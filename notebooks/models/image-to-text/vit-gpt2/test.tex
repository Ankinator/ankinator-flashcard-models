



\documentclass[aspectratio=3218]{beamer}


\input{../../../ressources/image-to-text-templates/English LaTeX beamer template (EMC, IMT, FEIT, OVGU, Stimulate)/packages}









\usepackage{beamer_ovgu_169}



\title[Short Title]{Title of the Presentation}
\author{John Doe}
\institute[Chair for EMC]{
	Chair for Electromagnetic Compatibility \\
	Institute for Medical Engineering \\
	Otto-von-Guericke-University Magdeburg, Germany
}
\mode<presentation>{\keywords{Put some keywords here, separated by commas, e.g. Non-Technical Project, Academic Writing}}
\date[01/01/2023]{Date of the Presentation, e.g. January 1th 2023}

\begin{document}

\begin{frame}
\frametitle<presentation>{Unrelated Title}


\begin{itemize}
\item Kfix+Kvar+At+Zt
\end{itemize}

\note[item]{}
\end{frame}
\begin{frame}
\frametitle<presentation>{Unrelated Title}

\begin{center}
\includegraphics[height=0.6\textheight]{/Users/I516998/Library/Application Support/Anki2/User 1/collection.media/img1963991816782229408.jpg}
\end{center}


\note[item]{}
\end{frame}
\begin{frame}
\frametitle<presentation>{Unrelated Title}

\begin{center}
\includegraphics[height=0.6\textheight]{/Users/I516998/Library/Application Support/Anki2/User 1/collection.media/img6198499762462923278.jpg}
\end{center}


\note[item]{}
\end{frame}
\begin{frame}
\frametitle<presentation>{Unrelated Title}


\begin{itemize}
\item G=U-K
\end{itemize}

\note[item]{}
\end{frame}
\begin{frame}
\frametitle<presentation>{Unrelated Title}



\note[item]{}
\end{frame}
\begin{frame}
\frametitle<presentation>{Unrelated Title}

\begin{center}
\includegraphics[height=0.6\textheight]{/Users/I516998/Library/Application Support/Anki2/User 1/collection.media/img8608825578963105513.jpg}
\end{center}


\note[item]{}
\end{frame}
\begin{frame}
\frametitle<presentation>{Unrelated Title}

\begin{center}
\includegraphics[height=0.6\textheight]{/Users/I516998/Library/Application Support/Anki2/User 1/collection.media/img1879412642540800749.jpg}
\end{center}


\note[item]{}
\end{frame}
\begin{frame}
\frametitle<presentation>{Unrelated Title}

\begin{center}
\includegraphics[height=0.6\textheight]{/Users/I516998/Library/Application Support/Anki2/User 1/collection.media/img4610769368882521779.jpg}
\end{center}


\note[item]{}
\end{frame}
\begin{frame}
\frametitle<presentation>{Unrelated Title}


\begin{itemize}
\item r=i1-[c01(i2-i1)/(c02-c01)]
\end{itemize}

\note[item]{}
\end{frame}
\begin{frame}
\frametitle<presentation>{Unrelated Title}


\begin{itemize}
\item Niedrigerer Zinssatz
\end{itemize}

\note[item]{}
\end{frame}
\begin{frame}
\frametitle<presentation>{Unrelated Title}


\begin{itemize}
\item Höherer Zinssatz 
\end{itemize}

\note[item]{}
\end{frame}
\begin{frame}
\frametitle<presentation>{Unrelated Title}


\begin{itemize}
\item (positiver) Kapitalwert des niedrigeren Zinssatzes
\end{itemize}

\note[item]{}
\end{frame}
\begin{frame}
\frametitle<presentation>{Unrelated Title}


\begin{itemize}
\item (Negativer)Kapitalwert des höheren Zinssatzes
\end{itemize}

\note[item]{}
\end{frame}
\begin{frame}
\frametitle<presentation>{Unrelated Title}

\begin{center}
\includegraphics[height=0.6\textheight]{/Users/I516998/Library/Application Support/Anki2/User 1/collection.media/img3974229742760914545.jpg}
\end{center}


\note[item]{}
\end{frame}
\begin{frame}
\frametitle<presentation>{Unrelated Title}


\begin{itemize}
\item G+Zt+At
\end{itemize}

\note[item]{}
\end{frame}
\begin{frame}
\frametitle<presentation>{Unrelated Title}


\begin{itemize}
\item (Gewinn+FK Zinsen)/(FK+EK)
\end{itemize}

\note[item]{}
\end{frame}
\begin{frame}
\frametitle<presentation>{Unrelated Title}


\begin{itemize}
\item Gewinn/EK
\end{itemize}

\note[item]{}
\end{frame}
\begin{frame}
\frametitle<presentation>{Unrelated Title}


\begin{itemize}
\item Gewinn/Umsatz
\end{itemize}

\note[item]{}
\end{frame}
\begin{frame}
\frametitle<presentation>{Unrelated Title}


\begin{itemize}
\item Abschreibung auf Forderungen an Wertberichtigung auf Forderungen
\end{itemize}

\note[item]{}
\end{frame}
\begin{frame}
\frametitle<presentation>{Unrelated Title}


\begin{itemize}
\item 1. Zweifelhafte Forderungen an Kundenforderungen
\item 2. Abschreibungen auf Forderungen an Zweifelhafte Forderungen
\end{itemize}

\note[item]{}
\end{frame}
\begin{frame}
\frametitle<presentation>{Unrelated Title}


\begin{itemize}
\item Forderungsverluste & Umsatzsteuer an Kundenforderungen
\end{itemize}

\note[item]{}
\end{frame}
\begin{frame}
\frametitle<presentation>{Unrelated Title}


\begin{itemize}
\item Bank an Erträge aus abgeschr. Ford. & USt.
\end{itemize}

\note[item]{}
\end{frame}
\begin{frame}
\frametitle<presentation>{Unrelated Title}


\begin{itemize}
\item Betrag d. letzten negativen Kapitalwertes (kummuliert)/Kapitalwert der ersten Periode mit positivem kummulierten Kapitalwert
\end{itemize}

\note[item]{}
\end{frame}
\begin{frame}
\frametitle<presentation>{Unrelated Title}


\begin{itemize}
\item    objektiver,beschreibender Charakter, inhaltlich überprüfbare
\end{itemize}

\note[item]{}
\end{frame}
\begin{frame}
\frametitle<presentation>{Unrelated Title}


\begin{itemize}
\item subjektiver Charakter, individuelle Wertung (wie die Welt sein sollte)
\end{itemize}

\note[item]{}
\end{frame}
\begin{frame}
\frametitle<presentation>{Unrelated Title}


\begin{itemize}
\item y = f(x) = ax+b-> zu erklärende Variable-> Beeinflussung durch andereVerniablen, reagiert auf Verändenung der unabhängigen Variable
\end{itemize}

\note[item]{}
\end{frame}
\begin{frame}
\frametitle<presentation>{Unrelated Title}


\begin{itemize}
\item y = f(x) = ax+b-> erklärende Variable-> beeinflusst abhängige Variable
\end{itemize}

\note[item]{}
\end{frame}
\begin{frame}
\frametitle<presentation>{Unrelated Title}


\begin{itemize}
\item -> nicht im Rahmen eines Modells erklärt-> äußerer Einfluss : z.B. Umweltkatastrophen, Kälte periode, Corona - Pandemie
\end{itemize}

\note[item]{}
\end{frame}
\begin{frame}
\frametitle<presentation>{Unrelated Title}

\begin{center}
\includegraphics[height=0.6\textheight]{/Users/I516998/Library/Application Support/Anki2/User 1/collection.media/image-64c771e8211adce57e8a6ea56aa73b8a03faeed5.png}
\end{center}

\begin{itemize}
\item -> alles was im Rahmen eines Modells erklärt wird-> wechselseitige Beziehung
\end{itemize}

\note[item]{}
\end{frame}
\begin{frame}
\frametitle<presentation>{Unrelated Title}


\begin{itemize}
\item -> Erreichen eines festen Zieles mit geringen (minimalen) Ressourcen (das Ziel ist fix, Mittel variabel)z.B. Ziel: neuer Laptop.Mittel: möglichst wenig Geld
\end{itemize}

\note[item]{}
\end{frame}

\end{document}