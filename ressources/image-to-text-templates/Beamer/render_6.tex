







\documentclass{beamer}
\usepackage[utf8]{inputenc}


\usepackage{comment}
\usetheme{Madrid}
  \usepackage[maxbibnames=99]{biblatex}
\definecolor{cvut_navy}{HTML}{0065BD}
\definecolor{cvut_blue}{HTML}{6AADE4}
\definecolor{cvut_gray}{HTML}{156570}
\usepackage{biblatex}


\usepackage[makeroom]{cancel}
\usepackage{threeparttable}
\usepackage[utf8]{inputenc}



\setbeamercolor{section in toc}{fg=black,bg=yellow} 
\setbeamercolor{alerted text}{fg=cvut_blue}
\usepackage{tikzsymbols}
\usepackage{textcomp}
\usepackage{parskip}
\definecolor{darkblue}{rgb}{0, 0, 0.5} 
\definecolor{babyblue}{rgb}{0.54, 0.81, 0.94}
\usepackage{pgf}
\usepackage{color,soul}

\usepackage{tcolorbox}
\tcbuselibrary{skins}
\usepackage{minted}
\usepackage{hyperref}
\usepackage{xcolor,soul}
\definecolor{lightblue}{rgb}{.90,.95,1}
\sethlcolor{lightblue}
\renewcommand<>{\hl}[1]{\only#2{\beameroriginal{\hl}}{#1}}
\setbeamertemplate{page number in head/foot}[framenumber]

\usepackage{empheq}
\usepackage{xcolor}
\definecolor{lightgreen}{HTML}{90EE90}
\newcommand{\boxedeq}[2]{\begin{empheq}[box={\fboxsep=6pt\fbox}]{align}\label{#1}#2\end{empheq}}
\newcommand{\coloredeq}[2]{\begin{empheq}[box=\colorbox{lightgreen}]{align}\label{#1}#2\end{empheq}}
\newcommand{\highlight}[1]{
  \colorbox{red!40}{$\displaystyle#1$}}
  
  
  \definecolor{babyblue}{rgb}{0.54, 0.81, 0.94}
\definecolor{babypink}{rgb}{0.96, 0.76, 0.76}
\definecolor{blue(ncs)}{rgb}{0.0, 0.53, 0.74}
\definecolor{pistachio}{rgb}{0.58, 0.77, 0.45}
\definecolor{darksalmon}{rgb}{0.91, 0.59, 0.48}
\definecolor{lightsalmonpink}{rgb}{1.0, 0.6, 0.6}
\definecolor{columbiablue}{rgb}{0.61, 0.87, 1.0}
\definecolor{corn}{rgb}{0.98, 0.93, 0.36}
\definecolor{jonquil}{rgb}{0.98, 0.85, 0.37}
\definecolor{bananayellow}{rgb}{1.0, 0.88, 0.21}
\newcommand{\bert}{\ensuremath{
  \mathchoice{\includegraphics[height=2ex]{Bert-pic-removebg-preview.png}} 
    {\includegraphics[height=2ex]{Bert-pic-removebg-preview.png}}
    {\includegraphics[height=1.5ex]{Bert-pic-removebg-preview.png}}
    {\includegraphics[height=1ex]{Bert-pic-removebg-preview.png}}
}}
  
  
\useoutertheme{infolines}



\usepackage{courier}

\usepackage{expl3}













\newcommand\scroll[4][§§]{
  \seq_set_split:Nnn\g_inputline_seq{#1}{#4}
  \seq_map_inline:Nn\g_inputline_seq{
    \seq_gput_right:Nx\g_linebuffer_seq{##1}
    \int_compare:nT{\seq_count:N\g_linebuffer_seq>#3}{
      \seq_gpop_left:NN\g_linebuffer_seq\dummy
    }
  }
  \fbox{\begin{minipage}[t][#3\baselineskip]{#2}
    \ttfamily
    \seq_map_inline:Nn\g_linebuffer_seq{\mbox{##1}\\}
  \end{minipage}}
}
\newcommand\clearbuf{\seq_gclear:N\g_linebuffer_seq}
\ExplSyntaxOff

\setbeamertemplate{headline}{
\begin{beamercolorbox}[colsep=1.5pt]{upper separation line head}
\end{beamercolorbox}
\begin{beamercolorbox}{section in head/foot}
    \vskip2pt\insertsectionnavigationhorizontal{\paperwidth}{}{\hskip0pt plus1filll}\vskip2pt
\end{beamercolorbox}

\begin{beamercolorbox}[colsep=1.5pt]{lower separation line head}
\end{beamercolorbox}
}
\makeatletter
\newcommand\SoulColor{
  \let\set@color\beamerorig@set@color
  \let\reset@color\beamerorig@reset@color}
\makeatother
\SoulColor
\usepackage{amsmath, bm}
\usepackage{tikz}
\setbeamercovered{dynamic}

\newcommand{\highlightt}[1]{
  \colorbox{blue!40}{$\displaystyle#1$}}


\newenvironment<>{problock}[1]{
  \begin{actionenv}#2
      \def\insertblocktitle{#1}
      \par
      \mode<presentation>{


        \setbeamercolor{block title}{fg=white,bg=cvut_blue}
       \setbeamercolor{block body}{fg=black,bg=white!50}
       \setbeamercolor{itemize item}{fg=orange!20!black}
       \setbeamertemplate{itemize item}[triangle]
     }
      \usebeamertemplate{block begin}
    \par\usebeamertemplate{block end}
    \end{actionenv}
    }
    
    \newcommand<>{\uncovergraphics}[2][{}]{

    \begin{tikzpicture}
    \node[anchor=south west,inner sep=0] (B) at (4,0)
        {\includegraphics[#1]{#2}};
    \alt#3{}{
        \fill [draw=none, fill=background, fill opacity=0.9] (B.north west) -- (B.north east) -- (B.south east) -- (B.south west) -- (B.north west) -- cycle;
    }
    \end{tikzpicture}
}




\newlength\dlf
\newcommand\alignedbox[3][yellow]{



  &
  \begingroup
  \settowidth\dlf{$\displaystyle #2$}
  \addtolength\dlf{\fboxsep+\fboxrule}
  \hspace{-\dlf}
  \fcolorbox{red}{#1}{$\displaystyle #2 #3$}
  \endgroup
}

\usepackage{collcell}
\usepackage{booktabs}
\usepackage{etoolbox}

\makeatletter

\usepackage{xcoffins}
\NewCoffin\tablecoffin
\NewDocumentCommand\Vcentre{m}
  {
    \SetHorizontalCoffin\tablecoffin{#1}
    \TypesetCoffin\tablecoffin[l,vc]
  }



\usepackage{pgfpages}





\setbeamertemplate{note page}{\pagecolor{gray!5}\insertnote}\usepackage{palatino}





\usepackage{xcolor}
\usepackage{soul}

\usepackage{etoolbox}
\makeatletter

\makeatother

\setbeamercolor*{palette primary}{bg=cvut_navy,fg=gray!20!white}
\setbeamercolor*{palette secondary}{bg=cvut_navy,fg=gray!20!white}


\setbeamercolor*{palette tertiary}{parent=palette primary}
\setbeamercolor*{palette quaternary}{fg=cvut_navy,bg=gray!5!white}
\setbeamercolor*{sidebar}{fg=cvut_navy,bg=gray!15!white}
\usepackage[first=0,last=9]{lcg}
\newcommand{\ra}{\rand0.\arabic{rand}}
\usepackage{color, colortbl}
\usepackage{stackengine,tikz}
\usepackage{transparent}
\usepackage{pgfpages}

\usepackage{booktabs}

\colorlet{Gray}{gray!30}




    
\newcommand*{\MinNumber}{0}

\newcommand*{\MaxNumber}{0.4}
\definecolor{bubblegum}{rgb}{0.99, 0.76, 0.8}
\newcommand{\ApplyGradient}[1]{
  \pgfmathsetmacro{\PercentColor}{100.0*(#1-\MinNumber)/(\MaxNumber-\MinNumber)}

  \edef\x{\noexpand\cellcolor{babyblue!\PercentColor}}\x\textcolor{black}{#1}
}
\newcolumntype{R}{>{\collectcell\ApplyGradient}{c}<{\endcollectcell}}


\setbeamercolor{titlelike}{parent=palette primary}
\setbeamercolor{frametitle}{parent=palette primary}

\setbeamercolor{B}{bg=red!30,fg=black}


\setbeamertemplate{section in toc}[default]
\setbeamercolor{itemize item }{fg=blue}
\setbeamertemplate{itemize item}[circle]

\setbeamercolor*{separation line}{}
\setbeamercolor*{fine separation line}{}

\setbeamertemplate{navigation symbols}{} 
\setbeamertemplate{caption}{\raggedright\insertcaption\par}




\setbeamercolor*{block title example}{fg=white,bg=purple!75!black}
\setbeamercolor*{block body example}{fg= black, bg= white}


\setbeamercolor{itemize item}{fg=cvut_navy}
\setbeamercolor{block title}{bg=red!30,fg=black}
\setbeamertemplate{subsection in toc}[subsections numbered]


\usepackage{eqnarray,amsmath}
\usepackage{amsfonts}
\usepackage{amssymb}
\usefonttheme{professionalfonts}
\usepackage{graphicx}
\usepackage{booktabs} 
\usepackage{bm}
\usepackage{mathtools}
\usepackage[utf8]{inputenc}
\usepackage[T1]{fontenc}
\usepackage{lmodern} 
\usepackage{booktabs}




\definecolor{NormalBlue}{RGB}{200,200,255}

\setbeamercolor{block title}{fg=black, bg=yellow}
\setbeamercolor{block2}{use=structure,fg=white,bg=purple!75!black}

\def\bq{\mbox{\kern.1ex\protect\raisebox{-1.3ex}[0pt][0pt]{''}\kern-.1ex}}
\def\eq{\mbox{\kern-.1ex``\kern.1ex}}
\def\ifundefined#1{\expandafter\ifx\csname#1\endcsname\relax }
\ifundefined{uv}
        \gdef\uv#1{\bq #1\eq}
\fi





\author[Thesis Defense]{Author}
\institute[]{Universitat Politècnica de Catalunya \\  Department of Computer Science
\vspace{2mm} \\ 
Advisors:  \\ Prof. John 

\vspace{2mm}}
\title[]{Presentation Title}


  \date[March 31, 2023]{\small{Thesis Defense} \\ \small{March 31, 2023}}





\begin{document}

\begin{frame}
\frametitle{Unrelated Title}


\begin{itemize}
\item 
Transparency
Inspection
Adaptation

\end{itemize}

\note[item]{}
\end{frame}
\begin{frame}
\frametitle{Unrelated Title}


\begin{itemize}
\item 
Product Owner (PO)
Scrum Master (SM)
Development Team (DT)

\end{itemize}

\note[item]{}
\end{frame}
\begin{frame}
\frametitle{Unrelated Title}


\begin{itemize}
\item 
Sprint Planning
Daily Scrum
Sprint review
Sprint Retrospective

\end{itemize}

\note[item]{}
\end{frame}
\begin{frame}
\frametitle{Unrelated Title}


\begin{itemize}
\item 
Product Backlog
Product Burn-up/down chart 
Sprint Backlog
Sprint Burn-up/down chart

\end{itemize}

\note[item]{}
\end{frame}
\begin{frame}
\frametitle{Unrelated Title}


\begin{itemize}
\item Translate the business needs in the product backlog
\end{itemize}

\note[item]{}
\end{frame}
\begin{frame}
\frametitle{Unrelated Title}


\begin{itemize}
\item The Product Owner
\end{itemize}

\note[item]{}
\end{frame}
\begin{frame}
\frametitle{Unrelated Title}


\begin{itemize}
\item The Product Owner
\end{itemize}

\note[item]{}
\end{frame}
\begin{frame}
\frametitle{Unrelated Title}


\begin{itemize}
\item Deliver potentially releasable increment of "Done"
\end{itemize}

\note[item]{}
\end{frame}
\begin{frame}
\frametitle{Unrelated Title}


\begin{itemize}
\item Self-organized
\end{itemize}

\note[item]{}
\end{frame}
\begin{frame}
\frametitle{Unrelated Title}


\begin{itemize}
\item Developer (no other)
\end{itemize}

\note[item]{}
\end{frame}
\begin{frame}
\frametitle{Unrelated Title}


\begin{itemize}
\item The Scrum Master
\end{itemize}

\note[item]{}
\end{frame}
\begin{frame}
\frametitle{Unrelated Title}


\begin{itemize}
\item • ensuring Scrum is understood and enacted• Remove impediments 
\end{itemize}

\note[item]{}
\end{frame}
\begin{frame}
\frametitle{Unrelated Title}


\begin{itemize}
\item Product Owner
\end{itemize}

\note[item]{}
\end{frame}
\begin{frame}
\frametitle{Unrelated Title}


\begin{itemize}
\item When the goal becomes obsolete
\end{itemize}

\note[item]{}
\end{frame}
\begin{frame}
\frametitle{Unrelated Title}


\begin{itemize}
\item At the beginning of each sprint
\end{itemize}

\note[item]{}
\end{frame}
\begin{frame}
\frametitle{Unrelated Title}


\begin{itemize}
\item Scrum Team (PO, SM, DT)
\end{itemize}

\note[item]{}
\end{frame}
\begin{frame}
\frametitle{Unrelated Title}


\begin{itemize}
\item max 8hrs for 1 month sprint
\end{itemize}

\note[item]{}
\end{frame}
\begin{frame}
\frametitle{Unrelated Title}


\begin{itemize}
\item • Selection of the PBI to do during the sprint• Discuss on how the work will be done
\end{itemize}

\note[item]{}
\end{frame}
\begin{frame}
\frametitle{Unrelated Title}


\begin{itemize}
\item Each days at the same time
\end{itemize}

\note[item]{}
\end{frame}
\begin{frame}
\frametitle{Unrelated Title}


\begin{itemize}
\item Every day at the same place
\end{itemize}

\note[item]{}
\end{frame}
\begin{frame}
\frametitle{Unrelated Title}


\begin{itemize}
\item Synchronize the team
\end{itemize}

\note[item]{}
\end{frame}
\begin{frame}
\frametitle{Unrelated Title}


\begin{itemize}
\item Development Team
\end{itemize}

\note[item]{}
\end{frame}
\begin{frame}
\frametitle{Unrelated Title}


\begin{itemize}
\item 15 minutes time-boxed
\end{itemize}

\note[item]{}
\end{frame}
\begin{frame}
\frametitle{Unrelated Title}


\begin{itemize}
\item o What did I do yesterday that helped the Development Team meet the Sprint Goal?o What will I do today to help the Development Team meet the Sprint Goal?o Do I see any impediment that prevents me or the Development Team from meeting the Sprint Goal?
\end{itemize}

\note[item]{}
\end{frame}
\begin{frame}
\frametitle{Unrelated Title}


\begin{itemize}
\item After the Daily Scrum
\end{itemize}

\note[item]{}
\end{frame}
\begin{frame}
\frametitle{Unrelated Title}


\begin{itemize}
\item At the beginning of the Sprint
\end{itemize}

\note[item]{}
\end{frame}
\begin{frame}
\frametitle{Unrelated Title}


\begin{itemize}
\item Scrum Team + Key stakeholders invited by the product owner
\end{itemize}

\note[item]{}
\end{frame}
\begin{frame}
\frametitle{Unrelated Title}


\begin{itemize}
\item 4 hours time-boxed for a 1 month sprint
\end{itemize}

\note[item]{}
\end{frame}
\begin{frame}
\frametitle{Unrelated Title}


\begin{itemize}
\item o Dev team demonstrate the work doneo Product owner says what is done / not doneo Feedback from outside invited stakeholderso Discuss what went well, where were problems, and how those problems were solved.o Discusses the product backlog as it stand o What to do next (input to next sprint) o Review timeline, Budget, capabilities
\end{itemize}

\note[item]{}
\end{frame}
\begin{frame}
\frametitle{Unrelated Title}


\begin{itemize}
\item No, they have opposite interestThe Product Owner try to push the Development TeamThe Scrum Master protect the Development Team
\end{itemize}

\note[item]{}
\end{frame}
\begin{frame}
\frametitle{Unrelated Title}


\begin{itemize}
\item The purpose of the select statement is to retrieve and display data from one or more database tables.
\end{itemize}

\note[item]{}
\end{frame}
\begin{frame}
\frametitle{Unrelated Title}


\begin{itemize}
\item The select statement.
\end{itemize}

\note[item]{}
\end{frame}
\begin{frame}
\frametitle{Unrelated Title}


\begin{itemize}
\item Specifies which columns are to appear in the output.
\end{itemize}

\note[item]{}
\end{frame}
\begin{frame}
\frametitle{Unrelated Title}


\begin{itemize}
\item Specifies the table(s) to be used.
\end{itemize}

\note[item]{}
\end{frame}
\begin{frame}
\frametitle{Unrelated Title}


\begin{itemize}
\item Filters the rows(s) subject to some condition.
\end{itemize}

\note[item]{}
\end{frame}
\begin{frame}
\frametitle{Unrelated Title}


\begin{itemize}
\item Forms groups of row(s) with the same column value.
\end{itemize}

\note[item]{}
\end{frame}
\begin{frame}
\frametitle{Unrelated Title}


\begin{itemize}
\item Filters the groups subject to some condition.
\end{itemize}

\note[item]{}
\end{frame}
\begin{frame}
\frametitle{Unrelated Title}


\begin{itemize}
\item Specifies the order of the output.
\end{itemize}

\note[item]{}
\end{frame}
\begin{frame}
\frametitle{Unrelated Title}


\begin{itemize}
\item The SELECT and FROM statement.
\end{itemize}

\note[item]{}
\end{frame}
\begin{frame}
\frametitle{Unrelated Title}


\begin{itemize}
\item AND, OR and NOT.
\end{itemize}

\note[item]{}
\end{frame}
\begin{frame}
\frametitle{Unrelated Title}


\begin{itemize}
\item The WHERE clause.
\end{itemize}

\note[item]{}
\end{frame}
\begin{frame}
\frametitle{Unrelated Title}


\begin{itemize}
\item NOT before AND / ORAND before OR
\end{itemize}

\note[item]{}
\end{frame}
\begin{frame}
\frametitle{Unrelated Title}


\begin{itemize}
\item The Like clause.
\end{itemize}

\note[item]{}
\end{frame}
\begin{frame}
\frametitle{Unrelated Title}


\begin{itemize}
\item The % character, used to find a value that matches part of a string or value.The _ character, representing a single character.
\end{itemize}

\note[item]{}
\end{frame}
\begin{frame}
\frametitle{Unrelated Title}


\begin{itemize}
\item COUNT, SUM, AVG, MIN, MAX
\end{itemize}

\note[item]{}
\end{frame}
\begin{frame}
\frametitle{Unrelated Title}


\begin{itemize}
\item COUNT returns the number of values in a specified column.
\end{itemize}

\note[item]{}
\end{frame}
\begin{frame}
\frametitle{Unrelated Title}


\begin{itemize}
\item SUM returns the sum of the values in a specified column.
\end{itemize}

\note[item]{}
\end{frame}
\begin{frame}
\frametitle{Unrelated Title}


\begin{itemize}
\item AVG returns the average of the values in a specified column.
\end{itemize}

\note[item]{}
\end{frame}
\begin{frame}
\frametitle{Unrelated Title}


\begin{itemize}
\item Min returns the smallest value in a specified column
\end{itemize}

\note[item]{}
\end{frame}
\begin{frame}
\frametitle{Unrelated Title}


\begin{itemize}
\item MAX returns the largest value in a specified column.
\end{itemize}

\note[item]{}
\end{frame}
\begin{frame}
\frametitle{Unrelated Title}


\begin{itemize}
\item A complete SELECT statement embedded within another SELECT statement
\end{itemize}

\note[item]{}
\end{frame}
\begin{frame}
\frametitle{Unrelated Title}


\begin{itemize}
\item A subquery
\end{itemize}

\note[item]{}
\end{frame}
\begin{frame}
\frametitle{Unrelated Title}


\begin{itemize}
\item The JOIN operation
\end{itemize}

\note[item]{}
\end{frame}
\begin{frame}
\frametitle{Unrelated Title}


\begin{itemize}
\item When we want to combine several columns from several tables into a result table.
\end{itemize}

\note[item]{}
\end{frame}
\begin{frame}
\frametitle{Unrelated Title}


\begin{itemize}
\item A new table where clientNo is the common JOIN column.
\end{itemize}

\note[item]{}
\end{frame}
\begin{frame}
\frametitle{Unrelated Title}


\begin{itemize}
\item The left table(1).
\end{itemize}

\note[item]{}
\end{frame}
\begin{frame}
\frametitle{Unrelated Title}


\begin{itemize}
\item The right table(2).
\end{itemize}

\note[item]{}
\end{frame}
\begin{frame}
\frametitle{Unrelated Title}


\begin{itemize}
\item They result in a NULL.
\end{itemize}

\note[item]{}
\end{frame}
\begin{frame}
\frametitle{Unrelated Title}


\begin{itemize}
\item The left table(1).
\end{itemize}

\note[item]{}
\end{frame}
\begin{frame}
\frametitle{Unrelated Title}


\begin{itemize}
\item The right table(2).
\end{itemize}

\note[item]{}
\end{frame}
\begin{frame}
\frametitle{Unrelated Title}


\begin{itemize}
\item All rows from the left table(1) and the right table(2).
\end{itemize}

\note[item]{}
\end{frame}
\begin{frame}
\frametitle{Unrelated Title}


\begin{itemize}
\item FULL OUTER JOIN
\end{itemize}

\note[item]{}
\end{frame}
\begin{frame}
\frametitle{Unrelated Title}


\begin{itemize}
\item INSERT, UPDATE and DELETE.
\end{itemize}

\note[item]{}
\end{frame}
\begin{frame}
\frametitle{Unrelated Title}


\begin{itemize}
\item Modify contents of a table in the database.
\end{itemize}

\note[item]{}
\end{frame}
\begin{frame}
\frametitle{Unrelated Title}


\begin{itemize}
\item To update one or more values in a specified column or columns of a named table.
\end{itemize}

\note[item]{}
\end{frame}
\begin{frame}
\frametitle{Unrelated Title}


\begin{itemize}
\item The UPDATE statement.
\end{itemize}

\note[item]{}
\end{frame}
\begin{frame}
\frametitle{Unrelated Title}


\begin{itemize}
\item The DELETE statement.
\end{itemize}

\note[item]{}
\end{frame}
\begin{frame}
\frametitle{Unrelated Title}


\begin{itemize}
\item Delete one or more rows from a named table.
\end{itemize}

\note[item]{}
\end{frame}
\begin{frame}
\frametitle{Unrelated Title}


\begin{itemize}
\item The truth values TRUE and FALSE.
\end{itemize}

\note[item]{}
\end{frame}
\begin{frame}
\frametitle{Unrelated Title}


\begin{itemize}
\item Boolean Data.
\end{itemize}

\note[item]{}
\end{frame}
\begin{frame}
\frametitle{Unrelated Title}


\begin{itemize}
\item It returns an UNKNOWN result.
\end{itemize}

\note[item]{}
\end{frame}
\begin{frame}
\frametitle{Unrelated Title}


\begin{itemize}
\item When we compare a NULL value or an UNKNOWN truth.
\end{itemize}

\note[item]{}
\end{frame}
\begin{frame}
\frametitle{Unrelated Title}


\begin{itemize}
\item A sequence of characters from an implementation defined character set(CHAR or VARCHAR).
\end{itemize}

\note[item]{}
\end{frame}
\begin{frame}
\frametitle{Unrelated Title}


\begin{itemize}
\item Character Data
\end{itemize}

\note[item]{}
\end{frame}
\begin{frame}
\frametitle{Unrelated Title}


\begin{itemize}
\item Defining bit strings.
\end{itemize}

\note[item]{}
\end{frame}
\begin{frame}
\frametitle{Unrelated Title}


\begin{itemize}
\item Bit Data
\end{itemize}

\note[item]{}
\end{frame}
\begin{frame}
\frametitle{Unrelated Title}


\begin{itemize}
\item A sequence of binary digits with either the value 0 or 1.
\end{itemize}

\note[item]{}
\end{frame}
\begin{frame}
\frametitle{Unrelated Title}


\begin{itemize}
\item When we need to define numbers with exact representations.
\end{itemize}

\note[item]{}
\end{frame}
\begin{frame}
\frametitle{Unrelated Title}


\begin{itemize}
\item Exact numeric data
\end{itemize}

\note[item]{}
\end{frame}
\begin{frame}
\frametitle{Unrelated Title}


\begin{itemize}
\item Excact numeric data.
\end{itemize}

\note[item]{}
\end{frame}
\begin{frame}
\frametitle{Unrelated Title}


\begin{itemize}
\item Defining numbers that do not have an exact representation (real numbers).
\end{itemize}

\note[item]{}
\end{frame}
\begin{frame}
\frametitle{Unrelated Title}


\begin{itemize}
\item Approximate numeric data
\end{itemize}

\note[item]{}
\end{frame}
\begin{frame}
\frametitle{Unrelated Title}


\begin{itemize}
\item When we want to define a point in time to a certain degree of accuracy.
\end{itemize}

\note[item]{}
\end{frame}
\begin{frame}
\frametitle{Unrelated Title}


\begin{itemize}
\item Date time data
\end{itemize}

\note[item]{}
\end{frame}
\begin{frame}
\frametitle{Unrelated Title}


\begin{itemize}
\item DATA TYPE DECLARATIONS
\end{itemize}

\note[item]{}
\end{frame}
\begin{frame}
\frametitle{Unrelated Title}


\begin{itemize}
\item DATA TYPE DECLARATIONS
\end{itemize}

\note[item]{}
\end{frame}
\begin{frame}
\frametitle{Unrelated Title}


\begin{itemize}
\item DATA TYPE DECLARATIONS
\end{itemize}

\note[item]{}
\end{frame}
\begin{frame}
\frametitle{Unrelated Title}


\begin{itemize}
\item DATA TYPE DECLARATIONS
\end{itemize}

\note[item]{}
\end{frame}
\begin{frame}
\frametitle{Unrelated Title}


\begin{itemize}
\item Columns must contain a valid value and are not allowed to contain NULLS.
\end{itemize}

\note[item]{}
\end{frame}
\begin{frame}
\frametitle{Unrelated Title}


\begin{itemize}
\item A set of legal values in a column, every colums has a domain.
\end{itemize}

\note[item]{}
\end{frame}
\begin{frame}
\frametitle{Unrelated Title}


\begin{itemize}
\item The CHECK clause and the CREATE DOMAIN statement.
\end{itemize}

\note[item]{}
\end{frame}
\begin{frame}
\frametitle{Unrelated Title}


\begin{itemize}
\item Check whether a value is in the range specified.
\end{itemize}

\note[item]{}
\end{frame}
\begin{frame}
\frametitle{Unrelated Title}


\begin{itemize}
\item CREATE DOMAIN
\end{itemize}

\note[item]{}
\end{frame}
\begin{frame}
\frametitle{Unrelated Title}


\begin{itemize}
\item Create our own data type with included CHECKs.
\end{itemize}

\note[item]{}
\end{frame}
\begin{frame}
\frametitle{Unrelated Title}


\begin{itemize}
\item Must contain a unique, non-null value for each row.
\end{itemize}

\note[item]{}
\end{frame}
\begin{frame}
\frametitle{Unrelated Title}


\begin{itemize}
\item The UNIQUE keyword.
\end{itemize}

\note[item]{}
\end{frame}
\begin{frame}
\frametitle{Unrelated Title}


\begin{itemize}
\item CREATE DOMAIN, ALTER DOMAIN, DROP DOMAINCREATE TABLE, ALTER TABLE, DROP TABLECREATE VIEW, DROP VIEW
\end{itemize}

\note[item]{}
\end{frame}
\begin{frame}
\frametitle{Unrelated Title}


\begin{itemize}
\item CREATE SCHEMA, DROP SCHEMACREATE TABLE, ALTER TABLE, DROP TABLECREATE VIEW, DROP VIEW
\end{itemize}

\note[item]{}
\end{frame}
\begin{frame}
\frametitle{Unrelated Title}


\begin{itemize}
\item CREATE SCHEMA, DROP SCHEMACREATE DOMAIN, ALTER DOMAIN, DROP DOMAINCREATE VIEW, DROP VIEW
\end{itemize}

\note[item]{}
\end{frame}
\begin{frame}
\frametitle{Unrelated Title}


\begin{itemize}
\item CREATE SCHEMA, DROP SCHEMACREATE DOMAIN, ALTER DOMAIN, DROP DOMAINCREATE TABLE, ALTER TABLE, DROP TABLE
\end{itemize}

\note[item]{}
\end{frame}
\begin{frame}
\frametitle{Unrelated Title}


\begin{itemize}
\item A view is a virtual table representing a subet columns / rows from one or more base tables or views.
\end{itemize}

\note[item]{}
\end{frame}
\begin{frame}
\frametitle{Unrelated Title}


\begin{itemize}
\item Ends the transactions successfully, making the database changes permanent.
\end{itemize}

\note[item]{}
\end{frame}
\begin{frame}
\frametitle{Unrelated Title}


\begin{itemize}
\item A COMMIT statement
\end{itemize}

\note[item]{}
\end{frame}
\begin{frame}
\frametitle{Unrelated Title}


\begin{itemize}
\item The ROLLBACK statement
\end{itemize}

\note[item]{}
\end{frame}
\begin{frame}
\frametitle{Unrelated Title}


\begin{itemize}
\item When we want to abort the transactions and back out any changes made by the transaction.
\end{itemize}

\note[item]{}
\end{frame}
\begin{frame}
\frametitle{Unrelated Title}


\begin{itemize}
\item READ WRITE.
\end{itemize}

\note[item]{}
\end{frame}
\begin{frame}
\frametitle{Unrelated Title}


\begin{itemize}
\item READ ONLY
\end{itemize}

\note[item]{}
\end{frame}
\begin{frame}
\frametitle{Unrelated Title}


\begin{itemize}
\item Authorization identifiers, ownerships and privileges.
\end{itemize}

\note[item]{}
\end{frame}
\begin{frame}
\frametitle{Unrelated Title}


\begin{itemize}
\item The Database administrator
\end{itemize}

\note[item]{}
\end{frame}
\begin{frame}
\frametitle{Unrelated Title}


\begin{itemize}
\item To pass privileges on to another user.
\end{itemize}

\note[item]{}
\end{frame}
\begin{frame}
\frametitle{Unrelated Title}


\begin{itemize}
\item The GRANT statement
\end{itemize}

\note[item]{}
\end{frame}
\begin{frame}
\frametitle{Unrelated Title}


\begin{itemize}
\item The REVOKE statement.
\end{itemize}

\note[item]{}
\end{frame}
\begin{frame}
\frametitle{Unrelated Title}


\begin{itemize}
\item When the owner wants to revoke privileges passed on to a user.
\end{itemize}

\note[item]{}
\end{frame}
\begin{frame}
\frametitle{Unrelated Title}


\begin{itemize}
\item USAGE, SELECT, DELETE, INSERT, UPDATE, REFERENCES.
\end{itemize}

\note[item]{}
\end{frame}
\begin{frame}
\frametitle{Unrelated Title}


\begin{itemize}
\item INSERT, UPDATE and REFERENCES.
\end{itemize}

\note[item]{}
\end{frame}
\begin{frame}
\frametitle{Unrelated Title}


\begin{itemize}
\item Allow a user obtaining the SQL priviliges to pass on SQL privileges.
\end{itemize}

\note[item]{}
\end{frame}
\begin{frame}
\frametitle{Unrelated Title}


\begin{itemize}
\item An optional declaration part, a mandatory excecutable part and an optional exceptions parts.
\end{itemize}

\note[item]{}
\end{frame}
\begin{frame}
\frametitle{Unrelated Title}


\begin{itemize}
\item The normal assign statement (:=)As the result of a SQL SELECT or FETCH statement
\end{itemize}

\note[item]{}
\end{frame}
\begin{frame}
\frametitle{Unrelated Title}


\begin{itemize}
\item Assign a variable in the excecutable part of the PL/SQL block.
\end{itemize}

\note[item]{}
\end{frame}
\begin{frame}
\frametitle{Unrelated Title}


\begin{itemize}
\item Stored procedures and functions.
\end{itemize}

\note[item]{}
\end{frame}
\begin{frame}
\frametitle{Unrelated Title}


\begin{itemize}
\item A function always returns a single value to the caller, a procedure does not.
\end{itemize}

\note[item]{}
\end{frame}
\begin{frame}
\frametitle{Unrelated Title}


\begin{itemize}
\item It has a specified name and data type.
\end{itemize}

\note[item]{}
\end{frame}
\begin{frame}
\frametitle{Unrelated Title}


\begin{itemize}
\item IN, OUT and IN OUT
\end{itemize}

\note[item]{}
\end{frame}
\begin{frame}
\frametitle{Unrelated Title}


\begin{itemize}
\item A parameter is used as an input value only
\end{itemize}

\note[item]{}
\end{frame}
\begin{frame}
\frametitle{Unrelated Title}


\begin{itemize}
\item A parameter is used as an output value only
\end{itemize}

\note[item]{}
\end{frame}
\begin{frame}
\frametitle{Unrelated Title}


\begin{itemize}
\item A parameter is used both as an input and output value
\end{itemize}

\note[item]{}
\end{frame}
\begin{frame}
\frametitle{Unrelated Title}


\begin{itemize}
\item Procedures, functions, variables, and SQL statements
\end{itemize}

\note[item]{}
\end{frame}
\begin{frame}
\frametitle{Unrelated Title}


\begin{itemize}
\item A specification and a body.
\end{itemize}

\note[item]{}
\end{frame}
\begin{frame}
\frametitle{Unrelated Title}


\begin{itemize}
\item Declares all public constucst of the package.
\end{itemize}

\note[item]{}
\end{frame}
\begin{frame}
\frametitle{Unrelated Title}


\begin{itemize}
\item Defines all constructs (public and private).
\end{itemize}

\note[item]{}
\end{frame}
\begin{frame}
\frametitle{Unrelated Title}


\begin{itemize}
\item An action the database should take when some event occurs in the application.
\end{itemize}

\note[item]{}
\end{frame}
\begin{frame}
\frametitle{Unrelated Title}


\begin{itemize}
\item Trigger
\end{itemize}

\note[item]{}
\end{frame}
\begin{frame}
\frametitle{Unrelated Title}


\begin{itemize}
\item Trigger body or trigger action.
\end{itemize}

\note[item]{}
\end{frame}
\begin{frame}
\frametitle{Unrelated Title}


\begin{itemize}
\item Row-level triggers and statement-level triggers.
\end{itemize}

\note[item]{}
\end{frame}
\begin{frame}
\frametitle{Unrelated Title}


\begin{itemize}
\item It executes for each row of the table that is affected by the triggering event.
\end{itemize}

\note[item]{}
\end{frame}
\begin{frame}
\frametitle{Unrelated Title}


\begin{itemize}
\item Only once even if multiple rows are affected by the triggering event.
\end{itemize}

\note[item]{}
\end{frame}
\begin{frame}
\frametitle{Unrelated Title}


\begin{itemize}
\item Triggers that when triggered are excecuted instead of the original SQL statement.
\end{itemize}

\note[item]{}
\end{frame}
\begin{frame}
\frametitle{Unrelated Title}


\begin{itemize}
\item That the trigger should be excecuted before an INSERT is applied to the table.
\end{itemize}

\note[item]{}
\end{frame}
\begin{frame}
\frametitle{Unrelated Title}


\begin{itemize}
\item The resources that enable the collection, management, control and dissemination of information.
\end{itemize}

\note[item]{}
\end{frame}
\begin{frame}
\frametitle{Unrelated Title}


\begin{itemize}
\item ISCL, SDLC and DSDLC
\end{itemize}

\note[item]{}
\end{frame}
\begin{frame}
\frametitle{Unrelated Title}


\begin{itemize}
\item Information Systems LifeCycle
\end{itemize}

\note[item]{}
\end{frame}
\begin{frame}
\frametitle{Unrelated Title}


\begin{itemize}
\item Software Development LifeCycle
\end{itemize}

\note[item]{}
\end{frame}
\begin{frame}
\frametitle{Unrelated Title}


\begin{itemize}
\item Database System Development LifeCycle
\end{itemize}

\note[item]{}
\end{frame}
\begin{frame}
\frametitle{Unrelated Title}


\begin{itemize}
\item The finalizing of a development after several iterations of trial and error.
\end{itemize}

\note[item]{}
\end{frame}
\begin{frame}
\frametitle{Unrelated Title}


\begin{itemize}
\item Enterprise activities within an organization such as marketing, personnel and stock control.
\end{itemize}

\note[item]{}
\end{frame}
\begin{frame}
\frametitle{Unrelated Title}


\begin{itemize}
\item 1 Database planning2 System definition3 Requirements collection and analysis4 Database design5 DBMS selection
\end{itemize}

\note[item]{}
\end{frame}
\begin{frame}
\frametitle{Unrelated Title}


\begin{itemize}
\item Planning how the stages of the lifecycle can be realized mos efficiently and effectively.
\end{itemize}

\note[item]{}
\end{frame}
\begin{frame}
\frametitle{Unrelated Title}


\begin{itemize}
\item Defining a mission statement and defining mission objectives.
\end{itemize}

\note[item]{}
\end{frame}
\begin{frame}
\frametitle{Unrelated Title}


\begin{itemize}
\item Define the major aims of the database system.
\end{itemize}

\note[item]{}
\end{frame}
\begin{frame}
\frametitle{Unrelated Title}


\begin{itemize}
\item Each objective should identify a task that the Database system must support 
\end{itemize}

\note[item]{}
\end{frame}
\begin{frame}
\frametitle{Unrelated Title}


\begin{itemize}
\item Entities, Attributes, (foreign key) relations
\end{itemize}

\note[item]{}
\end{frame}
\begin{frame}
\frametitle{Unrelated Title}


\begin{itemize}
\item Blue, one lease belongs to one and only one client.Red, a client can have 0 or 1 more leases.
\end{itemize}

\note[item]{}
\end{frame}
\begin{frame}
\frametitle{Unrelated Title}


\begin{itemize}
\item A column, or combination of columns that have a different value for each row.
\end{itemize}

\note[item]{}
\end{frame}
\begin{frame}
\frametitle{Unrelated Title}


\begin{itemize}
\item A primary key
\end{itemize}

\note[item]{}
\end{frame}
\begin{frame}
\frametitle{Unrelated Title}


\begin{itemize}
\item Defining the relations between tables.
\end{itemize}

\note[item]{}
\end{frame}
\begin{frame}
\frametitle{Unrelated Title}


\begin{itemize}
\item Navigation from table to table
\end{itemize}

\note[item]{}
\end{frame}
\begin{frame}
\frametitle{Unrelated Title}


\begin{itemize}
\item The name of the referenced table and fields.
\end{itemize}

\note[item]{}
\end{frame}
\begin{frame}
\frametitle{Unrelated Title}


\begin{itemize}
\item Singular not plural.
\end{itemize}

\note[item]{}
\end{frame}
\begin{frame}
\frametitle{Unrelated Title}


\begin{itemize}
\item Crowfoot notation
\end{itemize}

\note[item]{}
\end{frame}
\begin{frame}
\frametitle{Unrelated Title}


\begin{itemize}
\item Crowfoot notation
\end{itemize}

\note[item]{}
\end{frame}
\begin{frame}
\frametitle{Unrelated Title}


\begin{itemize}
\item Crowfoot notation
\end{itemize}

\note[item]{}
\end{frame}
\begin{frame}
\frametitle{Unrelated Title}


\begin{itemize}
\item Crowfoot notation
\end{itemize}

\note[item]{}
\end{frame}
\begin{frame}
\frametitle{Unrelated Title}


\begin{itemize}
\item Self referencing (recursive)
\end{itemize}

\note[item]{}
\end{frame}
\begin{frame}
\frametitle{Unrelated Title}


\begin{itemize}
\item Self referencing (recursive)
\end{itemize}

\note[item]{}
\end{frame}
\begin{frame}
\frametitle{Unrelated Title}


\begin{itemize}
\item Generalization
\end{itemize}

\note[item]{}
\end{frame}
\begin{frame}
\frametitle{Unrelated Title}


\begin{itemize}
\item Generalization
\end{itemize}

\note[item]{}
\end{frame}
\begin{frame}
\frametitle{Unrelated Title}


\begin{itemize}
\item Data definition language(DDL)Data manipulation language(DML)Data control language(DCL)
\end{itemize}

\note[item]{}
\end{frame}
\begin{frame}
\frametitle{Unrelated Title}


\begin{itemize}
\item DATE, TIME, DATETIME, TIMESTAMP, YEAR
\end{itemize}

\note[item]{}
\end{frame}
\begin{frame}
\frametitle{Unrelated Title}


\begin{itemize}
\item If the foreign key contains a value, that value must refer to an existing row in the parent table.
\end{itemize}

\note[item]{}
\end{frame}
\begin{frame}
\frametitle{Unrelated Title}


\begin{itemize}
\item An event and an action.
\end{itemize}

\note[item]{}
\end{frame}
\begin{frame}
\frametitle{Unrelated Title}


\begin{itemize}
\item Delete row from parent and delete matching rows in child.
\end{itemize}

\note[item]{}
\end{frame}
\begin{frame}
\frametitle{Unrelated Title}


\begin{itemize}
\item Delete row from parent and set FK column(s) in child to NULL
\end{itemize}

\note[item]{}
\end{frame}
\begin{frame}
\frametitle{Unrelated Title}


\begin{itemize}
\item You can't delete a given parent row if a child row references the value for that parent row.
\end{itemize}

\note[item]{}
\end{frame}
\begin{frame}
\frametitle{Unrelated Title}


\begin{itemize}
\item Do nothing.
\end{itemize}

\note[item]{}
\end{frame}
\begin{frame}
\frametitle{Unrelated Title}


\begin{itemize}
\item DATABASE PRIVILEGES
\end{itemize}

\note[item]{}
\end{frame}
\begin{frame}
\frametitle{Unrelated Title}


\begin{itemize}
\item In SELECT list and HAVING clause.
\end{itemize}

\note[item]{}
\end{frame}
\begin{frame}
\frametitle{Unrelated Title}


\begin{itemize}
\item A single column.
\end{itemize}

\note[item]{}
\end{frame}
\begin{frame}
\frametitle{Unrelated Title}


\begin{itemize}
\item SUM, AVG and MIN/MAX
\end{itemize}

\note[item]{}
\end{frame}
\begin{frame}
\frametitle{Unrelated Title}


\begin{itemize}
\item The DISTINCT keyword.
\end{itemize}

\note[item]{}
\end{frame}
\begin{frame}
\frametitle{Unrelated Title}


\begin{itemize}
\item WHERE filters individual rows, HAVING filters groups of rows.
\end{itemize}

\note[item]{}
\end{frame}
\begin{frame}
\frametitle{Unrelated Title}


\begin{itemize}
\item Returns records that have matching values in both tables.
\end{itemize}

\note[item]{}
\end{frame}
\begin{frame}
\frametitle{Unrelated Title}


\begin{itemize}
\item A WHERE clause
\end{itemize}

\note[item]{}
\end{frame}
\begin{frame}
\frametitle{Unrelated Title}


\begin{itemize}
\item The WHERE clause.
\end{itemize}

\note[item]{}
\end{frame}
\begin{frame}
\frametitle{Unrelated Title}


\begin{itemize}
\item The WHERE clause.
\end{itemize}

\note[item]{}
\end{frame}
\begin{frame}
\frametitle{Unrelated Title}


\begin{itemize}
\item Subquery
\end{itemize}

\note[item]{}
\end{frame}
\begin{frame}
\frametitle{Unrelated Title}


\begin{itemize}
\item If an underlying table is changed, the view on the tables remain the same.
\end{itemize}

\note[item]{}
\end{frame}
\begin{frame}
\frametitle{Unrelated Title}


\begin{itemize}
\item The same data can be presented in different ways.
\end{itemize}

\note[item]{}
\end{frame}
\begin{frame}
\frametitle{Unrelated Title}


\begin{itemize}
\item A given user may have access to query the view, but no access to base tables.
\end{itemize}

\note[item]{}
\end{frame}
\begin{frame}
\frametitle{Unrelated Title}


\begin{itemize}
\item All columns must refer to a single table (the view cannot join tables).
\end{itemize}

\note[item]{}
\end{frame}
\begin{frame}
\frametitle{Unrelated Title}


\begin{itemize}
\item Aggregate functions and subqueries
\end{itemize}

\note[item]{}
\end{frame}
\begin{frame}
\frametitle{Unrelated Title}


\begin{itemize}
\item Functions and stored procedures
\end{itemize}

\note[item]{}
\end{frame}
\begin{frame}
\frametitle{Unrelated Title}


\begin{itemize}
\item Use stored procedures or functions.
\end{itemize}

\note[item]{}
\end{frame}
\begin{frame}
\frametitle{Unrelated Title}


\begin{itemize}
\item We can isolate data tables.
\end{itemize}

\note[item]{}
\end{frame}
\begin{frame}
\frametitle{Unrelated Title}


\begin{itemize}
\item To execute a stored procedure.
\end{itemize}

\note[item]{}
\end{frame}
\begin{frame}
\frametitle{Unrelated Title}


\begin{itemize}
\item By SELECT, INSERT or UPDATE statements.
\end{itemize}

\note[item]{}
\end{frame}
\begin{frame}
\frametitle{Unrelated Title}


\begin{itemize}
\item By using an explicit CALL statement.
\end{itemize}

\note[item]{}
\end{frame}
\begin{frame}
\frametitle{Unrelated Title}


\begin{itemize}
\item View
\end{itemize}

\note[item]{}
\end{frame}
\begin{frame}
\frametitle{Unrelated Title}


\begin{itemize}
\item Function
\end{itemize}

\note[item]{}
\end{frame}
\begin{frame}
\frametitle{Unrelated Title}

\begin{center}
\includegraphics[width=0.9\textwidth,height=0.9\textheight,keepaspectratio]{/Users/I516998/Library/Application Support/Anki2/User 1/collection.media/paste-21612275433476.jpg}
\end{center}

\begin{itemize}
\item Because it is distincted by branch\_no
\end{itemize}

\note[item]{}
\end{frame}
\begin{frame}
\frametitle{Unrelated Title}


\begin{itemize}
\item Information System
\end{itemize}

\note[item]{}
\end{frame}
\begin{frame}
\frametitle{Unrelated Title}


\begin{itemize}
\item Compares the value of one expression to the value of another expression.
\end{itemize}

\note[item]{}
\end{frame}
\begin{frame}
\frametitle{Unrelated Title}


\begin{itemize}
\item Test whether the value of an expression falls withing a specified range of values.
\end{itemize}

\note[item]{}
\end{frame}
\begin{frame}
\frametitle{Unrelated Title}


\begin{itemize}
\item Test whether the value of an expression equals to one of a set of values.
\end{itemize}

\note[item]{}
\end{frame}
\begin{frame}
\frametitle{Unrelated Title}


\begin{itemize}
\item Test whether a string matches a specified pattern.
\end{itemize}

\note[item]{}
\end{frame}
\begin{frame}
\frametitle{Unrelated Title}


\begin{itemize}
\item Test whether a column has a null (unknown) value.
\end{itemize}

\note[item]{}
\end{frame}
\begin{frame}
\frametitle{Unrelated Title}

\begin{center}
\includegraphics[width=0.9\textwidth,height=0.9\textheight,keepaspectratio]{/Users/I516998/Library/Application Support/Anki2/User 1/collection.media/800px-Logistic-curve.svg.png}
\end{center}

\begin{itemize}
\item \[\sigma(t) = \frac{1}{1+e^{-t}}\]
\end{itemize}

\note[item]{}
\end{frame}
\begin{frame}
\frametitle{Unrelated Title}


\begin{itemize}
\item Sampling the mean of any random variable gives a normal distribution with the same mean and stddev \(\frac{\sigma} {\sqrt N}\).
\end{itemize}

\note[item]{}
\end{frame}
\begin{frame}
\frametitle{Unrelated Title}


\begin{itemize}
\item 1. Relationship is actually linear (y is linear combo of x's)2. No error in x values
\item 3. Errors are independent and4. have constant variance
\end{itemize}

\note[item]{}
\end{frame}
\begin{frame}
\frametitle{Unrelated Title}


\begin{itemize}
\item The chance the variable doesn't matter.
\end{itemize}

\note[item]{}
\end{frame}
\begin{frame}
\frametitle{Unrelated Title}


\begin{itemize}
\item The mean change in the dependent variable for a unit change in an independent variable (i.e., the slope), holding all other variables constant.
\end{itemize}

\note[item]{}
\end{frame}
\begin{frame}
\frametitle{Unrelated Title}


\begin{itemize}
\item Weight: minority instances higher
\item Metrics: precision, recall, F1, normalized accuracy (kappa), AUC...
\end{itemize}

\note[item]{}
\end{frame}
\begin{frame}
\frametitle{Unrelated Title}

\begin{center}
\includegraphics[width=0.9\textwidth,height=0.9\textheight,keepaspectratio]{/Users/I516998/Library/Application Support/Anki2/User 1/collection.media/biasvariance.png}
\end{center}

\begin{itemize}
\item Reducing Bias Increases Variance (and vice-versa)
\end{itemize}

\note[item]{}
\end{frame}
\begin{frame}
\frametitle{Unrelated Title}

\begin{center}
\includegraphics[width=0.9\textwidth,height=0.9\textheight,keepaspectratio]{/Users/I516998/Library/Application Support/Anki2/User 1/collection.media/500px-Ensemble_Bagging.svg.png}
\end{center}

\begin{itemize}
\item Bootstrap aggregatingTrain multiple models on subsamples and average predictions to reduce variance.
\end{itemize}

\note[item]{}
\end{frame}
\begin{frame}
\frametitle{Unrelated Title}

\begin{center}
\includegraphics[width=0.9\textwidth,height=0.9\textheight,keepaspectratio]{/Users/I516998/Library/Application Support/Anki2/User 1/collection.media/1000px-Ensemble_Boosting.svg.png}
\end{center}

\begin{itemize}
\item Sequentially training models, weighting mispredicted samples higher in the next model, and combining by accuracy.
\end{itemize}

\note[item]{}
\end{frame}
\begin{frame}
\frametitle{Unrelated Title}


\begin{itemize}
\item Recursively split the data into groups based on most discriminating feature; each leaf gives a prediction.
\end{itemize}

\note[item]{}
\end{frame}
\begin{frame}
\frametitle{Unrelated Title}


\begin{itemize}
\item \[P(A|B) = \frac{P(B|A) \ \ P(A)}{P(B)}\]
\end{itemize}

\note[item]{}
\end{frame}
\begin{frame}
\frametitle{Unrelated Title}

\begin{center}
\includegraphics[width=0.9\textwidth,height=0.9\textheight,keepaspectratio]{/Users/I516998/Library/Application Support/Anki2/User 1/collection.media/1G3imr4PVeU1SPSsZLW9ghA.png}
\end{center}

\begin{itemize}
\item Linear regression squashed to the range [0, 1] with a logistic function to do binary classification.
\end{itemize}

\note[item]{}
\end{frame}
\begin{frame}
\frametitle{Unrelated Title}

\begin{center}
\includegraphics[width=0.9\textwidth,height=0.9\textheight,keepaspectratio]{/Users/I516998/Library/Application Support/Anki2/User 1/collection.media/350px-Logit.svg.png}
\end{center}

\begin{itemize}
\item \[logit(p) = \ln\frac{p}{1 - p}\]Transforms probabilites to log-odds.
\end{itemize}

\note[item]{}
\end{frame}
\begin{frame}
\frametitle{Unrelated Title}


\begin{itemize}
\item Too many variables
\end{itemize}

\note[item]{}
\end{frame}
\begin{frame}
\frametitle{Unrelated Title}


\begin{itemize}
\item Cross-validation of metrics (accuracy, precision, recall, F1, AUC...)\(\chi^2\) between labels and predictionsLikelihood ratioWald test
\end{itemize}

\note[item]{}
\end{frame}
\begin{frame}
\frametitle{Unrelated Title}

\begin{center}
\includegraphics[width=0.9\textwidth,height=0.9\textheight,keepaspectratio]{/Users/I516998/Library/Application Support/Anki2/User 1/collection.media/558px-KMeans-Gaussian-data.svg.png}
\end{center}

\begin{itemize}
\item Start with k random samples as centroidsAssign each sample to nearest centroidRecompute centroids (average of cluster samples)Repeat until stationary
\end{itemize}

\note[item]{}
\end{frame}
\begin{frame}
\frametitle{Unrelated Title}

\begin{center}
\includegraphics[width=0.9\textwidth,height=0.9\textheight,keepaspectratio]{/Users/I516998/Library/Application Support/Anki2/User 1/collection.media/clustering-elbow-method.jpg}
\end{center}

\begin{itemize}
\item The "elbow method"
\end{itemize}

\note[item]{}
\end{frame}
\begin{frame}
\frametitle{Unrelated Title}

\begin{center}
\includegraphics[width=0.9\textwidth,height=0.9\textheight,keepaspectratio]{/Users/I516998/Library/Application Support/Anki2/User 1/collection.media/dbscan.png}
\end{center}

\begin{itemize}
\item Finds areas of high density (many neighbors)
\item Grows clusters of core points having at least m neighbors within distance epsPlus border points within epsOther points are outliers (noise)
\end{itemize}

\note[item]{}
\end{frame}
\begin{frame}
\frametitle{Unrelated Title}

\begin{center}
\includegraphics[width=0.9\textwidth,height=0.9\textheight,keepaspectratio]{/Users/I516998/Library/Application Support/Anki2/User 1/collection.media/spectral-clustering.png}
\end{center}

\begin{itemize}
\item Compute pairwise distance matrixReduce dimensionality (PCA)Use k-means
\end{itemize}

\note[item]{}
\end{frame}
\begin{frame}
\frametitle{Unrelated Title}

\begin{center}
\includegraphics[width=0.9\textwidth,height=0.9\textheight,keepaspectratio]{/Users/I516998/Library/Application Support/Anki2/User 1/collection.media/hierarchical-clustering.png}
\end{center}

\begin{itemize}
\item Usually bottom up (agglomerative):
\item Start with all points as clustersMerge close clusters (distance: sum-squared, mean, max)Repeat until target \# clusters remain.
\end{itemize}

\note[item]{}
\end{frame}
\begin{frame}
\frametitle{Unrelated Title}

\begin{center}
\includegraphics[width=0.9\textwidth,height=0.9\textheight,keepaspectratio]{/Users/I516998/Library/Application Support/Anki2/User 1/collection.media/cluster-evaluation.jpg}
\end{center}

\begin{itemize}
\item Compare within-cluster distances to between-cluster distances.
\end{itemize}

\note[item]{}
\end{frame}
\begin{frame}
\frametitle{Unrelated Title}


\begin{itemize}
\item Significance testing (hypothesis testing).  
\item Try two versions of something, compare their metrics with a significance test.
\end{itemize}

\note[item]{}
\end{frame}
\begin{frame}
\frametitle{Unrelated Title}


\begin{itemize}
\item \(\chi^2\) with 1 d.o.f.
\end{itemize}

\note[item]{}
\end{frame}
\begin{frame}
\frametitle{Unrelated Title}

\begin{center}
\includegraphics[width=0.9\textwidth,height=0.9\textheight,keepaspectratio]{/Users/I516998/Library/Application Support/Anki2/User 1/collection.media/210px-Gaussian_distribution.svg.png}
\end{center}

\begin{itemize}
\item Welch's t-test
\end{itemize}

\note[item]{}
\end{frame}
\begin{frame}
\frametitle{Unrelated Title}


\begin{itemize}
\item \(\chi^2\)with \(k - 1\) d.o.f. for \(k\) counts/classes.
\end{itemize}

\note[item]{}
\end{frame}
\begin{frame}
\frametitle{Unrelated Title}


\begin{itemize}
\item Independent samples
\end{itemize}

\note[item]{}
\end{frame}
\begin{frame}
\frametitle{Unrelated Title}


\begin{itemize}
\item \[t = \frac{\text{difference of means}}{\text{std err of mean}} = \frac{\bar{x} - \mu}{\sigma / \sqrt{N}}\]
\end{itemize}

\note[item]{}
\end{frame}
\begin{frame}
\frametitle{Unrelated Title}


\begin{itemize}
\item \[t = \frac{\text{difference of means}}{\text{std err of mean}} = \frac{\overline{x_1} - \overline{x_2}}{\sqrt{\frac{\sigma_1^2}{N_1} + \frac{\sigma_2^2}{N_2}}}\]
\end{itemize}

\note[item]{}
\end{frame}
\begin{frame}
\frametitle{Unrelated Title}


\begin{itemize}
\item The probability that, given no significance, you mistakenly find significance.
\end{itemize}

\note[item]{}
\end{frame}
\begin{frame}
\frametitle{Unrelated Title}


\begin{itemize}
\item Some tests will falsely show significance by random chance.
\end{itemize}

\note[item]{}
\end{frame}
\begin{frame}
\frametitle{Unrelated Title}


\begin{itemize}
\item You may get falsely significant results, as you are effectively running many significance tests (the multiple comparisons problem).
\end{itemize}

\note[item]{}
\end{frame}
\begin{frame}
\frametitle{Unrelated Title}


\begin{itemize}
\item Adding information to a model, often constraints, to avoid overfitting and/or make a problem well-defined.
\end{itemize}

\note[item]{}
\end{frame}
\begin{frame}
\frametitle{Unrelated Title}


\begin{itemize}
\item Adding a term for the L2 norm of the weights to the loss function of a model.
\item Penalizes large weights to reduce variance and overfitting.
\end{itemize}

\note[item]{}
\end{frame}
\begin{frame}
\frametitle{Unrelated Title}


\begin{itemize}
\item Adding a term for the L1 norm of the weights to the loss function of a model.
\item Penalizes a large number of (non-zero) weights, for feature selection and simpler models.
\end{itemize}

\note[item]{}
\end{frame}
\begin{frame}
\frametitle{Unrelated Title}

\begin{center}
\includegraphics[width=0.9\textwidth,height=0.9\textheight,keepaspectratio]{/Users/I516998/Library/Application Support/Anki2/User 1/collection.media/gradient-descent.png}
\end{center}

\begin{itemize}
\item Minimize the loss function with respect to the weights by taking small steps in the opposite direction of the gradient.
\end{itemize}

\note[item]{}
\end{frame}
\begin{frame}
\frametitle{Unrelated Title}

\begin{center}
\includegraphics[width=0.9\textwidth,height=0.9\textheight,keepaspectratio]{/Users/I516998/Library/Application Support/Anki2/User 1/collection.media/sgd.png}
\end{center}

\begin{itemize}
\item Like gradient descent, but at at each step, compute the gradient and update the weights using only a small number of random samples.
\end{itemize}

\note[item]{}
\end{frame}
\begin{frame}
\frametitle{Unrelated Title}


\begin{itemize}
\item Training just stores samples for fast neighbor lookup (kd-tree).  Prediction finds the k nearest neighbors and combines their labels (majority for classification, average for regression)
\end{itemize}

\note[item]{}
\end{frame}
\begin{frame}
\frametitle{Unrelated Title}


\begin{itemize}
\item Volume of the space grows exponentially, requring exponentially more data to learn functions.
\end{itemize}

\note[item]{}
\end{frame}
\begin{frame}
\frametitle{Unrelated Title}


\begin{itemize}
\item \[ -\frac{1}{N}\sum_{i=1}^n {y_i\log(p_i) + (1 - y_i)\log(1 - p_i)} \]
\end{itemize}

\note[item]{}
\end{frame}
\begin{frame}
\frametitle{Unrelated Title}


\begin{itemize}
\item \[ \frac{\text{true predicted positives}}{\text{all predicted positives}} \]\[ \frac{TP}{TP + FP} \]
\end{itemize}

\note[item]{}
\end{frame}
\begin{frame}
\frametitle{Unrelated Title}


\begin{itemize}
\item True Positive Rate\[ \frac{\text{correctly predicted positives}}{\text{actual positives}} \]\[ \frac{TP}{TP + FN} \]
\end{itemize}

\note[item]{}
\end{frame}
\begin{frame}
\frametitle{Unrelated Title}


\begin{itemize}
\item True Negative Rate\[ \frac{TN}{TN + FP} \]
\end{itemize}

\note[item]{}
\end{frame}
\begin{frame}
\frametitle{Unrelated Title}


\begin{itemize}
\item F1 = \( 2 \frac{precision * recall}{precision + recall} \)
\end{itemize}

\note[item]{}
\end{frame}
\begin{frame}
\frametitle{Unrelated Title}


\begin{itemize}
\item Plot of true positive rate against false positive rate.
\end{itemize}

\note[item]{}
\end{frame}
\begin{frame}
\frametitle{Unrelated Title}


\begin{itemize}
\item Area under the ROC curve.
\item The probability that a random positive sample scores higher than a random negative sample.
\end{itemize}

\note[item]{}
\end{frame}
\begin{frame}
\frametitle{Unrelated Title}

\begin{center}
\includegraphics[width=0.9\textwidth,height=0.9\textheight,keepaspectratio]{/Users/I516998/Library/Application Support/Anki2/User 1/collection.media/rfc_vs_dt1.png}
\end{center}

\begin{itemize}
\item Bagging multiple decision trees,randomly sampling both the samples and features
\end{itemize}

\note[item]{}
\end{frame}
\begin{frame}
\frametitle{Unrelated Title}


\begin{itemize}
\item Probability that a test finds significance given it is actually there.
\end{itemize}

\note[item]{}
\end{frame}
\begin{frame}
\frametitle{Unrelated Title}

\begin{center}
\includegraphics[width=0.9\textwidth,height=0.9\textheight,keepaspectratio]{/Users/I516998/Library/Application Support/Anki2/User 1/collection.media/849px-Singular-Value-Decomposition.svg.png}
\end{center}

\begin{itemize}
\item Factoring a matrix $M$ into $U\Sigma V^T$  where\[M = m \times n \\ U = \text{orthogonal } m \times m \\ \Sigma = \text{singular values (diagonal } m \times n )\\ V = \text{principal components (orthogonal } n \times n )\]Geometrically: a rotation / stretch / rotation onto the principal components.
\end{itemize}

\note[item]{}
\end{frame}
\begin{frame}
\frametitle{Unrelated Title}


\begin{itemize}
\item \[\text{P}(A \cap B) = \text{P}(A \mid B) \cdot \text{P}(B) \\
= \text{P}(B \mid A) \cdot \text{P}(A)\]
\end{itemize}

\note[item]{}
\end{frame}
\begin{frame}
\frametitle{Unrelated Title}


\begin{itemize}
\item The probability of exactly $k$ successes in $n$ independent trials, each with probability $p$
\item \[ {n \choose k} p^k(1-p)^{n-k} \\ \]\[ \mu = np \\ \]\[ \sigma^2 = np(1 - p) \]
\end{itemize}

\note[item]{}
\end{frame}
\begin{frame}
\frametitle{Unrelated Title}


\begin{itemize}
\item The probability of exactly \(k\) independent events in an interval, given an average of \( \lambda \) events per interval
\end{itemize}

\note[item]{}
\end{frame}
\begin{frame}
\frametitle{Unrelated Title}


\begin{itemize}
\item Probability of \(k\) independent trials to reach one success, each with probability \(p\)
\end{itemize}

\note[item]{}
\end{frame}
\begin{frame}
\frametitle{Unrelated Title}


\begin{itemize}
\item Probability of \(k\) successes before \(r\) failures, each independent with probability \(p\)
\end{itemize}

\note[item]{}
\end{frame}
\begin{frame}
\frametitle{Unrelated Title}


\begin{itemize}
\item Probability of \(k\) successes in \(n\) draws without replacement from \(N\) items containing \(K\) successes
\end{itemize}

\note[item]{}
\end{frame}
\begin{frame}
\frametitle{Unrelated Title}


\begin{itemize}
\item \[ \frac{1}{\sigma\sqrt{2\pi}}e^{-\frac{1}{2}\left ( \frac{x - \mu}{\sigma} \right )^2} \]
\end{itemize}

\note[item]{}
\end{frame}
\begin{frame}
\frametitle{Unrelated Title}

\begin{center}
\includegraphics[width=0.9\textwidth,height=0.9\textheight,keepaspectratio]{/Users/I516998/Library/Application Support/Anki2/User 1/collection.media/linear-independence.png}
\end{center}

\begin{itemize}
\item No vector can be written as a linear combination of the others.
\end{itemize}

\note[item]{}
\end{frame}
\begin{frame}
\frametitle{Unrelated Title}


\begin{itemize}
\item The dot product is zero.
\end{itemize}

\note[item]{}
\end{frame}
\begin{frame}
\frametitle{Unrelated Title}


\begin{itemize}
\item The set of all linear combinations of the vectors.
\end{itemize}

\note[item]{}
\end{frame}
\begin{frame}
\frametitle{Unrelated Title}


\begin{itemize}
\item The dimension of the space spanned by the columnsor The number of linearly independent vectors among the columns
\end{itemize}

\note[item]{}
\end{frame}
\begin{frame}
\frametitle{Unrelated Title}

\begin{center}
\includegraphics[width=0.9\textwidth,height=0.9\textheight,keepaspectratio]{/Users/I516998/Library/Application Support/Anki2/User 1/collection.media/Eigenvector-1.png}
\end{center}

\begin{itemize}
\item The vectors for which matrix multiplication is the same as scalar multiplication.
\end{itemize}

\note[item]{}
\end{frame}
\begin{frame}
\frametitle{Unrelated Title}


\begin{itemize}
\item A set of linearly independent vectors that span the space(\(n\) lin. indep. vectors for an \(n\)-dimensional space)
\end{itemize}

\note[item]{}
\end{frame}
\begin{frame}
\frametitle{Unrelated Title}


\begin{itemize}
\item The equation \(Ax = 0\) has only the trivial solution \(x = 0\).
\end{itemize}

\note[item]{}
\end{frame}
\begin{frame}
\frametitle{Unrelated Title}


\begin{itemize}
\item SVMs find the hyperplane of maximum distance (margin) from the nearest samples of each class (the "support vectors").
\end{itemize}

\note[item]{}
\end{frame}
\begin{frame}
\frametitle{Unrelated Title}

\begin{center}
\includegraphics[width=0.9\textwidth,height=0.9\textheight,keepaspectratio]{/Users/I516998/Library/Application Support/Anki2/User 1/collection.media/PCA.jpg}
\end{center}

\begin{itemize}
\item PCA finds a "rotation" (coordinate system, basis) of data to linearly independent features ordered by highest possible variance.
\end{itemize}

\note[item]{}
\end{frame}
\begin{frame}
\frametitle{Unrelated Title}


\begin{itemize}
\item Given the SVD of \(X = U\Sigma V^T\):
\end{itemize}

\note[item]{}
\end{frame}
\begin{frame}
\frametitle{Unrelated Title}


\begin{itemize}
\item The "elbow method" Plot explained variance vs. number of components and pick a point of diminishing returns.
\end{itemize}

\note[item]{}
\end{frame}
\begin{frame}
\frametitle{Unrelated Title}


\begin{itemize}
\item \[\operatorname{cov}(X,Y) = \operatorname{E}{\big[(X - \operatorname{E}[X])(Y - \operatorname{E}[Y])\big]} \\ 
= \operatorname{E}\left[X Y\right] - \operatorname{E}\left[X\right] \operatorname{E}\left[Y\right] \]For discrete data \(x_i\) and \(y_i\):\[ \frac{1}{N} \sum_i{(x_i - \bar{x})(y_i - \bar{y})} \]
\end{itemize}

\note[item]{}
\end{frame}
\begin{frame}
\frametitle{Unrelated Title}


\begin{itemize}
\item Algorithms with built-in selection
\end{itemize}

\note[item]{}
\end{frame}
\begin{frame}
\frametitle{Unrelated Title}


\begin{itemize}
\item Computing something many times on random subsamples of the data (with replacement) 
\item To estimate uncertainty or reduce variance
\end{itemize}

\note[item]{}
\end{frame}
\begin{frame}
\frametitle{Unrelated Title}


\begin{itemize}
\item Sensitive to the scale of featuresAssumes features with less variance are less importantAssumes (Gaussian) variance accurately characterizes featuresAssumes features are orthogonalOnly performs linear transformations (but see kernel PCA)Only removes linear correlation
\end{itemize}

\note[item]{}
\end{frame}
\begin{frame}
\frametitle{Unrelated Title}


\begin{itemize}
\item The cosine of the angle between two vectors.\[ \cos(\theta) = \frac{a \cdot b}{||a|| ||b||} \]1 for parallel vectors0 for orthogonal vectors-1 for opposite direction
\end{itemize}

\note[item]{}
\end{frame}
\begin{frame}
\frametitle{Unrelated Title}


\begin{itemize}
\item Compute the probability of each class given a sample, using Bayes' Theorem and assumping independenceReturn the most likely class
\item P(C_k \mid x) &= \frac{P(x \mid C_k)\ P(C_k)}{P(x)} && \text{Bayes Theorem} \\ 
\item &= \frac{P(C_k)\ P(x_1 \mid C_k) \cdots P(x_n \mid C_k)}{P(x_1) \cdots P(x_n)} && \text{Assuming independent } x_i 
\item \end{align} \]
\end{itemize}

\note[item]{}
\end{frame}
\begin{frame}
\frametitle{Unrelated Title}

\begin{center}
\includegraphics[width=0.9\textwidth,height=0.9\textheight,keepaspectratio]{/Users/I516998/Library/Application Support/Anki2/User 1/collection.media/power-law.png}
\end{center}

\begin{itemize}
\item \[ f(x) = cx^{-\alpha} \]
\end{itemize}

\note[item]{}
\end{frame}
\begin{frame}
\frametitle{Unrelated Title}


\begin{itemize}
\item View it as a \(t\)-test of significance of the difference \(\delta\):\[ \frac{\delta}{\sigma \over \sqrt{n}} \gt t \]
\end{itemize}

\note[item]{}
\end{frame}
\begin{frame}
\frametitle{Unrelated Title}


\begin{itemize}
\item \[ \mu \pm z\sigma \]or in general
\end{itemize}

\note[item]{}
\end{frame}
\begin{frame}
\frametitle{Unrelated Title}


\begin{itemize}
\item \[ P(A \mid B) = \frac{P(A \cap B)}{P(B)} \]
\end{itemize}

\note[item]{}
\end{frame}
\begin{frame}
\frametitle{Unrelated Title}


\begin{itemize}
\item \[ \frac{\text{nearest_dist - within_dist}}{\max(\text{nearest_dist, within_dist})} \]
\end{itemize}

\note[item]{}
\end{frame}
\begin{frame}
\frametitle{Unrelated Title}


\begin{itemize}
\item It lets you approximate most any random variable with a normal variable, and quantifies the error of the approximation.
\end{itemize}

\note[item]{}
\end{frame}
\begin{frame}
\frametitle{Unrelated Title}


\begin{itemize}
\item The basis formed by the \(k\) largest principal components minimizes the least-squares projection error in \(k\) dimensions.
\end{itemize}

\note[item]{}
\end{frame}
\begin{frame}
\frametitle{Unrelated Title}


\begin{itemize}
\item The eigenvectors and values of the covariance matrix \(X^TX\)
\end{itemize}

\note[item]{}
\end{frame}
\begin{frame}
\frametitle{Unrelated Title}

\begin{center}
\includegraphics[width=0.9\textwidth,height=0.9\textheight,keepaspectratio]{/Users/I516998/Library/Application Support/Anki2/User 1/collection.media/chi-square.png}
\end{center}

\begin{itemize}
\item Compares an actual discrete distribution \(a_i\) to an expected distribution \(x_i\).\[ \chi^2 = \sum^k_{i=1}{\frac{(x_i-a_i)^2}{x_i}} \]
\end{itemize}

\note[item]{}
\end{frame}
\begin{frame}
\frametitle{Unrelated Title}


\begin{itemize}
\item Covariance normalized by the standard deviations\[Corr(X,Y) = \frac{Cov(X, Y)}{\sigma_X \sigma_Y} = 
\frac{\operatorname{E}{\big[(X - \operatorname{E}[X])(Y - \operatorname{E}[Y])\big]}}{\sigma_X \sigma_Y}\]For discrete data \(x_i\) and \(y_i\):\[ \frac{\sum (x_i-\bar{x})(y_i-\bar{y})} {\sqrt{\sum (x_i-\bar{x})^2 \sum (y_i-\bar{y})^2 }} \]
\end{itemize}

\note[item]{}
\end{frame}
\begin{frame}
\frametitle{Unrelated Title}


\begin{itemize}
\item InterpretableMinimal feature engineeringNonlinear modelProvides feature importancePrediction is efficient
\end{itemize}

\note[item]{}
\end{frame}
\begin{frame}
\frametitle{Unrelated Title}


\begin{itemize}
\item Prone to overfitting (high variance)Hard to learn some simple functions (parity, xor)Decision boundaries always parallel to axes
\end{itemize}

\note[item]{}
\end{frame}
\begin{frame}
\frametitle{Unrelated Title}


\begin{itemize}
\item Can model non-linear relationships with kernelsFinds "best" model for given hyperparams, since error function has global minimum
\end{itemize}

\note[item]{}
\end{frame}
\begin{frame}
\frametitle{Unrelated Title}


\begin{itemize}
\item Sensitive to feature scalingSuffers from class imbalanceDistance metrics lose usefulness in high dimensions (curse of dimensionality)
\end{itemize}

\note[item]{}
\end{frame}
\begin{frame}
\frametitle{Unrelated Title}


\begin{itemize}
\item InterpretableEfficientCompact modelsWorks with small data
\end{itemize}

\note[item]{}
\end{frame}
\begin{frame}
\frametitle{Unrelated Title}


\begin{itemize}
\item Assumes features are independent (given the class)Assumes a distribution for continuous features (usually normal)Does not handle sparse data well
\item Fixed-sized model; diminishing returns with more data
\end{itemize}

\note[item]{}
\end{frame}
\begin{frame}
\frametitle{Unrelated Title}


\begin{itemize}
\item Slow and large in both training and prediction; don't scale wellNot great with multiclass problems
\end{itemize}

\note[item]{}
\end{frame}
\begin{frame}
\frametitle{Unrelated Title}


\begin{itemize}
\item <your response here>
\end{itemize}

\note[item]{}
\end{frame}
\begin{frame}
\frametitle{Unrelated Title}


\begin{itemize}
\item <your response here>
\end{itemize}

\note[item]{}
\end{frame}
\begin{frame}
\frametitle{Unrelated Title}


\begin{itemize}
\item <your response here>
\end{itemize}

\note[item]{}
\end{frame}
\begin{frame}
\frametitle{Unrelated Title}


\begin{itemize}
\item <your response here>
\end{itemize}

\note[item]{}
\end{frame}
\begin{frame}
\frametitle{Unrelated Title}


\begin{itemize}
\item <your response here>
\end{itemize}

\note[item]{}
\end{frame}
\begin{frame}
\frametitle{Unrelated Title}


\begin{itemize}
\item <your response here>
\end{itemize}

\note[item]{}
\end{frame}
\begin{frame}
\frametitle{Unrelated Title}


\begin{itemize}
\item <your response here>
\end{itemize}

\note[item]{}
\end{frame}
\begin{frame}
\frametitle{Unrelated Title}


\begin{itemize}
\item <your response here>
\end{itemize}

\note[item]{}
\end{frame}
\begin{frame}
\frametitle{Unrelated Title}


\begin{itemize}
\item <your response here>
\end{itemize}

\note[item]{}
\end{frame}
\begin{frame}
\frametitle{Unrelated Title}

\begin{center}
\includegraphics[width=0.9\textwidth,height=0.9\textheight,keepaspectratio]{/Users/I516998/Library/Application Support/Anki2/User 1/collection.media/neuron.jpg}
\end{center}

\begin{itemize}
\item Applies an activation function to the weighted sum of its inputs.
\end{itemize}

\note[item]{}
\end{frame}
\begin{frame}
\frametitle{Unrelated Title}


\begin{itemize}
\item Successive layers of neurons each recieve the same inputs, apply weights and an activation function, and produce an output.  The outputs of one layer are the inputs of the next.
\end{itemize}

\note[item]{}
\end{frame}
\begin{frame}
\frametitle{Unrelated Title}


\begin{itemize}
\item Backpropagation iteratively updates the weights of a neural network to minimize the error between the actual and desired outputs.
\end{itemize}

\note[item]{}
\end{frame}
\begin{frame}
\frametitle{Unrelated Title}

\begin{center}
\includegraphics[width=0.9\textwidth,height=0.9\textheight,keepaspectratio]{/Users/I516998/Library/Application Support/Anki2/User 1/collection.media/kernel-trick.png}
\end{center}

\begin{itemize}
\item Computing the results of vector dot products after non-linear mappings into high-dimesional space using only dot products in a low-dimensonal input space.
\end{itemize}

\note[item]{}
\end{frame}
\begin{frame}
\frametitle{Unrelated Title}


\begin{itemize}
\item the hardware
\end{itemize}

\note[item]{}
\end{frame}
\begin{frame}
\frametitle{Unrelated Title}


\begin{itemize}
\item Layers are designed so that each layer will only use functions and services of the layers below it. One layer can be designed, implemented, and debugged before any higher-level layers make use of its features.
\end{itemize}

\note[item]{}
\end{frame}
\begin{frame}
\frametitle{Unrelated Title}


\begin{itemize}
\item Layered OS implementations tend to be less efficient; when a high-level layer uses a low-level feature, control of execution may pass between several layers; this passing of control may involve duplication of data, passing of parameters, etc. Each layer adds some amount of overhead.
\end{itemize}

\note[item]{}
\end{frame}
\begin{frame}
\frametitle{Unrelated Title}


\begin{itemize}
\item The Mach microkernel, developed by researches at Carnegie Mellon in the 1980's.
\end{itemize}

\note[item]{}
\end{frame}
\begin{frame}
\frametitle{Unrelated Title}


\begin{itemize}
\item All non-critical system features are removed from the kernel and implemented as system or user-level programs. These programs communicate via message passing, sending and recieving messages between services. Message passing is facilitated by the kernel, which must offer the interface for interprocess communication (IPC).
\end{itemize}

\note[item]{}
\end{frame}
\begin{frame}
\frametitle{Unrelated Title}


\begin{itemize}
\item message passing
\end{itemize}

\note[item]{}
\end{frame}
\begin{frame}
\frametitle{Unrelated Title}


\begin{itemize}
\item 1. Ease of extending functionality (by adding services without modifying kernel code).2. Ease of porting from one hardware architecture to another.3. Greater security and reliability (more code runs in user spacei).
\end{itemize}

\note[item]{}
\end{frame}
\begin{frame}
\frametitle{Unrelated Title}


\begin{itemize}
\item Microkernels can suffer from performance decreases due to increased system function overhead (frequency of context-switching back-and-forth between kernel and user space when passing messages, etc).
\end{itemize}

\note[item]{}
\end{frame}
\begin{frame}
\frametitle{Unrelated Title}


\begin{itemize}
\item 1. Layered approach.2. Microkernel approach.3. Modular approach (dynamically loaded modules).
\end{itemize}

\note[item]{}
\end{frame}
\begin{frame}
\frametitle{Unrelated Title}


\begin{itemize}
\item macOS takes a hybrid approach, using both layered and microkernel design techniques. It incorporates three primary components:1. The Mach microkernel, which provides (a) memory management, (b) IPC and RPC communications, and (c) thread scheduling.2. The BSD kernel, which provides POSIX APIs (including Pthreads), a BSD command line interface, and support for network and file systems.3. Application environments and common services, including a graphical interface.
\end{itemize}

\note[item]{}
\end{frame}
\begin{frame}
\frametitle{Unrelated Title}


\begin{itemize}
\item An abstraction of one computer's hardware (CPU, memory, storage devices, etc) into multiple homogenous execution environments—creating the illusion that each environment posesses its own private computer.
\end{itemize}

\note[item]{}
\end{frame}
\begin{frame}
\frametitle{Unrelated Title}


\begin{itemize}
\item user mode
\end{itemize}

\note[item]{}
\end{frame}
\begin{frame}
\frametitle{Unrelated Title}


\begin{itemize}
\item virtual kernel mode
\end{itemize}

\note[item]{}
\end{frame}
\begin{frame}
\frametitle{Unrelated Title}


\begin{itemize}
\item Execution control moves to the virtual machine monitor running in virtual kernel mode (still in user mode). The monitor program may change the register contents and program counter for the virtual machine to simulate the effect of the system call.
\end{itemize}

\note[item]{}
\end{frame}
\begin{frame}
\frametitle{Unrelated Title}


\begin{itemize}
\item 1. VMware: Abstracts Intel x86 architecture into isolated virtual machines. Allows a host operating system (with VMware running) to also run multiple virtualized operating systems on the same machine. Each VM recieves its own virtual CPU, virtual memory, virtual devices, etc.2. The Java virtual machine (JVM): Consists of (a) the Java class loader, which verifies a class's architecture-independent bytecode, and (b) the Java interpreter. The JVM automatically manages memory using a garbage collector.3. Microsoft .NET Framework: Provides a virtual machine as an intermediary between an executing .NET program and the underlying hardware architecture. The virtual machine is implemented as the Common Language Runtime (CLR). Programs written in C# or VB.NET are compiled into architecture-agnostic intermediate "binaries" in the form of Microsoft Intermediate Language (MS-IL); the CLR loads these binaries and executes them on the underlying hardware using just-in-time (JIT) compilation.
\end{itemize}

\note[item]{}
\end{frame}
\begin{frame}
\frametitle{Unrelated Title}


\begin{itemize}
\item 1. What are the features of the CPU(s)—extended instruction set, floating-point operations, etc?2. How much physical memory is available?3. What hardware devices are available?4. OS-specific options (CPU-scheduling algorithm, process limit, etc).
\end{itemize}

\note[item]{}
\end{frame}
\begin{frame}
\frametitle{Unrelated Title}


\begin{itemize}
\item 1. Run hardware diagnostics.2. Initialize CPU register values, device controllers, memory contents, etc.3. Locate the kernel image (stored on a boot volume).4. Load the kernel into main memory.5. Start the kernel.
\end{itemize}

\note[item]{}
\end{frame}
\begin{frame}
\frametitle{Unrelated Title}


\begin{itemize}
\item The first instruction is fetched from a pre-defined address—usually the start of the bootstrap program. The memory location is a location in read-only memory (ROM), which stores the initial bootstrap program.
\end{itemize}

\note[item]{}
\end{frame}
\begin{frame}
\frametitle{Unrelated Title}


\begin{itemize}
\item RAM (SDRAM) forms each memory cell from a transistor and a capacitor. RAM cards can be read more quickly than ROM cards, which use a floating gate and control gate to store each persistent bit of data.
\end{itemize}

\note[item]{}
\end{frame}
\begin{frame}
\frametitle{Unrelated Title}


\begin{itemize}
\item Often, the initial bootstrap loader will read and copy a single block (block zero) from disk into memory. This block is known as the boot block.The code that resides in the boot block is often responsible for loading the remainder of the bootstrap program into main memory from elsewhere on disk.
\end{itemize}

\note[item]{}
\end{frame}
\begin{frame}
\frametitle{Unrelated Title}


\begin{itemize}
\item "boot disk" (or "system disk")
\end{itemize}

\note[item]{}
\end{frame}
\begin{frame}
\frametitle{Unrelated Title}


\begin{itemize}
\item A program in execution.
\end{itemize}

\note[item]{}
\end{frame}
\begin{frame}
\frametitle{Unrelated Title}


\begin{itemize}
\item jobs
\end{itemize}

\note[item]{}
\end{frame}
\begin{frame}
\frametitle{Unrelated Title}


\begin{itemize}
\item 1. A text section (i.e., program code).2. The data section (i.e., global variables).3. The state of the processor registers (including the program counter).4. A stack.5. A heap.
\end{itemize}

\note[item]{}
\end{frame}
\begin{frame}
\frametitle{Unrelated Title}


\begin{itemize}
\item 1. New (created).2. Running (currently executing its code).3. Waiting (on some event or signal).4. Ready (waiting to become the active process).5. Terminated (finished).
\end{itemize}

\note[item]{}
\end{frame}
\begin{frame}
\frametitle{Unrelated Title}


\begin{itemize}
\item Process control block (PCB) (also called a "task control block").
\end{itemize}

\note[item]{}
\end{frame}
\begin{frame}
\frametitle{Unrelated Title}


\begin{itemize}
\item 1. The process's current state.2. A program counter.3. The state of the CPU registers.4. CPU scheduling information (priority, queue pointer, etc).5. Memory management information (page tables, etc).6. I/O status information (list of open files, etc).7. Accounting information.
\end{itemize}

\note[item]{}
\end{frame}
\begin{frame}
\frametitle{Unrelated Title}


\begin{itemize}
\item the job queue.
\end{itemize}

\note[item]{}
\end{frame}
\begin{frame}
\frametitle{Unrelated Title}


\begin{itemize}
\item a ready queue.
\end{itemize}

\note[item]{}
\end{frame}
\begin{frame}
\frametitle{Unrelated Title}


\begin{itemize}
\item device queue.
\end{itemize}

\note[item]{}
\end{frame}
\begin{frame}
\frametitle{Unrelated Title}


\begin{itemize}
\item dispatched
\end{itemize}

\note[item]{}
\end{frame}
\begin{frame}
\frametitle{Unrelated Title}


\begin{itemize}
\item CPU-bound or I/O-bound
\end{itemize}

\note[item]{}
\end{frame}
\begin{frame}
\frametitle{Unrelated Title}


\begin{itemize}
\item 1. The values of the CPU registers.2. The process state.3. Memory management information associated with the process.
\end{itemize}

\note[item]{}
\end{frame}
\begin{frame}
\frametitle{Unrelated Title}


\begin{itemize}
\item process tree
\end{itemize}

\note[item]{}
\end{frame}
\begin{frame}
\frametitle{Unrelated Title}


\begin{itemize}
\item The init process.
\end{itemize}

\note[item]{}
\end{frame}
\begin{frame}
\frametitle{Unrelated Title}


\begin{itemize}
\item The ps command.
\end{itemize}

\note[item]{}
\end{frame}
\begin{frame}
\frametitle{Unrelated Title}


\begin{itemize}
\item The -e and -l flags.
\end{itemize}

\note[item]{}
\end{frame}
\begin{frame}
\frametitle{Unrelated Title}


\begin{itemize}
\item 1. The parent continues execution concurrently with the child.2. The parent waits until the child terminates before resuming execution.
\end{itemize}

\note[item]{}
\end{frame}
\begin{frame}
\frametitle{Unrelated Title}


\begin{itemize}
\item The fork() system call.
\end{itemize}

\note[item]{}
\end{frame}
\begin{frame}
\frametitle{Unrelated Title}


\begin{itemize}
\item 1. The parent process recieves the PID of the new child process.2. The child process recieves a NULL (0) value.
\end{itemize}

\note[item]{}
\end{frame}
\begin{frame}
\frametitle{Unrelated Title}


\begin{itemize}
\item address space
\end{itemize}

\note[item]{}
\end{frame}
\begin{frame}
\frametitle{Unrelated Title}


\begin{itemize}
\item To use the child process to load and run a different program (binary).
\end{itemize}

\note[item]{}
\end{frame}
\begin{frame}
\frametitle{Unrelated Title}


\begin{itemize}
\item It removes the (calling) parent process from the ready queue until the child process terminates.
\end{itemize}

\note[item]{}
\end{frame}
\begin{frame}
\frametitle{Unrelated Title}


\begin{itemize}
\item CreateProcess()
\end{itemize}

\note[item]{}
\end{frame}
\begin{frame}
\frametitle{Unrelated Title}


\begin{itemize}
\item The exit() system call.
\end{itemize}

\note[item]{}
\end{frame}
\begin{frame}
\frametitle{Unrelated Title}


\begin{itemize}
\item Deallocation of physical and/or virtual memory, open file handles, I/O buffers, etc.
\end{itemize}

\note[item]{}
\end{frame}
\begin{frame}
\frametitle{Unrelated Title}


\begin{itemize}
\item The init process.
\end{itemize}

\note[item]{}
\end{frame}
\begin{frame}
\frametitle{Unrelated Title}


\begin{itemize}
\item Unlike an independent process, a cooperating process can affect and be affected by other processes executing in the system.
\end{itemize}

\note[item]{}
\end{frame}
\begin{frame}
\frametitle{Unrelated Title}


\begin{itemize}
\item 1. Shared memory: Reading and writing of data to a shared region of memory.2. Message passing Exchanging messaging between cooperating processes.
\end{itemize}

\note[item]{}
\end{frame}
\begin{frame}
\frametitle{Unrelated Title}


\begin{itemize}
\item Shared memory.
\end{itemize}

\note[item]{}
\end{frame}
\begin{frame}
\frametitle{Unrelated Title}


\begin{itemize}
\item One process will request a shared memory segment from the operating system. The memory region will reside in this process' address space. Other processes must attach the region to their own address space.
\end{itemize}

\note[item]{}
\end{frame}
\begin{frame}
\frametitle{Unrelated Title}


\begin{itemize}
\item A producer process produces information (data) that is consumed by a consumer process. The producer and consumer must be synchronized so that the consumer does not try to consume an item that has not yet been produced.
\end{itemize}

\note[item]{}
\end{frame}
\begin{frame}
\frametitle{Unrelated Title}


\begin{itemize}
\item address space
\end{itemize}

\note[item]{}
\end{frame}
\begin{frame}
\frametitle{Unrelated Title}


\begin{itemize}
\item 1. send(message)2. receive(message)
\end{itemize}

\note[item]{}
\end{frame}
\begin{frame}
\frametitle{Unrelated Title}


\begin{itemize}
\item 1. Direct or indirect communcation (is there a message broker?).2. Synchronous or asynchronous communication.3. Symmetrical or asymmetrical message addressing.4. Automatic or explicit message buffering.5. Bounded or unbounded message buffering.6. Blocking or non-blocking send.7. Blocking or non-blocking recieve.
\end{itemize}

\note[item]{}
\end{frame}
\begin{frame}
\frametitle{Unrelated Title}


\begin{itemize}
\item Exactly 2 processes
\end{itemize}

\note[item]{}
\end{frame}
\begin{frame}
\frametitle{Unrelated Title}


\begin{itemize}
\item mailboxes (or ports).
\end{itemize}

\note[item]{}
\end{frame}
\begin{frame}
\frametitle{Unrelated Title}


\begin{itemize}
\item An integer value
\end{itemize}

\note[item]{}
\end{frame}
\begin{frame}
\frametitle{Unrelated Title}


\begin{itemize}
\item With direct communication, links exist between exactly 2 processes. Each process must explicitly identify the other.WIth indirect (mailbox) communication, multiple processes may share the same mailbox. Some arbitrary number of processes may be configured to receive messages from that mailbox.
\end{itemize}

\note[item]{}
\end{frame}
\begin{frame}
\frametitle{Unrelated Title}


\begin{itemize}
\item A valid message, or a null value.
\end{itemize}

\note[item]{}
\end{frame}
\begin{frame}
\frametitle{Unrelated Title}


\begin{itemize}
\item The shmget() system call.
\end{itemize}

\note[item]{}
\end{frame}
\begin{frame}
\frametitle{Unrelated Title}


\begin{itemize}
\item The shmat() system call.
\end{itemize}

\note[item]{}
\end{frame}
\begin{frame}
\frametitle{Unrelated Title}


\begin{itemize}
\item The shmdt() system call.
\end{itemize}

\note[item]{}
\end{frame}
\begin{frame}
\frametitle{Unrelated Title}


\begin{itemize}
\item message passing
\end{itemize}

\note[item]{}
\end{frame}
\begin{frame}
\frametitle{Unrelated Title}


\begin{itemize}
\item 1. Wait indefinitely until there is room in the mailbox.2. Wait at most n milliseconds.3. Do not wait. Return from the send() call immediately.4. Have the kernel temporarily "cache" the message on the process's behalf.
\end{itemize}

\note[item]{}
\end{frame}
\begin{frame}
\frametitle{Unrelated Title}


\begin{itemize}
\item A mailbox set.
\end{itemize}

\note[item]{}
\end{frame}
\begin{frame}
\frametitle{Unrelated Title}


\begin{itemize}
\item 1. msg_send()2. msg_receive()3. msg_rpc()
\end{itemize}

\note[item]{}
\end{frame}
\begin{frame}
\frametitle{Unrelated Title}


\begin{itemize}
\item The port_status() system call.
\end{itemize}

\note[item]{}
\end{frame}
\begin{frame}
\frametitle{Unrelated Title}


\begin{itemize}
\item Small messages (e.g., up to 256 bytes) can be stored directly in the message queue associated with a port. Large messages can be sent by (a) allocating a section object (a region of shared memory) to store the payload, and (b) sending a small message that contains a pointer and size information about the section object.
\end{itemize}

\note[item]{}
\end{frame}
\begin{frame}
\frametitle{Unrelated Title}


\begin{itemize}
\item callback mechanism
\end{itemize}

\note[item]{}
\end{frame}
\begin{frame}
\frametitle{Unrelated Title}


\begin{itemize}
\item An abstract endpoint for communication.
\end{itemize}

\note[item]{}
\end{frame}
\begin{frame}
\frametitle{Unrelated Title}


\begin{itemize}
\item 1. An IP address.2. A port number.
\end{itemize}

\note[item]{}
\end{frame}
\begin{frame}
\frametitle{Unrelated Title}


\begin{itemize}
\item The loopback address.
\end{itemize}

\note[item]{}
\end{frame}
\begin{frame}
\frametitle{Unrelated Title}


\begin{itemize}
\item "marshalling" (or "serializing")
\end{itemize}

\note[item]{}
\end{frame}
\begin{frame}
\frametitle{Unrelated Title}


\begin{itemize}
\item Internally, the two computers—caller and host—may represent data in a different way (big endian vs. little endian systems). Most RPC implementations use an external data representation (XDR) to marshal parameters over the network.
\end{itemize}

\note[item]{}
\end{frame}
\begin{frame}
\frametitle{Unrelated Title}


\begin{itemize}
\item The operating system may provide a rendevzous (or "matchmaker") daemon that listens on a fixed port. The daemon receives messages identifying the intended recipient in a port-agnostic way, and forwards the message to the appropriate recipient port (mailbox).
\end{itemize}

\note[item]{}
\end{frame}
\begin{frame}
\frametitle{Unrelated Title}


\begin{itemize}
\item Threads belonging to the same process share its code section (or "text section"), data section, and other resources such as open file handles and signals.
\end{itemize}

\note[item]{}
\end{frame}
\begin{frame}
\frametitle{Unrelated Title}


\begin{itemize}
\item Every thread recieve its own thread ID, program counter, register set (loaded and restored), and program stack.
\end{itemize}

\note[item]{}
\end{frame}
\begin{frame}
\frametitle{Unrelated Title}


\begin{itemize}
\item single-threaded, multi-threaded
\end{itemize}

\note[item]{}
\end{frame}
\begin{frame}
\frametitle{Unrelated Title}


\begin{itemize}
\item 1. System responsiveness: Certain process behaviors may continue to execute while other tasks are blocked.2. Resource sharing: A thread and its associated process oftens can share the same memory and other system resources.3. Economy: Generally it is more time consuming to create new processes as opposed to creating new threads.4. Processor utilization: In a multiprocessor architecture, we can increase throughput by running multiple threads concurrently on separate processors.
\end{itemize}

\note[item]{}
\end{frame}
\begin{frame}
\frametitle{Unrelated Title}


\begin{itemize}
\item user threads, kernel threads
\end{itemize}

\note[item]{}
\end{frame}
\begin{frame}
\frametitle{Unrelated Title}


\begin{itemize}
\item 1. Many-to-one: Many user-level threads get mapped to a single kernel thread. The entire process blocks if one thread makes a blocking system call. Only one thread may acces the kernel at any given time. Multiple threads cannot run in parallel on multiprocessors.2. One-to-one: Each user thread is mapped to a kernel thread. This improves concurrency, allowing multiple threads to run in parallel on multiprocessors. Users must be careful not to overtax the system with too many paired threads.3. Many-to-many: Many user-level threads are multiplexed to an equal or smaller number of kernel-level threads. This allows for true concurrency, unlike the many-to-one model. When a thread performs a blocking system call, the kernel can schedule another thread for execution.
\end{itemize}

\note[item]{}
\end{frame}
\begin{frame}
\frametitle{Unrelated Title}


\begin{itemize}
\item 1. POSIX Pthreads2. Win32 thread API3. Java thread API*
\end{itemize}

\note[item]{}
\end{frame}
\begin{frame}
\frametitle{Unrelated Title}


\begin{itemize}
\item specification, implementation
\end{itemize}

\note[item]{}
\end{frame}
\begin{frame}
\frametitle{Unrelated Title}


\begin{itemize}
\item In a function specified in the call to pthread_create().
\end{itemize}

\note[item]{}
\end{frame}
\begin{frame}
\frametitle{Unrelated Title}


\begin{itemize}
\item pthread_exit()
\end{itemize}

\note[item]{}
\end{frame}
\begin{frame}
\frametitle{Unrelated Title}


\begin{itemize}
\item The CreateThread() call.
\end{itemize}

\note[item]{}
\end{frame}
\begin{frame}
\frametitle{Unrelated Title}


\begin{itemize}
\item The CloseHandle() call.
\end{itemize}

\note[item]{}
\end{frame}
\begin{frame}
\frametitle{Unrelated Title}


\begin{itemize}
\item The WaitForSingleObject() system call.
\end{itemize}

\note[item]{}
\end{frame}
\begin{frame}
\frametitle{Unrelated Title}


\begin{itemize}
\item 1. Create a class inheriting from the Thread class, and override its run() method.2. Create a class that implements the Runnable interface (defines a run() method).
\end{itemize}

\note[item]{}
\end{frame}
\begin{frame}
\frametitle{Unrelated Title}


\begin{itemize}
\item Its start() method is called.
\end{itemize}

\note[item]{}
\end{frame}
\begin{frame}
\frametitle{Unrelated Title}


\begin{itemize}
\item join() method
\end{itemize}

\note[item]{}
\end{frame}
\begin{frame}
\frametitle{Unrelated Title}


\begin{itemize}
\item 1. The JVM allocates memory and initializes a new thread.2. The JVM calls the object's run() method, marking it as eligible to run.
\end{itemize}

\note[item]{}
\end{frame}
\begin{frame}
\frametitle{Unrelated Title}


\begin{itemize}
\item When it exits (returns) from its run() method.
\end{itemize}

\note[item]{}
\end{frame}
\begin{frame}
\frametitle{Unrelated Title}


\begin{itemize}
\item signals
\end{itemize}

\note[item]{}
\end{frame}
\begin{frame}
\frametitle{Unrelated Title}


\begin{itemize}
\item synchronous signals / asynchronous signals
\end{itemize}

\note[item]{}
\end{frame}
\begin{frame}
\frametitle{Unrelated Title}


\begin{itemize}
\item 1. An illegal memory access.2. An illegal divide-by-zero.
\end{itemize}

\note[item]{}
\end{frame}
\begin{frame}
\frametitle{Unrelated Title}


\begin{itemize}
\item A synchronous signal is immediately recieved by the process that generates the event (e.g., an attempted divide-by-zero instruction).An asynchronous signal is issued by something other than the process (e.g., a timer expiring), and may be handled after some delay.
\end{itemize}

\note[item]{}
\end{frame}
\begin{frame}
\frametitle{Unrelated Title}


\begin{itemize}
\item 1. A keyboard event (e.g., Ctrl-C to terminate the process).2. A timer expiring.
\end{itemize}

\note[item]{}
\end{frame}
\begin{frame}
\frametitle{Unrelated Title}


\begin{itemize}
\item A default signal handler (run by the kernel).
\end{itemize}

\note[item]{}
\end{frame}
\begin{frame}
\frametitle{Unrelated Title}


\begin{itemize}
\item 1. Deliver the signal to the most appropriate thread.2. Deliver the signal to every thread in the process.3. Deliver the signal to certain threads.4. Mark a single thread as the recipient for all signals.
\end{itemize}

\note[item]{}
\end{frame}
\begin{frame}
\frametitle{Unrelated Title}


\begin{itemize}
\item 1. The kill() call.2. The pthread_kill() call.
\end{itemize}

\note[item]{}
\end{frame}
\begin{frame}
\frametitle{Unrelated Title}


\begin{itemize}
\item 1. The time overhead of initializing new threads.2. The finite capacity of the host system (memory, processing power, etc).
\end{itemize}

\note[item]{}
\end{frame}
\begin{frame}
\frametitle{Unrelated Title}


\begin{itemize}
\item thread-specific data
\end{itemize}

\note[item]{}
\end{frame}
\begin{frame}
\frametitle{Unrelated Title}


\begin{itemize}
\item Between a (user) thread and a kernel thread, there often exists an intermediate data structure known as a lightweight thread. This structure represents a virtual processor on which to schedule threads. Each LWP gets paired to an available kernel thread, and each kernel thread will be scheduled to run on a physical processor.
\end{itemize}

\note[item]{}
\end{frame}
\begin{frame}
\frametitle{Unrelated Title}


\begin{itemize}
\item Whenever its underlying kernel thread blocks (e.g., waiting for I/O to complete).
\end{itemize}

\note[item]{}
\end{frame}
\begin{frame}
\frametitle{Unrelated Title}


\begin{itemize}
\item The kernel provides an application with a set of virtual processors (LWPs) on which to schedule threads. Using an upcall mechanism (supported by the thread library), the kernel informs the application about the ocurrences of certain events. The application gets notified when the kernel uses another LWP to run an upcall handler, which also "runs" on one of the virtual processors.
\end{itemize}

\note[item]{}
\end{frame}
\begin{frame}
\frametitle{Unrelated Title}


\begin{itemize}
\item The one-to-one model.
\end{itemize}

\note[item]{}
\end{frame}
\begin{frame}
\frametitle{Unrelated Title}


\begin{itemize}
\item 1. A unique thread ID.2. A register set (representing a processor state).3. A user stack and kernel stack.4. A private storage area (used by various runtime libraries and DLLs).
\end{itemize}

\note[item]{}
\end{frame}
\begin{frame}
\frametitle{Unrelated Title}


\begin{itemize}
\item processes from threads
\end{itemize}

\note[item]{}
\end{frame}
\begin{frame}
\frametitle{Unrelated Title}


\begin{itemize}
\item tasks
\end{itemize}

\note[item]{}
\end{frame}
\begin{frame}
\frametitle{Unrelated Title}


\begin{itemize}
\item clone() allows the programmer to pass a set of flags used to customize the set of resources that will be shared between the parent and child task. fork() does not expose these options. By default, fork() creates a copy in memory of all data structures associated with the parent task, and assigns them to the child task.
\end{itemize}

\note[item]{}
\end{frame}
\begin{frame}
\frametitle{Unrelated Title}


\begin{itemize}
\item 1. Memory address space.2. File system information.3. Signal handlers.4. Open files.
\end{itemize}

\note[item]{}
\end{frame}
\begin{frame}
\frametitle{Unrelated Title}


\begin{itemize}
\item By using pointers in the child task's task control block (PCB) (i.e.—task_struct) that point to the address of parent resources in the same memory address space.
\end{itemize}

\note[item]{}
\end{frame}
\begin{frame}
\frametitle{Unrelated Title}


\begin{itemize}
\item exponential
\end{itemize}

\note[item]{}
\end{frame}
\begin{frame}
\frametitle{Unrelated Title}


\begin{itemize}
\item A higher number of shorter CPU bursts.
\end{itemize}

\note[item]{}
\end{frame}
\begin{frame}
\frametitle{Unrelated Title}


\begin{itemize}
\item A lower number of longer CPU bursts.
\end{itemize}

\note[item]{}
\end{frame}
\begin{frame}
\frametitle{Unrelated Title}


\begin{itemize}
\item The short-term scheduler.
\end{itemize}

\note[item]{}
\end{frame}
\begin{frame}
\frametitle{Unrelated Title}


\begin{itemize}
\item 1. A FIFO queue.2. A priority queue.3. A self-balancing tree.4. An unsorted linked list.
\end{itemize}

\note[item]{}
\end{frame}
\begin{frame}
\frametitle{Unrelated Title}


\begin{itemize}
\item 1. When a running process switches to the "waiting" state (e.g., blocks or yields).2. When a running process switches to the "ready" state (e.g., interrupt occurs).3. When a waiting process switches to the ready state (e.g., I/O operation completes).4. When a process terminates (no longer running).
\end{itemize}

\note[item]{}
\end{frame}
\begin{frame}
\frametitle{Unrelated Title}


\begin{itemize}
\item A non-preemptive (or cooperative) scheduler will schedule a process on a CPU and allow it to run until it either yields (voluntarily) or terminates.A preemptive scheduler may suspend a task that is currently running on a CPU in favor of a different task, in response to some event (e.g., timer interrupt).
\end{itemize}

\note[item]{}
\end{frame}
\begin{frame}
\frametitle{Unrelated Title}


\begin{itemize}
\item shared data (memory)
\end{itemize}

\note[item]{}
\end{frame}
\begin{frame}
\frametitle{Unrelated Title}


\begin{itemize}
\item When a system call is invoked, disable context switching until either the call completes, or until it is blocked (e.g., on some I/O operation).
\end{itemize}

\note[item]{}
\end{frame}
\begin{frame}
\frametitle{Unrelated Title}


\begin{itemize}
\item During the time that interrupts are disabled, the system is unaware of potentially important events, and data loss may occur (i.e., a packet arrives on a network card, a key is pressed on the keyboard by the user, etc).
\end{itemize}

\note[item]{}
\end{frame}
\begin{frame}
\frametitle{Unrelated Title}


\begin{itemize}
\item The dispatcher.
\end{itemize}

\note[item]{}
\end{frame}
\begin{frame}
\frametitle{Unrelated Title}


\begin{itemize}
\item 1. CPU utilization.2. Throughput.3. Turnaround time.4. Waiting time.5. Response time.
\end{itemize}

\note[item]{}
\end{frame}
\begin{frame}
\frametitle{Unrelated Title}


\begin{itemize}
\item The number of processes that are completed per unit of time.
\end{itemize}

\note[item]{}
\end{frame}
\begin{frame}
\frametitle{Unrelated Title}


\begin{itemize}
\item turnaround time
\end{itemize}

\note[item]{}
\end{frame}
\begin{frame}
\frametitle{Unrelated Title}


\begin{itemize}
\item variance
\end{itemize}

\note[item]{}
\end{frame}
\begin{frame}
\frametitle{Unrelated Title}


\begin{itemize}
\item 1. First-come, first-served scheduling (FCFS).2. Shortest-job-first scheduling (SJF).3. Priority scheduling.4. Round-robin scheduling (RR).
\end{itemize}

\note[item]{}
\end{frame}
\begin{frame}
\frametitle{Unrelated Title}


\begin{itemize}
\item A FIFO queue.
\end{itemize}

\note[item]{}
\end{frame}
\begin{frame}
\frametitle{Unrelated Title}


\begin{itemize}
\item The resulting average wait time is generally not minimal, and variance may be significant.
\end{itemize}

\note[item]{}
\end{frame}
\begin{frame}
\frametitle{Unrelated Title}


\begin{itemize}
\item The (estimated) length of the process's next CPU burst.
\end{itemize}

\note[item]{}
\end{frame}
\begin{frame}
\frametitle{Unrelated Title}


\begin{itemize}
\item Average waiting time.
\end{itemize}

\note[item]{}
\end{frame}
\begin{frame}
\frametitle{Unrelated Title}


\begin{itemize}
\item An exponential average of the process's previous n CPU bursts:- Let [$]t_n[/$] be the length of the nth CPU burst.- Let [$]\mathrm{T}_{(n+1)}[/$] be the predicted value for the next CPU burst.- Let [$]\mathrm{A}[/$] determine the relative weight of recent vs. past measurements.For [$]0 \leq \mathrm{A} \leq 1[/$], [$]\mathrm{T}_{(n+1)} = \mathrm{A}*t_n + (1-\mathrm{A})*\mathrm{T}_n[/$]Each successive measurement carries less weight (in the resulting average) compared to any previous measurements.
\end{itemize}

\note[item]{}
\end{frame}
\begin{frame}
\frametitle{Unrelated Title}


\begin{itemize}
\item priority scheduling strategy
\end{itemize}

\note[item]{}
\end{frame}
\begin{frame}
\frametitle{Unrelated Title}


\begin{itemize}
\item Starvation occurs when a process is prevented from running indefinitely.
\end{itemize}

\note[item]{}
\end{frame}
\begin{frame}
\frametitle{Unrelated Title}


\begin{itemize}
\item Starvation.
\end{itemize}

\note[item]{}
\end{frame}
\begin{frame}
\frametitle{Unrelated Title}


\begin{itemize}
\item A time quantum (or "time-slice").
\end{itemize}

\note[item]{}
\end{frame}
\begin{frame}
\frametitle{Unrelated Title}


\begin{itemize}
\item Round-robin (RR)
\end{itemize}

\note[item]{}
\end{frame}
\begin{frame}
\frametitle{Unrelated Title}


\begin{itemize}
\item timer interrupt
\end{itemize}

\note[item]{}
\end{frame}
\begin{frame}
\frametitle{Unrelated Title}


\begin{itemize}
\item At the end of the ready queue.
\end{itemize}

\note[item]{}
\end{frame}
\begin{frame}
\frametitle{Unrelated Title}


\begin{itemize}
\item [$](n - 1) * q[/$]
\end{itemize}

\note[item]{}
\end{frame}
\begin{frame}
\frametitle{Unrelated Title}


\begin{itemize}
\item The time required to context-switch.
\end{itemize}

\note[item]{}
\end{frame}
\begin{frame}
\frametitle{Unrelated Title}


\begin{itemize}
\item Foreground processes / Background processes
\end{itemize}

\note[item]{}
\end{frame}
\begin{frame}
\frametitle{Unrelated Title}


\begin{itemize}
\item ready queues
\end{itemize}

\note[item]{}
\end{frame}
\begin{frame}
\frametitle{Unrelated Title}


\begin{itemize}
\item The "ready queue" is composed of multiple process queues. New processes are permanently assigned to different queues according to some measurable criteria: memory size, process type, explicit priority, etc. Each queue is given its own scheduling algorithm; interqueue scheduling must also occur (e.g., queue priority).
\end{itemize}

\note[item]{}
\end{frame}
\begin{frame}
\frametitle{Unrelated Title}


\begin{itemize}
\item It would introduce the ability for a process to move between queues according to some runtime characteristic (i.e., CPU burst behavior). The strategy effectively distributes (sorts) all processes among the process queues according to their runtime behaviors.
\end{itemize}

\note[item]{}
\end{frame}
\begin{frame}
\frametitle{Unrelated Title}


\begin{itemize}
\item Load sharing (or "load-balancing")
\end{itemize}

\note[item]{}
\end{frame}
\begin{frame}
\frametitle{Unrelated Title}


\begin{itemize}
\item homogeneous
\end{itemize}

\note[item]{}
\end{frame}
\begin{frame}
\frametitle{Unrelated Title}


\begin{itemize}
\item Select the next process to execute (from a ready queue).
\end{itemize}

\note[item]{}
\end{frame}
\begin{frame}
\frametitle{Unrelated Title}


\begin{itemize}
\item When a process runs on a CPU for some time, part of the CPU's cache memory is populated with data associated with that process (improving performance). When a process is migrated to a new CPU, the original CPU's cache memory must be invalidated and the new CPU's cache memory must be populated.
\end{itemize}

\note[item]{}
\end{frame}
\begin{frame}
\frametitle{Unrelated Title}


\begin{itemize}
\item processor affinity
\end{itemize}

\note[item]{}
\end{frame}
\begin{frame}
\frametitle{Unrelated Title}


\begin{itemize}
\item hard affinity
\end{itemize}

\note[item]{}
\end{frame}
\begin{frame}
\frametitle{Unrelated Title}


\begin{itemize}
\item "Push migration" or "pull migration"
\end{itemize}

\note[item]{}
\end{frame}
\begin{frame}
\frametitle{Unrelated Title}


\begin{itemize}
\item With push migration, a dedicated process periodically checks the size of each CPU's queue and uses process migration to balance out the loads. With pull migration, idle CPUs may pull a waiting task from the queue of busier CPU.
\end{itemize}

\note[item]{}
\end{frame}
\begin{frame}
\frametitle{Unrelated Title}


\begin{itemize}
\item hyperthreading
\end{itemize}

\note[item]{}
\end{frame}
\begin{frame}
\frametitle{Unrelated Title}


\begin{itemize}
\item A hardware feature. The hardware presents multiple logical processors to the operating system; the operating system can schedule tasks on each logical processor.
\end{itemize}

\note[item]{}
\end{frame}
\begin{frame}
\frametitle{Unrelated Title}


\begin{itemize}
\item Without kernel-level threads, the operating system must schedule the individual user-level processes. When kernel-level threads are used, the user-level threads are managed by a thread library (not visible to the kernel) and the kernel need only schedule its own kernel-level threads.
\end{itemize}

\note[item]{}
\end{frame}
\begin{frame}
\frametitle{Unrelated Title}


\begin{itemize}
\item On systems that implement kernel threads (for executing system calls), the set of threads belonging to one process compete (or contest) for time on available LWPs (assigned by the thread library).When the kernel must choose a kernel thread to run on an available (physical) CPU, it uses system contention scope (SCS), as the competition includes all threads in the system.
\end{itemize}

\note[item]{}
\end{frame}
\begin{frame}
\frametitle{Unrelated Title}


\begin{itemize}
\item 1. We can measure and estimate the distribution of CPU bursts and I/O bursts across processes over time; this normally produces an exponential model (or formula) giving the probability of a particular CPU burst occurring.2. We can model an arrival-time distribution for processes in the system.From these 2 distribution models, we can estimate the utilization, average queue length, average wait time, and other indicators of performance, for each scheduling algorithm.
\end{itemize}

\note[item]{}
\end{frame}
\begin{frame}
\frametitle{Unrelated Title}


\begin{itemize}
\item Let n be the average length of a service queue.Let W be the average waiting time in the queue.Let [$]\lambda[/$] be the average arrival rate for new processes in the queue.During the time W that a process waits in the queue, [$]\lambda * W[/$] new processes will arrive. If the system is in a steady state, then the number of processes leaving the queue must be equal to the number of processes arriving at the queue.Thus, the average number of processes sitting in a steady state queue must be:[$]n = \lambda * W[/$]
\end{itemize}

\note[item]{}
\end{frame}
\begin{frame}
\frametitle{Unrelated Title}


\begin{itemize}
\item It gives us the average length of a (steady state) service queue, given the arrival rate and average waiting time for processes in the queue. The formula is valid for any scheduling algorithm and arrival distribution.
\end{itemize}

\note[item]{}
\end{frame}
\begin{frame}
\frametitle{Unrelated Title}


\begin{itemize}
\item The average length of a service queue (when its in a steady state).
\end{itemize}

\note[item]{}
\end{frame}
\begin{frame}
\frametitle{Unrelated Title}


\begin{itemize}
\item Whenever 2 or more processes access and manipulate the same (shared) data concurrently, and the outcome of the execution depends on the order in which access takes place.
\end{itemize}

\note[item]{}
\end{frame}
\begin{frame}
\frametitle{Unrelated Title}


\begin{itemize}
\item Access (and perhaps modification) of shared data in the system.
\end{itemize}

\note[item]{}
\end{frame}
\begin{frame}
\frametitle{Unrelated Title}


\begin{itemize}
\item no other process be allowed to execute in their own critical sections.
\end{itemize}

\note[item]{}
\end{frame}
\begin{frame}
\frametitle{Unrelated Title}


\begin{itemize}
\item 1. Mutual exclusion: If one process is executing its critical section, no other process may also be executing a critical section.2. Progress: If no process is executing a critical section and some processes wish to enter their critical sections, then those processes must decide amongst themselves which process will enter its critical section next. This decision cannot be postponed indefinitely.3. Bounded waiting: When a given process P1 requests to enter its critical section, there must be a limit to the number of other processes that enter their own critical sections before P1 is allowed to enter.
\end{itemize}

\note[item]{}
\end{frame}
\begin{frame}
\frametitle{Unrelated Title}


\begin{itemize}
\item When one process enters its critical section, no other process in the system may enter its own critical section.
\end{itemize}

\note[item]{}
\end{frame}
\begin{frame}
\frametitle{Unrelated Title}


\begin{itemize}
\item The set of processes waiting to enter their critical sections must coordinate to decide which process may enter next. This decision cannot be delayed indefinitely.
\end{itemize}

\note[item]{}
\end{frame}
\begin{frame}
\frametitle{Unrelated Title}


\begin{itemize}
\item When one process, P1, requests to enter its critical section, it cannot be postponed indefinitely by other processes. The number of other processes that are allowed to enter before it (after the request is made) must be limited.
\end{itemize}

\note[item]{}
\end{frame}
\begin{frame}
\frametitle{Unrelated Title}


\begin{itemize}
\item 1. Preemptive kernels.2. Non-preemptive kernels.
\end{itemize}

\note[item]{}
\end{frame}
\begin{frame}
\frametitle{Unrelated Title}


\begin{itemize}
\item A preemptive kernel allows a process (P1) to be preempted by other process (P2) while the original process (P1) is executing in kernel mode.In a non-preemptive kernel, a process running in kernel mode is allowed to run until it exits from kernel mode, it blocks, or it yields control voluntarily.
\end{itemize}

\note[item]{}
\end{frame}
\begin{frame}
\frametitle{Unrelated Title}


\begin{itemize}
\item By allowing processes to continue execution while in kernel mode without interruption, processes will not create race conditions on kernel data structures. Only one process may be active in the kernel at a time.
\end{itemize}

\note[item]{}
\end{frame}
\begin{frame}
\frametitle{Unrelated Title}


\begin{itemize}
\item A preemptive kernel; these kernels have improved ability to satisfy precise timing requirements for processes in the system.
\end{itemize}

\note[item]{}
\end{frame}
\begin{frame}
\frametitle{Unrelated Title}


\begin{itemize}
\item A preemptive kernel can be more responsive, as processes lose the ability to run in kernel mode for an arbitrarily long period of time.
\end{itemize}

\note[item]{}
\end{frame}
\begin{frame}
\frametitle{Unrelated Title}


\begin{itemize}
\item 1. wantsToEnter: A two-element array of flags to indicate which processes wish to enter their critical sections. Both flags initialize to false.2. turn: An integer used by each process to indicate which process should be given priority to execute its critical section. Can be initialized to either 0 or 1.
\end{itemize}

\note[item]{}
\end{frame}
\begin{frame}
\frametitle{Unrelated Title}


\begin{itemize}
\item Decker's algorithm allows exactly 2 processes to coordinate the ordering of execution of their critical sections (i.e., use of a shared system resource)—enforcing mutual exclusion. The 2 processes use shared memory (variables) to communicate their intentions to each other.
\end{itemize}

\note[item]{}
\end{frame}
\begin{frame}
\frametitle{Unrelated Title}


\begin{itemize}
\item Busy-waiting.
\end{itemize}

\note[item]{}
\end{frame}
\begin{frame}
\frametitle{Unrelated Title}


\begin{itemize}
\item disabling interrupts whenever a process enters its critical section.
\end{itemize}

\note[item]{}
\end{frame}
\begin{frame}
\frametitle{Unrelated Title}


\begin{itemize}
\item boolean TestAndSet(boolean *target){    boolean value = *target;    *target = true;    return value;}
\end{itemize}

\note[item]{}
\end{frame}
\begin{frame}
\frametitle{Unrelated Title}


\begin{itemize}
\item Disabling interrupts on a multiprocessor can be time-consuming, as one processor needs to signal to all other processors that it wishes to disable interrupts. This would delay execution and decrease overall system performance.
\end{itemize}

\note[item]{}
\end{frame}
\begin{frame}
\frametitle{Unrelated Title}


\begin{itemize}
\item We mean that the actions are performed together in one uninterrupted unit of work.
\end{itemize}

\note[item]{}
\end{frame}
\begin{frame}
\frametitle{Unrelated Title}


\begin{itemize}
\item The TestAndSet instruction*.
\end{itemize}

\note[item]{}
\end{frame}
\begin{frame}
\frametitle{Unrelated Title}


\begin{itemize}
\item An integer variable.
\end{itemize}

\note[item]{}
\end{frame}
\begin{frame}
\frametitle{Unrelated Title}


\begin{itemize}
\item 1. wait(), or P() (proberen, "to test").2. signal(), or V() (verhogen, "to increment").
\end{itemize}

\note[item]{}
\end{frame}
\begin{frame}
\frametitle{Unrelated Title}


\begin{itemize}
\item wait(S){    while (S <= 0) { /* Busy wait… */ }        S--;}
\end{itemize}

\note[item]{}
\end{frame}
\begin{frame}
\frametitle{Unrelated Title}


\begin{itemize}
\item signal(S){    S++;}
\end{itemize}

\note[item]{}
\end{frame}
\begin{frame}
\frametitle{Unrelated Title}


\begin{itemize}
\item binary semaphore, or a counting semaphore
\end{itemize}

\note[item]{}
\end{frame}
\begin{frame}
\frametitle{Unrelated Title}


\begin{itemize}
\item 1. A binary semaphore's value is restricted to either '0' or '1'.2. A counting semaphore's value has an unrestricted domain ([$][0, +\infty][/$]).
\end{itemize}

\note[item]{}
\end{frame}
\begin{frame}
\frametitle{Unrelated Title}


\begin{itemize}
\item mutex locks
\end{itemize}

\note[item]{}
\end{frame}
\begin{frame}
\frametitle{Unrelated Title}


\begin{itemize}
\item binary semaphore
\end{itemize}

\note[item]{}
\end{frame}
\begin{frame}
\frametitle{Unrelated Title}


\begin{itemize}
\item Any situation where a finite number of instances (<1) of some system resource is available for use. A counting semaphore is initialized to the number of instances. When the semaphore's value reaches zero, this indicates that all instances of the resource are currently in use.
\end{itemize}

\note[item]{}
\end{frame}
\begin{frame}
\frametitle{Unrelated Title}


\begin{itemize}
\item /* Shared memory */mutex sync(0);/* Process 1 */// .../* statement A */signal(sync);// .../* Process 2 */// ...wait(sync);/* statement B */// ...
\end{itemize}

\note[item]{}
\end{frame}
\begin{frame}
\frametitle{Unrelated Title}


\begin{itemize}
\item A semaphore that is implemented with busy-waiting.
\end{itemize}

\note[item]{}
\end{frame}
\begin{frame}
\frametitle{Unrelated Title}


\begin{itemize}
\item A spinlock.
\end{itemize}

\note[item]{}
\end{frame}
\begin{frame}
\frametitle{Unrelated Title}


\begin{itemize}
\item When we expect the lock to be held by a process for a short span of time.
\end{itemize}

\note[item]{}
\end{frame}
\begin{frame}
\frametitle{Unrelated Title}


\begin{itemize}
\item We can have the kernel maintain a queue for all processes currently waiting for a given lock. A process is moved from the ready queue to the lock's waiting queue when it attempts to acquire the lock but is blocked (lock already in use).When a running process releases the lock, the kernel checks whether any processes are on that lock's waiting queue; if so, it may select the process at the front of the waiting queue to run, moving it back onto the processor's ready queue.This scheme removes any busy waiting, so that we avoid wasting CPU cycles on processes that can not progress until they acquire the lock.
\end{itemize}

\note[item]{}
\end{frame}
\begin{frame}
\frametitle{Unrelated Title}


\begin{itemize}
\item A negative value can be used to indicate the number of processes currently waiting for an instance of the resource.
\end{itemize}

\note[item]{}
\end{frame}
\begin{frame}
\frametitle{Unrelated Title}


\begin{itemize}
\item 1. A positive value indicates how many instances of the resource are still available.2. A zero value indicates that no resources are available, and no processes are currently waiting on a resource instance.3. A negative value indicates the number of processes currently waiting for a resource instance.
\end{itemize}

\note[item]{}
\end{frame}
\begin{frame}
\frametitle{Unrelated Title}


\begin{itemize}
\item Instead of having a process busy wait before entering its critical section, we "move" the busy-waiting to only the code responsible for requesting (wait()) and releasing (signal()) a semaphore lock. When coded properly, these 2 operations should be no more than 10 instructions.The amount of time spent busy-waiting would thus no longer be dependent on the length of time the lock is held—or waited on—by various processes. Instead, the time spent busy-waiting is constant in the number of times the semaphore is requested and released.
\end{itemize}

\note[item]{}
\end{frame}
\begin{frame}
\frametitle{Unrelated Title}


\begin{itemize}
\item The situation arises whenever two or more processes are each blocking on an event that can only be triggered by one of the other (blocked) processes.
\end{itemize}

\note[item]{}
\end{frame}
\begin{frame}
\frametitle{Unrelated Title}


\begin{itemize}
\item A process may acquire the semaphore and never release it, starving any other processes that may be waiting on the semaphore.
\end{itemize}

\note[item]{}
\end{frame}
\begin{frame}
\frametitle{Unrelated Title}


\begin{itemize}
\item fork() and wait()
\end{itemize}

\note[item]{}
\end{frame}
\begin{frame}
\frametitle{Unrelated Title}


\begin{itemize}
\item CreateProcess() and WaitForSingleObject()
\end{itemize}

\note[item]{}
\end{frame}
\begin{frame}
\frametitle{Unrelated Title}


\begin{itemize}
\item Suppose we have 2 types of task running in the system: one type that only reads from some shared data (i.e., readers), and another type that may read and write (i.e., writers).Two or more readers may access the same data concurrently with no adverse affects; however, if a writer and some other task—either a reader or a writer—access the data simultaneously, the state of the system may become chaotic.The readers-writers problem attempts to solve this problem by providing a protocol for accessing shared data while preserving the integrity of the system's state, without letting a reader or writer task starve.
\end{itemize}

\note[item]{}
\end{frame}
\begin{frame}
\frametitle{Unrelated Title}


\begin{itemize}
\item 1. The first readers-writers problem requires that no reader be kept waiting (blocked) unless a writer has already obtained permission to access the data.2. The second readers-writers problem requires that, once a writer is ready, the writer should perform its work as soon as possible (no new readers should preceed).
\end{itemize}

\note[item]{}
\end{frame}
\begin{frame}
\frametitle{Unrelated Title}


\begin{itemize}
\item We allocate two semaphores—one called 'mutex' and one called 'write', both initialized to '1'. We also allocate an integer counter ('readcount'), initialized to zero.The 'mutex' semaphore is used to safely access the 'readcount' counter; the 'write' semaphore is used by writer tasks entering their critical section, and by the first or last reader task enter its critical section.Writer tasks wait() on the 'write' mutex before writing and signal() on 'write' afterwards. New reader tasks must acquire the 'mutex' lock to increment (and later decrement) 'readcount'; any reader that is first-in-line wait()'s on 'write' before performing a read, later signal()'ing on 'write', allowing other tasks to access the data.
\end{itemize}

\note[item]{}
\end{frame}
\begin{frame}
\frametitle{Unrelated Title}


\begin{itemize}
\item /* Shared data */semaphore mutex(1);semaphore write(1);int readcount = 0;/* Writer task */while(True){    wait(write);    // Perform write...    signal(write);}/* Reader task */while(True){    wait(mutex);    readcount++;    if(readcount == 1)    {        wait(write);    }    signal(mutex);    // Perform read...    wait(mutex);    readcount--;    if(readcount == 0)    {        signal(write);    }    signal(mutex);}
\end{itemize}

\note[item]{}
\end{frame}
\begin{frame}
\frametitle{Unrelated Title}


\begin{itemize}
\item A lock that is acquired in read-mode may be used concurrently by multiple processes, provided that they also acquired the lock in read-mode. When the lock is acquired by a process in write-mode, no other process may also aquire it.
\end{itemize}

\note[item]{}
\end{frame}
\begin{frame}
\frametitle{Unrelated Title}


\begin{itemize}
\item When an application has more reader tasks than writer tasks; reader-writer locks permit multiple reader tasks to run concurrently, and so the overhead is made worth it by this performance gain.
\end{itemize}

\note[item]{}
\end{frame}
\begin{frame}
\frametitle{Unrelated Title}


\begin{itemize}
\item Five philosophers are sitting at a round table. At the center of the table is a bowl of rice, and in front of each philosopher is a single chopstick. At any given time, each philosopher is either thinking or eating.In order to eat, a philosopher must hold 2 chopsticks; when he is done eating, the philosopher puts down both chopsticks on the table—making them available to the other philosophers.The dining philosopher's problem welcomes solutions that allow the philosophers to think and eat in a manner that does not lead to deadlocks or starvation.
\end{itemize}

\note[item]{}
\end{frame}
\begin{frame}
\frametitle{Unrelated Title}


\begin{itemize}
\item The need to allocate several resources among several processes in a manner that avoids deadlocks and starvation.
\end{itemize}

\note[item]{}
\end{frame}
\begin{frame}
\frametitle{Unrelated Title}


\begin{itemize}
\item deadlocks and starvation
\end{itemize}

\note[item]{}
\end{frame}
\begin{frame}
\frametitle{Unrelated Title}


\begin{itemize}
\item Timing errors
\end{itemize}

\note[item]{}
\end{frame}
\begin{frame}
\frametitle{Unrelated Title}


\begin{itemize}
\item A monitor.
\end{itemize}

\note[item]{}
\end{frame}
\begin{frame}
\frametitle{Unrelated Title}


\begin{itemize}
\item A monitor is an abstract data type that allows the programmer to define a set of operations; the monitor guarantees mutual exclusion for processes that use these operations to operate on some shared state (data). The monitor includes private variables (state) than can only be accessed by these operations.
\end{itemize}

\note[item]{}
\end{frame}
\begin{frame}
\frametitle{Unrelated Title}


\begin{itemize}
\item wait() and signal()
\end{itemize}

\note[item]{}
\end{frame}
\begin{frame}
\frametitle{Unrelated Title}


\begin{itemize}
\item A monitor.
\end{itemize}

\note[item]{}
\end{frame}
\begin{frame}
\frametitle{Unrelated Title}


\begin{itemize}
\item semaphores
\end{itemize}

\note[item]{}
\end{frame}
\begin{frame}
\frametitle{Unrelated Title}


\begin{itemize}
\item mutex
\end{itemize}

\note[item]{}
\end{frame}
\begin{frame}
\frametitle{Unrelated Title}


\begin{itemize}
\item A semaphore.
\end{itemize}

\note[item]{}
\end{frame}
\begin{frame}
\frametitle{Unrelated Title}


\begin{itemize}
\item The entry set of a lock.
\end{itemize}

\note[item]{}
\end{frame}
\begin{frame}
\frametitle{Unrelated Title}


\begin{itemize}
\item The (possibly empty) queue of processes currently waiting on the lock.
\end{itemize}

\note[item]{}
\end{frame}
\begin{frame}
\frametitle{Unrelated Title}


\begin{itemize}
\item The synchronized keyword, which may preceed the return type in a class object's method declaration, enforces mutual exclusion on the class object's data by implicitly allocating a lock that is associated with the object.When multiple threads attempt to call the synchronized object method simultaneously, the Java runtime requires each thread to acquire the object's underlying lock before entering the method. The lock is released whenever a thread returns from the method.
\end{itemize}

\note[item]{}
\end{frame}
\begin{frame}
\frametitle{Unrelated Title}


\begin{itemize}
\item A simple mutex places the requesting process on a wait queue if the requested resource is already in use by another process.When a process attempts to acquire an adaptive mutex that is not available, the adaptive mutex checks whether the holding process is currently running (i.e., on another CPU) or in a wait queue.If the holding process is running, the adaptive mutex behaves like a spinlock—as the holding process is likely to finish soon, releasing the mutex. If the holding process is instead waiting, then the requesting process goes to sleep on a wait queue. When the holding process releases the mutex, the first task in the wait queue wakes up.
\end{itemize}

\note[item]{}
\end{frame}
\begin{frame}
\frametitle{Unrelated Title}


\begin{itemize}
\item A queue structure used by the kernel to hold a set of threads currently waiting to acquire a lock.
\end{itemize}

\note[item]{}
\end{frame}
\begin{frame}
\frametitle{Unrelated Title}


\begin{itemize}
\item Processes in a turnstile are organized (sorted) according to a priority-inheritance protocol: when a higher-priority thread is blocked on a lock held by a lower-priority thread, the lower-priority thread temporarily inherits the higher priority.
\end{itemize}

\note[item]{}
\end{frame}
\begin{frame}
\frametitle{Unrelated Title}


\begin{itemize}
\item When a higher-priority thread is blocked on a lock that is currently held by a lower-priority thread, the lower-priority thread will temporarily inherit the higher priority, reducing the wait time for the higher-priority thread.
\end{itemize}

\note[item]{}
\end{frame}
\begin{frame}
\frametitle{Unrelated Title}


\begin{itemize}
\item disabling interrupts while a thread is executing in a critical section.
\end{itemize}

\note[item]{}
\end{frame}
\begin{frame}
\frametitle{Unrelated Title}


\begin{itemize}
\item spinlocks
\end{itemize}

\note[item]{}
\end{frame}
\begin{frame}
\frametitle{Unrelated Title}


\begin{itemize}
\item mutexes
\end{itemize}

\note[item]{}
\end{frame}
\begin{frame}
\frametitle{Unrelated Title}


\begin{itemize}
\item We mean that the work associated with the operation must be performed in its entirety, or not performed at all.
\end{itemize}

\note[item]{}
\end{frame}
\begin{frame}
\frametitle{Unrelated Title}


\begin{itemize}
\item A collection of operations (or instructions) that performs a single logical function.
\end{itemize}

\note[item]{}
\end{frame}
\begin{frame}
\frametitle{Unrelated Title}


\begin{itemize}
\item A commit() operation (success) or an abort() operation (failure).
\end{itemize}

\note[item]{}
\end{frame}
\begin{frame}
\frametitle{Unrelated Title}


\begin{itemize}
\item A commit() operation.
\end{itemize}

\note[item]{}
\end{frame}
\begin{frame}
\frametitle{Unrelated Title}


\begin{itemize}
\item An abort() operation.
\end{itemize}

\note[item]{}
\end{frame}
\begin{frame}
\frametitle{Unrelated Title}


\begin{itemize}
\item rolled back
\end{itemize}

\note[item]{}
\end{frame}
\begin{frame}
\frametitle{Unrelated Title}


\begin{itemize}
\item 1. Relative speed.2. Relative capacity.3. Resiliance to failure.
\end{itemize}

\note[item]{}
\end{frame}
\begin{frame}
\frametitle{Unrelated Title}


\begin{itemize}
\item 1. Main memory.2. Cache memory.
\end{itemize}

\note[item]{}
\end{frame}
\begin{frame}
\frametitle{Unrelated Title}


\begin{itemize}
\item 1. Disk drives.2. Magnetic tape drives.
\end{itemize}

\note[item]{}
\end{frame}
\begin{frame}
\frametitle{Unrelated Title}


\begin{itemize}
\item Magnetic tapes
\end{itemize}

\note[item]{}
\end{frame}
\begin{frame}
\frametitle{Unrelated Title}


\begin{itemize}
\item write-ahead logging
\end{itemize}

\note[item]{}
\end{frame}
\begin{frame}
\frametitle{Unrelated Title}


\begin{itemize}
\item With write-ahead logging, the system maintains a log that records every write operation that modifies data:
\end{itemize}

\note[item]{}
\end{frame}
\begin{frame}
\frametitle{Unrelated Title}


\begin{itemize}
\item 1. When a transaction begins: <T starts>2. When a transaction performs a write: <T writes (…)>3. When a transaction is committed: <T commits>
\end{itemize}

\note[item]{}
\end{frame}
\begin{frame}
\frametitle{Unrelated Title}


\begin{itemize}
\item 1. A unique name for the transaction.2. The name of the data item (record) written to.3. The old value.4. The new value.
\end{itemize}

\note[item]{}
\end{frame}
\begin{frame}
\frametitle{Unrelated Title}


\begin{itemize}
\item 1. There is a performance cost, as each logic write requires 2 actual writes.2. There is a storage cost to logging events and information on disk.
\end{itemize}

\note[item]{}
\end{frame}
\begin{frame}
\frametitle{Unrelated Title}


\begin{itemize}
\item undo(T_i):Restores the value of all data updated by transaction T_i to the old values.redo(T_i):Sets the value of all dataa updated by transaction T_i to the new values.
\end{itemize}

\note[item]{}
\end{frame}
\begin{frame}
\frametitle{Unrelated Title}


\begin{itemize}
\item Both operations must be idempotent.
\end{itemize}

\note[item]{}
\end{frame}
\begin{frame}
\frametitle{Unrelated Title}


\begin{itemize}
\item We mean that, following a single invocation of the operation, an arbitrary number of succeeding operations must yield a identical result.
\end{itemize}

\note[item]{}
\end{frame}
\begin{frame}
\frametitle{Unrelated Title}


\begin{itemize}
\item If the write-ahead log contains the <T_i starts> record but does not contain the <T_i commits> record.
\end{itemize}

\note[item]{}
\end{frame}
\begin{frame}
\frametitle{Unrelated Title}


\begin{itemize}
\item When the log contains the <T_i starts> record as well as the <T_i commits> record.
\end{itemize}

\note[item]{}
\end{frame}
\begin{frame}
\frametitle{Unrelated Title}


\begin{itemize}
\item checkpoints
\end{itemize}

\note[item]{}
\end{frame}
\begin{frame}
\frametitle{Unrelated Title}


\begin{itemize}
\item 1. Output all log records residing in volatile storage to stable storage.2. Output all modified data residing in volatile storage to stable storage.3. Output a log record <checkpoint> to stable storage.
\end{itemize}

\note[item]{}
\end{frame}
\begin{frame}
\frametitle{Unrelated Title}


\begin{itemize}
\item Searching the log for the most recent <checkpoint> record and performing any restoration work associated with the set of transactions that followed in the log.
\end{itemize}

\note[item]{}
\end{frame}
\begin{frame}
\frametitle{Unrelated Title}


\begin{itemize}
\item When multiple (atomic) transactions may be executed concurrently, the execution of these transactions must be ordered in such a way that the final result is equivalent—regardless of the ordering.
\end{itemize}

\note[item]{}
\end{frame}
\begin{frame}
\frametitle{Unrelated Title}


\begin{itemize}
\item We could have all transactions depend on a common semaphore "mutex":1. Before beginning a transaction, a thread must first acquire the mutex.2. After committing or aborting the transaction, the thread must release the mutex.
\end{itemize}

\note[item]{}
\end{frame}
\begin{frame}
\frametitle{Unrelated Title}


\begin{itemize}
\item A schedule.
\end{itemize}

\note[item]{}
\end{frame}
\begin{frame}
\frametitle{Unrelated Title}


\begin{itemize}
\item A serial schedule.
\end{itemize}

\note[item]{}
\end{frame}
\begin{frame}
\frametitle{Unrelated Title}


\begin{itemize}
\item There exists n! ("n-factorial") possible valid serial schedules. 
\end{itemize}

\note[item]{}
\end{frame}
\begin{frame}
\frametitle{Unrelated Title}


\begin{itemize}
\item When they access the same data item and at least one of the operations is a write.
\end{itemize}

\note[item]{}
\end{frame}
\begin{frame}
\frametitle{Unrelated Title}


\begin{itemize}
\item When it can be transformed into a different (equivalent) serial schedule S', through a series of swaps of non-conflicting operations.
\end{itemize}

\note[item]{}
\end{frame}
\begin{frame}
\frametitle{Unrelated Title}


\begin{itemize}
\item 1. Locking protocols (e.g., two-phase locking).2. Timestamp-ordering protocols.
\end{itemize}

\note[item]{}
\end{frame}
\begin{frame}
\frametitle{Unrelated Title}


\begin{itemize}
\item 1. A growing phase.2. A shrinking phase.
\end{itemize}

\note[item]{}
\end{frame}
\begin{frame}
\frametitle{Unrelated Title}


\begin{itemize}
\item 1. During the growing phase, transactions may acquire locks but not release any.2. During the shrinking phase, transactions may release locks but not acquire any.
\end{itemize}

\note[item]{}
\end{frame}
\begin{frame}
\frametitle{Unrelated Title}


\begin{itemize}
\item 1. We can request() and release() devices.2. We can open() and close() files.3. We can allocate() and free() memory.
\end{itemize}

\note[item]{}
\end{frame}
\begin{frame}
\frametitle{Unrelated Title}


\begin{itemize}
\item A system table.
\end{itemize}

\note[item]{}
\end{frame}
\begin{frame}
\frametitle{Unrelated Title}


\begin{itemize}
\item When each process in the set is currently waiting for an event that can only be caused by another process in the set.
\end{itemize}

\note[item]{}
\end{frame}
\begin{frame}
\frametitle{Unrelated Title}


\begin{itemize}
\item 1. Mutual exclusion: We assume that each resource may only be used by one process at a time.2. Hold-and-wait: A process must be holding one resource while simultaneously waiting for another resource that is currently held by another process.3. No preemption: Resources may not be preempted by the kernel; they can only be released voluntarily by the process that acquired it.4. Circular wait: A set [$]{ P_0, P_1, \dots, P_n }[/$] must exist such that [$]P_0[/$] is waiting for a resource held by [$]P_1[/$], and so on, and [$]P_n[/$] is waiting for a resource from [$]P_0[/$].
\end{itemize}

\note[item]{}
\end{frame}
\begin{frame}
\frametitle{Unrelated Title}


\begin{itemize}
\item 1. Mutual exclusion.2. Hold and wait.3. No preemption.4. Circular wait.
\end{itemize}

\note[item]{}
\end{frame}
\begin{frame}
\frametitle{Unrelated Title}


\begin{itemize}
\item The circular wait condition implies the hold-and-wait condition.
\end{itemize}

\note[item]{}
\end{frame}
\begin{frame}
\frametitle{Unrelated Title}


\begin{itemize}
\item A directed graph.
\end{itemize}

\note[item]{}
\end{frame}
\begin{frame}
\frametitle{Unrelated Title}


\begin{itemize}
\item An application of directed graphs which represents the processes active in a system and the set of resource types available in the system.
\end{itemize}

\note[item]{}
\end{frame}
\begin{frame}
\frametitle{Unrelated Title}


\begin{itemize}
\item The set of vertices [$]V[/$] is composed of 2 disjoint sets:1. [$]P[/$], representing all active processes in the system.2. [$]R[/$], representing all resource types in the system.
\end{itemize}

\note[item]{}
\end{frame}
\begin{frame}
\frametitle{Unrelated Title}


\begin{itemize}
\item A directed edge from a process vertex [$]P_i[/$] to a resource type vertex [$]R_j[/$].
\end{itemize}

\note[item]{}
\end{frame}
\begin{frame}
\frametitle{Unrelated Title}


\begin{itemize}
\item A directed edge from a resource type vertex [$]R_i[/$] to a process vertex [$]P_j[/$].
\end{itemize}

\note[item]{}
\end{frame}
\begin{frame}
\frametitle{Unrelated Title}


\begin{itemize}
\item A request edge.
\end{itemize}

\note[item]{}
\end{frame}
\begin{frame}
\frametitle{Unrelated Title}


\begin{itemize}
\item An assignment edge.
\end{itemize}

\note[item]{}
\end{frame}
\begin{frame}
\frametitle{Unrelated Title}


\begin{itemize}
\item 1. If the system has one instance of each resource type, then a cycle in the graph implies that a deadlock has occurred between the processes involved in the cycle.2. If the system provides multiple instances of each resource type, then a cycle may mean that a deadlock has occurred between those processes—but not necessarily.
\end{itemize}

\note[item]{}
\end{frame}
\begin{frame}
\frametitle{Unrelated Title}


\begin{itemize}
\item 1. Deadlock prevention.2. Deadlock avoidance.3. Do nothing (ignore deadlocks).
\end{itemize}

\note[item]{}
\end{frame}
\begin{frame}
\frametitle{Unrelated Title}


\begin{itemize}
\item Deadlock prevention imposes restrictions on how request for resources can be made at runtime. Deadlock avoidance uses presupposed information to schedule processes in such a way that they do not lead to deadlocks.
\end{itemize}

\note[item]{}
\end{frame}
\begin{frame}
\frametitle{Unrelated Title}


\begin{itemize}
\item 1. Have processes request (and be allocated) all of the resources they need before starting execution (i.e., before making any additional system calls).2. Allow processes to request new resources only when they hold none.
\end{itemize}

\note[item]{}
\end{frame}
\begin{frame}
\frametitle{Unrelated Title}


\begin{itemize}
\item Impose a total ordering of all resource types [$]R = { R_1, R_2, \dots, R_n }[/$] by defining a one-to-one function F :[$]F: R \rightarrow N [/$], where N is the set of natural numbers (integers).
\end{itemize}

\note[item]{}
\end{frame}
\begin{frame}
\frametitle{Unrelated Title}


\begin{itemize}
\item 1. If a process is holding a resource of type [$]R_i[/$], then it may only request a resource of type [$]R_j[/$] only if [$]F(R_j) \geq F(R_i)[/$].2. If a process requests a resource of type [$]R_j[/$], it must first release any resource [$]R_i[/$] such that [$]F(R_i) \geq F(R_j)[/$].
\end{itemize}

\note[item]{}
\end{frame}
\begin{frame}
\frametitle{Unrelated Title}


\begin{itemize}
\item witness serves as a lock-order verifier: it protects critical sections of different processes by dynamically maintaining the relationship of lock orders in the system.
\end{itemize}

\note[item]{}
\end{frame}
\begin{frame}
\frametitle{Unrelated Title}


\begin{itemize}
\item It would need to know the maximum number of resources (of each type) that could be requested by each process in the system.
\end{itemize}

\note[item]{}
\end{frame}
\begin{frame}
\frametitle{Unrelated Title}


\begin{itemize}
\item 1. The maximum allocation demands (by resource type) of each process.2. The current number of allocated and available resources (by resource type).
\end{itemize}

\note[item]{}
\end{frame}
\begin{frame}
\frametitle{Unrelated Title}


\begin{itemize}
\item The system is in a safe state if there exists a safe sequence: a sequence of processes [$]<P_1, P_2, \dots, P_n>[/$] where, for each process [$]P_i[/$], the resource requests that [$]P_i[/$] can still make (up to its maximum) can be satisfied by the set of currently available resources plus the resources held by all [$]P_j[/$] (where [$]j \lT i[/$]).
\end{itemize}

\note[item]{}
\end{frame}
\begin{frame}
\frametitle{Unrelated Title}


\begin{itemize}
\item A sequence of processes [$]<P_1, P_2, \dots, P_n>[/$] where, for each process [$]P_i[/$], the resource requests that [$]P_i[/$] can still make—up to its maximum—can be satisfied by the set of currently available resources plus the resources held by all [$]P_j[/$] (where [$]j \lT i[/$]).
\end{itemize}

\note[item]{}
\end{frame}
\begin{frame}
\frametitle{Unrelated Title}


\begin{itemize}
\item Safe or unsafe.
\end{itemize}

\note[item]{}
\end{frame}
\begin{frame}
\frametitle{Unrelated Title}


\begin{itemize}
\item An unsafe state may lead to a deadlock, but it does not necessarily mean that a deadlock has occurred.
\end{itemize}

\note[item]{}
\end{frame}
\begin{frame}
\frametitle{Unrelated Title}


\begin{itemize}
\item 1. A request edge ([$]P \rightarrow R[/$]).2. An assignment edge ([$]R \rightarrow P[/$]).3. A claim edge ([$]P \rightarrow R[/$]).
\end{itemize}

\note[item]{}
\end{frame}
\begin{frame}
\frametitle{Unrelated Title}


\begin{itemize}
\item Cycle-detection algorithms.
\end{itemize}

\note[item]{}
\end{frame}
\begin{frame}
\frametitle{Unrelated Title}


\begin{itemize}
\item [$]\Omega(n^2)[/$], where n is the number of processes in the resource-allocation graph.
\end{itemize}

\note[item]{}
\end{frame}
\begin{frame}
\frametitle{Unrelated Title}


\begin{itemize}
\item When more than one instance of each resource type is offered by the system.
\end{itemize}

\note[item]{}
\end{frame}
\begin{frame}
\frametitle{Unrelated Title}


\begin{itemize}
\item We can use the algorithm to detect and prevent deadlocks in systems that offer multiple instances of each resource type.
\end{itemize}

\note[item]{}
\end{frame}
\begin{frame}
\frametitle{Unrelated Title}


\begin{itemize}
\item 1. available: An vector of length m indicaitng the number of available resources of each type.2. max: An [$]n \times m[/$] matrix defining the maximum number of allocations by type that a process may request during its lifecycle.3. allocation: An [$]n \times m[/$] matrix defining the current allocation of resources by type to each process.4. need: An [$]n \times m[/$] matrix indicating the current (remaining) need, per resource type, for each process.
\end{itemize}

\note[item]{}
\end{frame}
\begin{frame}
\frametitle{Unrelated Title}


\begin{itemize}
\item Vectors and matrices.
\end{itemize}

\note[item]{}
\end{frame}
\begin{frame}
\frametitle{Unrelated Title}


\begin{itemize}
\item 1. Let work and finish be vectors of length m and n, respectively.    a. Initialize work = available.    b. Initialize [$]finish_i[/$] = false for [$]i = 0, 1, \dots, n-1[/$].2. Find an i such that [$]finish_i[/$] == false and [$]need_i \leq work[/$]. If no such i exists, skip to step 4.3. Loop:    a. Add [$]allocation_i[/$] to work.    b. Set [$]finish_i[/$] to true.    c. Go to step 2.4. If [$]finish_i[/$] == true for all i, then the system is in a safe state.
\end{itemize}

\note[item]{}
\end{frame}
\begin{frame}
\frametitle{Unrelated Title}


\begin{itemize}
\item [$]\Omega(m \times n^2)[/$]
\end{itemize}

\note[item]{}
\end{frame}
\begin{frame}
\frametitle{Unrelated Title}


\begin{itemize}
\item When a request for resources is made by process [$]P_i[/$], the following steps are taken:1. If [$]request_i \leq need_i[/$], go to step 2. Otherwise, raise an error (as the process has exceeded its max claim).2. If [$]request_i \leq available[/$], go to step 3. Otherwise, [$]P_i[/$] must wait as the resources it needs are not currently available.3. Simulate the request by running the safety algorithm on an altered representation of the system state:    available = available - request_i    allocation_i = allocation_i + request_i    need_i = need_i - request_iIf the resulting state is safe, then the transaction is complete, and no additional work must be done; however, if the resulting state is unsafe, the [$]P_i[/$] must wait for [$]request_i[/$], and the previous allocation state is restored.
\end{itemize}

\note[item]{}
\end{frame}
\begin{frame}
\frametitle{Unrelated Title}


\begin{itemize}
\item A wait-for graph.
\end{itemize}

\note[item]{}
\end{frame}
\begin{frame}
\frametitle{Unrelated Title}


\begin{itemize}
\item A transformation (variant) of the system-allocation graph in which the set of resource nodes R is removed, collapsing the adjacent edges into the appropriate process vertices.
\end{itemize}

\note[item]{}
\end{frame}
\begin{frame}
\frametitle{Unrelated Title}


\begin{itemize}
\item 1. We removed the set of resource nodes R from the graph.2. We collapse the orphaned edges into the appropriate process vertices:For a pair of edges [$]{ E_{i,x}, E_{x,j} }[/$] where [$]E_{r,i,x}[/$] is a request edge originating from process [$]P_i[/$] and reaching resource [$]R_x[/$] and [$]E_{a,x,j}[/$] is an assignment edge originating from resource [$]R_x[/$] and reaching process [$]P_j[/$]……we remove R and collapse both edges into one wait edge [$]E_{w,i,j}[/$] such that [$]E_{w,i,j}[/$] originates at [$]P_i[/$] and reaches [$]P_j[/$].
\end{itemize}

\note[item]{}
\end{frame}
\begin{frame}
\frametitle{Unrelated Title}


\begin{itemize}
\item If the wait-for graph contains a cycle, then there is a deadlock. We can use an [$]\Omega(n^2)[/$] cycle-detection algorithm to check for cycles in the graph.
\end{itemize}

\note[item]{}
\end{frame}
\begin{frame}
\frametitle{Unrelated Title}


\begin{itemize}
\item When the system offers multiple instances of each resource type.
\end{itemize}

\note[item]{}
\end{frame}
\begin{frame}
\frametitle{Unrelated Title}


\begin{itemize}
\item 1. available: A vector of length m indicating the number of available resources of each type.2. allocation: An [$]n \times m[/$] matrix storing the number of resources of each type currently allocated to each process.3. request: An [$]n \times m[/$] matrix describing the currently outstanding resource requests for each process.
\end{itemize}

\note[item]{}
\end{frame}
\begin{frame}
\frametitle{Unrelated Title}


\begin{itemize}
\item The algorithm investigates every possible allocation sequence for the processes that have not yet finished their work.
\end{itemize}

\note[item]{}
\end{frame}
\begin{frame}
\frametitle{Unrelated Title}


\begin{itemize}
\item When the deadlock-detection algorithm is invoked by the operating system, it follows these steps:1. Let work and finish be vectors of length m and n, respectively. Initialize [$]work = available[/$]. For each process [$]P_i[/$], initialize [$]finish_i[/$] to either true or false depending on [$]allocation_i[/$].2. Find an index i such that [$]finish_i == false[/$] (process has not finished) and [$]request_i \leq work[/$]. If no index exists, go to Step 4.3. "Reclaim" the resources* used by [$]P_i[/$], and then go to Step 2.    [$]work = work + allocation_i[/$]    [$]finish_i = true[/$]4. If [$]finish_i == false[/$] for any i ([$]0 \leq i \leq n[/$]), then the system has a deadlock (and process [$]P_i[/$] is involved in the deadlock).
\end{itemize}

\note[item]{}
\end{frame}
\begin{frame}
\frametitle{Unrelated Title}


\begin{itemize}
\item 1. How often is a deadlock likely to occur?2. How many processes may be affected by a deadlock if one occurs?
\end{itemize}

\note[item]{}
\end{frame}
\begin{frame}
\frametitle{Unrelated Title}


\begin{itemize}
\item 1. Whenever a process requests a resource that cannot immediately be granted.2. Whenever CPU utilization drops below some threshold (e.g., < 40%).3. According to some timed frequency (e.g., every 500ms).
\end{itemize}

\note[item]{}
\end{frame}
\begin{frame}
\frametitle{Unrelated Title}


\begin{itemize}
\item 1. Terminate all processes involved in the deadlock.2. Terminate one involved process at a time until the deadlock is resolved.
\end{itemize}

\note[item]{}
\end{frame}
\begin{frame}
\frametitle{Unrelated Title}


\begin{itemize}
\item Terminating all processes involves less overhead, but risks greater waste of past computation time. The one-by-one approach may be less wasteful, but it requires more overhead cost: it requires us to re-execute our deadlock-detection algorithm after each termination.
\end{itemize}

\note[item]{}
\end{frame}
\begin{frame}
\frametitle{Unrelated Title}


\begin{itemize}
\item 1. What is the process's priority?2. How long has the process been running?3. How much longer do we expect the process to run?4. How many (and what types of) resources is the process currently using?5. How many more resources might the process request?6. Is the process categorized as interactive or batch?
\end{itemize}

\note[item]{}
\end{frame}
\begin{frame}
\frametitle{Unrelated Title}


\begin{itemize}
\item We could temporarily preempt one or more resources held by a process, giving them to another process–allowing it to finish its work.
\end{itemize}

\note[item]{}
\end{frame}
\begin{frame}
\frametitle{Unrelated Title}


\begin{itemize}
\item 1. We could record the state of all processes at some interval and roll back the victim process to a previous state, where the preempted resource(s) is not held.2. We could simply abort and restart the process—in which case all resources held by that process might be released.
\end{itemize}

\note[item]{}
\end{frame}
\begin{frame}
\frametitle{Unrelated Title}


\begin{itemize}
\item The program counter.
\end{itemize}

\note[item]{}
\end{frame}
\begin{frame}
\frametitle{Unrelated Title}


\begin{itemize}
\item Instruction operands.
\end{itemize}

\note[item]{}
\end{frame}
\begin{frame}
\frametitle{Unrelated Title}


\begin{itemize}
\item Normally one cycle.
\end{itemize}

\note[item]{}
\end{frame}
\begin{frame}
\frametitle{Unrelated Title}


\begin{itemize}
\item The memory bus.
\end{itemize}

\note[item]{}
\end{frame}
\begin{frame}
\frametitle{Unrelated Title}


\begin{itemize}
\item A base register and a limit register.
\end{itemize}

\note[item]{}
\end{frame}
\begin{frame}
\frametitle{Unrelated Title}


\begin{itemize}
\item The smallest (first) legal physical address belonging to the process's virtual address space.
\end{itemize}

\note[item]{}
\end{frame}
\begin{frame}
\frametitle{Unrelated Title}


\begin{itemize}
\item The size of the range of physical addresses, starting at the base.
\end{itemize}

\note[item]{}
\end{frame}
\begin{frame}
\frametitle{Unrelated Title}


\begin{itemize}
\item Normally the CPU issues a trap to the operating system, which treats the attempt as a fatal error and terminates the process.
\end{itemize}

\note[item]{}
\end{frame}
\begin{frame}
\frametitle{Unrelated Title}


\begin{itemize}
\item The operating system treats the trap as a fatal exception, and aborts the process.
\end{itemize}

\note[item]{}
\end{frame}
\begin{frame}
\frametitle{Unrelated Title}


\begin{itemize}
\item Privileged instructions.
\end{itemize}

\note[item]{}
\end{frame}
\begin{frame}
\frametitle{Unrelated Title}


\begin{itemize}
\item The input queue.
\end{itemize}

\note[item]{}
\end{frame}
\begin{frame}
\frametitle{Unrelated Title}


\begin{itemize}
\item 1. The compiler translates symbolic addresses into relocatable addresses;2. The linker/loader translates relocatable addresses into absolute addresses.
\end{itemize}

\note[item]{}
\end{frame}
\begin{frame}
\frametitle{Unrelated Title}


\begin{itemize}
\item A symbolic address is represented by variable/procesure names in the program's source code (e.g., count, drawRectangle, etc).A relocatable address is a numeric address that specifies a relative offset from some origin address—usually the base address of a data or code segment.
\end{itemize}

\note[item]{}
\end{frame}
\begin{frame}
\frametitle{Unrelated Title}


\begin{itemize}
\item At compile time.
\end{itemize}

\note[item]{}
\end{frame}
\begin{frame}
\frametitle{Unrelated Title}


\begin{itemize}
\item load time
\end{itemize}

\note[item]{}
\end{frame}
\begin{frame}
\frametitle{Unrelated Title}


\begin{itemize}
\item The memory-address register.
\end{itemize}

\note[item]{}
\end{frame}
\begin{frame}
\frametitle{Unrelated Title}


\begin{itemize}
\item Logical addresses, physical addresses
\end{itemize}

\note[item]{}
\end{frame}
\begin{frame}
\frametitle{Unrelated Title}


\begin{itemize}
\item The memory management unit (MMU).
\end{itemize}

\note[item]{}
\end{frame}
\begin{frame}
\frametitle{Unrelated Title}


\begin{itemize}
\item 1. Compile time (non-relocatable).2. Load time (relocatable between executions).3. Execution time (relocatable at any time by the operating system).
\end{itemize}

\note[item]{}
\end{frame}
\begin{frame}
\frametitle{Unrelated Title}


\begin{itemize}
\item Logical addresses (i.e., virtual addresses).
\end{itemize}

\note[item]{}
\end{frame}
\begin{frame}
\frametitle{Unrelated Title}


\begin{itemize}
\item All routines that might be called by a program are stored on disk, in a relocatable load format. A routine is not loaded from disk into memory until it is first called by the program. Loading can be performed by a relocatable linking loader. The loader must update the program's address tables to reflect the change.
\end{itemize}

\note[item]{}
\end{frame}
\begin{frame}
\frametitle{Unrelated Title}


\begin{itemize}
\item An unused routine is never loaded into main memory, thus reducing the overall memory footprint of the program.
\end{itemize}

\note[item]{}
\end{frame}
\begin{frame}
\frametitle{Unrelated Title}


\begin{itemize}
\item Dynamic linking allows dynamically linked libraries (DLLs) to be linked to the program at execution time.
\end{itemize}

\note[item]{}
\end{frame}
\begin{frame}
\frametitle{Unrelated Title}


\begin{itemize}
\item A stub is included in the image for each reference to a library routine. The stub indicates how to locate the library routine if it is resident in memory (or how to load the routine if it is not resident). The stub then replaces itself with the address of the loaded routine and executes it.
\end{itemize}

\note[item]{}
\end{frame}
\begin{frame}
\frametitle{Unrelated Title}


\begin{itemize}
\item With dynamic linking, we can load a library into some region of memory and share it with multiple programs.Because the operating system normally contains each process within its own distinct virtual address space, the operating system must provide a mechanism for sharing regions of memory between processes.
\end{itemize}

\note[item]{}
\end{frame}
\begin{frame}
\frametitle{Unrelated Title}


\begin{itemize}
\item A segment of disk space used to store non-active processes (and any associated data) that have been swapped out of main memory.
\end{itemize}

\note[item]{}
\end{frame}
\begin{frame}
\frametitle{Unrelated Title}


\begin{itemize}
\item Swap space.
\end{itemize}

\note[item]{}
\end{frame}
\begin{frame}
\frametitle{Unrelated Title}


\begin{itemize}
\item Swapping enables us to run more programs than might otherwise be able to fit into main memory. With swapping, to meet memory demands, we take a non-active process and migrate it (in its entirety) to the backing store on disk.Doing so makes room for another process to run (in main memory). When the process must run again, we migrate it back from the backing store into an available section of main memory.
\end{itemize}

\note[item]{}
\end{frame}
\begin{frame}
\frametitle{Unrelated Title}


\begin{itemize}
\item If the process is inactive but waiting on some I/O to arrive (e.g., opening a file). Normally, when a process initiates an I/O read, the I/O device is given the location of a buffer (in the process's virtual address space) to write to.If we initiate such an operation and then swap the associated process out with a new process in memory, the I/O would eventually complete—and attempt to write data to the new process's address space.
\end{itemize}

\note[item]{}
\end{frame}
\begin{frame}
\frametitle{Unrelated Title}


\begin{itemize}
\item Windows 3.1 (ca. 1992)
\end{itemize}

\note[item]{}
\end{frame}
\begin{frame}
\frametitle{Unrelated Title}


\begin{itemize}
\item Adjacent to the interrupt vector.
\end{itemize}

\note[item]{}
\end{frame}
\begin{frame}
\frametitle{Unrelated Title}


\begin{itemize}
\item Low memory (where the interrupt vector is often stored).
\end{itemize}

\note[item]{}
\end{frame}
\begin{frame}
\frametitle{Unrelated Title}


\begin{itemize}
\item When the operating system dispatcher executes a context-switch. New values are set according to fields stored in the new process's task control block (PCB).
\end{itemize}

\note[item]{}
\end{frame}
\begin{frame}
\frametitle{Unrelated Title}


\begin{itemize}
\item Transient operating system code.
\end{itemize}

\note[item]{}
\end{frame}
\begin{frame}
\frametitle{Unrelated Title}


\begin{itemize}
\item How to satisfy a memory request of a given size from a list of available blocks.
\end{itemize}

\note[item]{}
\end{frame}
\begin{frame}
\frametitle{Unrelated Title}


\begin{itemize}
\item 1. First fit: Allocate the first block that meets the size requirement.2. Best fit: Allocate the smallest block that meets the size requirement.3. Worst fit: Allocate the largest block that meets the size requirement.
\end{itemize}

\note[item]{}
\end{frame}
\begin{frame}
\frametitle{Unrelated Title}


\begin{itemize}
\item First fit and best fit.
\end{itemize}

\note[item]{}
\end{frame}
\begin{frame}
\frametitle{Unrelated Title}


\begin{itemize}
\item 1. External fragmentation is the result of many small "holes" lying outside of any process's allocated memory space.2. Internal fragmentation is the result of over-allocation of (unused) memory within a process's allocated memory space.
\end{itemize}

\note[item]{}
\end{frame}
\begin{frame}
\frametitle{Unrelated Title}


\begin{itemize}
\item External fragmentation.
\end{itemize}

\note[item]{}
\end{frame}
\begin{frame}
\frametitle{Unrelated Title}


\begin{itemize}
\item Internal fragmentation.
\end{itemize}

\note[item]{}
\end{frame}
\begin{frame}
\frametitle{Unrelated Title}


\begin{itemize}
\item Given N allocated blocks of memory, another [$]\frac{N}{2}[/$] blocks are expected to be lost to memory fragmentation. This rule generally holds true for first-fit allocation solutions.
\end{itemize}

\note[item]{}
\end{frame}
\begin{frame}
\frametitle{Unrelated Title}


\begin{itemize}
\item Compaction involves shuffling blocks of allocated and available memory in order to merge the available blocks. Doing so reduces the number of small, available blocks that are unlikely to be allocated.
\end{itemize}

\note[item]{}
\end{frame}
\begin{frame}
\frametitle{Unrelated Title}


\begin{itemize}
\item If the region of memory contains processes that are non-relocatable.
\end{itemize}

\note[item]{}
\end{frame}
\begin{frame}
\frametitle{Unrelated Title}


\begin{itemize}
\item Frames
\end{itemize}

\note[item]{}
\end{frame}
\begin{frame}
\frametitle{Unrelated Title}


\begin{itemize}
\item A block of physical memory defined by a memory-paging system. Frames normally share the same size.
\end{itemize}

\note[item]{}
\end{frame}
\begin{frame}
\frametitle{Unrelated Title}


\begin{itemize}
\item A logical block of memory, defined by a process or object in the system (such as a file or buffer). Pages of memory are mapped into frames (which share their size).
\end{itemize}

\note[item]{}
\end{frame}
\begin{frame}
\frametitle{Unrelated Title}


\begin{itemize}
\item Paging involves dividing physical memory into equally-sized frames, and breaking logical memory into block of the same size called pages. When a process is scheduled to run, its associated pages are copied from the backing store into frames of physical memory.
\end{itemize}

\note[item]{}
\end{frame}
\begin{frame}
\frametitle{Unrelated Title}


\begin{itemize}
\item 1. Which frames are currently free and which are occupied.2. Which page occupies which (occupied) physical frame.3. Which process owns which logical page.4. The number of total frames present in the system.
\end{itemize}

\note[item]{}
\end{frame}
\begin{frame}
\frametitle{Unrelated Title}


\begin{itemize}
\item 1. A page number.2. A page offset.
\end{itemize}

\note[item]{}
\end{frame}
\begin{frame}
\frametitle{Unrelated Title}


\begin{itemize}
\item It is determined by the hardware.
\end{itemize}

\note[item]{}
\end{frame}
\begin{frame}
\frametitle{Unrelated Title}


\begin{itemize}
\item A page table is a key-value store that maps (logical) pages of memory to (physical) frames of memory. The page number component of a logical address is used to index into a page table to determine a physical base address; the base address is used together with the page offset to determine an absolute physical address.
\end{itemize}

\note[item]{}
\end{frame}
\begin{frame}
\frametitle{Unrelated Title}


\begin{itemize}
\item External fragmentation.
\end{itemize}

\note[item]{}
\end{frame}
\begin{frame}
\frametitle{Unrelated Title}


\begin{itemize}
\item All physical memory is allocated in some fixed multiple of frames. Any unused frame may be allocated to a new process.
\end{itemize}

\note[item]{}
\end{frame}
\begin{frame}
\frametitle{Unrelated Title}


\begin{itemize}
\item All memory is allocated in some number of frames. A process's memory usage normally would not coincide with a page or frame boundary, so a portion of the last frame allocated to the process may not be used in its entirety.
\end{itemize}

\note[item]{}
\end{frame}
\begin{frame}
\frametitle{Unrelated Title}


\begin{itemize}
\item We expect each process to waste one half-page of memory.
\end{itemize}

\note[item]{}
\end{frame}
\begin{frame}
\frametitle{Unrelated Title}


\begin{itemize}
\item • Smaller pages reduce internal fragmentation (less wasted memory).• Larger pages reduce the size of the system's page tables (less granular).• Larger pages result in more efficient disk I/O.
\end{itemize}

\note[item]{}
\end{frame}
\begin{frame}
\frametitle{Unrelated Title}


\begin{itemize}
\item 4 kilobytes or 8 kilobytes
\end{itemize}

\note[item]{}
\end{frame}
\begin{frame}
\frametitle{Unrelated Title}


\begin{itemize}
\item 4 bytes
\end{itemize}

\note[item]{}
\end{frame}
\begin{frame}
\frametitle{Unrelated Title}


\begin{itemize}
\item The table could address [$]2^32[/$] physical page frames.
\end{itemize}

\note[item]{}
\end{frame}
\begin{frame}
\frametitle{Unrelated Title}


\begin{itemize}
\item A frame table.
\end{itemize}

\note[item]{}
\end{frame}
\begin{frame}
\frametitle{Unrelated Title}


\begin{itemize}
\item The frame table records which frames are currently mapped to a page.The page table holds the mapping between virtual page addresses and physical frame addresses. It may also hold auxiliary information such as a valid bit, a modified ("dirty") bit, the process / address space associated with the page, etc.
\end{itemize}

\note[item]{}
\end{frame}
\begin{frame}
\frametitle{Unrelated Title}


\begin{itemize}
\item When the dispatcher switches one active process out for another, it must use a process-specific page table to overwrite the existing values stored in the hardware page table.
\end{itemize}

\note[item]{}
\end{frame}
\begin{frame}
\frametitle{Unrelated Title}


\begin{itemize}
\item Privileged instructions (that must be executed in kernel mode).
\end{itemize}

\note[item]{}
\end{frame}
\begin{frame}
\frametitle{Unrelated Title}


\begin{itemize}
\item A set of hardware registers.
\end{itemize}

\note[item]{}
\end{frame}
\begin{frame}
\frametitle{Unrelated Title}


\begin{itemize}
\item The operating system stores each process's page table somewhere in main memory. When the dispatcher performs a context switch, it updates a value stored in the CPU's page-table base register (PTBR) to match the value defined in the new process's task control block.
\end{itemize}

\note[item]{}
\end{frame}
\begin{frame}
\frametitle{Unrelated Title}


\begin{itemize}
\item Because each (user space) memory access would actually require 2 lookups:• One to look up the starting location of the process's page table, and• A second to look up the absolute address that was calculated after indexing into the page table.
\end{itemize}

\note[item]{}
\end{frame}
\begin{frame}
\frametitle{Unrelated Title}


\begin{itemize}
\item A hardware cache (table) composed of high-speed associative memory. The TLB stores key-value (or "tag-value") entries—in which each key represents a logical page number and each value is a physical frame number.The CPU presents the page number of a logical address to the TLB, which performs a fast lookup and returns a matching frame number (if one is found).
\end{itemize}

\note[item]{}
\end{frame}
\begin{frame}
\frametitle{Unrelated Title}


\begin{itemize}
\item The size of the TLB cache normally ranges from 64 to 1024 entries.
\end{itemize}

\note[item]{}
\end{frame}
\begin{frame}
\frametitle{Unrelated Title}


\begin{itemize}
\item 1. Least recently used (LRU).2. Random selection.
\end{itemize}

\note[item]{}
\end{frame}
\begin{frame}
\frametitle{Unrelated Title}


\begin{itemize}
\item This means that those entries cannot be evicted from the TLB. Most wired-down entries point to kernel data and routines (e.g., a pointer to the ready queue, etc).
\end{itemize}

\note[item]{}
\end{frame}
\begin{frame}
\frametitle{Unrelated Title}


\begin{itemize}
\item Each entry in the cache can associated (labeled) with a unique process. If the currently-running process causes the CPU to present a page number associated with a different process, the TLB treats it as a cache miss; this eventually generates a memory access violation for the running process.
\end{itemize}

\note[item]{}
\end{frame}
\begin{frame}
\frametitle{Unrelated Title}


\begin{itemize}
\item Without ASIDs, we'd have to flush (erase) the TLB with every context switch; otherwise, a valid page number may be mapped to an incorrect physical address (i.e., within a frame that is not mapped to the new process).
\end{itemize}

\note[item]{}
\end{frame}
\begin{frame}
\frametitle{Unrelated Title}


\begin{itemize}
\item This bit is set by the operating system to indicate whether or not the entry's associated page lies inside of the current process's logical address space. If the bit is not set (i.e., invalid), then the process may not look up into the entry (this would result in a trap).
\end{itemize}

\note[item]{}
\end{frame}
\begin{frame}
\frametitle{Unrelated Title}


\begin{itemize}
\item Suppose we have a procedure that starts executing on behalf of one process—only to have it be interrupted by another process that also begins executing the same procedure (before the initial execution completes).If the second execution can complete—allowing the first to continue—without changing the final result (i.e., no side effects), then we can say that this procedure (code) is re-entrant.Re-entrant code must not be modifiable by the code itself (i.e., no shared state).
\end{itemize}

\note[item]{}
\end{frame}
\begin{frame}
\frametitle{Unrelated Title}


\begin{itemize}
\item Re-entrant.
\end{itemize}

\note[item]{}
\end{frame}
\begin{frame}
\frametitle{Unrelated Title}


\begin{itemize}
\item A forward-mapped page table (as address translation works from the outer page table inward).
\end{itemize}

\note[item]{}
\end{frame}
\begin{frame}
\frametitle{Unrelated Title}


\begin{itemize}
\item Three or more address components (e.g., outer page number, inner page number, and page offset).
\end{itemize}

\note[item]{}
\end{frame}
\begin{frame}
\frametitle{Unrelated Title}


\begin{itemize}
\item The section number.
\end{itemize}

\note[item]{}
\end{frame}
\begin{frame}
\frametitle{Unrelated Title}


\begin{itemize}
\item [$]N + 1[/$]
\end{itemize}

\note[item]{}
\end{frame}
\begin{frame}
\frametitle{Unrelated Title}


\begin{itemize}
\item When the system's address space is quite large (i.e., with 64-bit architectures).
\end{itemize}

\note[item]{}
\end{frame}
\begin{frame}
\frametitle{Unrelated Title}


\begin{itemize}
\item We the single-level or multi-level page table structure with a hash table. The hash table maps virtual page numbers (as keys) to entries that contain a mapped physical frame address.
\end{itemize}

\note[item]{}
\end{frame}
\begin{frame}
\frametitle{Unrelated Title}


\begin{itemize}
\item Keys are hashed and mapped to a linked list of entries; each entry consists of 3 components:1. A virtual page number (i.e., key).2. The associated mapped physical frame number.3. A pointer to the next element in the linked list.When a virtual address is presented by the CPU, the memory unit hashes the page number and performs a lookup into the hash table. If an associated list of entries exists in the table, the procedure iterates through the entries, looking for a match (i.e., key that was hashed must match the first component in the entry).
\end{itemize}

\note[item]{}
\end{frame}
\begin{frame}
\frametitle{Unrelated Title}


\begin{itemize}
\item The operating system can quickly calculate the offset (location) of the page number's corresponding physical address entry in the table.
\end{itemize}

\note[item]{}
\end{frame}
\begin{frame}
\frametitle{Unrelated Title}


\begin{itemize}
\item The process is unlikely to use every page that belongs to its address space, so some table space may be wasted.
\end{itemize}

\note[item]{}
\end{frame}
\begin{frame}
\frametitle{Unrelated Title}


\begin{itemize}
\item An inverted page table has one entry for each physical frame of memory in the system. Each entry consists of the set of virtual (logical) page numbers (and its associated address-space identifier) that map to that frame. Thus, it follows that only one page table has to be allocated for the entire system (all processes).
\end{itemize}

\note[item]{}
\end{frame}
\begin{frame}
\frametitle{Unrelated Title}


\begin{itemize}
\item Given a logical page number, more time is needed to perform a lookup as we might have to check every index in the table looking for a match.
\end{itemize}

\note[item]{}
\end{frame}
\begin{frame}
\frametitle{Unrelated Title}


\begin{itemize}
\item A memory-management technique that divides a process's logical address space into several segments, each normally associated with some component of the program—code, data, stack, shared libraries, etc.
\end{itemize}

\note[item]{}
\end{frame}
\begin{frame}
\frametitle{Unrelated Title}


\begin{itemize}
\item Each segment is assigned a name, a segment number, and a length (i.e., size). In a segmented memory scheme, each logical address consists of a two-tuple  (<segment-number, offset>).
\end{itemize}

\note[item]{}
\end{frame}
\begin{frame}
\frametitle{Unrelated Title}


\begin{itemize}
\item The compiler.
\end{itemize}

\note[item]{}
\end{frame}
\begin{frame}
\frametitle{Unrelated Title}


\begin{itemize}
\item 1. Code segment (i.e., "text segment").2. Data segments (i.e., "data segment" and "BSS segment").3. Heap segment.4. Stack segment(s) (for each thread).5. The standard C library.
\end{itemize}

\note[item]{}
\end{frame}
\begin{frame}
\frametitle{Unrelated Title}


\begin{itemize}
\item Segment table.
\end{itemize}

\note[item]{}
\end{frame}
\begin{frame}
\frametitle{Unrelated Title}


\begin{itemize}
\item Segment base / segment limit
\end{itemize}

\note[item]{}
\end{frame}
\begin{frame}
\frametitle{Unrelated Title}


\begin{itemize}
\item Base-limit register pairs.
\end{itemize}

\note[item]{}
\end{frame}
\begin{frame}
\frametitle{Unrelated Title}


\begin{itemize}
\item 1. Programs can be larger than what would otherwise fit into physical memory.2. Programmers are freed from the concerns of memory storage limitations.3. Processes can share data and files easily (via shared virtual memory pages).
\end{itemize}

\note[item]{}
\end{frame}
\begin{frame}
\frametitle{Unrelated Title}


\begin{itemize}
\item Demand paging
\end{itemize}

\note[item]{}
\end{frame}
\begin{frame}
\frametitle{Unrelated Title}


\begin{itemize}
\item 1. Code for handling unusual error conditions.2. Program data structures that are rarely filled entirely.3. Code supporting options and features that may be used rarely.
\end{itemize}

\note[item]{}
\end{frame}
\begin{frame}
\frametitle{Unrelated Title}


\begin{itemize}
\item 1. Programs would not be constrained to the limits of physical memory. They could use an extremely large virtual address space managed by the operating system.2. Each active process would require less memory to run, allowing for a greater degree of multiprogramming; this could improve CPU utilization and throughput.3. Programs would not need to be swapped to disk as often, reducing the amount of I/O performed by the system.
\end{itemize}

\note[item]{}
\end{frame}
\begin{frame}
\frametitle{Unrelated Title}


\begin{itemize}
\item With a virtual memory scheme, we introduce a logical memory space that is separate from the physical memory. This logical (or virtual) address space can be much larger than the system's physical memory constraints.For each process in the system, we define a new logical (virtual) address space beginning at some logical base address—normally address 0x00000000.
\end{itemize}

\note[item]{}
\end{frame}
\begin{frame}
\frametitle{Unrelated Title}


\begin{itemize}
\item Upwards in memory
\end{itemize}

\note[item]{}
\end{frame}
\begin{frame}
\frametitle{Unrelated Title}


\begin{itemize}
\item Downwards in memory.
\end{itemize}

\note[item]{}
\end{frame}
\begin{frame}
\frametitle{Unrelated Title}


\begin{itemize}
\item A virtual address space that includes empty regions, or "holes".
\end{itemize}

\note[item]{}
\end{frame}
\begin{frame}
\frametitle{Unrelated Title}


\begin{itemize}
\item Files and memory (data).
\end{itemize}

\note[item]{}
\end{frame}
\begin{frame}
\frametitle{Unrelated Title}


\begin{itemize}
\item 1. System libraries can be loaded into one location in memory and mapped into the virtual address spaces of several different processes. This alleviates the need to store multiple copies of the libraries in memory.2. Using shared pages of memory, we can offer a faster means of process creation; a child process can be quickly set up to share its parent's virtual address space and mappings.
\end{itemize}

\note[item]{}
\end{frame}
\begin{frame}
\frametitle{Unrelated Title}


\begin{itemize}
\item A paging strategy in which pages are loaded into memory on-demand: pages are only loaded from disk when they are needed for execution. Pages that are never accessed are thus never loaded into memory.
\end{itemize}

\note[item]{}
\end{frame}
\begin{frame}
\frametitle{Unrelated Title}


\begin{itemize}
\item Swapping
\end{itemize}

\note[item]{}
\end{frame}
\begin{frame}
\frametitle{Unrelated Title}


\begin{itemize}
\item pages
\end{itemize}

\note[item]{}
\end{frame}
\begin{frame}
\frametitle{Unrelated Title}


\begin{itemize}
\item We establish a convention where the presence of an invalid bit indicates that either (a) the page is not valid (i.e., outside of the process's address space), or (b) the page is valid but currently not memory-resident.When updating a process's page table, we can choose to mark the non-resident pages as invalid, or store the disk address of the page as part of the entry.
\end{itemize}

\note[item]{}
\end{frame}
\begin{frame}
\frametitle{Unrelated Title}


\begin{itemize}
\item A hardware exception that occurs when one process attempts to access a page that is not resident in memory.
\end{itemize}

\note[item]{}
\end{frame}
\begin{frame}
\frametitle{Unrelated Title}


\begin{itemize}
\item It must determine whether or not the offending memory reference was valid (i.e., in a page that was not memory-resident) or invalid (i.e., in a page that is outside of the process's address space).
\end{itemize}

\note[item]{}
\end{frame}
\begin{frame}
\frametitle{Unrelated Title}


\begin{itemize}
\item A fault caused by a valid reference occurs because the referenced page belongs to the process but is not memory-resident (i.e., the page is located on disk).A fault caused by an invalid reference occurs because the process attempted to access a memory address that exists outside of the process's address space.
\end{itemize}

\note[item]{}
\end{frame}
\begin{frame}
\frametitle{Unrelated Title}


\begin{itemize}
\item By terminating the process that issued the invalid reference.
\end{itemize}

\note[item]{}
\end{frame}
\begin{frame}
\frametitle{Unrelated Title}


\begin{itemize}
\item 1. If the memory reference is deemed valid, we need to load the page in from disk.2. Schedule an I/O operation to read the page in from disk (the request may have to wait in the device's I/O queue).3. Update the page tables (including the process's page table structure), marking the page as resident.4. Roll back the program counter, restarting the instruction that triggered the page fault (this may require the processor to fetch operands again).
\end{itemize}

\note[item]{}
\end{frame}
\begin{frame}
\frametitle{Unrelated Title}


\begin{itemize}
\item It allows us to store the POPTs in the same physical frames of memory that store our process data.
\end{itemize}

\note[item]{}
\end{frame}
\begin{frame}
\frametitle{Unrelated Title}


\begin{itemize}
\item Pure demand paging has the operating system begin executing a process with no pages in memory. Execution starts with the instruction pointer pointing to an address in a non-resident page. The process immediately faults, causing the instruction(s) to be paged into memory.
\end{itemize}

\note[item]{}
\end{frame}
\begin{frame}
\frametitle{Unrelated Title}


\begin{itemize}
\item We can have the hardware (microcode) attempt to access the start and end addresses of the regions involved in the operation (i.e., [$]source_start[/$], [$]source_end[/$], [$]dest_start[/$], [$]dest_end[/$]). If any page fault is to occur, it can happen and be handled before any data is actually changed in memory.
\end{itemize}

\note[item]{}
\end{frame}
\begin{frame}
\frametitle{Unrelated Title}


\begin{itemize}
\item 10 nanoseconds
\end{itemize}

\note[item]{}
\end{frame}
\begin{frame}
\frametitle{Unrelated Title}


\begin{itemize}
\item 200 nanoseconds
\end{itemize}

\note[item]{}
\end{frame}
\begin{frame}
\frametitle{Unrelated Title}


\begin{itemize}
\item [$]ma_e = (1 - p) \times ma + p \times pf [/$]
\end{itemize}

\note[item]{}
\end{frame}
\begin{frame}
\frametitle{Unrelated Title}


\begin{itemize}
\item If the current process cannot continue executing until a page has been paged from disk, the operating system can save the process state (i.e., registers, stack, etc) and schedule a new process to run on the CPU. When the system receives an interrupt from the I/O subsystem (i.e., "page is ready"), it can take the waiting process off of the wait queue and move it back to the ready queue.
\end{itemize}

\note[item]{}
\end{frame}
\begin{frame}
\frametitle{Unrelated Title}


\begin{itemize}
\item About 3 milliseconds.
\end{itemize}

\note[item]{}
\end{frame}
\begin{frame}
\frametitle{Unrelated Title}


\begin{itemize}
\item About 5 milliseconds.
\end{itemize}

\note[item]{}
\end{frame}
\begin{frame}
\frametitle{Unrelated Title}


\begin{itemize}
\item About 50 microseconds.
\end{itemize}

\note[item]{}
\end{frame}
\begin{frame}
\frametitle{Unrelated Title}


\begin{itemize}
\item Page-fault rate.
\end{itemize}

\note[item]{}
\end{frame}
\begin{frame}
\frametitle{Unrelated Title}


\begin{itemize}
\item 3 milliseconds of latency +5 milliseconds for a disk seek +50 microseconds for transfer =Roughly 8 milliseconds.
\end{itemize}

\note[item]{}
\end{frame}
\begin{frame}
\frametitle{Unrelated Title}


\begin{itemize}
\item Swap space is allocated in much larger blocks. Also, read and write operations to swap space are not subject to file lookups, indirect allocation methods, and other operations used by file system implementations.
\end{itemize}

\note[item]{}
\end{frame}
\begin{frame}
\frametitle{Unrelated Title}


\begin{itemize}
\item The operating system can copy the entire binary file from the file system into swap space, and then perform demand-paging directly from swap space. The upfront cost of copying on disk may be offset by the faster paging that results.
\end{itemize}

\note[item]{}
\end{frame}
\begin{frame}
\frametitle{Unrelated Title}


\begin{itemize}
\item The process's program stack and heap.
\end{itemize}

\note[item]{}
\end{frame}
\begin{frame}
\frametitle{Unrelated Title}


\begin{itemize}
\item If a page is marked read-only (e.g., a page storing a binary file, or other read-only data), we can evict it from memory without writing it back to swap space. In this scenario, the page can be read again from the file system the next time it's needed.
\end{itemize}

\note[item]{}
\end{frame}
\begin{frame}
\frametitle{Unrelated Title}


\begin{itemize}
\item A copy-on-write scheme allows multiple processes (e.g., a parent and child) to share the same page (i.e., frame) in memory under the guise that each process owns it. We mark a page as "copy-on-write" and we only copy the page (to another frame in memory) when one of the processes attempts to modify data within that page. Writable pages that are never modified are never copied in memory.
\end{itemize}

\note[item]{}
\end{frame}
\begin{frame}
\frametitle{Unrelated Title}


\begin{itemize}
\item In modern Unix variants, the fork() system call leverages copy-on-write features that allow a child process to have certain elements initially "shadow" the parent's.
\end{itemize}

\note[item]{}
\end{frame}
\begin{frame}
\frametitle{Unrelated Title}


\begin{itemize}
\item We could maintain a pool of free pages from which to allocate; we can quickly take pages from this pool when a process attempts to write to copy-on-write pages.
\end{itemize}

\note[item]{}
\end{frame}
\begin{frame}
\frametitle{Unrelated Title}


\begin{itemize}
\item With copy-on-write, we assign the process a virtual page that maps to a dedicated "zero page" initialized by the operating system. We can wait to generate an actual zero-filled page when the process actually needs to modify the memory.
\end{itemize}

\note[item]{}
\end{frame}
\begin{frame}
\frametitle{Unrelated Title}


\begin{itemize}
\item When a process requests a large (zero-filled) array, the operating system can assign and map several virtual pages to the system's zero page. Actual frames of memory can be allocated on-demand whenever the program writes values to the array.
\end{itemize}

\note[item]{}
\end{frame}
\begin{frame}
\frametitle{Unrelated Title}


\begin{itemize}
\item The vfork() system call suspends the parent process, temporarily assigning the parent's address space to the child. When the child terminates, the parent regains its address space, and any changes made by the child will be visible to the parent.In contrast, fork() uses copy-on-write to assign a "new" address space to the child. Any modifications made by the child to memory in its address space will cause pages to be copied into dedicated frames of memory for the child.
\end{itemize}

\note[item]{}
\end{frame}
\begin{frame}
\frametitle{Unrelated Title}


\begin{itemize}
\item 1. Pages allocated for a process.2. Buffers for I/O.
\end{itemize}

\note[item]{}
\end{frame}
\begin{frame}
\frametitle{Unrelated Title}


\begin{itemize}
\item 1. The page must be written to swap space (i.e., copied to disk).2. The system's page tables must be updated to indicate that the page is no longer resident in memory.
\end{itemize}

\note[item]{}
\end{frame}
\begin{frame}
\frametitle{Unrelated Title}


\begin{itemize}
\item We can shorten the page-fault service time by explicitly marking each frame as either "clean" (unmodified) or "dirty" (modified). Unmodified pages can be selected first as victims as they need not be written back to disk. When the unmodified page is needed again by a process, it can be read back from disk in its original form.
\end{itemize}

\note[item]{}
\end{frame}
\begin{frame}
\frametitle{Unrelated Title}


\begin{itemize}
\item When a frame (page) that is resident in memory is selected for eviction (to make room for another page that is needed), we call it a "victim frame".
\end{itemize}

\note[item]{}
\end{frame}
\begin{frame}
\frametitle{Unrelated Title}


\begin{itemize}
\item 1. A frame-allocation algorithm.2. A page-replacement algorithm.
\end{itemize}

\note[item]{}
\end{frame}
\begin{frame}
\frametitle{Unrelated Title}


\begin{itemize}
\item How many frames of memory to allocate to each active process.
\end{itemize}

\note[item]{}
\end{frame}
\begin{frame}
\frametitle{Unrelated Title}


\begin{itemize}
\item Which page (frame) to evict from memory when a new page is needed in memory.
\end{itemize}

\note[item]{}
\end{frame}
\begin{frame}
\frametitle{Unrelated Title}


\begin{itemize}
\item We look at which algorithms lead to the lowest page-fault rate.
\end{itemize}

\note[item]{}
\end{frame}
\begin{frame}
\frametitle{Unrelated Title}


\begin{itemize}
\item A string of memory references (accesses).
\end{itemize}

\note[item]{}
\end{frame}
\begin{frame}
\frametitle{Unrelated Title}


\begin{itemize}
\item We can use a reference string (either generated randomly or gathered from a trace) to simulate a long series of memory accesses, using one page-replacement algorithm at a time and recording the number of page faults that are generated. By looking at the page-fault rate for each algorithm, we can compare their relative behavior and performance.
\end{itemize}

\note[item]{}
\end{frame}
\begin{frame}
\frametitle{Unrelated Title}


\begin{itemize}
\item Generally, as the number of frames increases, the page-fault rate decreases.
\end{itemize}

\note[item]{}
\end{frame}
\begin{frame}
\frametitle{Unrelated Title}


\begin{itemize}
\item 1. First-in, first-out (FIFO).2. Least recently used (LRU).3. Least frequently used (LFU).4. Most frequently used (MFU).
\end{itemize}

\note[item]{}
\end{frame}
\begin{frame}
\frametitle{Unrelated Title}


\begin{itemize}
\item A FIFO queue. Victim pages are taken from the head of the queue. When a new page is brought into memory, it is added at the tail of the queue.
\end{itemize}

\note[item]{}
\end{frame}
\begin{frame}
\frametitle{Unrelated Title}


\begin{itemize}
\item A phenomena that is sometimes seen in which increasing the number of frames allocated to a process actually increases the page-fault rate for the process.
\end{itemize}

\note[item]{}
\end{frame}
\begin{frame}
\frametitle{Unrelated Title}


\begin{itemize}
\item future knowledge of each process's memory accesses.
\end{itemize}

\note[item]{}
\end{frame}
\begin{frame}
\frametitle{Unrelated Title}


\begin{itemize}
\item • The FIFO algorithm uses the time when a page was last brought into memory.• An optimal algorithm would use the time when a page is to be used next.
\end{itemize}

\note[item]{}
\end{frame}
\begin{frame}
\frametitle{Unrelated Title}


\begin{itemize}
\item The recent past, and the near future.
\end{itemize}

\note[item]{}
\end{frame}
\begin{frame}
\frametitle{Unrelated Title}


\begin{itemize}
\item 1. Counters: Each page-table entry includes a time-of-use field; whenever the page is referenced, we copy the value of the processor's system clock register to this field. The page with the smallest (oldest) timestamp is selected for replacement.2. Stack: Whenever a page is referenced, it is moved to the top of the stack; the page at the bottom of the stack is selected for replacement. This can be implemented efficiently using a doubly-linked list.
\end{itemize}

\note[item]{}
\end{frame}
\begin{frame}
\frametitle{Unrelated Title}


\begin{itemize}
\item An algorithm for which it can be shown that: the set of pages [$]P_n[/$] in memory for n frames is always a subset of the set of pages [$]P_{n+1}[/$] that would be in memory with [$]n + 1[/$] frames.
\end{itemize}

\note[item]{}
\end{frame}
\begin{frame}
\frametitle{Unrelated Title}


\begin{itemize}
\item A bit given to each entry in the page table. The bit is set by the hardware whenever an address within the associated page page is accessed by a process.
\end{itemize}

\note[item]{}
\end{frame}
\begin{frame}
\frametitle{Unrelated Title}


\begin{itemize}
\item As processes run in the system and access memory, the reference bits of various pages are set (to 1) by the hardware. Our operating system can read these bits and prefer to evict those pages that have not had their reference bit set. Thus, more recently used pages are more likely to remain resident.
\end{itemize}

\note[item]{}
\end{frame}
\begin{frame}
\frametitle{Unrelated Title}


\begin{itemize}
\item A basic LRU page-replacement algorithm that relies on a single reference bit associated with each entry in the page table:• The algorithm organizes entries into a circular queue.• A moving queue cursor (pointer) points to the next page to be replaced.• When a frame is needed, the cursor advances to the next unreference page, clearing the reference bits of any referenced pages that is passes over.Thus, when a page is referenced by a program, it gets a "second chance" before becoming eligible again for eviction.
\end{itemize}

\note[item]{}
\end{frame}
\begin{frame}
\frametitle{Unrelated Title}


\begin{itemize}
\item The clock algorithm.
\end{itemize}

\note[item]{}
\end{frame}
\begin{frame}
\frametitle{Unrelated Title}


\begin{itemize}
\item 1. Initialize a system timer to fire at some regular interval—say, every 100 milliseconds. Have the timer callback copy the value of the reference bit for each page table entry into an 8-bit field associated with that entry.2. Before copying the bit, right-shift the existing value stored in the byte to make room for the new bit. The new bit is copied to the highest-order position in the byte.3. Thus, at most, we can record the last 8 "reference states" of each page. If we interpret the byte as an unsigned integer, we can have our page-replacement algorithm prefer to evict pages with the lowest value—as these have been referenced less recently.
\end{itemize}

\note[item]{}
\end{frame}
\begin{frame}
\frametitle{Unrelated Title}


\begin{itemize}
\item We can use the reference bit together with the dirty bit to classify the page into one of four classes:(0,0) - Neither recently used nor modified.(0,1) - Not recently used, but modified.(1,0) - Recently used, but not modified.(1,1) - Recently used and modified.How these classes are used to select a victim is determined by protocol.
\end{itemize}

\note[item]{}
\end{frame}
\begin{frame}
\frametitle{Unrelated Title}


\begin{itemize}
\item (0,0) - Neither recently used nor modified.(0,1) - Not recently used, but modified.(1,0) - Recently used, but not modified.(1,1) - Recently used and modified.
\end{itemize}

\note[item]{}
\end{frame}
\begin{frame}
\frametitle{Unrelated Title}


\begin{itemize}
\item The operating system may select a free frame from its frame pool and use it to store the new page. This way, the new page does not have to wait for the victim page to be written to disk (this can happen asynchronously at a time when the disk is idle).
\end{itemize}

\note[item]{}
\end{frame}
\begin{frame}
\frametitle{Unrelated Title}


\begin{itemize}
\item Write the victim page to a frame in the free frame pool before writing it out to disk. If a process immediately references our victim page, we can quickly map the victim page back into the process's address space (and cancel the write-back to disk).
\end{itemize}

\note[item]{}
\end{frame}
\begin{frame}
\frametitle{Unrelated Title}


\begin{itemize}
\item 1. Databases: These applications often provide their own memory management and I/O buffering, and duplicating these features at the operating system level can be wasteful and inefficient.2. Data warehouses: These applications may perform massive sequential disk reads, followed by computations an disk writes. An MFU replacement policy would actually out-perform a LRU policy in this case.
\end{itemize}

\note[item]{}
\end{frame}
\begin{frame}
\frametitle{Unrelated Title}


\begin{itemize}
\item The maximum possible number of page references (for a single instruction) is determined by the instruction set architecture.
\end{itemize}

\note[item]{}
\end{frame}
\begin{frame}
\frametitle{Unrelated Title}


\begin{itemize}
\item Each indirect memory reference may be referencing memory inside a different page frame. Thus, N indirect references may result in the processor referencing N unique page frames in memory. If this number is greater than the maximum frame allocation for the process, then the operation can never complete.
\end{itemize}

\note[item]{}
\end{frame}
\begin{frame}
\frametitle{Unrelated Title}


\begin{itemize}
\item A frame-allocation algorithm that divides the available frames equally amongst all active processes in the system. If frames cannot be divided evenly, the remaining frames can be used as a free-frame pool.
\end{itemize}

\note[item]{}
\end{frame}
\begin{frame}
\frametitle{Unrelated Title}


\begin{itemize}
\item A frame-allocation algorithm that allocates available frames to processes according to the virtual memory needs (usage) of each process.
\end{itemize}

\note[item]{}
\end{frame}
\begin{frame}
\frametitle{Unrelated Title}


\begin{itemize}
\item [$]a_i = \frac{s_i }{S} \times m[/$]
\end{itemize}

\note[item]{}
\end{frame}
\begin{frame}
\frametitle{Unrelated Title}


\begin{itemize}
\item As the number of running processes increases, the number of pages available to each process decreases.
\end{itemize}

\note[item]{}
\end{frame}
\begin{frame}
\frametitle{Unrelated Title}


\begin{itemize}
\item • The amount of virtual memory used by a process.• The relative priority of each process.• A combination of size and priority.
\end{itemize}

\note[item]{}
\end{frame}
\begin{frame}
\frametitle{Unrelated Title}


\begin{itemize}
\item With global replacement, a process in need of additional pages may evict frame currently storing pages belonging to another process. With local replacement, a process may only replace frames storing pages that belong to that process.
\end{itemize}

\note[item]{}
\end{frame}
\begin{frame}
\frametitle{Unrelated Title}


\begin{itemize}
\item Local replacement
\end{itemize}

\note[item]{}
\end{frame}
\begin{frame}
\frametitle{Unrelated Title}


\begin{itemize}
\item Global replacement
\end{itemize}

\note[item]{}
\end{frame}
\begin{frame}
\frametitle{Unrelated Title}


\begin{itemize}
\item Because global replacement can be shown to have greater system throughput.
\end{itemize}

\note[item]{}
\end{frame}
\begin{frame}
\frametitle{Unrelated Title}


\begin{itemize}
\item With a local replacement protocol, a process (program) has some degree of control over its page-fault rate. The program may operate using a fixed set of pages, and, thus, it may be optimized by the programmer to target a certain expected rate.With a global replacement protocol, other processes may evict pages that are currently being used by the process, increasing its page-fault rate in a manner that the process has no control over.
\end{itemize}

\note[item]{}
\end{frame}
\begin{frame}
\frametitle{Unrelated Title}


\begin{itemize}
\item Thrashing is a phenomenon that occurs when certain circumstances cause the paging rate of a process (or, consequently, a system) to be very high.
\end{itemize}

\note[item]{}
\end{frame}
\begin{frame}
\frametitle{Unrelated Title}


\begin{itemize}
\item Thrashing
\end{itemize}

\note[item]{}
\end{frame}
\begin{frame}
\frametitle{Unrelated Title}


\begin{itemize}
\item A local replacement strategy, as one process that is thrashing cannot "steal" frames from another process, causing that process to thrash in turn.
\end{itemize}

\note[item]{}
\end{frame}
\begin{frame}
\frametitle{Unrelated Title}


\begin{itemize}
\item When a process thrashes, it places many paging requests on the device queue. This increases the average time needed to service requests from other processes that inevitably page-fault (e.g., to access new instructions and data).
\end{itemize}

\note[item]{}
\end{frame}
\begin{frame}
\frametitle{Unrelated Title}


\begin{itemize}
\item From locality to locality (in memory)
\end{itemize}

\note[item]{}
\end{frame}
\begin{frame}
\frametitle{Unrelated Title}


\begin{itemize}
\item A locality is a set of pages that are actively being used together by a process. As a program executes, it typically enters (or visits) several localities.
\end{itemize}

\note[item]{}
\end{frame}
\begin{frame}
\frametitle{Unrelated Title}


\begin{itemize}
\item working set (window)
\end{itemize}

\note[item]{}
\end{frame}
\begin{frame}
\frametitle{Unrelated Title}


\begin{itemize}
\item The working-set model uses a parameter ([$]\theta[/$]) to define a process's working-set window. The window is defined by the [$]\theta[/$] most recently referenced pages. If a page is in active use, it will be in the process's working set.
\end{itemize}

\note[item]{}
\end{frame}
\begin{frame}
\frametitle{Unrelated Title}


\begin{itemize}
\item locality
\end{itemize}

\note[item]{}
\end{frame}
\begin{frame}
\frametitle{Unrelated Title}


\begin{itemize}
\item size
\end{itemize}

\note[item]{}
\end{frame}
\begin{frame}
\frametitle{Unrelated Title}


\begin{itemize}
\item At any given time, a process needs [$]WSS_i[/$] frames. An increase in the size of a process's working set is an increase in the amount of memory that it needs to be resident in order to continue execution. A decrease in this number is a decrease in frame demand.
\end{itemize}

\note[item]{}
\end{frame}
\begin{frame}
\frametitle{Unrelated Title}


\begin{itemize}
\item [$]D = \sum{WSS_i}[/$]
\end{itemize}

\note[item]{}
\end{frame}
\begin{frame}
\frametitle{Unrelated Title}


\begin{itemize}
\item The total number of available frames in the system.
\end{itemize}

\note[item]{}
\end{frame}
\begin{frame}
\frametitle{Unrelated Title}


\begin{itemize}
\item By selecting one or more processes to suspend—moving its process state and associated pages out of main memory and into the backing store. Doing so causes a certain number of frames to become available to other processes, which may fix the thrashing.
\end{itemize}

\note[item]{}
\end{frame}
\begin{frame}
\frametitle{Unrelated Title}


\begin{itemize}
\item For each page table entry, we include a field storing one or more "history bits". We then set up a timer to fire at some fixed interval. When we get the timer interrupt, we copy the page entry's reference bit (set by the hardware) into the high-order history bit. We then shift any existing history bits towards the lowest-order position. We also clear the reference bit before the next timer interrupt.
\end{itemize}

\note[item]{}
\end{frame}
\begin{frame}
\frametitle{Unrelated Title}


\begin{itemize}
\item 1. Increase the number of history bits allocated for the page table entry.2. Increase the frequency of the timer interrupt (i.e., reference-checking routine).
\end{itemize}

\note[item]{}
\end{frame}
\begin{frame}
\frametitle{Unrelated Title}


\begin{itemize}
\item Its page-fault frequency.
\end{itemize}

\note[item]{}
\end{frame}
\begin{frame}
\frametitle{Unrelated Title}


\begin{itemize}
\item An upper bound and a lower bound.
\end{itemize}

\note[item]{}
\end{frame}
\begin{frame}
\frametitle{Unrelated Title}


\begin{itemize}
\item We map a disk block to a page (or several pages) in main memory. Doing so allows us to leverage the paging system to read in and store portions of a file on-demand.
\end{itemize}

\note[item]{}
\end{frame}
\begin{frame}
\frametitle{Unrelated Title}


\begin{itemize}
\item After the requested portions of the file are brought into main memory (as one or more frames), the operating system can work with the data directly in memory—initially bypassing expensive disk I/O and file-system operations.
\end{itemize}

\note[item]{}
\end{frame}
\begin{frame}
\frametitle{Unrelated Title}


\begin{itemize}
\item 1. If the file data was modified, it may need to be written to the backing store. A new write request would be placed on the storage device I/O queue.2. The file mapping can be removed from the process's virtual address space (i.e., the page table entries for the pages mapping the file can be marked as invalid).3. The underlying memory frames can be returned to the free-frame pool.
\end{itemize}

\note[item]{}
\end{frame}
\begin{frame}
\frametitle{Unrelated Title}


\begin{itemize}
\item Multiple processes can map the same file (in memory) to their own sets of pages. Each set of page table entries can be set up to point to the same underlying frames holding the file data. This allows a single copy of the data to exist in memory.
\end{itemize}

\note[item]{}
\end{frame}
\begin{frame}
\frametitle{Unrelated Title}


\begin{itemize}
\item When the file is mapped into each process's virtual address space, we can mark the mapping as copy-on-write-enabled. When a process attempts to write its own data to the file, new frames are allocated to the process to store a copy of the file data.
\end{itemize}

\note[item]{}
\end{frame}
\begin{frame}
\frametitle{Unrelated Title}


\begin{itemize}
\item The mmap() system call.
\end{itemize}

\note[item]{}
\end{frame}
\begin{frame}
\frametitle{Unrelated Title}


\begin{itemize}
\item The shmget() and shmat() system calls.
\end{itemize}

\note[item]{}
\end{frame}
\begin{frame}
\frametitle{Unrelated Title}


\begin{itemize}
\item Memory-mapped files.
\end{itemize}

\note[item]{}
\end{frame}
\begin{frame}
\frametitle{Unrelated Title}


\begin{itemize}
\item 1. CreateFileMapping()2. MapViewOfFile()3. UnmapViewOfFile()
\end{itemize}

\note[item]{}
\end{frame}
\begin{frame}
\frametitle{Unrelated Title}


\begin{itemize}
\item Named objects.
\end{itemize}

\note[item]{}
\end{frame}
\begin{frame}
\frametitle{Unrelated Title}


\begin{itemize}
\item A hardware technique wherein a range of memory addresses are mapped to the dedicated registers of an I/O device. Whenever a process reads from or writes to these addresses, data is actually transferred to and from the device register(s).
\end{itemize}

\note[item]{}
\end{frame}
\begin{frame}
\frametitle{Unrelated Title}


\begin{itemize}
\item 1. A video controller, which normally has a fast response time.2. A modem, whose serial I/O port may need to consume data very quickly.
\end{itemize}

\note[item]{}
\end{frame}
\begin{frame}
\frametitle{Unrelated Title}


\begin{itemize}
\item I/O ports
\end{itemize}

\note[item]{}
\end{frame}
\begin{frame}
\frametitle{Unrelated Title}


\begin{itemize}
\item The system can memory-map the device's control and data registers to virtual addresses that are addressable by the CPU. The CPU can write one byte to the data registers and set a bit in the control register, indicating to the device that new data is available. The device clears the control bit before the CPU provides the next byte of data.
\end{itemize}

\note[item]{}
\end{frame}
\begin{frame}
\frametitle{Unrelated Title}


\begin{itemize}
\item • Programmed I/O has the CPU frequently poll to check whether or not the control bit for the I/O device has been set (or unset) by the device. If the device is ready to receive new data, the CPU can place a new byte in the device register.• Interrupt-driven I/O requires the I/O device to issue a hardware interrupt to the CPU to indicate that the device is ready to take new data (over the I/O port or bus).
\end{itemize}

\note[item]{}
\end{frame}
\begin{frame}
\frametitle{Unrelated Title}


\begin{itemize}
\item 1. Many kernel data structures are less than 1 page in size. This can lead to external fragmentation, which prevents the kernel from keeping a small footprint.2. The kernel may interface with certain devices that expect an associated region of memory (i.e., a buffer) to be contiguous in order for the device to operate correctly.
\end{itemize}

\note[item]{}
\end{frame}
\begin{frame}
\frametitle{Unrelated Title}


\begin{itemize}
\item The buddy system aims to produce as many contiguous allocations as possible within a dedicated segment of memory. A kernel can use this scheme to manage its own memory allocations for internal data structures, etc.
\end{itemize}

\note[item]{}
\end{frame}
\begin{frame}
\frametitle{Unrelated Title}


\begin{itemize}
\item A power-of-two allocator.
\end{itemize}

\note[item]{}
\end{frame}
\begin{frame}
\frametitle{Unrelated Title}


\begin{itemize}
\item With the buddy system, we start with a dedicated root segment of memory and continually divide part of it in half until the resulting pair of child segments [$]A_L[/$] and [$]A_R[/$] are as small as possible while still satisfying the request size. One of these child segments is chosen to satisfy the request, and it is marked as "used" by the allocator.When a segment is freed by the kernel, it can be "coalesced" into neighboring available segments in a recursive manner. Free all allocations would result in a single free segment (that is, the root segment).
\end{itemize}

\note[item]{}
\end{frame}
\begin{frame}
\frametitle{Unrelated Title}


\begin{itemize}
\item With a power-of-two allocator, we cannot guarantee that less than 50% of the dedicated memory will be wasted due to internal fragmentation.
\end{itemize}

\note[item]{}
\end{frame}
\begin{frame}
\frametitle{Unrelated Title}


\begin{itemize}
\item Internal fragmentation
\end{itemize}

\note[item]{}
\end{frame}
\begin{frame}
\frametitle{Unrelated Title}


\begin{itemize}
\item We reserve a series of physically contiguous pages in memory, referring to it as a "cache". Each cache is associated with a particular kernel data structure, and is subdivided into multiple "slabs" of equal size; each slab is sized to a multiple of the associated data structure's size.The kernel populates each cache (and, thus, its slabs) with instances of the data structure. When a kernel process requires a new instance of the structure, we select a slab from the corresponding cache and mark one of its instances as "used".
\end{itemize}

\note[item]{}
\end{frame}
\begin{frame}
\frametitle{Unrelated Title}


\begin{itemize}
\item 1. Process control blocks / process descriptors.2. File objects.3. Semaphores.
\end{itemize}

\note[item]{}
\end{frame}
\begin{frame}
\frametitle{Unrelated Title}


\begin{itemize}
\item Kernel objects.
\end{itemize}

\note[item]{}
\end{frame}
\begin{frame}
\frametitle{Unrelated Title}


\begin{itemize}
\item 1. Empty: All objects in the slab are marked as "free".2. Full: All objects in the slab are marked as "used".3. Partially full: The slab holds a mix of free and used objects.
\end{itemize}

\note[item]{}
\end{frame}
\begin{frame}
\frametitle{Unrelated Title}


\begin{itemize}
\item 1. Fragmentation does not occur, as this scheme's unit of allocation perfectly matches the size of the object in memory being requested. It is not possible to request more or less memory than what is required to store the object.2. By pre-allocating and pre-initializing a number of object instances, requests for an instance can be satisfied very quickly.
\end{itemize}

\note[item]{}
\end{frame}
\begin{frame}
\frametitle{Unrelated Title}


\begin{itemize}
\item When a given kernel object is allocated and freed frequently by the system.
\end{itemize}

\note[item]{}
\end{frame}
\begin{frame}
\frametitle{Unrelated Title}


\begin{itemize}
\item A paging technique in which a page-fault may cause multiple pages to be brought in, in hopes of bringing more of a process's locality into memory.
\end{itemize}

\note[item]{}
\end{frame}
\begin{frame}
\frametitle{Unrelated Title}


\begin{itemize}
\item 1. When a process first begins executing, at which point it may have none of its pages resident in memory.2. When a process leaves the wait queue (e.g., after some I/O completes) and resumes execution, at which point its pages may have been replaced.
\end{itemize}

\note[item]{}
\end{frame}
\begin{frame}
\frametitle{Unrelated Title}


\begin{itemize}
\item Working set.
\end{itemize}

\note[item]{}
\end{frame}
\begin{frame}
\frametitle{Unrelated Title}


\begin{itemize}
\item 1. Fragmentation decreases with a smaller page size, resulting in better memory utilization across the entire system.2. Approximating program localities (i.e., working sets) becomes more accurate with a smaller page size, as we're working with a finer page resolution.
\end{itemize}

\note[item]{}
\end{frame}
\begin{frame}
\frametitle{Unrelated Title}


\begin{itemize}
\item 1. Using larger pages allows for fewer page table entries, and, thus, a smaller memory footprint for our page table(s).2. A larger page size means larger reads and writes to disk. This is generally more efficient (over time) as data transfer time is a very small percentage of the total time needed for disk operations.
\end{itemize}

\note[item]{}
\end{frame}
\begin{frame}
\frametitle{Unrelated Title}


\begin{itemize}
\item Roughly 1%.
\end{itemize}

\note[item]{}
\end{frame}
\begin{frame}
\frametitle{Unrelated Title}


\begin{itemize}
\item A TLB's reach is the total amount of (physical) memory that is directly accessible through the TLB cache.
\end{itemize}

\note[item]{}
\end{frame}
\begin{frame}
\frametitle{Unrelated Title}


\begin{itemize}
\item The table size (i.e., number of entries) multiplied by the page (frame) size.
\end{itemize}

\note[item]{}
\end{frame}
\begin{frame}
\frametitle{Unrelated Title}


\begin{itemize}
\item Memory accesses within the stack are more likely to be near the top than elsewhere. The closer we get to the top of the stack (and its associated memory area), the more likely it is that the next stack reference will occur in that area.
\end{itemize}

\note[item]{}
\end{frame}
\begin{frame}
\frametitle{Unrelated Title}


\begin{itemize}
\item A hash table scatters its entries across an allocated region of memory. Entries have no ordering, so sequential access to memory does not normally occur within the hash table structure. Thus, frequent access to large hash tables may cause a high frequency of cache memory misses and/or page faults.
\end{itemize}

\note[item]{}
\end{frame}
\begin{frame}
\frametitle{Unrelated Title}


\begin{itemize}
\item The compiler (or loader) chooses where in a program's logical address space to place instructions for routines, etc. These tools can elect to place each routine so that it does not cross page boundaries, and, thus, do not trigger as many page-faults during exection.
\end{itemize}

\note[item]{}
\end{frame}
\begin{frame}
\frametitle{Unrelated Title}


\begin{itemize}
\item The loader can "pack" multiple routines into individual pages in such a way that related routines are placed in the same page, or neighboring pages. This can improve locality and reduce the page-fault rate of the program.
\end{itemize}

\note[item]{}
\end{frame}
\begin{frame}
\frametitle{Unrelated Title}


\begin{itemize}
\item A program allocates memory from the heap, using pointers (variables storing heap addresses) to track its allocations. Heap allocators are typically implemented with free-lists, in which successive heap allocations may not necessarily be contiguous in the heap's underlying physical memory. Hence, accesses to various objects in the heap may cause frequent cache misses and/or page faults.
\end{itemize}

\note[item]{}
\end{frame}
\begin{frame}
\frametitle{Unrelated Title}


\begin{itemize}
\item "Locking" a frame marks it as "not eligible for eviction" by the page-replacement algorithm. Pages mapped to the locked frame will remain resident in memory until the frame is unlocked. We can use a bit field in each frame table entry to indicate whether or not the frame is currently locked.
\end{itemize}

\note[item]{}
\end{frame}
\begin{frame}
\frametitle{Unrelated Title}


\begin{itemize}
\item 1. When we initiate an I/O device operation—passing the address of an I/O buffer—we'd want to lock the pages of the buffer until the device writes it data back.2. We may choose to lock any page that is allocated for the kernel. Doing so simplifies the kernel design and improves kernel performance.
\end{itemize}

\note[item]{}
\end{frame}
\begin{frame}
\frametitle{Unrelated Title}


\begin{itemize}
\item 1. We could design our kernel to never have the I/O devices write directly to user-space memory. Instead, we'd pass the device the address of a buffer in kernel-space, and eventually copy the data from the kernel buffer into the process's space.2. We could allow pages (and their underlying frames) to be "locked", making them as ineligible for replacement by the page-replacement algorithm. We can do so by including a "lock bit" in each entry of the frame table.
\end{itemize}

\note[item]{}
\end{frame}
\begin{frame}
\frametitle{Unrelated Title}


\begin{itemize}
\item Instead of only reading in the requested page from the backing store, we also read in several neighboring pages. Clustering is a form of pre-paging.
\end{itemize}

\note[item]{}
\end{frame}
\begin{frame}
\frametitle{Unrelated Title}


\begin{itemize}
\item platters
\end{itemize}

\note[item]{}
\end{frame}
\begin{frame}
\frametitle{Unrelated Title}


\begin{itemize}
\item tracks
\end{itemize}

\note[item]{}
\end{frame}
\begin{frame}
\frametitle{Unrelated Title}


\begin{itemize}
\item sectors
\end{itemize}

\note[item]{}
\end{frame}
\begin{frame}
\frametitle{Unrelated Title}


\begin{itemize}
\item cylinder
\end{itemize}

\note[item]{}
\end{frame}
\begin{frame}
\frametitle{Unrelated Title}


\begin{itemize}
\item The seek time and the positioning time both refer to the time needed to move the disc arm to the desired cylinder.
\end{itemize}

\note[item]{}
\end{frame}
\begin{frame}
\frametitle{Unrelated Title}


\begin{itemize}
\item rotational latency
\end{itemize}

\note[item]{}
\end{frame}
\begin{frame}
\frametitle{Unrelated Title}


\begin{itemize}
\item memory-mapped I/O ports
\end{itemize}

\note[item]{}
\end{frame}
\begin{frame}
\frametitle{Unrelated Title}


\begin{itemize}
\item The host controller is directly connected to the computer, in that it's located at the computer's end of the bus. The disc controller is normally built into the disc drive, and so it sits at the device end of the bus.
\end{itemize}

\note[item]{}
\end{frame}
\begin{frame}
\frametitle{Unrelated Title}


\begin{itemize}
\item Disc controllers typically have a built-in cache which sits between the host controller (i.e., data bus) and the physical disc storage (i.e., magnetic platters).
\end{itemize}

\note[item]{}
\end{frame}
\begin{frame}
\frametitle{Unrelated Title}


\begin{itemize}
\item A logical block is the smallest unit of transfer to and from a disc device.
\end{itemize}

\note[item]{}
\end{frame}
\begin{frame}
\frametitle{Unrelated Title}


\begin{itemize}
\item low-level formatting
\end{itemize}

\note[item]{}
\end{frame}
\begin{frame}
\frametitle{Unrelated Title}


\begin{itemize}
\item These discs' outer tracks have a higher bit density than their inner tracks; thus, more data fits into the longer, outer tracks. The disc hardware increases the rotational speed as the read-write head moves closer to the center tracks, allowing the read/write speed to stay constant during operations.
\end{itemize}

\note[item]{}
\end{frame}
\begin{frame}
\frametitle{Unrelated Title}


\begin{itemize}
\item In these devices, the density of bits decreases as we move from the inner tracks to the outer tracks. This allows the disc to spin at a constant velocity regardless of where the read-write head is positioned, maintaining a constant read/write speed.
\end{itemize}

\note[item]{}
\end{frame}
\begin{frame}
\frametitle{Unrelated Title}


\begin{itemize}
\item 1. Constant linear velocity is normally used by CD-ROM and DVD-ROM drives.2. Constant angular velocity is normally used by hard discs.
\end{itemize}

\note[item]{}
\end{frame}
\begin{frame}
\frametitle{Unrelated Title}


\begin{itemize}
\item 1. Host-attached storage (i.e., connected through local I/O ports).2. Network-attached storage (i.e., connected by some distributed system).
\end{itemize}

\note[item]{}
\end{frame}
\begin{frame}
\frametitle{Unrelated Title}


\begin{itemize}
\item 1. IDE2. SATA3. SCSI4. FC (fiber channel)
\end{itemize}

\note[item]{}
\end{frame}
\begin{frame}
\frametitle{Unrelated Title}


\begin{itemize}
\item remote procedure call
\end{itemize}

\note[item]{}
\end{frame}
\begin{frame}
\frametitle{Unrelated Title}


\begin{itemize}
\item disk-scheduling algorithm
\end{itemize}

\note[item]{}
\end{frame}
\begin{frame}
\frametitle{Unrelated Title}


\begin{itemize}
\item The FCFS algorithm is technically "fair", but it does not take into account the relative locations of the pending read and write operations on the disc; by ignoring this, the algorithm often leads to more disc latency than necessary for a given set of requests. 
\end{itemize}

\note[item]{}
\end{frame}
\begin{frame}
\frametitle{Unrelated Title}


\begin{itemize}
\item The SSTF algorithm does not prevent starvation of requests; a request that is far away from the current disc location may be delayed indefinitely by a stream of closeby requests.
\end{itemize}

\note[item]{}
\end{frame}
\begin{frame}
\frametitle{Unrelated Title}


\begin{itemize}
\item The elevator algorithm.
\end{itemize}

\note[item]{}
\end{frame}
\begin{frame}
\frametitle{Unrelated Title}


\begin{itemize}
\item The circular SCAN (C-SCAN) algorithm immediately jumps to the opposite end of the disc once it reaches one end, without scheduling a "return trip".
\end{itemize}

\note[item]{}
\end{frame}
\begin{frame}
\frametitle{Unrelated Title}


\begin{itemize}
\item On any given pass in either direction, a LOOK scheduler will only move the disk head to the further (most extreme) location in the request queue; once that final request is serviced, the scheduler immediately reverses the direction of the disc arm, repeating the process in the opposite direction.
\end{itemize}

\note[item]{}
\end{frame}
\begin{frame}
\frametitle{Unrelated Title}


\begin{itemize}
\item 1. FCFS (first-come, first-served)2. SSTF (shortest-seek-time-first)3. SCAN (i.e., "elevator")4. C-SCAN (circular SCAN)5. LOOK (i.e., "lazy elevator")6. C-LOOK (circular LOOK)
\end{itemize}

\note[item]{}
\end{frame}
\begin{frame}
\frametitle{Unrelated Title}


\begin{itemize}
\item SSTF and LOOK
\end{itemize}

\note[item]{}
\end{frame}
\begin{frame}
\frametitle{Unrelated Title}


\begin{itemize}
\item 1. The file-allocation method (i.e., operating on contiguous vs. linked or indexed files).2. The locations of the directories and index blocks on disc (these must be read in order to locate all parts of a file on disc).
\end{itemize}

\note[item]{}
\end{frame}
\begin{frame}
\frametitle{Unrelated Title}


\begin{itemize}
\item The operating system does not know the physical locations of logical blocks on disc; thus, the operating system can't schedule for improved rotational latency.
\end{itemize}

\note[item]{}
\end{frame}
\begin{frame}
\frametitle{Unrelated Title}


\begin{itemize}
\item The operating system may want to place certain constraints on order-of-service; for example, demand paging may take priority over application I/O, and writes are more urgent than reads if the cache is running out of free pages.
\end{itemize}

\note[item]{}
\end{frame}
\begin{frame}
\frametitle{Unrelated Title}


\begin{itemize}
\item physical formatting
\end{itemize}

\note[item]{}
\end{frame}
\begin{frame}
\frametitle{Unrelated Title}


\begin{itemize}
\item A single block consists of:1. A header segment2. A data segment (usually 512 bytes).3. A trailer segment.
\end{itemize}

\note[item]{}
\end{frame}
\begin{frame}
\frametitle{Unrelated Title}


\begin{itemize}
\item The header and trailer hold metadata—such as a sector number and an error-correcting code (ECC)—which can be used by the disk controller.
\end{itemize}

\note[item]{}
\end{frame}
\begin{frame}
\frametitle{Unrelated Title}


\begin{itemize}
\item The process fills the disk with a special data structure for each sector. The structure holds the sector's data as well as some metadata used by the disk controller. Logical blocks are mapped to these sectors on disk.
\end{itemize}

\note[item]{}
\end{frame}
\begin{frame}
\frametitle{Unrelated Title}


\begin{itemize}
\item A larger sector size means that fewer auxiliary (header and footer) segments need to be store on disk—allowing a larger percentage of the storage media to store data.
\end{itemize}

\note[item]{}
\end{frame}
\begin{frame}
\frametitle{Unrelated Title}


\begin{itemize}
\item A smaller sector size allows more sectors to fit on each track.
\end{itemize}

\note[item]{}
\end{frame}
\begin{frame}
\frametitle{Unrelated Title}


\begin{itemize}
\item Low-level formatting populates the disk with a data structure for each sector.Logical formatting uses these sectors (addressed via logical blocks) to create a new file system on the disk; initial file system data structures can include maps of free and allocated space (i.e., a FAT, or inodes) and an initial empty directory.
\end{itemize}

\note[item]{}
\end{frame}
\begin{frame}
\frametitle{Unrelated Title}


\begin{itemize}
\item clusters
\end{itemize}

\note[item]{}
\end{frame}
\begin{frame}
\frametitle{Unrelated Title}


\begin{itemize}
\item A cluster is a logical grouping of contiguous blocks on disk. A file system can use clustering to increase sequential access and reduce random access.
\end{itemize}

\note[item]{}
\end{frame}
\begin{frame}
\frametitle{Unrelated Title}


\begin{itemize}
\item By grouping contiguous blocks into clusters, the file system can promote sequential access and reduce random access during file system operations.
\end{itemize}

\note[item]{}
\end{frame}
\begin{frame}
\frametitle{Unrelated Title}


\begin{itemize}
\item A boot disk (or a system disk).
\end{itemize}

\note[item]{}
\end{frame}
\begin{frame}
\frametitle{Unrelated Title}


\begin{itemize}
\item boot blocks.
\end{itemize}

\note[item]{}
\end{frame}
\begin{frame}
\frametitle{Unrelated Title}


\begin{itemize}
\item The location on disc where Windows stores its initial bootstrap code.
\end{itemize}

\note[item]{}
\end{frame}
\begin{frame}
\frametitle{Unrelated Title}


\begin{itemize}
\item master boot record (MBR).
\end{itemize}

\note[item]{}
\end{frame}
\begin{frame}
\frametitle{Unrelated Title}


\begin{itemize}
\item boot partition
\end{itemize}

\note[item]{}
\end{frame}
\begin{frame}
\frametitle{Unrelated Title}


\begin{itemize}
\item A disk partition that holds the code for the operating system and device drivers.
\end{itemize}

\note[item]{}
\end{frame}
\begin{frame}
\frametitle{Unrelated Title}


\begin{itemize}
\item The boot partition.
\end{itemize}

\note[item]{}
\end{frame}
\begin{frame}
\frametitle{Unrelated Title}


\begin{itemize}
\item boot sector
\end{itemize}

\note[item]{}
\end{frame}
\begin{frame}
\frametitle{Unrelated Title}


\begin{itemize}
\item Following a soft error (e.g., a block read), the damanaged data can be restored using an error-correcting mechanism (i.e., ECC). Hard errors are irrecoverable, however, and result in data loss.
\end{itemize}

\note[item]{}
\end{frame}
\begin{frame}
\frametitle{Unrelated Title}


\begin{itemize}
\item The implementation will be relatively simple and quick, as the file system already supports basic operations needed to manage the swap space (create file, name it, allocate its space, etc).
\end{itemize}

\note[item]{}
\end{frame}
\begin{frame}
\frametitle{Unrelated Title}


\begin{itemize}
\item It's an inefficient strategy, as extra overhead is involved in using file system calls to manage a "raw" portion of the disk (as swap space).
\end{itemize}

\note[item]{}
\end{frame}
\begin{frame}
\frametitle{Unrelated Title}


\begin{itemize}
\item An operating system could use a separate swap space manager to allocate and manage a "raw" area of the disk. Using specialized algorithms, this approach can improve the speed of the operating system's page-swapping operations.
\end{itemize}

\note[item]{}
\end{frame}
\begin{frame}
\frametitle{Unrelated Title}


\begin{itemize}
\item The entire swap space is reinitialized on boot, so any resulting fragmentation will only last as long as the system is running.
\end{itemize}

\note[item]{}
\end{frame}
\begin{frame}
\frametitle{Unrelated Title}


\begin{itemize}
\item Linux divides each swap area into 4kB page slots used to hold swapped pages. For each swap area, Linux allocates a swap map. The swap map holds an integer value for each page in the swap area. A zero value indicates that the page slot is available to take a page. A positive value indicates that the slot is currently occupied—it also serves to indicate the number of processes currently sharing that page.
\end{itemize}

\note[item]{}
\end{frame}
\begin{frame}
\frametitle{Unrelated Title}


\begin{itemize}
\item "Redundant Array of Inexpensive Disks" (or, more recently, "Redundant Array of Indendent Disks")
\end{itemize}

\note[item]{}
\end{frame}
\begin{frame}
\frametitle{Unrelated Title}


\begin{itemize}
\item redundancy
\end{itemize}

\note[item]{}
\end{frame}
\begin{frame}
\frametitle{Unrelated Title}


\begin{itemize}
\item A storage reliability technique that involves duplicating each logical disk on two or more physical disks; every write to the logical disk is carried out to all physical disks.
\end{itemize}

\note[item]{}
\end{frame}
\begin{frame}
\frametitle{Unrelated Title}


\begin{itemize}
\item mean time to failure
\end{itemize}

\note[item]{}
\end{frame}
\begin{frame}
\frametitle{Unrelated Title}


\begin{itemize}
\item mean time to repair
\end{itemize}

\note[item]{}
\end{frame}
\begin{frame}
\frametitle{Unrelated Title}


\begin{itemize}
\item 1. Always write blocks to the second disk only after the write to the first completes.2. Integrate a non-volatile RAM (NVRAM) cache into the RAID array (assuming the cache supports error protection and/or correction, such as ECC or mirroring).
\end{itemize}

\note[item]{}
\end{frame}
\begin{frame}
\frametitle{Unrelated Title}


\begin{itemize}
\item Storing equal portions of some data set across several storage devices, such that a complete copy can be constructed by reading the portions stored on each device.
\end{itemize}

\note[item]{}
\end{frame}
\begin{frame}
\frametitle{Unrelated Title}


\begin{itemize}
\item Bit-level striping
\end{itemize}

\note[item]{}
\end{frame}
\begin{frame}
\frametitle{Unrelated Title}


\begin{itemize}
\item A variant of data striping in which the bits of each byte (of some region of data) are distributed across several disks (one disk for each bit of the byte).
\end{itemize}

\note[item]{}
\end{frame}
\begin{frame}
\frametitle{Unrelated Title}


\begin{itemize}
\item A data striping scheme in which the individual blocks of a file are striped across several disks.
\end{itemize}

\note[item]{}
\end{frame}
\begin{frame}
\frametitle{Unrelated Title}


\begin{itemize}
\item 1. Bit-level striping.2. Block-level striping.
\end{itemize}

\note[item]{}
\end{frame}
\begin{frame}
\frametitle{Unrelated Title}


\begin{itemize}
\item block-level striping
\end{itemize}

\note[item]{}
\end{frame}
\begin{frame}
\frametitle{Unrelated Title}


\begin{itemize}
\item parallelism
\end{itemize}

\note[item]{}
\end{frame}
\begin{frame}
\frametitle{Unrelated Title}


\begin{itemize}
\item 1. Increased throughput for (multiple) small accesses (achieved w/ load-balancing).2. Reduced response time for large accesses.
\end{itemize}

\note[item]{}
\end{frame}
\begin{frame}
\frametitle{Unrelated Title}


\begin{itemize}
\item RAID levels
\end{itemize}

\note[item]{}
\end{frame}
\begin{frame}
\frametitle{Unrelated Title}


\begin{itemize}
\item A parity bit is a separate bit that is allocated for each byte in a memory system.
\end{itemize}

\note[item]{}
\end{frame}
\begin{frame}
\frametitle{Unrelated Title}


\begin{itemize}
\item Parity bits can be used to detect single-bit errors in a given byte that is stored. The memory system can read the entire byte and check it against the parity bit—if one does not reflect the other, then an error has occurred.
\end{itemize}

\note[item]{}
\end{frame}
\begin{frame}
\frametitle{Unrelated Title}


\begin{itemize}
\item The parity bit is meant to indicate the number of bits in the associated byte that are set. An unset parity bit indicates an even number of set bits; otherwise, it indicates an odd number of set bits.
\end{itemize}

\note[item]{}
\end{frame}
\begin{frame}
\frametitle{Unrelated Title}


\begin{itemize}
\item No. It can detect single-bit errors, but more error-correcting bits must be used in order for the memory system to determine the position of the erroneous bit.
\end{itemize}

\note[item]{}
\end{frame}
\begin{frame}
\frametitle{Unrelated Title}


\begin{itemize}
\item This organizational scheme involves striping the bits of each byte across several storage disks and using a final disk to store the parity of that byte. If, during an operation, a given disk detects a bad sector read, the remaining bits and be used in conjunction with the parity bit to determine the correct value of the misread bit.
\end{itemize}

\note[item]{}
\end{frame}
\begin{frame}
\frametitle{Unrelated Title}


\begin{itemize}
\item The operating system scheduling write operations for a block storage device must perform writes at the block level; the entire block must be read into main memory, modified, and then written back out to the device.
\end{itemize}

\note[item]{}
\end{frame}
\begin{frame}
\frametitle{Unrelated Title}


\begin{itemize}
\item The cycle conducted by an operating system in order to change data on a block storage device. The targeted block must be read into main memory, modified, and then written back to the device (block-level granularity).
\end{itemize}

\note[item]{}
\end{frame}
\begin{frame}
\frametitle{Unrelated Title}


\begin{itemize}
\item If an operation writes to a single data block, the corresponding parity block must also be updated (written) by the storage system.
\end{itemize}

\note[item]{}
\end{frame}
\begin{frame}
\frametitle{Unrelated Title}


\begin{itemize}
\item A storage scheme in which some number of data blocks are striped to an equal number of disks. A final disk is used to maintain associated parity blocks, which the system can use to detect and fix bad sector reads on any of the other disks.
\end{itemize}

\note[item]{}
\end{frame}
\begin{frame}
\frametitle{Unrelated Title}


\begin{itemize}
\item In the distributed parity scheme, data and parity information are spread across all of the disks. For each logical block, N disks stores its data while one stores its parity. The disk that is responsible for storing the parity of a given logical block can be determined by that block's logical index.
\end{itemize}

\note[item]{}
\end{frame}
\begin{frame}
\frametitle{Unrelated Title}


\begin{itemize}
\item error-correcting
\end{itemize}

\note[item]{}
\end{frame}
\begin{frame}
\frametitle{Unrelated Title}


\begin{itemize}
\item Block-level striping of data is done across some number of disks (e.g., 4 disks). No mechanisms are put in place for redundancy or error correction.
\end{itemize}

\note[item]{}
\end{frame}
\begin{frame}
\frametitle{Unrelated Title}


\begin{itemize}
\item Level 1 introduces redundancy by mirroring each striping disk with its own copy.
\end{itemize}

\note[item]{}
\end{frame}
\begin{frame}
\frametitle{Unrelated Title}


\begin{itemize}
\item Level 1
\end{itemize}

\note[item]{}
\end{frame}
\begin{frame}
\frametitle{Unrelated Title}


\begin{itemize}
\item Level 2 replaces the disk mirroring with disks that store error-correcting party bits. Level 2 reduces storage costs, as 4 disks of data can be made reliable using 3 error-correcting parity disks (instead of the 4 disks required for mirroring).
\end{itemize}

\note[item]{}
\end{frame}
\begin{frame}
\frametitle{Unrelated Title}


\begin{itemize}
\item Level 3 makes use of the fact that, for a given disk, the disk controller can detect a bad sector read. Level 3 organizes disks according to a bit-interleaving parity scheme: one bit from each byte is written to an associated disk, while an additional disk is used to store the parity bit for that byte (a total of N+1 disks).If a bad sector read occurs, we can use the other disk's bits together with the parity bit to determine the correct values of the misread bit.
\end{itemize}

\note[item]{}
\end{frame}
\begin{frame}
\frametitle{Unrelated Title}


\begin{itemize}
\item With Level 3, a single byte of data is distributed across N disks. A read or write can activate all disks in parallel, performing the operation in 1/N-th the time required by Levels 0 and 1.
\end{itemize}

\note[item]{}
\end{frame}
\begin{frame}
\frametitle{Unrelated Title}


\begin{itemize}
\item Level 3 (as well as Level 2) requires us to compute and update the parity information of each byte written. This overhead can result in significantly slower writes.
\end{itemize}

\note[item]{}
\end{frame}
\begin{frame}
\frametitle{Unrelated Title}


\begin{itemize}
\item RAID Level 3 uses bit-level striping of data while Level 4 uses block-level striping. In a Level 3 system, the bits of each bytes are striped across N disks; in a Level 4 system, the blocks of each file are striped across N disks. Both schemes use an extra disk to store parity information.
\end{itemize}

\note[item]{}
\end{frame}
\begin{frame}
\frametitle{Unrelated Title}


\begin{itemize}
\item In a Level 3 system, the bits of each byte can be read in parallel from multiple disks, providing a high data-transfer rate. In a Level 4 system, a block read only requires access to 1 disk, allowing multiple block reads to run in parallel.This results in a higher overall I/O rate with Level 4 for larger reads; since the parity blocks are also written in parallel, larger writes also have high I/O rates.
\end{itemize}

\note[item]{}
\end{frame}
\begin{frame}
\frametitle{Unrelated Title}


\begin{itemize}
\item Level 5 distributes parity information across all disks; instead of allocating 1 disk to store all parity blocks, Level 5 stripes the parity blocks across the same disks that are storing data.
\end{itemize}

\note[item]{}
\end{frame}
\begin{frame}
\frametitle{Unrelated Title}


\begin{itemize}
\item Level 6 replaces parity information (i.e., parity blocks) with bit-fields capable of detecting and correcting errors in the event that 2 disks both fail. These bit-fields are often referred to as error-correcting codes. These codes require more bits to be allocated than just a single parity bit.
\end{itemize}

\note[item]{}
\end{frame}
\begin{frame}
\frametitle{Unrelated Title}


\begin{itemize}
\item RAID Level 0 + 1
\end{itemize}

\note[item]{}
\end{frame}
\begin{frame}
\frametitle{Unrelated Title}


\begin{itemize}
\item 1. Kernel (software)2. Host bus-adapter (HBA) hardware3. Storage array hardware4. A SAN interconnect layer (by disk virtualization devices)
\end{itemize}

\note[item]{}
\end{frame}
\begin{frame}
\frametitle{Unrelated Title}


\begin{itemize}
\item The automatic duplication of writes between separate storage sites, for redundancy and disaster recovery.
\end{itemize}

\note[item]{}
\end{frame}
\begin{frame}
\frametitle{Unrelated Title}


\begin{itemize}
\item replication
\end{itemize}

\note[item]{}
\end{frame}
\begin{frame}
\frametitle{Unrelated Title}


\begin{itemize}
\item hot spare (disk)
\end{itemize}

\note[item]{}
\end{frame}
\begin{frame}
\frametitle{Unrelated Title}


\begin{itemize}
\item A disk that is designed to be used as a backup (in the event that another disk fails).
\end{itemize}

\note[item]{}
\end{frame}
\begin{frame}
\frametitle{Unrelated Title}


\begin{itemize}
\item It must never lose data, in any event.
\end{itemize}

\note[item]{}
\end{frame}
\begin{frame}
\frametitle{Unrelated Title}


\begin{itemize}
\item replication (across multiple disks)
\end{itemize}

\note[item]{}
\end{frame}
\begin{frame}
\frametitle{Unrelated Title}


\begin{itemize}
\item 1. Successful completion (correct write).2. Partial failure (some sectors written, and perhaps corrupted).3. Total failure (original data is intact).
\end{itemize}

\note[item]{}
\end{frame}
\begin{frame}
\frametitle{Unrelated Title}


\begin{itemize}
\item 1. Write the data to the first physical disk.2. Write the same data to the second physical disk, following the first write.3. After the second write completes (if it completes), declare the operation a success.
\end{itemize}

\note[item]{}
\end{frame}
\begin{frame}
\frametitle{Unrelated Title}


\begin{itemize}
\item an NVRAM cache
\end{itemize}

\note[item]{}
\end{frame}
\begin{frame}
\frametitle{Unrelated Title}


\begin{itemize}
\item Provide the cache with a battery for backup power.
\end{itemize}

\note[item]{}
\end{frame}
\begin{frame}
\frametitle{Unrelated Title}


\begin{itemize}
\item When shining a laser at a magnetized spot (i.e., on a magnetic platter), the orientation of the spot's magnetic field will determine the polarization (clockwise or counter-clockwise) of the reflected laser beam.
\end{itemize}

\note[item]{}
\end{frame}
\begin{frame}
\frametitle{Unrelated Title}


\begin{itemize}
\item A magneto-optic disk.
\end{itemize}

\note[item]{}
\end{frame}
\begin{frame}
\frametitle{Unrelated Title}


\begin{itemize}
\item The read-write head of a magneto-optic disk is positioning further away from the surface of the platter, making it difficult for the head to detect the individal magnetic orientation of a single written bit. Thus, a laser beam is used to detect bit values.
\end{itemize}

\note[item]{}
\end{frame}
\begin{frame}
\frametitle{Unrelated Title}


\begin{itemize}
\item A magneto-optic disk's magnetic platter has a protective coated, while a hard drive's does not.
\end{itemize}

\note[item]{}
\end{frame}
\begin{frame}
\frametitle{Unrelated Title}


\begin{itemize}
\item laser beams
\end{itemize}

\note[item]{}
\end{frame}
\begin{frame}
\frametitle{Unrelated Title}


\begin{itemize}
\item 1. CD-RW drives2. DVD-RW drives
\end{itemize}

\note[item]{}
\end{frame}
\begin{frame}
\frametitle{Unrelated Title}


\begin{itemize}
\item Because a tape mechanism uses sequential seeks (i.e., fast-forward and rewind) to move from one location to another. A disk can seek to a new random location much more quickly.
\end{itemize}

\note[item]{}
\end{frame}
\begin{frame}
\frametitle{Unrelated Title}


\begin{itemize}
\item backup copies of disk data
\end{itemize}

\note[item]{}
\end{frame}
\begin{frame}
\frametitle{Unrelated Title}


\begin{itemize}
\item 1. File systems2. Raw disk (i.e., array of blocks)
\end{itemize}

\note[item]{}
\end{frame}
\begin{frame}
\frametitle{Unrelated Title}


\begin{itemize}
\item seek()read()write()
\end{itemize}

\note[item]{}
\end{frame}
\begin{frame}
\frametitle{Unrelated Title}


\begin{itemize}
\item Because locate() moves the tape head to a specific logical block, instead of only to a specific cylinder and/or track (which holds many blocks).
\end{itemize}

\note[item]{}
\end{frame}
\begin{frame}
\frametitle{Unrelated Title}


\begin{itemize}
\item The average data rate during a large data transfer.
\end{itemize}

\note[item]{}
\end{frame}
\begin{frame}
\frametitle{Unrelated Title}


\begin{itemize}
\item The average data rate over the entire I/O operation, including the time to seek() or locate().
\end{itemize}

\note[item]{}
\end{frame}
\begin{frame}
\frametitle{Unrelated Title}


\begin{itemize}
\item Because the removable disk may be exposed to dust, changes in temperature and humidity, mechanical shock, etc.
\end{itemize}

\note[item]{}
\end{frame}
\begin{frame}
\frametitle{Unrelated Title}


\begin{itemize}
\item An optical disk.
\end{itemize}

\note[item]{}
\end{frame}
\begin{frame}
\frametitle{Unrelated Title}


\begin{itemize}
\item Because the optical disk is given a plastic or glass coating that protects against head crashes.
\end{itemize}

\note[item]{}
\end{frame}
\begin{frame}
\frametitle{Unrelated Title}


\begin{itemize}
\item I/O subsystem
\end{itemize}

\note[item]{}
\end{frame}
\begin{frame}
\frametitle{Unrelated Title}


\begin{itemize}
\item A program associated with an I/O device that presents (or implements) a standardized device-access interface to the operating system kernel's I/O subsystem.
\end{itemize}

\note[item]{}
\end{frame}
\begin{frame}
\frametitle{Unrelated Title}


\begin{itemize}
\item A logical connection (or communication) point between two components in a computer system.
\end{itemize}

\note[item]{}
\end{frame}
\begin{frame}
\frametitle{Unrelated Title}


\begin{itemize}
\item A bus
\end{itemize}

\note[item]{}
\end{frame}
\begin{frame}
\frametitle{Unrelated Title}


\begin{itemize}
\item A bus
\end{itemize}

\note[item]{}
\end{frame}
\begin{frame}
\frametitle{Unrelated Title}


\begin{itemize}
\item A common set of wires used by hardware components to communicate with each other. Components must use an established, shared protocol to communicate over these wires.
\end{itemize}

\note[item]{}
\end{frame}
\begin{frame}
\frametitle{Unrelated Title}


\begin{itemize}
\item port (of communication)
\end{itemize}

\note[item]{}
\end{frame}
\begin{frame}
\frametitle{Unrelated Title}


\begin{itemize}
\item The PCI bus
\end{itemize}

\note[item]{}
\end{frame}
\begin{frame}
\frametitle{Unrelated Title}


\begin{itemize}
\item expansion buses
\end{itemize}

\note[item]{}
\end{frame}
\begin{frame}
\frametitle{Unrelated Title}


\begin{itemize}
\item A piece of hardware that can operate a port, a bus, or a device.
\end{itemize}

\note[item]{}
\end{frame}
\begin{frame}
\frametitle{Unrelated Title}


\begin{itemize}
\item controller
\end{itemize}

\note[item]{}
\end{frame}
\begin{frame}
\frametitle{Unrelated Title}


\begin{itemize}
\item A category of hardware controller that is implemented as a separate circuit board that plugs into the computer through a bus.
\end{itemize}

\note[item]{}
\end{frame}
\begin{frame}
\frametitle{Unrelated Title}


\begin{itemize}
\item The disk controller
\end{itemize}

\note[item]{}
\end{frame}
\begin{frame}
\frametitle{Unrelated Title}


\begin{itemize}
\item data and control (signals)
\end{itemize}

\note[item]{}
\end{frame}
\begin{frame}
\frametitle{Unrelated Title}


\begin{itemize}
\item Memory-mapped I/O
\end{itemize}

\note[item]{}
\end{frame}
\begin{frame}
\frametitle{Unrelated Title}


\begin{itemize}
\item 1. A status register.2. A control register.3. A data-in register.3. A data-out register.
\end{itemize}

\note[item]{}
\end{frame}
\begin{frame}
\frametitle{Unrelated Title}


\begin{itemize}
\item A producer-consumer relationship
\end{itemize}

\note[item]{}
\end{frame}
\begin{frame}
\frametitle{Unrelated Title}


\begin{itemize}
\item A busy bit (set and cleared in a status register).
\end{itemize}

\note[item]{}
\end{frame}
\begin{frame}
\frametitle{Unrelated Title}


\begin{itemize}
\item The command-ready bit (stored in a command register).
\end{itemize}

\note[item]{}
\end{frame}
\begin{frame}
\frametitle{Unrelated Title}


\begin{itemize}
\item 1. The host reads the busy bit until it is cleared.2. The host places a byte of data into the data-out register. It also sets the write bit in the command register and sets the command-ready bit.3. The device controller notices the command-ready bit is set, and it sets the busy bit.4. The device controller sees the write command in the command register; it reads the byte of data in the data-out register and writes it to the device's storage.5. After the I/O is performed, the device controller clears the command-ready bit, clears the busy bit, and clears the error bit (indicating that the I/O succeeded).
\end{itemize}

\note[item]{}
\end{frame}
\begin{frame}
\frametitle{Unrelated Title}


\begin{itemize}
\item polling
\end{itemize}

\note[item]{}
\end{frame}
\begin{frame}
\frametitle{Unrelated Title}


\begin{itemize}
\item busy-waiting
\end{itemize}

\note[item]{}
\end{frame}
\begin{frame}
\frametitle{Unrelated Title}


\begin{itemize}
\item By executing a logical AND operation (using a register mask).
\end{itemize}

\note[item]{}
\end{frame}
\begin{frame}
\frametitle{Unrelated Title}


\begin{itemize}
\item 1. Read the device register into a CPU register.2. Logical-AND to extract the status bit into another CPU register.3. Branch (BR) if not zero.4. Repeat.
\end{itemize}

\note[item]{}
\end{frame}
\begin{frame}
\frametitle{Unrelated Title}


\begin{itemize}
\item 1. Polling (or busy-waiting) the device register.2. Having the device issue interrupts to the CPU.
\end{itemize}

\note[item]{}
\end{frame}
\begin{frame}
\frametitle{Unrelated Title}


\begin{itemize}
\item The interrupt request line.
\end{itemize}

\note[item]{}
\end{frame}
\begin{frame}
\frametitle{Unrelated Title}


\begin{itemize}
\item An electric wire used to carry interrupt signals to the CPU.
\end{itemize}

\note[item]{}
\end{frame}
\begin{frame}
\frametitle{Unrelated Title}


\begin{itemize}
\item After every executed instruction.
\end{itemize}

\note[item]{}
\end{frame}
\begin{frame}
\frametitle{Unrelated Title}


\begin{itemize}
\item Raises and interrupt by asserting a signal.
\end{itemize}

\note[item]{}
\end{frame}
\begin{frame}
\frametitle{Unrelated Title}


\begin{itemize}
\item Catches an interrupt and dispatches it.
\end{itemize}

\note[item]{}
\end{frame}
\begin{frame}
\frametitle{Unrelated Title}


\begin{itemize}
\item Clear the interrupt
\end{itemize}

\note[item]{}
\end{frame}
\begin{frame}
\frametitle{Unrelated Title}


\begin{itemize}
\item 1. It performs a save-state of the currently running process.2. It begins executing thethe interrupt handler code (stored at a fixed address in memory).
\end{itemize}

\note[item]{}
\end{frame}
\begin{frame}
\frametitle{Unrelated Title}


\begin{itemize}
\item An interrupt controller.
\end{itemize}

\note[item]{}
\end{frame}
\begin{frame}
\frametitle{Unrelated Title}


\begin{itemize}
\item So that the CPU can disable (or mask) a certain set of (low priority) interrupts while still checking for non-maskable (high priority) while executing critical sections. If the CPU had only one interrupt request line, it could not mask out a subset of signals.
\end{itemize}

\note[item]{}
\end{frame}
\begin{frame}
\frametitle{Unrelated Title}


\begin{itemize}
\item A table, normally stored in low memory, which contains the memory addresses of every interrupt handler loaded by the operating system. Thus, each entry in the table points to the location of an associated block of code in memory.
\end{itemize}

\note[item]{}
\end{frame}
\begin{frame}
\frametitle{Unrelated Title}


\begin{itemize}
\item The CPU normally has two separate interrupt request lines, allowing a high-priority signal to still reach the CPU while the low-priority handler is running.
\end{itemize}

\note[item]{}
\end{frame}
\begin{frame}
\frametitle{Unrelated Title}


\begin{itemize}
\item 32 (addresses 0 through 31)
\end{itemize}

\note[item]{}
\end{frame}
\begin{frame}
\frametitle{Unrelated Title}


\begin{itemize}
\item 32-255
\end{itemize}

\note[item]{}
\end{frame}
\begin{frame}
\frametitle{Unrelated Title}


\begin{itemize}
\item 8 bits
\end{itemize}

\note[item]{}
\end{frame}
\begin{frame}
\frametitle{Unrelated Title}


\begin{itemize}
\item The Pentium uses an 8-bit address space, allowing for 256 unique addresses.
\end{itemize}

\note[item]{}
\end{frame}
\begin{frame}
\frametitle{Unrelated Title}


\begin{itemize}
\item When the operating system is loaded, it probes the system's hardware buses to determine which devices are connected. Based on this information, it loads the corresponding interrupt handlers into the interrupt vector.
\end{itemize}

\note[item]{}
\end{frame}
\begin{frame}
\frametitle{Unrelated Title}


\begin{itemize}
\item 1. Divide-by-zero.2. Accessing a protected (or invalid) memory address.3. Accessing an address that is not memory-resident (i.e., page fault).4. Attempting to execute a privileged instruction.
\end{itemize}

\note[item]{}
\end{frame}
\begin{frame}
\frametitle{Unrelated Title}


\begin{itemize}
\item 1. Switching the processor mode.2. Input-output control.3. Timer management.4. Interrupt management.
\end{itemize}

\note[item]{}
\end{frame}
\begin{frame}
\frametitle{Unrelated Title}


\begin{itemize}
\item A trap
\end{itemize}

\note[item]{}
\end{frame}
\begin{frame}
\frametitle{Unrelated Title}


\begin{itemize}
\item An interrupt that is caused by software.
\end{itemize}

\note[item]{}
\end{frame}
\begin{frame}
\frametitle{Unrelated Title}


\begin{itemize}
\item 1. The high-priority interrupt handler can record the I/O status, clear the device interrupt, initial the pending I/O request, and then raise a low-priority interrupt to finish the work.2. The low-priority interrupt can copy the read data out of the kernel buffer into the application (user) space, and then call the scheduler to place the associated task back onto the ready queue.
\end{itemize}

\note[item]{}
\end{frame}
\begin{frame}
\frametitle{Unrelated Title}


\begin{itemize}
\item Programmed I/O (PIO)
\end{itemize}

\note[item]{}
\end{frame}
\begin{frame}
\frametitle{Unrelated Title}


\begin{itemize}
\item An I/O scheme in which data is read or written one-byte-at-a-time, using data-in and data-out registers.
\end{itemize}

\note[item]{}
\end{frame}
\begin{frame}
\frametitle{Unrelated Title}


\begin{itemize}
\item A hardware feature that allows device controllers and other components to issue reads and writes to the memory controller without direct reliance on the CPU.
\end{itemize}

\note[item]{}
\end{frame}
\begin{frame}
\frametitle{Unrelated Title}


\begin{itemize}
\item The DMA controller
\end{itemize}

\note[item]{}
\end{frame}
\begin{frame}
\frametitle{Unrelated Title}


\begin{itemize}
\item A special purpose processor that can operate the memory bus directly, placing addresses on the bus to perform data transfers.
\end{itemize}

\note[item]{}
\end{frame}
\begin{frame}
\frametitle{Unrelated Title}


\begin{itemize}
\item The CPU writes command information into a DMA control block in memory, and then passes the address of the block to the DMS controller via a device register. The DMA controller then performs the requested memory operation independently. When the operation completes, the DMA controller raises an interrupt to the CPU to signal that it's done.
\end{itemize}

\note[item]{}
\end{frame}
\begin{frame}
\frametitle{Unrelated Title}


\begin{itemize}
\item A data structure written by the CPU that describes a desired memory operation.
\end{itemize}

\note[item]{}
\end{frame}
\begin{frame}
\frametitle{Unrelated Title}


\begin{itemize}
\item 1. A source pointer (e.g., from a kernel-space buffer).2. A destination pointer (e.g., to a user-space buffer.3. A byte count.
\end{itemize}

\note[item]{}
\end{frame}
\begin{frame}
\frametitle{Unrelated Title}


\begin{itemize}
\item bus-mastering
\end{itemize}

\note[item]{}
\end{frame}
\begin{frame}
\frametitle{Unrelated Title}


\begin{itemize}
\item No. Only one component may use the memory bus at a time.
\end{itemize}

\note[item]{}
\end{frame}
\begin{frame}
\frametitle{Unrelated Title}


\begin{itemize}
\item The CPU still has access to primary and secondary cache memory.
\end{itemize}

\note[item]{}
\end{frame}
\begin{frame}
\frametitle{Unrelated Title}


\begin{itemize}
\item Two wires—DMA-request and DMA-acknowledge—are used to coordinate requests.
\end{itemize}

\note[item]{}
\end{frame}
\begin{frame}
\frametitle{Unrelated Title}


\begin{itemize}
\item 1. A device controller places a signal on the DMA-request line when a byte of data is ready to be transferred into memory.2. The DMA controller then takes control of the memory bus, placing the intended address on the memory-address line and placing a signal on DMA-acknowledge.3. The device controller then transfers the byte of data to memory and clears the DMA-request line.
\end{itemize}

\note[item]{}
\end{frame}
\begin{frame}
\frametitle{Unrelated Title}


\begin{itemize}
\item A standardized set of I/O functions that an I/O device may support.
\end{itemize}

\note[item]{}
\end{frame}
\begin{frame}
\frametitle{Unrelated Title}


\begin{itemize}
\item Kernel module
\end{itemize}

\note[item]{}
\end{frame}
\begin{frame}
\frametitle{Unrelated Title}


\begin{itemize}
\item I/O interface
\end{itemize}

\note[item]{}
\end{frame}
\begin{frame}
\frametitle{Unrelated Title}


\begin{itemize}
\item 1. Block I/O.2. Character-stream I/O.3. Memory-mapped file access.4. Network sockets.
\end{itemize}

\note[item]{}
\end{frame}
\begin{frame}
\frametitle{Unrelated Title}


\begin{itemize}
\item The "I/O control" system call allows processes to use or interact with custom (non-standard) functionality supported by a connected I/O device.
\end{itemize}

\note[item]{}
\end{frame}
\begin{frame}
\frametitle{Unrelated Title}


\begin{itemize}
\item 1. A file descriptor (connecting the calling task to the driver by referencing a specific device managed by that driver).2. An integer specifying the command supported by the device driver.3. A pointer to some arbitrary data structure (can provide control and/or data).
\end{itemize}

\note[item]{}
\end{frame}
\begin{frame}
\frametitle{Unrelated Title}


\begin{itemize}
\item A file-system interface.
\end{itemize}

\note[item]{}
\end{frame}
\begin{frame}
\frametitle{Unrelated Title}


\begin{itemize}
\item Raw I/O
\end{itemize}

\note[item]{}
\end{frame}
\begin{frame}
\frametitle{Unrelated Title}


\begin{itemize}
\item Block I/O device driver.
\end{itemize}

\note[item]{}
\end{frame}
\begin{frame}
\frametitle{Unrelated Title}


\begin{itemize}
\item Processes read and write data to and from the device at a file level. A process can request a new or existing file to be mapped into its virtual memory address space, allowing the program to manipulate the file data in memory. Sections of the in-memory file buffer (i.e., file copy) can be written back to secondary storage as-needed.
\end{itemize}

\note[item]{}
\end{frame}
\begin{frame}
\frametitle{Unrelated Title}


\begin{itemize}
\item Demand paging
\end{itemize}

\note[item]{}
\end{frame}
\begin{frame}
\frametitle{Unrelated Title}


\begin{itemize}
\item It could use the interface to manage a swap space (or swap file) on disk.
\end{itemize}

\note[item]{}
\end{frame}
\begin{frame}
\frametitle{Unrelated Title}


\begin{itemize}
\item 1. get(): Moves the device's next available character into memory.2. put(): Moves a given character from memory over to the device.
\end{itemize}

\note[item]{}
\end{frame}
\begin{frame}
\frametitle{Unrelated Title}


\begin{itemize}
\item 1. Keyboards.2. Mice.3. Network modems.4. Printers.5. Audio boards.
\end{itemize}

\note[item]{}
\end{frame}
\begin{frame}
\frametitle{Unrelated Title}


\begin{itemize}
\item 1. Create a new local socket.2. Connect a local socket to a remote address (i.e., another process).3. Listen for remote processes waiting to connect to a local socket.4. Send and receive messages over the socket connection.
\end{itemize}

\note[item]{}
\end{frame}
\begin{frame}
\frametitle{Unrelated Title}


\begin{itemize}
\item 1. Get the current time.2. Get the elapsed time.3. Register a timer to execute X at time T.
\end{itemize}

\note[item]{}
\end{frame}
\begin{frame}
\frametitle{Unrelated Title}


\begin{itemize}
\item 1. The scheduler uses a timer to preempt a process at the end of its time slice.2. The disk I/O subsystem can use a timer to flush any dirty cache buffers.3. The network subsystem can use timers to cancel operations that take too long.
\end{itemize}

\note[item]{}
\end{frame}
\begin{frame}
\frametitle{Unrelated Title}


\begin{itemize}
\item The process is moved from the ready queue (run queue) onto the wait queue. It is moved back to the ready queue by the operating system once the system call returns.
\end{itemize}

\note[item]{}
\end{frame}
\begin{frame}
\frametitle{Unrelated Title}


\begin{itemize}
\item The process receives status information from the operating system immediately—the data may or may not contain actionable information.
\end{itemize}

\note[item]{}
\end{frame}
\begin{frame}
\frametitle{Unrelated Title}


\begin{itemize}
\item Immediately.
\end{itemize}

\note[item]{}
\end{frame}
\begin{frame}
\frametitle{Unrelated Title}


\begin{itemize}
\item For asynchronous system calls, the caller provides a callback function invoked by the operating system when a certain asynchronous event has occurred. Both non-blocking and asynchronous calls return to the process immediatey.
\end{itemize}

\note[item]{}
\end{frame}
\begin{frame}
\frametitle{Unrelated Title}


\begin{itemize}
\item wait queue
\end{itemize}

\note[item]{}
\end{frame}
\begin{frame}
\frametitle{Unrelated Title}


\begin{itemize}
\item A vector, maintained by the operating system, with an entry for each device connected to the system. Each entry stores information about the device and any pending I/O requests associated with it.
\end{itemize}

\note[item]{}
\end{frame}
\begin{frame}
\frametitle{Unrelated Title}


\begin{itemize}
\item 1. The device's type.2. The device's address.3. The device's current state (idle, busy, etc).4. The device's set of pending I/O requests (i.e., its wait queue).
\end{itemize}

\note[item]{}
\end{frame}
\begin{frame}
\frametitle{Unrelated Title}


\begin{itemize}
\item The device-status table.
\end{itemize}

\note[item]{}
\end{frame}
\begin{frame}
\frametitle{Unrelated Title}


\begin{itemize}
\item A memory area that stores data while they are transferred between two devices, or between a device and an application.
\end{itemize}

\note[item]{}
\end{frame}
\begin{frame}
\frametitle{Unrelated Title}


\begin{itemize}
\item • A cache is always backed by an original copy somewhere (i.e., in main memory)• A buffer may hold the only existing copy of some data (until the buffer's memory is copied to some other location).
\end{itemize}

\note[item]{}
\end{frame}
\begin{frame}
\frametitle{Unrelated Title}


\begin{itemize}
\item 1. To accommodate a speed mismatch between a producer and a consumer (e.g., receiving a file in packets from a network and writing the file to disk).2. To adapt between devices that have different data-transfer sizes (e.g., a packet reassembly buffer on a receiving host's end).3. To support copy semantics for application I/O (e.g., to prevent subsequent unintended modification to the data that is still waiting to be copied by the kernel).
\end{itemize}

\note[item]{}
\end{frame}
\begin{frame}
\frametitle{Unrelated Title}


\begin{itemize}
\item Double buffering facilitates a producer-consumer relationship by allocating one buffer to be filled with (i.e., written) new data (by the producer) while another buffer is used (i.e., read) by the consumer. Using two distinct buffers prevents one participant from interfering with the actions of the other.
\end{itemize}

\note[item]{}
\end{frame}
\begin{frame}
\frametitle{Unrelated Title}


\begin{itemize}
\item Because double buffering decouples the producer from the consumer, relaxing the timing requirements between participants.
\end{itemize}

\note[item]{}
\end{frame}
\begin{frame}
\frametitle{Unrelated Title}


\begin{itemize}
\item The kernel could copy the region of user-space memory into a kernel-space buffer before returning from the system call. The data in the kernel buffer is what would be copied (presumably later) into a target device buffer (i.e., I/O request buffer).
\end{itemize}

\note[item]{}
\end{frame}
\begin{frame}
\frametitle{Unrelated Title}


\begin{itemize}
\item When the operating system reads a block of data off a disk, the data is placed into a buffer in memory (e.g., memory-mapped file). Assuming the buffer has not been modified by a task, the operating system could use this buffer as an in-memory cache whenever some other task asks to read the same disk block.
\end{itemize}

\note[item]{}
\end{frame}
\begin{frame}
\frametitle{Unrelated Title}


\begin{itemize}
\item A buffer that holds output for a device that cannot accept interleaved data streams
\end{itemize}

\note[item]{}
\end{frame}
\begin{frame}
\frametitle{Unrelated Title}


\begin{itemize}
\item Simultaneous Peripheral Operation Online
\end{itemize}

\note[item]{}
\end{frame}
\begin{frame}
\frametitle{Unrelated Title}


\begin{itemize}
\item 1. Spooling2. Explicit device allocation (and deallocation) to a process (mutual exclusion).
\end{itemize}

\note[item]{}
\end{frame}
\begin{frame}
\frametitle{Unrelated Title}


\begin{itemize}
\item privileged instructions
\end{itemize}

\note[item]{}
\end{frame}
\begin{frame}
\frametitle{Unrelated Title}


\begin{itemize}
\item message passing
\end{itemize}

\note[item]{}
\end{frame}
\begin{frame}
\frametitle{Unrelated Title}


\begin{itemize}
\item kernel data structures
\end{itemize}

\note[item]{}
\end{frame}
\begin{frame}
\frametitle{Unrelated Title}


\begin{itemize}
\item object-oriented methods
\end{itemize}

\note[item]{}
\end{frame}
\begin{frame}
\frametitle{Unrelated Title}


\begin{itemize}
\item The drive prefix is mapped to a specific I/O port address via a device table. The "c:" prefix value is hard-coded to point to the primary hard disk.
\end{itemize}

\note[item]{}
\end{frame}
\begin{frame}
\frametitle{Unrelated Title}


\begin{itemize}
\item 1. The system uses the first component of the file path (i.e., the path prefix) to lookup into the mount table, retrieving a device name.2. The device name is used to search through the associated file-system directory structure for an identifier of the form <major, minor>. The major value identifies the device driver, while the minor value is an index into the driver's device table.3. The system passes the minor identifier to the device driver, which performs the device table lookup to retrieve the port address (or memory-mapped address) of the appropriate device controller.
\end{itemize}

\note[item]{}
\end{frame}
\begin{frame}
\frametitle{Unrelated Title}


\begin{itemize}
\item A feature of UNIX System V that allows programmers to cleanly assemble pipelines of device driver code to facilitate interaction between a user process and a device.
\end{itemize}

\note[item]{}
\end{frame}
\begin{frame}
\frametitle{Unrelated Title}


\begin{itemize}
\item A full-duplex connection
\end{itemize}

\note[item]{}
\end{frame}
\begin{frame}
\frametitle{Unrelated Title}


\begin{itemize}
\item 1. Stream head: Interfaces with the user-space process.2. Driver end: Interfaces with the I/O device.3. Stream module: One or more modules sitting between the head and the end.
\end{itemize}

\note[item]{}
\end{frame}
\begin{frame}
\frametitle{Unrelated Title}


\begin{itemize}
\item The user-space process.
\end{itemize}

\note[item]{}
\end{frame}
\begin{frame}
\frametitle{Unrelated Title}


\begin{itemize}
\item The I/O device.
\end{itemize}

\note[item]{}
\end{frame}
\begin{frame}
\frametitle{Unrelated Title}


\begin{itemize}
\item Stream modules
\end{itemize}

\note[item]{}
\end{frame}
\begin{frame}
\frametitle{Unrelated Title}


\begin{itemize}
\item A read queue and a write queue.
\end{itemize}

\note[item]{}
\end{frame}
\begin{frame}
\frametitle{Unrelated Title}


\begin{itemize}
\item Read queues and write queues
\end{itemize}

\note[item]{}
\end{frame}
\begin{frame}
\frametitle{Unrelated Title}


\begin{itemize}
\item The ioctl() system call.
\end{itemize}

\note[item]{}
\end{frame}
\begin{frame}
\frametitle{Unrelated Title}


\begin{itemize}
\item This system call can be used to add a new module to an existing stream.
\end{itemize}

\note[item]{}
\end{frame}
\begin{frame}
\frametitle{Unrelated Title}


\begin{itemize}
\item Flow control
\end{itemize}

\note[item]{}
\end{frame}
\begin{frame}
\frametitle{Unrelated Title}


\begin{itemize}
\item Without flow control, a module may send too many messages to the next module's queues, which may have limited space.
\end{itemize}

\note[item]{}
\end{frame}
\begin{frame}
\frametitle{Unrelated Title}


\begin{itemize}
\item 1. read() or getmsg()2. write() or putmsg()
\end{itemize}

\note[item]{}
\end{frame}
\begin{frame}
\frametitle{Unrelated Title}


\begin{itemize}
\item 1. Copying data from a device into main memory.2. Copying data from a kernel-space buffer into a user-space buffer.
\end{itemize}

\note[item]{}
\end{frame}
\begin{frame}
\frametitle{Unrelated Title}


\begin{itemize}
\item 1. Reduce the number of context switches.2. Reduce how often we need to copy data (i.e., buffers) around in memory.3. Reduce the frequency of device interrupts (e.g., coalescing I/O requests).4. Increase concurrency by having device controllers and I/O channels communicate with a DMA controller, bypassing the CPU.5. Move the most common primitive I/O operations into the hardware.
\end{itemize}

\note[item]{}
\end{frame}
\begin{frame}
\frametitle{Unrelated Title}


\begin{itemize}
\item 1. In application space (easy, but probably too slow).2. In kernel space (more difficult, but more performant; requires greater care).3. In the device controller / hardware (most expensive, but the fastest).
\end{itemize}

\note[item]{}
\end{frame}
\begin{frame}
\frametitle{Unrelated Title}


\begin{itemize}
\item A specific variant or implementation of a command-line interpreter.
\end{itemize}

\note[item]{}
\end{frame}
\begin{frame}
\frametitle{Unrelated Title}


\begin{itemize}
\item 1. Bourne shell2. C shell3. Bourne-Again shell4. Korn shell
\end{itemize}

\note[item]{}
\end{frame}
\begin{frame}
\frametitle{Unrelated Title}


\begin{itemize}
\item system programs
\end{itemize}

\note[item]{}
\end{frame}
\begin{frame}
\frametitle{Unrelated Title}


\begin{itemize}
\item It identifies a specific system program (on disk) that corresponds to the command, loads the program into main memory, and executes it, forwarding any parameters specified by the user.
\end{itemize}

\note[item]{}
\end{frame}
\begin{frame}
\frametitle{Unrelated Title}


\begin{itemize}
\item The Xerox Alto (in 1973)
\end{itemize}

\note[item]{}
\end{frame}
\begin{frame}
\frametitle{Unrelated Title}


\begin{itemize}
\item A mechanism through which a user application can use the operating system's services in a safe and controlled manner.
\end{itemize}

\note[item]{}
\end{frame}
\begin{frame}
\frametitle{Unrelated Title}


\begin{itemize}
\item A set of functions that are available to an application programmer (provided by a system, library, or other component external to the main program), including the parameters passed to each function and set of possible return values.
\end{itemize}

\note[item]{}
\end{frame}
\begin{frame}
\frametitle{Unrelated Title}


\begin{itemize}
\item 1. POSIX (for Unix-like systems, Linux, and macOS).2. Win32 (for Windows systems).3. Java (for the Java virtual machine).
\end{itemize}

\note[item]{}
\end{frame}
\begin{frame}
\frametitle{Unrelated Title}


\begin{itemize}
\item An API provided by a programming language runtime that implements a set of common system calls for a given operating system, allowing the application programmer to more easily make calls to a kernel.
\end{itemize}

\note[item]{}
\end{frame}
\begin{frame}
\frametitle{Unrelated Title}


\begin{itemize}
\item A system-call interface
\end{itemize}

\note[item]{}
\end{frame}
\begin{frame}
\frametitle{Unrelated Title}


\begin{itemize}
\item 1. Pass the parameters in general purpose registers.2. Pass the parameters on the current process's program stack.3. Store the parameters in a block of memory and pass a pointer through a register.
\end{itemize}

\note[item]{}
\end{frame}
\begin{frame}
\frametitle{Unrelated Title}


\begin{itemize}
\item The debugger could invoke a system call to switch the CPU into "single-step mode"; in this mode, the CPU executes a trap after every single instruction. The trap can be caught by the operating system and propagated to the debugger, allowing the programmer to inspect the new state of the program using the debugger.
\end{itemize}

\note[item]{}
\end{frame}
\begin{frame}
\frametitle{Unrelated Title}


\begin{itemize}
\item virtual
\end{itemize}

\note[item]{}
\end{frame}
\begin{frame}
\frametitle{Unrelated Title}


\begin{itemize}
\item 1. Message passing.2. Shared memory.
\end{itemize}

\note[item]{}
\end{frame}
\begin{frame}
\frametitle{Unrelated Title}


\begin{itemize}
\item mechanism from policy
\end{itemize}

\note[item]{}
\end{frame}
\begin{frame}
\frametitle{Unrelated Title}


\begin{itemize}
\item How to do something
\end{itemize}

\note[item]{}
\end{frame}
\begin{frame}
\frametitle{Unrelated Title}


\begin{itemize}
\item What will be done.
\end{itemize}

\note[item]{}
\end{frame}
\begin{frame}
\frametitle{Unrelated Title}


\begin{itemize}
\item Separation of mechanism from policy.
\end{itemize}

\note[item]{}
\end{frame}
\begin{frame}
\frametitle{Unrelated Title}


\begin{itemize}
\item 1. Faster to write.2. More compact.3. Easier to understand (and debug).4. Easier to port to another architecture (i.e., ISA).
\end{itemize}

\note[item]{}
\end{frame}
\begin{frame}
\frametitle{Unrelated Title}


\begin{itemize}
\item 1. Possibly reduced speed*.2. Increased storage needs.
\end{itemize}

\note[item]{}
\end{frame}
\begin{frame}
\frametitle{Unrelated Title}


\begin{itemize}
\item A linker
\end{itemize}

\note[item]{}
\end{frame}
\begin{frame}
\frametitle{Unrelated Title}


\begin{itemize}
\item A loader
\end{itemize}

\note[item]{}
\end{frame}
\begin{frame}
\frametitle{Unrelated Title}


\begin{itemize}
\item a filename, a unique (numeric) identifier
\end{itemize}

\note[item]{}
\end{frame}
\begin{frame}
\frametitle{Unrelated Title}


\begin{itemize}
\item 1. File name.2. Unque identifier (for the file system).3. File type (on some systems).4. Location (on a secondary storage device)..5. Size (in bytes, words, or blocks).6. Access control information.
\end{itemize}

\note[item]{}
\end{frame}
\begin{frame}
\frametitle{Unrelated Title}


\begin{itemize}
\item 1. Create2. Read3. Write4. Reposition5. Delete6. Truncate7. Append8. Rename
\end{itemize}

\note[item]{}
\end{frame}
\begin{frame}
\frametitle{Unrelated Title}


\begin{itemize}
\item 1. create() a new file in the file system.2. read() data from the existing file into a buffer.3. write() data from the buffer to the new file (copy).
\end{itemize}

\note[item]{}
\end{frame}
\begin{frame}
\frametitle{Unrelated Title}


\begin{itemize}
\item open
\end{itemize}

\note[item]{}
\end{frame}
\begin{frame}
\frametitle{Unrelated Title}


\begin{itemize}
\item Access-mode information
\end{itemize}

\note[item]{}
\end{frame}
\begin{frame}
\frametitle{Unrelated Title}


\begin{itemize}
\item 1. Create2. Read-only3. Read-write4. Append-only
\end{itemize}

\note[item]{}
\end{frame}
\begin{frame}
\frametitle{Unrelated Title}


\begin{itemize}
\item A pointer to the associated file entry in the system's open-file table.
\end{itemize}

\note[item]{}
\end{frame}
\begin{frame}
\frametitle{Unrelated Title}


\begin{itemize}
\item 1. A system-wide open-file table.2. A per-process open-file table.
\end{itemize}

\note[item]{}
\end{frame}
\begin{frame}
\frametitle{Unrelated Title}


\begin{itemize}
\item Certain information about a file is independent of any particular process; this includes the file's location on disk, the file's size, its modified dates, etc. We can save memory by storing these attributes in one place in (kernel) memory.
\end{itemize}

\note[item]{}
\end{frame}
\begin{frame}
\frametitle{Unrelated Title}


\begin{itemize}
\item 1. Current file pointer (for operations on the file).2. File access mode (for the process).
\end{itemize}

\note[item]{}
\end{frame}
\begin{frame}
\frametitle{Unrelated Title}


\begin{itemize}
\item 1. The file access mode (read-write, etc).2. The current file pointer.
\end{itemize}

\note[item]{}
\end{frame}
\begin{frame}
\frametitle{Unrelated Title}


\begin{itemize}
\item Each entry in the table includes an open count. This field tracks the number of processes that have opened the file and not yet surrendered its file handle. The count is incremented and decremented whenever a process open()'s and close()'s the file, respectively.When the count reaches zero, then no processes are using the file, and its associated table entry is removed.
\end{itemize}

\note[item]{}
\end{frame}
\begin{frame}
\frametitle{Unrelated Title}


\begin{itemize}
\item A file extension (separated by a dot)
\end{itemize}

\note[item]{}
\end{frame}
\begin{frame}
\frametitle{Unrelated Title}


\begin{itemize}
\item 1. ".com"2. ".exe"3. ".bat" (for batch files)
\end{itemize}

\note[item]{}
\end{frame}
\begin{frame}
\frametitle{Unrelated Title}


\begin{itemize}
\item Hints
\end{itemize}

\note[item]{}
\end{frame}
\begin{frame}
\frametitle{Unrelated Title}


\begin{itemize}
\item magic numbers
\end{itemize}

\note[item]{}
\end{frame}
\begin{frame}
\frametitle{Unrelated Title}


\begin{itemize}
\item They indicate the file's type to the operating system.
\end{itemize}

\note[item]{}
\end{frame}
\begin{frame}
\frametitle{Unrelated Title}


\begin{itemize}
\item A stream of bytes
\end{itemize}

\note[item]{}
\end{frame}
\begin{frame}
\frametitle{Unrelated Title}


\begin{itemize}
\item On Unix, the operating system makes no assumptions about the type or format of a file based on the file extension.
\end{itemize}

\note[item]{}
\end{frame}
\begin{frame}
\frametitle{Unrelated Title}


\begin{itemize}
\item Packed
\end{itemize}

\note[item]{}
\end{frame}
\begin{frame}
\frametitle{Unrelated Title}


\begin{itemize}
\item The process of mapping the logical records in a file (e.g., rows in a spreadsheet file) onto physical blocks of storage on a disk device.
\end{itemize}

\note[item]{}
\end{frame}
\begin{frame}
\frametitle{Unrelated Title}


\begin{itemize}
\item Internal fragmentation
\end{itemize}

\note[item]{}
\end{frame}
\begin{frame}
\frametitle{Unrelated Title}


\begin{itemize}
\item Because writes to disk must be done at block-level granulaity; a file's data size is unlikely to be a perfect multiple of the device's block size, some some space is normally wasted in the final block of data.
\end{itemize}

\note[item]{}
\end{frame}
\begin{frame}
\frametitle{Unrelated Title}


\begin{itemize}
\item A tape model
\end{itemize}

\note[item]{}
\end{frame}
\begin{frame}
\frametitle{Unrelated Title}


\begin{itemize}
\item 1. Seek (forward or backward).2. Read next.3. Write next.
\end{itemize}

\note[item]{}
\end{frame}
\begin{frame}
\frametitle{Unrelated Title}


\begin{itemize}
\item 1. Sequential access.2. Direct (or random) access.
\end{itemize}

\note[item]{}
\end{frame}
\begin{frame}
\frametitle{Unrelated Title}


\begin{itemize}
\item Because the block number is interpreted relative to the first block in the file (as determined by reading a file index).
\end{itemize}

\note[item]{}
\end{frame}
\begin{frame}
\frametitle{Unrelated Title}


\begin{itemize}
\item A data structure, normally stored on disk as well as in memory, that maps the logical records of a file to the physical blocks that store them on disk.
\end{itemize}

\note[item]{}
\end{frame}
\begin{frame}
\frametitle{Unrelated Title}


\begin{itemize}
\item The index could be used to efficiently find the block containing a given logical record; this significantly reduces the amount of disk I/O needed to locate the record.
\end{itemize}

\note[item]{}
\end{frame}
\begin{frame}
\frametitle{Unrelated Title}


\begin{itemize}
\item 1. We can find a logical record's associated disk block and access it much faster.2. We reduce the level of disk I/O requested by our programs, improving overall system performance.
\end{itemize}

\note[item]{}
\end{frame}
\begin{frame}
\frametitle{Unrelated Title}


\begin{itemize}
\item 1. The index must also be stored on disk, increasing storage requirements.2. Additional processing time is needed to generate (and update) the index.
\end{itemize}

\note[item]{}
\end{frame}
\begin{frame}
\frametitle{Unrelated Title}


\begin{itemize}
\item Binary search
\end{itemize}

\note[item]{}
\end{frame}
\begin{frame}
\frametitle{Unrelated Title}


\begin{itemize}
\item We could implement a multi-level index; searching the top-level index (using a given record as the key) would yield a pointer to a second-level index whose entries are pointers to the actual blocks.
\end{itemize}

\note[item]{}
\end{frame}
\begin{frame}
\frametitle{Unrelated Title}


\begin{itemize}
\item Because a single-level table for the file might not fit into memory.
\end{itemize}

\note[item]{}
\end{frame}
\begin{frame}
\frametitle{Unrelated Title}


\begin{itemize}
\item A volume
\end{itemize}

\note[item]{}
\end{frame}
\begin{frame}
\frametitle{Unrelated Title}


\begin{itemize}
\item disk
\end{itemize}

\note[item]{}
\end{frame}
\begin{frame}
\frametitle{Unrelated Title}


\begin{itemize}
\item The device directory (or, simply, the directory).
\end{itemize}

\note[item]{}
\end{frame}
\begin{frame}
\frametitle{Unrelated Title}


\begin{itemize}
\item 1. List all files.2. Search for files (possibly using a search pattern).3. Create a new file.4. Rename a file.5. Delete a file.6. Traverse all files (i.e., file-system traversal).
\end{itemize}

\note[item]{}
\end{frame}
\begin{frame}
\frametitle{Unrelated Title}


\begin{itemize}
\item 12 (8 for a file name and—optionally—1 for a dot and 3 for an extension)
\end{itemize}

\note[item]{}
\end{frame}
\begin{frame}
\frametitle{Unrelated Title}


\begin{itemize}
\item 8.3 filename convention.
\end{itemize}

\note[item]{}
\end{frame}
\begin{frame}
\frametitle{Unrelated Title}


\begin{itemize}
\item 1. File name and extension.2. File attributes (including file type).3. Lettercase information (internal).4. Date created.5. Date modified.6. Address of extended attributes (EA) data (if created).7. Address of the first cluster storing file data.8. File size.
\end{itemize}

\note[item]{}
\end{frame}
\begin{frame}
\frametitle{Unrelated Title}


\begin{itemize}
\item The size (in bits) of each entry in the file allocation table (FAT).
\end{itemize}

\note[item]{}
\end{frame}
\begin{frame}
\frametitle{Unrelated Title}


\begin{itemize}
\item 255 characters
\end{itemize}

\note[item]{}
\end{frame}
\begin{frame}
\frametitle{Unrelated Title}


\begin{itemize}
\item Yes (4096 characters)
\end{itemize}

\note[item]{}
\end{frame}
\begin{frame}
\frametitle{Unrelated Title}


\begin{itemize}
\item 1. It offers a simple mechanism for protecting one user's data from another.2. It allows multiple users to create files with the same file name.
\end{itemize}

\note[item]{}
\end{frame}
\begin{frame}
\frametitle{Unrelated Title}


\begin{itemize}
\item On some systems, it is a table storing entries that each point to the file directory of a user in the system. The MFD is indexed by user name (or account number).
\end{itemize}

\note[item]{}
\end{frame}
\begin{frame}
\frametitle{Unrelated Title}


\begin{itemize}
\item A synonym for a (fully qualified) file path.
\end{itemize}

\note[item]{}
\end{frame}
\begin{frame}
\frametitle{Unrelated Title}


\begin{itemize}
\item 1. A device (volume) identifier.2. A user directory.3. A file name.4. A file extension.
\end{itemize}

\note[item]{}
\end{frame}
\begin{frame}
\frametitle{Unrelated Title}


\begin{itemize}
\item 1. Loaders.2. Assemblers.3. Compilers.4. Command-line programs.5. System utilities.
\end{itemize}

\note[item]{}
\end{frame}
\begin{frame}
\frametitle{Unrelated Title}


\begin{itemize}
\item The sequence of directories that are searched when resolving the location of a file.
\end{itemize}

\note[item]{}
\end{frame}
\begin{frame}
\frametitle{Unrelated Title}


\begin{itemize}
\item A directory that is associated with a process that can be used to resolve files or perform other actions as the process runs. A process's current working directory (CWD) normally defaults to the directory containing the program file.
\end{itemize}

\note[item]{}
\end{frame}
\begin{frame}
\frametitle{Unrelated Title}


\begin{itemize}
\item Normally, the current directory of the parent process.
\end{itemize}

\note[item]{}
\end{frame}
\begin{frame}
\frametitle{Unrelated Title}


\begin{itemize}
\item Making a system call.
\end{itemize}

\note[item]{}
\end{frame}
\begin{frame}
\frametitle{Unrelated Title}


\begin{itemize}
\item A process's current directory.
\end{itemize}

\note[item]{}
\end{frame}
\begin{frame}
\frametitle{Unrelated Title}


\begin{itemize}
\item With a strict tree structure, we cannot share the same file between multiple logical locations in the file-system.
\end{itemize}

\note[item]{}
\end{frame}
\begin{frame}
\frametitle{Unrelated Title}


\begin{itemize}
\item A tree structure
\end{itemize}

\note[item]{}
\end{frame}
\begin{frame}
\frametitle{Unrelated Title}


\begin{itemize}
\item A pointer to another file or directory on a file-system.
\end{itemize}

\note[item]{}
\end{frame}
\begin{frame}
\frametitle{Unrelated Title}


\begin{itemize}
\item We elect not to follow these links whenever we are traversing directories.
\end{itemize}

\note[item]{}
\end{frame}
\begin{frame}
\frametitle{Unrelated Title}


\begin{itemize}
\item A graph structure means that two different internal nodes (i.e., subdirectories) may end up pointing to the same leaf node (i.e., file). This means that a single file may have more than one absolute path from the root.
\end{itemize}

\note[item]{}
\end{frame}
\begin{frame}
\frametitle{Unrelated Title}


\begin{itemize}
\item 1. Traversals: How do we avoid revisiting files that have already been visited?2. Delete semantics: What happens if a file pointed to by multiple links is deleted off disk?
\end{itemize}

\note[item]{}
\end{frame}
\begin{frame}
\frametitle{Unrelated Title}


\begin{itemize}
\item 1. Search the file-system for all links to the file and delete them.2. Wait for a user to attempt to resolve a dangling link, and respond with an invalid access error (as though requesting a non-existant file from the file-system).2. Do nothing.
\end{itemize}

\note[item]{}
\end{frame}
\begin{frame}
\frametitle{Unrelated Title}


\begin{itemize}
\item These systems are designed to do nothing. The user is responsible for recognizing the invalid link and handling it themselves.
\end{itemize}

\note[item]{}
\end{frame}
\begin{frame}
\frametitle{Unrelated Title}


\begin{itemize}
\item The directory entry for a hard link points to the associated file's inode, not to one of its directory entries.
\end{itemize}

\note[item]{}
\end{frame}
\begin{frame}
\frametitle{Unrelated Title}


\begin{itemize}
\item Unix stores a reference count (or hard-link count) for a file inside the file's information block (or inode). This count is incremented and decremented whenever a hard-link is created or deleted, respectively. When the count reaches zero, the system may remove the actual inode from the file-system.
\end{itemize}

\note[item]{}
\end{frame}
\begin{frame}
\frametitle{Unrelated Title}


\begin{itemize}
\item Subdirectories (as these structures may in-turn contain links).
\end{itemize}

\note[item]{}
\end{frame}
\begin{frame}
\frametitle{Unrelated Title}


\begin{itemize}
\item The system can arbitrarily limit the number of directories that are accessed during a traversal (or search).
\end{itemize}

\note[item]{}
\end{frame}
\begin{frame}
\frametitle{Unrelated Title}


\begin{itemize}
\item garbage collection
\end{itemize}

\note[item]{}
\end{frame}
\begin{frame}
\frametitle{Unrelated Title}


\begin{itemize}
\item By relaxing the acyclic requirement on the graph, we introduce the possibility of self-referencing (or cycles). This can lead to situations where a file has a non-zero reference count and yet it cannot be reached via a traversal from the root (i.e., "islands"). A garbage collection algorithm can test the reachability of each file in the file-system graph.
\end{itemize}

\note[item]{}
\end{frame}
\begin{frame}
\frametitle{Unrelated Title}


\begin{itemize}
\item Running these algorithms on a disk-based file-system can be extremely time consuming.
\end{itemize}

\note[item]{}
\end{frame}
\begin{frame}
\frametitle{Unrelated Title}


\begin{itemize}
\item Mounted
\end{itemize}

\note[item]{}
\end{frame}
\begin{frame}
\frametitle{Unrelated Title}


\begin{itemize}
\item The location (in the system's directory structure) where a file-system volume is attached (or mounted).
\end{itemize}

\note[item]{}
\end{frame}
\begin{frame}
\frametitle{Unrelated Title}


\begin{itemize}
\item The operating system could ask the supporting device driver to read the directory structure off the device and verify that it follows the expected file-system format.
\end{itemize}

\note[item]{}
\end{frame}
\begin{frame}
\frametitle{Unrelated Title}


\begin{itemize}
\item 1. Can a volume be mounted at a directory that already contains files?2. Can the same volume be mounted repeatedly at different logical locations?
\end{itemize}

\note[item]{}
\end{frame}
\begin{frame}
\frametitle{Unrelated Title}


\begin{itemize}
\item A system configuration file.
\end{itemize}

\note[item]{}
\end{frame}
\begin{frame}
\frametitle{Unrelated Title}


\begin{itemize}
\item An owner and a group.
\end{itemize}

\note[item]{}
\end{frame}
\begin{frame}
\frametitle{Unrelated Title}


\begin{itemize}
\item The owner may change file attributes, grant file access to other users, and perform any other action associated with the file.
\end{itemize}

\note[item]{}
\end{frame}
\begin{frame}
\frametitle{Unrelated Title}


\begin{itemize}
\item The set of users who can be assigned certain file-sharing permissions.
\end{itemize}

\note[item]{}
\end{frame}
\begin{frame}
\frametitle{Unrelated Title}


\begin{itemize}
\item File attributes
\end{itemize}

\note[item]{}
\end{frame}
\begin{frame}
\frametitle{Unrelated Title}


\begin{itemize}
\item A many-to-many client-server model.
\end{itemize}

\note[item]{}
\end{frame}
\begin{frame}
\frametitle{Unrelated Title}


\begin{itemize}
\item The active directory protocol.
\end{itemize}

\note[item]{}
\end{frame}
\begin{frame}
\frametitle{Unrelated Title}


\begin{itemize}
\item Secure single sign-on (SSO)
\end{itemize}

\note[item]{}
\end{frame}
\begin{frame}
\frametitle{Unrelated Title}


\begin{itemize}
\item 1. Active directory.2. Lightweight directory-access protocol (LDAP).
\end{itemize}

\note[item]{}
\end{frame}
\begin{frame}
\frametitle{Unrelated Title}


\begin{itemize}
\item 1. Mechanical disk failure (i.e., head crash, bad sector read, power loss, etc).2. On-disk directory structure (i.e., corrupted due to hardware malfunction or bug).3. Failure in the disk controller.4. Failures over the serial bus or cable.5. Failures in the host-adapter hardware.6. Software bugs in the disk's device driver program.
\end{itemize}

\note[item]{}
\end{frame}
\begin{frame}
\frametitle{Unrelated Title}


\begin{itemize}
\item 1. Consistency semantics.2. Failure semantics.
\end{itemize}

\note[item]{}
\end{frame}
\begin{frame}
\frametitle{Unrelated Title}


\begin{itemize}
\item A set of rules governing how resources may be accessed simultaneously in a multi-user (e.g., networked) system.
\end{itemize}

\note[item]{}
\end{frame}
\begin{frame}
\frametitle{Unrelated Title}


\begin{itemize}
\item Immediately.
\end{itemize}

\note[item]{}
\end{frame}
\begin{frame}
\frametitle{Unrelated Title}


\begin{itemize}
\item No. Processes (and users) must contend for access to the file. The file is an exclusive resource, and processes must wait to acquire exclusive access to it before operating on it.
\end{itemize}

\note[item]{}
\end{frame}
\begin{frame}
\frametitle{Unrelated Title}


\begin{itemize}
\item The set of accesses or operations performed on a file by a process, after a call to open() and before a matching call to close().
\end{itemize}

\note[item]{}
\end{frame}
\begin{frame}
\frametitle{Unrelated Title}


\begin{itemize}
\item 1. Read access ('r') (may include name, attributes, etc).2. Write access ('w') (may include append, delete, etc).3. Execute access ('x').
\end{itemize}

\note[item]{}
\end{frame}
\begin{frame}
\frametitle{Unrelated Title}


\begin{itemize}
\item A list of user names and associated access permissions for a given file or resource.
\end{itemize}

\note[item]{}
\end{frame}
\begin{frame}
\frametitle{Unrelated Title}


\begin{itemize}
\item Yes. Files can only have 1 group in Unix.
\end{itemize}

\note[item]{}
\end{frame}
\begin{frame}
\frametitle{Unrelated Title}


\begin{itemize}
\item 1. Read: 'r'2. Write: 'w'3. Execute: 'x'
\end{itemize}

\note[item]{}
\end{frame}
\begin{frame}
\frametitle{Unrelated Title}


\begin{itemize}
\item 9 bits (3 bits for user, 3 bits for group, etc)
\end{itemize}

\note[item]{}
\end{frame}
\begin{frame}
\frametitle{Unrelated Title}


\begin{itemize}
\item 1. Access permissions (owner, group, and system-wide).2. Number of links to the file.3. Owner name.4. Group name.5. File size (in bytes).6. Last modified date.7. File name.
\end{itemize}

\note[item]{}
\end{frame}
\begin{frame}
\frametitle{Unrelated Title}


\begin{itemize}
\item 1. Listing of files.2. Creation and deletion of files.3. Renaming.
\end{itemize}

\note[item]{}
\end{frame}
\begin{frame}
\frametitle{Unrelated Title}


\begin{itemize}
\item The device driver translates high-level commands (such as "retrieve block XYZ") from the operating system into low-level, hardware-specific instructions or signals that are recieved by the I/O hardware controller. Thus, a device driver serves as an effective abstraction layer for accessing I/O devices.
\end{itemize}

\note[item]{}
\end{frame}
\begin{frame}
\frametitle{Unrelated Title}


\begin{itemize}
\item 1. The operating system.2. The logical file system.3. The file-organization module.4. The basic file system.5. The device driver.
\end{itemize}

\note[item]{}
\end{frame}
\begin{frame}
\frametitle{Unrelated Title}


\begin{itemize}
\item The file-organization module.
\end{itemize}

\note[item]{}
\end{frame}
\begin{frame}
\frametitle{Unrelated Title}


\begin{itemize}
\item The file-organization module translates a file's logical block addresses into physical block addresses on the disk. It typically includes a free-space manager to track block allocation on disk.
\end{itemize}

\note[item]{}
\end{frame}
\begin{frame}
\frametitle{Unrelated Title}


\begin{itemize}
\item A component in a file-system that tracks which blocks are allocated and unallocated, and provides unallocated blocks to the file-system when requested.
\end{itemize}

\note[item]{}
\end{frame}
\begin{frame}
\frametitle{Unrelated Title}


\begin{itemize}
\item Free space
\end{itemize}

\note[item]{}
\end{frame}
\begin{frame}
\frametitle{Unrelated Title}


\begin{itemize}
\item The logical file-system.
\end{itemize}

\note[item]{}
\end{frame}
\begin{frame}
\frametitle{Unrelated Title}


\begin{itemize}
\item The directory structure
\end{itemize}

\note[item]{}
\end{frame}
\begin{frame}
\frametitle{Unrelated Title}


\begin{itemize}
\item The file control block (FCB) data structure.
\end{itemize}

\note[item]{}
\end{frame}
\begin{frame}
\frametitle{Unrelated Title}


\begin{itemize}
\item Metadata about a file (separate from the actual file data).
\end{itemize}

\note[item]{}
\end{frame}
\begin{frame}
\frametitle{Unrelated Title}


\begin{itemize}
\item 1. File size.2. Permissions.3. Owner, group, and ACL information.4. Access dates.5. Pointer(s) to file data blocks.
\end{itemize}

\note[item]{}
\end{frame}
\begin{frame}
\frametitle{Unrelated Title}


\begin{itemize}
\item The logical file-system layer.
\end{itemize}

\note[item]{}
\end{frame}
\begin{frame}
\frametitle{Unrelated Title}


\begin{itemize}
\item A layered design reduces code duplication and allows one layer to support multiple (different) implementations of high-level layers (i.e., different logical file-systems).
\end{itemize}

\note[item]{}
\end{frame}
\begin{frame}
\frametitle{Unrelated Title}


\begin{itemize}
\item 1. The basic file system layer.2. The device driver (i.e., I/O control code).
\end{itemize}

\note[item]{}
\end{frame}
\begin{frame}
\frametitle{Unrelated Title}


\begin{itemize}
\item 1. Files could be any number of bytes, and not strictly aligned to a block length.2. A master directory (or root directory) provided a tree structure file-system, where subdirectories could contain their own subdirectories, and so on.3. A new set of APIs—including open(), read(), and close()—for operating on files.4. The user of special (designated) system files that could be operated on via standard APIs in order to interface with certain operating system features and logical devices (i.e., "Everything is a file").
\end{itemize}

\note[item]{}
\end{frame}
\begin{frame}
\frametitle{Unrelated Title}


\begin{itemize}
\item 1. Long file names.2. Symbolic links.
\end{itemize}

\note[item]{}
\end{frame}
\begin{frame}
\frametitle{Unrelated Title}


\begin{itemize}
\item The extended file system (e.g., ext2, ext3).
\end{itemize}

\note[item]{}
\end{frame}
\begin{frame}
\frametitle{Unrelated Title}


\begin{itemize}
\item The boot control block (alternately, "boot block" or "partition boot sector").
\end{itemize}

\note[item]{}
\end{frame}
\begin{frame}
\frametitle{Unrelated Title}


\begin{itemize}
\item A disk block that holds information about a storage partition.
\end{itemize}

\note[item]{}
\end{frame}
\begin{frame}
\frametitle{Unrelated Title}


\begin{itemize}
\item 1. The number of blocks on the partition.2. The partition's block size.3. The free block count and free-block pointers.4. The free FCB count and FCB pointers.
\end{itemize}

\note[item]{}
\end{frame}
\begin{frame}
\frametitle{Unrelated Title}


\begin{itemize}
\item The logical file-system layer.
\end{itemize}

\note[item]{}
\end{frame}
\begin{frame}
\frametitle{Unrelated Title}


\begin{itemize}
\item A pointer to an entry in the process's open-file table.
\end{itemize}

\note[item]{}
\end{frame}
\begin{frame}
\frametitle{Unrelated Title}


\begin{itemize}
\item A file descriptor (or, a file handle).
\end{itemize}

\note[item]{}
\end{frame}
\begin{frame}
\frametitle{Unrelated Title}


\begin{itemize}
\item A volume is the logical storage space that comprises a file-system. A partition is a region (normally a subregion) of a physical storage disk. A logical volume may span one or more partitions, and those partitions may span one or more physical disks.
\end{itemize}

\note[item]{}
\end{frame}
\begin{frame}
\frametitle{Unrelated Title}


\begin{itemize}
\item The disk (or partition) does not store a file-system image.
\end{itemize}

\note[item]{}
\end{frame}
\begin{frame}
\frametitle{Unrelated Title}


\begin{itemize}
\item The disk (or partition) stores a file-system image.
\end{itemize}

\note[item]{}
\end{frame}
\begin{frame}
\frametitle{Unrelated Title}


\begin{itemize}
\item At boot time, no file-system drivers are loaded, so the system cannot access or interpret data stored on a file-system's volume.
\end{itemize}

\note[item]{}
\end{frame}
\begin{frame}
\frametitle{Unrelated Title}


\begin{itemize}
\item The operating system's kernel image (and sometimes other system files).
\end{itemize}

\note[item]{}
\end{frame}
\begin{frame}
\frametitle{Unrelated Title}


\begin{itemize}
\item Unix allocates a flag field and a pointer field in each directory's inode:1. Unix sets the flag to indicate that the directory represents a file-system volume.2. The pointer is set to point to the volume's entry in the system's mount table.3. The table entry stores a pointer to the superblock on the volume's device (disk).The system can follow these pointers to resolve directory and file information, allowing users to traverse from one file-system directory to another.
\end{itemize}

\note[item]{}
\end{frame}
\begin{frame}
\frametitle{Unrelated Title}


\begin{itemize}
\item 1. Doing so abstracts away details that are unique to each individual file-system; it provides a consistent API that application programmers can use to access and manipulate files regardless of the particular file-systems that are used.2. Extending this idea, it allows one machine reading data from one file-system to share files with other machines using other file-systems. We can leverage this common software layer to associate network-wide unique IDs with files, creating a network-wide file namespace.
\end{itemize}

\note[item]{}
\end{frame}
\begin{frame}
\frametitle{Unrelated Title}


\begin{itemize}
\item vnodes
\end{itemize}

\note[item]{}
\end{frame}
\begin{frame}
\frametitle{Unrelated Title}


\begin{itemize}
\item A data structure used by the virtual file system (VFS) layer to store a file's network-wide unique identifier and other information.
\end{itemize}

\note[item]{}
\end{frame}
\begin{frame}
\frametitle{Unrelated Title}


\begin{itemize}
\item 1. superblock: Represents an entire file-system.2. dentry: Represents an individual directory entry.3. inode: Represents a file.4. file: Represents an open file.
\end{itemize}

\note[item]{}
\end{frame}
\begin{frame}
\frametitle{Unrelated Title}


\begin{itemize}
\item A function table.
\end{itemize}

\note[item]{}
\end{frame}
\begin{frame}
\frametitle{Unrelated Title}


\begin{itemize}
\item Linear search (must look for existing file or conflicting files.
\end{itemize}

\note[item]{}
\end{frame}
\begin{frame}
\frametitle{Unrelated Title}


\begin{itemize}
\item While still storing a directory's files in a linked list, we could simultaneously store pointers to each of those list entries in a hash table. Filenames would then be hashed (when the file is created) and used to retrieve a pointer to the matching file entry in the linked list. This scheme allows us to retrieve a specific file entry (for open, delete, etc) in constant time instead of linear time.
\end{itemize}

\note[item]{}
\end{frame}
\begin{frame}
\frametitle{Unrelated Title}


\begin{itemize}
\item 1. Contiguous allocation.2. Linked allocation.3. Indexed allocation.
\end{itemize}

\note[item]{}
\end{frame}
\begin{frame}
\frametitle{Unrelated Title}


\begin{itemize}
\item Contiguous blocks
\end{itemize}

\note[item]{}
\end{frame}
\begin{frame}
\frametitle{Unrelated Title}


\begin{itemize}
\item 1. The address of the first block of data.2. The length of the file data (in blocks).
\end{itemize}

\note[item]{}
\end{frame}
\begin{frame}
\frametitle{Unrelated Title}


\begin{itemize}
\item Assuming that the limits of a contiguous allocation file-system are accetable to the operating system, then this strategy benefits from minimum disk head movement and seek time.
\end{itemize}

\note[item]{}
\end{frame}
\begin{frame}
\frametitle{Unrelated Title}


\begin{itemize}
\item How to satisfy a request of size n from a list of free holes.
\end{itemize}

\note[item]{}
\end{frame}
\begin{frame}
\frametitle{Unrelated Title}


\begin{itemize}
\item We could request a new set of free blocks (called an extent) and use space at the end of the file's existing block range to store the address and block size of the extent. The file-system can detect these entries and jump to any additional extents during read and/or write operations. This scheme can effectively distribute a file's data across multiple non-contiguous block regions.
\end{itemize}

\note[item]{}
\end{frame}
\begin{frame}
\frametitle{Unrelated Title}


\begin{itemize}
\item Each file is stored as a linked list of blocks on disk. A file's directory entry stores the addresses of the first and last block entries of the list. In addition to data, each block stores a pointer to the next block in the list.
\end{itemize}

\note[item]{}
\end{frame}
\begin{frame}
\frametitle{Unrelated Title}


\begin{itemize}
\item 1. A contiguous allocation scheme can cause significant external fragmentation. A linked allocation exhibits no external fragmentation, as any free block can be used to store data for any file.2. A contiguous allocation scheme requires the application programmer to request a specific size for the initial file; as the size of a file may need to grow over time, this can result in significant internal fragmentation across the file-system.
\end{itemize}

\note[item]{}
\end{frame}
\begin{frame}
\frametitle{Unrelated Title}


\begin{itemize}
\item When file access patterns are more random-access than sequential.
\end{itemize}

\note[item]{}
\end{frame}
\begin{frame}
\frametitle{Unrelated Title}


\begin{itemize}
\item The linked strategy requires more storage, as we need to store pointers inside of each block (thus creating the linked-list structure). The size of the pointers and the size of the blocks determines the blocks' storage efficiency (i.e., pointers vs. data).
\end{itemize}

\note[item]{}
\end{frame}
\begin{frame}
\frametitle{Unrelated Title}


\begin{itemize}
\item Block clusters
\end{itemize}

\note[item]{}
\end{frame}
\begin{frame}
\frametitle{Unrelated Title}


\begin{itemize}
\item A logical grouping of contiguous blocks on disk used by a clustering allocation strategy. Such a strategy effectively makes our allocation operations less granular.
\end{itemize}

\note[item]{}
\end{frame}
\begin{frame}
\frametitle{Unrelated Title}


\begin{itemize}
\item By grouping together neighboring blocks into logical clusters for allocation, we can significantly reduce the ratio of structural metadata (i.e., list pointers) to file data.
\end{itemize}

\note[item]{}
\end{frame}
\begin{frame}
\frametitle{Unrelated Title}


\begin{itemize}
\item 1. Clustering improves disk throughput, as it reduces disk seeking.2. Clustering improves storage efficiency by requiring less space to store data structure information (i.e., list entry pointers) and free-list data structures.
\end{itemize}

\note[item]{}
\end{frame}
\begin{frame}
\frametitle{Unrelated Title}


\begin{itemize}
\item Clustering increases internal fragmentation by increasing the minimum allocation unit from a block to a cluster.
\end{itemize}

\note[item]{}
\end{frame}
\begin{frame}
\frametitle{Unrelated Title}


\begin{itemize}
\item We could use a doubly-linked list structure; if a forward pointer is damaged, we could traverse the backward pointers (from the tail) to locate all data blocks.
\end{itemize}

\note[item]{}
\end{frame}
\begin{frame}
\frametitle{Unrelated Title}


\begin{itemize}
\item For each file in the file-system, a block—called the index block—is allocated to store the array of block locations holding the file's data. This strategy is similar to the paging strategies used by operating systems to support virtual memory schemes.
\end{itemize}

\note[item]{}
\end{frame}
\begin{frame}
\frametitle{Unrelated Title}


\begin{itemize}
\item Because we might allocate an entire (index) block for a file that has very few actual blocks of data (i.e., wasted index space).
\end{itemize}

\note[item]{}
\end{frame}
\begin{frame}
\frametitle{Unrelated Title}


\begin{itemize}
\item We could allocate space for the first few data block addresses (i.e., direct blocks) directly inside of the file's inode, and then allocate a full index block only for larger files. The inode can then also contain pointers to the full index block(s).
\end{itemize}

\note[item]{}
\end{frame}
\begin{frame}
\frametitle{Unrelated Title}


\begin{itemize}
\item 1. We could link together multiple index blocks for a single file such that one index block pointers to the next index block. As a file grows, additional index blocks would be added to the linked list.2. We could implement a multi-level indexing scheme wherein an indirect block stores addresses pointing to individual index blocks; those index blocks in turn point to actual data blocks; this scheme can be generalized to support some arbitrary number of levels.
\end{itemize}

\note[item]{}
\end{frame}
\begin{frame}
\frametitle{Unrelated Title}


\begin{itemize}
\item File access pattern (i.e., sequential vs. random access).
\end{itemize}

\note[item]{}
\end{frame}
\begin{frame}
\frametitle{Unrelated Title}


\begin{itemize}
\item The free-space list.
\end{itemize}

\note[item]{}
\end{frame}
\begin{frame}
\frametitle{Unrelated Title}


\begin{itemize}
\item A list that maintains the set of all unallocated (free) disk blocks from which the file-system can allocate a block to store file data.
\end{itemize}

\note[item]{}
\end{frame}
\begin{frame}
\frametitle{Unrelated Title}


\begin{itemize}
\item 1. Using a bitmask (or bit vector).2. Using a linked-list structure (i.e., each free block points to the next).
\end{itemize}

\note[item]{}
\end{frame}
\begin{frame}
\frametitle{Unrelated Title}


\begin{itemize}
\item A bitmask can encode the state (allocated or free) of each block using a single bit; thus, it is the smallest data structure capable of storing the state of all unique blocks.
\end{itemize}

\note[item]{}
\end{frame}
\begin{frame}
\frametitle{Unrelated Title}


\begin{itemize}
\item Most processors have a special instruction that takes a word value as an operand and produces the offset (in the word) of the first bit that is set (or zero). Thus, we can use this to scan through sequential words to find the first word containing a set bit. We can use the word and bit offsets to calculate the address of the first free block:(# of bits per word) * (# of 0-value words scanned) + (offset of the first set bit)
\end{itemize}

\note[item]{}
\end{frame}
\begin{frame}
\frametitle{Unrelated Title}


\begin{itemize}
\item Traversing a linked-list encoded across the set of all free blocks requires extensive non-sequential disk I/O seeks.
\end{itemize}

\note[item]{}
\end{frame}
\begin{frame}
\frametitle{Unrelated Title}


\begin{itemize}
\item 100 Gb / 4,096 bytes per block = 25,000,000 blocks ~= 25 Mb bitmask size
\end{itemize}

\note[item]{}
\end{frame}
\begin{frame}
\frametitle{Unrelated Title}


\begin{itemize}
\item A linked list structure is encoded directly within the file allocation table; thus, storage of the list structure is not distributed across the entire set of free blocks on disk.
\end{itemize}

\note[item]{}
\end{frame}
\begin{frame}
\frametitle{Unrelated Title}


\begin{itemize}
\item Dedicate a number of free blocks to serve as address blocks: all words in the block except for the last is used to store the address of an actual free block in the system; the last word is used to store a pointer pointing to the next address block.
\end{itemize}

\note[item]{}
\end{frame}
\begin{frame}
\frametitle{Unrelated Title}


\begin{itemize}
\item 1. Less seek time between reading a file's starting block address from its metadata entry (i.e., inode) and actually reading file data.2. Less seek time between writing to a file's data block and updating the file's metadata (i.e., last modified date) on disk.
\end{itemize}

\note[item]{}
\end{frame}
\begin{frame}
\frametitle{Unrelated Title}


\begin{itemize}
\item Unified virtual memory.
\end{itemize}

\note[item]{}
\end{frame}
\begin{frame}
\frametitle{Unrelated Title}


\begin{itemize}
\item A technique in which the (virtual memory) page cache is used as a cache for file data brought into memory from disk. File data is updated on disk whenever dirty pages are flushed by the virtual memory system.
\end{itemize}

\note[item]{}
\end{frame}
\begin{frame}
\frametitle{Unrelated Title}


\begin{itemize}
\item A phenomenon that occurs when memory-mapped files that are brought into a buffer cache are subsequently cached in the page cache, wasting memory, CPU cycles, and I/O cycles.
\end{itemize}

\note[item]{}
\end{frame}
\begin{frame}
\frametitle{Unrelated Title}


\begin{itemize}
\item Double-caching
\end{itemize}

\note[item]{}
\end{frame}
\begin{frame}
\frametitle{Unrelated Title}


\begin{itemize}
\item When choosing which page of virtual memory to evict (i.e., when available memory runs low), a priority paging scheme places greater priority on preserving processes (i.e., programs and program data mapped into virtual memory) rather than file data (i.e., mapped into virtual memory from the file-system).
\end{itemize}

\note[item]{}
\end{frame}
\begin{frame}
\frametitle{Unrelated Title}


\begin{itemize}
\item Because the "writes" simply copy data in the file buffer to pages in the virtual memory page cache—to be written back to disk at some later point by the kernel. Reads must bring in new data using disk I/O (unless there are already clean pages of data in the page cache).
\end{itemize}

\note[item]{}
\end{frame}
\begin{frame}
\frametitle{Unrelated Title}


\begin{itemize}
\item Consistency checker programs
\end{itemize}

\note[item]{}
\end{frame}
\begin{frame}
\frametitle{Unrelated Title}


\begin{itemize}
\item A program that uses the file-system's directory structure as well as on-disk data to correct inconsistencies in the file-system state (when possible). A consistency checker can be run at system reboot time, or manually by a system admin.
\end{itemize}

\note[item]{}
\end{frame}
\begin{frame}
\frametitle{Unrelated Title}


\begin{itemize}
\item System crashes, loss of power, etc.
\end{itemize}

\note[item]{}
\end{frame}
\begin{frame}
\frametitle{Unrelated Title}


\begin{itemize}
\item 1. fsck (for Unix).2. chkdsk (for MS-DOS).
\end{itemize}

\note[item]{}
\end{frame}
\begin{frame}
\frametitle{Unrelated Title}


\begin{itemize}
\item Journaling file-systems.
\end{itemize}

\note[item]{}
\end{frame}
\begin{frame}
\frametitle{Unrelated Title}


\begin{itemize}
\item A buffer that overwrites older entries with newer entries (in a circular manner) when there is no free space left in the buffer.
\end{itemize}

\note[item]{}
\end{frame}
\begin{frame}
\frametitle{Unrelated Title}


\begin{itemize}
\item Doing this allows us to add journal entries using more sequential I/O—as the set of blocks holding log entries is contiguous on disk.
\end{itemize}

\note[item]{}
\end{frame}
\begin{frame}
\frametitle{Unrelated Title}


\begin{itemize}
\item circular buffer
\end{itemize}

\note[item]{}
\end{frame}
\begin{frame}
\frametitle{Unrelated Title}


\begin{itemize}
\item The system writes a new entry into the file-system's (circular) log buffer describing the metadata operations, and then returns to the calling application.
\end{itemize}

\note[item]{}
\end{frame}
\begin{frame}
\frametitle{Unrelated Title}


\begin{itemize}
\item A client-server relationship*.
\end{itemize}

\note[item]{}
\end{frame}
\begin{frame}
\frametitle{Unrelated Title}


\begin{itemize}
\item 1. RPC primitives.2. An external data representation (XDR) protocol.
\end{itemize}

\note[item]{}
\end{frame}
\begin{frame}
\frametitle{Unrelated Title}


\begin{itemize}
\item 1. The mounting protocol.2. The remote file access protocol.
\end{itemize}

\note[item]{}
\end{frame}
\begin{frame}
\frametitle{Unrelated Title}


\begin{itemize}
\item 1. The list of local filesystems that may be mounted by clients.2. The list of remote client machines that are permitted to mount them.3. Optionally, specific permissions for each client.
\end{itemize}

\note[item]{}
\end{frame}
\begin{frame}
\frametitle{Unrelated Title}


\begin{itemize}
\item A file handle, which the client can use as a key for subsequent NFS requests.
\end{itemize}

\note[item]{}
\end{frame}
\begin{frame}
\frametitle{Unrelated Title}


\begin{itemize}
\item 1. A file-system identifier.2. The inode number of the directory* that was mounted.
\end{itemize}

\note[item]{}
\end{frame}
\begin{frame}
\frametitle{Unrelated Title}


\begin{itemize}
\item 1. Reading a set of directory entries (i.e., directory listing).2. Manipulating directories and links.3. Reading and writing files.4. Manipulating file attributes.5. Searching for items in a directory.
\end{itemize}

\note[item]{}
\end{frame}
\begin{frame}
\frametitle{Unrelated Title}


\begin{itemize}
\item NFS is a set of stateless protocols. Servers do not track information about a client between individual client accesses.
\end{itemize}

\note[item]{}
\end{frame}
\begin{frame}
\frametitle{Unrelated Title}


\begin{itemize}
\item Kernel threads
\end{itemize}

\note[item]{}
\end{frame}
\begin{frame}
\frametitle{Unrelated Title}


\begin{itemize}
\item Path-name translation takes a filepath as input and produces the set of (component name, directory vnode) pairs for all components in the filepath.
\end{itemize}

\note[item]{}
\end{frame}
\begin{frame}
\frametitle{Unrelated Title}


\begin{itemize}
\item Any remote component along the file path may in fact be a mounted directory from some other (remote) file-system. Thus, it isn't possible to perform one lookup operation that is more than one level deep at a time.
\end{itemize}

\note[item]{}
\end{frame}
\begin{frame}
\frametitle{Unrelated Title}


\begin{itemize}
\item 1. The file attributes cached (TTL ~60 seconds).2. The file data (block) cache.
\end{itemize}

\note[item]{}
\end{frame}
\begin{frame}
\frametitle{Unrelated Title}


\begin{itemize}
\item ◦ Base tech (low/◦ Key tech (medium/◦ Pacing tech (high/future)
\end{itemize}

\note[item]{}
\end{frame}
\begin{frame}
\frametitle{Unrelated Title}


\begin{itemize}
\item ◦ New product/qualitative change to existing ◦ Process innovation◦ Completely new market◦ New source of raw materials or other input◦ Change in the industrial organization
\end{itemize}

\note[item]{}
\end{frame}
\begin{frame}
\frametitle{Unrelated Title}


\begin{itemize}
\item 1. Macroeconomic perspective2. Microeconomic perspective
\end{itemize}

\note[item]{}
\end{frame}
\begin{frame}
\frametitle{Unrelated Title}


\begin{itemize}
\item 1. Creative destruction (old stuff is replaced with new stuff)2. Neighborhoods of equilibrium (innovation is centered both spatial and temporary)
\end{itemize}

\note[item]{}
\end{frame}
\begin{frame}
\frametitle{Unrelated Title}


\begin{itemize}
\item 1. Necessity of temporary monopolies -> higher incentive to innovate2. Benefit of entrepreneurial activity -> those smf disturbe the status quo 
\end{itemize}

\note[item]{}
\end{frame}
\begin{frame}
\frametitle{Unrelated Title}


\begin{itemize}
\item A value increase of an existing resouce.
\end{itemize}

\note[item]{}
\end{frame}
\begin{frame}
\frametitle{Unrelated Title}


\begin{itemize}
\item - erodes the competitve advantage of the incumbents- triggered by new market entrants
\end{itemize}

\note[item]{}
\end{frame}
\begin{frame}
\frametitle{Unrelated Title}


\begin{itemize}
\item - the optimal way to build a component- the tech that is required to build a component
\end{itemize}

\note[item]{}
\end{frame}
\begin{frame}
\frametitle{Unrelated Title}


\begin{itemize}
\item - integration of different components in a company- communication of the single components
\end{itemize}

\note[item]{}
\end{frame}
\begin{frame}
\frametitle{Unrelated Title}


\begin{itemize}
\item disruptive, revolutionary
\end{itemize}

\note[item]{}
\end{frame}
\begin{frame}
\frametitle{Unrelated Title}


\begin{itemize}
\item evolutionary / sustaining
\end{itemize}

\note[item]{}
\end{frame}
\begin{frame}
\frametitle{Unrelated Title}


\begin{itemize}
\item ◦ Normative level (values/culture/◦ Strategic level (competitive advantage/how)◦ Operative level (results/what)
\end{itemize}

\note[item]{}
\end{frame}
\begin{frame}
\frametitle{Unrelated Title}


\begin{itemize}
\item ◦ Long-term goal◦ Must be sustainable ◦ Should create a competitive advantage
\end{itemize}

\note[item]{}
\end{frame}
\begin{frame}
\frametitle{Unrelated Title}


\begin{itemize}
\item A high demand for a existing product, but with another configuration.
\end{itemize}

\note[item]{}
\end{frame}
\begin{frame}
\frametitle{Unrelated Title}


\begin{itemize}
\item The invention of a new technology that creates new and more satisfied solutions for customers.
\end{itemize}

\note[item]{}
\end{frame}
\begin{frame}
\frametitle{Unrelated Title}


\begin{itemize}
\item ▪ Mismatch between innovation ideas and corporate intrest▪ Focus lies on what can be researched
\end{itemize}

\note[item]{}
\end{frame}
\begin{frame}
\frametitle{Unrelated Title}


\begin{itemize}
\item Satisfied needs are might be only those with minor impact.Probability of radical innovation is lower.
\end{itemize}

\note[item]{}
\end{frame}
\begin{frame}
\frametitle{Unrelated Title}


\begin{itemize}
\item Maximum returns with Minimal risk
\end{itemize}

\note[item]{}
\end{frame}
\begin{frame}
\frametitle{Unrelated Title}


\begin{itemize}
\item Stars, questionmarks, cash cows, poor dogs
\end{itemize}

\note[item]{}
\end{frame}
\begin{frame}
\frametitle{Unrelated Title}


\begin{itemize}
\item disinvest: bottom-left areaselect: diagonal top-left to bottom-rightinvest: upper-right area
\end{itemize}

\note[item]{}
\end{frame}
\begin{frame}
\frametitle{Unrelated Title}


\begin{itemize}
\item An optimal /balanced resource allocation.
\item The visualization can reduce the complexety of a problem.
\end{itemize}

\note[item]{}
\end{frame}
\begin{frame}
\frametitle{Unrelated Title}


\begin{itemize}
\item Clayton Christinson
\end{itemize}

\note[item]{}
\end{frame}
\begin{frame}
\frametitle{Unrelated Title}


\begin{itemize}
\item - Serves lower tier customers- Has a lower price with a higher margine than the incumbents product
\end{itemize}

\note[item]{}
\end{frame}
\begin{frame}
\frametitle{Unrelated Title}


\begin{itemize}
\item To the higher margins of the premium serctor of a product.This process pushes the incumbent out of the market.
\end{itemize}

\note[item]{}
\end{frame}
\begin{frame}
\frametitle{Unrelated Title}


\begin{itemize}
\item Customers want more and more features (innovations)Company can not defend its lower price segments anymore -> potential danger of new market entrants 
\end{itemize}

\note[item]{}
\end{frame}
\begin{frame}
\frametitle{Unrelated Title}


\begin{itemize}
\item idea -> discovery -> research -> development -> invention
\end{itemize}

\note[item]{}
\end{frame}
\begin{frame}
\frametitle{Unrelated Title}


\begin{itemize}
\item the marketlaunch
\end{itemize}

\note[item]{}
\end{frame}
\begin{frame}
\frametitle{Unrelated Title}


\begin{itemize}
\item Lot of input (tech and research) small amount of ideas make it through it.
\end{itemize}

\note[item]{}
\end{frame}
\begin{frame}
\frametitle{Unrelated Title}


\begin{itemize}
\item  idea generation → idea exploration → product development → commercializationeach phase is seperated by gateways
\end{itemize}

\note[item]{}
\end{frame}
\begin{frame}
\frametitle{Unrelated Title}


\begin{itemize}
\item     • Product quality: robustness and reliability of a product    • Product cost: costs of manufacturing  → what’s the profit?    • Development time: quick earnings?    • Development cost: amount initial investment    • Development Capabilities: knowledge and expertise to build and mass produce a product
\end{itemize}

\note[item]{}
\end{frame}
\begin{frame}
\frametitle{Unrelated Title}


\begin{itemize}
\item - Technology monitoring- Technology evaluation
\end{itemize}

\note[item]{}
\end{frame}
\begin{frame}
\frametitle{Unrelated Title}


\begin{itemize}
\item - Product life cycles 
\item - Performance development via the S-curve- Ansoff Matrix
\end{itemize}

\note[item]{}
\end{frame}
\begin{frame}
\frametitle{Unrelated Title}


\begin{itemize}
\item - Identification of customer needs- Analysis of potential market
\end{itemize}

\note[item]{}
\end{frame}
\begin{frame}
\frametitle{Unrelated Title}


\begin{itemize}
\item - delphi- positioning- conjoint
\end{itemize}

\note[item]{}
\end{frame}
\begin{frame}
\frametitle{Unrelated Title}


\begin{itemize}
\item - competition monitoring- strategy to profit form innovation- how to enter and exit the market
\end{itemize}

\note[item]{}
\end{frame}
\begin{frame}
\frametitle{Unrelated Title}


\begin{itemize}
\item CustomerInnovatorSupplierImitator
\end{itemize}

\note[item]{}
\end{frame}
\begin{frame}
\frametitle{Unrelated Title}


\begin{itemize}
\item - Appropriability regime (is innovation easy to protect)- Life cycle phase- Complementary assets
\end{itemize}

\note[item]{}
\end{frame}
\begin{frame}
\frametitle{Unrelated Title}


\begin{itemize}
\item many new innovations companies are trying and testing new stuff
\end{itemize}

\note[item]{}
\end{frame}
\begin{frame}
\frametitle{Unrelated Title}


\begin{itemize}
\item The dominant design.
\end{itemize}

\note[item]{}
\end{frame}
\begin{frame}
\frametitle{Unrelated Title}


\begin{itemize}
\item Asset that can increase the power of a new innovation.
\item Increases the base value of a product.
\end{itemize}

\note[item]{}
\end{frame}
\begin{frame}
\frametitle{Unrelated Title}


\begin{itemize}
\item generic, cospecialized, specialized
\end{itemize}

\note[item]{}
\end{frame}
\begin{frame}
\frametitle{Unrelated Title}


\begin{itemize}
\item Teece
\end{itemize}

\note[item]{}
\end{frame}
\begin{frame}
\frametitle{Unrelated Title}


\begin{itemize}
\item - sunken cost bias- not spending all money - too influential managers
\end{itemize}

\note[item]{}
\end{frame}
\begin{frame}
\frametitle{Unrelated Title}


\begin{itemize}
\item - Technical- Scheduling- Approval- Market
\end{itemize}

\note[item]{}
\end{frame}
\begin{frame}
\frametitle{Unrelated Title}


\begin{itemize}
\item 1. Choose Criteria -> Filter your favorite criteria for different the projects2. Choose Weight -> every chosen criteria gets its own weight3. Evaluate Projects -> Assign evaluation to each project and criteria (highly subjective)
\end{itemize}

\note[item]{}
\end{frame}
\begin{frame}
\frametitle{Unrelated Title}


\begin{itemize}
\item A set of estimated future cashflows. -> can be error proneThe average weighted cost per capital.
\end{itemize}

\note[item]{}
\end{frame}
\begin{frame}
\frametitle{Unrelated Title}


\begin{itemize}
\item Can make assumptions of uncetain events.
\end{itemize}

\note[item]{}
\end{frame}
\begin{frame}
\frametitle{Unrelated Title}


\begin{itemize}
\item A Calculation of the present value of future costs and profits.
\end{itemize}

\note[item]{}
\end{frame}
\begin{frame}
\frametitle{Unrelated Title}


\begin{itemize}
\item Scoring, Decision trees, Discounted Cash Flow.
\end{itemize}

\note[item]{}
\end{frame}
\begin{frame}
\frametitle{Unrelated Title}


\begin{itemize}
\item tech breakthrough -> new product -> increased sale -> more R&D investment
\end{itemize}

\note[item]{}
\end{frame}
\begin{frame}
\frametitle{Unrelated Title}


\begin{itemize}
\item - input is a mix of external and internal tech bases- Spin-offs to target total different markets are possible- New tech can be added (licensing or acquisition)
\end{itemize}

\note[item]{}
\end{frame}
\begin{frame}
\frametitle{Unrelated Title}


\begin{itemize}
\item Chesbrough
\end{itemize}

\note[item]{}
\end{frame}
\begin{frame}
\frametitle{Unrelated Title}


\begin{itemize}
\item - Solution information side- Need information side
\end{itemize}

\note[item]{}
\end{frame}
\begin{frame}
\frametitle{Unrelated Title}


\begin{itemize}
\item Side that develops products -> efficiency in the innovation process
\end{itemize}

\note[item]{}
\end{frame}
\begin{frame}
\frametitle{Unrelated Title}


\begin{itemize}
\item Generates product ideas -> effectiveness of the innovation process
\end{itemize}

\note[item]{}
\end{frame}
\begin{frame}
\frametitle{Unrelated Title}


\begin{itemize}
\item Innovation comes from user -> user revel their innovations for free
\end{itemize}

\note[item]{}
\end{frame}
\begin{frame}
\frametitle{Unrelated Title}


\begin{itemize}
\item selling the innovation
\end{itemize}

\note[item]{}
\end{frame}
\begin{frame}
\frametitle{Unrelated Title}


\begin{itemize}
\item Use the innovation (typically lead user)
\end{itemize}

\note[item]{}
\end{frame}
\begin{frame}
\frametitle{Unrelated Title}


\begin{itemize}
\item The lower End or unserved market.
\end{itemize}

\note[item]{}
\end{frame}
\begin{frame}
\frametitle{Unrelated Title}


\begin{itemize}
\item Incumbent is slowly squeezed out of the market. New tech is getting better or/and cheaper.
\end{itemize}

\note[item]{}
\end{frame}
\begin{frame}
\frametitle{Unrelated Title}


\begin{itemize}
\item Innovating for more features for the customer -> lower sections are unserved-> New market entrant can replace highly innovative company
\end{itemize}

\note[item]{}
\end{frame}
\begin{frame}
\frametitle{Unrelated Title}


\begin{itemize}
\item base / key/ pacing (was introduced recently and has high competetive advantage)
\end{itemize}

\note[item]{}
\end{frame}
\begin{frame}
\frametitle{Unrelated Title}


\begin{itemize}
\item Embryonic (a lot of value is added, nearly no users) -> pacing -> key -> base tech (little value added, lot of users)
\end{itemize}

\note[item]{}
\end{frame}
\begin{frame}
\frametitle{Unrelated Title}


\begin{itemize}
\item Slow startRise of technical performanceReach of limits
\end{itemize}

\note[item]{}
\end{frame}
\begin{frame}
\frametitle{Unrelated Title}


\begin{itemize}
\item lot of investment -> tiny performance gain
\end{itemize}

\note[item]{}
\end{frame}
\begin{frame}
\frametitle{Unrelated Title}


\begin{itemize}
\item strategic early detectionstrategic planningImplementation → overcome innovation barriersStrategic Control → monitoring of the development
\end{itemize}

\note[item]{}
\end{frame}
\begin{frame}
\frametitle{Unrelated Title}


\begin{itemize}
\item - Technology monitoring- Technology roadmaps
\end{itemize}

\note[item]{}
\end{frame}
\begin{frame}
\frametitle{Unrelated Title}


\begin{itemize}
\item Product developmentDiversificationMarket PenetrationMarket Development
\end{itemize}

\note[item]{}
\end{frame}
\begin{frame}
\frametitle{Unrelated Title}


\begin{itemize}
\item ProductMarket
\end{itemize}

\note[item]{}
\end{frame}
\begin{frame}
\frametitle{Unrelated Title}


\begin{itemize}
\item Generate innovation or feedback with users that are ahead of the trend.
\end{itemize}

\note[item]{}
\end{frame}
\begin{frame}
\frametitle{Unrelated Title}


\begin{itemize}
\item - stickieness of innovation (high transfer cost of information)- ability to profit from innovation
\end{itemize}

\note[item]{}
\end{frame}
\begin{frame}
\frametitle{Unrelated Title}


\begin{itemize}
\item the need infromation
\end{itemize}

\note[item]{}
\end{frame}
\begin{frame}
\frametitle{Unrelated Title}


\begin{itemize}
\item solution information
\end{itemize}

\note[item]{}
\end{frame}
\begin{frame}
\frametitle{Unrelated Title}


\begin{itemize}
\item Enhances own reputation.Own innovation can be imporved by others.Benefits of networkeffects
\end{itemize}

\note[item]{}
\end{frame}
\begin{frame}
\frametitle{Unrelated Title}


\begin{itemize}
\item LUs are often not the customer
\end{itemize}

\note[item]{}
\end{frame}
\begin{frame}
\frametitle{Unrelated Title}


\begin{itemize}
\item technology analysisdemand analysiscompetition analysis
\end{itemize}

\note[item]{}
\end{frame}
\begin{frame}
\frametitle{Unrelated Title}


\begin{itemize}
\item Lack of data.Unfinished product. (features are hard to anticipate)Bet on the future.
\end{itemize}

\note[item]{}
\end{frame}
\begin{frame}
\frametitle{Unrelated Title}


\begin{itemize}
\item - delphi analysis (develop of market long term view)- positioning analysis (which market segment is attractive)- conjoint analysis (which property of the product has the highest customer value)
\end{itemize}

\note[item]{}
\end{frame}
\begin{frame}
\frametitle{Unrelated Title}


\begin{itemize}
\item No. Exert group is invited, that fills out a survey.
\end{itemize}

\note[item]{}
\end{frame}
\begin{frame}
\frametitle{Unrelated Title}


\begin{itemize}
\item Is a itereative, multi-step process.
\end{itemize}

\note[item]{}
\end{frame}
\begin{frame}
\frametitle{Unrelated Title}


\begin{itemize}
\item Advantage:- Longterm forecast- Experts can solve complex problems and avoid pitfalls
\end{itemize}

\note[item]{}
\end{frame}
\begin{frame}
\frametitle{Unrelated Title}


\begin{itemize}
\item - time consuming- no direct open discussion
\end{itemize}

\note[item]{}
\end{frame}
\begin{frame}
\frametitle{Unrelated Title}


\begin{itemize}
\item 1. List potential dimensions. (e.g. brand image, price)2. Choos segmentation area. (filter)3. Split market into segments. (e.g. high, low)
\end{itemize}

\note[item]{}
\end{frame}
\begin{frame}
\frametitle{Unrelated Title}


\begin{itemize}
\item Segmentations must be meaningful Internally homogeneousExternally heterogeneous
\end{itemize}

\note[item]{}
\end{frame}
\begin{frame}
\frametitle{Unrelated Title}


\begin{itemize}
\item Identify the property setting of a product that is valued most by the customer.
\end{itemize}

\note[item]{}
\end{frame}
\begin{frame}
\frametitle{Unrelated Title}


\begin{itemize}
\item Multiple products are presented.The potential customer can set an order or choose the best candidate.The needed product characteristics are deduced.
\end{itemize}

\note[item]{}
\end{frame}
\begin{frame}
\frametitle{Unrelated Title}


\begin{itemize}
\item Revelas unknown needs of a customer (even unknown to the customer him/herself)
\end{itemize}

\note[item]{}
\end{frame}
\begin{frame}
\frametitle{Unrelated Title}


\begin{itemize}
\item Spreading of the innovation’s use over time. (non linear, due to network externaities)
\end{itemize}

\note[item]{}
\end{frame}
\begin{frame}
\frametitle{Unrelated Title}


\begin{itemize}
\item Single unit of demand that decides to use the new product.
\end{itemize}

\note[item]{}
\end{frame}
\begin{frame}
\frametitle{Unrelated Title}


\begin{itemize}
\item AwarenessFormation of opinionDecisionValidation
\end{itemize}

\note[item]{}
\end{frame}
\begin{frame}
\frametitle{Unrelated Title}


\begin{itemize}
\item Chasem is located between the early adopters and the early majority.
\end{itemize}

\note[item]{}
\end{frame}
\begin{frame}
\frametitle{Unrelated Title}


\begin{itemize}
\item Innovators -> ealry adopters -> early majority -> late majority -> laggards
\end{itemize}

\note[item]{}
\end{frame}
\begin{frame}
\frametitle{Unrelated Title}


\begin{itemize}
\item Drastic slow down of the adoption rate.Markteing must address customer groups differently.
\end{itemize}

\note[item]{}
\end{frame}
\begin{frame}
\frametitle{Unrelated Title}


\begin{itemize}
\item The Side with the Solution Information (How to produce a new product?)
\end{itemize}

\note[item]{}
\end{frame}
\begin{frame}
\frametitle{Unrelated Title}


\begin{itemize}
\item Wisdom of crowd -> user and communitiesBroadcast search  -> contests, tournaments
\end{itemize}

\note[item]{}
\end{frame}
\begin{frame}
\frametitle{Unrelated Title}


\begin{itemize}
\item Accumulated knowledge created by crowd members
\end{itemize}

\note[item]{}
\end{frame}
\begin{frame}
\frametitle{Unrelated Title}


\begin{itemize}
\item Request for help -> hope to find and expert
\end{itemize}

\note[item]{}
\end{frame}
\begin{frame}
\frametitle{Unrelated Title}


\begin{itemize}
\item - Difficult problem that need high-value knowledge- Problem that needs experimentation / multiple solutions
\end{itemize}

\note[item]{}
\end{frame}
\begin{frame}
\frametitle{Unrelated Title}


\begin{itemize}
\item Screening -> filter iteratively for LUsPyramiding -> ask one LU for another LU (uses networks)User Competitions -> Ideation contests
\end{itemize}

\note[item]{}
\end{frame}
\begin{frame}
\frametitle{Unrelated Title}


\begin{itemize}
\item learning effects -> users do not want to learn new stuffnetwork externalites -> lot of complementary goods are existing
\end{itemize}

\note[item]{}
\end{frame}
\begin{frame}
\frametitle{Unrelated Title}


\begin{itemize}
\item Fluid phase: lot of experiments with product design (pre paradigmatic)Specific phase: industry uses same tech base (paradigmatic)
\end{itemize}

\note[item]{}
\end{frame}
\begin{frame}
\frametitle{Unrelated Title}


\begin{itemize}
\item - unclear cost target- little information flow- no clear strategic objective
\end{itemize}

\note[item]{}
\end{frame}
\begin{frame}
\frametitle{Unrelated Title}


\begin{itemize}
\item We people speak with others that are close to them.
\end{itemize}

\note[item]{}
\end{frame}
\begin{frame}
\frametitle{Unrelated Title}


\begin{itemize}
\item - letter shaped buildings -> extreme isolation- no coffe rooms- leader not in the center of the building- misplacement of vertical connecors
\end{itemize}

\note[item]{}
\end{frame}
\begin{frame}
\frametitle{Unrelated Title}


\begin{itemize}
\item Stuff that is placed at a workspace to show others the own value mindset or status.
\end{itemize}

\note[item]{}
\end{frame}
\begin{frame}
\frametitle{Unrelated Title}


\begin{itemize}
\item Decomposition: break problem up into feasible componentsRepresentation: represent problems as simple as possible
\end{itemize}

\note[item]{}
\end{frame}
\begin{frame}
\frametitle{Unrelated Title}


\begin{itemize}
\item Problem requires diversity of approches (experimentation).
\end{itemize}

\note[item]{}
\end{frame}
\begin{frame}
\frametitle{Unrelated Title}


\begin{itemize}
\item Innovation requires a cumulative knowledge base -> aggregation of diverse inputs.
\end{itemize}

\note[item]{}
\end{frame}
\begin{frame}
\frametitle{Unrelated Title}


\begin{itemize}
\item Small percent of research population does generate more than 50% of the output.
\end{itemize}

\note[item]{}
\end{frame}
\begin{frame}
\frametitle{Unrelated Title}


\begin{itemize}
\item Stratup guy is mor riskyDoes not need highpay checkNot so interested in job security
\end{itemize}

\note[item]{}
\end{frame}
\begin{frame}
\frametitle{Unrelated Title}


\begin{itemize}
\item FounderJoinerEstablished firm employee
\end{itemize}

\note[item]{}
\end{frame}
\begin{frame}
\frametitle{Unrelated Title}


\begin{itemize}
\item not knowing / not wanting / being not able / no permission
\end{itemize}

\note[item]{}
\end{frame}
\begin{frame}
\frametitle{Unrelated Title}


\begin{itemize}
\item         ◦ skill promotor (knowledge) -> not knowing / being not able        ◦ power promotor (material stuff like money)  -> not wanting        ◦ process promotor (is very good at creating connections) -> no permission
\end{itemize}

\note[item]{}
\end{frame}
\begin{frame}
\frametitle{Unrelated Title}


\begin{itemize}
\item sticky informationlocal search bais
\end{itemize}

\note[item]{}
\end{frame}
\begin{frame}
\frametitle{Unrelated Title}


\begin{itemize}
\item Expert toolkit:- Higher learning cost- Allows for creation of behavior.Basic toolkit:- Used in B2C - Offeres options -> user can only choose.
\end{itemize}

\note[item]{}
\end{frame}
\begin{frame}
\frametitle{Unrelated Title}


\begin{itemize}
\item Hippel
\end{itemize}

\note[item]{}
\end{frame}
\begin{frame}
\frametitle{Unrelated Title}


\begin{itemize}
\item - use innovation in own product- exclude others- direct profits (licence fees)- indirect profits (complementary goods)
\end{itemize}

\note[item]{}
\end{frame}
\begin{frame}
\frametitle{Unrelated Title}


\begin{itemize}
\item A patent must be renewed after 20 years.
\end{itemize}

\note[item]{}
\end{frame}
\begin{frame}
\frametitle{Unrelated Title}


\begin{itemize}
\item Invention must be new,commercially usableand must be based on inventive activity.
\end{itemize}

\note[item]{}
\end{frame}
\begin{frame}
\frametitle{Unrelated Title}


\begin{itemize}
\item - Creates cost for rivals- Image and signaling (for young firms that are seeking for VC)
\end{itemize}

\note[item]{}
\end{frame}
\begin{frame}
\frametitle{Unrelated Title}


\begin{itemize}
\item - The opportunity is lost due to secrecy- Detection of patent violation is hard to enforce
\end{itemize}

\note[item]{}
\end{frame}
\begin{frame}
\frametitle{Unrelated Title}


\begin{itemize}
\item No, it is a protection that enables appropriation. (like: exclusion, licensing)
\end{itemize}

\note[item]{}
\end{frame}
\begin{frame}
\frametitle{Unrelated Title}


\begin{itemize}
\item Complemnetary assets for production, sales and branding.
\end{itemize}

\note[item]{}
\end{frame}
\begin{frame}
\frametitle{Unrelated Title}


\begin{itemize}
\item A competitor can independently discover the invention.
\end{itemize}

\note[item]{}
\end{frame}
\begin{frame}
\frametitle{Unrelated Title}


\begin{itemize}
\item The intellectual property rights and strong property rights via patents.
\end{itemize}

\note[item]{}
\end{frame}
\begin{frame}
\frametitle{Unrelated Title}


\begin{itemize}
\item Granting others the use of the own IPR.
\end{itemize}

\note[item]{}
\end{frame}
\begin{frame}
\frametitle{Unrelated Title}


\begin{itemize}
\item - Establish and maintaining the intellectual property rights- Monitoring and prosecution of infringements
\end{itemize}

\note[item]{}
\end{frame}
\begin{frame}
\frametitle{Unrelated Title}


\begin{itemize}
\item Does prevent patent wars and therefore supports innovation.
\end{itemize}

\note[item]{}
\end{frame}
\begin{frame}
\frametitle{Unrelated Title}


\begin{itemize}
\item Budle of patents is offered to a competitor.
\end{itemize}

\note[item]{}
\end{frame}
\begin{frame}
\frametitle{Unrelated Title}


\begin{itemize}
\item - Increased demand for inputs- Increased demand for complements
\end{itemize}

\note[item]{}
\end{frame}
\begin{frame}
\frametitle{Unrelated Title}


\begin{itemize}
\item Biotech, pharmaceuticals and chemicals
\end{itemize}

\note[item]{}
\end{frame}
\begin{frame}
\frametitle{Unrelated Title}


\begin{itemize}
\item - Preferred treatment to alternatives- Continuation of the present program
\end{itemize}

\note[item]{}
\end{frame}
\begin{frame}
\frametitle{Unrelated Title}


\begin{itemize}
\item ExternalInternal
\end{itemize}

\note[item]{}
\end{frame}
\begin{frame}
\frametitle{Unrelated Title}


\begin{itemize}
\item - cognitive (not knowing)- psychological (not wanting)- resource (not abel)- organizational (not allowed)
\end{itemize}

\note[item]{}
\end{frame}
\begin{frame}
\frametitle{Unrelated Title}


\begin{itemize}
\item - skill barrier- motivation barrier- startegic barrier- operative barrier
\end{itemize}

\note[item]{}
\end{frame}
\begin{frame}
\frametitle{Unrelated Title}


\begin{itemize}
\item Expert Promotor (has high technological knowlege -> solves ignorance)Power Promotor (uses incentives or punishment -> solves unwillingness)Process Promotor (connects between administrative barriers)
\end{itemize}

\note[item]{}
\end{frame}
\begin{frame}
\frametitle{Unrelated Title}


\begin{itemize}
\item Smart peopleInternal and external R&DBeing firstProfi form innovationProfit and innovationIP
\end{itemize}

\note[item]{}
\end{frame}
\begin{frame}
\frametitle{Unrelated Title}


\begin{itemize}
\item Crossing firm boundaries in the search for knowledge / solution.
\end{itemize}

\note[item]{}
\end{frame}
\begin{frame}
\frametitle{Unrelated Title}


\begin{itemize}
\item ITIL is the world's most widely used service management framework.
\end{itemize}

\note[item]{}
\end{frame}
\begin{frame}
\frametitle{Unrelated Title}


\begin{itemize}
\item AXELOS
 is a joint venture set up in 2014 by the Government of the United 
Kingdom and Capita, to develop, manage and operate qualifications in 
best practice such as ITIL.
\end{itemize}

\note[item]{}
\end{frame}
\begin{frame}
\frametitle{Unrelated Title}


\begin{itemize}
\item CompTIA
\end{itemize}

\note[item]{}
\end{frame}
\begin{frame}
\frametitle{Unrelated Title}


\begin{itemize}
\item ISACA
\end{itemize}

\note[item]{}
\end{frame}
\begin{frame}
\frametitle{Unrelated Title}


\begin{itemize}
\item (ISC)² 
\end{itemize}

\note[item]{}
\end{frame}
\begin{frame}
\frametitle{Unrelated Title}


\begin{itemize}
\item AXELOS
\end{itemize}

\note[item]{}
\end{frame}
\begin{frame}
\frametitle{Unrelated Title}


\begin{itemize}
\item ITIL® Specialist - Create, Deliver, & SupportITIL® Specialist – Drive Stakeholder Value
\end{itemize}

\note[item]{}
\end{frame}
\begin{frame}
\frametitle{Unrelated Title}


\begin{itemize}
\item This is a course on the ITIL® Strategic Leader (SL) path.
\end{itemize}

\note[item]{}
\end{frame}
\begin{frame}
\frametitle{Unrelated Title}


\begin{itemize}
\item ITIL® Specialist - Create, Deliver, & Support
\end{itemize}

\note[item]{}
\end{frame}
\begin{frame}
\frametitle{Unrelated Title}


\begin{itemize}
\item ITIL Foundation
\end{itemize}

\note[item]{}
\end{frame}
\begin{frame}
\frametitle{Unrelated Title}


\begin{itemize}
\item ITIL® Specialist – Drive Stakeholder Value
\end{itemize}

\note[item]{}
\end{frame}
\begin{frame}
\frametitle{Unrelated Title}


\begin{itemize}
\item The candidate has 75 minutes to answer 40 questions if not taking the exam in their native or working language.
\item candidate has 60 minutes to complete the 40-question exam. If they not 
\item taking it in their native or working language, then they have 75 minutes
\item to complete the exam.
\end{itemize}

\note[item]{}
\end{frame}
\begin{frame}
\frametitle{Unrelated Title}


\begin{itemize}
\item 40
\end{itemize}

\note[item]{}
\end{frame}
\begin{frame}
\frametitle{Unrelated Title}


\begin{itemize}
\item ITIL
 4 provides organizations with a comprehensive framework for Information
 Technology Service Management (ITSM). It is designed to ensure that an 
effective, efficient, flexible, coordinated and integrated system for 
governance and management of IT services is established and continually 
improving in the organization.
\end{itemize}

\note[item]{}
\end{frame}
\begin{frame}
\frametitle{Unrelated Title}


\begin{itemize}
\item Information Technology Service Management (ITSM)
\end{itemize}

\note[item]{}
\end{frame}
\begin{frame}
\frametitle{Unrelated Title}


\begin{itemize}
\item Information libraries
\item term “library” has been officially removed and deprecated in ITIL 4. 
\item ITIL 4 offers a comprehensive framework for Information Technology 
\item Service Management (ITSM).
\end{itemize}

\note[item]{}
\end{frame}
\begin{frame}
\frametitle{Unrelated Title}


\begin{itemize}
\item They are IT-enabled
\item all services are IT-enabled, meaning there is tremendous benefit for 
\item organizations in creating, expanding, and improving their IT service 
\item management capability.
\end{itemize}

\note[item]{}
\end{frame}
\begin{frame}
\frametitle{Unrelated Title}


\begin{itemize}
\item Service management
\end{itemize}

\note[item]{}
\end{frame}
\begin{frame}
\frametitle{Unrelated Title}


\begin{itemize}
\item Services
\end{itemize}

\note[item]{}
\end{frame}
\begin{frame}
\frametitle{Unrelated Title}


\begin{itemize}
\item on-premise
\end{itemize}

\note[item]{}
\end{frame}
\begin{frame}
\frametitle{Unrelated Title}

\begin{center}
\includegraphics[width=0.9\textwidth,height=0.9\textheight,keepaspectratio]{/Users/I516998/Library/Application Support/Anki2/User 1/collection.media/Figure-1.1-on-Page-3-of-ITIL-Foundation..png}
\end{center}

\begin{itemize}
\item ITIL Service Value System (SVS)
\end{itemize}

\note[item]{}
\end{frame}
\begin{frame}
\frametitle{Unrelated Title}


\begin{itemize}
\item ITIL® 4
\end{itemize}

\note[item]{}
\end{frame}
\begin{frame}
\frametitle{Unrelated Title}


\begin{itemize}
\item To remember
\end{itemize}

\note[item]{}
\end{frame}
\begin{frame}
\frametitle{Unrelated Title}


\begin{itemize}
\item official ITIL 4 question types
\end{itemize}

\note[item]{}
\end{frame}
\begin{frame}
\frametitle{Unrelated Title}


\begin{itemize}
\item Understand 7 ITIL practices
\end{itemize}

\note[item]{}
\end{frame}
\begin{frame}
\frametitle{Unrelated Title}


\begin{itemize}
\item 2000
\end{itemize}

\note[item]{}
\end{frame}
\begin{frame}
\frametitle{Unrelated Title}


\begin{itemize}
\item 65%
\end{itemize}

\note[item]{}
\end{frame}
\begin{frame}
\frametitle{Unrelated Title}


\begin{itemize}
\item ITIL® Strategist - Direct, Plan, & Improve
\end{itemize}

\note[item]{}
\end{frame}
\begin{frame}
\frametitle{Unrelated Title}


\begin{itemize}
\item valid levels of the ITIL 4 certification scheme
\end{itemize}

\note[item]{}
\end{frame}
\begin{frame}
\frametitle{Unrelated Title}


\begin{itemize}
\item these technologies offer fresh opportunities for value creation.
\end{itemize}

\note[item]{}
\end{frame}
\begin{frame}
\frametitle{Unrelated Title}


\begin{itemize}
\item Bloom's level 1, which represents approximately 77.5% simply involves to remember.  Remembering is the lowest level of learning in the cognitive domain in Bloom's taxonomy, and typically doesn't bring about a change in behavior.
\item Bloom's level 2, which is about 22.5% of the questions, involves to understand.
\end{itemize}

\note[item]{}
\end{frame}
\begin{frame}
\frametitle{Unrelated Title}


\begin{itemize}
\item ITIL StrategistDirect, Plan, & ImproveITIL Leader Digital & IT Strategy
\end{itemize}

\note[item]{}
\end{frame}
\begin{frame}
\frametitle{Unrelated Title}


\begin{itemize}
\item ITIL Specialist, Create, Deliver, & SupportITIL Specialist, Drive Stakeholder ValueITIL Specialist, High Velocity ITITIL Strategist, Direct, Plan, & Improve
\end{itemize}

\note[item]{}
\end{frame}
\begin{frame}
\frametitle{Unrelated Title}


\begin{itemize}
\item -standard-missing word-list-what is not
\end{itemize}

\note[item]{}
\end{frame}
\begin{frame}
\frametitle{Unrelated Title}


\begin{itemize}
\item Service offering
\end{itemize}

\note[item]{}
\end{frame}
\begin{frame}
\frametitle{Unrelated Title}


\begin{itemize}
\item Service offering
\end{itemize}

\note[item]{}
\end{frame}
\begin{frame}
\frametitle{Unrelated Title}


\begin{itemize}
\item service action
\end{itemize}

\note[item]{}
\end{frame}
\begin{frame}
\frametitle{Unrelated Title}


\begin{itemize}
\item A product is a configuration of resources, created by the organization that will be potentially valuable for their customers.
\end{itemize}

\note[item]{}
\end{frame}
\begin{frame}
\frametitle{Unrelated Title}


\begin{itemize}
\item product
\end{itemize}

\note[item]{}
\end{frame}
\begin{frame}
\frametitle{Unrelated Title}


\begin{itemize}
\item service
\end{itemize}

\note[item]{}
\end{frame}
\begin{frame}
\frametitle{Unrelated Title}


\begin{itemize}
\item A 
service is simply a means of enabling value co-creation by facilitating 
outcomes that customers want to achieve without the customer having to 
manage specific costs and risks.
\end{itemize}

\note[item]{}
\end{frame}
\begin{frame}
\frametitle{Unrelated Title}


\begin{itemize}
\item service provision 
\end{itemize}

\note[item]{}
\end{frame}
\begin{frame}
\frametitle{Unrelated Title}


\begin{itemize}
\item Utilization of the provider's resourcesRequesting of service actions to fulfil
\end{itemize}

\note[item]{}
\end{frame}
\begin{frame}
\frametitle{Unrelated Title}


\begin{itemize}
\item Service relationship management
\end{itemize}

\note[item]{}
\end{frame}
\begin{frame}
\frametitle{Unrelated Title}


\begin{itemize}
\item Service relationship management
\end{itemize}

\note[item]{}
\end{frame}
\begin{frame}
\frametitle{Unrelated Title}


\begin{itemize}
\item Service actions
\end{itemize}

\note[item]{}
\end{frame}
\begin{frame}
\frametitle{Unrelated Title}


\begin{itemize}
\item Service provisioning
\end{itemize}

\note[item]{}
\end{frame}
\begin{frame}
\frametitle{Unrelated Title}


\begin{itemize}
\item Service consumption
\end{itemize}

\note[item]{}
\end{frame}
\begin{frame}
\frametitle{Unrelated Title}


\begin{itemize}
\item Costs
\end{itemize}

\note[item]{}
\end{frame}
\begin{frame}
\frametitle{Unrelated Title}


\begin{itemize}
\item Outputs
\end{itemize}

\note[item]{}
\end{frame}
\begin{frame}
\frametitle{Unrelated Title}


\begin{itemize}
\item Outcomes
\end{itemize}

\note[item]{}
\end{frame}
\begin{frame}
\frametitle{Unrelated Title}


\begin{itemize}
\item Risks
\end{itemize}

\note[item]{}
\end{frame}
\begin{frame}
\frametitle{Unrelated Title}


\begin{itemize}
\item costs removed from the consumer by the service and costs imposed on the consumer by the service.
\end{itemize}

\note[item]{}
\end{frame}
\begin{frame}
\frametitle{Unrelated Title}


\begin{itemize}
\item Warranty
\end{itemize}

\note[item]{}
\end{frame}
\begin{frame}
\frametitle{Unrelated Title}


\begin{itemize}
\item Warranty typically addresses such areas as availability of a service, its capacity, levels of security and continuity.
\end{itemize}

\note[item]{}
\end{frame}
\begin{frame}
\frametitle{Unrelated Title}


\begin{itemize}
\item value
\end{itemize}

\note[item]{}
\end{frame}
\begin{frame}
\frametitle{Unrelated Title}


\begin{itemize}
\item Utility
\end{itemize}

\note[item]{}
\end{frame}
\begin{frame}
\frametitle{Unrelated Title}


\begin{itemize}
\item Purposeful
\end{itemize}

\note[item]{}
\end{frame}
\begin{frame}
\frametitle{Unrelated Title}


\begin{itemize}
\item Co-creation
\end{itemize}

\note[item]{}
\end{frame}
\begin{frame}
\frametitle{Unrelated Title}


\begin{itemize}
\item Service offerings
\end{itemize}

\note[item]{}
\end{frame}
\begin{frame}
\frametitle{Unrelated Title}


\begin{itemize}
\item Service actions
\end{itemize}

\note[item]{}
\end{frame}
\begin{frame}
\frametitle{Unrelated Title}


\begin{itemize}
\item co-creation
\end{itemize}

\note[item]{}
\end{frame}
\begin{frame}
\frametitle{Unrelated Title}


\begin{itemize}
\item Service consumption
\end{itemize}

\note[item]{}
\end{frame}
\begin{frame}
\frametitle{Unrelated Title}


\begin{itemize}
\item concerning consumer roles
\end{itemize}

\note[item]{}
\end{frame}
\begin{frame}
\frametitle{Unrelated Title}


\begin{itemize}
\item combined
\end{itemize}

\note[item]{}
\end{frame}
\begin{frame}
\frametitle{Unrelated Title}


\begin{itemize}
\item the core consumer roles and relationships
\end{itemize}

\note[item]{}
\end{frame}
\begin{frame}
\frametitle{Unrelated Title}


\begin{itemize}
\item Service consumer roles may often have conflicting interests
\end{itemize}

\note[item]{}
\end{frame}
\begin{frame}
\frametitle{Unrelated Title}


\begin{itemize}
\item The customer role
\end{itemize}

\note[item]{}
\end{frame}
\begin{frame}
\frametitle{Unrelated Title}


\begin{itemize}
\item Shareholders
\end{itemize}

\note[item]{}
\end{frame}
\begin{frame}
\frametitle{Unrelated Title}


\begin{itemize}
\item Partners and suppliers
\end{itemize}

\note[item]{}
\end{frame}
\begin{frame}
\frametitle{Unrelated Title}


\begin{itemize}
\item Society, community, and charitable organizations
\end{itemize}

\note[item]{}
\end{frame}
\begin{frame}
\frametitle{Unrelated Title}


\begin{itemize}
\item Service provider employees
\end{itemize}

\note[item]{}
\end{frame}
\begin{frame}
\frametitle{Unrelated Title}


\begin{itemize}
\item Shareholders
\end{itemize}

\note[item]{}
\end{frame}
\begin{frame}
\frametitle{Unrelated Title}


\begin{itemize}
\item components of service provisioning
\end{itemize}

\note[item]{}
\end{frame}
\begin{frame}
\frametitle{Unrelated Title}


\begin{itemize}
\item Service Relationship
\end{itemize}

\note[item]{}
\end{frame}
\begin{frame}
\frametitle{Unrelated Title}


\begin{itemize}
\item organization as provider and consumer of services
\end{itemize}

\note[item]{}
\end{frame}
\begin{frame}
\frametitle{Unrelated Title}


\begin{itemize}
\item Utility
\end{itemize}

\note[item]{}
\end{frame}
\begin{frame}
\frametitle{Unrelated Title}


\begin{itemize}
\item core consumer roles
\end{itemize}

\note[item]{}
\end{frame}
\begin{frame}
\frametitle{Unrelated Title}


\begin{itemize}
\item consumer could contribute to the reduction of risk
\end{itemize}

\note[item]{}
\end{frame}
\begin{frame}
\frametitle{Unrelated Title}


\begin{itemize}
\item Products
\end{itemize}

\note[item]{}
\end{frame}
\begin{frame}
\frametitle{Unrelated Title}


\begin{itemize}
\item Developing the distinctive proficiencies of the organization
\end{itemize}

\note[item]{}
\end{frame}
\begin{frame}
\frametitle{Unrelated Title}


\begin{itemize}
\item Employees
\end{itemize}

\note[item]{}
\end{frame}
\begin{frame}
\frametitle{Unrelated Title}


\begin{itemize}
\item Sponsor
\end{itemize}

\note[item]{}
\end{frame}
\begin{frame}
\frametitle{Unrelated Title}


\begin{itemize}
\item Warranty
\end{itemize}

\note[item]{}
\end{frame}
\begin{frame}
\frametitle{Unrelated Title}


\begin{itemize}
\item Risk
\end{itemize}

\note[item]{}
\end{frame}
\begin{frame}
\frametitle{Unrelated Title}


\begin{itemize}
\item Service consumer
\end{itemize}

\note[item]{}
\end{frame}
\begin{frame}
\frametitle{Unrelated Title}


\begin{itemize}
\item valid stakeholders in value
\end{itemize}

\note[item]{}
\end{frame}
\begin{frame}
\frametitle{Unrelated Title}


\begin{itemize}
\item service offerings
\end{itemize}

\note[item]{}
\end{frame}
\begin{frame}
\frametitle{Unrelated Title}


\begin{itemize}
\item Service provisioning
\end{itemize}

\note[item]{}
\end{frame}
\begin{frame}
\frametitle{Unrelated Title}


\begin{itemize}
\item Organizationservice providersservice consumersother stakeholders
\end{itemize}

\note[item]{}
\end{frame}
\begin{frame}
\frametitle{Unrelated Title}


\begin{itemize}
\item customers, users and sponsors.
\end{itemize}

\note[item]{}
\end{frame}
\begin{frame}
\frametitle{Unrelated Title}


\begin{itemize}
\item service relationship managementservice provisionservice consumption
\end{itemize}

\note[item]{}
\end{frame}
\begin{frame}
\frametitle{Unrelated Title}


\begin{itemize}
\item outcome
\item utilityrisk
\end{itemize}

\note[item]{}
\end{frame}
\begin{frame}
\frametitle{Unrelated Title}


\begin{itemize}
\item Service management is defined as a set of specialized organizational capabilities for enabling value to customers in the form of services.
\end{itemize}

\note[item]{}
\end{frame}
\begin{frame}
\frametitle{Unrelated Title}


\begin{itemize}
\item value is the perceived benefits, usefulness and importance of something
\end{itemize}

\note[item]{}
\end{frame}
\begin{frame}
\frametitle{Unrelated Title}


\begin{itemize}
\item The purpose of an organization is to create value for stakeholders.
\end{itemize}

\note[item]{}
\end{frame}
\begin{frame}
\frametitle{Unrelated Title}


\begin{itemize}
\item -the nature of value-the aspects and scope of the involved stakeholders-the way that value creation is enabled through services
\end{itemize}

\note[item]{}
\end{frame}
\begin{frame}
\frametitle{Unrelated Title}


\begin{itemize}
\item -board of directors
\item -bondholders-lenders (creditors)-venture capitalists-vendors-strategic partners-trade unions-government agencies-suppliers
\end{itemize}

\note[item]{}
\end{frame}
\begin{frame}
\frametitle{Unrelated Title}


\begin{itemize}
\item customerusersponsor
\end{itemize}

\note[item]{}
\end{frame}
\begin{frame}
\frametitle{Unrelated Title}


\begin{itemize}
\item this is achieved through the provision and consumption of services.
\end{itemize}

\note[item]{}
\end{frame}
\begin{frame}
\frametitle{Unrelated Title}


\begin{itemize}
\item -organizations and people-information and technology-partners and suppliers-value streams and processes
\end{itemize}

\note[item]{}
\end{frame}
\begin{frame}
\frametitle{Unrelated Title}


\begin{itemize}
\item -political factors-economical factors-social factors-technological factors-legal factors-environmental factors
\end{itemize}

\note[item]{}
\end{frame}
\begin{frame}
\frametitle{Unrelated Title}


\begin{itemize}
\item service value system
\end{itemize}

\note[item]{}
\end{frame}
\begin{frame}
\frametitle{Unrelated Title}


\begin{itemize}
\item wasteful work, the duplication of efforts, or worse, work that conflicts with what's being done elsewhere in the organization.
\end{itemize}

\note[item]{}
\end{frame}
\begin{frame}
\frametitle{Unrelated Title}


\begin{itemize}
\item four dimensions
\end{itemize}

\note[item]{}
\end{frame}
\begin{frame}
\frametitle{Unrelated Title}


\begin{itemize}
\item organizations and people.
\end{itemize}

\note[item]{}
\end{frame}
\begin{frame}
\frametitle{Unrelated Title}


\begin{itemize}
\item organization's culture
\end{itemize}

\note[item]{}
\end{frame}
\begin{frame}
\frametitle{Unrelated Title}


\begin{itemize}
\item Organisations and People
\end{itemize}

\note[item]{}
\end{frame}
\begin{frame}
\frametitle{Unrelated Title}


\begin{itemize}
\item broad, general knowledge of the other areas of the organization, combined with a deep specialization in certain fields.
\end{itemize}

\note[item]{}
\end{frame}
\begin{frame}
\frametitle{Unrelated Title}


\begin{itemize}
\item -communication and collaboration-updating skills and competencies-broad knowledge plus deep specialization -having a common objective of facilitating value creation-understanding management and leadership styles-breaking down silos
\end{itemize}

\note[item]{}
\end{frame}
\begin{frame}
\frametitle{Unrelated Title}


\begin{itemize}
\item The second dimension of service management is information and technology.
\end{itemize}

\note[item]{}
\end{frame}
\begin{frame}
\frametitle{Unrelated Title}


\begin{itemize}
\item information and technology
\end{itemize}

\note[item]{}
\end{frame}
\begin{frame}
\frametitle{Unrelated Title}


\begin{itemize}
\item service management 
\end{itemize}

\note[item]{}
\end{frame}
\begin{frame}
\frametitle{Unrelated Title}


\begin{itemize}
\item value creation
\end{itemize}

\note[item]{}
\end{frame}
\begin{frame}
\frametitle{Unrelated Title}


\begin{itemize}
\item EnvironmentalEconomic
\item Economic – policies on currencies, inflation and a host of other factors have far-reaching consequences. For instance, an increase in the inflation rate would mean that prices of products or services might have to increase
\item Social – demographic, behavioural and attitudes need to be considered. An example might be how well a business can adapt to an ageing population
\item Technological – rapid changes in technology adoption, as well as increased automation, present opportunities (a chance to take costs out of a business) as well as challenges (customer expectations can be higher)
\item Legal – from HR and regulatory compliance to health and safety and data protection, there are a range of laws that impact on businesses and affect day-to-day operations of business
\item Environmental – climate change, pollution and waste management are among a number of factors that impact on how businesses operate
\end{itemize}

\note[item]{}
\end{frame}
\begin{frame}
\frametitle{Unrelated Title}


\begin{itemize}
\item the PESTLE
\end{itemize}

\note[item]{}
\end{frame}
\begin{frame}
\frametitle{Unrelated Title}


\begin{itemize}
\item the PESTLE
\end{itemize}

\note[item]{}
\end{frame}
\begin{frame}
\frametitle{Unrelated Title}


\begin{itemize}
\item The PESTLE framework is:
\item (P) Political(E) Economic(S) Social(T) Technological(L) Legal(E) Environmental.
\end{itemize}

\note[item]{}
\end{frame}
\begin{frame}
\frametitle{Unrelated Title}


\begin{itemize}
\item Value streams
\end{itemize}

\note[item]{}
\end{frame}
\begin{frame}
\frametitle{Unrelated Title}


\begin{itemize}
\item Timeliness
\end{itemize}

\note[item]{}
\end{frame}
\begin{frame}
\frametitle{Unrelated Title}


\begin{itemize}
\item value stream
\end{itemize}

\note[item]{}
\end{frame}
\begin{frame}
\frametitle{Unrelated Title}


\begin{itemize}
\item Processes
\end{itemize}

\note[item]{}
\end{frame}
\begin{frame}
\frametitle{Unrelated Title}


\begin{itemize}
\item organization and people
\end{itemize}

\note[item]{}
\end{frame}
\begin{frame}
\frametitle{Unrelated Title}


\begin{itemize}
\item service partnerships
\end{itemize}

\note[item]{}
\end{frame}
\begin{frame}
\frametitle{Unrelated Title}


\begin{itemize}
\item Clear separation of roles, responsibilities, and authority
\end{itemize}

\note[item]{}
\end{frame}
\begin{frame}
\frametitle{Unrelated Title}


\begin{itemize}
\item sharing common goals/risks
\end{itemize}

\note[item]{}
\end{frame}
\begin{frame}
\frametitle{Unrelated Title}


\begin{itemize}
\item collaboration to achieve desired outcomes
\end{itemize}

\note[item]{}
\end{frame}
\begin{frame}
\frametitle{Unrelated Title}


\begin{itemize}
\item formal contracts and agreements
\end{itemize}

\note[item]{}
\end{frame}
\begin{frame}
\frametitle{Unrelated Title}


\begin{itemize}
\item Roles and responsibilities
\end{itemize}

\note[item]{}
\end{frame}
\begin{frame}
\frametitle{Unrelated Title}


\begin{itemize}
\item “information and technology”
\end{itemize}

\note[item]{}
\end{frame}
\begin{frame}
\frametitle{Unrelated Title}


\begin{itemize}
\item “partners and suppliers”
\end{itemize}

\note[item]{}
\end{frame}
\begin{frame}
\frametitle{Unrelated Title}


\begin{itemize}
\item the interfaces between their specializations and roles
\end{itemize}

\note[item]{}
\end{frame}
\begin{frame}
\frametitle{Unrelated Title}


\begin{itemize}
\item skills and competencies
\end{itemize}

\note[item]{}
\end{frame}
\begin{frame}
\frametitle{Unrelated Title}


\begin{itemize}
\item service management
\end{itemize}

\note[item]{}
\end{frame}
\begin{frame}
\frametitle{Unrelated Title}


\begin{itemize}
\item Workflow management
\end{itemize}

\note[item]{}
\end{frame}
\begin{frame}
\frametitle{Unrelated Title}


\begin{itemize}
\item Blockchain services (or Blockchain-as-a-Service)
\end{itemize}

\note[item]{}
\end{frame}
\begin{frame}
\frametitle{Unrelated Title}


\begin{itemize}
\item Inventory
\end{itemize}

\note[item]{}
\end{frame}
\begin{frame}
\frametitle{Unrelated Title}


\begin{itemize}
\item service management 
\end{itemize}

\note[item]{}
\end{frame}
\begin{frame}
\frametitle{Unrelated Title}


\begin{itemize}
\item Artificial intelligence services (Artificial Intelligence-as-a-Service)
\end{itemize}

\note[item]{}
\end{frame}
\begin{frame}
\frametitle{Unrelated Title}


\begin{itemize}
\item Knowledge basesInventory systems
\end{itemize}

\note[item]{}
\end{frame}
\begin{frame}
\frametitle{Unrelated Title}


\begin{itemize}
\item Social
\end{itemize}

\note[item]{}
\end{frame}
\begin{frame}
\frametitle{Unrelated Title}


\begin{itemize}
\item Resource scarcity
\end{itemize}

\note[item]{}
\end{frame}
\begin{frame}
\frametitle{Unrelated Title}


\begin{itemize}
\item Value streams
\end{itemize}

\note[item]{}
\end{frame}
\begin{frame}
\frametitle{Unrelated Title}


\begin{itemize}
\item Demand patterns
\end{itemize}

\note[item]{}
\end{frame}
\begin{frame}
\frametitle{Unrelated Title}


\begin{itemize}
\item Political
\end{itemize}

\note[item]{}
\end{frame}
\begin{frame}
\frametitle{Unrelated Title}


\begin{itemize}
\item Availability
\end{itemize}

\note[item]{}
\end{frame}
\begin{frame}
\frametitle{Unrelated Title}


\begin{itemize}
\item Non-value-adding activities
\end{itemize}

\note[item]{}
\end{frame}
\begin{frame}
\frametitle{Unrelated Title}


\begin{itemize}
\item ITIL 4 dimensions
\end{itemize}

\note[item]{}
\end{frame}
\begin{frame}
\frametitle{Unrelated Title}


\begin{itemize}
\item Social
\end{itemize}

\note[item]{}
\end{frame}
\begin{frame}
\frametitle{Unrelated Title}


\begin{itemize}
\item -activities-practices-teams-authorities -responsibilities
\end{itemize}

\note[item]{}
\end{frame}
\begin{frame}
\frametitle{Unrelated Title}


\begin{itemize}
\item opportunity, which represents options or possibilities to add value
\end{itemize}

\note[item]{}
\end{frame}
\begin{frame}
\frametitle{Unrelated Title}


\begin{itemize}
\item The outcome of SVS is value. The SVS can enable the creation of many different types of value for a wide group of stakeholders.
\end{itemize}

\note[item]{}
\end{frame}
\begin{frame}
\frametitle{Unrelated Title}


\begin{itemize}
\item Difficulty acting quickly to take advantage of opportunities. 
\item company. 
\item increasing visibility and reducing hidden agendas. 
\item removing practices and habits that will become silos.
\end{itemize}

\note[item]{}
\end{frame}
\begin{frame}
\frametitle{Unrelated Title}


\begin{itemize}
\item According to ITIL 4, the three core types of service consumers are sponsors, customers, and users.
\end{itemize}

\note[item]{}
\end{frame}
\begin{frame}
\frametitle{Unrelated Title}


\begin{itemize}
\item outcomes
\end{itemize}

\note[item]{}
\end{frame}
\begin{frame}
\frametitle{Unrelated Title}


\begin{itemize}
\item sponsors
\end{itemize}

\note[item]{}
\end{frame}
\begin{frame}
\frametitle{Unrelated Title}


\begin{itemize}
\item users
\end{itemize}

\note[item]{}
\end{frame}
\begin{frame}
\frametitle{Unrelated Title}


\begin{itemize}
\item This is a
 set of interconnected activities that an organization performs in order
 to deliver a valuable product or service to its consumers.The service value chain is also used to facilitate value realization.
\end{itemize}

\note[item]{}
\end{frame}
\begin{frame}
\frametitle{Unrelated Title}


\begin{itemize}
\item Guiding principles
\end{itemize}

\note[item]{}
\end{frame}
\begin{frame}
\frametitle{Unrelated Title}


\begin{itemize}
\item continual improvement in the service value chain.
\end{itemize}

\note[item]{}
\end{frame}
\begin{frame}
\frametitle{Unrelated Title}


\begin{itemize}
\item ITIL Continual Improvement Model
\end{itemize}

\note[item]{}
\end{frame}
\begin{frame}
\frametitle{Unrelated Title}


\begin{itemize}
\item These practices are each designed to accomplish specific organizational goals.
\end{itemize}

\note[item]{}
\end{frame}
\begin{frame}
\frametitle{Unrelated Title}

\begin{center}
\includegraphics[width=0.9\textwidth,height=0.9\textheight,keepaspectratio]{/Users/I516998/Library/Application Support/Anki2/User 1/collection.media/paste-769df05ddd6dfce810dbcb372fad6122448e0518.png}
\end{center}

\begin{itemize}
\item Guiding Principles
\end{itemize}

\note[item]{}
\end{frame}
\begin{frame}
\frametitle{Unrelated Title}


\begin{itemize}
\item Governance
\end{itemize}

\note[item]{}
\end{frame}
\begin{frame}
\frametitle{Unrelated Title}


\begin{itemize}
\item Practices
\end{itemize}

\note[item]{}
\end{frame}
\begin{frame}
\frametitle{Unrelated Title}


\begin{itemize}
\item Value
\end{itemize}

\note[item]{}
\end{frame}
\begin{frame}
\frametitle{Unrelated Title}


\begin{itemize}
\item A value stream
\end{itemize}

\note[item]{}
\end{frame}
\begin{frame}
\frametitle{Unrelated Title}


\begin{itemize}
\item Opportunity and Demand
\end{itemize}

\note[item]{}
\end{frame}
\begin{frame}
\frametitle{Unrelated Title}


\begin{itemize}
\item Demand
\end{itemize}

\note[item]{}
\end{frame}
\begin{frame}
\frametitle{Unrelated Title}


\begin{itemize}
\item Opportunities
\end{itemize}

\note[item]{}
\end{frame}
\begin{frame}
\frametitle{Unrelated Title}


\begin{itemize}
\item Utility
\end{itemize}

\note[item]{}
\end{frame}
\begin{frame}
\frametitle{Unrelated Title}


\begin{itemize}
\item Warranty
\end{itemize}

\note[item]{}
\end{frame}
\begin{frame}
\frametitle{Unrelated Title}


\begin{itemize}
\item Silo
\end{itemize}

\note[item]{}
\end{frame}
\begin{frame}
\frametitle{Unrelated Title}


\begin{itemize}
\item DevOps is the collaboration and integration of Development (programming) and Operations (systems and/or applications).
\end{itemize}

\note[item]{}
\end{frame}
\begin{frame}
\frametitle{Unrelated Title}


\begin{itemize}
\item Silos
\end{itemize}

\note[item]{}
\end{frame}
\begin{frame}
\frametitle{Unrelated Title}


\begin{itemize}
\item Goodhart’s
 law relates to the role of measurement and states that “when a measure 
becomes a target, it ceases to be a good measure”.
\end{itemize}

\note[item]{}
\end{frame}
\begin{frame}
\frametitle{Unrelated Title}


\begin{itemize}
\item Goodhart’s
 law
\end{itemize}

\note[item]{}
\end{frame}
\begin{frame}
\frametitle{Unrelated Title}


\begin{itemize}
\item A feedback loop
\end{itemize}

\note[item]{}
\end{frame}
\begin{frame}
\frametitle{Unrelated Title}


\begin{itemize}
\item Practices
\end{itemize}

\note[item]{}
\end{frame}
\begin{frame}
\frametitle{Unrelated Title}


\begin{itemize}
\item deliver value according to ITIL
\end{itemize}

\note[item]{}
\end{frame}
\begin{frame}
\frametitle{Unrelated Title}


\begin{itemize}
\item "siloed" organizations
\end{itemize}

\note[item]{}
\end{frame}
\begin{frame}
\frametitle{Unrelated Title}


\begin{itemize}
\item activitiespracticesteamsauthoritiesresponsibilities
\end{itemize}

\note[item]{}
\end{frame}
\begin{frame}
\frametitle{Unrelated Title}


\begin{itemize}
\item service value system
\end{itemize}

\note[item]{}
\end{frame}
\begin{frame}
\frametitle{Unrelated Title}


\begin{itemize}
\item value output
\end{itemize}

\note[item]{}
\end{frame}
\begin{frame}
\frametitle{Unrelated Title}


\begin{itemize}
\item Guiding principles
\end{itemize}

\note[item]{}
\end{frame}
\begin{frame}
\frametitle{Unrelated Title}


\begin{itemize}
\item Value
\end{itemize}

\note[item]{}
\end{frame}
\begin{frame}
\frametitle{Unrelated Title}


\begin{itemize}
\item Control
\end{itemize}

\note[item]{}
\end{frame}
\begin{frame}
\frametitle{Unrelated Title}


\begin{itemize}
\item Direct the organizationControl
\end{itemize}

\note[item]{}
\end{frame}
\begin{frame}
\frametitle{Unrelated Title}


\begin{itemize}
\item ITIL SVS
\end{itemize}

\note[item]{}
\end{frame}
\begin{frame}
\frametitle{Unrelated Title}


\begin{itemize}
\item Knowledge and information, from design and transition and obtain/build. 
\end{itemize}

\note[item]{}
\end{frame}
\begin{frame}
\frametitle{Unrelated Title}


\begin{itemize}
\item Contract and agreement requirements for engage.
\end{itemize}

\note[item]{}
\end{frame}
\begin{frame}
\frametitle{Unrelated Title}


\begin{itemize}
\item Portfolio decisions provided by plan.
\end{itemize}

\note[item]{}
\end{frame}
\begin{frame}
\frametitle{Unrelated Title}


\begin{itemize}
\item Service value chain
\end{itemize}

\note[item]{}
\end{frame}
\begin{frame}
\frametitle{Unrelated Title}


\begin{itemize}
\item Design and transition
\end{itemize}

\note[item]{}
\end{frame}
\begin{frame}
\frametitle{Unrelated Title}


\begin{itemize}
\item service value chain
\end{itemize}

\note[item]{}
\end{frame}
\begin{frame}
\frametitle{Unrelated Title}


\begin{itemize}
\item Value streams
\end{itemize}

\note[item]{}
\end{frame}
\begin{frame}
\frametitle{Unrelated Title}


\begin{itemize}
\item Products and services
\end{itemize}

\note[item]{}
\end{frame}
\begin{frame}
\frametitle{Unrelated Title}


\begin{itemize}
\item Engage
\end{itemize}

\note[item]{}
\end{frame}
\begin{frame}
\frametitle{Unrelated Title}

\begin{center}
\includegraphics[width=0.9\textwidth,height=0.9\textheight,keepaspectratio]{/Users/I516998/Library/Application Support/Anki2/User 1/collection.media/The Service Value Chain.png}
\end{center}

\begin{itemize}
\item good understanding of stakeholders' needs
\end{itemize}

\note[item]{}
\end{frame}
\begin{frame}
\frametitle{Unrelated Title}


\begin{itemize}
\item EngageDesign & transition
\end{itemize}

\note[item]{}
\end{frame}
\begin{frame}
\frametitle{Unrelated Title}


\begin{itemize}
\item Engage
\end{itemize}

\note[item]{}
\end{frame}
\begin{frame}
\frametitle{Unrelated Title}


\begin{itemize}
\item EngageDesign & transition
\end{itemize}

\note[item]{}
\end{frame}
\begin{frame}
\frametitle{Unrelated Title}


\begin{itemize}
\item Deliver & support service value chain activity
\end{itemize}

\note[item]{}
\end{frame}
\begin{frame}
\frametitle{Unrelated Title}


\begin{itemize}
\item To ensure
 a shared understanding of the vision, current status, and improvement 
direction for all four dimensions and all products and services across 
the organization
\end{itemize}

\note[item]{}
\end{frame}
\begin{frame}
\frametitle{Unrelated Title}


\begin{itemize}
\item value streams
\end{itemize}

\note[item]{}
\end{frame}
\begin{frame}
\frametitle{Unrelated Title}


\begin{itemize}
\item To ensure that products and services continually meet stakeholder expectations for quality, costs, and time to market
\end{itemize}

\note[item]{}
\end{frame}
\begin{frame}
\frametitle{Unrelated Title}


\begin{itemize}
\item Deliver & support service value chain activity
\end{itemize}

\note[item]{}
\end{frame}
\begin{frame}
\frametitle{Unrelated Title}


\begin{itemize}
\item Improve service value chain activity
\end{itemize}

\note[item]{}
\end{frame}
\begin{frame}
\frametitle{Unrelated Title}


\begin{itemize}
\item Improve
\end{itemize}

\note[item]{}
\end{frame}
\begin{frame}
\frametitle{Unrelated Title}


\begin{itemize}
\item Improve
\end{itemize}

\note[item]{}
\end{frame}
\begin{frame}
\frametitle{Unrelated Title}


\begin{itemize}
\item ImproveEngage
\end{itemize}

\note[item]{}
\end{frame}
\begin{frame}
\frametitle{Unrelated Title}

\begin{center}
\includegraphics[width=0.9\textwidth,height=0.9\textheight,keepaspectratio]{/Users/I516998/Library/Application Support/Anki2/User 1/collection.media/The Service Value Chain.png}
\end{center}

\begin{itemize}
\item Engage
\item Improve
\end{itemize}

\note[item]{}
\end{frame}
\begin{frame}
\frametitle{Unrelated Title}


\begin{itemize}
\item Plan
\end{itemize}

\note[item]{}
\end{frame}
\begin{frame}
\frametitle{Unrelated Title}

\begin{center}
\includegraphics[width=0.9\textwidth,height=0.9\textheight,keepaspectratio]{/Users/I516998/Library/Application Support/Anki2/User 1/collection.media/The Service Value Chain.png}
\end{center}

\begin{itemize}
\item Plan
\end{itemize}

\note[item]{}
\end{frame}
\begin{frame}
\frametitle{Unrelated Title}

\begin{center}
\includegraphics[width=0.9\textwidth,height=0.9\textheight,keepaspectratio]{/Users/I516998/Library/Application Support/Anki2/User 1/collection.media/The Service Value Chain.png}
\end{center}

\begin{itemize}
\item EngagePlan
\end{itemize}

\note[item]{}
\end{frame}
\begin{frame}
\frametitle{Unrelated Title}


\begin{itemize}
\item Engage
\end{itemize}

\note[item]{}
\end{frame}
\begin{frame}
\frametitle{Unrelated Title}

\begin{center}
\includegraphics[width=0.9\textwidth,height=0.9\textheight,keepaspectratio]{/Users/I516998/Library/Application Support/Anki2/User 1/collection.media/The Service Value Chain.png}
\end{center}

\begin{itemize}
\item continual improvement
\end{itemize}

\note[item]{}
\end{frame}
\begin{frame}
\frametitle{Unrelated Title}

\begin{center}
\includegraphics[width=0.9\textwidth,height=0.9\textheight,keepaspectratio]{/Users/I516998/Library/Application Support/Anki2/User 1/collection.media/paste-89ee87fd4c57b7a8115fb84632b0fba9b653a683.png}
\end{center}

\begin{itemize}
\item value stream
\end{itemize}

\note[item]{}
\end{frame}
\begin{frame}
\frametitle{Unrelated Title}

\begin{center}
\includegraphics[width=0.9\textwidth,height=0.9\textheight,keepaspectratio]{/Users/I516998/Library/Application Support/Anki2/User 1/collection.media/ValueStreamMapParts.png}
\end{center}

\begin{itemize}
\item value stream
\end{itemize}

\note[item]{}
\end{frame}
\begin{frame}
\frametitle{Unrelated Title}


\begin{itemize}
\item Value streams 
\end{itemize}

\note[item]{}
\end{frame}
\begin{frame}
\frametitle{Unrelated Title}


\begin{itemize}
\item delivery and performance
\end{itemize}

\note[item]{}
\end{frame}
\begin{frame}
\frametitle{Unrelated Title}

\begin{center}
\includegraphics[width=0.9\textwidth,height=0.9\textheight,keepaspectratio]{/Users/I516998/Library/Application Support/Anki2/User 1/collection.media/The Service Value Chain.png}
\end{center}

\begin{itemize}
\item service value chain
\end{itemize}

\note[item]{}
\end{frame}
\begin{frame}
\frametitle{Unrelated Title}


\begin{itemize}
\item service value chain
\end{itemize}

\note[item]{}
\end{frame}
\begin{frame}
\frametitle{Unrelated Title}

\begin{center}
\includegraphics[width=0.9\textwidth,height=0.9\textheight,keepaspectratio]{/Users/I516998/Library/Application Support/Anki2/User 1/collection.media/The Service Value Chain.png}
\end{center}

\begin{itemize}
\item Obtain/build
\end{itemize}

\note[item]{}
\end{frame}
\begin{frame}
\frametitle{Unrelated Title}


\begin{itemize}
\item To ensure that products and services 
continually meet stakeholder expectations for quality, costs, and time to market
\end{itemize}

\note[item]{}
\end{frame}
\begin{frame}
\frametitle{Unrelated Title}


\begin{itemize}
\item Deliver & support
\end{itemize}

\note[item]{}
\end{frame}
\begin{frame}
\frametitle{Unrelated Title}


\begin{itemize}
\item To assist
 in carrying out specific combinations of scenario-based activities and 
practices to better perform certain tasks or respond to situations
\end{itemize}

\note[item]{}
\end{frame}
\begin{frame}
\frametitle{Unrelated Title}


\begin{itemize}
\item Obtain/build
\end{itemize}

\note[item]{}
\end{frame}
\begin{frame}
\frametitle{Unrelated Title}


\begin{itemize}
\item Design & transition
\end{itemize}

\note[item]{}
\end{frame}
\begin{frame}
\frametitle{Unrelated Title}


\begin{itemize}
\item To ensure that products and services 
continually meet stakeholder expectations for quality, costs, and time to market
\end{itemize}

\note[item]{}
\end{frame}
\begin{frame}
\frametitle{Unrelated Title}


\begin{itemize}
\item To provide a good understanding of stakeholder needs,
continual engagement with all stakeholders, transparency, and good relationships with all stakeholders
\end{itemize}

\note[item]{}
\end{frame}
\begin{frame}
\frametitle{Unrelated Title}


\begin{itemize}
\item Plan
\end{itemize}

\note[item]{}
\end{frame}
\begin{frame}
\frametitle{Unrelated Title}


\begin{itemize}
\item To ensure continual improvement of products, services, 
and practices across all value chain activities and the four dimensions of service management
\end{itemize}

\note[item]{}
\end{frame}
\begin{frame}
\frametitle{Unrelated Title}


\begin{itemize}
\item Design and transition
\end{itemize}

\note[item]{}
\end{frame}
\begin{frame}
\frametitle{Unrelated Title}


\begin{itemize}
\item To ensure continual improvement of products, services, and practices across the four dimensions of service management
\end{itemize}

\note[item]{}
\end{frame}
\begin{frame}
\frametitle{Unrelated Title}


\begin{itemize}
\item In the value chain activity engage, the purpose is to provide a good understanding of stakeholder needs, continual engagement with all stakeholders, transparency, and good relationships with all stakeholders.
\end{itemize}

\note[item]{}
\end{frame}
\begin{frame}
\frametitle{Unrelated Title}


\begin{itemize}
\item Engage
\end{itemize}

\note[item]{}
\end{frame}
\begin{frame}
\frametitle{Unrelated Title}


\begin{itemize}
\item To ensure a shared understanding of the vision, current status, and improvement direction for all four dimensions
\end{itemize}

\note[item]{}
\end{frame}
\begin{frame}
\frametitle{Unrelated Title}


\begin{itemize}
\item Deliver & Support
\end{itemize}

\note[item]{}
\end{frame}
\begin{frame}
\frametitle{Unrelated Title}


\begin{itemize}
\item value chains
\end{itemize}

\note[item]{}
\end{frame}
\begin{frame}
\frametitle{Unrelated Title}

\begin{center}
\includegraphics[width=0.9\textwidth,height=0.9\textheight,keepaspectratio]{/Users/I516998/Library/Application Support/Anki2/User 1/collection.media/The Service Value Chain.png}
\end{center}

\begin{itemize}
\item Obtain/build
\end{itemize}

\note[item]{}
\end{frame}
\begin{frame}
\frametitle{Unrelated Title}

\begin{center}
\includegraphics[width=0.9\textwidth,height=0.9\textheight,keepaspectratio]{/Users/I516998/Library/Application Support/Anki2/User 1/collection.media/ITILv4 SVC.png}
\end{center}

\begin{itemize}
\item Obtain/build
\end{itemize}

\note[item]{}
\end{frame}
\begin{frame}
\frametitle{Unrelated Title}


\begin{itemize}
\item To ensure that products and services continually meet stakeholder expectations for quality and costs
\end{itemize}

\note[item]{}
\end{frame}
\begin{frame}
\frametitle{Unrelated Title}


\begin{itemize}
\item Engage
\end{itemize}

\note[item]{}
\end{frame}
\begin{frame}
\frametitle{Unrelated Title}


\begin{itemize}
\item Service Value Chain
\end{itemize}

\note[item]{}
\end{frame}
\begin{frame}
\frametitle{Unrelated Title}


\begin{itemize}
\item value streams
\end{itemize}

\note[item]{}
\end{frame}
\begin{frame}
\frametitle{Unrelated Title}


\begin{itemize}
\item Improve
\end{itemize}

\note[item]{}
\end{frame}
\begin{frame}
\frametitle{Unrelated Title}


\begin{itemize}
\item plan
\end{itemize}

\note[item]{}
\end{frame}
\begin{frame}
\frametitle{Unrelated Title}


\begin{itemize}
\item ITIL 4 Service Value Chain
\end{itemize}

\note[item]{}
\end{frame}
\begin{frame}
\frametitle{Unrelated Title}


\begin{itemize}
\item Deliver and support
\end{itemize}

\note[item]{}
\end{frame}
\begin{frame}
\frametitle{Unrelated Title}


\begin{itemize}
\item Holistic
\end{itemize}

\note[item]{}
\end{frame}
\begin{frame}
\frametitle{Unrelated Title}


\begin{itemize}
\item Iterative
\end{itemize}

\note[item]{}
\end{frame}
\begin{frame}
\frametitle{Unrelated Title}

\begin{center}
\includegraphics[width=0.9\textwidth,height=0.9\textheight,keepaspectratio]{/Users/I516998/Library/Application Support/Anki2/User 1/collection.media/7 Guiding Principales of ITIL 4.png}
\end{center}

\begin{itemize}
\item Automated
\end{itemize}

\note[item]{}
\end{frame}
\begin{frame}
\frametitle{Unrelated Title}


\begin{itemize}
\item Holistic
\end{itemize}

\note[item]{}
\end{frame}
\begin{frame}
\frametitle{Unrelated Title}


\begin{itemize}
\item Understand and identify the service consumer
\end{itemize}

\note[item]{}
\end{frame}
\begin{frame}
\frametitle{Unrelated Title}


\begin{itemize}
\item Understand the consumer’s perspective of value
\end{itemize}

\note[item]{}
\end{frame}
\begin{frame}
\frametitle{Unrelated Title}


\begin{itemize}
\item The customer experience (CX)
\end{itemize}

\note[item]{}
\end{frame}
\begin{frame}
\frametitle{Unrelated Title}


\begin{itemize}
\item Map value to the intended outcomes
\end{itemize}

\note[item]{}
\end{frame}
\begin{frame}
\frametitle{Unrelated Title}


\begin{itemize}
\item is being served (the service consumer)
\end{itemize}

\note[item]{}
\end{frame}
\begin{frame}
\frametitle{Unrelated Title}


\begin{itemize}
\item Optimize
\end{itemize}

\note[item]{}
\end{frame}
\begin{frame}
\frametitle{Unrelated Title}


\begin{itemize}
\item Automation
\end{itemize}

\note[item]{}
\end{frame}
\begin{frame}
\frametitle{Unrelated Title}


\begin{itemize}
\item measurement
\end{itemize}

\note[item]{}
\end{frame}
\begin{frame}
\frametitle{Unrelated Title}

\begin{center}
\includegraphics[width=0.9\textwidth,height=0.9\textheight,keepaspectratio]{/Users/I516998/Library/Application Support/Anki2/User 1/collection.media/7 Guiding Principales of ITIL 4.png}
\end{center}

\begin{itemize}
\item Collaborate
\end{itemize}

\note[item]{}
\end{frame}
\begin{frame}
\frametitle{Unrelated Title}


\begin{itemize}
\item optimized
\end{itemize}

\note[item]{}
\end{frame}
\begin{frame}
\frametitle{Unrelated Title}


\begin{itemize}
\item Feedback loop
\end{itemize}

\note[item]{}
\end{frame}
\begin{frame}
\frametitle{Unrelated Title}


\begin{itemize}
\item timebox
\end{itemize}

\note[item]{}
\end{frame}
\begin{frame}
\frametitle{Unrelated Title}


\begin{itemize}
\item A feedback loop
\end{itemize}

\note[item]{}
\end{frame}
\begin{frame}
\frametitle{Unrelated Title}


\begin{itemize}
\item A service value chain
\end{itemize}

\note[item]{}
\end{frame}
\begin{frame}
\frametitle{Unrelated Title}


\begin{itemize}
\item ecosystem
\end{itemize}

\note[item]{}
\end{frame}
\begin{frame}
\frametitle{Unrelated Title}

\begin{center}
\includegraphics[width=0.9\textwidth,height=0.9\textheight,keepaspectratio]{/Users/I516998/Library/Application Support/Anki2/User 1/collection.media/7 Guiding Principales of ITIL 4.png}
\end{center}

\begin{itemize}
\item UniversalEnduring
\end{itemize}

\note[item]{}
\end{frame}
\begin{frame}
\frametitle{Unrelated Title}


\begin{itemize}
\item universal and enduring 
\end{itemize}

\note[item]{}
\end{frame}
\begin{frame}
\frametitle{Unrelated Title}

\begin{center}
\includegraphics[width=0.9\textwidth,height=0.9\textheight,keepaspectratio]{/Users/I516998/Library/Application Support/Anki2/User 1/collection.media/7 Guiding Principales of ITIL 4.png}
\end{center}

\begin{itemize}
\item Keep it simple and practical
\end{itemize}

\note[item]{}
\end{frame}
\begin{frame}
\frametitle{Unrelated Title}

\begin{center}
\includegraphics[width=0.9\textwidth,height=0.9\textheight,keepaspectratio]{/Users/I516998/Library/Application Support/Anki2/User 1/collection.media/7 Guiding Principales of ITIL 4 (1).png}
\end{center}

\begin{itemize}
\item Collaborate and promote visibility
\end{itemize}

\note[item]{}
\end{frame}
\begin{frame}
\frametitle{Unrelated Title}

\begin{center}
\includegraphics[width=0.9\textwidth,height=0.9\textheight,keepaspectratio]{/Users/I516998/Library/Application Support/Anki2/User 1/collection.media/7 Guiding Principales of ITIL 4.png}
\end{center}

\begin{itemize}
\item Keep it simple and practical
\end{itemize}

\note[item]{}
\end{frame}
\begin{frame}
\frametitle{Unrelated Title}

\begin{center}
\includegraphics[width=0.9\textwidth,height=0.9\textheight,keepaspectratio]{/Users/I516998/Library/Application Support/Anki2/User 1/collection.media/7 Guiding Principales of ITIL 4.png}
\end{center}

\begin{itemize}
\item Think and work holistically
\end{itemize}

\note[item]{}
\end{frame}
\begin{frame}
\frametitle{Unrelated Title}

\begin{center}
\includegraphics[width=0.9\textwidth,height=0.9\textheight,keepaspectratio]{/Users/I516998/Library/Application Support/Anki2/User 1/collection.media/7 Guiding Principales of ITIL 4 (1).png}
\end{center}

\begin{itemize}
\item Progress iteratively with feedback
\end{itemize}

\note[item]{}
\end{frame}
\begin{frame}
\frametitle{Unrelated Title}

\begin{center}
\includegraphics[width=0.9\textwidth,height=0.9\textheight,keepaspectratio]{/Users/I516998/Library/Application Support/Anki2/User 1/collection.media/7 Guiding Principales of ITIL 4.png}
\end{center}

\begin{itemize}
\item collaborate and promote visibility
\end{itemize}

\note[item]{}
\end{frame}
\begin{frame}
\frametitle{Unrelated Title}


\begin{itemize}
\item service level and operation level
\end{itemize}

\note[item]{}
\end{frame}
\begin{frame}
\frametitle{Unrelated Title}


\begin{itemize}
\item the guiding principle of collaborate and promote visibility
\end{itemize}

\note[item]{}
\end{frame}
\begin{frame}
\frametitle{Unrelated Title}


\begin{itemize}
\item collaborate and promote visibility
\end{itemize}

\note[item]{}
\end{frame}
\begin{frame}
\frametitle{Unrelated Title}


\begin{itemize}
\item consensus
\end{itemize}

\note[item]{}
\end{frame}
\begin{frame}
\frametitle{Unrelated Title}


\begin{itemize}
\item start where you are
\end{itemize}

\note[item]{}
\end{frame}
\begin{frame}
\frametitle{Unrelated Title}

\begin{center}
\includegraphics[width=0.9\textwidth,height=0.9\textheight,keepaspectratio]{/Users/I516998/Library/Application Support/Anki2/User 1/collection.media/7 Guiding Principales of ITIL 4.png}
\end{center}

\begin{itemize}
\item start where you are
\end{itemize}

\note[item]{}
\end{frame}
\begin{frame}
\frametitle{Unrelated Title}


\begin{itemize}
\item reporting and reality
\end{itemize}

\note[item]{}
\end{frame}
\begin{frame}
\frametitle{Unrelated Title}


\begin{itemize}
\item unbiased
\end{itemize}

\note[item]{}
\end{frame}
\begin{frame}
\frametitle{Unrelated Title}


\begin{itemize}
\item rarely necessary and often wasteful
\end{itemize}

\note[item]{}
\end{frame}
\begin{frame}
\frametitle{Unrelated Title}


\begin{itemize}
\item Keep it simple and practical
\end{itemize}

\note[item]{}
\end{frame}
\begin{frame}
\frametitle{Unrelated Title}


\begin{itemize}
\item Collaborate and promote visibility
\end{itemize}

\note[item]{}
\end{frame}
\begin{frame}
\frametitle{Unrelated Title}

\begin{center}
\includegraphics[width=0.9\textwidth,height=0.9\textheight,keepaspectratio]{/Users/I516998/Library/Application Support/Anki2/User 1/collection.media/7 Guiding Principales of ITIL 4.png}
\end{center}

\begin{itemize}
\item collaboration
\end{itemize}

\note[item]{}
\end{frame}
\begin{frame}
\frametitle{Unrelated Title}


\begin{itemize}
\item Real accomplishmentUnderstanding
\end{itemize}

\note[item]{}
\end{frame}
\begin{frame}
\frametitle{Unrelated Title}


\begin{itemize}
\item focus on value
\end{itemize}

\note[item]{}
\end{frame}
\begin{frame}
\frametitle{Unrelated Title}


\begin{itemize}
\item Recognize that sometimes nothing from the current state is reusable
\end{itemize}

\note[item]{}
\end{frame}
\begin{frame}
\frametitle{Unrelated Title}


\begin{itemize}
\item start where you are
\end{itemize}

\note[item]{}
\end{frame}
\begin{frame}
\frametitle{Unrelated Title}


\begin{itemize}
\item Progress iteratively with feedback
\end{itemize}

\note[item]{}
\end{frame}
\begin{frame}
\frametitle{Unrelated Title}


\begin{itemize}
\item Understand the customer experience (CX)
\end{itemize}

\note[item]{}
\end{frame}
\begin{frame}
\frametitle{Unrelated Title}

\begin{center}
\includegraphics[width=0.9\textwidth,height=0.9\textheight,keepaspectratio]{/Users/I516998/Library/Application Support/Anki2/User 1/collection.media/7 Guiding Principales of ITIL 4.png}
\end{center}

\begin{itemize}
\item Start where you are
\end{itemize}

\note[item]{}
\end{frame}
\begin{frame}
\frametitle{Unrelated Title}


\begin{itemize}
\item concerning the guiding principle of think and work holistically
\end{itemize}

\note[item]{}
\end{frame}
\begin{frame}
\frametitle{Unrelated Title}


\begin{itemize}
\item Feedback loop
\end{itemize}

\note[item]{}
\end{frame}
\begin{frame}
\frametitle{Unrelated Title}


\begin{itemize}
\item what to keep
\end{itemize}

\note[item]{}
\end{frame}
\begin{frame}
\frametitle{Unrelated Title}


\begin{itemize}
\item Look for patterns and interactions between elements
\end{itemize}

\note[item]{}
\end{frame}
\begin{frame}
\frametitle{Unrelated Title}

\begin{center}
\includegraphics[width=0.9\textwidth,height=0.9\textheight,keepaspectratio]{/Users/I516998/Library/Application Support/Anki2/User 1/collection.media/7 Guiding Principales of ITIL 4.png}
\end{center}

\begin{itemize}
\item think and work holistically
\end{itemize}

\note[item]{}
\end{frame}
\begin{frame}
\frametitle{Unrelated Title}


\begin{itemize}
\item Guiding principle
\end{itemize}

\note[item]{}
\end{frame}
\begin{frame}
\frametitle{Unrelated Title}


\begin{itemize}
\item automation
\end{itemize}

\note[item]{}
\end{frame}
\begin{frame}
\frametitle{Unrelated Title}


\begin{itemize}
\item -Focus on Value
\item -Start Where You Are
\item -Progress Iteratively with Feedback
\item -Collaborate and Promote Visibility
\item -Think and Work Holistically
\item -Keep it Simple and Practical
\item -Optimize and Automate
\end{itemize}

\note[item]{}
\end{frame}
\begin{frame}
\frametitle{Unrelated Title}


\begin{itemize}
\item Automation
\end{itemize}

\note[item]{}
\end{frame}
\begin{frame}
\frametitle{Unrelated Title}


\begin{itemize}
\item focus on valuestart where you
\item progress iteratively with feedbackcollaborate and promote visibilitythink and work holisticallykeep it simple and practicaloptimize and automate
\end{itemize}

\note[item]{}
\end{frame}
\begin{frame}
\frametitle{Unrelated Title}


\begin{itemize}
\item Do not be afraid to ask questionsObserve and measure existing services and methods
\end{itemize}

\note[item]{}
\end{frame}
\begin{frame}
\frametitle{Unrelated Title}


\begin{itemize}
\item Look at what exists as objectively as possible. 
\end{itemize}

\note[item]{}
\end{frame}
\begin{frame}
\frametitle{Unrelated Title}


\begin{itemize}
\item CIR (Continual Improvement Register)
\end{itemize}

\note[item]{}
\end{frame}
\begin{frame}
\frametitle{Unrelated Title}


\begin{itemize}
\item CX
\end{itemize}

\note[item]{}
\end{frame}
\begin{frame}
\frametitle{Unrelated Title}


\begin{itemize}
\item A service level agreement (SLA)
\end{itemize}

\note[item]{}
\end{frame}
\begin{frame}
\frametitle{Unrelated Title}


\begin{itemize}
\item The CIR is the Continual Improvement Register. 
\end{itemize}

\note[item]{}
\end{frame}
\begin{frame}
\frametitle{Unrelated Title}


\begin{itemize}
\item The SVS is the service value system. 
\end{itemize}

\note[item]{}
\end{frame}
\begin{frame}
\frametitle{Unrelated Title}


\begin{itemize}
\item Practice
\end{itemize}

\note[item]{}
\end{frame}
\begin{frame}
\frametitle{Unrelated Title}


\begin{itemize}
\item A 
registerThe continual improvement
 register
\end{itemize}

\note[item]{}
\end{frame}
\begin{frame}
\frametitle{Unrelated Title}


\begin{itemize}
\item Practices
\end{itemize}

\note[item]{}
\end{frame}
\begin{frame}
\frametitle{Unrelated Title}


\begin{itemize}
\item A value stream
\end{itemize}

\note[item]{}
\end{frame}
\begin{frame}
\frametitle{Unrelated Title}


\begin{itemize}
\item A process
\end{itemize}

\note[item]{}
\end{frame}
\begin{frame}
\frametitle{Unrelated Title}


\begin{itemize}
\item ConfidentialityIntegrity
\end{itemize}

\note[item]{}
\end{frame}
\begin{frame}
\frametitle{Unrelated Title}


\begin{itemize}
\item Outsourcing
\end{itemize}

\note[item]{}
\end{frame}
\begin{frame}
\frametitle{Unrelated Title}


\begin{itemize}
\item Maintainability
\end{itemize}

\note[item]{}
\end{frame}
\begin{frame}
\frametitle{Unrelated Title}


\begin{itemize}
\item Confidentiality
\end{itemize}

\note[item]{}
\end{frame}
\begin{frame}
\frametitle{Unrelated Title}


\begin{itemize}
\item Integrity
\end{itemize}

\note[item]{}
\end{frame}
\begin{frame}
\frametitle{Unrelated Title}


\begin{itemize}
\item Single sourcing
\end{itemize}

\note[item]{}
\end{frame}
\begin{frame}
\frametitle{Unrelated Title}


\begin{itemize}
\item Insourcing
\end{itemize}

\note[item]{}
\end{frame}
\begin{frame}
\frametitle{Unrelated Title}


\begin{itemize}
\item Multi-sourcing
\end{itemize}

\note[item]{}
\end{frame}
\begin{frame}
\frametitle{Unrelated Title}


\begin{itemize}
\item Single sourcing
\end{itemize}

\note[item]{}
\end{frame}
\begin{frame}
\frametitle{Unrelated Title}


\begin{itemize}
\item Outsourcing
\end{itemize}

\note[item]{}
\end{frame}
\begin{frame}
\frametitle{Unrelated Title}


\begin{itemize}
\item Continual improvementInformation security managementRelationship managementSupplier Management
\end{itemize}

\note[item]{}
\end{frame}
\begin{frame}
\frametitle{Unrelated Title}


\begin{itemize}
\item Portfolio management
\end{itemize}

\note[item]{}
\end{frame}
\begin{frame}
\frametitle{Unrelated Title}


\begin{itemize}
\item Supplier management
\end{itemize}

\note[item]{}
\end{frame}
\begin{frame}
\frametitle{Unrelated Title}


\begin{itemize}
\item Information security management
\end{itemize}

\note[item]{}
\end{frame}
\begin{frame}
\frametitle{Unrelated Title}


\begin{itemize}
\item Relationship management
\end{itemize}

\note[item]{}
\end{frame}
\begin{frame}
\frametitle{Unrelated Title}


\begin{itemize}
\item Availability management
\end{itemize}

\note[item]{}
\end{frame}
\begin{frame}
\frametitle{Unrelated Title}


\begin{itemize}
\item the relationship management practice
\end{itemize}

\note[item]{}
\end{frame}
\begin{frame}
\frametitle{Unrelated Title}


\begin{itemize}
\item a “practice”
\end{itemize}

\note[item]{}
\end{frame}
\begin{frame}
\frametitle{Unrelated Title}


\begin{itemize}
\item This is a goal of the supplier management practice
\end{itemize}

\note[item]{}
\end{frame}
\begin{frame}
\frametitle{Unrelated Title}


\begin{itemize}
\item relationship management practice
\end{itemize}

\note[item]{}
\end{frame}
\begin{frame}
\frametitle{Unrelated Title}


\begin{itemize}
\item continual improvement
\end{itemize}

\note[item]{}
\end{frame}
\begin{frame}
\frametitle{Unrelated Title}


\begin{itemize}
\item supplier sourcing model
\end{itemize}

\note[item]{}
\end{frame}
\begin{frame}
\frametitle{Unrelated Title}


\begin{itemize}
\item Plan
\end{itemize}

\note[item]{}
\end{frame}
\begin{frame}
\frametitle{Unrelated Title}


\begin{itemize}
\item 14
\end{itemize}

\note[item]{}
\end{frame}
\begin{frame}
\frametitle{Unrelated Title}


\begin{itemize}
\item Information security management
\end{itemize}

\note[item]{}
\end{frame}
\begin{frame}
\frametitle{Unrelated Title}


\begin{itemize}
\item information security management
\end{itemize}

\note[item]{}
\end{frame}
\begin{frame}
\frametitle{Unrelated Title}


\begin{itemize}
\item Improve
\end{itemize}

\note[item]{}
\end{frame}
\begin{frame}
\frametitle{Unrelated Title}


\begin{itemize}
\item 17
\end{itemize}

\note[item]{}
\end{frame}
\begin{frame}
\frametitle{Unrelated Title}


\begin{itemize}
\item Supplier management
\end{itemize}

\note[item]{}
\end{frame}
\begin{frame}
\frametitle{Unrelated Title}


\begin{itemize}
\item general management
\end{itemize}

\note[item]{}
\end{frame}
\begin{frame}
\frametitle{Unrelated Title}


\begin{itemize}
\item Integrity 
\end{itemize}

\note[item]{}
\end{frame}
\begin{frame}
\frametitle{Unrelated Title}


\begin{itemize}
\item relationship management
\end{itemize}

\note[item]{}
\end{frame}
\begin{frame}
\frametitle{Unrelated Title}


\begin{itemize}
\item relationship management
\end{itemize}

\note[item]{}
\end{frame}
\begin{frame}
\frametitle{Unrelated Title}


\begin{itemize}
\item To ensure services deliver agreed levels of availability to meet the needs of customers and users.
\end{itemize}

\note[item]{}
\end{frame}
\begin{frame}
\frametitle{Unrelated Title}


\begin{itemize}
\item availability
\end{itemize}

\note[item]{}
\end{frame}
\begin{frame}
\frametitle{Unrelated Title}


\begin{itemize}
\item Negotiating and agreeing achievable targets for availability
\end{itemize}

\note[item]{}
\end{frame}
\begin{frame}
\frametitle{Unrelated Title}


\begin{itemize}
\item Mean time between failures (MTBF)
\item User outage minutesNumber of lost transactions
\end{itemize}

\note[item]{}
\end{frame}
\begin{frame}
\frametitle{Unrelated Title}


\begin{itemize}
\item Normal changesStandard changesEmergency changes
\end{itemize}

\note[item]{}
\end{frame}
\begin{frame}
\frametitle{Unrelated Title}


\begin{itemize}
\item Maximize valueControl costsManage risksSupport decision-makingComply with regulations
\end{itemize}

\note[item]{}
\end{frame}
\begin{frame}
\frametitle{Unrelated Title}


\begin{itemize}
\item To ensure that services deliver agreed levels of accessibility to meet the needs of customers and users
\end{itemize}

\note[item]{}
\end{frame}
\begin{frame}
\frametitle{Unrelated Title}


\begin{itemize}
\item To maximize the number of IT changes by ensuring that risks have been properly assessed
\end{itemize}

\note[item]{}
\end{frame}
\begin{frame}
\frametitle{Unrelated Title}


\begin{itemize}
\item This is the definition of capacity and performance management.
\end{itemize}

\note[item]{}
\end{frame}
\begin{frame}
\frametitle{Unrelated Title}


\begin{itemize}
\item This is the definition of incident management.
\end{itemize}

\note[item]{}
\end{frame}
\begin{frame}
\frametitle{Unrelated Title}


\begin{itemize}
\item Availability
\end{itemize}

\note[item]{}
\end{frame}
\begin{frame}
\frametitle{Unrelated Title}


\begin{itemize}
\item To reduce
 the likelihood and impact of incidents by identifying actual and 
potential causes and managing workarounds and known errors
\end{itemize}

\note[item]{}
\end{frame}
\begin{frame}
\frametitle{Unrelated Title}


\begin{itemize}
\item To systematically observe services and service components, and record and report selected changes in events
\end{itemize}

\note[item]{}
\end{frame}
\begin{frame}
\frametitle{Unrelated Title}


\begin{itemize}
\item This is the definition of availability management.
\end{itemize}

\note[item]{}
\end{frame}
\begin{frame}
\frametitle{Unrelated Title}


\begin{itemize}
\item To plan and manage the full lifecycle of IT assets
\end{itemize}

\note[item]{}
\end{frame}
\begin{frame}
\frametitle{Unrelated Title}


\begin{itemize}
\item problem management
\end{itemize}

\note[item]{}
\end{frame}
\begin{frame}
\frametitle{Unrelated Title}


\begin{itemize}
\item Service capacity
\end{itemize}

\note[item]{}
\end{frame}
\begin{frame}
\frametitle{Unrelated Title}


\begin{itemize}
\item Service capacity
\end{itemize}

\note[item]{}
\end{frame}
\begin{frame}
\frametitle{Unrelated Title}


\begin{itemize}
\item Service performance 
\end{itemize}

\note[item]{}
\end{frame}
\begin{frame}
\frametitle{Unrelated Title}


\begin{itemize}
\item Capacity planning
\end{itemize}

\note[item]{}
\end{frame}
\begin{frame}
\frametitle{Unrelated Title}


\begin{itemize}
\item Performance
\end{itemize}

\note[item]{}
\end{frame}
\begin{frame}
\frametitle{Unrelated Title}


\begin{itemize}
\item Swarming
\end{itemize}

\note[item]{}
\end{frame}
\begin{frame}
\frametitle{Unrelated Title}


\begin{itemize}
\item collective analysis
\end{itemize}

\note[item]{}
\end{frame}
\begin{frame}
\frametitle{Unrelated Title}


\begin{itemize}
\item Mean time between failures (MTBF) 
\end{itemize}

\note[item]{}
\end{frame}
\begin{frame}
\frametitle{Unrelated Title}


\begin{itemize}
\item Escalation
\end{itemize}

\note[item]{}
\end{frame}
\begin{frame}
\frametitle{Unrelated Title}


\begin{itemize}
\item Mean time to restore service (MTRS)
\end{itemize}

\note[item]{}
\end{frame}
\begin{frame}
\frametitle{Unrelated Title}


\begin{itemize}
\item monitoring and event management
\end{itemize}

\note[item]{}
\end{frame}
\begin{frame}
\frametitle{Unrelated Title}


\begin{itemize}
\item monitoring and event management
\end{itemize}

\note[item]{}
\end{frame}
\begin{frame}
\frametitle{Unrelated Title}


\begin{itemize}
\item Automation
\end{itemize}

\note[item]{}
\end{frame}
\begin{frame}
\frametitle{Unrelated Title}


\begin{itemize}
\item This is an availability management activity.
\end{itemize}

\note[item]{}
\end{frame}
\begin{frame}
\frametitle{Unrelated Title}


\begin{itemize}
\item availability
\end{itemize}

\note[item]{}
\end{frame}
\begin{frame}
\frametitle{Unrelated Title}


\begin{itemize}
\item managed
\end{itemize}

\note[item]{}
\end{frame}
\begin{frame}
\frametitle{Unrelated Title}


\begin{itemize}
\item A listed change
\end{itemize}

\note[item]{}
\end{frame}
\begin{frame}
\frametitle{Unrelated Title}


\begin{itemize}
\item Standard change
\end{itemize}

\note[item]{}
\end{frame}
\begin{frame}
\frametitle{Unrelated Title}


\begin{itemize}
\item Normal change
\end{itemize}

\note[item]{}
\end{frame}
\begin{frame}
\frametitle{Unrelated Title}


\begin{itemize}
\item Emergency change
\end{itemize}

\note[item]{}
\end{frame}
\begin{frame}
\frametitle{Unrelated Title}


\begin{itemize}
\item asset management
\end{itemize}

\note[item]{}
\end{frame}
\begin{frame}
\frametitle{Unrelated Title}


\begin{itemize}
\item Asset management
\end{itemize}

\note[item]{}
\end{frame}
\begin{frame}
\frametitle{Unrelated Title}


\begin{itemize}
\item Software asset management (SAM)
\end{itemize}

\note[item]{}
\end{frame}
\begin{frame}
\frametitle{Unrelated Title}


\begin{itemize}
\item IT asset management (ITAM)
\end{itemize}

\note[item]{}
\end{frame}
\begin{frame}
\frametitle{Unrelated Title}


\begin{itemize}
\item change control management practice
\end{itemize}

\note[item]{}
\end{frame}
\begin{frame}
\frametitle{Unrelated Title}


\begin{itemize}
\item Event
\end{itemize}

\note[item]{}
\end{frame}
\begin{frame}
\frametitle{Unrelated Title}


\begin{itemize}
\item Capacity and performance management
\end{itemize}

\note[item]{}
\end{frame}
\begin{frame}
\frametitle{Unrelated Title}


\begin{itemize}
\item Assists with improvement planningAssists in communicationHelps avoid conflictsHelps with incidents and problems
\end{itemize}

\note[item]{}
\end{frame}
\begin{frame}
\frametitle{Unrelated Title}


\begin{itemize}
\item Known error
\end{itemize}

\note[item]{}
\end{frame}
\begin{frame}
\frametitle{Unrelated Title}


\begin{itemize}
\item incident management
\end{itemize}

\note[item]{}
\end{frame}
\begin{frame}
\frametitle{Unrelated Title}


\begin{itemize}
\item monitoring and event management
\end{itemize}

\note[item]{}
\end{frame}
\begin{frame}
\frametitle{Unrelated Title}


\begin{itemize}
\item asset management
\end{itemize}

\note[item]{}
\end{frame}
\begin{frame}
\frametitle{Unrelated Title}


\begin{itemize}
\item Asset register
\end{itemize}

\note[item]{}
\end{frame}
\begin{frame}
\frametitle{Unrelated Title}


\begin{itemize}
\item incident management
\end{itemize}

\note[item]{}
\end{frame}
\begin{frame}
\frametitle{Unrelated Title}


\begin{itemize}
\item MTBF
\end{itemize}

\note[item]{}
\end{frame}
\begin{frame}
\frametitle{Unrelated Title}


\begin{itemize}
\item MTRS
\end{itemize}

\note[item]{}
\end{frame}
\begin{frame}
\frametitle{Unrelated Title}


\begin{itemize}
\item problem management
\end{itemize}

\note[item]{}
\end{frame}
\begin{frame}
\frametitle{Unrelated Title}


\begin{itemize}
\item Service performance
\end{itemize}

\note[item]{}
\end{frame}
\begin{frame}
\frametitle{Unrelated Title}


\begin{itemize}
\item Deployment management
\end{itemize}

\note[item]{}
\end{frame}
\begin{frame}
\frametitle{Unrelated Title}


\begin{itemize}
\item Service request management
\end{itemize}

\note[item]{}
\end{frame}
\begin{frame}
\frametitle{Unrelated Title}


\begin{itemize}
\item Release management
\end{itemize}

\note[item]{}
\end{frame}
\begin{frame}
\frametitle{Unrelated Title}


\begin{itemize}
\item Deployment management
\end{itemize}

\note[item]{}
\end{frame}
\begin{frame}
\frametitle{Unrelated Title}


\begin{itemize}
\item Service continuity management
\end{itemize}

\note[item]{}
\end{frame}
\begin{frame}
\frametitle{Unrelated Title}


\begin{itemize}
\item service continuity management
\end{itemize}

\note[item]{}
\end{frame}
\begin{frame}
\frametitle{Unrelated Title}


\begin{itemize}
\item Identify new configuration items
\end{itemize}

\note[item]{}
\end{frame}
\begin{frame}
\frametitle{Unrelated Title}


\begin{itemize}
\item The organization’s reputation and corporate brand
\end{itemize}

\note[item]{}
\end{frame}
\begin{frame}
\frametitle{Unrelated Title}


\begin{itemize}
\item Service continuity management
\end{itemize}

\note[item]{}
\end{frame}
\begin{frame}
\frametitle{Unrelated Title}


\begin{itemize}
\item Service continuity management
\end{itemize}

\note[item]{}
\end{frame}
\begin{frame}
\frametitle{Unrelated Title}


\begin{itemize}
\item service configuration management 
\end{itemize}

\note[item]{}
\end{frame}
\begin{frame}
\frametitle{Unrelated Title}


\begin{itemize}
\item Design and Transition
\end{itemize}

\note[item]{}
\end{frame}
\begin{frame}
\frametitle{Unrelated Title}


\begin{itemize}
\item Verify CMDB record accuracy
\end{itemize}

\note[item]{}
\end{frame}
\begin{frame}
\frametitle{Unrelated Title}


\begin{itemize}
\item Service Desk management
\end{itemize}

\note[item]{}
\end{frame}
\begin{frame}
\frametitle{Unrelated Title}


\begin{itemize}
\item Configuration management
\end{itemize}

\note[item]{}
\end{frame}
\begin{frame}
\frametitle{Unrelated Title}


\begin{itemize}
\item Intelligent telephony systemsWorkflow systems
\end{itemize}

\note[item]{}
\end{frame}
\begin{frame}
\frametitle{Unrelated Title}


\begin{itemize}
\item A recovery point objective (RPO)
\end{itemize}

\note[item]{}
\end{frame}
\begin{frame}
\frametitle{Unrelated Title}


\begin{itemize}
\item Feature flags
\end{itemize}

\note[item]{}
\end{frame}
\begin{frame}
\frametitle{Unrelated Title}


\begin{itemize}
\item Intelligent telephony systemsWorkforce managementResource planning systemsWorkflow systemsRecordingQuality controlDashboardsMonitoring tools
\end{itemize}

\note[item]{}
\end{frame}
\begin{frame}
\frametitle{Unrelated Title}


\begin{itemize}
\item centralized service desk
\end{itemize}

\note[item]{}
\end{frame}
\begin{frame}
\frametitle{Unrelated Title}


\begin{itemize}
\item centralized service desk
\end{itemize}

\note[item]{}
\end{frame}
\begin{frame}
\frametitle{Unrelated Title}


\begin{itemize}
\item SLA
\end{itemize}

\note[item]{}
\end{frame}
\begin{frame}
\frametitle{Unrelated Title}


\begin{itemize}
\item A service level agreement (SLA)
\end{itemize}

\note[item]{}
\end{frame}
\begin{frame}
\frametitle{Unrelated Title}


\begin{itemize}
\item tools, data, and information
\end{itemize}

\note[item]{}
\end{frame}
\begin{frame}
\frametitle{Unrelated Title}


\begin{itemize}
\item vital business functions and their dependencies
\end{itemize}

\note[item]{}
\end{frame}
\begin{frame}
\frametitle{Unrelated Title}


\begin{itemize}
\item RTORTO (Recovery Time Objective), is a metric that defines the time to recover your IT infrastructure and services following a disaster to ensure business continuity.
\end{itemize}

\note[item]{}
\end{frame}
\begin{frame}
\frametitle{Unrelated Title}


\begin{itemize}
\item Release management
\end{itemize}

\note[item]{}
\end{frame}
\begin{frame}
\frametitle{Unrelated Title}


\begin{itemize}
\item Service continuity management
\end{itemize}

\note[item]{}
\end{frame}
\begin{frame}
\frametitle{Unrelated Title}


\begin{itemize}
\item Service configuration management
\end{itemize}

\note[item]{}
\end{frame}
\begin{frame}
\frametitle{Unrelated Title}


\begin{itemize}
\item Service level management
\end{itemize}

\note[item]{}
\end{frame}
\begin{frame}
\frametitle{Unrelated Title}


\begin{itemize}
\item Release management
\end{itemize}

\note[item]{}
\end{frame}
\begin{frame}
\frametitle{Unrelated Title}


\begin{itemize}
\item Where are we now?
\end{itemize}

\note[item]{}
\end{frame}
\begin{frame}
\frametitle{Unrelated Title}


\begin{itemize}
\item What is the vision?
\end{itemize}

\note[item]{}
\end{frame}
\begin{frame}
\frametitle{Unrelated Title}


\begin{itemize}
\item Where are we now?
\end{itemize}

\note[item]{}
\end{frame}
\begin{frame}
\frametitle{Unrelated Title}


\begin{itemize}
\item Where do we want to be?
\end{itemize}

\note[item]{}
\end{frame}
\begin{frame}
\frametitle{Unrelated Title}


\begin{itemize}
\item How do we get there?
\end{itemize}

\note[item]{}
\end{frame}
\begin{frame}
\frametitle{Unrelated Title}


\begin{itemize}
\item Initiation
\end{itemize}

\note[item]{}
\end{frame}
\begin{frame}
\frametitle{Unrelated Title}


\begin{itemize}
\item fulfil
\end{itemize}

\note[item]{}
\end{frame}
\begin{frame}
\frametitle{Unrelated Title}


\begin{itemize}
\item standard approval process
\end{itemize}

\note[item]{}
\end{frame}
\begin{frame}
\frametitle{Unrelated Title}


\begin{itemize}
\item Acknowledgement
\end{itemize}

\note[item]{}
\end{frame}
\begin{frame}
\frametitle{Unrelated Title}


\begin{itemize}
\item Service requests
\end{itemize}

\note[item]{}
\end{frame}
\begin{frame}
\frametitle{Unrelated Title}


\begin{itemize}
\item Identify new configuration itemsAdd the CIs to the CMSUpdate configuration dataVerify record accuracyAudit applicationsAudit infrastructure
\end{itemize}

\note[item]{}
\end{frame}
\begin{frame}
\frametitle{Unrelated Title}

\begin{center}
\includegraphics[width=0.9\textwidth,height=0.9\textheight,keepaspectratio]{/Users/I516998/Library/Application Support/Anki2/User 1/collection.media/paste-18c18032586075bdaa8716f448f09716d70c2727.png}
\end{center}


\note[item]{}
\end{frame}
\begin{frame}
\frametitle{Unrelated Title}

\begin{center}
\includegraphics[width=0.9\textwidth,height=0.9\textheight,keepaspectratio]{/Users/I516998/Library/Application Support/Anki2/User 1/collection.media/paste-0db2dcd503149dd389b2b16d0539988f89aa5a30.png}
\end{center}


\note[item]{}
\end{frame}
\begin{frame}
\frametitle{Unrelated Title}


\begin{itemize}
\item Define the vision of the initiative
\end{itemize}

\note[item]{}
\end{frame}
\begin{frame}
\frametitle{Unrelated Title}


\begin{itemize}
\item Continuous deployment
\end{itemize}

\note[item]{}
\end{frame}
\begin{frame}
\frametitle{Unrelated Title}


\begin{itemize}
\item service desk single point of contact
\end{itemize}

\note[item]{}
\end{frame}
\begin{frame}
\frametitle{Unrelated Title}


\begin{itemize}
\item Feature flags
\end{itemize}

\note[item]{}
\end{frame}
\begin{frame}
\frametitle{Unrelated Title}


\begin{itemize}
\item To ensure accurate information about the configuration of services
\end{itemize}

\note[item]{}
\end{frame}
\begin{frame}
\frametitle{Unrelated Title}


\begin{itemize}
\item common service requests
\end{itemize}

\note[item]{}
\end{frame}
\begin{frame}
\frametitle{Unrelated Title}


\begin{itemize}
\item CMDB
\item configuration management database
\end{itemize}

\note[item]{}
\end{frame}
\begin{frame}
\frametitle{Unrelated Title}


\begin{itemize}
\item How do we get there?
\end{itemize}

\note[item]{}
\end{frame}
\begin{frame}
\frametitle{Unrelated Title}


\begin{itemize}
\item Pull deployment
\end{itemize}

\note[item]{}
\end{frame}
\begin{frame}
\frametitle{Unrelated Title}


\begin{itemize}
\item disaster for an organization
\end{itemize}

\note[item]{}
\end{frame}
\begin{frame}
\frametitle{Unrelated Title}


\begin{itemize}
\item Service request management
\end{itemize}

\note[item]{}
\end{frame}
\begin{frame}
\frametitle{Unrelated Title}


\begin{itemize}
\item Act
\end{itemize}

\note[item]{}
\end{frame}
\begin{frame}
\frametitle{Unrelated Title}


\begin{itemize}
\item SLA
\end{itemize}

\note[item]{}
\end{frame}
\begin{frame}
\frametitle{Unrelated Title}


\begin{itemize}
\item BIA - Business Impact Analysis 
\end{itemize}

\note[item]{}
\end{frame}
\begin{frame}
\frametitle{Unrelated Title}


\begin{itemize}
\item Service level management
\end{itemize}

\note[item]{}
\end{frame}
\begin{frame}
\frametitle{Unrelated Title}


\begin{itemize}
\item Release
\end{itemize}

\note[item]{}
\end{frame}
\begin{frame}
\frametitle{Unrelated Title}


\begin{itemize}
\item Opportunity and demand
\item opportunity and demand. Improving the organization or adding value for 
\item stakeholders represent SVS opportunities. Whereas demand refers to the 
\item consumer’s need for the products or services. The consumer may be 
\item internal to the service provider or external.
\end{itemize}

\note[item]{}
\end{frame}
\begin{frame}
\frametitle{Unrelated Title}


\begin{itemize}
\item Different combinations of ITIL practices
\item resources for performing specific work, is needed for the value chain 
\item activities to convert inputs into outputs. These resources may be 
\item internal or external to the service provider.
\end{itemize}

\note[item]{}
\end{frame}
\begin{frame}
\frametitle{Unrelated Title}


\begin{itemize}
\item Tangible itemsaccess to resourcesservice actions
\end{itemize}

\note[item]{}
\end{frame}
\begin{frame}
\frametitle{Unrelated Title}


\begin{itemize}
\item incident management practice
\end{itemize}

\note[item]{}
\end{frame}
\begin{frame}
\frametitle{Unrelated Title}


\begin{itemize}
\item To ensure that products and services continually meet stakeholder expectations for quality, cost, and time to market
\end{itemize}

\note[item]{}
\end{frame}
\begin{frame}
\frametitle{Unrelated Title}


\begin{itemize}
\item Progress iteratively with feedback
\item Using feedback before, throughout, and after each iteration will ensure that actions are focused and appropriate, even if circumstances change.
\end{itemize}

\note[item]{}
\end{frame}
\begin{frame}
\frametitle{Unrelated Title}


\begin{itemize}
\item Partners and suppliers
\end{itemize}

\note[item]{}
\end{frame}
\begin{frame}
\frametitle{Unrelated Title}


\begin{itemize}
\item Release management
\end{itemize}

\note[item]{}
\end{frame}
\begin{frame}
\frametitle{Unrelated Title}


\begin{itemize}
\item components and activities
\end{itemize}

\note[item]{}
\end{frame}
\begin{frame}
\frametitle{Unrelated Title}


\begin{itemize}
\item An outcome
\end{itemize}

\note[item]{}
\end{frame}
\begin{frame}
\frametitle{Unrelated Title}


\begin{itemize}
\item service value chain, an operating model
\end{itemize}

\note[item]{}
\end{frame}
\begin{frame}
\frametitle{Unrelated Title}


\begin{itemize}
\item Service configuration management
\end{itemize}

\note[item]{}
\end{frame}
\begin{frame}
\frametitle{Unrelated Title}


\begin{itemize}
\item Value streams and processes
\end{itemize}

\note[item]{}
\end{frame}
\begin{frame}
\frametitle{Unrelated Title}


\begin{itemize}
\item The purpose of the change control process is to maximize the number of successful service and product changes by ensuring that risks have been properly assessed, authorizing changes to proceed, and managing the change schedule. The scope of change control is defined by each organization. It will typically include all IT infrastructure, applications, documentation, processes, supplier relationships, and anything else that might directly or indirectly impact a product or service.
\end{itemize}

\note[item]{}
\end{frame}
\begin{frame}
\frametitle{Unrelated Title}


\begin{itemize}
\item depend upon each other
\item Organizations should not limit themselves to one or two of the principles. The relevance of each principle needs to be considered and how they apply together. While not all principles will be essential in every situation, at the very least, each one should be reviewed to determine their appropriateness for a given situation.
\end{itemize}

\note[item]{}
\end{frame}
\begin{frame}
\frametitle{Unrelated Title}


\begin{itemize}
\item Service consumption
\end{itemize}

\note[item]{}
\end{frame}
\begin{frame}
\frametitle{Unrelated Title}


\begin{itemize}
\item Think and work holistically
\item Taking a holistic approach to service management includes establishing an understanding of how all the parts of an organization work together in an integrated way.
\end{itemize}

\note[item]{}
\end{frame}
\begin{frame}
\frametitle{Unrelated Title}


\begin{itemize}
\item Service level management
\end{itemize}

\note[item]{}
\end{frame}
\begin{frame}
\frametitle{Unrelated Title}


\begin{itemize}
\item Problem identification
\end{itemize}

\note[item]{}
\end{frame}
\begin{frame}
\frametitle{Unrelated Title}


\begin{itemize}
\item Monitoring and event management
\end{itemize}

\note[item]{}
\end{frame}
\begin{frame}
\frametitle{Unrelated Title}


\begin{itemize}
\item Start where you are
\end{itemize}

\note[item]{}
\end{frame}
\begin{frame}
\frametitle{Unrelated Title}


\begin{itemize}
\item A formal process for logging and managing unplanned interruptions to a service
\end{itemize}

\note[item]{}
\end{frame}
\begin{frame}
\frametitle{Unrelated Title}


\begin{itemize}
\item Risk
\end{itemize}

\note[item]{}
\end{frame}
\begin{frame}
\frametitle{Unrelated Title}


\begin{itemize}
\item Error control
\end{itemize}

\note[item]{}
\end{frame}
\begin{frame}
\frametitle{Unrelated Title}


\begin{itemize}
\item incident
\end{itemize}

\note[item]{}
\end{frame}
\begin{frame}
\frametitle{Unrelated Title}


\begin{itemize}
\item Focus on value
\end{itemize}

\note[item]{}
\end{frame}
\begin{frame}
\frametitle{Unrelated Title}


\begin{itemize}
\item Sponsor
\end{itemize}

\note[item]{}
\end{frame}
\begin{frame}
\frametitle{Unrelated Title}


\begin{itemize}
\item a configuration item
\end{itemize}

\note[item]{}
\end{frame}
\begin{frame}
\frametitle{Unrelated Title}


\begin{itemize}
\item Guiding principles
\item Critical success factors (CSFs) are a necessary precondition for the achievement of intended results. Continual improvement practices involve aligning an organization's practices and services with changing business needs through the ongoing identification and improvement of all elements involved in the effective management of products and services. Architecture management practices provide an understanding of all the different elements that make up an organization and how those elements relate to one another.
\end{itemize}

\note[item]{}
\end{frame}
\begin{frame}
\frametitle{Unrelated Title}


\begin{itemize}
\item service request management practice
\end{itemize}

\note[item]{}
\end{frame}
\begin{frame}
\frametitle{Unrelated Title}


\begin{itemize}
\item Policies should be created as to what service requests will have limited or no additional approvals during the fulfillment procedure
\end{itemize}

\note[item]{}
\end{frame}
\begin{frame}
\frametitle{Unrelated Title}


\begin{itemize}
\item Information security management
\end{itemize}

\note[item]{}
\end{frame}
\begin{frame}
\frametitle{Unrelated Title}


\begin{itemize}
\item service request management
\end{itemize}

\note[item]{}
\end{frame}
\begin{frame}
\frametitle{Unrelated Title}


\begin{itemize}
\item future incidents
\end{itemize}

\note[item]{}
\end{frame}
\begin{frame}
\frametitle{Unrelated Title}


\begin{itemize}
\item change control
\end{itemize}

\note[item]{}
\end{frame}
\begin{frame}
\frametitle{Unrelated Title}


\begin{itemize}
\item service desk practice
\end{itemize}

\note[item]{}
\end{frame}
\begin{frame}
\frametitle{Unrelated Title}


\begin{itemize}
\item how changes should be handled
\end{itemize}

\note[item]{}
\end{frame}
\begin{frame}
\frametitle{Unrelated Title}


\begin{itemize}
\item incident management
\end{itemize}

\note[item]{}
\end{frame}
\begin{frame}
\frametitle{Unrelated Title}


\begin{itemize}
\item Incidents are escalated to a temporary team
\end{itemize}

\note[item]{}
\end{frame}
\begin{frame}
\frametitle{Unrelated Title}


\begin{itemize}
\item Service level management involves collating and analyzing information from several different sources including customer engagement, customer feedback, operational metrics, and business metrics.
\item Customer engagement involves initial listening, discovery, and information capture on which to base metrics, measurement, and ongoing progress discussions.
\end{itemize}

\note[item]{}
\end{frame}
\begin{frame}
\frametitle{Unrelated Title}


\begin{itemize}
\item Problem management 
\end{itemize}

\note[item]{}
\end{frame}
\begin{frame}
\frametitle{Unrelated Title}


\begin{itemize}
\item problem management practice
\end{itemize}

\note[item]{}
\end{frame}
\begin{frame}
\frametitle{Unrelated Title}


\begin{itemize}
\item service management
\end{itemize}

\note[item]{}
\end{frame}
\begin{frame}
\frametitle{Unrelated Title}


\begin{itemize}
\item Collaborate and promote visibility
\end{itemize}

\note[item]{}
\end{frame}
\begin{frame}
\frametitle{Unrelated Title}


\begin{itemize}
\item Keep it simple and practical
\end{itemize}

\note[item]{}
\end{frame}
\begin{frame}
\frametitle{Unrelated Title}


\begin{itemize}
\item Optimize and automate
\end{itemize}

\note[item]{}
\end{frame}
\begin{frame}
\frametitle{Unrelated Title}


\begin{itemize}
\item guiding principles
\item Organizations should not limit themselves to one or two of the principles. The relevance of each principle needs to be considered and how they apply together. While not all principles will be essential in every situation, at the very least, each one should be reviewed to determine their appropriateness for a given situation.
\end{itemize}

\note[item]{}
\end{frame}
\begin{frame}
\frametitle{Unrelated Title}


\begin{itemize}
\item Information and technology
\end{itemize}

\note[item]{}
\end{frame}
\begin{frame}
\frametitle{Unrelated Title}


\begin{itemize}
\item Organizations and people
\end{itemize}

\note[item]{}
\end{frame}
\begin{frame}
\frametitle{Unrelated Title}


\begin{itemize}
\item To ensure a shared understanding of the vision, current status and improvement direction for all four service management dimensions and all products and services across the organization.
\end{itemize}

\note[item]{}
\end{frame}
\begin{frame}
\frametitle{Unrelated Title}

\begin{center}
\includegraphics[width=0.9\textwidth,height=0.9\textheight,keepaspectratio]{/Users/I516998/Library/Application Support/Anki2/User 1/collection.media/paste-e2c5000f8343de5954dd5ff4cba763763a728213.jpg}
\end{center}

\begin{itemize}
\item Error - wrong answers, low accuracy, underfitting
\end{itemize}

\note[item]{}
\end{frame}
\begin{frame}
\frametitle{Unrelated Title}

\begin{center}
\includegraphics[width=0.9\textwidth,height=0.9\textheight,keepaspectratio]{/Users/I516998/Library/Application Support/Anki2/User 1/collection.media/paste-e2c5000f8343de5954dd5ff4cba763763a728213.jpg}
\end{center}

\begin{itemize}
\item Change in predictions with changes in training.
\end{itemize}

\note[item]{}
\end{frame}
\begin{frame}
\frametitle{Unrelated Title}

\begin{center}
\includegraphics[width=0.9\textwidth,height=0.9\textheight,keepaspectratio]{/Users/I516998/Library/Application Support/Anki2/User 1/collection.media/paste-9468185cbf252cd0d1ec910aa6a022baaaecd658.jpg}
\end{center}

\begin{itemize}
\item Low accuracy, due to insufficient model complexity or training
\end{itemize}

\note[item]{}
\end{frame}
\begin{frame}
\frametitle{Unrelated Title}

\begin{center}
\includegraphics[width=0.9\textwidth,height=0.9\textheight,keepaspectratio]{/Users/I516998/Library/Application Support/Anki2/User 1/collection.media/paste-9468185cbf252cd0d1ec910aa6a022baaaecd658.jpg}
\end{center}

\begin{itemize}
\item Poor generalization, from too much model complexity or over training
\end{itemize}

\note[item]{}
\end{frame}
\begin{frame}
\frametitle{Unrelated Title}


\begin{itemize}
\item Boosting where sucsessive models predict the error of the previous model; the error is equivalent to the gradient of the loss function.
\end{itemize}

\note[item]{}
\end{frame}
\begin{frame}
\frametitle{Unrelated Title}


\begin{itemize}
\item Out of all values of all features, choose the one that best splits the targets
\end{itemize}

\note[item]{}
\end{frame}
\begin{frame}
\frametitle{Unrelated Title}

\begin{center}
\includegraphics[width=0.9\textwidth,height=0.9\textheight,keepaspectratio]{/Users/I516998/Library/Application Support/Anki2/User 1/collection.media/kernel-trick.png}
\end{center}

\begin{itemize}
\item Efficiently mapping data into higher-dimensional spaces where it is more easily separable.
\end{itemize}

\note[item]{}
\end{frame}
\begin{frame}
\frametitle{Unrelated Title}


\begin{itemize}
\item Choose the best features as ranked by variance, correlation with labels, mutual information, etc.
\end{itemize}

\note[item]{}
\end{frame}
\begin{frame}
\frametitle{Unrelated Title}


\begin{itemize}
\item Find the best single feature via cross-validation.  Then add each remaining feature, one at a time, to find the best pair of features.  Repeat, adding one feature each iteration, until performance does not improve significantly.
\end{itemize}

\note[item]{}
\end{frame}
\begin{frame}
\frametitle{Unrelated Title}


\begin{itemize}
\item Use a model that gives feature importances
\end{itemize}

\note[item]{}
\end{frame}
\begin{frame}
\frametitle{Unrelated Title}


\begin{itemize}
\item Decision trees
\item LASSO (linear regression with L1 regularization)
\end{itemize}

\note[item]{}
\end{frame}
\begin{frame}
\frametitle{Unrelated Title}


\begin{itemize}
\item A matrix factorization such that the most significant components form the best approximation for a given number of dimensions.
\end{itemize}

\note[item]{}
\end{frame}
\begin{frame}
\frametitle{Unrelated Title}

\begin{center}
\includegraphics[width=0.9\textwidth,height=0.9\textheight,keepaspectratio]{/Users/I516998/Library/Application Support/Anki2/User 1/collection.media/svm.png}
\end{center}

\begin{itemize}
\item SVMs can find non-linear separating hyperplanes by mapping data into a higher number of dimensions with the kernel trick.
\end{itemize}

\note[item]{}
\end{frame}
\begin{frame}
\frametitle{Unrelated Title}


\begin{itemize}
\item A set of specialized organizational capabilities for enabling value to customers in the form of services.
\end{itemize}

\note[item]{}
\end{frame}
\begin{frame}
\frametitle{Unrelated Title}


\begin{itemize}
\item comprehensive framework for IT service management.
\end{itemize}

\note[item]{}
\end{frame}
\begin{frame}
\frametitle{Unrelated Title}


\begin{itemize}
\item these are increase  in assets (or decrease liabilities), that result in an increase in owners’ equity (other than by a contribution by the owner), achieved by providing goods or services to customers.
\end{itemize}

\note[item]{}
\end{frame}
\begin{frame}
\frametitle{Unrelated Title}


\begin{itemize}
\item these
are decreases in assets (or increases in liabilities) that lead to a decrease
in owners equity (other than by a withdrawal by the owner) 
\end{itemize}

\note[item]{}
\end{frame}
\begin{frame}
\frametitle{Unrelated Title}


\begin{itemize}
\item these
are present economic resources controlled by an entity as a result
of past events, which have the potential to produce future economic
benefits. 
\end{itemize}

\note[item]{}
\end{frame}
\begin{frame}
\frametitle{Unrelated Title}


\begin{itemize}
\item these
are present obligations of an entity to transfer economic resources as a result of a past event.



 


\end{itemize}

\note[item]{}
\end{frame}
\begin{frame}
\frametitle{Unrelated Title}


\begin{itemize}
\item this
is a residual amount after liabilities have been deducted from assets.
\end{itemize}

\note[item]{}
\end{frame}
\begin{frame}
\frametitle{Unrelated Title}


\begin{itemize}
\item the
information should be available for decision-makers in time that it is useful
for decision making. 
\end{itemize}

\note[item]{}
\end{frame}
\begin{frame}
\frametitle{Unrelated Title}


\begin{itemize}
\item the
accounting information should be presented in a format which can be understood
by users with a reasonable knowledge of business and economic activities and
they can comprehend their meaning. Information should be presented concisely
and clearly. 
\end{itemize}

\note[item]{}
\end{frame}
\begin{frame}
\frametitle{Unrelated Title}


\begin{itemize}
\item refers
to the usefulness of financial information in helping users make decisions.
Relevant financial information will assist the users in forming predictions
about the outcomes of past, present or future events. 
\end{itemize}

\note[item]{}
\end{frame}
\begin{frame}
\frametitle{Unrelated Title}


\begin{itemize}
\item information
reported must be a faithful representation of the real-world economic event it
represents. Information should be complete, free from material error, accurate
and neutral (without bias)
\end{itemize}

\note[item]{}
\end{frame}
\begin{frame}
\frametitle{Unrelated Title}


\begin{itemize}
\item ensures
that users identify similarities and differences in financial reports with
other entities or over different reporting periods because of consistent
accounting standards and policies.  
\end{itemize}

\note[item]{}
\end{frame}
\begin{frame}
\frametitle{Unrelated Title}


\begin{itemize}
\item the
premise that financial information is supported by evidence that can be used to
check its accuracy. The financial information should allow different independent
and knowledgeable observers to agree that a event has been faithfully
represented
\end{itemize}

\note[item]{}
\end{frame}
\begin{frame}
\frametitle{Unrelated Title}


\begin{itemize}
\item the ongoing life of the business is divided into
equal, arbitrary reporting periods e.g a month, quarter or year in order to
compare results, measure performance and prepare reports. Reports should
reflect the period in which the transaction occurs. In each reporting period
revenue earned is compared with expenses incurred so as to calculate an
accurate profit for the reporting period.
\end{itemize}

\note[item]{}
\end{frame}
\begin{frame}
\frametitle{Unrelated Title}


\begin{itemize}
\item revenue
Is recognised when it is earned, not when the cash is received whereas expenses
are recognised when they are incurred, not when the cash is paid. Under accrual
accounting, profit is determined by subtracting the expenses incurred from the
revenue earned in order to accurately determine profit for that period
\end{itemize}

\note[item]{}
\end{frame}
\begin{frame}
\frametitle{Unrelated Title}


\begin{itemize}
\item a business will continue to operate and will not
be wound up in the foreseeable future and records are kept on this basis.
\end{itemize}

\note[item]{}
\end{frame}
\begin{frame}
\frametitle{Unrelated Title}


\begin{itemize}
\item a business has its own financial status and is
completely separate from its owner and other entities. A record of assets, liabilities
and business activities of the entity are kept separate from those of the owner
of the entity as well as from those of other entities.
\end{itemize}

\note[item]{}
\end{frame}
\begin{frame}
\frametitle{Unrelated Title}


\begin{itemize}
\item output
\end{itemize}

\note[item]{}
\end{frame}
\begin{frame}
\frametitle{Unrelated Title}


\begin{itemize}
\item An output
\end{itemize}

\note[item]{}
\end{frame}
\begin{frame}
\frametitle{Unrelated Title}


\begin{itemize}
\item outcome
\end{itemize}

\note[item]{}
\end{frame}
\begin{frame}
\frametitle{Unrelated Title}


\begin{itemize}
\item An outcome
\end{itemize}

\note[item]{}
\end{frame}
\begin{frame}
\frametitle{Unrelated Title}


\begin{itemize}
\item The consumer contributes to the reduction of risk
\end{itemize}

\note[item]{}
\end{frame}
\begin{frame}
\frametitle{Unrelated Title}


\begin{itemize}
\item The consumer contributes to the reduction of risk
\end{itemize}

\note[item]{}
\end{frame}
\begin{frame}
\frametitle{Unrelated Title}


\begin{itemize}
\item Utility
\end{itemize}

\note[item]{}
\end{frame}
\begin{frame}
\frametitle{Unrelated Title}


\begin{itemize}
\item Warranty
\end{itemize}

\note[item]{}
\end{frame}
\begin{frame}
\frametitle{Unrelated Title}


\begin{itemize}
\item Dimension 1: Organizations & People
\end{itemize}

\note[item]{}
\end{frame}
\begin{frame}
\frametitle{Unrelated Title}


\begin{itemize}
\item Dimension 2: Information & Technology
\end{itemize}

\note[item]{}
\end{frame}
\begin{frame}
\frametitle{Unrelated Title}


\begin{itemize}
\item IT services
\end{itemize}

\note[item]{}
\end{frame}
\begin{frame}
\frametitle{Unrelated Title}


\begin{itemize}
\item ● Dimension 2: Information & Technology
\end{itemize}

\note[item]{}
\end{frame}
\begin{frame}
\frametitle{Unrelated Title}


\begin{itemize}
\item ● Dimension 3: Partners & Suppliers
\end{itemize}

\note[item]{}
\end{frame}
\begin{frame}
\frametitle{Unrelated Title}


\begin{itemize}
\item ● Dimension 3: Partners & Suppliers
\end{itemize}

\note[item]{}
\end{frame}
\begin{frame}
\frametitle{Unrelated Title}


\begin{itemize}
\item ● Dimension 3: Partners & Suppliers
\end{itemize}

\note[item]{}
\end{frame}
\begin{frame}
\frametitle{Unrelated Title}


\begin{itemize}
\item ● Dimension 3: Partners & Suppliers
\end{itemize}

\note[item]{}
\end{frame}
\begin{frame}
\frametitle{Unrelated Title}


\begin{itemize}
\item ● Dimension 3: Partners & Suppliers
\end{itemize}

\note[item]{}
\end{frame}
\begin{frame}
\frametitle{Unrelated Title}


\begin{itemize}
\item ● Dimension 4: Value Streams & Processes
\end{itemize}

\note[item]{}
\end{frame}
\begin{frame}
\frametitle{Unrelated Title}


\begin{itemize}
\item ● Dimension 4: Value Streams & Processes
\end{itemize}

\note[item]{}
\end{frame}
\begin{frame}
\frametitle{Unrelated Title}


\begin{itemize}
\item ● Dimension 4: Value Streams & Processes
\end{itemize}

\note[item]{}
\end{frame}
\begin{frame}
\frametitle{Unrelated Title}


\begin{itemize}
\item process automation
\end{itemize}

\note[item]{}
\end{frame}
\begin{frame}
\frametitle{Unrelated Title}


\begin{itemize}
\item interrelated
\item interacting
\end{itemize}

\note[item]{}
\end{frame}
\begin{frame}
\frametitle{Unrelated Title}


\begin{itemize}
\item PESTLE
\end{itemize}

\note[item]{}
\end{frame}
\begin{frame}
\frametitle{Unrelated Title}


\begin{itemize}
\item interconnected
\end{itemize}

\note[item]{}
\end{frame}
\begin{frame}
\frametitle{Unrelated Title}


\begin{itemize}
\item Guiding Principle
\end{itemize}

\note[item]{}
\end{frame}
\begin{frame}
\frametitle{Unrelated Title}


\begin{itemize}
\item continual improvement
\end{itemize}

\note[item]{}
\end{frame}
\begin{frame}
\frametitle{Unrelated Title}


\begin{itemize}
\item ● Focus on Value
\end{itemize}

\note[item]{}
\end{frame}
\begin{frame}
\frametitle{Unrelated Title}


\begin{itemize}
\item ● Start Where You Are
\end{itemize}

\note[item]{}
\end{frame}
\begin{frame}
\frametitle{Unrelated Title}


\begin{itemize}
\item iterative manner
\end{itemize}

\note[item]{}
\end{frame}
\begin{frame}
\frametitle{Unrelated Title}


\begin{itemize}
\item ● Collaborate and Promote Visibility
\end{itemize}

\note[item]{}
\end{frame}
\begin{frame}
\frametitle{Unrelated Title}


\begin{itemize}
\item holistic
\end{itemize}

\note[item]{}
\end{frame}
\begin{frame}
\frametitle{Unrelated Title}


\begin{itemize}
\item ● Keep it Simple and Practical
\end{itemize}

\note[item]{}
\end{frame}
\begin{frame}
\frametitle{Unrelated Title}


\begin{itemize}
\item Optimization
\end{itemize}

\note[item]{}
\end{frame}
\begin{frame}
\frametitle{Unrelated Title}


\begin{itemize}
\item Automation
\end{itemize}

\note[item]{}
\end{frame}
\begin{frame}
\frametitle{Unrelated Title}


\begin{itemize}
\item ● Optimize and Automate
\end{itemize}

\note[item]{}
\end{frame}
\begin{frame}
\frametitle{Unrelated Title}


\begin{itemize}
\item obtain/build
\end{itemize}

\note[item]{}
\end{frame}
\begin{frame}
\frametitle{Unrelated Title}


\begin{itemize}
\item deliver and support
\end{itemize}

\note[item]{}
\end{frame}
\begin{frame}
\frametitle{Unrelated Title}


\begin{itemize}
\item continual improvement
\end{itemize}

\note[item]{}
\end{frame}
\begin{frame}
\frametitle{Unrelated Title}


\begin{itemize}
\item What is the Vision?
\end{itemize}

\note[item]{}
\end{frame}
\begin{frame}
\frametitle{Unrelated Title}


\begin{itemize}
\item Where are we now?
\end{itemize}

\note[item]{}
\end{frame}
\begin{frame}
\frametitle{Unrelated Title}


\begin{itemize}
\item Where do we want to be?
\end{itemize}

\note[item]{}
\end{frame}
\begin{frame}
\frametitle{Unrelated Title}


\begin{itemize}
\item How do we get there?
\end{itemize}

\note[item]{}
\end{frame}
\begin{frame}
\frametitle{Unrelated Title}


\begin{itemize}
\item Take action
\end{itemize}

\note[item]{}
\end{frame}
\begin{frame}
\frametitle{Unrelated Title}


\begin{itemize}
\item Did we get there?
\end{itemize}

\note[item]{}
\end{frame}
\begin{frame}
\frametitle{Unrelated Title}


\begin{itemize}
\item practices and services
\end{itemize}

\note[item]{}
\end{frame}
\begin{frame}
\frametitle{Unrelated Title}


\begin{itemize}
\item ● ** Continual Improvement
\end{itemize}

\note[item]{}
\end{frame}
\begin{frame}
\frametitle{Unrelated Title}


\begin{itemize}
\item Leaders
\end{itemize}

\note[item]{}
\end{frame}
\begin{frame}
\frametitle{Unrelated Title}


\begin{itemize}
\item Continual improvement team
\end{itemize}

\note[item]{}
\end{frame}
\begin{frame}
\frametitle{Unrelated Title}


\begin{itemize}
\item Everyone in the org
\end{itemize}

\note[item]{}
\end{frame}
\begin{frame}
\frametitle{Unrelated Title}


\begin{itemize}
\item Partners and suppliers
\end{itemize}

\note[item]{}
\end{frame}
\begin{frame}
\frametitle{Unrelated Title}


\begin{itemize}
\item planning activitiesmethodstechniques
\end{itemize}

\note[item]{}
\end{frame}
\begin{frame}
\frametitle{Unrelated Title}


\begin{itemize}
\item Improve
\end{itemize}

\note[item]{}
\end{frame}
\begin{frame}
\frametitle{Unrelated Title}


\begin{itemize}
\item EngageDesign and transitionObtain/BuildDeliver and support
\end{itemize}

\note[item]{}
\end{frame}
\begin{frame}
\frametitle{Unrelated Title}


\begin{itemize}
\item information security management
\end{itemize}

\note[item]{}
\end{frame}
\begin{frame}
\frametitle{Unrelated Title}


\begin{itemize}
\item relationship management
\end{itemize}

\note[item]{}
\end{frame}
\begin{frame}
\frametitle{Unrelated Title}


\begin{itemize}
\item supplier management 
\end{itemize}

\note[item]{}
\end{frame}
\begin{frame}
\frametitle{Unrelated Title}


\begin{itemize}
\item There are 17 General Management Practices:
\end{itemize}

\note[item]{}
\end{frame}
\begin{frame}
\frametitle{Unrelated Title}


\begin{itemize}
\item change control
\end{itemize}

\note[item]{}
\end{frame}
\begin{frame}
\frametitle{Unrelated Title}


\begin{itemize}
\item Standard
\end{itemize}

\note[item]{}
\end{frame}
\begin{frame}
\frametitle{Unrelated Title}


\begin{itemize}
\item Normal
\end{itemize}

\note[item]{}
\end{frame}
\begin{frame}
\frametitle{Unrelated Title}


\begin{itemize}
\item Emergency
\end{itemize}

\note[item]{}
\end{frame}
\begin{frame}
\frametitle{Unrelated Title}


\begin{itemize}
\item StandardNormalEmergency
\end{itemize}

\note[item]{}
\end{frame}
\begin{frame}
\frametitle{Unrelated Title}


\begin{itemize}
\item change authority
\end{itemize}

\note[item]{}
\end{frame}
\begin{frame}
\frametitle{Unrelated Title}


\begin{itemize}
\item incident management
\end{itemize}

\note[item]{}
\end{frame}
\begin{frame}
\frametitle{Unrelated Title}


\begin{itemize}
\item An incident is an unplanned interruption to a service, or reduction in the quality of service.
\end{itemize}

\note[item]{}
\end{frame}
\begin{frame}
\frametitle{Unrelated Title}


\begin{itemize}
\item Design
\end{itemize}

\note[item]{}
\end{frame}
\begin{frame}
\frametitle{Unrelated Title}


\begin{itemize}
\item Prioritize
\end{itemize}

\note[item]{}
\end{frame}
\begin{frame}
\frametitle{Unrelated Title}


\begin{itemize}
\item swarming
\end{itemize}

\note[item]{}
\end{frame}
\begin{frame}
\frametitle{Unrelated Title}


\begin{itemize}
\item problem management
\end{itemize}

\note[item]{}
\end{frame}
\begin{frame}
\frametitle{Unrelated Title}


\begin{itemize}
\item problem
\end{itemize}

\note[item]{}
\end{frame}
\begin{frame}
\frametitle{Unrelated Title}


\begin{itemize}
\item analyzed
\end{itemize}

\note[item]{}
\end{frame}
\begin{frame}
\frametitle{Unrelated Title}


\begin{itemize}
\item Problem Management
\end{itemize}

\note[item]{}
\end{frame}
\begin{frame}
\frametitle{Unrelated Title}


\begin{itemize}
\item service desk
\end{itemize}

\note[item]{}
\end{frame}
\begin{frame}
\frametitle{Unrelated Title}


\begin{itemize}
\item virtual service desk
\end{itemize}

\note[item]{}
\end{frame}
\begin{frame}
\frametitle{Unrelated Title}


\begin{itemize}
\item service level management
\end{itemize}

\note[item]{}
\end{frame}
\begin{frame}
\frametitle{Unrelated Title}


\begin{itemize}
\item Provides the end to end visibility
\end{itemize}

\note[item]{}
\end{frame}
\begin{frame}
\frametitle{Unrelated Title}


\begin{itemize}
\item Customer engagement
\end{itemize}

\note[item]{}
\end{frame}
\begin{frame}
\frametitle{Unrelated Title}


\begin{itemize}
\item ▪ Customer feedback
\end{itemize}

\note[item]{}
\end{frame}
\begin{frame}
\frametitle{Unrelated Title}


\begin{itemize}
\item service request management
\end{itemize}

\note[item]{}
\end{frame}
\begin{frame}
\frametitle{Unrelated Title}

\begin{center}
\includegraphics[width=0.9\textwidth,height=0.9\textheight,keepaspectratio]{/Users/I516998/Library/Application Support/Anki2/User 1/collection.media/paste-010fe897f78c3c298c677ffc57ca6dd9d06940e6.png}
\end{center}

\begin{itemize}
\item Initiation -> Approval -> Fulfillment
\end{itemize}

\note[item]{}
\end{frame}
\begin{frame}
\frametitle{Unrelated Title}


\begin{itemize}
\item Service requests
\end{itemize}

\note[item]{}
\end{frame}
\begin{frame}
\frametitle{Unrelated Title}


\begin{itemize}
\item Some examples of a service request
\end{itemize}

\note[item]{}
\end{frame}
\begin{frame}
\frametitle{Unrelated Title}


\begin{itemize}
\item Policies
\end{itemize}

\note[item]{}
\end{frame}
\begin{frame}
\frametitle{Unrelated Title}


\begin{itemize}
\item IT asset
\end{itemize}

\note[item]{}
\end{frame}
\begin{frame}
\frametitle{Unrelated Title}


\begin{itemize}
\item monitoring and event management
\end{itemize}

\note[item]{}
\end{frame}
\begin{frame}
\frametitle{Unrelated Title}


\begin{itemize}
\item new and changed
\end{itemize}

\note[item]{}
\end{frame}
\begin{frame}
\frametitle{Unrelated Title}


\begin{itemize}
\item service configuration management
\end{itemize}

\note[item]{}
\end{frame}
\begin{frame}
\frametitle{Unrelated Title}


\begin{itemize}
\item Technical management
\end{itemize}

\note[item]{}
\end{frame}
\begin{frame}
\frametitle{Unrelated Title}


\begin{itemize}
\item  ▪ * Deployment Management ▪ Infrastructure and Platform Management ▪ Software Development and Management
\end{itemize}

\note[item]{}
\end{frame}
\begin{frame}
\frametitle{Unrelated Title}


\begin{itemize}
\item deployment management
\end{itemize}

\note[item]{}
\end{frame}
\begin{frame}
\frametitle{Unrelated Title}


\begin{itemize}
\item A partnership refers to an unincorporated business with unlimited liability which has two or more owners known as "partners" conducting business activities. The responsibilities and profits  of the business are shared between partners.
\end{itemize}

\note[item]{}
\end{frame}
\begin{frame}
\frametitle{Unrelated Title}


\begin{itemize}
\item A group of workers operating together to meet common goals.
\end{itemize}

\note[item]{}
\end{frame}
\begin{frame}
\frametitle{Unrelated Title}


\begin{itemize}
\item Sole traders or sole proprietorships refers to an individual who owns and runs an unincorporated personal business with unlimited liability. 
\end{itemize}

\note[item]{}
\end{frame}
\begin{frame}
\frametitle{Unrelated Title}


\begin{itemize}
\item A Business Plan is a planning tool that serves as a blueprint to address the issues of a startup business. It sets out the business idea, aims and objectives, and should outline the business organisation and each of the four business functions: Finance, Marketing, Operations and Human Resources. It is meant for investors/banks to help them decide on whether to invest/approve loans
\end{itemize}

\note[item]{}
\end{frame}
\begin{frame}
\frametitle{Unrelated Title}


\begin{itemize}
\item They are any good produced by a business that is used and required for another firm to operate (such as machinery or buildings)
\end{itemize}

\note[item]{}
\end{frame}
\begin{frame}
\frametitle{Unrelated Title}


\begin{itemize}
\item An entrepreneur is a risk tasking person who starts a business, while providing finance and direction.
\end{itemize}

\note[item]{}
\end{frame}
\begin{frame}
\frametitle{Unrelated Title}


\begin{itemize}
\item An Intrapreneur is a person that acts as an entrepreneur within a large corporation who takes direct responsibility for turning an idea into a profitable finished product through entrepreneurial talents.
\end{itemize}

\note[item]{}
\end{frame}
\begin{frame}
\frametitle{Unrelated Title}


\begin{itemize}
\item Revenue refers to the amount of money earned by a business for providing goods and services to consumers.
\end{itemize}

\note[item]{}
\end{frame}
\begin{frame}
\frametitle{Unrelated Title}


\begin{itemize}
\item The management department is one of the interdependent business functions that is responsible for identifying, anticipating and satisfying customers' needs and wants in order for the busiess to be profitable.
\end{itemize}

\note[item]{}
\end{frame}
\begin{frame}
\frametitle{Unrelated Title}


\begin{itemize}
\item Any tangible or non tangle output produced by a businessed that is purchased by either commercial or private customers
\end{itemize}

\note[item]{}
\end{frame}
\begin{frame}
\frametitle{Unrelated Title}


\begin{itemize}
\item The HR department is one of the interdependent business functions that is responsible for employment-related matters, such as the hiring process, payroll, benefits, and labor relations.
\end{itemize}

\note[item]{}
\end{frame}
\begin{frame}
\frametitle{Unrelated Title}


\begin{itemize}
\item A business is a decision-making organization with the purpose of using inputs, the factors of production, to produce outputs or products, in order to satisfy needs and wants of consumers, people, organisations or states.
\end{itemize}

\note[item]{}
\end{frame}
\begin{frame}
\frametitle{Unrelated Title}


\begin{itemize}
\item A physical or tangible product of a business that satisfies consumer needs or wants.
\end{itemize}

\note[item]{}
\end{frame}
\begin{frame}
\frametitle{Unrelated Title}


\begin{itemize}
\item Services are non-tangible products of a business, that product tangible effects consumers either need or want.
\end{itemize}

\note[item]{}
\end{frame}
\begin{frame}
\frametitle{Unrelated Title}


\begin{itemize}
\item Needs refer to basic human necessities such as clothes, food, and water, which individuals can not survive without.
\end{itemize}

\note[item]{}
\end{frame}
\begin{frame}
\frametitle{Unrelated Title}


\begin{itemize}
\item Wants are products that people desire but do not require (e.g. new smartphones, a vacation)
\end{itemize}

\note[item]{}
\end{frame}
\begin{frame}
\frametitle{Unrelated Title}


\begin{itemize}
\item The Finance and Accounts department is one of the interdependent business functions that is responsible for managing, recording, and reporting of a business's financial accounts or money, as well as complying with legal requirements such as taxes/
\end{itemize}

\note[item]{}
\end{frame}
\begin{frame}
\frametitle{Unrelated Title}


\begin{itemize}
\item Also know as "Operation Management' and "Producton", it is one of the interdependent business functions resposible for the production process of the business (transforming inputs into outputs), as well as distribution to customers.
\end{itemize}

\note[item]{}
\end{frame}
\begin{frame}
\frametitle{Unrelated Title}


\begin{itemize}
\item Capital is refers to man-made goods like machines, buildings, vehicles, and equipment needed for business to operate.Land refers to raw materials and natural resources that are used in making a product.· Labor or manpower is the physical & mental efforts of people to produce products/services.  Enterprise is the management, organization, and planning of other three factors of production.
\end{itemize}

\note[item]{}
\end{frame}
\begin{frame}
\frametitle{Unrelated Title}


\begin{itemize}
\item The primary sector refers to business activity involving the extraction, harvesting, and conversion of natural resources. This tends to account for a large percentage of business activity of less economically developed countries. Because it has very little value added, as an economy grows, the country becomes less reliant towards this sector.
\end{itemize}

\note[item]{}
\end{frame}
\begin{frame}
\frametitle{Unrelated Title}


\begin{itemize}
\item The secondary sector include businesses that are involved with the manufacturing of goods to either individuals or to firms. This has a higher amount of value added compared to the primary sector. It is known to be the weatlh generating sectors and as such majority of developing countries have a dominant secondary sector.
\end{itemize}

\note[item]{}
\end{frame}
\begin{frame}
\frametitle{Unrelated Title}


\begin{itemize}
\item Business in the tertiary sectors provide services to either individuals or other businesses. This sector relies of goods produced from the primary and secondary sectors as inputs.They are often dominant in more economically developed countries.
\end{itemize}

\note[item]{}
\end{frame}
\begin{frame}
\frametitle{Unrelated Title}


\begin{itemize}
\item Sectorial change refers to the phenomenon, when the dominant sector in an economy shifts seen when there is a shift in the relative share of national output and employment attributed to each business sector.
\end{itemize}

\note[item]{}
\end{frame}
\begin{frame}
\frametitle{Unrelated Title}


\begin{itemize}
\item The quartenary sector is a subset of the tertiary sector.  It provides services, specifically ones involved in intellectual, knowledge based activities that generate and share information, to businesses and individuals. Because of its nature, it requires a highly educated workforce most often found in more economically developed countries.
\end{itemize}

\note[item]{}
\end{frame}
\begin{frame}
\frametitle{Unrelated Title}


\begin{itemize}
\item Value added refers to the amount the value of the product increased during the production process. This is calculated by subtracting the value of the final good by the value of the raw materials and components.
\end{itemize}

\note[item]{}
\end{frame}
\begin{frame}
\frametitle{Unrelated Title}


\begin{itemize}
\item Industrialisation refers to sectorial change from the primary sector, the extraction and havesting of raw materials, to the secondary sector, manufacturing of goods. The result of this is more refined goods with higher export potential, an increase in the standard of education, and the creation of more job opportunities.
\end{itemize}

\note[item]{}
\end{frame}
\begin{frame}
\frametitle{Unrelated Title}


\begin{itemize}
\item Deindustrialisation refer to the sectorial shift of the secondary sector, which focuses on creating goods, towards the tertiary sector, that is involved with providing services.
\end{itemize}

\note[item]{}
\end{frame}
\begin{frame}
\frametitle{Unrelated Title}


\begin{itemize}
\item Growth, Earnings, Transferance, Challenge, Autonomy, Security, and Hobbies
\end{itemize}

\note[item]{}
\end{frame}
\begin{frame}
\frametitle{Unrelated Title}


\begin{itemize}
\item Entreprenuers are the owners of the business, while intrapreneurs are employees of the business.Entreprenuers take high risks, while intrapreneurs take medium to low risks.
\end{itemize}

\note[item]{}
\end{frame}
\begin{frame}
\frametitle{Unrelated Title}


\begin{itemize}
\item The private sector comprises of businesses and organizations that are owned, financed and controlled by private individuals or entities, whose goal is to make a profit
\end{itemize}

\note[item]{}
\end{frame}
\begin{frame}
\frametitle{Unrelated Title}


\begin{itemize}
\item The public sector includes organisations that are owned, financed and run by the national or local government. They provide public goods, that would otherwise be underproduced in the economy, such as education, security, and healthcare. 
\end{itemize}

\note[item]{}
\end{frame}
\begin{frame}
\frametitle{Unrelated Title}


\begin{itemize}
\item They are organisations that are wholly owned and controlled by the government or state.
\end{itemize}

\note[item]{}
\end{frame}
\begin{frame}
\frametitle{Unrelated Title}


\begin{itemize}
\item Unlimited liabilities refers to the owners of an unincorporated business to be seen as the same legal entity, putting the debt incurred by the business on the onwer.
\end{itemize}

\note[item]{}
\end{frame}
\begin{frame}
\frametitle{Unrelated Title}


\begin{itemize}
\item 1. A sole proprietorship is easy to set up and does not have many legal barriers or requirements.
\item 2. A sole proprietorship enjoys complete profits that the business may bring.
\end{itemize}

\note[item]{}
\end{frame}
\begin{frame}
\frametitle{Unrelated Title}


\begin{itemize}
\item 1. unlimited liability
\end{itemize}

\note[item]{}
\end{frame}
\begin{frame}
\frametitle{Unrelated Title}


\begin{itemize}
\item Also known as sleeping partners, are investors that have a financial stake in a partnership and are eligible for a portion of the profits, but do not wish to actively take part in running it.
\end{itemize}

\note[item]{}
\end{frame}
\begin{frame}
\frametitle{Unrelated Title}


\begin{itemize}
\item Also know as a pertnership deed, is a legal contract drawn between partners of a partnership, which states the responsiblity, portion of profits, and finance of each member. Furthermore, it contains terms should there be a withdrawal or introduction of a partner, or the termination of the partnership.
\end{itemize}

\note[item]{}
\end{frame}
\begin{frame}
\frametitle{Unrelated Title}


\begin{itemize}
\item 1. Due to the higher amount of stakeholders in the business, they enjoy a better level of financial strength compared to Sole Proprietorships as more members can contribute to the costs. It is also easier to obtain sources of external finance,as it poses a lower risk.
\item 2. Partnerships also enjoy the benefits of shared expertise as well as having specialisation. This could greatly improve efficiencyand increase their customer base. 
\end{itemize}

\note[item]{}
\end{frame}
\begin{frame}
\frametitle{Unrelated Title}


\begin{itemize}
\item 1. Partnerships still have unlimited liability, which make the owners more vulnerable should the business fail.
\item 2. Partnerships also have a lack of continuity, despite it being an improvement from sole proprietorships, as a withdrawl or death of a partner can cause problems and require an amended deed of partnership
\end{itemize}

\note[item]{}
\end{frame}
\begin{frame}
\frametitle{Unrelated Title}


\begin{itemize}
\item A company is an incorporation business that is owned by shareholders, who enjoy limited liability, and run by its board of directors.
\end{itemize}

\note[item]{}
\end{frame}
\begin{frame}
\frametitle{Unrelated Title}


\begin{itemize}
\item It is an incorporated busiess that is unable to raise share capital from the general public in the public stock market. 
\end{itemize}

\note[item]{}
\end{frame}
\begin{frame}
\frametitle{Unrelated Title}


\begin{itemize}
\item This is an incorporate business that is able to advertise and sell shares to the general public via the stock exchange.
\end{itemize}

\note[item]{}
\end{frame}
\begin{frame}
\frametitle{Unrelated Title}


\begin{itemize}
\item The memorandum of association and the articles of incorporation (or association)
\end{itemize}

\note[item]{}
\end{frame}
\begin{frame}
\frametitle{Unrelated Title}


\begin{itemize}
\item The memorandum of association is a brief document regardingfundamental details of a company such as its name, and purpose
\end{itemize}

\note[item]{}
\end{frame}
\begin{frame}
\frametitle{Unrelated Title}


\begin{itemize}
\item It is a document that ennumerates the internal regulations and procedures of the company, as well as administrative issues
\end{itemize}

\note[item]{}
\end{frame}
\begin{frame}
\frametitle{Unrelated Title}


\begin{itemize}
\item This document is a licence given to a company after the submission of the memorandom of  association and articles of association (incorporation). It declares the business as incorporated, or a seperate legal entity from its owners, allowing the business and its shareholders to enjoy limited liability.
\end{itemize}

\note[item]{}
\end{frame}
\begin{frame}
\frametitle{Unrelated Title}


\begin{itemize}
\item It occurs when a business first sells all or part of the business to external stakeholders in the buinesss initial public offering (IPO)
\end{itemize}

\note[item]{}
\end{frame}
\begin{frame}
\frametitle{Unrelated Title}


\begin{itemize}
\item It is a meeting that allows its shareholders to vote on the business's operations. Typically, an AGM would include shareholders voting and electing the board of directors, inquiring about the business to executives and higher management, and approving financial accounts.
\end{itemize}

\note[item]{}
\end{frame}
\begin{frame}
\frametitle{Unrelated Title}


\begin{itemize}
\item 1. They are able to raise large amounts of capital by selling shares without habing interest charges.
\end{itemize}

\note[item]{}
\end{frame}
\begin{frame}
\frametitle{Unrelated Title}


\begin{itemize}
\item 1. Due to their larger size, interactions can feel impersonal to customers and employees
\end{itemize}

\note[item]{}
\end{frame}
\begin{frame}
\frametitle{Unrelated Title}


\begin{itemize}
\item They are revenue generating businesses with social objectivesat the core of their operations.
\end{itemize}

\note[item]{}
\end{frame}
\begin{frame}
\frametitle{Unrelated Title}


\begin{itemize}
\item They are for-profit social enterprises owned and runned by their members (e.g. employees or customers) with a goal of creating value for their members by operating in a socially responsible way. Members enjoy voting rights and a portion of profits.
\end{itemize}

\note[item]{}
\end{frame}
\begin{frame}
\frametitle{Unrelated Title}


\begin{itemize}
\item They are owned by the customers who buy goods or services for personal use. In most cases, members get access to goods and services at lower prices than those charged at traditional commercial businesses.
\end{itemize}

\note[item]{}
\end{frame}
\begin{frame}
\frametitle{Unrelated Title}


\begin{itemize}
\item They are set up, owned and organized by their employees. By operativing as an enterprise, members are provided with work
\end{itemize}

\note[item]{}
\end{frame}
\begin{frame}
\frametitle{Unrelated Title}


\begin{itemize}
\item They are cooperatives that join and support each other to process or market their products. Farmer cooperatives are a common example of producer cooperatives.
\end{itemize}

\note[item]{}
\end{frame}
\begin{frame}
\frametitle{Unrelated Title}


\begin{itemize}
\item 1. Members have a key stake in the cooperatives, and hence are more active in its success. This leads to higher motivation and productivity.
\end{itemize}

\note[item]{}
\end{frame}
\begin{frame}
\frametitle{Unrelated Title}


\begin{itemize}
\item 1. They might be inefficient managers and employees as cooperatives often don't pay high salaries and bonuses as incentives
\end{itemize}

\note[item]{}
\end{frame}
\begin{frame}
\frametitle{Unrelated Title}


\begin{itemize}
\item They are a type of financial service provided aimed at entreprenues of small businesses, who have limited means of obtaining finances.
\end{itemize}

\note[item]{}
\end{frame}
\begin{frame}
\frametitle{Unrelated Title}


\begin{itemize}
\item 1. They are accessible as many banks would not lend small amounts of money to businesses and can help those in poverty become financially independent
\end{itemize}

\note[item]{}
\end{frame}
\begin{frame}
\frametitle{Unrelated Title}


\begin{itemize}
\item Public private partnerships or public private enterprises occur when the governments works together with the private sector to jointly provide goods or services.
\end{itemize}

\note[item]{}
\end{frame}
\begin{frame}
\frametitle{Unrelated Title}


\begin{itemize}
\item NGOs or private voluntary organisations are non-profit social enterprises that operate in the private sector for the benefit of society.
\end{itemize}

\note[item]{}
\end{frame}
\begin{frame}
\frametitle{Unrelated Title}


\begin{itemize}
\item * Operational NGOs* Advocacy NGOs
\end{itemize}

\note[item]{}
\end{frame}
\begin{frame}
\frametitle{Unrelated Title}


\begin{itemize}
\item They are established for a specific objective or purpose and are often involved in relief based and community projects
\end{itemize}

\note[item]{}
\end{frame}
\begin{frame}
\frametitle{Unrelated Title}


\begin{itemize}
\item They are organisations with a more aggressive approach to promote or defend a cause and striving to raise awareness via direct action (e.g. lobbying, mass demonstrations and public relations)
\end{itemize}

\note[item]{}
\end{frame}
\begin{frame}
\frametitle{Unrelated Title}


\begin{itemize}
\item * Social benefits* Tax exemptions for NPOs* Tax incentives for donors* Limited liability* Public recognition and trust
\end{itemize}

\note[item]{}
\end{frame}
\begin{frame}
\frametitle{Unrelated Title}


\begin{itemize}
\item 1. Bureaucracy
\item 2. Disincentive effects: a lack of profit motive can cause problems for workers
\end{itemize}

\note[item]{}
\end{frame}
\begin{frame}
\frametitle{Unrelated Title}


\begin{itemize}
\item It is an abstract, broad statement that outlines an organization's aspirations or where wants to be in the future, answering the question “what we will be.” It doesn't often change and is a guiding belief of how the business should operate and acts as a source of inspiration for internal stakeholders. 
\end{itemize}

\note[item]{}
\end{frame}
\begin{frame}
\frametitle{Unrelated Title}


\begin{itemize}
\item It is a clear, concise, and specific declaration of an organization's fundamental purpose or why the business exists. It symbolises the business's goals and ambtions enabling stakeholders to understand the desired level of performance.
\end{itemize}

\note[item]{}
\end{frame}
\begin{frame}
\frametitle{Unrelated Title}


\begin{itemize}
\item They are general and long-term goals of an organization, often in vague and unquantifiables statements. They give general purpose and direction for an organization.
\end{itemize}

\note[item]{}
\end{frame}
\begin{frame}
\frametitle{Unrelated Title}


\begin{itemize}
\item They are specific, quantifiable, short to medium term targets to acheive an organization's aims. 
\end{itemize}

\note[item]{}
\end{frame}
\begin{frame}
\frametitle{Unrelated Title}


\begin{itemize}
\item They help businesses measure and control performance, motivate internal stakeholders, and direct decision making of a firm.
\end{itemize}

\note[item]{}
\end{frame}
\begin{frame}
\frametitle{Unrelated Title}


\begin{itemize}
\item They are long term plans of action used to achieve the strategic objectives of an organisation.
\end{itemize}

\note[item]{}
\end{frame}
\begin{frame}
\frametitle{Unrelated Title}


\begin{itemize}
\item They are short-term methods used to achieve an organization's tactical objective. They have specific targets and timelines and are set to facilitate strategies of an organization.
\end{itemize}

\note[item]{}
\end{frame}
\begin{frame}
\frametitle{Unrelated Title}


\begin{itemize}
\item They are short term, specific goals that guide the operation of specific functions of an organization. Some examples are survival and sales revenue maximisation.
\end{itemize}

\note[item]{}
\end{frame}
\begin{frame}
\frametitle{Unrelated Title}


\begin{itemize}
\item They are longer-term goals of a business. Some examples included: profit maximisation, growth, market standing, and image and reputation 
\end{itemize}

\note[item]{}
\end{frame}
\begin{frame}
\frametitle{Unrelated Title}


\begin{itemize}
\item 1. Corporate culture
\item 2. Type and size of organisation
\end{itemize}

\note[item]{}
\end{frame}
\begin{frame}
\frametitle{Unrelated Title}


\begin{itemize}
\item * State of economy* Government constraints* Presence of pressure groups* New technology
\end{itemize}

\note[item]{}
\end{frame}
\begin{frame}
\frametitle{Unrelated Title}


\begin{itemize}
\item They are ethical and social obligrations placed upon business to behave in an ethical and socially responsible way.
\end{itemize}

\note[item]{}
\end{frame}
\begin{frame}
\frametitle{Unrelated Title}


\begin{itemize}
\item * Self-interest attitude* Altruistic attitude* Strategic attitude
\end{itemize}

\note[item]{}
\end{frame}
\begin{frame}
\frametitle{Unrelated Title}


\begin{itemize}
\item 1. Improved corporate image2. Customer loyalty3. Cost cutting (e.g. reduce packing, less litigation costs)4. Staff morale and motivation
\end{itemize}

\note[item]{}
\end{frame}
\begin{frame}
\frametitle{Unrelated Title}


\begin{itemize}
\item 1. Compliance costs2. Lower profits3. Stakeholder conflict4. CSR as subjective
\end{itemize}

\note[item]{}
\end{frame}
\begin{frame}
\frametitle{Unrelated Title}


\begin{itemize}
\item It is a management tool used to assess the position of a business at a point in time looking at internal and external factors which hinder and benefit the business
\end{itemize}

\note[item]{}
\end{frame}
\begin{frame}
\frametitle{Unrelated Title}


\begin{itemize}
\item 1. It is quick and simple2. It has a wide range of applications3. It is useful in determing an organization's position4. It encourages foresight and proactive thinking5. It reduces the risks of decision making
\end{itemize}

\note[item]{}
\end{frame}
\begin{frame}
\frametitle{Unrelated Title}


\begin{itemize}
\item 1. It is simplistic and doesn't demand further analysis2. The model is static3. It is only useful if decision makers are open and realistic about the weaknesses4. It can't be used in isolation
\end{itemize}

\note[item]{}
\end{frame}
\begin{frame}
\frametitle{Unrelated Title}

\begin{center}
\includegraphics[width=0.9\textwidth,height=0.9\textheight,keepaspectratio]{/Users/I516998/Library/Application Support/Anki2/User 1/collection.media/Screenshot 2021-05-20 at 5.28.28 PM.png}
\end{center}

\begin{itemize}
\item It is an analytical tools used by management to choose and devise various product and market growth strategies. 
\end{itemize}

\note[item]{}
\end{frame}
\begin{frame}
\frametitle{Unrelated Title}


\begin{itemize}
\item It is a medium risk growth strategy where a firm sells new products in existing markets. It relies heavily on product extension strategies and brand development.
\end{itemize}

\note[item]{}
\end{frame}
\begin{frame}
\frametitle{Unrelated Title}


\begin{itemize}
\item It is a medium risk growth strategy where a firm sells existing products in new markets. This can be done through new distribution channels or appealing to new audiences via promotinal strategies.
\end{itemize}

\note[item]{}
\end{frame}
\begin{frame}
\frametitle{Unrelated Title}


\begin{itemize}
\item It is a high-risk growth strategy that involves selling new products in new markets. They are two types of diversification: related diversification and unrelated diversification.
\end{itemize}

\note[item]{}
\end{frame}
\begin{frame}
\frametitle{Unrelated Title}


\begin{itemize}
\item They are any person or organization with a direct interest or is affected by the acitivities and performance of a business.
\end{itemize}

\note[item]{}
\end{frame}
\begin{frame}
\frametitle{Unrelated Title}


\begin{itemize}
\item * Employees (with a goal of financial benefits, job security and progression, and improved working conditions)
\item * Managers and directors (long term financial health of organization, profit maximisation and a mazimisation of personal benefits)
\end{itemize}

\note[item]{}
\end{frame}
\begin{frame}
\frametitle{Unrelated Title}


\begin{itemize}
\item * Customers
\item * Suppliers * Pressure groups* Competitors* Government
\end{itemize}

\note[item]{}
\end{frame}
\begin{frame}
\frametitle{Unrelated Title}

\begin{center}
\includegraphics[width=0.9\textwidth,height=0.9\textheight,keepaspectratio]{/Users/I516998/Library/Application Support/Anki2/User 1/collection.media/Screenshot 2021-05-20 at 5.47.31 PM.png}
\end{center}

\begin{itemize}
\item It is a model that assess the relative interest and power of stakeholders
\end{itemize}

\note[item]{}
\end{frame}
\begin{frame}
\frametitle{Unrelated Title}


\begin{itemize}
\item They are situations where stakeholders have disagreements on matters due to differing options leading to arguments and tension
\end{itemize}

\note[item]{}
\end{frame}
\begin{frame}
\frametitle{Unrelated Title}


\begin{itemize}
\item It examines the external environment factors of a business. It looks at social, technological, economic, ethical, political, and legal, and environmental facors.
\end{itemize}

\note[item]{}
\end{frame}
\begin{frame}
\frametitle{Unrelated Title}


\begin{itemize}
\item They are cost saving benefits of oprating on a larger scale
\end{itemize}

\note[item]{}
\end{frame}
\begin{frame}
\frametitle{Unrelated Title}


\begin{itemize}
\item They arise when the unit costs increase due the size of an organization.
\end{itemize}

\note[item]{}
\end{frame}
\begin{frame}
\frametitle{Unrelated Title}


\begin{itemize}
\item AC = Total Cost (TC) / Quantity (Q)
\end{itemize}

\note[item]{}
\end{frame}
\begin{frame}
\frametitle{Unrelated Title}


\begin{itemize}
\item It is when larger firms can obtain finances easier at better lending rates due to their lower level of risk
\end{itemize}

\note[item]{}
\end{frame}
\begin{frame}
\frametitle{Unrelated Title}


\begin{itemize}
\item It is when larger organizations can benefits from specialisation of managerials positions allowing for higher productivity
\end{itemize}

\note[item]{}
\end{frame}
\begin{frame}
\frametitle{Unrelated Title}


\begin{itemize}
\item Larger firms are able to use more sophisticated machinery to product their product, which fixed costs can be psoread over a large scale of output. 
\end{itemize}

\note[item]{}
\end{frame}
\begin{frame}
\frametitle{Unrelated Title}


\begin{itemize}
\item The costs of marketing can spread of a large scale of products. Furthermore, they are able to use marketing materials on a larger amount of customers and markets
\end{itemize}

\note[item]{}
\end{frame}
\begin{frame}
\frametitle{Unrelated Title}


\begin{itemize}
\item They benefit from the division of labour and specialisation of the workforce allowing for greater productivity
\end{itemize}

\note[item]{}
\end{frame}
\begin{frame}
\frametitle{Unrelated Title}


\begin{itemize}
\item Larger firms can reduce the price of resources by gaining discounts for buying in bulk 
\end{itemize}

\note[item]{}
\end{frame}
\begin{frame}
\frametitle{Unrelated Title}


\begin{itemize}
\item Conglomerates are able to spread their risk over a range of different products and industries.
\end{itemize}

\note[item]{}
\end{frame}
\begin{frame}
\frametitle{Unrelated Title}


\begin{itemize}
\item * Techonological progress* Improved transportation networks* Skilled labour* Regional Specialisation
\end{itemize}

\note[item]{}
\end{frame}
\begin{frame}
\frametitle{Unrelated Title}


\begin{itemize}
\item 1. Lack of control and coordination2. Poor working relationships3. Lack of motivation due to repetitive tasks4. Bureaucracy5. Complacency
\end{itemize}

\note[item]{}
\end{frame}
\begin{frame}
\frametitle{Unrelated Title}


\begin{itemize}
\item * Market share* Total revenue* Size of workforce* Profit* Capital Employed
\end{itemize}

\note[item]{}
\end{frame}
\begin{frame}
\frametitle{Unrelated Title}


\begin{itemize}
\item * Easier to set up at a lower cost
\item * independence in decision making
\end{itemize}

\note[item]{}
\end{frame}
\begin{frame}
\frametitle{Unrelated Title}


\begin{itemize}
\item * Greater access to financial resources* Attract better skilled staff* Have brand recognition* Less risk* Economies of scale
\end{itemize}

\note[item]{}
\end{frame}
\begin{frame}
\frametitle{Unrelated Title}


\begin{itemize}
\item Also know as organic growth, occurs when an organization expands using its own resources
\end{itemize}

\note[item]{}
\end{frame}
\begin{frame}
\frametitle{Unrelated Title}


\begin{itemize}
\item Also known as inorganic growth, occurs when a business relies on third party organization for growthh
\end{itemize}

\note[item]{}
\end{frame}
\begin{frame}
\frametitle{Unrelated Title}


\begin{itemize}
\item 1. Changing prices2. Effective promotion3. Improved products4. Better distribution network5. Preferential credit6. Increased capital expenditure7. Improved training and development
\end{itemize}

\note[item]{}
\end{frame}
\begin{frame}
\frametitle{Unrelated Title}


\begin{itemize}
\item 1. Better control and coordination
\end{itemize}

\note[item]{}
\end{frame}
\begin{frame}
\frametitle{Unrelated Title}


\begin{itemize}
\item 1. Diseconomies of scale
\end{itemize}

\note[item]{}
\end{frame}
\begin{frame}
\frametitle{Unrelated Title}


\begin{itemize}
\item It is a form of external when two firms agree to combine and form a new company.
\end{itemize}

\note[item]{}
\end{frame}
\begin{frame}
\frametitle{Unrelated Title}


\begin{itemize}
\item It is a form of external growth when a company buys controlling interest and absorbs another firm.
\end{itemize}

\note[item]{}
\end{frame}
\begin{frame}
\frametitle{Unrelated Title}


\begin{itemize}
\item * Horizontal integration (firms operating in the same industry)
\item * Vertical integration (businesses that are at different stages of production)
\end{itemize}

\note[item]{}
\end{frame}
\begin{frame}
\frametitle{Unrelated Title}


\begin{itemize}
\item 1. Greater market share2. Economies of scales3. Synergy4. Survival5. Diversification
\end{itemize}

\note[item]{}
\end{frame}
\begin{frame}
\frametitle{Unrelated Title}


\begin{itemize}
\item It is an external source of growth that occurs when two or more businesses split costs, risk, control and rewards of a business project by setting up a new legal entity.
\end{itemize}

\note[item]{}
\end{frame}
\begin{frame}
\frametitle{Unrelated Title}


\begin{itemize}
\item 1. Synergy2. Spreading costs and risks3. Entry to foreign markets4. Relatively cheap5. Competitive advantages 6. High success rate
\end{itemize}

\note[item]{}
\end{frame}
\begin{frame}
\frametitle{Unrelated Title}


\begin{itemize}
\item 1. Clasing corporate cultures
\end{itemize}

\note[item]{}
\end{frame}
\begin{frame}
\frametitle{Unrelated Title}


\begin{itemize}
\item Similar to joint ventures, a strategic alliances where two or more businesses cooperate in a business venture for mutual benefit without the creation of a seperate legal entity.
\end{itemize}

\note[item]{}
\end{frame}
\begin{frame}
\frametitle{Unrelated Title}


\begin{itemize}
\item * Benefit from shared expertise and resources* Seperate legal entities* Synergies and economies of scale
\end{itemize}

\note[item]{}
\end{frame}
\begin{frame}
\frametitle{Unrelated Title}


\begin{itemize}
\item * Easier to enter and exit, but less stable* Can sometimes only be short term* Vulnerable to mistakes and malpractice
\end{itemize}

\note[item]{}
\end{frame}
\begin{frame}
\frametitle{Unrelated Title}


\begin{itemize}
\item A method of external growth that is a form of business ownership where a person or business buys a license to trade using another firm's name, logos, brands, and trademarks in return the franchisee (the purchasers of a franchise) pays the franchisor a royalty payment based on their sales revenue.
\end{itemize}

\note[item]{}
\end{frame}
\begin{frame}
\frametitle{Unrelated Title}


\begin{itemize}
\item 1. They can experience rapid growth without high amount of capital investments, making it cheaper than internal growth
\end{itemize}

\note[item]{}
\end{frame}
\begin{frame}
\frametitle{Unrelated Title}


\begin{itemize}
\item 1. Low risk
\end{itemize}

\note[item]{}
\end{frame}
\begin{frame}
\frametitle{Unrelated Title}


\begin{itemize}
\item 1. Risk in allowing others to use the franchisor's name as they can damage the reputation of the whole business
\end{itemize}

\note[item]{}
\end{frame}
\begin{frame}
\frametitle{Unrelated Title}


\begin{itemize}
\item 1. They can't use their own initiative to try new ideas
\end{itemize}

\note[item]{}
\end{frame}
\begin{frame}
\frametitle{Unrelated Title}


\begin{itemize}
\item It is an organization that operates, owns, or controls production and service facilities in two or more countries
\end{itemize}

\note[item]{}
\end{frame}
\begin{frame}
\frametitle{Unrelated Title}


\begin{itemize}
\item * Economies of scale
\item * Spread risks* Have powerful brand names* Exploit growth oppurtunites in untapped overseas markets* Cost-saving benefits of operating overseas
\end{itemize}

\note[item]{}
\end{frame}
\begin{frame}
\frametitle{Unrelated Title}


\begin{itemize}
\item * Provide a significant number of employment opportunities.* Provide extra revenue to local supplies and industries* Provides more choices on consumers in host countries* Can force local firms to improve operational efficiency, etc.* Transfer techinical knowledge and benchmarking practices* Added tax revenue to host economy
\end{itemize}

\note[item]{}
\end{frame}
\begin{frame}
\frametitle{Unrelated Title}


\begin{itemize}
\item * Local business may lose cutomers, market share and profit* Foreign companies may not be socially responsible* Can destory local competitors* May lead to social tension
\end{itemize}

\note[item]{}
\end{frame}
\begin{frame}
\frametitle{Unrelated Title}


\begin{itemize}
\item It is a qualitative organizational planning tool, used to identify and graphical represent of the root causes of a problem or use by management.
\end{itemize}

\note[item]{}
\end{frame}
\begin{frame}
\frametitle{Unrelated Title}

\begin{center}
\includegraphics[width=0.9\textwidth,height=0.9\textheight,keepaspectratio]{/Users/I516998/Library/Application Support/Anki2/User 1/collection.media/Screenshot 2021-05-21 at 1.15.44 PM.png}
\includegraphics[width=0.9\textwidth,height=0.9\textheight,keepaspectratio]{/Users/I516998/Library/Application Support/Anki2/User 1/collection.media/Screenshot 2021-05-21 at 1.15.54 PM.png}
\end{center}


\note[item]{}
\end{frame}
\begin{frame}
\frametitle{Unrelated Title}


\begin{itemize}
\item It is a quantitative decision making tool used as a diagrammatic reprentation of the different options that are available to a business, showing their probable outcomes by calcuating the expected values of each decision.
\end{itemize}

\note[item]{}
\end{frame}
\begin{frame}
\frametitle{Unrelated Title}

\begin{center}
\includegraphics[width=0.9\textwidth,height=0.9\textheight,keepaspectratio]{/Users/I516998/Library/Application Support/Anki2/User 1/collection.media/Screenshot 2021-05-21 at 1.18.57 PM.png}
\end{center}


\note[item]{}
\end{frame}
\begin{frame}
\frametitle{Unrelated Title}


\begin{itemize}
\item 1. It shows a variety of probable causes of a given problem
\item 2. It is easy to follow in order to determine the root causes and consequences of a problem as it is a visual tool
\end{itemize}

\note[item]{}
\end{frame}
\begin{frame}
\frametitle{Unrelated Title}


\begin{itemize}
\item 1. Causes and effects are not always interrelated, making theconstruction of the fishbone more difficult and less meaningful
\item 2. As a qualitative decision-making tool, it does not include any quantifiable data so it is difficult to ascertain how much each factor actually contributes to the problem
\end{itemize}

\note[item]{}
\end{frame}
\begin{frame}
\frametitle{Unrelated Title}


\begin{itemize}
\item 1. They allow managers to set out problems in a clear and logical manner
\end{itemize}

\note[item]{}
\end{frame}
\begin{frame}
\frametitle{Unrelated Title}


\begin{itemize}
\item 1. The probabilities given in a decision tree are only estimates andsubject to forecasting errors.
\item involved in decision-making.
\item decision is actually made, yet time lags are often inevitable in
\item the real business world.
\item can be deliberately biased to justify the preference of the
\item management.
\end{itemize}

\note[item]{}
\end{frame}
\begin{frame}
\frametitle{Unrelated Title}


\begin{itemize}
\item The force field analysis is a planning and decision making framwork used to examine driving forces, factors for the change, and restraining forces (or factors the block and hinder change). 
\end{itemize}

\note[item]{}
\end{frame}
\begin{frame}
\frametitle{Unrelated Title}

\begin{center}
\includegraphics[width=0.9\textwidth,height=0.9\textheight,keepaspectratio]{/Users/I516998/Library/Application Support/Anki2/User 1/collection.media/Screenshot 2021-05-21 at 1.30.23 PM.png}
\end{center}


\note[item]{}
\end{frame}
\begin{frame}
\frametitle{Unrelated Title}


\begin{itemize}
\item 1. As a quantitative planning tool, it makes the decision-making process more objective and logical
\item 2. It offers a simple visual representation of the forces for and against change
\end{itemize}

\note[item]{}
\end{frame}
\begin{frame}
\frametitle{Unrelated Title}


\begin{itemize}
\item 1. Qualitative factors affecting decision making are ignored and/or
difficult to quantify
2. The omission of certain driving or restraining forces can alter the
outcome quite drastically
3.The weighting of the forces is subjective, leaving room for
potential bias
\end{itemize}

\note[item]{}
\end{frame}
\begin{frame}
\frametitle{Unrelated Title}


\begin{itemize}
\item It is a management tool that plans and schedules a particular project plotted against a timeframe to help project managers determine length of the project.
\end{itemize}

\note[item]{}
\end{frame}
\begin{frame}
\frametitle{Unrelated Title}

\begin{center}
\includegraphics[width=0.9\textwidth,height=0.9\textheight,keepaspectratio]{/Users/I516998/Library/Application Support/Anki2/User 1/collection.media/Screenshot 2021-05-21 at 1.41.15 PM.png}
\end{center}


\note[item]{}
\end{frame}
\begin{frame}
\frametitle{Unrelated Title}


\begin{itemize}
\item 1. Show the dependencies between different activities in order to
minimize the time needed to complete a project
2. There are wide applications, e.g. scheduling production
processes, employee work rosters, and holiday schedules
3. Help managers to set realistic deadlines for the various activities
of a project
4. Simple to interpret and understand
5. Allow managers to monitor progress and take corrective measures
\end{itemize}

\note[item]{}
\end{frame}
\begin{frame}
\frametitle{Unrelated Title}


\begin{itemize}
\item 1. The length of time (each bar) does not necessarily correlate with the amount of work or resources involved for each activity
2. Need to be monitored and may need regular updating
3. Complex projects may be difficult to display on a one-page Gantt chart
4. Its simplicity means that a Gantt chart may not provide enough detail or information for complex projects
5. Based on and reliant on the estimates of the timings of each task
\end{itemize}

\note[item]{}
\end{frame}
\begin{frame}
\frametitle{Unrelated Title}


\begin{itemize}
\item The spending on fixed assets and capital equipment of a business. This is also known as investment expenditure.
\end{itemize}

\note[item]{}
\end{frame}
\begin{frame}
\frametitle{Unrelated Title}


\begin{itemize}
\item The need for business to finance their daily and routine operations. 
\end{itemize}

\note[item]{}
\end{frame}
\begin{frame}
\frametitle{Unrelated Title}


\begin{itemize}
\item The costs that the business must meet before it start producing and selling it productions. It is the capital needed by an entrepreneur to set up a business.
\end{itemize}

\note[item]{}
\end{frame}
\begin{frame}
\frametitle{Unrelated Title}


\begin{itemize}
\item The costs that the business must meet in the course of the daily process of producing and selling its porudctions. The capital needed to pay tfor the raw materials day-to-day running costs and credit offered to customers
\end{itemize}

\note[item]{}
\end{frame}
\begin{frame}
\frametitle{Unrelated Title}


\begin{itemize}
\item Current Assets - Current Liabilities
\end{itemize}

\note[item]{}
\end{frame}
\begin{frame}
\frametitle{Unrelated Title}


\begin{itemize}
\item 1. Expanding an existing business2. A business in difficulties
\end{itemize}

\note[item]{}
\end{frame}
\begin{frame}
\frametitle{Unrelated Title}


\begin{itemize}
\item This is money or capital. which is obtained from within the business itself. It is raised from the business's own assets or retained profits.
\end{itemize}

\note[item]{}
\end{frame}
\begin{frame}
\frametitle{Unrelated Title}


\begin{itemize}
\item This is money obtained from individuals or institutions outside of the business
\end{itemize}

\note[item]{}
\end{frame}
\begin{frame}
\frametitle{Unrelated Title}


\begin{itemize}
\item They profit working capital needed by the businesses for day-to-day operation. This are finances that are used within 12 months.
\end{itemize}

\note[item]{}
\end{frame}
\begin{frame}
\frametitle{Unrelated Title}


\begin{itemize}
\item This is finance which is available for more than 12 months and less than 5 years
\end{itemize}

\note[item]{}
\end{frame}
\begin{frame}
\frametitle{Unrelated Title}


\begin{itemize}
\item This is finance, which is available for more than 5 years. This is used fund capital expenditure, and business expansion.
\end{itemize}

\note[item]{}
\end{frame}
\begin{frame}
\frametitle{Unrelated Title}


\begin{itemize}
\item A short term source of  internal finance for sole traders and partnership that come mostly from their own personal saving.
\end{itemize}

\note[item]{}
\end{frame}
\begin{frame}
\frametitle{Unrelated Title}


\begin{itemize}
\item Retained profits refer to the surplus of funds reinvested in the business rather than being distributed as dividends. This is a long term source of internal finance.
\end{itemize}

\note[item]{}
\end{frame}
\begin{frame}
\frametitle{Unrelated Title}


\begin{itemize}
\item 1. Easy to use (few formalities)
\end{itemize}

\note[item]{}
\end{frame}
\begin{frame}
\frametitle{Unrelated Title}


\begin{itemize}
\item 1. Owner has to sacrifice their personal funds (risk)
\end{itemize}

\note[item]{}
\end{frame}
\begin{frame}
\frametitle{Unrelated Title}


\begin{itemize}
\item 1. Easy to use (few formalities)2. Does not have to be paid externally (will not require conditions, time, interest etc.)3. Owner(s) do not have to use their personal funds
\end{itemize}

\note[item]{}
\end{frame}
\begin{frame}
\frametitle{Unrelated Title}


\begin{itemize}
\item 1. Requires prior profits (cannot be used as startup capital)
\end{itemize}

\note[item]{}
\end{frame}
\begin{frame}
\frametitle{Unrelated Title}


\begin{itemize}
\item This is a short term internal source of finance where businesses can sell their dormant or fixed assets to raise finance.
\end{itemize}

\note[item]{}
\end{frame}
\begin{frame}
\frametitle{Unrelated Title}


\begin{itemize}
\item 1. Provides the business with an opportunity to dispose of fixed assets that are no longer needed 
\end{itemize}

\note[item]{}
\end{frame}
\begin{frame}
\frametitle{Unrelated Title}


\begin{itemize}
\item 1. It can compromise the firm’s ability to raise working capital in the case where there are insufficient resources for production
\end{itemize}

\note[item]{}
\end{frame}
\begin{frame}
\frametitle{Unrelated Title}


\begin{itemize}
\item It is a medium to long term source of external finance, where in a limited liability company raises money from selling off shares.
\end{itemize}

\note[item]{}
\end{frame}
\begin{frame}
\frametitle{Unrelated Title}


\begin{itemize}
\item It is a medium to long term source of external finance that is obtained from commercial finance lenders.
\end{itemize}

\note[item]{}
\end{frame}
\begin{frame}
\frametitle{Unrelated Title}


\begin{itemize}
\item 1. Permanent capital2. No interest charges
\end{itemize}

\note[item]{}
\end{frame}
\begin{frame}
\frametitle{Unrelated Title}


\begin{itemize}
\item 1. Loss of control2. Dividends will be expected
\end{itemize}

\note[item]{}
\end{frame}
\begin{frame}
\frametitle{Unrelated Title}


\begin{itemize}
\item 1. Enable the borrower to repay in regular installments, making loan capital more accessible and affordable for many businesses as it is not burdened by having to pay a large sum of money
\end{itemize}

\note[item]{}
\end{frame}
\begin{frame}
\frametitle{Unrelated Title}


\begin{itemize}
\item 1. Interest is charged on the amount of borrowed funds. The interest rate can be fixed or variable rate
\end{itemize}

\note[item]{}
\end{frame}
\begin{frame}
\frametitle{Unrelated Title}


\begin{itemize}
\item They are an external source of short term finance, that allows a business to temporarily overdraw from its account.
\end{itemize}

\note[item]{}
\end{frame}
\begin{frame}
\frametitle{Unrelated Title}


\begin{itemize}
\item They allow flexibility for businesses who face occasional cash flow problems
\end{itemize}

\note[item]{}
\end{frame}
\begin{frame}
\frametitle{Unrelated Title}


\begin{itemize}
\item 1. Can be repayable on demand without prior notice from lender
\end{itemize}

\note[item]{}
\end{frame}
\begin{frame}
\frametitle{Unrelated Title}


\begin{itemize}
\item They are a source of long term loan capital securedagainst a specific asset. Debenture holders do not have any ownership rights but usually get some interest on their investment and are paid dividends (if awarded) before shareholders receive any dividends.
\end{itemize}

\note[item]{}
\end{frame}
\begin{frame}
\frametitle{Unrelated Title}


\begin{itemize}
\item It is a long-term source of loan capital which involves the financier demanding the borrower has collateral (a fixed asset such as property that provides financial security in case the borrower fails to repay the loan).
\end{itemize}

\note[item]{}
\end{frame}
\begin{frame}
\frametitle{Unrelated Title}


\begin{itemize}
\item A common source of short term external finance that enables a business to obtain goods or services from a supplier without having to pay for these immediately. The usual trade credit period is between one and two months. Some suppliers offer a price discount for customers who pay their invoices earlier.
\end{itemize}

\note[item]{}
\end{frame}
\begin{frame}
\frametitle{Unrelated Title}


\begin{itemize}
\item 1. Allows businesses to delay payment so that they don't have to pay the full amount immediately and can instead wait for a more convenient time in the near future
\end{itemize}

\note[item]{}
\end{frame}
\begin{frame}
\frametitle{Unrelated Title}


\begin{itemize}
\item 1. Requires a strong existing relationship with the supplier
\end{itemize}

\note[item]{}
\end{frame}
\begin{frame}
\frametitle{Unrelated Title}


\begin{itemize}
\item Grants are a form of financial assistance from the government, given to qualifying businesses to aid their operations, e.g. business start-ups and R&D
(research and development). They are long term sources of external finances.
\end{itemize}

\note[item]{}
\end{frame}
\begin{frame}
\frametitle{Unrelated Title}


\begin{itemize}
\item Sometimes does not require to be paid back (free money), this is because grants are financial gifts from the government or other agency to encourage output
\end{itemize}

\note[item]{}
\end{frame}
\begin{frame}
\frametitle{Unrelated Title}


\begin{itemize}
\item 1. They are not a realistic source of finance for many firms, especially larger organisations
\end{itemize}

\note[item]{}
\end{frame}
\begin{frame}
\frametitle{Unrelated Title}


\begin{itemize}
\item They are funded by the government to lower a firm’s production costs as output provides extended benefits to society. They are sources of long term external finance.
\end{itemize}

\note[item]{}
\end{frame}
\begin{frame}
\frametitle{Unrelated Title}


\begin{itemize}
\item 1. Helps to cut production costs and increase the demand for their goods and services by being able to charge lower prices
\end{itemize}

\note[item]{}
\end{frame}
\begin{frame}
\frametitle{Unrelated Title}


\begin{itemize}
\item 1. Not easily available, even for small businesses 
\end{itemize}

\note[item]{}
\end{frame}
\begin{frame}
\frametitle{Unrelated Title}


\begin{itemize}
\item It is a financial service whereby a factor (such as a bank) collects debts on behalf of other businesses, in return for a fee, often in the form of a commision. This is a source of short term external finance.
\end{itemize}

\note[item]{}
\end{frame}
\begin{frame}
\frametitle{Unrelated Title}


\begin{itemize}
\item 1. Relatively fast source of finance
\end{itemize}

\note[item]{}
\end{frame}
\begin{frame}
\frametitle{Unrelated Title}


\begin{itemize}
\item 1. Debt factoring service will often take a cut meaning that you are not given the full amount that you are otherwise owed
\end{itemize}

\note[item]{}
\end{frame}
\begin{frame}
\frametitle{Unrelated Title}


\begin{itemize}
\item Obtaining the use of equipment or vehicles and paying a rental or leasing charge over a fixed period of time. This avoids the need for the business to raise long-term capital to buy the asset.
\end{itemize}

\note[item]{}
\end{frame}
\begin{frame}
\frametitle{Unrelated Title}


\begin{itemize}
\item 1. Release cash for other purposes within the business
\end{itemize}

\note[item]{}
\end{frame}
\begin{frame}
\frametitle{Unrelated Title}


\begin{itemize}
\item 1. The lessee never owns the asset. Ownership remains with the lessor (the leasing company) before, during and after the learning contract
\end{itemize}

\note[item]{}
\end{frame}
\begin{frame}
\frametitle{Unrelated Title}


\begin{itemize}
\item Risk capital invested in business start-ups or expanding small businesses with high potential growth, but do not find it easy to obtain finances from other sources. This is a medium term source of finance.
\end{itemize}

\note[item]{}
\end{frame}
\begin{frame}
\frametitle{Unrelated Title}


\begin{itemize}
\item 1. Useful for small businesses and inexperienced entrepreneurs who are unable to raise sufficient finance of their own, as those which are unable to secure bank loans or those which are not permitted to sell shares on a stock exchange
\end{itemize}

\note[item]{}
\end{frame}
\begin{frame}
\frametitle{Unrelated Title}


\begin{itemize}
\item 1. Use of it dilutes control and ownership for the business start-up, as investors will take a share of the business 
\end{itemize}

\note[item]{}
\end{frame}
\begin{frame}
\frametitle{Unrelated Title}


\begin{itemize}
\item They are wealthy entrepreneurs who risk their own money in small-medium businesses that have high potential growth. They are medium to long term sources of finance.
\end{itemize}

\note[item]{}
\end{frame}
\begin{frame}
\frametitle{Unrelated Title}


\begin{itemize}
\item 1. Free to make investment decisions quickly2. No interests/repayments needed
\end{itemize}

\note[item]{}
\end{frame}
\begin{frame}
\frametitle{Unrelated Title}


\begin{itemize}
\item 1. Owner loses some control to the business angel2. You’d need to give up a share of your business3. Takes longer to find a business angel investor
\end{itemize}

\note[item]{}
\end{frame}
\begin{frame}
\frametitle{Unrelated Title}


\begin{itemize}
\item They refer to the expenditure or outflows of money made by the business in the process of producing a product.
\end{itemize}

\note[item]{}
\end{frame}
\begin{frame}
\frametitle{Unrelated Title}


\begin{itemize}
\item It refers to the amount paid by the customer to purchase a product.
\end{itemize}

\note[item]{}
\end{frame}
\begin{frame}
\frametitle{Unrelated Title}


\begin{itemize}
\item It is the money a business receives from the sale of products to customers.
\end{itemize}

\note[item]{}
\end{frame}
\begin{frame}
\frametitle{Unrelated Title}


\begin{itemize}
\item It is the positive difference between revenues and costs.
\end{itemize}

\note[item]{}
\end{frame}
\begin{frame}
\frametitle{Unrelated Title}


\begin{itemize}
\item They are costs of production that a business has to pay regardless of how much it produces or sells. These costs exist even without output.
\end{itemize}

\note[item]{}
\end{frame}
\begin{frame}
\frametitle{Unrelated Title}


\begin{itemize}
\item They are costs of production that change in direct proportion to the level of output of sales.
\end{itemize}

\note[item]{}
\end{frame}
\begin{frame}
\frametitle{Unrelated Title}


\begin{itemize}
\item They are the total costs incured by the business. This is calculated by adding the total variable costs to the total fixed costs.
\end{itemize}

\note[item]{}
\end{frame}
\begin{frame}
\frametitle{Unrelated Title}


\begin{itemize}
\item They are costs that contain an element of both fixed and variable costs. They tend to change when production or sales exceed a certain level of output.
\end{itemize}

\note[item]{}
\end{frame}
\begin{frame}
\frametitle{Unrelated Title}


\begin{itemize}
\item They are costs that are directly linked to the production of a specific production, so they change with the level of output. 
\end{itemize}

\note[item]{}
\end{frame}
\begin{frame}
\frametitle{Unrelated Title}


\begin{itemize}
\item They are costs which do not directly link to the production or sale of a specfic product.
\end{itemize}

\note[item]{}
\end{frame}
\begin{frame}
\frametitle{Unrelated Title}


\begin{itemize}
\item Price * Quantity Sold
\end{itemize}

\note[item]{}
\end{frame}
\begin{frame}
\frametitle{Unrelated Title}


\begin{itemize}
\item TC = TFC + TVC
\end{itemize}

\note[item]{}
\end{frame}
\begin{frame}
\frametitle{Unrelated Title}


\begin{itemize}
\item TVC = Q * AVC
\end{itemize}

\note[item]{}
\end{frame}
\begin{frame}
\frametitle{Unrelated Title}


\begin{itemize}
\item TFC = AFC * Q
\end{itemize}

\note[item]{}
\end{frame}
\begin{frame}
\frametitle{Unrelated Title}


\begin{itemize}
\item They are means of revenue of a business. Beyond sales, this can include: advertising revenue, transaction fees, franchise costs / royalties, sponsorships, subcription fees, merchandise, dividends, donations, interest earnings and subventions.
\end{itemize}

\note[item]{}
\end{frame}
\begin{frame}
\frametitle{Unrelated Title}


\begin{itemize}
\item It refers to the sum of money that remains after all diret and variable costs have been taken away from the sales revenue
\end{itemize}

\note[item]{}
\end{frame}
\begin{frame}
\frametitle{Unrelated Title}


\begin{itemize}
\item Price - Average Variable Costs
\end{itemize}

\note[item]{}
\end{frame}
\begin{frame}
\frametitle{Unrelated Title}


\begin{itemize}
\item (P - AVC) * Q
\end{itemize}

\note[item]{}
\end{frame}
\begin{frame}
\frametitle{Unrelated Title}


\begin{itemize}
\item Total contribution - TFC orTotal Revenue- Total Costs
\end{itemize}

\note[item]{}
\end{frame}
\begin{frame}
\frametitle{Unrelated Title}


\begin{itemize}
\item It is a management tool that can be used to determine the level of sales that must be made in order to earn a profit. 
\end{itemize}

\note[item]{}
\end{frame}
\begin{frame}
\frametitle{Unrelated Title}


\begin{itemize}
\item TC = TRFixed Cost / Unit Contribution or TFC / P - AVC
\end{itemize}

\note[item]{}
\end{frame}
\begin{frame}
\frametitle{Unrelated Title}

\begin{center}
\includegraphics[width=0.9\textwidth,height=0.9\textheight,keepaspectratio]{/Users/I516998/Library/Application Support/Anki2/User 1/collection.media/Screenshot 2021-06-01 at 1.21.42 AM.png}
\end{center}


\note[item]{}
\end{frame}
\begin{frame}
\frametitle{Unrelated Title}


\begin{itemize}
\item It is where the total costs of production is equal the total revenue.
\end{itemize}

\note[item]{}
\end{frame}
\begin{frame}
\frametitle{Unrelated Title}


\begin{itemize}
\item It refers to quantity of sales or output required to break-even
\end{itemize}

\note[item]{}
\end{frame}
\begin{frame}
\frametitle{Unrelated Title}


\begin{itemize}
\item = (Fixed Cost + Target Profit) / (Price - Variable Cost per Unit)
\end{itemize}

\note[item]{}
\end{frame}
\begin{frame}
\frametitle{Unrelated Title}


\begin{itemize}
\item It is the desired or expected profit from a business.
\end{itemize}

\note[item]{}
\end{frame}
\begin{frame}
\frametitle{Unrelated Title}


\begin{itemize}
\item It shows the extent to which quantity demand exceeds the BEQ.
\end{itemize}

\note[item]{}
\end{frame}
\begin{frame}
\frametitle{Unrelated Title}


\begin{itemize}
\item Output - BEQ
\end{itemize}

\note[item]{}
\end{frame}
\begin{frame}
\frametitle{Unrelated Title}


\begin{itemize}
\item Actual Output  / Maximum Output
\end{itemize}

\note[item]{}
\end{frame}
\begin{frame}
\frametitle{Unrelated Title}


\begin{itemize}
\item 1. It is a visual tool, so it is easy to interpret2. It is useful strategic decision making tool3. Useful for analysing and predicting the impacts of changes in price
\end{itemize}

\note[item]{}
\end{frame}
\begin{frame}
\frametitle{Unrelated Title}


\begin{itemize}
\item 1. Assumes linear costs and sales revenue2. Assumes one product is being produced3. Assumes business will sell all outputs4. Ignores qualittaive issues5. Static model6. Relies on accurate information
\end{itemize}

\note[item]{}
\end{frame}
\begin{frame}
\frametitle{Unrelated Title}


\begin{itemize}
\item 1. Show the financial performance of the business
\end{itemize}

\note[item]{}
\end{frame}
\begin{frame}
\frametitle{Unrelated Title}


\begin{itemize}
\item 1. Integrity2. Objectivity3. Professional competence4. Confidentiality5. Professional behaviour
\end{itemize}

\note[item]{}
\end{frame}
\begin{frame}
\frametitle{Unrelated Title}


\begin{itemize}
\item It is a financial statement of a firm's trading activity over a period of time.
\end{itemize}

\note[item]{}
\end{frame}
\begin{frame}
\frametitle{Unrelated Title}


\begin{itemize}
\item 1. trading account2. Profit and loss account3. Appropriation Account
\end{itemize}

\note[item]{}
\end{frame}
\begin{frame}
\frametitle{Unrelated Title}


\begin{itemize}
\item = Sales Revenue - Cost of goods sold
\end{itemize}

\note[item]{}
\end{frame}
\begin{frame}
\frametitle{Unrelated Title}


\begin{itemize}
\item COGS = Opening stock + Purchases - Closing
\end{itemize}

\note[item]{}
\end{frame}
\begin{frame}
\frametitle{Unrelated Title}


\begin{itemize}
\item It is the direct costs of the goods that are actually sold.
\end{itemize}

\note[item]{}
\end{frame}
\begin{frame}
\frametitle{Unrelated Title}


\begin{itemize}
\item 1. Reduce costs2. Raise revenue
\end{itemize}

\note[item]{}
\end{frame}
\begin{frame}
\frametitle{Unrelated Title}


\begin{itemize}
\item It shows the net profit (or loss) after all costs have been deducted from the organization’s revenues.
\end{itemize}

\note[item]{}
\end{frame}
\begin{frame}
\frametitle{Unrelated Title}


\begin{itemize}
\item = Gross profit - Expenses
\end{itemize}

\note[item]{}
\end{frame}
\begin{frame}
\frametitle{Unrelated Title}


\begin{itemize}
\item 1. Reduce expenses2. Improve gross profit
\end{itemize}

\note[item]{}
\end{frame}
\begin{frame}
\frametitle{Unrelated Title}

\begin{center}
\includegraphics[width=0.9\textwidth,height=0.9\textheight,keepaspectratio]{/Users/I516998/Library/Application Support/Anki2/User 1/collection.media/Screenshot 2021-06-01 at 1.48.09 AM.png}
\end{center}


\note[item]{}
\end{frame}
\begin{frame}
\frametitle{Unrelated Title}


\begin{itemize}
\item 1. No guarantee that future performance is linked to past perfomance 
\end{itemize}

\note[item]{}
\end{frame}
\begin{frame}
\frametitle{Unrelated Title}


\begin{itemize}
\item It shows the value of an organization's assets and liabilities at a particular point in time.
\end{itemize}

\note[item]{}
\end{frame}
\begin{frame}
\frametitle{Unrelated Title}


\begin{itemize}
\item They are the item of value that a business owns.
\end{itemize}

\note[item]{}
\end{frame}
\begin{frame}
\frametitle{Unrelated Title}


\begin{itemize}
\item They are debts the business ownes to others
\end{itemize}

\note[item]{}
\end{frame}
\begin{frame}
\frametitle{Unrelated Title}


\begin{itemize}
\item They are long term assets used to product goods and service. 
\end{itemize}

\note[item]{}
\end{frame}
\begin{frame}
\frametitle{Unrelated Title}


\begin{itemize}
\item They are short-term, liquid assets of the business that are intended to be used up within a year (i.e. cash, debtors and stock)
\end{itemize}

\note[item]{}
\end{frame}
\begin{frame}
\frametitle{Unrelated Title}


\begin{itemize}
\item They are short-term debts that need to be repaid within 12 months (i.e. Overdrafts, creditors, short-term loans)
\end{itemize}

\note[item]{}
\end{frame}
\begin{frame}
\frametitle{Unrelated Title}


\begin{itemize}
\item Also known as net current assets, is the amount of money available to the business for the day-to-day operations. It is needed to fund business activity and trade or operational activities. 
\end{itemize}

\note[item]{}
\end{frame}
\begin{frame}
\frametitle{Unrelated Title}


\begin{itemize}
\item They are the overall value of a firm's assets after all liabilities are accounted for
\end{itemize}

\note[item]{}
\end{frame}
\begin{frame}
\frametitle{Unrelated Title}


\begin{itemize}
\item Total assets - Total liabilites.
\end{itemize}

\note[item]{}
\end{frame}
\begin{frame}
\frametitle{Unrelated Title}


\begin{itemize}
\item Total assets - Total liabilities = Net assets = Equity
\end{itemize}

\note[item]{}
\end{frame}
\begin{frame}
\frametitle{Unrelated Title}

\begin{center}
\includegraphics[width=0.9\textwidth,height=0.9\textheight,keepaspectratio]{/Users/I516998/Library/Application Support/Anki2/User 1/collection.media/Screenshot 2021-06-01 at 1.57.22 AM.png}
\end{center}


\note[item]{}
\end{frame}
\begin{frame}
\frametitle{Unrelated Title}


\begin{itemize}
\item They are non-physical assets that add value to an organization.
\end{itemize}

\note[item]{}
\end{frame}
\begin{frame}
\frametitle{Unrelated Title}


\begin{itemize}
\item 1. Goodwill2. Patents3. Copyright4. Trademarks5. Branding
\end{itemize}

\note[item]{}
\end{frame}
\begin{frame}
\frametitle{Unrelated Title}


\begin{itemize}
\item It is the decline in the value of a fixed asset over time, due to usage and newer models or better technologies.
\end{itemize}

\note[item]{}
\end{frame}
\begin{frame}
\frametitle{Unrelated Title}


\begin{itemize}
\item Annual depreciation = (Purchase cost - Residual Value) / Lifespan
\end{itemize}

\note[item]{}
\end{frame}
\begin{frame}
\frametitle{Unrelated Title}

\begin{center}
\includegraphics[width=0.9\textwidth,height=0.9\textheight,keepaspectratio]{/Users/I516998/Library/Application Support/Anki2/User 1/collection.media/Screenshot 2021-06-01 at 2.01.09 AM.png}
\end{center}


\note[item]{}
\end{frame}
\begin{frame}
\frametitle{Unrelated Title}


\begin{itemize}
\item Net book value * Depreciation rate
\end{itemize}

\note[item]{}
\end{frame}
\begin{frame}
\frametitle{Unrelated Title}

\begin{center}
\includegraphics[width=0.9\textwidth,height=0.9\textheight,keepaspectratio]{/Users/I516998/Library/Application Support/Anki2/User 1/collection.media/Screenshot 2021-06-01 at 2.03.06 AM.png}
\end{center}


\note[item]{}
\end{frame}
\begin{frame}
\frametitle{Unrelated Title}


\begin{itemize}
\item * Easy to calculate* Makes it easier to compare different years
\end{itemize}

\note[item]{}
\end{frame}
\begin{frame}
\frametitle{Unrelated Title}


\begin{itemize}
\item * Depreciate significantly more in an early part and thus misleading
\end{itemize}

\note[item]{}
\end{frame}
\begin{frame}
\frametitle{Unrelated Title}


\begin{itemize}
\item * More realistic
\end{itemize}

\note[item]{}
\end{frame}
\begin{frame}
\frametitle{Unrelated Title}


\begin{itemize}
\item * More time consuming
\end{itemize}

\note[item]{}
\end{frame}
\begin{frame}
\frametitle{Unrelated Title}


\begin{itemize}
\item They are quantitative management tools for analysing and judging the financial performance of a business.
\end{itemize}

\note[item]{}
\end{frame}
\begin{frame}
\frametitle{Unrelated Title}


\begin{itemize}
\item 1. Examine and assess a firm's financial position2. Compare actual figures to projected figures3. Aids decision making
\end{itemize}

\note[item]{}
\end{frame}
\begin{frame}
\frametitle{Unrelated Title}


\begin{itemize}
\item 1. Historical comparisions2. Inter-firm or competitive comparisons
\end{itemize}

\note[item]{}
\end{frame}
\begin{frame}
\frametitle{Unrelated Title}


\begin{itemize}
\item It examines profit in relation to other figures, such as ratio of profit to revenue. They are relevent to profit-seeking businesses.
\end{itemize}

\note[item]{}
\end{frame}
\begin{frame}
\frametitle{Unrelated Title}


\begin{itemize}
\item They show how well a firm's resources have been used, such as amount of profit generated from capital used.
\end{itemize}

\note[item]{}
\end{frame}
\begin{frame}
\frametitle{Unrelated Title}


\begin{itemize}
\item It is a profitability of a firm that shows the value of gross profit as a percentage of sales revenue. 
\end{itemize}

\note[item]{}
\end{frame}
\begin{frame}
\frametitle{Unrelated Title}


\begin{itemize}
\item GPM = (Gross Profit / Sales Revenue) * 100
\end{itemize}

\note[item]{}
\end{frame}
\begin{frame}
\frametitle{Unrelated Title}


\begin{itemize}
\item 1. Raise revenue2. Reduce direct costs
\end{itemize}

\note[item]{}
\end{frame}
\begin{frame}
\frametitle{Unrelated Title}


\begin{itemize}
\item It is a profitability ratio shows the percentage of sales turnover that is converted into net profit. 
\end{itemize}

\note[item]{}
\end{frame}
\begin{frame}
\frametitle{Unrelated Title}


\begin{itemize}
\item NPM = (Net profit / Sales Revenue) * 100
\end{itemize}

\note[item]{}
\end{frame}
\begin{frame}
\frametitle{Unrelated Title}


\begin{itemize}
\item 1. Reduce expenses2. reduce direct costs3. Raise revenue
\end{itemize}

\note[item]{}
\end{frame}
\begin{frame}
\frametitle{Unrelated Title}


\begin{itemize}
\item The return on capital employed
(ROCE) ratio measures a firm’s
efficiency and profitability in
relation to its size (as measured by
the firm’s capital employed).
\end{itemize}

\note[item]{}
\end{frame}
\begin{frame}
\frametitle{Unrelated Title}


\begin{itemize}
\item (Net profit before interest and tax) / Capital employed *100
\end{itemize}

\note[item]{}
\end{frame}
\begin{frame}
\frametitle{Unrelated Title}


\begin{itemize}
\item 1. Increase sales revenues2. Decrease costs of production3. Decrease expenses4. Sell unproductive assets
\end{itemize}

\note[item]{}
\end{frame}
\begin{frame}
\frametitle{Unrelated Title}


\begin{itemize}
\item They are financial ratios that look at a firm's ability to pay its debts
\end{itemize}

\note[item]{}
\end{frame}
\begin{frame}
\frametitle{Unrelated Title}


\begin{itemize}
\item It is a short term liquidity ratio which calculates the ability of a business to meet its debts within the next 12 months.
\end{itemize}

\note[item]{}
\end{frame}
\begin{frame}
\frametitle{Unrelated Title}


\begin{itemize}
\item Current Assets / Current Liabilities
\end{itemize}

\note[item]{}
\end{frame}
\begin{frame}
\frametitle{Unrelated Title}


\begin{itemize}
\item 1. Increase current assets2. Decrease liabilities
\end{itemize}

\note[item]{}
\end{frame}
\begin{frame}
\frametitle{Unrelated Title}


\begin{itemize}
\item It is a liquidity ratio that measure a firm's ability to meet its short-term debts, ignoring stock because it may be difficult to turn into cash.
\end{itemize}

\note[item]{}
\end{frame}
\begin{frame}
\frametitle{Unrelated Title}


\begin{itemize}
\item (Current Asset - stock) / Current Liabilities
\end{itemize}

\note[item]{}
\end{frame}
\begin{frame}
\frametitle{Unrelated Title}


\begin{itemize}
\item 1. Improve current assets2. Decrease current liabilities3. Improve stock control management
\end{itemize}

\note[item]{}
\end{frame}
\begin{frame}
\frametitle{Unrelated Title}


\begin{itemize}
\item 1.5 - 2.0: 1
\end{itemize}

\note[item]{}
\end{frame}
\begin{frame}
\frametitle{Unrelated Title}


\begin{itemize}
\item 1: 1
\end{itemize}

\note[item]{}
\end{frame}
\begin{frame}
\frametitle{Unrelated Title}


\begin{itemize}
\item 20%
\end{itemize}

\note[item]{}
\end{frame}
\begin{frame}
\frametitle{Unrelated Title}


\begin{itemize}
\item It measures the number of times a firm sells its stocks within a time period, usually a year. This indicates the speed at which a firm sells and replenishes all its stock.
\end{itemize}

\note[item]{}
\end{frame}
\begin{frame}
\frametitle{Unrelated Title}


\begin{itemize}
\item (number of times) = COGS / Average Stock(number of days) = Average stock / COGS * 365
\end{itemize}

\note[item]{}
\end{frame}
\begin{frame}
\frametitle{Unrelated Title}


\begin{itemize}
\item 1. Reduce inventory of finished goods2. Reduce inventories for raw materials3. Introduce just in time inventory management
\end{itemize}

\note[item]{}
\end{frame}
\begin{frame}
\frametitle{Unrelated Title}


\begin{itemize}
\item Also known as the debt collection period, is a ratio that measures the number of days it takes a firm, on average, to collect money from its debtors.
\end{itemize}

\note[item]{}
\end{frame}
\begin{frame}
\frametitle{Unrelated Title}


\begin{itemize}
\item Debtors / Sales revenue * 365
\end{itemize}

\note[item]{}
\end{frame}
\begin{frame}
\frametitle{Unrelated Title}


\begin{itemize}
\item 30 - 60 days
\end{itemize}

\note[item]{}
\end{frame}
\begin{frame}
\frametitle{Unrelated Title}


\begin{itemize}
\item 1. Impose super charages to late payers2. Give incentives to pay earlier3. Only allow cash4. Use debt factoring
\end{itemize}

\note[item]{}
\end{frame}
\begin{frame}
\frametitle{Unrelated Title}


\begin{itemize}
\item It measures the number of days  it takes, on average, for a business to pay its trade creditors.
\end{itemize}

\note[item]{}
\end{frame}
\begin{frame}
\frametitle{Unrelated Title}


\begin{itemize}
\item = Creditors / COGS * 365
\end{itemize}

\note[item]{}
\end{frame}
\begin{frame}
\frametitle{Unrelated Title}


\begin{itemize}
\item 1. Delay payments2. Request extended credit terms3. Change suppliers with extended credit terms
\end{itemize}

\note[item]{}
\end{frame}
\begin{frame}
\frametitle{Unrelated Title}


\begin{itemize}
\item It is used to assess a firm's long-term liquidity position. This is done by examining the firm's capital employed that is financed by long term debt.
\end{itemize}

\note[item]{}
\end{frame}
\begin{frame}
\frametitle{Unrelated Title}


\begin{itemize}
\item Long term liabilities / Capital employed * 100
\end{itemize}

\note[item]{}
\end{frame}
\begin{frame}
\frametitle{Unrelated Title}


\begin{itemize}
\item > 50%
\end{itemize}

\note[item]{}
\end{frame}
\begin{frame}
\frametitle{Unrelated Title}


\begin{itemize}
\item Get finance and pay back loansKeep retained profit levels high to increase capital employed
\end{itemize}

\note[item]{}
\end{frame}
\begin{frame}
\frametitle{Unrelated Title}


\begin{itemize}
\item They are cash received from the sale of goods and services
\end{itemize}

\note[item]{}
\end{frame}
\begin{frame}
\frametitle{Unrelated Title}


\begin{itemize}
\item They are used to pay for the operation costs of businesses.
\end{itemize}

\note[item]{}
\end{frame}
\begin{frame}
\frametitle{Unrelated Title}


\begin{itemize}
\item They are the difference between cash inflow and outflow
\end{itemize}

\note[item]{}
\end{frame}
\begin{frame}
\frametitle{Unrelated Title}

\begin{center}
\includegraphics[width=0.9\textwidth,height=0.9\textheight,keepaspectratio]{/Users/I516998/Library/Application Support/Anki2/User 1/collection.media/Screenshot 2021-06-01 at 2.51.46 AM.png}
\end{center}


\note[item]{}
\end{frame}
\begin{frame}
\frametitle{Unrelated Title}


\begin{itemize}
\item They are financial documents that show the expected movement of cash in and out of a business
\end{itemize}

\note[item]{}
\end{frame}
\begin{frame}
\frametitle{Unrelated Title}


\begin{itemize}
\item 1. Helps assess the financial health of the business2. Help anticipate and identify periods of liquidity problems3. Aid business planning
\end{itemize}

\note[item]{}
\end{frame}
\begin{frame}
\frametitle{Unrelated Title}

\begin{center}
\includegraphics[width=0.9\textwidth,height=0.9\textheight,keepaspectratio]{/Users/I516998/Library/Application Support/Anki2/User 1/collection.media/Screenshot 2021-06-01 at 2.53.47 AM.png}
\end{center}


\note[item]{}
\end{frame}
\begin{frame}
\frametitle{Unrelated Title}


\begin{itemize}
\item It is the amount of cash at the beginnning of a trading period. 
\end{itemize}

\note[item]{}
\end{frame}
\begin{frame}
\frametitle{Unrelated Title}


\begin{itemize}
\item It is the amount of cash at the end of a trading period.
\end{itemize}

\note[item]{}
\end{frame}
\begin{frame}
\frametitle{Unrelated Title}


\begin{itemize}
\item  Opening balance + Net cash flow
\end{itemize}

\note[item]{}
\end{frame}
\begin{frame}
\frametitle{Unrelated Title}


\begin{itemize}
\item 1. Overtrading2. Overborrowing3. Overstocking4. Poor credit control5. Unforeseen changes
\end{itemize}

\note[item]{}
\end{frame}
\begin{frame}
\frametitle{Unrelated Title}


\begin{itemize}
\item 1. Reduce cash outflows2. Improve cash inflows
\end{itemize}

\note[item]{}
\end{frame}
\begin{frame}
\frametitle{Unrelated Title}


\begin{itemize}
\item 1. Seek preferential credit terms2. Find alternative suppliers3. Better stock control4. Reduce expenses5. Leasing
\end{itemize}

\note[item]{}
\end{frame}
\begin{frame}
\frametitle{Unrelated Title}


\begin{itemize}
\item 1. Tigher credit control2. Cash payments only3. Change pricing policy4. Improve product portfolio
\end{itemize}

\note[item]{}
\end{frame}
\begin{frame}
\frametitle{Unrelated Title}


\begin{itemize}
\item 1. Marketing (inaccurate market research)2. Human resource (less productive workforce or conflicts = unfavourable effect on cash flow)3. Operations management (machine failure)4. Competitors5. Changing tastes6. Economic change
\end{itemize}

\note[item]{}
\end{frame}
\begin{frame}
\frametitle{Unrelated Title}


\begin{itemize}
\item Refers to the purchase of an asset with the potential to yield future financial benefit. 
\end{itemize}

\note[item]{}
\end{frame}
\begin{frame}
\frametitle{Unrelated Title}


\begin{itemize}
\item It is a quantitative decision making tool used to assess and justify the capital expenditure of a firm in terms of whether it will be financially worthday
\end{itemize}

\note[item]{}
\end{frame}
\begin{frame}
\frametitle{Unrelated Title}


\begin{itemize}
\item It measures the time it takes for an investement project to earn enough profit to recover the initial cost of investment
\end{itemize}

\note[item]{}
\end{frame}
\begin{frame}
\frametitle{Unrelated Title}


\begin{itemize}
\item Cost of investment / Annual net cash flow
\end{itemize}

\note[item]{}
\end{frame}
\begin{frame}
\frametitle{Unrelated Title}


\begin{itemize}
\item 1. Simply and quickest method2. Useful for firms with cash flow problems3. Allows to see whether it will break-even on the purchase of an asset4. Compare different investment projects with different costs5. Assess projects which will yield a quick return for shareholders
\end{itemize}

\note[item]{}
\end{frame}
\begin{frame}
\frametitle{Unrelated Title}


\begin{itemize}
\item 1. Contribution / month is not constant2. Only focuses on time3. Encourage a short-term approach4. PBP not suitable for some firms5. Prone to errors
\end{itemize}

\note[item]{}
\end{frame}
\begin{frame}
\frametitle{Unrelated Title}

\begin{center}
\includegraphics[width=0.9\textwidth,height=0.9\textheight,keepaspectratio]{/Users/I516998/Library/Application Support/Anki2/User 1/collection.media/Screenshot 2021-06-01 at 3.10.36 AM.png}
\end{center}


\note[item]{}
\end{frame}
\begin{frame}
\frametitle{Unrelated Title}


\begin{itemize}
\item It calculates the average profit on an investment project as a percentage of the amount invested.
\end{itemize}

\note[item]{}
\end{frame}
\begin{frame}
\frametitle{Unrelated Title}

\begin{center}
\includegraphics[width=0.9\textwidth,height=0.9\textheight,keepaspectratio]{/Users/I516998/Library/Application Support/Anki2/User 1/collection.media/Screenshot 2021-06-01 at 3.14.24 AM.png}
\end{center}


\note[item]{}
\end{frame}
\begin{frame}
\frametitle{Unrelated Title}


\begin{itemize}
\item 1. Simple and straightforward2. Focues on profitability3. Evaluate perfomrance4. Useful to compare the attractiveness of a range of different projects
\end{itemize}

\note[item]{}
\end{frame}
\begin{frame}
\frametitle{Unrelated Title}


\begin{itemize}
\item 1. Ignores timings of cash flow2. Only focuses on profits3. Only predictions
\end{itemize}

\note[item]{}
\end{frame}
\begin{frame}
\frametitle{Unrelated Title}


\begin{itemize}
\item It is an investment appraisal techinique that calculates the real value of an investment project by discounting the value of future cash flows. 
\end{itemize}

\note[item]{}
\end{frame}
\begin{frame}
\frametitle{Unrelated Title}


\begin{itemize}
\item Sum of present value - Cost
\end{itemize}

\note[item]{}
\end{frame}
\begin{frame}
\frametitle{Unrelated Title}


\begin{itemize}
\item It is a financial plan of expected revenue and expenditure for an organization, or a department within an organization, for a given time period. It can be stated in terms of financial targets such as planned sales revenues, costs, cash flow or profits. 
\end{itemize}

\note[item]{}
\end{frame}
\begin{frame}
\frametitle{Unrelated Title}


\begin{itemize}
\item 1. Planning2. Cash flow forecasting3. Prioritizing4. Controlling5. Target setting6. Accountability7. Benchmarking8. Motivating
\end{itemize}

\note[item]{}
\end{frame}
\begin{frame}
\frametitle{Unrelated Title}


\begin{itemize}
\item 1. Preparing and updating budgets can be expensive and time consuming2. If they are inflexible and unrealistic, staff may be demotivated3. It is difficult to set realistic budgets4. Corporate culture may cause budget holders may exaggerate budgets5. Budgeting is only based on forecast data6. Don't allows budgets to be carried over and thus have no incentive to save
\end{itemize}

\note[item]{}
\end{frame}
\begin{frame}
\frametitle{Unrelated Title}


\begin{itemize}
\item It is a department or division within an organization that is responsible and held accountable for its own costs. They don't generate any revenue
\end{itemize}

\note[item]{}
\end{frame}
\begin{frame}
\frametitle{Unrelated Title}


\begin{itemize}
\item they are departments or divisions within an organization that are responsible and accountable for both its costs and revenues
\end{itemize}

\note[item]{}
\end{frame}
\begin{frame}
\frametitle{Unrelated Title}


\begin{itemize}
\item 1. Monitoring and controling2. Enhancing decision making3. Motivational4. Accountability5. Coordinations6. Planning and guidance
\end{itemize}

\note[item]{}
\end{frame}
\begin{frame}
\frametitle{Unrelated Title}


\begin{itemize}
\item 1. Flexible2. Incremental3. Zero budgeting
\end{itemize}

\note[item]{}
\end{frame}
\begin{frame}
\frametitle{Unrelated Title}


\begin{itemize}
\item 1. Available finance2. Historical data3. Organizational objectives4. Benchmarking5. Negotiations
\end{itemize}

\note[item]{}
\end{frame}
\begin{frame}
\frametitle{Unrelated Title}


\begin{itemize}
\item 1. Unforeseen changes and inflexible 2. Overestimating budgets can cause complacency3. They aren't carried fowards, so surplus isn't an incentive to spend within budgets4. Senior managers, who set budgets, have no direct involvement in running the department5. Only useful for businesses with stable sales and costs6. Badly allocated budget can harm quality of products7. It is very time consuming8. Budget holders can compete and create conflict9. Ignores qualitative factors
\end{itemize}

\note[item]{}
\end{frame}
\begin{frame}
\frametitle{Unrelated Title}


\begin{itemize}
\item Common done in highly competitive markets, it is the plan to lower the cost budgers, but raise the sales budgets. Incremental budgeting uses the previous year's figure as the basis for the budget. As such, each department doesn't have to justify its whole budget. By doing this, the firm is able to achieve higher productuivty due to increased pressure. IHowever, incremental budgeting does not allow for unforeseen events.
\end{itemize}

\note[item]{}
\end{frame}
\begin{frame}
\frametitle{Unrelated Title}


\begin{itemize}
\item It is the approach of setting budgets wherein in all budget holders must justify their whole budget every year. While it is time consuming, as it requires a fundamental review of the work and significance of each budget holding section each year, it adds incentive for managers to defend the work of their sections. Futhermore, it is able to reflect the changing situations in the business world.
\end{itemize}

\note[item]{}
\end{frame}
\begin{frame}
\frametitle{Unrelated Title}


\begin{itemize}
\item 1. Having targets to work towards -> positive impact on motivation2. Compare actual perfomrnace and identify areas of strength and weakness3. Assessment and comparison of performance of divisions and managers4. Monitoring and decision making
\end{itemize}

\note[item]{}
\end{frame}
\begin{frame}
\frametitle{Unrelated Title}


\begin{itemize}
\item 1. Can create competition between profit centres2. Some indrect costs are impossible to allocate to cost and profit centres accurately causing arbitrary and inaccurate overhead cost allocations3. Reasons for good or bad performance may be due to external factors out of their control
\end{itemize}

\note[item]{}
\end{frame}
\begin{frame}
\frametitle{Unrelated Title}


\begin{itemize}
\item 1. They are forced to be more accountable to department's contribution towards costs
\end{itemize}

\note[item]{}
\end{frame}
\begin{frame}
\frametitle{Unrelated Title}


\begin{itemize}
\item 1. Allocating indirect costs is subjective. Profits can change due to arbitrary placement of costs2. Performance can be affected to factors beyond control3. Data collection needs to be accurate and thus is expensive and time consuming4. Add pressure and stress on staff -> productivity and motivational issues5. Less focus on CSR due to compliance costs6. Tension and conflict in the organization
\end{itemize}

\note[item]{}
\end{frame}
\begin{frame}
\frametitle{Unrelated Title}


\begin{itemize}
\item A variance exists when the actual outcome differs from the budgeted figures. They are usually measure monthly and are classified as favourable or adverse.
\end{itemize}

\note[item]{}
\end{frame}
\begin{frame}
\frametitle{Unrelated Title}


\begin{itemize}
\item Variance = Actual Outcome - Budgeted outcome.
\end{itemize}

\note[item]{}
\end{frame}
\begin{frame}
\frametitle{Unrelated Title}


\begin{itemize}
\item (Actual Outcome - Budgeted Figure) / Budgeted Figure * 100
\end{itemize}

\note[item]{}
\end{frame}
\begin{frame}
\frametitle{Unrelated Title}


\begin{itemize}
\item Kfix+Kvar+At+Zt
\end{itemize}

\note[item]{}
\end{frame}
\begin{frame}
\frametitle{Unrelated Title}

\begin{center}
\includegraphics[width=0.9\textwidth,height=0.9\textheight,keepaspectratio]{/Users/I516998/Library/Application Support/Anki2/User 1/collection.media/img1963991816782229408.jpg}
\end{center}


\note[item]{}
\end{frame}
\begin{frame}
\frametitle{Unrelated Title}

\begin{center}
\includegraphics[width=0.9\textwidth,height=0.9\textheight,keepaspectratio]{/Users/I516998/Library/Application Support/Anki2/User 1/collection.media/img6198499762462923278.jpg}
\end{center}


\note[item]{}
\end{frame}
\begin{frame}
\frametitle{Unrelated Title}


\begin{itemize}
\item G=U-K
\end{itemize}

\note[item]{}
\end{frame}
\begin{frame}
\frametitle{Unrelated Title}

\begin{center}
\includegraphics[width=0.9\textwidth,height=0.9\textheight,keepaspectratio]{/Users/I516998/Library/Application Support/Anki2/User 1/collection.media/img8608825578963105513.jpg}
\end{center}


\note[item]{}
\end{frame}
\begin{frame}
\frametitle{Unrelated Title}

\begin{center}
\includegraphics[width=0.9\textwidth,height=0.9\textheight,keepaspectratio]{/Users/I516998/Library/Application Support/Anki2/User 1/collection.media/img1879412642540800749.jpg}
\end{center}


\note[item]{}
\end{frame}
\begin{frame}
\frametitle{Unrelated Title}

\begin{center}
\includegraphics[width=0.9\textwidth,height=0.9\textheight,keepaspectratio]{/Users/I516998/Library/Application Support/Anki2/User 1/collection.media/img4610769368882521779.jpg}
\end{center}


\note[item]{}
\end{frame}
\begin{frame}
\frametitle{Unrelated Title}


\begin{itemize}
\item r=i1-[c01(i2-i1)/(c02-c01)]
\end{itemize}

\note[item]{}
\end{frame}
\begin{frame}
\frametitle{Unrelated Title}


\begin{itemize}
\item Niedrigerer Zinssatz
\end{itemize}

\note[item]{}
\end{frame}
\begin{frame}
\frametitle{Unrelated Title}


\begin{itemize}
\item Höherer Zinssatz 
\end{itemize}

\note[item]{}
\end{frame}
\begin{frame}
\frametitle{Unrelated Title}


\begin{itemize}
\item (positiver) Kapitalwert des niedrigeren Zinssatzes
\end{itemize}

\note[item]{}
\end{frame}
\begin{frame}
\frametitle{Unrelated Title}


\begin{itemize}
\item (Negativer)Kapitalwert des höheren Zinssatzes
\end{itemize}

\note[item]{}
\end{frame}
\begin{frame}
\frametitle{Unrelated Title}

\begin{center}
\includegraphics[width=0.9\textwidth,height=0.9\textheight,keepaspectratio]{/Users/I516998/Library/Application Support/Anki2/User 1/collection.media/img3974229742760914545.jpg}
\end{center}


\note[item]{}
\end{frame}
\begin{frame}
\frametitle{Unrelated Title}


\begin{itemize}
\item G+Zt+At
\end{itemize}

\note[item]{}
\end{frame}
\begin{frame}
\frametitle{Unrelated Title}


\begin{itemize}
\item (Gewinn+FK Zinsen)/(FK+EK)
\end{itemize}

\note[item]{}
\end{frame}
\begin{frame}
\frametitle{Unrelated Title}


\begin{itemize}
\item Gewinn/EK
\end{itemize}

\note[item]{}
\end{frame}
\begin{frame}
\frametitle{Unrelated Title}


\begin{itemize}
\item Gewinn/Umsatz
\end{itemize}

\note[item]{}
\end{frame}
\begin{frame}
\frametitle{Unrelated Title}


\begin{itemize}
\item Abschreibung auf Forderungen an Wertberichtigung auf Forderungen
\end{itemize}

\note[item]{}
\end{frame}
\begin{frame}
\frametitle{Unrelated Title}


\begin{itemize}
\item 1. Zweifelhafte Forderungen an Kundenforderungen
\item 2. Abschreibungen auf Forderungen an Zweifelhafte Forderungen
\end{itemize}

\note[item]{}
\end{frame}
\begin{frame}
\frametitle{Unrelated Title}


\begin{itemize}
\item Forderungsverluste & Umsatzsteuer an Kundenforderungen
\end{itemize}

\note[item]{}
\end{frame}
\begin{frame}
\frametitle{Unrelated Title}


\begin{itemize}
\item Bank an Erträge aus abgeschr. Ford. & USt.
\end{itemize}

\note[item]{}
\end{frame}
\begin{frame}
\frametitle{Unrelated Title}


\begin{itemize}
\item Betrag d. letzten negativen Kapitalwertes (kummuliert)/Kapitalwert der ersten Periode mit positivem kummulierten Kapitalwert
\end{itemize}

\note[item]{}
\end{frame}
\begin{frame}
\frametitle{Unrelated Title}


\begin{itemize}
\item    objektiver,beschreibender Charakter, inhaltlich überprüfbare
\end{itemize}

\note[item]{}
\end{frame}
\begin{frame}
\frametitle{Unrelated Title}


\begin{itemize}
\item subjektiver Charakter, individuelle Wertung (wie die Welt sein sollte)
\end{itemize}

\note[item]{}
\end{frame}
\begin{frame}
\frametitle{Unrelated Title}


\begin{itemize}
\item y = f(x) = ax+b-> zu erklärende Variable-> Beeinflussung durch andereVerniablen, reagiert auf Verändenung der unabhängigen Variable
\end{itemize}

\note[item]{}
\end{frame}
\begin{frame}
\frametitle{Unrelated Title}


\begin{itemize}
\item y = f(x) = ax+b-> erklärende Variable-> beeinflusst abhängige Variable
\end{itemize}

\note[item]{}
\end{frame}
\begin{frame}
\frametitle{Unrelated Title}


\begin{itemize}
\item -> nicht im Rahmen eines Modells erklärt-> äußerer Einfluss : z.B. Umweltkatastrophen, Kälte periode, Corona - Pandemie
\end{itemize}

\note[item]{}
\end{frame}
\begin{frame}
\frametitle{Unrelated Title}

\begin{center}
\includegraphics[width=0.9\textwidth,height=0.9\textheight,keepaspectratio]{/Users/I516998/Library/Application Support/Anki2/User 1/collection.media/image-64c771e8211adce57e8a6ea56aa73b8a03faeed5.png}
\end{center}

\begin{itemize}
\item -> alles was im Rahmen eines Modells erklärt wird-> wechselseitige Beziehung
\end{itemize}

\note[item]{}
\end{frame}
\begin{frame}
\frametitle{Unrelated Title}


\begin{itemize}
\item -> Erreichen eines festen Zieles mit geringen (minimalen) Ressourcen (das Ziel ist fix, Mittel variabel)z.B. Ziel: neuer Laptop.Mittel: möglichst wenig Geld
\end{itemize}

\note[item]{}
\end{frame}
\begin{frame}
\frametitle{Unrelated Title}


\begin{itemize}
\item -> Das Wirtschaftssubjekt erzielt mit gegebenen Mitteln das best mögliche Ergebnis (Mittel sind fix, Ziel variabel)z. B. Ziel: best möglicher LaptopMittel: 600 €
\end{itemize}

\note[item]{}
\end{frame}
\begin{frame}
\frametitle{Unrelated Title}


\begin{itemize}
\item -> wird bevorzugt durch höherwertiges Gut ersetzt
\end{itemize}

\note[item]{}
\end{frame}
\begin{frame}
\frametitle{Unrelated Title}


\begin{itemize}
\item -> gleichzeitig und in bestimmten Verhältnis nachgefragte Güter-> nur die Kombination stiftet dem Konsumenten einen Nutzen-> z.Bsp. Auto und Kraftstoff-> Steigt der Preis des einen Gutes, sinkt die Nachfrage nach dem anderen Gut
\end{itemize}

\note[item]{}
\end{frame}
\begin{frame}
\frametitle{Unrelated Title}


\begin{itemize}
\item -> gleichwertige, austauschbare Guter-> z.B. Salzbretzel und Salzstangen-> Steigt der Preis des einen Gutes, steigt die Nachfrage nach dem anderen Gut
\end{itemize}

\note[item]{}
\end{frame}
\begin{frame}
\frametitle{Unrelated Title}

\begin{center}
\includegraphics[width=0.9\textwidth,height=0.9\textheight,keepaspectratio]{/Users/I516998/Library/Application Support/Anki2/User 1/collection.media/image-4939730c1ac0c2af290e1660f30782aa507cf7e8.png}
\end{center}


\note[item]{}
\end{frame}
\begin{frame}
\frametitle{Unrelated Title}

\begin{center}
\includegraphics[width=0.9\textwidth,height=0.9\textheight,keepaspectratio]{/Users/I516998/Library/Application Support/Anki2/User 1/collection.media/image-7909c792e66290c955dc0a002fbbc99bc8d94da5.png}
\end{center}


\note[item]{}
\end{frame}
\begin{frame}
\frametitle{Unrelated Title}


\begin{itemize}
\item -> ist ein Maß, das uns Auskunft gibt, wie Anbieter und Nachfrager auf Veränderungen der Marktlage reagieren-> Ist ein einheitsloses Maß für die Empfindlichkeit der nachgefragtenoder der angebotenen Menge-> Im Allgemeinen misst die Elastizität die Empfindlichkeit einer Variablen im Hinblick auf eine andere.=> Sie gibt die prozentuale Änderung einer Variablen in Folge einer 1-prozentigen Änderung einer anderen Variablen an
\end{itemize}

\note[item]{}
\end{frame}
\begin{frame}
\frametitle{Unrelated Title}

\begin{center}
\includegraphics[width=0.9\textwidth,height=0.9\textheight,keepaspectratio]{/Users/I516998/Library/Application Support/Anki2/User 1/collection.media/image-36ed6232a5b0bbfd83f6ff59b15c00d51b048617.png}
\end{center}


\note[item]{}
\end{frame}
\begin{frame}
\frametitle{Unrelated Title}

\begin{center}
\includegraphics[width=0.9\textwidth,height=0.9\textheight,keepaspectratio]{/Users/I516998/Library/Application Support/Anki2/User 1/collection.media/image-b2270d0f1c0b65a6cf0c5a0a1c079674c8cf0dae.png}
\end{center}


\note[item]{}
\end{frame}
\begin{frame}
\frametitle{Unrelated Title}


\begin{itemize}
\item • Die Nachfragemenge reagiert nicht sehr stark auf Preisveränderungen• Die Preiselastizität der Nachfrage ist kleiner als 1• 0 < |E| < 1
\end{itemize}

\note[item]{}
\end{frame}
\begin{frame}
\frametitle{Unrelated Title}


\begin{itemize}
\item • Die Nachfragemenge reagiert stark auf Preisveränderungen.• Die Preiselastizität ist größer als 1 • 1 < |E| < ♾
\end{itemize}

\note[item]{}
\end{frame}
\begin{frame}
\frametitle{Unrelated Title}


\begin{itemize}
\item • Die Nachfragemenge reagiert nicht auf Preisveränderungen• |E| = 0
\end{itemize}

\note[item]{}
\end{frame}
\begin{frame}
\frametitle{Unrelated Title}


\begin{itemize}
\item • Preisveränderungen führen zu einer unendlichen Veränderung der Nachfragemenge• |E| = ♾
\end{itemize}

\note[item]{}
\end{frame}
\begin{frame}
\frametitle{Unrelated Title}


\begin{itemize}
\item • Die Nachfragemenge verändert sich um den gleichen Prozentsatz wie der Preis• |E| = 1
\end{itemize}

\note[item]{}
\end{frame}
\begin{frame}
\frametitle{Unrelated Title}

\begin{center}
\includegraphics[width=0.9\textwidth,height=0.9\textheight,keepaspectratio]{/Users/I516998/Library/Application Support/Anki2/User 1/collection.media/image-d55f55707c73d3891369c8c553659b01a3311740.png}
\end{center}

\begin{itemize}
\item Einkommenselastizität der Nachfrage = prozentuale Änderung der Nachfragemenge /prozentuale Einkommensänderung
\end{itemize}

\note[item]{}
\end{frame}
\begin{frame}
\frametitle{Unrelated Title}


\begin{itemize}
\item • positive Analyse, d.h. Analyse dessen was der Fall ist und Herstellung von Zusammenhängen zwischen verschiedenen Begriffen.
\end{itemize}

\note[item]{}
\end{frame}
\begin{frame}
\frametitle{Unrelated Title}


\begin{itemize}
\item •normative Analyse, d.h. Analyse dessen, was der Fall sein soll,• Bewertung verschiedener Allokationen, insbesondere derMarktlosung.
\end{itemize}

\note[item]{}
\end{frame}
\begin{frame}
\frametitle{Unrelated Title}


\begin{itemize}
\item -> Ökonomische Wollfahrt der Käufer durch Marktteilnahme-> Differenz zwischen Zahlungsbereitschaft und tatsächlich gezahltem Preis
\end{itemize}

\note[item]{}
\end{frame}
\begin{frame}
\frametitle{Unrelated Title}


\begin{itemize}
\item -> ökonomische Wohlfahrt der Verkäufer durch Marktteilnahme-> Differenz zwischen tatsächlichem Preis und Preis, zu dem sie zu verkaufen bereit sind
\end{itemize}

\note[item]{}
\end{frame}
\begin{frame}
\frametitle{Unrelated Title}


\begin{itemize}
\item (Nachfrageüberschuss)QD> QS.•Beispiele: Wohnungsknappheit, Benzinknappheit
\end{itemize}

\note[item]{}
\end{frame}
\begin{frame}
\frametitle{Unrelated Title}


\begin{itemize}
\item Minderung der Gesamtrente durch eine Steuer
\end{itemize}

\note[item]{}
\end{frame}
\begin{frame}
\frametitle{Unrelated Title}


\begin{itemize}
\item Eine Zahlung an Käufer und Verkäufer mit dem Ziel,die Einkommen zu erhöhen oder die Produktionskosten zu senkenund dadurch dem Empfänger der Subvention einen Vorteil zuverschaffen.
\end{itemize}

\note[item]{}
\end{frame}
\begin{frame}
\frametitle{Unrelated Title}


\begin{itemize}
\item 1. Vollständigkeit und Rangordnungsfähigkeit   •Konsumenten können Güterbündel vergleichen und sie aufreihen2. Für die meisten Güter gilt: "Mehr" ist besser als "weniger"   •Nichtsättigung und "kostenlose Entsorgung3.Transitivität   •sorgt für logisch konsistente Präferenzen und Rankings4.Je mehr ein Konsument von einem bestimmten Gut hat, desto weniger ist er bereit, ein anderes Gut aufzugeben, um noch mehr von diesem Gut zu erhalten.    •Wird auch als "abnehmender Grenznutzen" bezeichnet
\end{itemize}

\note[item]{}
\end{frame}
\begin{frame}
\frametitle{Unrelated Title}


\begin{itemize}
\item ist der zusätzliche Nutzen, den ein Konsument durch eine zusätzliche Einheit eines materiellen Gutes oder einer Dienstleistung erhält.
\end{itemize}

\note[item]{}
\end{frame}
\begin{frame}
\frametitle{Unrelated Title}

\begin{center}
\includegraphics[width=0.9\textwidth,height=0.9\textheight,keepaspectratio]{/Users/I516998/Library/Application Support/Anki2/User 1/collection.media/image-45db37487146d9d5f7917a9754f9da632f978ca7.png}
\end{center}


\note[item]{}
\end{frame}
\begin{frame}
\frametitle{Unrelated Title}

\begin{center}
\includegraphics[width=0.9\textwidth,height=0.9\textheight,keepaspectratio]{/Users/I516998/Library/Application Support/Anki2/User 1/collection.media/image-0a9614154db41c2ebd6b3d30c33a1fb425c96c48.png}
\end{center}


\note[item]{}
\end{frame}
\begin{frame}
\frametitle{Unrelated Title}

\begin{center}
\includegraphics[width=0.9\textwidth,height=0.9\textheight,keepaspectratio]{/Users/I516998/Library/Application Support/Anki2/User 1/collection.media/image-ef994671bb0c8ac6883de922a7ea246535a2d2e4.png}
\includegraphics[width=0.9\textwidth,height=0.9\textheight,keepaspectratio]{/Users/I516998/Library/Application Support/Anki2/User 1/collection.media/image-60ca01bdccb293a49e7e6deaf15572d3eaa80085.png}
\includegraphics[width=0.9\textwidth,height=0.9\textheight,keepaspectratio]{/Users/I516998/Library/Application Support/Anki2/User 1/collection.media/image-2f8649070fea63721f938a7f7a63b027f3f5c9c9.png}
\end{center}


\note[item]{}
\end{frame}
\begin{frame}
\frametitle{Unrelated Title}

\begin{center}
\includegraphics[width=0.9\textwidth,height=0.9\textheight,keepaspectratio]{/Users/I516998/Library/Application Support/Anki2/User 1/collection.media/image-433b56172a96c65f963fc6a6f9e15ecefca3c96b.png}
\end{center}


\note[item]{}
\end{frame}
\begin{frame}
\frametitle{Unrelated Title}


\begin{itemize}
\item (Realismus, Konstruktivismus,.. )-> weder zu beweisen noch zu widerlegen 
\end{itemize}

\note[item]{}
\end{frame}
\begin{frame}
\frametitle{Unrelated Title}


\begin{itemize}
\item 1. Definiendum ( Begriff, dessen Bedeutung festgelegt werden soll) 2. Definiens (Begriffe, die den Inhalt des Definiendums festlegen und begrenzen) -> weder wahr noch falsch-> Konventionen, die nichts über die Wirklichkeit aussagen -> keine Aussage über das "Wesen" von Tatbeständen 
\end{itemize}

\note[item]{}
\end{frame}
\begin{frame}
\frametitle{Unrelated Title}


\begin{itemize}
\item Begriff, dessen Bedeutung festgelegt werden soll
\end{itemize}

\note[item]{}
\end{frame}
\begin{frame}
\frametitle{Unrelated Title}


\begin{itemize}
\item Begriffe, die den Inhalt des Definiendums festlegen und begrenzen
\end{itemize}

\note[item]{}
\end{frame}
\begin{frame}
\frametitle{Unrelated Title}


\begin{itemize}
\item -> Im Definiens Begriff enthalten, deren Bedeutung nicht festgelegt ist => Folgedefinitionen nötigFolge -> unendliches definieren Abbruchkriterium -> Bedeutung der Begriffe kann als bekannt vorausgesetzt werden
\end{itemize}

\note[item]{}
\end{frame}
\begin{frame}
\frametitle{Unrelated Title}


\begin{itemize}
\item -> Definiendum ist Teil des Definiens 
\end{itemize}

\note[item]{}
\end{frame}
\begin{frame}
\frametitle{Unrelated Title}


\begin{itemize}
\item Variablen sind zusammenfassende Klassen von Prädikatoren  -> Beispiel: Geschlecht, Einkommen, Stellung im Beruf 
\end{itemize}

\note[item]{}
\end{frame}
\begin{frame}
\frametitle{Unrelated Title}


\begin{itemize}
\item -> Sonderform von Variablen-> Dispositionen sind situationsübergreifende Reaktionstendenzen Beispiel: Einstellungen, Handlungsbereitschaft, Fähigkeit 
\end{itemize}

\note[item]{}
\end{frame}
\begin{frame}
\frametitle{Unrelated Title}


\begin{itemize}
\item -> Kausalanalyse (z.B. familiäres Umfeld und Bildungserfolg-> Dimensionsanalyse (z.B Messung einer Dispositionsvariable) -> Konstruktion von Typologien ( z.B Schichten, Milieus) 
\end{itemize}

\note[item]{}
\end{frame}
\begin{frame}
\frametitle{Unrelated Title}


\begin{itemize}
\item -> Nichtzirkularität-> Präzision-> Konsistenz -> Relevanz:            -> empirisch            -> theoretisch            -> Praktisch  
\end{itemize}

\note[item]{}
\end{frame}
\begin{frame}
\frametitle{Unrelated Title}


\begin{itemize}
\item Wahrheitskriterium: Übereinstimmung mit den Tatsachen (Wirklichkeit), Korrespondenztheorie der Wirklichkeit 
\end{itemize}

\note[item]{}
\end{frame}
\begin{frame}
\frametitle{Unrelated Title}


\begin{itemize}
\item Wahrheitskriterium: Übereinstimmung mit übergeordneten Werten (Normen), wissenschaftlich keine Letztbegründung 
\end{itemize}

\note[item]{}
\end{frame}
\begin{frame}
\frametitle{Unrelated Title}


\begin{itemize}
\item >Singuläre Aussage -> raumzeitlich fixierte Aussage über Individuen und Ereignisse; auch sozialwissenschaftliche Aussagen 
\end{itemize}

\note[item]{}
\end{frame}
\begin{frame}
\frametitle{Unrelated Title}


\begin{itemize}
\item -> Deterministische Aussage (Ausnahmen sind nicht erlaubt) -> Probabilistische Aussage (Ausnahmen sind erlaubt) 
\end{itemize}

\note[item]{}
\end{frame}
\begin{frame}
\frametitle{Unrelated Title}


\begin{itemize}
\item -> gemessen an ausgeschlossenen Sachverhalten ("potenzielle Falsifikatoren") Viele potenzielle Falsifikationen -> viel Informationsgehalt => schneller scheitern an der Wirklichkeit 
\end{itemize}

\note[item]{}
\end{frame}
\begin{frame}
\frametitle{Unrelated Title}


\begin{itemize}
\item Die Deterministische Aussage-> da diese keine Ausnahmen erlaubt und leichter zu widerlegen ist, folglich mehr potentielle Falsifikatoren aufweist. Als eine probabilistische Aussage, die Ausnahmen erlaubt und so weniger potentielle Falsifikationen entstehen können.
\end{itemize}

\note[item]{}
\end{frame}
\begin{frame}
\frametitle{Unrelated Title}


\begin{itemize}
\item Je mehr Beschränkungen eine Aussage hat, desto weniger können potentielle Falsifikatoren entstehen. Folglich enthalten solche Aussagen einen geringeren Informationsgehalt 
\end{itemize}

\note[item]{}
\end{frame}
\begin{frame}
\frametitle{Unrelated Title}


\begin{itemize}
\item Recruiting, Compensation, Karriere(Weiterbildung)
\end{itemize}

\note[item]{}
\end{frame}
\begin{frame}
\frametitle{Unrelated Title}

\begin{center}
\includegraphics[width=0.9\textwidth,height=0.9\textheight,keepaspectratio]{/Users/I516998/Library/Application Support/Anki2/User 1/collection.media/paste-47e378c466a4edc8a787c751c8ab9d7aa58c84f0.jpg}
\end{center}

\begin{itemize}
\item DataInformationBusiness intelligenceKnowledge
\end{itemize}

\note[item]{}
\end{frame}
\begin{frame}
\frametitle{Unrelated Title}


\begin{itemize}
\item Structured data has a defined length, type, and format and includes numbers, dates, or strings such as Customer Address, typically stored in a a relational database or spreadsheet and accounts for about 20 percent of the data that surrounds us. Examples include sensor data, point-of-sale data, and web log data, as well as human-generated structured data like input data, click-stream data, or gaming data.Unstructured data is not defined and does not follow a specified format and is typically free-form text such as emails, Twitter tweets, and text messages and accounts for about 80 percent of the data that surrounds us. Examples include satellite images, scientific atmosphere data, and radar data, as well as human-generated text messages, social media data, and emails.Big data is a collection of large complex data sets, including structured and unstructured data, which cannot be analyzed using traditional database methods and tools.
\end{itemize}

\note[item]{}
\end{frame}
\begin{frame}
\frametitle{Unrelated Title}


\begin{itemize}
\item Data is a raw building block that has not been shaped, processed, or analyzed and frequently appears disorganized and unfriendly.
\item Information is data converted into a meaningful and useful context and structure that is valuable when making informed business decisions. A report is a document containing data organized in a table, matrix, or graphical format allowing users to easily comprehend and understand information.
\item A static report is created once based on data that does not change such as a sales report from last year or salary report from five years ago.
\item A dynamic report changes automatically during creation such as daily stock market prices or available inventory.
\end{itemize}

\note[item]{}
\end{frame}
\begin{frame}
\frametitle{Unrelated Title}


\begin{itemize}
\item Business intelligence (BI) is information collected from multiple sources such as suppliers, customers, competitors, partners, and industries that analyzes patterns, trends, and relationships for strategic decision making. BI manipulates multiple variables such as interest rates, weather conditions, and even gas prices.Analytics is the science of fact-based decision making, and business analytics is the scientific process of transforming data into insight for making better decisions.
\end{itemize}

\note[item]{}
\end{frame}
\begin{frame}
\frametitle{Unrelated Title}

\begin{center}
\includegraphics[width=0.9\textwidth,height=0.9\textheight,keepaspectratio]{/Users/I516998/Library/Application Support/Anki2/User 1/collection.media/paste-8220823c2c786f22726e27ad85811831670ec883.png}
\end{center}

\begin{itemize}
\item Analytics can range from simple reports to advanced optimization models.
\item Descriptive analytics uses techniques that describe past performance and history.
\item Predictive analytics uses techniques that extract information from data and use it to predict future trends and identify behavioral patterns.
\item Prescriptive analytics uses techniques that create models indicating the best decision to make or course of action to take.
\end{itemize}

\note[item]{}
\end{frame}
\begin{frame}
\frametitle{Unrelated Title}


\begin{itemize}
\item A business unit is a segment of a company (such as accounting, production,
marketing) representing a specific business function. The terms department,
functional area, and business unit are used interchangeably and corporations
are typically organized by business unit such as:
·        
Accounting: Records, measures,
and reports monetary transactions.
·        
Finance: Deals with strategic
financial issues including money, banking, credit, investments, and assets.
·        
Human resources: Maintains
policies, plans, and procedures for the effective management of employees.
·        
Marketing: Supports sales by
planning, pricing, and promoting goods or services.
·        
Operations management: Manages
the process of converting or transforming resources into goods or services.
·        
Sales: Performs the function of
selling goods or services.
\end{itemize}

\note[item]{}
\end{frame}
\begin{frame}
\frametitle{Unrelated Title}


\begin{itemize}
\item A self-managed computer model named after and patterned on the human body's autonomic nervous system
\end{itemize}

\note[item]{}
\end{frame}
\begin{frame}
\frametitle{Unrelated Title}


\begin{itemize}
\item The network security model necessary to accommodate the emergence of multiple perimeters and moving parts on the network and increasingly advance threats targeting enterprises
\end{itemize}

\note[item]{}
\end{frame}
\begin{frame}
\frametitle{Unrelated Title}


\begin{itemize}
\item Software that carries out some set of operations on behalf of a user or another program with some degree of independence or autonomy and employs some knowledge of representation of the user’s goals or desires
\end{itemize}

\note[item]{}
\end{frame}
\begin{frame}
\frametitle{Unrelated Title}


\begin{itemize}
\item The small program stored on a PC or portable device that monitors emails and faxes messages and phone calls
\end{itemize}

\note[item]{}
\end{frame}
\begin{frame}
\frametitle{Unrelated Title}


\begin{itemize}
\item A type of artificial intelligence that enables computers to both understand concepts in the environment and also to learn
\end{itemize}

\note[item]{}
\end{frame}
\begin{frame}
\frametitle{Unrelated Title}


\begin{itemize}
\item A concept that extends the Internet of Things emphasis on machine-to-machine communication to describe a more complex system that encompasses people and processes
\end{itemize}

\note[item]{}
\end{frame}
\begin{frame}
\frametitle{Unrelated Title}


\begin{itemize}
\item Builds three-dimensional solid objects from a digital model layer by layer using an additive process
\end{itemize}

\note[item]{}
\end{frame}
\begin{frame}
\frametitle{Unrelated Title}


\begin{itemize}
\item A band of the physical, virtual, and electronic environment creating a real-time ambient environment that changes as the user moves from one place to another
\end{itemize}

\note[item]{}
\end{frame}
\begin{frame}
\frametitle{Unrelated Title}


\begin{itemize}
\item A communication system created by linking two or more devices and establishing a standard methodology in which they can communicate
\end{itemize}

\note[item]{}
\end{frame}
\begin{frame}
\frametitle{Unrelated Title}


\begin{itemize}
\item Wireless media carry electromagnetic signals that represent the binary digits of data communications using radio or microwave frequencies. As a networking medium, wireless is not restricted to conductors or pathways, as are copper and fiber media. Wireless media provides the greatest mobility options of all media.
\end{itemize}

\note[item]{}
\end{frame}
\begin{frame}
\frametitle{Unrelated Title}


\begin{itemize}
\item Cable that can carry a wide range of frequencies with low signal loss
\end{itemize}

\note[item]{}
\end{frame}
\begin{frame}
\frametitle{Unrelated Title}


\begin{itemize}
\item The technology for the transmission of data as light impulses along a glass wire or fiber
\end{itemize}

\note[item]{}
\end{frame}
\begin{frame}
\frametitle{Unrelated Title}


\begin{itemize}
\item A cable composed of four (or more) copper wires twisted around each other within a plastic sheath
\end{itemize}

\note[item]{}
\end{frame}
\begin{frame}
\frametitle{Unrelated Title}


\begin{itemize}
\item Transmission material (usually copper) that carries electrical signals
\end{itemize}

\note[item]{}
\end{frame}
\begin{frame}
\frametitle{Unrelated Title}


\begin{itemize}
\item Various types of media used to carry electrical signals between computers
\end{itemize}

\note[item]{}
\end{frame}
\begin{frame}
\frametitle{Unrelated Title}


\begin{itemize}
\item The sixth generation Internet protocol that replaced the fifth generation Internet protocol
\end{itemize}

\note[item]{}
\end{frame}
\begin{frame}
\frametitle{Unrelated Title}


\begin{itemize}
\item Provides the technical foundation for the public internet as well as for private networks
\end{itemize}

\note[item]{}
\end{frame}
\begin{frame}
\frametitle{Unrelated Title}


\begin{itemize}
\item Ethernet is a family of wired computer networking technologies commonly used in local area networks (LAN), metropolitan area networks (MAN) and wide area networks (WAN). Systems communicating over Ethernet divide a stream of data into shorter pieces called frames.
\end{itemize}

\note[item]{}
\end{frame}
\begin{frame}
\frametitle{Unrelated Title}


\begin{itemize}
\item Capability of two or more computer systems to share data and resources even though they are made by different manufacturers
\end{itemize}

\note[item]{}
\end{frame}
\begin{frame}
\frametitle{Unrelated Title}


\begin{itemize}
\item The topological structure of a network that may be depicted physically or logically. It is an application of graph theory wherein communicating devices are modeled as nodes and the connections between the devices are modeled as links or lines between the nodes.
\end{itemize}

\note[item]{}
\end{frame}
\begin{frame}
\frametitle{Unrelated Title}


\begin{itemize}
\item A standard that specifies the format of data as well as the rules to be followed during transmission
\end{itemize}

\note[item]{}
\end{frame}
\begin{frame}
\frametitle{Unrelated Title}


\begin{itemize}
\item An intelligent connecting device that examines each packet of data it receives and then decides which way to send it towards its destination
\end{itemize}

\note[item]{}
\end{frame}
\begin{frame}
\frametitle{Unrelated Title}


\begin{itemize}
\item Occurs when the sending computer divides a message into a number of efficiently sized units called packets each of which contains the address of the destination computer
\end{itemize}

\note[item]{}
\end{frame}
\begin{frame}
\frametitle{Unrelated Title}


\begin{itemize}
\item The operating system that runs a network steering information between computers and managing security and users
\end{itemize}

\note[item]{}
\end{frame}
\begin{frame}
\frametitle{Unrelated Title}


\begin{itemize}
\item A model for applications in which the bulk of the back-end processing such as performing a physical search of a database takes place on a server while the front-end processing which involves communicating with users is handled by the clients
\end{itemize}

\note[item]{}
\end{frame}
\begin{frame}
\frametitle{Unrelated Title}


\begin{itemize}
\item A computer network that relies on the computing power and bandwidth of the participants in the network rather than centralized server
\end{itemize}

\note[item]{}
\end{frame}
\begin{frame}
\frametitle{Unrelated Title}


\begin{itemize}
\item A large computer network usually spanning a city
\end{itemize}

\note[item]{}
\end{frame}
\begin{frame}
\frametitle{Unrelated Title}


\begin{itemize}
\item Spans large geographic area such as a state province or country
\end{itemize}

\note[item]{}
\end{frame}
\begin{frame}
\frametitle{Unrelated Title}


\begin{itemize}
\item Designed to connect a group of computers in proximity to each other such as in an office building a school or a home
\end{itemize}

\note[item]{}
\end{frame}
\begin{frame}
\frametitle{Unrelated Title}


\begin{itemize}
\item Enables the transmission data over public or private networks
\end{itemize}

\note[item]{}
\end{frame}
\begin{frame}
\frametitle{Unrelated Title}


\begin{itemize}
\item Applications that use location information to provide a service
\end{itemize}

\note[item]{}
\end{frame}
\begin{frame}
\frametitle{Unrelated Title}


\begin{itemize}
\item Spatial databases in the coding process that takes a digital map feature and assigns it an attribute that serves as a unique ID (tract number, node number) or classification (soil type, zoning category)
\end{itemize}

\note[item]{}
\end{frame}
\begin{frame}
\frametitle{Unrelated Title}


\begin{itemize}
\item Identifies the geographical location of features and boundaries on earth such as natural or constructed features oceans and more
\end{itemize}

\note[item]{}
\end{frame}
\begin{frame}
\frametitle{Unrelated Title}


\begin{itemize}
\item Occurs when paper maps are laid age to age and items that run across maps, but do not match are reconfigured to match
\end{itemize}

\note[item]{}
\end{frame}
\begin{frame}
\frametitle{Unrelated Title}


\begin{itemize}
\item The science and art of making an illustrated map or chart
\end{itemize}

\note[item]{}
\end{frame}
\begin{frame}
\frametitle{Unrelated Title}


\begin{itemize}
\item The time of a day of expected arrival at a certain destination typically used for navigation applications
\end{itemize}

\note[item]{}
\end{frame}
\begin{frame}
\frametitle{Unrelated Title}


\begin{itemize}
\item The time remaining before reaching a destination using the present speed typically used for navigation applications
\end{itemize}

\note[item]{}
\end{frame}
\begin{frame}
\frametitle{Unrelated Title}


\begin{itemize}
\item A round, coin-size object that is uniquely numbered and hidden in geocache
\end{itemize}

\note[item]{}
\end{frame}
\begin{frame}
\frametitle{Unrelated Title}


\begin{itemize}
\item Represent an East West measurement of position
\end{itemize}

\note[item]{}
\end{frame}
\begin{frame}
\frametitle{Unrelated Title}


\begin{itemize}
\item Represent the north south measurement of position
\end{itemize}

\note[item]{}
\end{frame}
\begin{frame}
\frametitle{Unrelated Title}


\begin{itemize}
\item Uses GPS tracking to track vehicles
\end{itemize}

\note[item]{}
\end{frame}
\begin{frame}
\frametitle{Unrelated Title}


\begin{itemize}
\item A satellite-based navigation system providing extremely accurate position time and speed information
\end{itemize}

\note[item]{}
\end{frame}
\begin{frame}
\frametitle{Unrelated Title}


\begin{itemize}
\item Use plastic or conductive polymers instead of silicon-based microchips allow them to be washed or exposed to weather without damage the chips
\end{itemize}

\note[item]{}
\end{frame}
\begin{frame}
\frametitle{Unrelated Title}


\begin{itemize}
\item Occurs when a company places active or semi-passive RF ID tags on expensive products or asset to gather data on the item’s location with a little or no manual intervention
\end{itemize}

\note[item]{}
\end{frame}
\begin{frame}
\frametitle{Unrelated Title}


\begin{itemize}
\item Include a battery to run the microchip circuitry but communicate by drawing power from the RFID reader
\end{itemize}

\note[item]{}
\end{frame}
\begin{frame}
\frametitle{Unrelated Title}


\begin{itemize}
\item Have their own transmitter and power source typically a battery
\end{itemize}

\note[item]{}
\end{frame}
\begin{frame}
\frametitle{Unrelated Title}


\begin{itemize}
\item Do not have a power source
\end{itemize}

\note[item]{}
\end{frame}
\begin{frame}
\frametitle{Unrelated Title}


\begin{itemize}
\item An electronic identification device that is made up of a chip and antenna
\end{itemize}

\note[item]{}
\end{frame}
\begin{frame}
\frametitle{Unrelated Title}


\begin{itemize}
\item A transmitter receiver that reads the contents of RFID tags in the area
\end{itemize}

\note[item]{}
\end{frame}
\begin{frame}
\frametitle{Unrelated Title}


\begin{itemize}
\item Administrator and delivers Applications to corporate and personal smart phones and tablets
\end{itemize}

\note[item]{}
\end{frame}
\begin{frame}
\frametitle{Unrelated Title}


\begin{itemize}
\item The bleeding of personal and business use of technology devices and applications
\end{itemize}

\note[item]{}
\end{frame}
\begin{frame}
\frametitle{Unrelated Title}


\begin{itemize}
\item The practice of tagging pavement with code displayed where Wi-Fi access is available
\end{itemize}

\note[item]{}
\end{frame}
\begin{frame}
\frametitle{Unrelated Title}


\begin{itemize}
\item A wireless security protocol to protect Wi-Fi networks
\end{itemize}

\note[item]{}
\end{frame}
\begin{frame}
\frametitle{Unrelated Title}


\begin{itemize}
\item Encryption algorithm design to protect wireless transmission data
\end{itemize}

\note[item]{}
\end{frame}
\begin{frame}
\frametitle{Unrelated Title}


\begin{itemize}
\item A space station that orbits the earth receiving and transmitting signals from earth base stations over the wide area
\end{itemize}

\note[item]{}
\end{frame}
\begin{frame}
\frametitle{Unrelated Title}


\begin{itemize}
\item Allow mobile voice calls to be made over broad band networks creating under the right network conditions clear audio and fewer dropped calls
\end{itemize}

\note[item]{}
\end{frame}
\begin{frame}
\frametitle{Unrelated Title}


\begin{itemize}
\item A method of sending audio and video files over the Internet in such a way that you can view the file while it is being transferred
\end{itemize}

\note[item]{}
\end{frame}
\begin{frame}
\frametitle{Unrelated Title}


\begin{itemize}
\item Offers more advanced computing ability and connectivity than basic cell phones
\end{itemize}

\note[item]{}
\end{frame}
\begin{frame}
\frametitle{Unrelated Title}


\begin{itemize}
\item A wide area network that uses radio signals to transmit and receive data
\end{itemize}

\note[item]{}
\end{frame}
\begin{frame}
\frametitle{Unrelated Title}


\begin{itemize}
\item Communication technology and at providing high-speed wireless data over metropolitan area networks
\end{itemize}

\note[item]{}
\end{frame}
\begin{frame}
\frametitle{Unrelated Title}


\begin{itemize}
\item Metropolitan area network that uses radio signals to transmit and receive data
\end{itemize}

\note[item]{}
\end{frame}
\begin{frame}
\frametitle{Unrelated Title}


\begin{itemize}
\item A common standard for wireless networking
\end{itemize}

\note[item]{}
\end{frame}
\begin{frame}
\frametitle{Unrelated Title}


\begin{itemize}
\item An organization that researches and Institute electrical standards for communication and other technologies
\end{itemize}

\note[item]{}
\end{frame}
\begin{frame}
\frametitle{Unrelated Title}


\begin{itemize}
\item Designated locations where Wi-Fi access points are publicly available
\end{itemize}

\note[item]{}
\end{frame}
\begin{frame}
\frametitle{Unrelated Title}


\begin{itemize}
\item Include the inner workings of a Wi-Fi service or utilities including the signal transmitters towers or post along with additional equipment required to send out Wi-Fi signal
\end{itemize}

\note[item]{}
\end{frame}
\begin{frame}
\frametitle{Unrelated Title}


\begin{itemize}
\item A means by which portable devices can connect wirelessly to a local network and use an access point that send and receive data via radio waves
\end{itemize}

\note[item]{}
\end{frame}
\begin{frame}
\frametitle{Unrelated Title}


\begin{itemize}
\item Multiple transmitters and receivers allow sending and receiving greater amount of data than traditional networking devices
\end{itemize}

\note[item]{}
\end{frame}
\begin{frame}
\frametitle{Unrelated Title}


\begin{itemize}
\item Enable device to connect to a wireless network to communicate with each other
\end{itemize}

\note[item]{}
\end{frame}
\begin{frame}
\frametitle{Unrelated Title}


\begin{itemize}
\item The computer or network device that serves as the interface between devices Internet work
\end{itemize}

\note[item]{}
\end{frame}
\begin{frame}
\frametitle{Unrelated Title}


\begin{itemize}
\item A local area network that uses radio signals to transmit and receive data over distance over a few hundred feet
\end{itemize}

\note[item]{}
\end{frame}
\begin{frame}
\frametitle{Unrelated Title}


\begin{itemize}
\item Wireless PAN technology that transmits signals over short distance among cell phones and computers and other devices
\end{itemize}

\note[item]{}
\end{frame}
\begin{frame}
\frametitle{Unrelated Title}


\begin{itemize}
\item Provides communication for devices owned by a single user that work over a short distance
\end{itemize}

\note[item]{}
\end{frame}
\begin{frame}
\frametitle{Unrelated Title}


\begin{itemize}
\item Data converted into a meaningful and useful context
\end{itemize}

\note[item]{}
\end{frame}
\begin{frame}
\frametitle{Unrelated Title}


\begin{itemize}
\item A business function like accounting and human resources which moves information about people products and processes across the company to facilitate decision making and problem-solving
\end{itemize}

\note[item]{}
\end{frame}
\begin{frame}
\frametitle{Unrelated Title}


\begin{itemize}
\item Information that returns to its original transmitter input transform or output and modifies the transmitters actions
\end{itemize}

\note[item]{}
\end{frame}
\begin{frame}
\frametitle{Unrelated Title}


\begin{itemize}
\item A way of monitoring the entire system by viewing multiple inputs begin processed or transformed to produce output while continuously gathering feedback on each part
\end{itemize}

\note[item]{}
\end{frame}
\begin{frame}
\frametitle{Unrelated Title}


\begin{itemize}
\item The rate at which goods and services are produced based upon total output and total input
\end{itemize}

\note[item]{}
\end{frame}
\begin{frame}
\frametitle{Unrelated Title}


\begin{itemize}
\item The process where a business takes raw materials and processes them or converts them into finished products for it's good or service
\end{itemize}

\note[item]{}
\end{frame}
\begin{frame}
\frametitle{Unrelated Title}


\begin{itemize}
\item Task performed by people that customers will buy to satisfy a want or need
\end{itemize}

\note[item]{}
\end{frame}
\begin{frame}
\frametitle{Unrelated Title}


\begin{itemize}
\item Material items or production that customers will buy to satisfy a want or need
\end{itemize}

\note[item]{}
\end{frame}
\begin{frame}
\frametitle{Unrelated Title}


\begin{itemize}
\item A collection of parts that link to achieve a common purpose
\end{itemize}

\note[item]{}
\end{frame}
\begin{frame}
\frametitle{Unrelated Title}


\begin{itemize}
\item Occurs when a business unit is unable to freely communicate with other business unit making it difficult or impossible for organization to work cross functionally
\end{itemize}

\note[item]{}
\end{frame}
\begin{frame}
\frametitle{Unrelated Title}


\begin{itemize}
\item Individual valued for their ability to interpret and analyze information
\end{itemize}

\note[item]{}
\end{frame}
\begin{frame}
\frametitle{Unrelated Title}


\begin{itemize}
\item A segment of a company such as accounting, production, marketing representing a specific business function
\end{itemize}

\note[item]{}
\end{frame}
\begin{frame}
\frametitle{Unrelated Title}


\begin{itemize}
\item Skills experience and expertise coupled with information and intelligence that create a person’s intellectual resource
\end{itemize}

\note[item]{}
\end{frame}
\begin{frame}
\frametitle{Unrelated Title}


\begin{itemize}
\item Use data about people's behaviors to understand intent and predict future actions
\end{itemize}

\note[item]{}
\end{frame}
\begin{frame}
\frametitle{Unrelated Title}


\begin{itemize}
\item Extract information from data and use it to predict future trends is it identify behavior patterns
\end{itemize}

\note[item]{}
\end{frame}
\begin{frame}
\frametitle{Unrelated Title}


\begin{itemize}
\item The science of fact-based decision making
\end{itemize}

\note[item]{}
\end{frame}
\begin{frame}
\frametitle{Unrelated Title}


\begin{itemize}
\item A report that changes automatically during creation
\end{itemize}

\note[item]{}
\end{frame}
\begin{frame}
\frametitle{Unrelated Title}


\begin{itemize}
\item A report created once based on data that does not change
\end{itemize}

\note[item]{}
\end{frame}
\begin{frame}
\frametitle{Unrelated Title}


\begin{itemize}
\item A document containing data organized in a table matrix or graphical format allow users to easily comprehend and understand information
\end{itemize}

\note[item]{}
\end{frame}
\begin{frame}
\frametitle{Unrelated Title}


\begin{itemize}
\item A view of data at a particular moment in time
\end{itemize}

\note[item]{}
\end{frame}
\begin{frame}
\frametitle{Unrelated Title}


\begin{itemize}
\item A collection of large complex data set including structured and unstructured data which cannot be analyzed using traditional database methods and tools
\end{itemize}

\note[item]{}
\end{frame}
\begin{frame}
\frametitle{Unrelated Title}


\begin{itemize}
\item Data that is not defined and does not follow a specific format and is typical free form text such as email Twitter tweet and text messages
\end{itemize}

\note[item]{}
\end{frame}
\begin{frame}
\frametitle{Unrelated Title}


\begin{itemize}
\item Data that humans in interaction with computer generate
\end{itemize}

\note[item]{}
\end{frame}
\begin{frame}
\frametitle{Unrelated Title}


\begin{itemize}
\item Data created by a machine without human intervention
\end{itemize}

\note[item]{}
\end{frame}
\begin{frame}
\frametitle{Unrelated Title}


\begin{itemize}
\item Data that has a defined length type and format and include numbers, dates, or strains such as customer address
\end{itemize}

\note[item]{}
\end{frame}
\begin{frame}
\frametitle{Unrelated Title}


\begin{itemize}
\item Refers to devices that connect directly to other devices
\end{itemize}

\note[item]{}
\end{frame}
\begin{frame}
\frametitle{Unrelated Title}


\begin{itemize}
\item A world where interconnect Internet enabled devices or things have the ability to collect and share data without human intervention
\end{itemize}

\note[item]{}
\end{frame}
\begin{frame}
\frametitle{Unrelated Title}


\begin{itemize}
\item A time when infinite quantities of facts are widely available to anyone who can use a computer
\end{itemize}

\note[item]{}
\end{frame}
\begin{frame}
\frametitle{Unrelated Title}


\begin{itemize}
\item The confirmation or validation of an event or object
\end{itemize}

\note[item]{}
\end{frame}
\begin{frame}
\frametitle{Unrelated Title}


\begin{itemize}
\item An approach to business intelligent BI that incorporate agile software development methodologies to accelerate and improve the outcomes of BI initiatives
\end{itemize}

\note[item]{}
\end{frame}
\begin{frame}
\frametitle{Unrelated Title}


\begin{itemize}
\item The time it takes for him to comprehend the analysis results and determine an appropriate action
\end{itemize}

\note[item]{}
\end{frame}
\begin{frame}
\frametitle{Unrelated Title}


\begin{itemize}
\item The time from which data are made available to the time when analysis is complete
\end{itemize}

\note[item]{}
\end{frame}
\begin{frame}
\frametitle{Unrelated Title}


\begin{itemize}
\item Time stamps information collected at a particular frequency
\end{itemize}

\note[item]{}
\end{frame}
\begin{frame}
\frametitle{Unrelated Title}


\begin{itemize}
\item A statistical process that finds the way to make a design, system or decision as effective as possible for example finding the values of controllable variables that determine maximal productivity or minimal waste
\end{itemize}

\note[item]{}
\end{frame}
\begin{frame}
\frametitle{Unrelated Title}


\begin{itemize}
\item Describes technologies that allow users to see or visualize data to transform information into a business perspective
\end{itemize}

\note[item]{}
\end{frame}
\begin{frame}
\frametitle{Unrelated Title}


\begin{itemize}
\item Moves beyond Excel graphs and charts into sophisticated analysis techniques such as controls, instruments, maps, time-series graphs, and more
\end{itemize}

\note[item]{}
\end{frame}
\begin{frame}
\frametitle{Unrelated Title}


\begin{itemize}
\item Occurs when the user goes into an emotional state of over-analysis (or over-thinking) a situation so that decision or action is never taken in effect paralyzing the outcome.
\end{itemize}

\note[item]{}
\end{frame}
\begin{frame}
\frametitle{Unrelated Title}


\begin{itemize}
\item A business analytics specialist who uses visual tools to help people understand complex data
\end{itemize}

\note[item]{}
\end{frame}
\begin{frame}
\frametitle{Unrelated Title}


\begin{itemize}
\item Present the results of data analysis displaying the patterns relationships and trends in a graphical format
\end{itemize}

\note[item]{}
\end{frame}
\begin{frame}
\frametitle{Unrelated Title}


\begin{itemize}
\item Extracts knowledge from data by performing statistical analysis data mining and advanced analytics on big data to identify trends market changes and other relevant information
\end{itemize}

\note[item]{}
\end{frame}
\begin{frame}
\frametitle{Unrelated Title}


\begin{itemize}
\item The application of big data analytics to smaller data sets in near real or real time in order to solve a problem or create business value
\end{itemize}

\note[item]{}
\end{frame}
\begin{frame}
\frametitle{Unrelated Title}


\begin{itemize}
\item A data value that is numerically distant from most of the other data points in a set data
\end{itemize}

\note[item]{}
\end{frame}
\begin{frame}
\frametitle{Unrelated Title}


\begin{itemize}
\item The process of identifying rare or unexpected items or events in data set that do not conform to other items in the data set
\end{itemize}

\note[item]{}
\end{frame}
\begin{frame}
\frametitle{Unrelated Title}


\begin{itemize}
\item The common term for the representation of multidimensional information
\end{itemize}

\note[item]{}
\end{frame}
\begin{frame}
\frametitle{Unrelated Title}


\begin{itemize}
\item A statement about what will happen or might happen in the future for example predicting future sales or employee turnover
\end{itemize}

\note[item]{}
\end{frame}
\begin{frame}
\frametitle{Unrelated Title}


\begin{itemize}
\item A variety of techniques to find patterns and relationships in large volumes of information that predict future behavior and guide decision making
\end{itemize}

\note[item]{}
\end{frame}
\begin{frame}
\frametitle{Unrelated Title}


\begin{itemize}
\item The process of organizing data into categories or group for its most effective and efficient use
\end{itemize}

\note[item]{}
\end{frame}
\begin{frame}
\frametitle{Unrelated Title}


\begin{itemize}
\item A technique used to divide an information set into mutually exclusive groups such that the members of each group are as close together as possible to one another and the different groups are as far apart as possible
\end{itemize}

\note[item]{}
\end{frame}
\begin{frame}
\frametitle{Unrelated Title}


\begin{itemize}
\item Reveals the relationship between variables along with the nature and frequency of the relationships
\end{itemize}

\note[item]{}
\end{frame}
\begin{frame}
\frametitle{Unrelated Title}


\begin{itemize}
\item Evaluates such items as websites and checkout scanner information to detect customers buying behavior and predict future behavior by identifying affinities among customers choices of products and services
\end{itemize}

\note[item]{}
\end{frame}
\begin{frame}
\frametitle{Unrelated Title}


\begin{itemize}
\item Determine values for an unknown continuous variable behavior or estimated future value
\end{itemize}

\note[item]{}
\end{frame}
\begin{frame}
\frametitle{Unrelated Title}


\begin{itemize}
\item A data mining algorithm that analyzes a customer’s purchases and actions on a website and then uses that data to recommend complementary products
\end{itemize}

\note[item]{}
\end{frame}
\begin{frame}
\frametitle{Unrelated Title}


\begin{itemize}
\item The process of sharing information to ensure consistency between multiple data sources
\end{itemize}

\note[item]{}
\end{frame}
\begin{frame}
\frametitle{Unrelated Title}


\begin{itemize}
\item The process of collecting statistics and information about data in an existing source
\end{itemize}

\note[item]{}
\end{frame}
\begin{frame}
\frametitle{Unrelated Title}


\begin{itemize}
\item The process of analyzing data to extract information not offered by the raw data alone
\end{itemize}

\note[item]{}
\end{frame}
\begin{frame}
\frametitle{Unrelated Title}


\begin{itemize}
\item Processes and manages algorithms across many machines in a computing environment
\end{itemize}

\note[item]{}
\end{frame}
\begin{frame}
\frametitle{Unrelated Title}


\begin{itemize}
\item Equipment used to capture information and commands
\end{itemize}

\note[item]{}
\end{frame}
\begin{frame}
\frametitle{Unrelated Title}


\begin{itemize}
\item Specialty software paid for on license basis or per-use basis or usage-based licensing
\end{itemize}

\note[item]{}
\end{frame}
\begin{frame}
\frametitle{Unrelated Title}


\begin{itemize}
\item Enable any qualified users within the organization to install the software regardless of whether the computer is on a network
\end{itemize}

\note[item]{}
\end{frame}
\begin{frame}
\frametitle{Unrelated Title}


\begin{itemize}
\item Enable anyone on the network to install and use the software
\end{itemize}

\note[item]{}
\end{frame}
\begin{frame}
\frametitle{Unrelated Title}


\begin{itemize}
\item Restricts the use of the software to one user at a time
\end{itemize}

\note[item]{}
\end{frame}
\begin{frame}
\frametitle{Unrelated Title}


\begin{itemize}
\item Occurs when the software vendors release a new version of the software making significant changes to the programs
\end{itemize}

\note[item]{}
\end{frame}
\begin{frame}
\frametitle{Unrelated Title}


\begin{itemize}
\item Occurs when the software vendor release updates to software to fix problems or enhance features
\end{itemize}

\note[item]{}
\end{frame}
\begin{frame}
\frametitle{Unrelated Title}


\begin{itemize}
\item Contains course information such as a syllabus and assignments and offers drop boxes for quizzes and homework along with a grade book
\end{itemize}

\note[item]{}
\end{frame}
\begin{frame}
\frametitle{Unrelated Title}


\begin{itemize}
\item Software handles contact information appointment task lists and email
\end{itemize}

\note[item]{}
\end{frame}
\begin{frame}
\frametitle{Unrelated Title}


\begin{itemize}
\item Used for specific information processing needs including payroll customer relationship management project management training and many others
\end{itemize}

\note[item]{}
\end{frame}
\begin{frame}
\frametitle{Unrelated Title}


\begin{itemize}
\item Enables a user to return to the previous OS
\end{itemize}

\note[item]{}
\end{frame}
\begin{frame}
\frametitle{Unrelated Title}


\begin{itemize}
\item Occurs if the system is failing and will load only the most essential parts of the OS and will not run many of the background operating utilities
\end{itemize}

\note[item]{}
\end{frame}
\begin{frame}
\frametitle{Unrelated Title}


\begin{itemize}
\item Works like a wristwatch and uses a battery mounted on the motherboard to provide power when the computer is turned off
\end{itemize}

\note[item]{}
\end{frame}
\begin{frame}
\frametitle{Unrelated Title}


\begin{itemize}
\item A windows feature that provides a group of options that set default values for the Windows OS
\end{itemize}

\note[item]{}
\end{frame}
\begin{frame}
\frametitle{Unrelated Title}


\begin{itemize}
\item Provides additional functionality to the operating system
\end{itemize}

\note[item]{}
\end{frame}
\begin{frame}
\frametitle{Unrelated Title}


\begin{itemize}
\item Allows more than one piece of software to be used at a time
\end{itemize}

\note[item]{}
\end{frame}
\begin{frame}
\frametitle{Unrelated Title}


\begin{itemize}
\item Used for a single purpose in computer appliances and special purpose applications such as an automobile ATM or media player
\end{itemize}

\note[item]{}
\end{frame}
\begin{frame}
\frametitle{Unrelated Title}


\begin{itemize}
\item Provides the user with the option of choosing the OS when the computer is turned on
\end{itemize}

\note[item]{}
\end{frame}
\begin{frame}
\frametitle{Unrelated Title}


\begin{itemize}
\item Controls the application software and manages how the hardware device works together
\end{itemize}

\note[item]{}
\end{frame}
\begin{frame}
\frametitle{Unrelated Title}


\begin{itemize}
\item Controls how the various technology tools work together along with the application software
\end{itemize}

\note[item]{}
\end{frame}
\begin{frame}
\frametitle{Unrelated Title}


\begin{itemize}
\item An electronic book they can be read or a computer or special reading device
\end{itemize}

\note[item]{}
\end{frame}
\begin{frame}
\frametitle{Unrelated Title}


\begin{itemize}
\item A computer dedicated to a single function such as a Calculator or computer game
\end{itemize}

\note[item]{}
\end{frame}
\begin{frame}
\frametitle{Unrelated Title}


\begin{itemize}
\item Equipment used to send information and receive it from one location to another
\end{itemize}

\note[item]{}
\end{frame}
\begin{frame}
\frametitle{Unrelated Title}


\begin{itemize}
\item Equipment used to see hear or otherwise accept the results of information processing requests
\end{itemize}

\note[item]{}
\end{frame}
\begin{frame}
\frametitle{Unrelated Title}


\begin{itemize}
\item The set of instructions that the hardware executes to carry out specific task
\end{itemize}

\note[item]{}
\end{frame}
\begin{frame}
\frametitle{Unrelated Title}


\begin{itemize}
\item A pen like device used to tap the screen to enter command
\end{itemize}

\note[item]{}
\end{frame}
\begin{frame}
\frametitle{Unrelated Title}


\begin{itemize}
\item Input devices designed for special applications for use by people with different type of special needs
\end{itemize}

\note[item]{}
\end{frame}
\begin{frame}
\frametitle{Unrelated Title}


\begin{itemize}
\item An all-electronic storage device that is an alternative to a hard disk and is faster than hard disk because there is zero latency (no read/write head to move)
\end{itemize}

\note[item]{}
\end{frame}
\begin{frame}
\frametitle{Unrelated Title}


\begin{itemize}
\item Secondary storage medium that uses several rigid disks coated with a magnetically sensitive materials and house together with the recording head in a hermetically sealed Mechanism
\end{itemize}

\note[item]{}
\end{frame}
\begin{frame}
\frametitle{Unrelated Title}


\begin{itemize}
\item A secondary storage medium that uses a strip of 10 plastic coated with magnetic system of recording medium
\end{itemize}

\note[item]{}
\end{frame}
\begin{frame}
\frametitle{Unrelated Title}


\begin{itemize}
\item Secondary storage medium that uses magnetic techniques to store and retrieve data on this or tape coat it with magnetic sensitive material
\end{itemize}

\note[item]{}
\end{frame}
\begin{frame}
\frametitle{Unrelated Title}


\begin{itemize}
\item Roughly 1 trillion bytes
\end{itemize}

\note[item]{}
\end{frame}
\begin{frame}
\frametitle{Unrelated Title}


\begin{itemize}
\item Roughly 1 billion bytes
\end{itemize}

\note[item]{}
\end{frame}
\begin{frame}
\frametitle{Unrelated Title}


\begin{itemize}
\item Roughly 1 million bytes
\end{itemize}

\note[item]{}
\end{frame}
\begin{frame}
\frametitle{Unrelated Title}


\begin{itemize}
\item Consists of equipment designed to store large volumes of data for long term storage
\end{itemize}

\note[item]{}
\end{frame}
\begin{frame}
\frametitle{Unrelated Title}


\begin{itemize}
\item Provides nonvolatile memory for range of portable devices including computers digital cameras MP3 players and PDA
\end{itemize}

\note[item]{}
\end{frame}
\begin{frame}
\frametitle{Unrelated Title}


\begin{itemize}
\item Contains high-capacity storage that holds data such as captured images music or text files
\end{itemize}

\note[item]{}
\end{frame}
\begin{frame}
\frametitle{Unrelated Title}


\begin{itemize}
\item A special type of rewritable read only memory ROM that is compact and portable
\end{itemize}

\note[item]{}
\end{frame}
\begin{frame}
\frametitle{Unrelated Title}


\begin{itemize}
\item Dose not require constant power to function
\end{itemize}

\note[item]{}
\end{frame}
\begin{frame}
\frametitle{Unrelated Title}


\begin{itemize}
\item The portion of a computers primary storage that does not lose its contents when on switches off the power
\end{itemize}

\note[item]{}
\end{frame}
\begin{frame}
\frametitle{Unrelated Title}


\begin{itemize}
\item A small unit of ultra-fast memory that is used to store recently accessed or frequently accessed data so that the CPU does not have to retrieve this data from slower memory circuits such as well
\end{itemize}

\note[item]{}
\end{frame}
\begin{frame}
\frametitle{Unrelated Title}


\begin{itemize}
\item Must have constant power of function contents are lost when the computers electric supply fails
\end{itemize}

\note[item]{}
\end{frame}
\begin{frame}
\frametitle{Unrelated Title}


\begin{itemize}
\item Refers to RAMs complete loss of stored information is power is interrupted
\end{itemize}

\note[item]{}
\end{frame}
\begin{frame}
\frametitle{Unrelated Title}


\begin{itemize}
\item The computers primary working memory in which program instructions and data or stored so that they can be access directly by the CPU via the process has been external data bus
\end{itemize}

\note[item]{}
\end{frame}
\begin{frame}
\frametitle{Unrelated Title}


\begin{itemize}
\item Computers main memory which consists of the random-access memory (RAM) cache memory and read-only memory (ROM) that is directly accessible to the CPU
\end{itemize}

\note[item]{}
\end{frame}
\begin{frame}
\frametitle{Unrelated Title}


\begin{itemize}
\item Limit the number of instructions the CPU can execute to increase processing speed
\end{itemize}

\note[item]{}
\end{frame}
\begin{frame}
\frametitle{Unrelated Title}


\begin{itemize}
\item Type of CPU that can recognize as many as 100 or more instruction enough to carry out most computations directly
\end{itemize}

\note[item]{}
\end{frame}
\begin{frame}
\frametitle{Unrelated Title}


\begin{itemize}
\item The number of billions of CPU cycles per second
\end{itemize}

\note[item]{}
\end{frame}
\begin{frame}
\frametitle{Unrelated Title}


\begin{itemize}
\item The number of millions of CPU cycles per second
\end{itemize}

\note[item]{}
\end{frame}
\begin{frame}
\frametitle{Unrelated Title}


\begin{itemize}
\item Performs all arithmetic operations for example addition and subtraction and all logic operations such as sorting and comparing numbers
\end{itemize}

\note[item]{}
\end{frame}
\begin{frame}
\frametitle{Unrelated Title}


\begin{itemize}
\item Interprets software instructions and literally tells the other hardware devices what to do based on the software instructions
\end{itemize}

\note[item]{}
\end{frame}
\begin{frame}
\frametitle{Unrelated Title}


\begin{itemize}
\item The actual hardware that interprets and executes the program software instructions and coordinate how all the other hardware devices work together
\end{itemize}

\note[item]{}
\end{frame}
\begin{frame}
\frametitle{Unrelated Title}


\begin{itemize}
\item Electronic device operating under the control of instructions stored in its own memory that can except manipulate in store data
\end{itemize}

\note[item]{}
\end{frame}
\begin{frame}
\frametitle{Unrelated Title}


\begin{itemize}
\item Occurs when a company examines its data to determine if it can meet business expectations, while identifying possible data gaps or where missing data might exist
\end{itemize}

\note[item]{}
\end{frame}
\begin{frame}
\frametitle{Unrelated Title}


\begin{itemize}
\item Responsible for ensuring the policies and procedures are implemented across the organization and acts as a liaison between the MIS department and the business
\end{itemize}

\note[item]{}
\end{frame}
\begin{frame}
\frametitle{Unrelated Title}


\begin{itemize}
\item Refers to the overall management of the availability, usability, integrity and security of the company data
\end{itemize}

\note[item]{}
\end{frame}
\begin{frame}
\frametitle{Unrelated Title}


\begin{itemize}
\item Includes the tests and evaluations used to determine compliance with data governance polices to ensure correctness of data
\end{itemize}

\note[item]{}
\end{frame}
\begin{frame}
\frametitle{Unrelated Title}


\begin{itemize}
\item Maintains information about various types of object (inventory), events (transactions), people (employees), and places (warehouses)
\end{itemize}

\note[item]{}
\end{frame}
\begin{frame}
\frametitle{Unrelated Title}


\begin{itemize}
\item Creates, read, updates, and deletes data in a database while controlling access and security
\end{itemize}

\note[item]{}
\end{frame}
\begin{frame}
\frametitle{Unrelated Title}


\begin{itemize}
\item Users write lines of code to answer questions against a database
\end{itemize}

\note[item]{}
\end{frame}
\begin{frame}
\frametitle{Unrelated Title}


\begin{itemize}
\item Details about data
\end{itemize}

\note[item]{}
\end{frame}
\begin{frame}
\frametitle{Unrelated Title}


\begin{itemize}
\item A type of database the stores information in the form of logically related two dimensional tables.
\end{itemize}

\note[item]{}
\end{frame}
\begin{frame}
\frametitle{Unrelated Title}


\begin{itemize}
\item Allows users to create read update and delete data in a relational database
\end{itemize}

\note[item]{}
\end{frame}
\begin{frame}
\frametitle{Unrelated Title}


\begin{itemize}
\item Stores information about a person place thing transaction or event
\end{itemize}

\note[item]{}
\end{frame}
\begin{frame}
\frametitle{Unrelated Title}


\begin{itemize}
\item The data elements associated with an entity
\end{itemize}

\note[item]{}
\end{frame}
\begin{frame}
\frametitle{Unrelated Title}


\begin{itemize}
\item A collection of related data elements
\end{itemize}

\note[item]{}
\end{frame}
\begin{frame}
\frametitle{Unrelated Title}


\begin{itemize}
\item A field (or group of fields) that uniquely identifies a given entity in a table
\end{itemize}

\note[item]{}
\end{frame}
\begin{frame}
\frametitle{Unrelated Title}


\begin{itemize}
\item A primary key of one table that appears as an attribute in another table and acts to provide a logical relationship between the two tables
\end{itemize}

\note[item]{}
\end{frame}
\begin{frame}
\frametitle{Unrelated Title}


\begin{itemize}
\item The physical storage of information on a storage device such as a hard disk
\end{itemize}

\note[item]{}
\end{frame}
\begin{frame}
\frametitle{Unrelated Title}


\begin{itemize}
\item Focuses on how users logically access information to meet their particular business needs
\end{itemize}

\note[item]{}
\end{frame}
\begin{frame}
\frametitle{Unrelated Title}


\begin{itemize}
\item The time it takes for data to be stored or retrieved
\end{itemize}

\note[item]{}
\end{frame}
\begin{frame}
\frametitle{Unrelated Title}


\begin{itemize}
\item The duplication of data or the storage of the same data in multiple places
\end{itemize}

\note[item]{}
\end{frame}
\begin{frame}
\frametitle{Unrelated Title}


\begin{itemize}
\item A measure of the quality of information
\end{itemize}

\note[item]{}
\end{frame}
\begin{frame}
\frametitle{Unrelated Title}


\begin{itemize}
\item The rules that help ensure the quality of information
\end{itemize}

\note[item]{}
\end{frame}
\begin{frame}
\frametitle{Unrelated Title}


\begin{itemize}
\item Defines how a company performs a certain aspect of its business and typically results in either a yes/no or true/false answer
\end{itemize}

\note[item]{}
\end{frame}
\begin{frame}
\frametitle{Unrelated Title}


\begin{itemize}
\item Enforces business rules vital to an organization's success and often requires more insight and knowledge than relational integrity constraints
\end{itemize}

\note[item]{}
\end{frame}
\begin{frame}
\frametitle{Unrelated Title}


\begin{itemize}
\item Manages accounting data and financial processes within the enterprise with functions such as general ledger accounts payable accounts receivable budgeting and asset management
\end{itemize}

\note[item]{}
\end{frame}
\begin{frame}
\frametitle{Unrelated Title}


\begin{itemize}
\item A set of routines protocols and tools for building software applications
\end{itemize}

\note[item]{}
\end{frame}
\begin{frame}
\frametitle{Unrelated Title}


\begin{itemize}
\item Simulate human intelligence such as the ability to reason and learn
\end{itemize}

\note[item]{}
\end{frame}
\begin{frame}
\frametitle{Unrelated Title}


\begin{itemize}
\item Communication such as email in which the message and the response do not occur at the same time
\end{itemize}

\note[item]{}
\end{frame}
\begin{frame}
\frametitle{Unrelated Title}


\begin{itemize}
\item The viewing of the physical world with computer generated layers of information added to it
\end{itemize}

\note[item]{}
\end{frame}
\begin{frame}
\frametitle{Unrelated Title}


\begin{itemize}
\item An online journal that allows users to post their own comments graphics and video
\end{itemize}

\note[item]{}
\end{frame}
\begin{frame}
\frametitle{Unrelated Title}


\begin{itemize}
\item A plan that details how a company creates delivers and generates revenues
\end{itemize}

\note[item]{}
\end{frame}
\begin{frame}
\frametitle{Unrelated Title}


\begin{itemize}
\item Applies to businesses buying from and selling to each other over the internet
\end{itemize}

\note[item]{}
\end{frame}
\begin{frame}
\frametitle{Unrelated Title}


\begin{itemize}
\item Applies to any business that sells its products or services to consumers over the internet
\end{itemize}

\note[item]{}
\end{frame}
\begin{frame}
\frametitle{Unrelated Title}


\begin{itemize}
\item Gather products details and issues resolution information that can be automatically generated into a script for the representative to read to the customer
\end{itemize}

\note[item]{}
\end{frame}
\begin{frame}
\frametitle{Unrelated Title}


\begin{itemize}
\item Guides user through marketing campaigns by performing such task as campaign definition planning scheduling segmentation and success analysis
\end{itemize}

\note[item]{}
\end{frame}
\begin{frame}
\frametitle{Unrelated Title}


\begin{itemize}
\item Allows customers to click a button and talk with a representative via internet
\end{itemize}

\note[item]{}
\end{frame}
\begin{frame}
\frametitle{Unrelated Title}


\begin{itemize}
\item Any proprietary software licensed under exclusive legal right of the copyright holder
\end{itemize}

\note[item]{}
\end{frame}
\begin{frame}
\frametitle{Unrelated Title}


\begin{itemize}
\item Stores manages and processes data and applications over the internet rather than on personal computer or server
\end{itemize}

\note[item]{}
\end{frame}
\begin{frame}
\frametitle{Unrelated Title}


\begin{itemize}
\item A set of tools that supports the work of teams or groups by facilitating the sharing and flow of information
\end{itemize}

\note[item]{}
\end{frame}
\begin{frame}
\frametitle{Unrelated Title}


\begin{itemize}
\item Collaborating and tapping into the core knowledge of all employees’ partners and customers
\end{itemize}

\note[item]{}
\end{frame}
\begin{frame}
\frametitle{Unrelated Title}


\begin{itemize}
\item Applies to customers offering goods and services to each other on the internet such as EBay
\end{itemize}

\note[item]{}
\end{frame}
\begin{frame}
\frametitle{Unrelated Title}


\begin{itemize}
\item Maintains customer contact information and identifies prospective customers for future sales using tools such as organizational chart detailed customer notes and supplemental sales information
\end{itemize}

\note[item]{}
\end{frame}
\begin{frame}
\frametitle{Unrelated Title}


\begin{itemize}
\item Helps companies manage the creation storage editing and publication of their website content
\end{itemize}

\note[item]{}
\end{frame}
\begin{frame}
\frametitle{Unrelated Title}


\begin{itemize}
\item Traditional components included in most ERP systems that primarily focus on internal operations
\end{itemize}

\note[item]{}
\end{frame}
\begin{frame}
\frametitle{Unrelated Title}


\begin{itemize}
\item Selling additional products or services to an existing customer
\end{itemize}

\note[item]{}
\end{frame}
\begin{frame}
\frametitle{Unrelated Title}


\begin{itemize}
\item Sources capital for a project by raising many small amounts from a large number of individuals typically via the internet
\end{itemize}

\note[item]{}
\end{frame}
\begin{frame}
\frametitle{Unrelated Title}


\begin{itemize}
\item Refers to the wisdom of the crowd
\end{itemize}

\note[item]{}
\end{frame}
\begin{frame}
\frametitle{Unrelated Title}


\begin{itemize}
\item Divides a market into categories that share similar attributes such as age location gender habits and so on
\end{itemize}

\note[item]{}
\end{frame}
\begin{frame}
\frametitle{Unrelated Title}


\begin{itemize}
\item A part of operational CRM that automates service requests, complaints, products returns, and information requests
\end{itemize}

\note[item]{}
\end{frame}
\begin{frame}
\frametitle{Unrelated Title}


\begin{itemize}
\item Model information using OLAP which provides assistance in evaluating and choosing among a different course of action
\end{itemize}

\note[item]{}
\end{frame}
\begin{frame}
\frametitle{Unrelated Title}


\begin{itemize}
\item Tracks key performance indicators KPIs critical success factors CSF by compliance information from multiple sources and tailoring to meet your needs
\end{itemize}

\note[item]{}
\end{frame}
\begin{frame}
\frametitle{Unrelated Title}


\begin{itemize}
\item A plan that details how a company creates delivers and generates revenues on the internet
\end{itemize}

\note[item]{}
\end{frame}
\begin{frame}
\frametitle{Unrelated Title}


\begin{itemize}
\item Involves the use of strategies and technologies to transform governments by improving the delivery of services and enhancing the quality of interaction between the citizen consumer within all branches of government
\end{itemize}

\note[item]{}
\end{frame}
\begin{frame}
\frametitle{Unrelated Title}


\begin{itemize}
\item Manages the transportation and storage of goods
\end{itemize}

\note[item]{}
\end{frame}
\begin{frame}
\frametitle{Unrelated Title}


\begin{itemize}
\item Provides employees with a subset of CRM applications available through a web browser
\end{itemize}

\note[item]{}
\end{frame}
\begin{frame}
\frametitle{Unrelated Title}


\begin{itemize}
\item Represents a new approach to middleware by packaging together commonly used functionality, such as providing prebuilt link to popular enterprise applications which reduces the time necessary to develop solutions that integrate applications from multiple vendors
\end{itemize}

\note[item]{}
\end{frame}
\begin{frame}
\frametitle{Unrelated Title}


\begin{itemize}
\item The business to business (B2B) purchase and sale of supplies and services over the internet
\end{itemize}

\note[item]{}
\end{frame}
\begin{frame}
\frametitle{Unrelated Title}


\begin{itemize}
\item An online version of a retail store where customers can shop at any hour
\end{itemize}

\note[item]{}
\end{frame}
\begin{frame}
\frametitle{Unrelated Title}


\begin{itemize}
\item Specialized DSS that support senior level executives within the organization
\end{itemize}

\note[item]{}
\end{frame}
\begin{frame}
\frametitle{Unrelated Title}


\begin{itemize}
\item Computerized advisory programs that imitate the reasoning processes of experts in solving difficult problems
\end{itemize}

\note[item]{}
\end{frame}
\begin{frame}
\frametitle{Unrelated Title}


\begin{itemize}
\item Consists of anything that can be documented archived and codified often with the help of it
\end{itemize}

\note[item]{}
\end{frame}
\begin{frame}
\frametitle{Unrelated Title}


\begin{itemize}
\item The extra components that meet the organizational need not covered by the core components and primarily focus on external operations
\end{itemize}

\note[item]{}
\end{frame}
\begin{frame}
\frametitle{Unrelated Title}


\begin{itemize}
\item A magazine published only in electronic form on a computer network
\end{itemize}

\note[item]{}
\end{frame}
\begin{frame}
\frametitle{Unrelated Title}


\begin{itemize}
\item A mathematical method of handling imprecise or subjective information
\end{itemize}

\note[item]{}
\end{frame}
\begin{frame}
\frametitle{Unrelated Title}


\begin{itemize}
\item An artificial intelligence system that mimics the evolutionary survival of the fittest process to generate increasingly better solutions to a problem
\end{itemize}

\note[item]{}
\end{frame}
\begin{frame}
\frametitle{Unrelated Title}


\begin{itemize}
\item Find the input necessary to achieve a goal such as desired level of output
\end{itemize}

\note[item]{}
\end{frame}
\begin{frame}
\frametitle{Unrelated Title}


\begin{itemize}
\item Refers to the level of detail in the model or the decision-making process
\end{itemize}

\note[item]{}
\end{frame}
\begin{frame}
\frametitle{Unrelated Title}


\begin{itemize}
\item A keyword or phrase used to identify a topic and is preceded by a hash or pound sign (#)
\end{itemize}

\note[item]{}
\end{frame}
\begin{frame}
\frametitle{Unrelated Title}


\begin{itemize}
\item Tracks employee information including payroll benefits compensation and performance assessment and assures compliance with the legal requirements of multiple jurisdictions and tax authorities
\end{itemize}

\note[item]{}
\end{frame}
\begin{frame}
\frametitle{Unrelated Title}


\begin{itemize}
\item Splits the ERP functions between an on-premises ERP system and one or more functions handled as software as a service in the cloud
\end{itemize}

\note[item]{}
\end{frame}
\begin{frame}
\frametitle{Unrelated Title}


\begin{itemize}
\item The set of ideas about how all information in a given context should be organized
\end{itemize}

\note[item]{}
\end{frame}
\begin{frame}
\frametitle{Unrelated Title}


\begin{itemize}
\item A service that enables instant or real time communication between people
\end{itemize}

\note[item]{}
\end{frame}
\begin{frame}
\frametitle{Unrelated Title}


\begin{itemize}
\item A special purpose knowledge-based information system that accomplished specific tasks on behalf of its users
\end{itemize}

\note[item]{}
\end{frame}
\begin{frame}
\frametitle{Unrelated Title}


\begin{itemize}
\item Various commercial application of artificial intelligent
\end{itemize}

\note[item]{}
\end{frame}
\begin{frame}
\frametitle{Unrelated Title}


\begin{itemize}
\item A company that provides access to the internet for a monthly fee
\end{itemize}

\note[item]{}
\end{frame}
\begin{frame}
\frametitle{Unrelated Title}


\begin{itemize}
\item Involves capturing classifying evaluating retrieving and sharing information assets in a way that provides context for effective decisions and actions
\end{itemize}

\note[item]{}
\end{frame}
\begin{frame}
\frametitle{Unrelated Title}


\begin{itemize}
\item Supports the capturing organization and dissemination of knowledge (i.e., know-how) throughout an organization
\end{itemize}

\note[item]{}
\end{frame}
\begin{frame}
\frametitle{Unrelated Title}


\begin{itemize}
\item An old system that is fast approaching or beyond the end of its useful life within an organization
\end{itemize}

\note[item]{}
\end{frame}
\begin{frame}
\frametitle{Unrelated Title}


\begin{itemize}
\item Compile customer information from a variety of sources and segment it for different marketing campaigns
\end{itemize}

\note[item]{}
\end{frame}
\begin{frame}
\frametitle{Unrelated Title}


\begin{itemize}
\item Employees are constantly evaluating company operations to hone the firm’s ability to identify adapt to a leverage change
\end{itemize}

\note[item]{}
\end{frame}
\begin{frame}
\frametitle{Unrelated Title}


\begin{itemize}
\item WYSIWYGs (what you see is what you get) for mashups that provide a visual interface to build a mashup often allowing the user to drag and drop data points into a web application
\end{itemize}

\note[item]{}
\end{frame}
\begin{frame}
\frametitle{Unrelated Title}


\begin{itemize}
\item A website or web application that uses content from more than one source to create a completely new product or service
\end{itemize}

\note[item]{}
\end{frame}
\begin{frame}
\frametitle{Unrelated Title}


\begin{itemize}
\item The practice of sending brief posts (140 to 200 characters) to a personal blog either publicly or to a private group of subscribers who can read the posts as IMs or as text messages
\end{itemize}

\note[item]{}
\end{frame}
\begin{frame}
\frametitle{Unrelated Title}


\begin{itemize}
\item Several different types of software that sit in the middle of and provide connectivity between two or more software application
\end{itemize}

\note[item]{}
\end{frame}
\begin{frame}
\frametitle{Unrelated Title}


\begin{itemize}
\item The ability to purchase goods and services through a wireless internet-enabled device
\end{itemize}

\note[item]{}
\end{frame}
\begin{frame}
\frametitle{Unrelated Title}


\begin{itemize}
\item A simplified representation or abstraction of reality
\end{itemize}

\note[item]{}
\end{frame}
\begin{frame}
\frametitle{Unrelated Title}


\begin{itemize}
\item The process within a genetic algorithm of randomly trying combinations and evaluating the success or failure of the outcome
\end{itemize}

\note[item]{}
\end{frame}
\begin{frame}
\frametitle{Unrelated Title}


\begin{itemize}
\item An online marketing concept in which the advertiser attempts to gain attention by providing content in the context of the users experience in terms of its content format style or placement
\end{itemize}

\note[item]{}
\end{frame}
\begin{frame}
\frametitle{Unrelated Title}


\begin{itemize}
\item Describes how products in a network increase in value to users as the number of users increases
\end{itemize}

\note[item]{}
\end{frame}
\begin{frame}
\frametitle{Unrelated Title}


\begin{itemize}
\item A category of AI that attempts to emulate the way the human brain works
\end{itemize}

\note[item]{}
\end{frame}
\begin{frame}
\frametitle{Unrelated Title}


\begin{itemize}
\item Includes a server at a physical location using an internal network for internal access and firewall for remote user's access
\end{itemize}

\note[item]{}
\end{frame}
\begin{frame}
\frametitle{Unrelated Title}


\begin{itemize}
\item The manipulation of information to create business intelligence in support of strategy decision making
\end{itemize}

\note[item]{}
\end{frame}
\begin{frame}
\frametitle{Unrelated Title}


\begin{itemize}
\item The capture of transaction and event information using technology to (1) process the information according to the define business rules (2) to store the information and (3) update existing information to reflect the new information
\end{itemize}

\note[item]{}
\end{frame}
\begin{frame}
\frametitle{Unrelated Title}


\begin{itemize}
\item Refers to any software whose source code is made available free for any third party to review and modify
\end{itemize}

\note[item]{}
\end{frame}
\begin{frame}
\frametitle{Unrelated Title}


\begin{itemize}
\item Consists of nonproprietary hardware and software based on publicly known standards that allow third parties to create add-on products to plug into or interoperate with the system
\end{itemize}

\note[item]{}
\end{frame}
\begin{frame}
\frametitle{Unrelated Title}


\begin{itemize}
\item Supports traditional transactional processing for day-to-day front office operations or systems that deal directly with the customers
\end{itemize}

\note[item]{}
\end{frame}
\begin{frame}
\frametitle{Unrelated Title}


\begin{itemize}
\item Employees develop, control, and maintain core business activities required to run the day-to-day operations
\end{itemize}

\note[item]{}
\end{frame}
\begin{frame}
\frametitle{Unrelated Title}


\begin{itemize}
\item Targets sales opportunities by finding new customers or companies for future sales
\end{itemize}

\note[item]{}
\end{frame}
\begin{frame}
\frametitle{Unrelated Title}


\begin{itemize}
\item An extension of goal seeking analysis find the optimum value for a target variable by repeatedly changing other variables subject to specified constraints
\end{itemize}

\note[item]{}
\end{frame}
\begin{frame}
\frametitle{Unrelated Title}


\begin{itemize}
\item Focuses on keeping vendors satisfied by managing alliance partner and reseller relationship that provide customers with the optimal sales channel
\end{itemize}

\note[item]{}
\end{frame}
\begin{frame}
\frametitle{Unrelated Title}


\begin{itemize}
\item Generates revenue each time users click on a link that them directly to an online agent waiting for a call
\end{itemize}

\note[item]{}
\end{frame}
\begin{frame}
\frametitle{Unrelated Title}


\begin{itemize}
\item Generates revenue each time a user clicks on a link to retailers’ website
\end{itemize}

\note[item]{}
\end{frame}
\begin{frame}
\frametitle{Unrelated Title}


\begin{itemize}
\item Generates revenue each time a website visitor is converted to a customer
\end{itemize}

\note[item]{}
\end{frame}
\begin{frame}
\frametitle{Unrelated Title}


\begin{itemize}
\item Converts an audio broadcast to a digital music player
\end{itemize}

\note[item]{}
\end{frame}
\begin{frame}
\frametitle{Unrelated Title}


\begin{itemize}
\item Handles the various aspects of production planning and execution such as demand forecasting production scheduling job cost accounting and quality control
\end{itemize}

\note[item]{}
\end{frame}
\begin{frame}
\frametitle{Unrelated Title}


\begin{itemize}
\item A web format used to publish frequently updated works such as blogs news headlines audio and video in a standardized format
\end{itemize}

\note[item]{}
\end{frame}
\begin{frame}
\frametitle{Unrelated Title}


\begin{itemize}
\item Occurs when system updates informational the same rate it receives it
\end{itemize}

\note[item]{}
\end{frame}
\begin{frame}
\frametitle{Unrelated Title}


\begin{itemize}
\item Automatically tracks all the steps in the sales process
\end{itemize}

\note[item]{}
\end{frame}
\begin{frame}
\frametitle{Unrelated Title}


\begin{itemize}
\item Automates each phase of the sales process helping individual sales representatives coordinate and organize all their accounts
\end{itemize}

\note[item]{}
\end{frame}
\begin{frame}
\frametitle{Unrelated Title}


\begin{itemize}
\item Combines art along with science to determine how to make URLs more attractive to search engines resulting in higher search engine ranking
\end{itemize}

\note[item]{}
\end{frame}
\begin{frame}
\frametitle{Unrelated Title}


\begin{itemize}
\item Evaluates variables that search engines use to determine where URL appears on the list of search results
\end{itemize}

\note[item]{}
\end{frame}
\begin{frame}
\frametitle{Unrelated Title}


\begin{itemize}
\item Website software that finds other pages based on keyword matching
\end{itemize}

\note[item]{}
\end{frame}
\begin{frame}
\frametitle{Unrelated Title}


\begin{itemize}
\item A self-photograph placed on a social media website
\end{itemize}

\note[item]{}
\end{frame}
\begin{frame}
\frametitle{Unrelated Title}


\begin{itemize}
\item A component of Web 3.0 that describes things in a way that computers can understand
\end{itemize}

\note[item]{}
\end{frame}
\begin{frame}
\frametitle{Unrelated Title}


\begin{itemize}
\item The study of the impact on other variables when one variable is changed repeatedly
\end{itemize}

\note[item]{}
\end{frame}
\begin{frame}
\frametitle{Unrelated Title}


\begin{itemize}
\item Software that will search several retailer websites and provide a comparison each retailers offering including price and availability
\end{itemize}

\note[item]{}
\end{frame}
\begin{frame}
\frametitle{Unrelated Title}


\begin{itemize}
\item Allows users to share organize search and manage bookmarks
\end{itemize}

\note[item]{}
\end{frame}
\begin{frame}
\frametitle{Unrelated Title}


\begin{itemize}
\item Represents the interconnection of relationships in a social network
\end{itemize}

\note[item]{}
\end{frame}
\begin{frame}
\frametitle{Unrelated Title}


\begin{itemize}
\item Refers to websites that rely on user participation and user contributed content
\end{itemize}

\note[item]{}
\end{frame}
\begin{frame}
\frametitle{Unrelated Title}


\begin{itemize}
\item Maps group contacts identifying who knows each other and who works together
\end{itemize}

\note[item]{}
\end{frame}
\begin{frame}
\frametitle{Unrelated Title}


\begin{itemize}
\item An application that connects people by matching profiles information, or the practice of expanding your business and or social contacts by constructing a personal network
\end{itemize}

\note[item]{}
\end{frame}
\begin{frame}
\frametitle{Unrelated Title}


\begin{itemize}
\item Describe the collaborative activity of marking shared online content with keywords or tags as a way to organize it for future navigation filtering or search
\end{itemize}

\note[item]{}
\end{frame}
\begin{frame}
\frametitle{Unrelated Title}


\begin{itemize}
\item Delivers applications over the cloud using a pay-per-use revenue model
\end{itemize}

\note[item]{}
\end{frame}
\begin{frame}
\frametitle{Unrelated Title}


\begin{itemize}
\item Modifies existing software according to the business's or user's requirements
\end{itemize}

\note[item]{}
\end{frame}
\begin{frame}
\frametitle{Unrelated Title}


\begin{itemize}
\item Contains instructions written by a programmer specifying the actions to be performed by computer software
\end{itemize}

\note[item]{}
\end{frame}
\begin{frame}
\frametitle{Unrelated Title}


\begin{itemize}
\item Describes the original transaction record along with details such as it stated purpose and amount spent and includes cash receipt canceled checks invoice customer refund employee timesheet etc.
\end{itemize}

\note[item]{}
\end{frame}
\begin{frame}
\frametitle{Unrelated Title}


\begin{itemize}
\item Managers develop overall business strategies goals and objectives as part of the company strategic plan
\end{itemize}

\note[item]{}
\end{frame}
\begin{frame}
\frametitle{Unrelated Title}


\begin{itemize}
\item Involves situations where establish process offers potential solutions
\end{itemize}

\note[item]{}
\end{frame}
\begin{frame}
\frametitle{Unrelated Title}


\begin{itemize}
\item Focuses on keeping supplies satisfied by evaluating and categorizing suppliers for different projects which optimizes supplier selection
\end{itemize}

\note[item]{}
\end{frame}
\begin{frame}
\frametitle{Unrelated Title}


\begin{itemize}
\item Communication that occurs at the same time such as IM or chat
\end{itemize}

\note[item]{}
\end{frame}
\begin{frame}
\frametitle{Unrelated Title}


\begin{itemize}
\item The knowledge contained in people’s heads
\end{itemize}

\note[item]{}
\end{frame}
\begin{frame}
\frametitle{Unrelated Title}


\begin{itemize}
\item Specific keywords or phrases incorporated into website content for means of classification taxonomy
\end{itemize}

\note[item]{}
\end{frame}
\begin{frame}
\frametitle{Unrelated Title}


\begin{itemize}
\item The scientific classification of organisms into group based on similarities of structure or origin
\end{itemize}

\note[item]{}
\end{frame}
\begin{frame}
\frametitle{Unrelated Title}


\begin{itemize}
\item The basic business system that serves the operational level analysis in our organization
\end{itemize}

\note[item]{}
\end{frame}
\begin{frame}
\frametitle{Unrelated Title}


\begin{itemize}
\item Incurs in situation in which no procedures or rules exist to guide decision making towards the correct choice
\end{itemize}

\note[item]{}
\end{frame}
\begin{frame}
\frametitle{Unrelated Title}


\begin{itemize}
\item Increasing the value of the sale
\end{itemize}

\note[item]{}
\end{frame}
\begin{frame}
\frametitle{Unrelated Title}


\begin{itemize}
\item A form of predictive analytics for marketing campaigns that attempts to identify target markets or people who could be convinced to buy products
\end{itemize}

\note[item]{}
\end{frame}
\begin{frame}
\frametitle{Unrelated Title}


\begin{itemize}
\item Content created and updated by many users for many users
\end{itemize}

\note[item]{}
\end{frame}
\begin{frame}
\frametitle{Unrelated Title}


\begin{itemize}
\item Allows people at two or more locations to interact via two-way video and audio transmissions simultaneously as well as share document data computer display and whiteboards
\end{itemize}

\note[item]{}
\end{frame}
\begin{frame}
\frametitle{Unrelated Title}


\begin{itemize}
\item A computer stimulated environment that can be a simulation of the real world or an imaginary world
\end{itemize}

\note[item]{}
\end{frame}
\begin{frame}
\frametitle{Unrelated Title}

\begin{center}
\includegraphics[width=0.9\textwidth,height=0.9\textheight,keepaspectratio]{/Users/I516998/Library/Application Support/Anki2/User 1/collection.media/paste-9cf6edb93de6f9b5cc48a407804fccdca3f38a42.jpg}
\end{center}

\begin{itemize}
\item The next generation of Internet use—a more mature, distinctive communications platform characterized by new qualities such as collaboration, sharing, and free.
\end{itemize}

\note[item]{}
\end{frame}
\begin{frame}
\frametitle{Unrelated Title}


\begin{itemize}
\item Allows customers to use the web find answers to their questions or solutions to their problem
\end{itemize}

\note[item]{}
\end{frame}
\begin{frame}
\frametitle{Unrelated Title}


\begin{itemize}
\item Blends videoconferencing with document sharing and allows the user to deliver a presentation over the web to group of geographically dispersed participants
\end{itemize}

\note[item]{}
\end{frame}
\begin{frame}
\frametitle{Unrelated Title}


\begin{itemize}
\item A locally stored URL or the address of a file or internet page saved as a shortcut
\end{itemize}

\note[item]{}
\end{frame}
\begin{frame}
\frametitle{Unrelated Title}


\begin{itemize}
\item Occurs when a website has stored enough data about a person's likes and dislikes to fashion offers more likely to appeal to that person
\end{itemize}

\note[item]{}
\end{frame}
\begin{frame}
\frametitle{Unrelated Title}


\begin{itemize}
\item Checks the impact of a change in an assumption on the proposed solution
\end{itemize}

\note[item]{}
\end{frame}
\begin{frame}
\frametitle{Unrelated Title}


\begin{itemize}
\item A type of collaborative web page that allows users to add remove and change content which can be easily organized and reorganized as required
\end{itemize}

\note[item]{}
\end{frame}
\begin{frame}
\frametitle{Unrelated Title}


\begin{itemize}
\item And for customer satisfaction through early a continuous delivery of useful software components developed by an iterative process and with design point that uses the bare minimum requirement
\end{itemize}

\note[item]{}
\end{frame}
\begin{frame}
\frametitle{Unrelated Title}


\begin{itemize}
\item Analyze and user business requirement and refining projects goals into define functions and operations of the intended system
\end{itemize}

\note[item]{}
\end{frame}
\begin{frame}
\frametitle{Unrelated Title}


\begin{itemize}
\item Represents the current state of the operation that has been mapped without any specific improvements or changes to existing processes
\end{itemize}

\note[item]{}
\end{frame}
\begin{frame}
\frametitle{Unrelated Title}


\begin{itemize}
\item A technique for generating ideas by encouraging participating to offer as many ideas as possible in a short period of time without any analysis until all the ideas have been exhausted
\end{itemize}

\note[item]{}
\end{frame}
\begin{frame}
\frametitle{Unrelated Title}


\begin{itemize}
\item Defects in code of an information system
\end{itemize}

\note[item]{}
\end{frame}
\begin{frame}
\frametitle{Unrelated Title}


\begin{itemize}
\item A graphical notation that depicts the steps in a business process
\end{itemize}

\note[item]{}
\end{frame}
\begin{frame}
\frametitle{Unrelated Title}


\begin{itemize}
\item A graphic description of a process showing the sequence and process task which is developed for a specific purpose and from a selected viewpoint
\end{itemize}

\note[item]{}
\end{frame}
\begin{frame}
\frametitle{Unrelated Title}


\begin{itemize}
\item The activity of creating a detailed flowchart or process map of a work process showing its input task and activities in a structured sequence
\end{itemize}

\note[item]{}
\end{frame}
\begin{frame}
\frametitle{Unrelated Title}


\begin{itemize}
\item A patent that protects a specific set of procedures for conducting a particular business activity
\end{itemize}

\note[item]{}
\end{frame}
\begin{frame}
\frametitle{Unrelated Title}


\begin{itemize}
\item The analysis and redesign of workflow within and between enterprises
\end{itemize}

\note[item]{}
\end{frame}
\begin{frame}
\frametitle{Unrelated Title}


\begin{itemize}
\item The is standardized set of activities that accomplish a specific task, such as processing a customer’s order. Business processes transform a set of inputs into a set of outputs by using people and tools.
\end{itemize}

\note[item]{}
\end{frame}
\begin{frame}
\frametitle{Unrelated Title}


\begin{itemize}
\item The specific business requests the system must meet to be successful, so the analysis phase is critical because business requirements drive the entire system development efforts
\end{itemize}

\note[item]{}
\end{frame}
\begin{frame}
\frametitle{Unrelated Title}


\begin{itemize}
\item A leadership plan that achieves a specific set of goals or objectives
\end{itemize}

\note[item]{}
\end{frame}
\begin{frame}
\frametitle{Unrelated Title}


\begin{itemize}
\item The ability of buyers to affect the price they must pay for an item. Manipulated with switching costs and loyalty programs.
\end{itemize}

\note[item]{}
\end{frame}
\begin{frame}
\frametitle{Unrelated Title}


\begin{itemize}
\item Person or event that is the catalyst for implementing major changes for a system to meet business changes
\end{itemize}

\note[item]{}
\end{frame}
\begin{frame}
\frametitle{Unrelated Title}


\begin{itemize}
\item Defines that how what when and who regarding the flow of project information to stakeholders and is key for managing expectations
\end{itemize}

\note[item]{}
\end{frame}
\begin{frame}
\frametitle{Unrelated Title}


\begin{itemize}
\item A feature of a product or service that an organization's customers place a greater value on than similar offering from a competitor
\end{itemize}

\note[item]{}
\end{frame}
\begin{frame}
\frametitle{Unrelated Title}


\begin{itemize}
\item the process of gathering information about the competitive environment, including competitors' plans, activities, and products, to improve a company's ability to succeed
\end{itemize}

\note[item]{}
\end{frame}
\begin{frame}
\frametitle{Unrelated Title}


\begin{itemize}
\item Tools are software suite the automated system analysis design and development
\end{itemize}

\note[item]{}
\end{frame}
\begin{frame}
\frametitle{Unrelated Title}


\begin{itemize}
\item The process of transferring information from a legacy system to a new system
\end{itemize}

\note[item]{}
\end{frame}
\begin{frame}
\frametitle{Unrelated Title}


\begin{itemize}
\item Business processes such as manufacturing goods selling products and providing service that make up the primary activities in a value chain
\end{itemize}

\note[item]{}
\end{frame}
\begin{frame}
\frametitle{Unrelated Title}


\begin{itemize}
\item Makes system changes to repair design flaws coding errors or implementation issues
\end{itemize}

\note[item]{}
\end{frame}
\begin{frame}
\frametitle{Unrelated Title}


\begin{itemize}
\item Identified as the longest stretch of dependent activities and measuring them from start to finish
\end{itemize}

\note[item]{}
\end{frame}
\begin{frame}
\frametitle{Unrelated Title}


\begin{itemize}
\item Help organization's segment their customers into categories such as best and worst customers
\end{itemize}

\note[item]{}
\end{frame}
\begin{frame}
\frametitle{Unrelated Title}


\begin{itemize}
\item Results in a product or service that is received by an organization's external customer
\end{itemize}

\note[item]{}
\end{frame}
\begin{frame}
\frametitle{Unrelated Title}


\begin{itemize}
\item Illustrates the movement of information between external entities and the process and data store within the system
\end{itemize}

\note[item]{}
\end{frame}
\begin{frame}
\frametitle{Unrelated Title}


\begin{itemize}
\item A logical relationship that exists between the project task or between a project task and a milestone
\end{itemize}

\note[item]{}
\end{frame}
\begin{frame}
\frametitle{Unrelated Title}


\begin{itemize}
\item Involves describing the desire features and operations of the system including screen layouts business rules process diagrams pseudo of code and other documents
\end{itemize}

\note[item]{}
\end{frame}
\begin{frame}
\frametitle{Unrelated Title}


\begin{itemize}
\item Involves taking all of the detailed design documents from the design phase and transferring them into the actual system
\end{itemize}

\note[item]{}
\end{frame}
\begin{frame}
\frametitle{Unrelated Title}


\begin{itemize}
\item Build a small-scale representation or working model of the system to ensure it meets the user in business requirement
\end{itemize}

\note[item]{}
\end{frame}
\begin{frame}
\frametitle{Unrelated Title}


\begin{itemize}
\item Continuously changing and provides business solutions to ever changing business operation
\end{itemize}

\note[item]{}
\end{frame}
\begin{frame}
\frametitle{Unrelated Title}


\begin{itemize}
\item A feature of a product or service that customers have come to expect an entry competitors must offer the same for survival
\end{itemize}

\note[item]{}
\end{frame}
\begin{frame}
\frametitle{Unrelated Title}


\begin{itemize}
\item The person or group who provides the financial resources for the project
\end{itemize}

\note[item]{}
\end{frame}
\begin{frame}
\frametitle{Unrelated Title}


\begin{itemize}
\item A Markup language for document containing structured information
\end{itemize}

\note[item]{}
\end{frame}
\begin{frame}
\frametitle{Unrelated Title}


\begin{itemize}
\item Breaks a project into tiny phases and developers cannot continue on the next phase until the first phase is completed
\end{itemize}

\note[item]{}
\end{frame}
\begin{frame}
\frametitle{Unrelated Title}


\begin{itemize}
\item The measure of the tangible and intangible of an information system
\end{itemize}

\note[item]{}
\end{frame}
\begin{frame}
\frametitle{Unrelated Title}


\begin{itemize}
\item Occurs when an organization can significantly impact its market share by being first to market with a competitive advantage
\end{itemize}

\note[item]{}
\end{frame}
\begin{frame}
\frametitle{Unrelated Title}


\begin{itemize}
\item Programming language that looks similar to human languages
\end{itemize}

\note[item]{}
\end{frame}
\begin{frame}
\frametitle{Unrelated Title}


\begin{itemize}
\item A simple bar chart that lists project tasks vertically against the projects time frame listed horizontally
\end{itemize}

\note[item]{}
\end{frame}
\begin{frame}
\frametitle{Unrelated Title}


\begin{itemize}
\item The interface to an information system
\end{itemize}

\note[item]{}
\end{frame}
\begin{frame}
\frametitle{Unrelated Title}


\begin{itemize}
\item A group of people who respond to internal system user question
\end{itemize}

\note[item]{}
\end{frame}
\begin{frame}
\frametitle{Unrelated Title}


\begin{itemize}
\item Impulse placing the system into production so users can begin to perform actual business operations with the system
\end{itemize}

\note[item]{}
\end{frame}
\begin{frame}
\frametitle{Unrelated Title}


\begin{itemize}
\item A common approach using the professional expertise within an organization to develop and maintain the Organization information technology systems
\end{itemize}

\note[item]{}
\end{frame}
\begin{frame}
\frametitle{Unrelated Title}


\begin{itemize}
\item Difficult to quantify or measure
\end{itemize}

\note[item]{}
\end{frame}
\begin{frame}
\frametitle{Unrelated Title}


\begin{itemize}
\item Consist of a series of tiny projects
\end{itemize}

\note[item]{}
\end{frame}
\begin{frame}
\frametitle{Unrelated Title}


\begin{itemize}
\item A session where employees meet sometimes for several days to define or define or review the business requirement for the system
\end{itemize}

\note[item]{}
\end{frame}
\begin{frame}
\frametitle{Unrelated Title}


\begin{itemize}
\item A trigger that enables a projects manager to close the project prior to completion
\end{itemize}

\note[item]{}
\end{frame}
\begin{frame}
\frametitle{Unrelated Title}


\begin{itemize}
\item The capability of service to be joined on demand to create composite services or disassembled just as easily into their functional component
\end{itemize}

\note[item]{}
\end{frame}
\begin{frame}
\frametitle{Unrelated Title}


\begin{itemize}
\item Rewards customers based on their spending
\end{itemize}

\note[item]{}
\end{frame}
\begin{frame}
\frametitle{Unrelated Title}


\begin{itemize}
\item The organization performs changes corrections additions and upgrades to ensure the system continues to meet business goals
\end{itemize}

\note[item]{}
\end{frame}
\begin{frame}
\frametitle{Unrelated Title}


\begin{itemize}
\item A set of policies procedures standard process practice tools technique and Tess that people apply to TECO management challenges
\end{itemize}

\note[item]{}
\end{frame}
\begin{frame}
\frametitle{Unrelated Title}


\begin{itemize}
\item Contracting an outsourcing agreement with a company in a nearby country
\end{itemize}

\note[item]{}
\end{frame}
\begin{frame}
\frametitle{Unrelated Title}


\begin{itemize}
\item Languages that group data and corresponding processes into objects
\end{itemize}

\note[item]{}
\end{frame}
\begin{frame}
\frametitle{Unrelated Title}


\begin{itemize}
\item Supports general business processes and does not require any specific software customization to meet the organization needs
\end{itemize}

\note[item]{}
\end{frame}
\begin{frame}
\frametitle{Unrelated Title}


\begin{itemize}
\item Using organization from developing countries to write code and develop system
\end{itemize}

\note[item]{}
\end{frame}
\begin{frame}
\frametitle{Unrelated Title}


\begin{itemize}
\item Runs over the Internet or on a CD or DVD and employees complete the training on their own time at their own pace
\end{itemize}

\note[item]{}
\end{frame}
\begin{frame}
\frametitle{Unrelated Title}


\begin{itemize}
\item The process of engaging another company within the same country for services
\end{itemize}

\note[item]{}
\end{frame}
\begin{frame}
\frametitle{Unrelated Title}


\begin{itemize}
\item An arrangement by which one organization provides a service or services for another organization that chooses not to perform them inhouse
\end{itemize}

\note[item]{}
\end{frame}
\begin{frame}
\frametitle{Unrelated Title}


\begin{itemize}
\item A graphical network model that depicts a project's task and the relationship between those tasks
\end{itemize}

\note[item]{}
\end{frame}
\begin{frame}
\frametitle{Unrelated Title}


\begin{itemize}
\item Involves establishing a high-level plan of the intended project and determine project goals
\end{itemize}

\note[item]{}
\end{frame}
\begin{frame}
\frametitle{Unrelated Title}

\begin{center}
\includegraphics[width=0.9\textwidth,height=0.9\textheight,keepaspectratio]{/Users/I516998/Library/Application Support/Anki2/User 1/collection.media/paste-27c43e6b84a5b8e7725bf5852a23683c5dabe894.jpg}
\end{center}

\begin{itemize}
\item Porter’s Five Forces Model analyzes the competitive forces within the environment in which a company operates to assess the potential for profitability in an industry. Its purpose is to combat these competitive forces by identifying opportunities, competitive advantages, and competitive intelligence. If the forces are strong, they increase competition; if the forces are weak, they decrease competition.
\end{itemize}

\note[item]{}
\end{frame}
\begin{frame}
\frametitle{Unrelated Title}

\begin{center}
\includegraphics[width=0.9\textwidth,height=0.9\textheight,keepaspectratio]{/Users/I516998/Library/Application Support/Anki2/User 1/collection.media/paste-8bc7c788aba8acdbd610fe911a25123effbb4a3c.jpg}
\includegraphics[width=0.9\textwidth,height=0.9\textheight,keepaspectratio]{/Users/I516998/Library/Application Support/Anki2/User 1/collection.media/paste-f947aeb0ba1ac2af13b25551e15935df88c3e780.jpg}
\end{center}

\begin{itemize}
\item Generic business strategies that are neither organization nor industry specific and can be applied to any business, product, or service.
\item These three generic business strategies for entering a new market are: (1) broad cost leadership, (2) broad differentiation, and (3) focused strategy.
\item Broad strategies reach a large market segment.
\item Focused strategies target a niche or unique market with either cost leadership or differentiation.
\item Trying to be all things to all people is a recipe for disaster, since doing so makes it difficult to project a consistent image to the entire marketplace. For this reason, Porter suggests adopting only one of the three generic strategies.
\end{itemize}

\note[item]{}
\end{frame}
\begin{frame}
\frametitle{Unrelated Title}


\begin{itemize}
\item Makes a system change to reduce the chance of future system failures
\end{itemize}

\note[item]{}
\end{frame}
\begin{frame}
\frametitle{Unrelated Title}

\begin{center}
\includegraphics[width=0.9\textwidth,height=0.9\textheight,keepaspectratio]{/Users/I516998/Library/Application Support/Anki2/User 1/collection.media/paste-02f7f9b54667b31e2082038c1770593da66cbcd4.jpg}
\end{center}

\begin{itemize}
\item Found at the bottom of the value chain, these include business processes that acquire raw materials and manufacture, deliver, market, sell, and provide after-sales services.
\end{itemize}

\note[item]{}
\end{frame}
\begin{frame}
\frametitle{Unrelated Title}


\begin{itemize}
\item The person responsible for the end-to-end functioning of a business process
\end{itemize}

\note[item]{}
\end{frame}
\begin{frame}
\frametitle{Unrelated Title}


\begin{itemize}
\item An advantage that occurs when a company develops unique difference in its products with the intent to influence demand
\end{itemize}

\note[item]{}
\end{frame}
\begin{frame}
\frametitle{Unrelated Title}


\begin{itemize}
\item Factor that is considered to be true real or certain without proof or demonstration
\end{itemize}

\note[item]{}
\end{frame}
\begin{frame}
\frametitle{Unrelated Title}


\begin{itemize}
\item Specific factor that can limit options
\end{itemize}

\note[item]{}
\end{frame}
\begin{frame}
\frametitle{Unrelated Title}


\begin{itemize}
\item Any measurable tangible verifiable outcome result or item that is produced to complete a project or part of a project
\end{itemize}

\note[item]{}
\end{frame}
\begin{frame}
\frametitle{Unrelated Title}


\begin{itemize}
\item An internal department that oversees all organizational projects
\end{itemize}

\note[item]{}
\end{frame}
\begin{frame}
\frametitle{Unrelated Title}


\begin{itemize}
\item The application of knowledge skills tools and techniques to project activities in order to meet or exceed stakeholders need an expectation from a project
\end{itemize}

\note[item]{}
\end{frame}
\begin{frame}
\frametitle{Unrelated Title}


\begin{itemize}
\item Represents key dates when certain group of activities must be performed
\end{itemize}

\note[item]{}
\end{frame}
\begin{frame}
\frametitle{Unrelated Title}


\begin{itemize}
\item Quantifiable criteria that must be met for the project to be considered a success
\end{itemize}

\note[item]{}
\end{frame}
\begin{frame}
\frametitle{Unrelated Title}


\begin{itemize}
\item A formal approved document that manages and control project execution
\end{itemize}

\note[item]{}
\end{frame}
\begin{frame}
\frametitle{Unrelated Title}


\begin{itemize}
\item Defines the specification for product output of the project and is key for managing expectations, controlling scope and completing other planning efforts
\end{itemize}

\note[item]{}
\end{frame}
\begin{frame}
\frametitle{Unrelated Title}


\begin{itemize}
\item Links the project to the organizations overall business goals
\end{itemize}

\note[item]{}
\end{frame}
\begin{frame}
\frametitle{Unrelated Title}


\begin{itemize}
\item Describe the business needs the problem the project will solve and the justification requirement and current boundaries for the project
\end{itemize}

\note[item]{}
\end{frame}
\begin{frame}
\frametitle{Unrelated Title}


\begin{itemize}
\item Individuals and organizations actively involved in the project or whose interests might be affected as a result of project execution or project completion
\end{itemize}

\note[item]{}
\end{frame}
\begin{frame}
\frametitle{Unrelated Title}


\begin{itemize}
\item Emphasize extensive user involvement in the rapid and evolutionary construction of working prototype of a system to accelerate the systems development process
\end{itemize}

\note[item]{}
\end{frame}
\begin{frame}
\frametitle{Unrelated Title}


\begin{itemize}
\item Provides a framework for breaking down the development of software into four gates
\end{itemize}

\note[item]{}
\end{frame}
\begin{frame}
\frametitle{Unrelated Title}


\begin{itemize}
\item The process of managing changes to the business requirements throughout the project
\end{itemize}

\note[item]{}
\end{frame}
\begin{frame}
\frametitle{Unrelated Title}


\begin{itemize}
\item Prioritize all of the business requirements by order of importance to the company
\end{itemize}

\note[item]{}
\end{frame}
\begin{frame}
\frametitle{Unrelated Title}


\begin{itemize}
\item Defines all project roles and indicates what responsibilities are associated with each role
\end{itemize}

\note[item]{}
\end{frame}
\begin{frame}
\frametitle{Unrelated Title}


\begin{itemize}
\item High when competition is fierce in a market and low when competitors are more complacent. Although competition is always more intense in some industries than in others, the overall trend is toward increased competition in almost every industry.
\end{itemize}

\note[item]{}
\end{frame}
\begin{frame}
\frametitle{Unrelated Title}


\begin{itemize}
\item A program he made it that provides for interactive modules to a website
\end{itemize}

\note[item]{}
\end{frame}
\begin{frame}
\frametitle{Unrelated Title}


\begin{itemize}
\item Uses small teams to produce small pieces of deliverable software using sprints or 30-day intervals to achieve an appointed goal
\end{itemize}

\note[item]{}
\end{frame}
\begin{frame}
\frametitle{Unrelated Title}


\begin{itemize}
\item A business-driven enterprise architecture that supports integrating a business as linked repeatable activities task or service
\end{itemize}

\note[item]{}
\end{frame}
\begin{frame}
\frametitle{Unrelated Title}


\begin{itemize}
\item The system users actual signature indicating they approve all of the business requirement
\end{itemize}

\note[item]{}
\end{frame}
\begin{frame}
\frametitle{Unrelated Title}


\begin{itemize}
\item A business task such as checking a potential customer’s credit rating when opening a new account
\end{itemize}

\note[item]{}
\end{frame}
\begin{frame}
\frametitle{Unrelated Title}


\begin{itemize}
\item A discipline approach for constructing information systems through the use of common method techniques or tools
\end{itemize}

\note[item]{}
\end{frame}
\begin{frame}
\frametitle{Unrelated Title}


\begin{itemize}
\item A person or organization with an interest in a particular place or issue.
\end{itemize}

\note[item]{}
\end{frame}
\begin{frame}
\frametitle{Unrelated Title}


\begin{itemize}
\item Uses a systematic approach in an attempt to improve business effectiveness and efficiency continuously
\end{itemize}

\note[item]{}
\end{frame}
\begin{frame}
\frametitle{Unrelated Title}


\begin{itemize}
\item Periodic reviews of actual performance versus expected performance
\end{itemize}

\note[item]{}
\end{frame}
\begin{frame}
\frametitle{Unrelated Title}


\begin{itemize}
\item Supplier power is the suppliers’ ability to influence the prices they charge for supplies (including materials, labor, and services).If supplier power is high, the supplier can influence the industry by:• Charging higher prices.• Limiting quality or services.• Shifting costs to industry participants.
\end{itemize}

\note[item]{}
\end{frame}
\begin{frame}
\frametitle{Unrelated Title}


\begin{itemize}
\item Involves the management of information flows between and among activities in supply chain to maximize total supply chain effectiveness and corporate profitability
\end{itemize}

\note[item]{}
\end{frame}
\begin{frame}
\frametitle{Unrelated Title}


\begin{itemize}
\item Consist of all parties involved directly or indirectly in the procurement of a product or raw material
\end{itemize}

\note[item]{}
\end{frame}
\begin{frame}
\frametitle{Unrelated Title}

\begin{center}
\includegraphics[width=0.9\textwidth,height=0.9\textheight,keepaspectratio]{/Users/I516998/Library/Application Support/Anki2/User 1/collection.media/paste-02f7f9b54667b31e2082038c1770593da66cbcd4.jpg}
\end{center}

\begin{itemize}
\item Found along the top of the value chain and includes business processes such as include firm infrastructure, human resource management, technology development, and procurement that support the primary value activities.
\end{itemize}

\note[item]{}
\end{frame}
\begin{frame}
\frametitle{Unrelated Title}


\begin{itemize}
\item Documents the steps or activities of a workflow by grouping activities into swim lanes which are horizontal or vertical columns containing all associated activities for that category department
\end{itemize}

\note[item]{}
\end{frame}
\begin{frame}
\frametitle{Unrelated Title}


\begin{itemize}
\item The costs that can make customers reluctant to switch to another product or service
\end{itemize}

\note[item]{}
\end{frame}
\begin{frame}
\frametitle{Unrelated Title}

\begin{center}
\includegraphics[width=0.9\textwidth,height=0.9\textheight,keepaspectratio]{/Users/I516998/Library/Application Support/Anki2/User 1/collection.media/paste-8a867ebe37d7fa2352c0d93a606b738ac9f915e7.jpg}
\end{center}

\begin{itemize}
\item Evaluate and organization strength weakness opportunities and threats to identify significant influences that work for or against business strategies
\item A SWOT analysis evaluates an organization’s Strengths, Weaknesses, Opportunities, and Threats to identify significant influences that work for or against business strategies.
\item Strengths and weaknesses originate inside an organization. Opportunities and threats originate outside an organization, and cannot always be anticipated or controlled.
\end{itemize}

\note[item]{}
\end{frame}
\begin{frame}
\frametitle{Unrelated Title}


\begin{itemize}
\item The overall process for developing information systems from planning and analysis through implementation and maintenance
\end{itemize}

\note[item]{}
\end{frame}
\begin{frame}
\frametitle{Unrelated Title}


\begin{itemize}
\item Easy to quantify and typical measured to determine the success or failure of a project
\end{itemize}

\note[item]{}
\end{frame}
\begin{frame}
\frametitle{Unrelated Title}


\begin{itemize}
\item The detailed step the system must perform along with the expected results of each step
\end{itemize}

\note[item]{}
\end{frame}
\begin{frame}
\frametitle{Unrelated Title}


\begin{itemize}
\item Involves bringing all the project pieces together into a special testing environment to test for errors bugs and interoperability and verify that the system meets all of the business requirements defined in the analysis phase
\end{itemize}

\note[item]{}
\end{frame}
\begin{frame}
\frametitle{Unrelated Title}


\begin{itemize}
\item High when it is easy for new competitors to enter a market and low when there are significant entry barriers to entering a market. An entry barrier is a feature of a product or service that customers have come to expect and entering competitors must offer the same for survival.
\end{itemize}

\note[item]{}
\end{frame}
\begin{frame}
\frametitle{Unrelated Title}


\begin{itemize}
\item High when there are many alternatives to a product or service and low when they are a few alternatives from which to choose. A company can reduce the threat of substitutes by offering additional value through wider product distribution. Companies can also offer various add-on services, making the substitute product less of a threat.
\end{itemize}

\note[item]{}
\end{frame}
\begin{frame}
\frametitle{Unrelated Title}


\begin{itemize}
\item Show the results of applying changes improvement opportunities to the current as-is process model
\end{itemize}

\note[item]{}
\end{frame}
\begin{frame}
\frametitle{Unrelated Title}


\begin{itemize}
\item How late how do use the system and how to troubleshoot issues or problems
\end{itemize}

\note[item]{}
\end{frame}
\begin{frame}
\frametitle{Unrelated Title}


\begin{itemize}
\item Views of firm as a series of business processes that each add value to the product or service. Value chain analysis is a useful tool for determining how to create the greatest possible value for customers. The goal of value chain analysis is to identify processes in which the firm can add value for the customer and create a competitive advantage for itself, with a cost advantage or product differentiation.
\item The value chain groups a firm’s activities into two categories, primary value activities, and support value activities.
\end{itemize}

\note[item]{}
\end{frame}
\begin{frame}
\frametitle{Unrelated Title}


\begin{itemize}
\item A sequence of phases in which the output of each phase becomes the input for the next
\end{itemize}

\note[item]{}
\end{frame}
\begin{frame}
\frametitle{Unrelated Title}


\begin{itemize}
\item An open standard way of supporting interoperability
\end{itemize}

\note[item]{}
\end{frame}
\begin{frame}
\frametitle{Unrelated Title}


\begin{itemize}
\item Monitor processes to ensure tasks activities and responsibilities are executed as specified
\end{itemize}

\note[item]{}
\end{frame}
\begin{frame}
\frametitle{Unrelated Title}


\begin{itemize}
\item Includes the tasks activities and responsibilities required to execute each step in a business process
\end{itemize}

\note[item]{}
\end{frame}
\begin{frame}
\frametitle{Unrelated Title}


\begin{itemize}
\item Held in a classroom environment is led by instructor
\end{itemize}

\note[item]{}
\end{frame}
\begin{frame}
\frametitle{Unrelated Title}


\begin{itemize}
\item Unrestricted access to the entire system
\end{itemize}

\note[item]{}
\end{frame}
\begin{frame}
\frametitle{Unrelated Title}


\begin{itemize}
\item Introduced by the national institute of standards and technology (NIST) AES is an encryption standard designed to keep government information secure
\end{itemize}

\note[item]{}
\end{frame}
\begin{frame}
\frametitle{Unrelated Title}


\begin{itemize}
\item Software that although purporting to serve some useful function and often fulfilling that function also allows internet advertisers to display advertisements without the consent of the computer user
\end{itemize}

\note[item]{}
\end{frame}
\begin{frame}
\frametitle{Unrelated Title}


\begin{itemize}
\item Includes the hardware software and telecommunications equipment that when combined provide the underlining foundation to support the organization goals
\end{itemize}

\note[item]{}
\end{frame}
\begin{frame}
\frametitle{Unrelated Title}


\begin{itemize}
\item Scales and searches hard drive to prevent detect and remove known viruses, adware, and spyware
\end{itemize}

\note[item]{}
\end{frame}
\begin{frame}
\frametitle{Unrelated Title}


\begin{itemize}
\item A method for confirming users’ identities
\end{itemize}

\note[item]{}
\end{frame}
\begin{frame}
\frametitle{Unrelated Title}


\begin{itemize}
\item The process of providing a user with permission including access level and abilities such as file access hours of access and amount of allocated storage space
\end{itemize}

\note[item]{}
\end{frame}
\begin{frame}
\frametitle{Unrelated Title}


\begin{itemize}
\item An exact copy of a system's information
\end{itemize}

\note[item]{}
\end{frame}
\begin{frame}
\frametitle{Unrelated Title}


\begin{itemize}
\item The identification of a user based on a physical characteristic such as a fingerprint iris face voice or handwriting
\end{itemize}

\note[item]{}
\end{frame}
\begin{frame}
\frametitle{Unrelated Title}


\begin{itemize}
\item Details how a company recovers and restores critical business operations and systems after a disaster or extended disruption
\end{itemize}

\note[item]{}
\end{frame}
\begin{frame}
\frametitle{Unrelated Title}


\begin{itemize}
\item A process that identifies all critical business functions and the effect that a specific disaster may have upon them
\end{itemize}

\note[item]{}
\end{frame}
\begin{frame}
\frametitle{Unrelated Title}


\begin{itemize}
\item Determines future environmental infrastructure requirements to ensure high-quality system performance
\end{itemize}

\note[item]{}
\end{frame}
\begin{frame}
\frametitle{Unrelated Title}


\begin{itemize}
\item Represents the maximum output a system can deliver, for example the capacity of a hard drive represents the size or volume
\end{itemize}

\note[item]{}
\end{frame}
\begin{frame}
\frametitle{Unrelated Title}


\begin{itemize}
\item A trusted a third-party such as Verisign that validates users’ identities by means of visuals certificates
\end{itemize}

\note[item]{}
\end{frame}
\begin{frame}
\frametitle{Unrelated Title}


\begin{itemize}
\item A separate facility that does not haven computer equipment but is place where employees can move after a disaster
\end{itemize}

\note[item]{}
\end{frame}
\begin{frame}
\frametitle{Unrelated Title}


\begin{itemize}
\item Occurs when organizations use software that filters content to prevent the transmission of unauthorized information
\end{itemize}

\note[item]{}
\end{frame}
\begin{frame}
\frametitle{Unrelated Title}


\begin{itemize}
\item The science that studies encryption which is the hiding of messages so that only the sender and receiver can read them
\end{itemize}

\note[item]{}
\end{frame}
\begin{frame}
\frametitle{Unrelated Title}


\begin{itemize}
\item The use of computer and networking technologies against persons or property to intimidate or coerce governments individuals or any segment of social to attain political religious or ideological goals
\end{itemize}

\note[item]{}
\end{frame}
\begin{frame}
\frametitle{Unrelated Title}


\begin{itemize}
\item An organized attempt by a country's military to disrupt or destroy information and communication systems for another country
\end{itemize}

\note[item]{}
\end{frame}
\begin{frame}
\frametitle{Unrelated Title}


\begin{itemize}
\item Decodes information and is the opposite of encrypted
\end{itemize}

\note[item]{}
\end{frame}
\begin{frame}
\frametitle{Unrelated Title}


\begin{itemize}
\item Malicious agent designed by spammers and other Internet attackers to farm email addresses of websites or deposit spyware on machines
\end{itemize}

\note[item]{}
\end{frame}
\begin{frame}
\frametitle{Unrelated Title}


\begin{itemize}
\item A data file that identifies individuals or organizations online and is comparable to a digital signature
\end{itemize}

\note[item]{}
\end{frame}
\begin{frame}
\frametitle{Unrelated Title}


\begin{itemize}
\item Detailed process for recovering information or a system in the event of a catastrophic disaster
\end{itemize}

\note[item]{}
\end{frame}
\begin{frame}
\frametitle{Unrelated Title}


\begin{itemize}
\item Refers to a period of time when a system is unavailable
\end{itemize}

\note[item]{}
\end{frame}
\begin{frame}
\frametitle{Unrelated Title}


\begin{itemize}
\item A computer attack where an attacker accesses a wireless computer network, intercepts data, uses network services, and/or sends attack instructions without entering the office or organization that owns the network
\end{itemize}

\note[item]{}
\end{frame}
\begin{frame}
\frametitle{Unrelated Title}


\begin{itemize}
\item Looking through peoples trash another way hackers obtain information
\end{itemize}

\note[item]{}
\end{frame}
\begin{frame}
\frametitle{Unrelated Title}


\begin{itemize}
\item An infrastructure built for notifying people in the event of an emergency
\end{itemize}

\note[item]{}
\end{frame}
\begin{frame}
\frametitle{Unrelated Title}


\begin{itemize}
\item Ensures a company is ready to respond to an emergency in an organized timely and effective manner
\end{itemize}

\note[item]{}
\end{frame}
\begin{frame}
\frametitle{Unrelated Title}


\begin{itemize}
\item A sudden unexpected event requiring immediate action due to potential threat to health and safety the environment or property
\end{itemize}

\note[item]{}
\end{frame}
\begin{frame}
\frametitle{Unrelated Title}


\begin{itemize}
\item Scrambles information into an alternative form that requires a key or password to decrypt the information
\end{itemize}

\note[item]{}
\end{frame}
\begin{frame}
\frametitle{Unrelated Title}


\begin{itemize}
\item Occurs when the primary machine recovers and resumes operations taking over from the secondary server
\end{itemize}

\note[item]{}
\end{frame}
\begin{frame}
\frametitle{Unrelated Title}


\begin{itemize}
\item Occurs when redundant storage server offers an exact replica of the real time data and if the primary server crashes the users are automatically directed to the secondary server or backup server
\end{itemize}

\note[item]{}
\end{frame}
\begin{frame}
\frametitle{Unrelated Title}


\begin{itemize}
\item The ability for a system to respond to unexpected failures or system crashes as the backup system immediately and automatically takes over with no loss of service
\end{itemize}

\note[item]{}
\end{frame}
\begin{frame}
\frametitle{Unrelated Title}


\begin{itemize}
\item Hardware and or software that guards a private network by analyzing the information leaving and entering the network
\end{itemize}

\note[item]{}
\end{frame}
\begin{frame}
\frametitle{Unrelated Title}


\begin{itemize}
\item Experts in technology who use their knowledge to break into computer and computer networks either for profit or simply for the challenge
\end{itemize}

\note[item]{}
\end{frame}
\begin{frame}
\frametitle{Unrelated Title}


\begin{itemize}
\item Occurs when a system is continuously operational at all times
\end{itemize}

\note[item]{}
\end{frame}
\begin{frame}
\frametitle{Unrelated Title}


\begin{itemize}
\item Attacker grants themselves the same access levels they already have but assumes the identity of another user
\end{itemize}

\note[item]{}
\end{frame}
\begin{frame}
\frametitle{Unrelated Title}


\begin{itemize}
\item A separate and fully equipped facility where the company can move immediately after a disaster and resume business
\end{itemize}

\note[item]{}
\end{frame}
\begin{frame}
\frametitle{Unrelated Title}


\begin{itemize}
\item The forging of someone's identity for the purpose of fraud
\end{itemize}

\note[item]{}
\end{frame}
\begin{frame}
\frametitle{Unrelated Title}


\begin{itemize}
\item The process responsible for managing how incidents are identified and corrected
\end{itemize}

\note[item]{}
\end{frame}
\begin{frame}
\frametitle{Unrelated Title}


\begin{itemize}
\item Contains all of the details of an incident
\end{itemize}

\note[item]{}
\end{frame}
\begin{frame}
\frametitle{Unrelated Title}


\begin{itemize}
\item Unplanned interruption of service
\end{itemize}

\note[item]{}
\end{frame}
\begin{frame}
\frametitle{Unrelated Title}


\begin{itemize}
\item Details how an organization will implement the information security policy
\end{itemize}

\note[item]{}
\end{frame}
\begin{frame}
\frametitle{Unrelated Title}


\begin{itemize}
\item The category of computer security that addresses the protection of data from unauthorized disclosure and confirmation of data source authenticity
\end{itemize}

\note[item]{}
\end{frame}
\begin{frame}
\frametitle{Unrelated Title}


\begin{itemize}
\item Identifies the rules required to maintain information security
\end{itemize}

\note[item]{}
\end{frame}
\begin{frame}
\frametitle{Unrelated Title}


\begin{itemize}
\item A broad term encompassing the protection of information from accidental or intentional misuse by persons inside or outside an organization
\end{itemize}

\note[item]{}
\end{frame}
\begin{frame}
\frametitle{Unrelated Title}


\begin{itemize}
\item Legitimate users who purposely or accidentally misuse their access to the environment and cause some kind of business affecting incident
\end{itemize}

\note[item]{}
\end{frame}
\begin{frame}
\frametitle{Unrelated Title}


\begin{itemize}
\item Features full time monitoring tools that search for patterns in network traffic to identify intruders
\end{itemize}

\note[item]{}
\end{frame}
\begin{frame}
\frametitle{Unrelated Title}


\begin{itemize}
\item Requires more than two means of authentication such as what the user knows (password) what the user has security token and what the user is biometric verification
\end{itemize}

\note[item]{}
\end{frame}
\begin{frame}
\frametitle{Unrelated Title}


\begin{itemize}
\item String of alphanumeric characters used to authenticate a user and provide access to a system
\end{itemize}

\note[item]{}
\end{frame}
\begin{frame}
\frametitle{Unrelated Title}


\begin{itemize}
\item Measure how quickly a system performs a certain process or transaction
\end{itemize}

\note[item]{}
\end{frame}
\begin{frame}
\frametitle{Unrelated Title}


\begin{itemize}
\item Uses a zombie farm often by an organized crime association to launch a massive phishing attack
\end{itemize}

\note[item]{}
\end{frame}
\begin{frame}
\frametitle{Unrelated Title}


\begin{itemize}
\item Reroutes requests for legitimate websites to false websites
\end{itemize}

\note[item]{}
\end{frame}
\begin{frame}
\frametitle{Unrelated Title}


\begin{itemize}
\item A masquerading attack that combines spam with spoofing
\end{itemize}

\note[item]{}
\end{frame}
\begin{frame}
\frametitle{Unrelated Title}


\begin{itemize}
\item Technique to gain personal information for the purpose of identity theft usually by means of fraudulent email
\end{itemize}

\note[item]{}
\end{frame}
\begin{frame}
\frametitle{Unrelated Title}


\begin{itemize}
\item Refers to the ability of an application to operate on different devices or software platforms such as a different operating system
\end{itemize}

\note[item]{}
\end{frame}
\begin{frame}
\frametitle{Unrelated Title}


\begin{itemize}
\item A form of social engineering in which one individual lies to obtain confidential data about another individual
\end{itemize}

\note[item]{}
\end{frame}
\begin{frame}
\frametitle{Unrelated Title}


\begin{itemize}
\item A network intrusion attack that takes advantage of programming errors or design flaws to grant the attacker elevated access to the network and its associated data and applications
\end{itemize}

\note[item]{}
\end{frame}
\begin{frame}
\frametitle{Unrelated Title}


\begin{itemize}
\item Encryption system that uses two keys a public key that everyone can have and a private key for only the recipient
\end{itemize}

\note[item]{}
\end{frame}
\begin{frame}
\frametitle{Unrelated Title}


\begin{itemize}
\item A form of malicious software that infects your computer and asks for money
\end{itemize}

\note[item]{}
\end{frame}
\begin{frame}
\frametitle{Unrelated Title}


\begin{itemize}
\item The ability to get a system up and running in the event of a system crash or failure and includes restoring the information backup
\end{itemize}

\note[item]{}
\end{frame}
\begin{frame}
\frametitle{Unrelated Title}


\begin{itemize}
\item Ensure all systems are functioning correctly and providing accurate information
\end{itemize}

\note[item]{}
\end{frame}
\begin{frame}
\frametitle{Unrelated Title}


\begin{itemize}
\item Describes how well a system can scale up or adapt to increased demand of growth
\end{itemize}

\note[item]{}
\end{frame}
\begin{frame}
\frametitle{Unrelated Title}


\begin{itemize}
\item The traditional security process which requires a username and password
\end{itemize}

\note[item]{}
\end{frame}
\begin{frame}
\frametitle{Unrelated Title}


\begin{itemize}
\item A device that is around the same size as credit card containing embedded technologies that can store information and small amount of software to perform some limited processing
\end{itemize}

\note[item]{}
\end{frame}
\begin{frame}
\frametitle{Unrelated Title}


\begin{itemize}
\item Hackers use their social skills to trick people into revealing Access credentials or other valuable information
\end{itemize}

\note[item]{}
\end{frame}
\begin{frame}
\frametitle{Unrelated Title}


\begin{itemize}
\item A phishing expedition which the emails are carefully designed to target a particular person or organization
\end{itemize}

\note[item]{}
\end{frame}
\begin{frame}
\frametitle{Unrelated Title}


\begin{itemize}
\item A special class of adware that collects data about the user and transmit it over the internet without the user’s knowledge or permission
\end{itemize}

\note[item]{}
\end{frame}
\begin{frame}
\frametitle{Unrelated Title}


\begin{itemize}
\item Identifies ways that a company can grow in terms of computing resources while simultaneously becoming less dependent on hardware and energy consumption
\end{itemize}

\note[item]{}
\end{frame}
\begin{frame}
\frametitle{Unrelated Title}


\begin{itemize}
\item Occurs when the ability of a company to operate is impaired because of a hardware software or data outage
\end{itemize}

\note[item]{}
\end{frame}
\begin{frame}
\frametitle{Unrelated Title}


\begin{itemize}
\item Focus specifically on prioritizing the order for restoring hardware software and data across the organization that best meets business recovery requirements
\end{itemize}

\note[item]{}
\end{frame}
\begin{frame}
\frametitle{Unrelated Title}


\begin{itemize}
\item Computer viruses that wait for a specific data before executing instructions
\end{itemize}

\note[item]{}
\end{frame}
\begin{frame}
\frametitle{Unrelated Title}


\begin{itemize}
\item Small electronic devices that change user passwords automatically
\end{itemize}

\note[item]{}
\end{frame}
\begin{frame}
\frametitle{Unrelated Title}


\begin{itemize}
\item Requires the user to provide two means of authentication what the user knows password and what the user has security token
\end{itemize}

\note[item]{}
\end{frame}
\begin{frame}
\frametitle{Unrelated Title}


\begin{itemize}
\item When a system is not operating or cannot be used
\end{itemize}

\note[item]{}
\end{frame}
\begin{frame}
\frametitle{Unrelated Title}


\begin{itemize}
\item The degree to which a system is easy to learn efficient and satisfying to use
\end{itemize}

\note[item]{}
\end{frame}
\begin{frame}
\frametitle{Unrelated Title}


\begin{itemize}
\item Attackers grant themselves a higher access level such as administrator allowing the attacker to perform illegal actions such as running unauthorized code or deleting data
\end{itemize}

\note[item]{}
\end{frame}
\begin{frame}
\frametitle{Unrelated Title}


\begin{itemize}
\item Software written with malicious intent to cause annoyance of damage
\end{itemize}

\note[item]{}
\end{frame}
\begin{frame}
\frametitle{Unrelated Title}


\begin{itemize}
\item A phone scam that attempts to defraud people by asking them to call a bogus telephone number to confirm their account information
\end{itemize}

\note[item]{}
\end{frame}
\begin{frame}
\frametitle{Unrelated Title}


\begin{itemize}
\item A set of measurable characteristics of a human voice that uniquely identifies as individual
\end{itemize}

\note[item]{}
\end{frame}
\begin{frame}
\frametitle{Unrelated Title}


\begin{itemize}
\item A system weakness they can be exploited by a threat, for example a password that is never changed or system left on while an employee calls to lunch
\end{itemize}

\note[item]{}
\end{frame}
\begin{frame}
\frametitle{Unrelated Title}


\begin{itemize}
\item A separate facility with computer equipment that requires installation and configuration
\end{itemize}

\note[item]{}
\end{frame}
\begin{frame}
\frametitle{Unrelated Title}


\begin{itemize}
\item Brings together people from industry disability organizations government and research labs from around the world to develop guidelines and resources to help make the web accessible to people with disabilities including auditory cognitive neurological physical speech and visual disabilities
\end{itemize}

\note[item]{}
\end{frame}
\begin{frame}
\frametitle{Unrelated Title}


\begin{itemize}
\item Means that people with disabilities including visual auditory cognitive neurological physical display can use the web
\end{itemize}

\note[item]{}
\end{frame}
\begin{frame}
\frametitle{Unrelated Title}


\begin{itemize}
\item Spread itself not only from file to file but also from computer to computer without user intervention
\end{itemize}

\note[item]{}
\end{frame}
\begin{frame}
\frametitle{Unrelated Title}


\begin{itemize}
\item A group of computers on which a hacker has planted zombie programs
\end{itemize}

\note[item]{}
\end{frame}
\begin{frame}
\frametitle{Unrelated Title}


\begin{itemize}
\item a program that secretly takes over another computer for the purpose of launching attacks on other computers
\end{itemize}

\note[item]{}
\end{frame}
\begin{frame}
\frametitle{Unrelated Title}


\begin{itemize}
\item A policy that a user must agree to follow in order to be provided access to a network or to the internet
\end{itemize}

\note[item]{}
\end{frame}
\begin{frame}
\frametitle{Unrelated Title}


\begin{itemize}
\item States that email users will not send unsolicited emails or spam
\end{itemize}

\note[item]{}
\end{frame}
\begin{frame}
\frametitle{Unrelated Title}


\begin{itemize}
\item Offers a cloud based big data service to help organizations analyze massive amounts of data to solve business dilemmas
\end{itemize}

\note[item]{}
\end{frame}
\begin{frame}
\frametitle{Unrelated Title}


\begin{itemize}
\item A policy allows employees to use their personal mobile devices and computers to access enterprise data and applications
\end{itemize}

\note[item]{}
\end{frame}
\begin{frame}
\frametitle{Unrelated Title}


\begin{itemize}
\item A process improvement approach that contains 22 process areas
\end{itemize}

\note[item]{}
\end{frame}
\begin{frame}
\frametitle{Unrelated Title}


\begin{itemize}
\item The carbon dioxide and carbon monoxide in the atmosphere produced by business processes and systems
\end{itemize}

\note[item]{}
\end{frame}
\begin{frame}
\frametitle{Unrelated Title}


\begin{itemize}
\item A law that protects minors from accessing inappropriate material on the internet
\end{itemize}

\note[item]{}
\end{frame}
\begin{frame}
\frametitle{Unrelated Title}


\begin{itemize}
\item Refers to environmentally responsible use, manufacture, and disposal of technology products and computer equipment
\end{itemize}

\note[item]{}
\end{frame}
\begin{frame}
\frametitle{Unrelated Title}


\begin{itemize}
\item The abuse of pay per click pay per call and pay per conversion revenue models by repeatedly clicking on a link to increase charges or costs for the advertiser
\end{itemize}

\note[item]{}
\end{frame}
\begin{frame}
\frametitle{Unrelated Title}


\begin{itemize}
\item When a company uses its own computing infrastructure for normal usage and accesses the cloud when it needs to scale for high peak load requirements ensuring a sudden spike in usage does not result in poor performance or system crashes
\end{itemize}

\note[item]{}
\end{frame}
\begin{frame}
\frametitle{Unrelated Title}


\begin{itemize}
\item An individual who monitors and provisions cloud resources similar to server administrator at an individual company
\end{itemize}

\note[item]{}
\end{frame}
\begin{frame}
\frametitle{Unrelated Title}


\begin{itemize}
\item The software that makes the benefits of cloud computing possible such as multi tenancy
\end{itemize}

\note[item]{}
\end{frame}
\begin{frame}
\frametitle{Unrelated Title}


\begin{itemize}
\item Key for evaluating internal controls, such as Human Resources, logistics, information technology, risk, legal, marketing and sales operations, financial functions, procurement, and reporting
\end{itemize}

\note[item]{}
\end{frame}
\begin{frame}
\frametitle{Unrelated Title}


\begin{itemize}
\item Serves a specific community with common business models security requirements and compliance considerations
\end{itemize}

\note[item]{}
\end{frame}
\begin{frame}
\frametitle{Unrelated Title}


\begin{itemize}
\item A computer crime where a competitor or disgruntled employee increases a company's search advertising costs by repeatedly clicking on the advertiser's link
\end{itemize}

\note[item]{}
\end{frame}
\begin{frame}
\frametitle{Unrelated Title}


\begin{itemize}
\item The act of conforming, acquiescing, or yielding
\end{itemize}

\note[item]{}
\end{frame}
\begin{frame}
\frametitle{Unrelated Title}


\begin{itemize}
\item The assurance that messages and information are available only to those who are authorized to view them
\end{itemize}

\note[item]{}
\end{frame}
\begin{frame}
\frametitle{Unrelated Title}


\begin{itemize}
\item Legal protection for the expression of an idea, such as a song book or video game
\end{itemize}

\note[item]{}
\end{frame}
\begin{frame}
\frametitle{Unrelated Title}


\begin{itemize}
\item Companies acknowledged responsibility to society
\end{itemize}

\note[item]{}
\end{frame}
\begin{frame}
\frametitle{Unrelated Title}


\begin{itemize}
\item Software that is manufactured to look like the real thing and sold as such
\end{itemize}

\note[item]{}
\end{frame}
\begin{frame}
\frametitle{Unrelated Title}


\begin{itemize}
\item Threats negative remarks or defamatory comments transmitted via the internet or posted on a website
\end{itemize}

\note[item]{}
\end{frame}
\begin{frame}
\frametitle{Unrelated Title}


\begin{itemize}
\item A problem that occurs when someone registers purposely misspelled variations of well-known domain names
\end{itemize}

\note[item]{}
\end{frame}
\begin{frame}
\frametitle{Unrelated Title}


\begin{itemize}
\item Facilitates the accessibility of business-critical data in timely secure and affordable manner
\end{itemize}

\note[item]{}
\end{frame}
\begin{frame}
\frametitle{Unrelated Title}


\begin{itemize}
\item A facility used to house management information systems and associated components such as telecommunications and storage systems
\end{itemize}

\note[item]{}
\end{frame}
\begin{frame}
\frametitle{Unrelated Title}


\begin{itemize}
\item Occurs when an organization moves employees outside its firewall that a growing movement to change the way corporations address technology security
\end{itemize}

\note[item]{}
\end{frame}
\begin{frame}
\frametitle{Unrelated Title}


\begin{itemize}
\item A technological solution that allows publishers to control their digital media to discourage limit or prevent illegal copying and distribution
\end{itemize}

\note[item]{}
\end{frame}
\begin{frame}
\frametitle{Unrelated Title}


\begin{itemize}
\item Offers backup service that use cloud resources to protect applications and data from disruption caused by disaster
\end{itemize}

\note[item]{}
\end{frame}
\begin{frame}
\frametitle{Unrelated Title}


\begin{itemize}
\item Coordinates the process of transporting materials from a manufacture to distribution centers to the final customer
\end{itemize}

\note[item]{}
\end{frame}
\begin{frame}
\frametitle{Unrelated Title}


\begin{itemize}
\item Means that the MIS Infrastructure can be automatically scaled up or down based on need requirements
\end{itemize}

\note[item]{}
\end{frame}
\begin{frame}
\frametitle{Unrelated Title}


\begin{itemize}
\item Refers to the ability of a company to identify search gather seize or export digital information in responding to a litigation audit investigation or information inquiry
\end{itemize}

\note[item]{}
\end{frame}
\begin{frame}
\frametitle{Unrelated Title}


\begin{itemize}
\item Details the extent to which email messages may be read by others
\end{itemize}

\note[item]{}
\end{frame}
\begin{frame}
\frametitle{Unrelated Title}


\begin{itemize}
\item Stating explicitly how when and where the company monitors it employees
\end{itemize}

\note[item]{}
\end{frame}
\begin{frame}
\frametitle{Unrelated Title}


\begin{itemize}
\item The amount of energy consumed by business processes and systems
\end{itemize}

\note[item]{}
\end{frame}
\begin{frame}
\frametitle{Unrelated Title}


\begin{itemize}
\item Includes the plans for how an organization will build, deploy, use, and share it data processes and MIS assets
\end{itemize}

\note[item]{}
\end{frame}
\begin{frame}
\frametitle{Unrelated Title}


\begin{itemize}
\item Policies and procedures that address the ethical use of computers and internet usage in the business environment
\end{itemize}

\note[item]{}
\end{frame}
\begin{frame}
\frametitle{Unrelated Title}


\begin{itemize}
\item Contains general principles to guide computer user behavior
\end{itemize}

\note[item]{}
\end{frame}
\begin{frame}
\frametitle{Unrelated Title}


\begin{itemize}
\item Refers to discarded, obsolete, or broken electronic devices
\end{itemize}

\note[item]{}
\end{frame}
\begin{frame}
\frametitle{Unrelated Title}


\begin{itemize}
\item A general term for a set of standards governing the collection and use of personal data and addressing issues of privacy and accuracy
\end{itemize}

\note[item]{}
\end{frame}
\begin{frame}
\frametitle{Unrelated Title}


\begin{itemize}
\item The effects of geography on the economic realities of international business activities
\end{itemize}

\note[item]{}
\end{frame}
\begin{frame}
\frametitle{Unrelated Title}


\begin{itemize}
\item Provides the ability to locate track and predict the movement of every component for material anywhere upstream or downstream in the supply chain
\end{itemize}

\note[item]{}
\end{frame}
\begin{frame}
\frametitle{Unrelated Title}


\begin{itemize}
\item Method or system of government for management or control
\end{itemize}

\note[item]{}
\end{frame}
\begin{frame}
\frametitle{Unrelated Title}


\begin{itemize}
\item Built using environment friendly materials and designed to save energy
\end{itemize}

\note[item]{}
\end{frame}
\begin{frame}
\frametitle{Unrelated Title}


\begin{itemize}
\item A collection of computers often geographically dispersed they are coordinated to solve a common problem
\end{itemize}

\note[item]{}
\end{frame}
\begin{frame}
\frametitle{Unrelated Title}


\begin{itemize}
\item Includes two or more private public or community clouds but each cloud remains separate and is only linked by technology that enable data and application portability
\end{itemize}

\note[item]{}
\end{frame}
\begin{frame}
\frametitle{Unrelated Title}


\begin{itemize}
\item Governs the ethical and moral issues arising from the development and use of information technologies as well as the creation collection duplication distribution and processing of information itself (with or without the aid of computer technologies)
\end{itemize}

\note[item]{}
\end{frame}
\begin{frame}
\frametitle{Unrelated Title}


\begin{itemize}
\item Contains general principles regarding information privacy
\end{itemize}

\note[item]{}
\end{frame}
\begin{frame}
\frametitle{Unrelated Title}


\begin{itemize}
\item Concerns the legal right or general expectation of individuals, groups, or institutions to determine for themselves when and to what extent information about them is communicated to others
\end{itemize}

\note[item]{}
\end{frame}
\begin{frame}
\frametitle{Unrelated Title}


\begin{itemize}
\item An organization that sets guidelines and supporting tools for IT governance that is accepted worldwide and generally used by auditors and companies as a way to integrate technology to implement controls and meet specific business objectives
\end{itemize}

\note[item]{}
\end{frame}
\begin{frame}
\frametitle{Unrelated Title}


\begin{itemize}
\item Delivers hardware networking capabilities including the use of servers networking and storage over the cloud using a pay per use revenue model
\end{itemize}

\note[item]{}
\end{frame}
\begin{frame}
\frametitle{Unrelated Title}


\begin{itemize}
\item Intangible creative work that is embodied in physical form and includes copyrights trademarks and patents
\end{itemize}

\note[item]{}
\end{frame}
\begin{frame}
\frametitle{Unrelated Title}


\begin{itemize}
\item A non-governmental organization established in 1947 to promote the development of world standards to facilitate the international exchange of goods and services
\end{itemize}

\note[item]{}
\end{frame}
\begin{frame}
\frametitle{Unrelated Title}


\begin{itemize}
\item Government attempts to control internet traffic thus preventing some material from being viewed by a country's citizens
\end{itemize}

\note[item]{}
\end{frame}
\begin{frame}
\frametitle{Unrelated Title}


\begin{itemize}
\item Contains general principles to guide the proper use of the internet
\end{itemize}

\note[item]{}
\end{frame}
\begin{frame}
\frametitle{Unrelated Title}


\begin{itemize}
\item Provides control and disability to the status of individual items maintained in inventory
\end{itemize}

\note[item]{}
\end{frame}
\begin{frame}
\frametitle{Unrelated Title}


\begin{itemize}
\item A massive amount of email to a specific person or system that can cause that user's server to stop functioning
\end{itemize}

\note[item]{}
\end{frame}
\begin{frame}
\frametitle{Unrelated Title}


\begin{itemize}
\item Sales forecast to make sure that needed parts and materials are available at the right time and place in a specific company
\end{itemize}

\note[item]{}
\end{frame}
\begin{frame}
\frametitle{Unrelated Title}


\begin{itemize}
\item Refers to the computer chip performance per dollar doubling every 18 months
\end{itemize}

\note[item]{}
\end{frame}
\begin{frame}
\frametitle{Unrelated Title}


\begin{itemize}
\item A single instance of a system serves multiple customers
\end{itemize}

\note[item]{}
\end{frame}
\begin{frame}
\frametitle{Unrelated Title}


\begin{itemize}
\item Combines the physical resources such as servers, processors, and operating systems from the applications
\end{itemize}

\note[item]{}
\end{frame}
\begin{frame}
\frametitle{Unrelated Title}


\begin{itemize}
\item A contractual stipulation to ensure that eBusiness participants do not deny their online actions
\end{itemize}

\note[item]{}
\end{frame}
\begin{frame}
\frametitle{Unrelated Title}


\begin{itemize}
\item Deals with the day-to-day procedures for performing work including scheduling inventory and process management
\end{itemize}

\note[item]{}
\end{frame}
\begin{frame}
\frametitle{Unrelated Title}


\begin{itemize}
\item A user receives emails by choosing to allow permissions to incoming emails
\end{itemize}

\note[item]{}
\end{frame}
\begin{frame}
\frametitle{Unrelated Title}


\begin{itemize}
\item Receiving emails by choosing to deny permission to incoming emails
\end{itemize}

\note[item]{}
\end{frame}
\begin{frame}
\frametitle{Unrelated Title}


\begin{itemize}
\item An exclusive right to use, sell, and license the use of an invention granted by a government to the inventor
\end{itemize}

\note[item]{}
\end{frame}
\begin{frame}
\frametitle{Unrelated Title}


\begin{itemize}
\item Tangible protection such as alarms guards fireproof doors fences and vaults
\end{itemize}

\note[item]{}
\end{frame}
\begin{frame}
\frametitle{Unrelated Title}


\begin{itemize}
\item The unauthorized use duplication distribution or sale of copyrighted software
\end{itemize}

\note[item]{}
\end{frame}
\begin{frame}
\frametitle{Unrelated Title}


\begin{itemize}
\item Supports the deployment of entire systems including hardware networking and application using a pay per use revenue model
\end{itemize}

\note[item]{}
\end{frame}
\begin{frame}
\frametitle{Unrelated Title}


\begin{itemize}
\item The right to be left alone when you want to control your personal possessions and not to be observed without your consent
\end{itemize}

\note[item]{}
\end{frame}
\begin{frame}
\frametitle{Unrelated Title}


\begin{itemize}
\item Serves only one customer or organization and can be located on the customers premises or off the customer's premises
\end{itemize}

\note[item]{}
\end{frame}
\begin{frame}
\frametitle{Unrelated Title}


\begin{itemize}
\item Describes all the activities managers do to help companies create goods
\end{itemize}

\note[item]{}
\end{frame}
\begin{frame}
\frametitle{Unrelated Title}


\begin{itemize}
\item Promotes massive global industrywide applications offered to the general public
\end{itemize}

\note[item]{}
\end{frame}
\begin{frame}
\frametitle{Unrelated Title}


\begin{itemize}
\item Delivers electronically using two-way digital technology
\end{itemize}

\note[item]{}
\end{frame}
\begin{frame}
\frametitle{Unrelated Title}


\begin{itemize}
\item The process of monitoring and responding to what is being said about a company individual product or brand
\end{itemize}

\note[item]{}
\end{frame}
\begin{frame}
\frametitle{Unrelated Title}


\begin{itemize}
\item Outlines the corporate guidelines or principles governing employee online communications
\end{itemize}

\note[item]{}
\end{frame}
\begin{frame}
\frametitle{Unrelated Title}


\begin{itemize}
\item Unsolicited email
\end{itemize}

\note[item]{}
\end{frame}
\begin{frame}
\frametitle{Unrelated Title}


\begin{itemize}
\item Combines multiple network storage device so they appear to be the single stores device
\end{itemize}

\note[item]{}
\end{frame}
\begin{frame}
\frametitle{Unrelated Title}


\begin{itemize}
\item Consists of several standalone businesses
\end{itemize}

\note[item]{}
\end{frame}
\begin{frame}
\frametitle{Unrelated Title}


\begin{itemize}
\item Focus on the long-range planning such as plant size location and type process to be used
\end{itemize}

\note[item]{}
\end{frame}
\begin{frame}
\frametitle{Unrelated Title}


\begin{itemize}
\item Refers to the safe disposal of MIS assets at the end of their life cycle
\end{itemize}

\note[item]{}
\end{frame}
\begin{frame}
\frametitle{Unrelated Title}


\begin{itemize}
\item Describes the production, management, use, and disposal of technology in a way that minimizes damage to the environment
\end{itemize}

\note[item]{}
\end{frame}
\begin{frame}
\frametitle{Unrelated Title}


\begin{itemize}
\item The ability to present the resources of a single computer as if it is a collection of separate computers (VM) each with its own virtual CPUs network interfaces storage and operating system
\end{itemize}

\note[item]{}
\end{frame}
\begin{frame}
\frametitle{Unrelated Title}


\begin{itemize}
\item Focuses on producing goods and services as efficiently as possible within the strategic plan
\end{itemize}

\note[item]{}
\end{frame}
\begin{frame}
\frametitle{Unrelated Title}


\begin{itemize}
\item Anti-spamming approach where the receiving computer launches a return attack against the spammer sending email messages back to the computer that originated the suspected spam
\end{itemize}

\note[item]{}
\end{frame}
\begin{frame}
\frametitle{Unrelated Title}


\begin{itemize}
\item An act or object that poses a danger to assets
\end{itemize}

\note[item]{}
\end{frame}
\begin{frame}
\frametitle{Unrelated Title}


\begin{itemize}
\item When business data flows across international boundaries over the telecommunications networks of global information systems
\end{itemize}

\note[item]{}
\end{frame}
\begin{frame}
\frametitle{Unrelated Title}


\begin{itemize}
\item The technical core especially in manufacturing organization the actual conversion of input and output
\end{itemize}

\note[item]{}
\end{frame}
\begin{frame}
\frametitle{Unrelated Title}


\begin{itemize}
\item Tracks and analyze the movement of materials and products to ensure the delivery of materials and finished goods it is the right time the right place and lowest cost
\end{itemize}

\note[item]{}
\end{frame}
\begin{frame}
\frametitle{Unrelated Title}


\begin{itemize}
\item A problem that occurs when someone registers purposely misspelled variations of well-known domain names
\end{itemize}

\note[item]{}
\end{frame}
\begin{frame}
\frametitle{Unrelated Title}


\begin{itemize}
\item Reuses of refurbishes e-waste and creates a new product
\end{itemize}

\note[item]{}
\end{frame}
\begin{frame}
\frametitle{Unrelated Title}


\begin{itemize}
\item Offers a pay per use revenue model similar to metered service such as gas or electricity
\end{itemize}

\note[item]{}
\end{frame}
\begin{frame}
\frametitle{Unrelated Title}


\begin{itemize}
\item The term used to describe the difference between the cost of input and the value of price of output
\end{itemize}

\note[item]{}
\end{frame}
\begin{frame}
\frametitle{Unrelated Title}


\begin{itemize}
\item The creation of a virtual rather than actual version of a computer resource such as operating system a server storage Device or Network resources
\end{itemize}

\note[item]{}
\end{frame}
\begin{frame}
\frametitle{Unrelated Title}


\begin{itemize}
\item The theft of a websites name that occurs when someone posing as a site’s administrator changes the ownership of the domain name assigned to the website to another website owner
\end{itemize}

\note[item]{}
\end{frame}
\begin{frame}
\frametitle{Unrelated Title}


\begin{itemize}
\item Tracks activities of people using such measures as number of keystrokes error rate and number of transactions processed
\end{itemize}

\note[item]{}
\end{frame}
\begin{frame}
\frametitle{Unrelated Title}


\begin{itemize}
\item Information Type that encompasses all of the information contained within a single business process or unit of work, and its primary purpose is to support daily operational tasks.
\end{itemize}

\note[item]{}
\end{frame}
\begin{frame}
\frametitle{Unrelated Title}


\begin{itemize}
\item Information Type that encompasses all organizational information, and its primary purpose is to support the performing of managerial analysis tasks.
\end{itemize}

\note[item]{}
\end{frame}
\begin{frame}
\frametitle{Unrelated Title}


\begin{itemize}
\item A data dictionary compiles all of the metadata about the data elements in the data model.
\end{itemize}

\note[item]{}
\end{frame}
\begin{frame}
\frametitle{Unrelated Title}

\begin{center}
\includegraphics[width=0.9\textwidth,height=0.9\textheight,keepaspectratio]{/Users/I516998/Library/Application Support/Anki2/User 1/collection.media/clip_image001-6f18fe0b4fe52adbb381bfd90acba776109392e3.jpg}
\end{center}

\begin{itemize}
\item Increased Flexibility: Equally Flexibility in allowing each user to access the information in whatever way best suits his or her needs using logical and physical views.
\item Increased Scalability And Performance: scalable to handle the massive volumes of information and the large numbers of users, and perform quickly under heavy use. Data latency is the time it takes for data to be stored or retrieved.
\item Reduced Information Redundancy: by recording each piece of information in only one place of the database, relational sdatabase eliminate information redundancy. This saves disk space, makes performing information updates easier, and improves information quality.
\item Increased Information Integrity (Quality): Ensures quality of information through integrity constraints both (1) relational and (2) business critical.
\item Increased Information Security: databases offer many security features including passwords to provide authentication, access levels to determine who can access the data, and access controls to determine what type of access they have to the information.
\end{itemize}

\note[item]{}
\end{frame}
\begin{frame}
\frametitle{Unrelated Title}


\begin{itemize}
\item An interactive website kept constantly updated and relevant to the needs of its customers using a database. 
\item Data-driven capabilities are especially useful to offer large amounts of information to customers and partners based on unique search requirements.
\end{itemize}

\note[item]{}
\end{frame}
\begin{frame}
\frametitle{Unrelated Title}


\begin{itemize}
\item An integration allows separate systems to communicate directly with each other, eliminating the need for manual entry into multiple systems.
\item Without integrations, an organization will (1) spend considerable time entering the same information in multiple systems and (2) suffer from the low quality and inconsistency typically embedded in redundant information. Integrations ensure the consistency of it across multiple systems. A forward integration takes information entered into a given system and sends it automatically to all downstream systems and processes. A backward integration takes information entered into a given system and sends it automatically to all upstream systems and processes.
\item Organizations can also build a central repository for a particular type of information.
\end{itemize}

\note[item]{}
\end{frame}
\begin{frame}
\frametitle{Unrelated Title}


\begin{itemize}
\item 6.2. Describe a database, a database management system, and the
relational database model.
A database maintains information about various
types of objects (inventory), events (transactions), people (employees), and
places (warehouses). A database management system (DBMS) creates, reads,
updates, and deletes data in a database while controlling access and security.
A DBMS provides methodologies for creating, updating, storing, and retrieving
data in a database. In addition, a DBMS provides facilities for controlling
data access and security, allowing data sharing and enforcing data integrity.
The relational database model allows users to create, read, update, and delete
data in a relational database.
\end{itemize}

\note[item]{}
\end{frame}
\begin{frame}
\frametitle{Unrelated Title}


\begin{itemize}
\item 6.3. Identify the business advantages of a relational database.
Many business managers are familiar with Excel and other spreadsheet
programs they can use to store business data. Although spreadsheets are
excellent for supporting some data analysis, they offer limited functionality
in terms of security, accessibility, and flexibility and can rarely scale to
support business growth. From a business perspective, relational databases
offer many advantages over using a text document or a spreadsheet, including
increased flexibility, increased scalability and performance, reduced
information redundancy, increased information integrity (quality), and
increased information security.
\end{itemize}

\note[item]{}
\end{frame}
\begin{frame}
\frametitle{Unrelated Title}


\begin{itemize}
\item 6.4. Explain the business benefits of a data-driven website.
A data-driven website is an interactive website
kept constantly updated and relevant to the needs of its customers using a
database. Data-driven capabilities are especially useful when the website
offers a great deal of information, products, or services because visitors are
frequently annoyed if they are buried under an avalanche of information when
searching a website. Many companies use the web to make some of the information
in their internal databases available to customers and business partners.
\end{itemize}

\note[item]{}
\end{frame}
\begin{frame}
\frametitle{Unrelated Title}


\begin{itemize}
\item 6.5. Explain why an organization would want to integrate its
databases.
An integration allows separate systems to communicate
directly with each other. An organization can choose from two integration
methods. The first is to create forward and backward integrations that link
processes (and their underlying databases) in the value chain. A forward
integration takes information entered into a given system and sends it
automatically to all downstream systems and processes. A backward integration
takes information entered into a given system and sends it automatically to all
upstream systems and processes.
\end{itemize}

\note[item]{}
\end{frame}
\begin{frame}
\frametitle{Unrelated Title}


\begin{itemize}
\item Using the wrong information can lead managers to make erroneous decisions.
\item Erroneous decisions in turn can cost time, money, reputations, and even jobs.
\item Some of the serious business consequences that occur due to using low-quality information to make decisions are: ·         Inability to accurately track customers. ·         Difficulty identifying the organization’s most valuable customers. ·         Inability to identify selling opportunities. ·         Lost revenue opportunities from marketing to nonexistent customers. ·         The cost of sending nondeliverable mail. ·         Difficulty tracking revenue because of inaccurate invoices. ·         Inability to build strong relationships with customers. 
\item Using high-quality information can significantly improve the chances of making a good decision and directly increase an organization’s bottom line. For example, one company discovered that even with its large number of golf courses, Phoenix, Arizona, is not a good place to sell golf clubs because they are mostly tourists.
\item  
\item  
\end{itemize}

\note[item]{}
\end{frame}
\begin{frame}
\frametitle{Unrelated Title}


\begin{itemize}
\item Information is a vital resource and users need to be educated on what they can and cannot do with it. To ensure a firm manages its information correctly, it will need special policies and procedures establishing rules on how the information is organized, updated, maintained, and accessed.This information policy is data governance.
\item Data governance refers to the overall management of the availability, usability, integrity, and security of company data. Master data management (MDM) is the practice of gathering data and ensuring that it is uniform, accurate, consistent, and complete, including such entities as customers, suppliers, products, sales, employees, and other critical entities that are commonly integrated across organizational systems. 
\item A company that supports a data governance program has a defined a policy that specifies who is accountable for various portions or aspects of the data, including its accuracy, accessibility, consistency, timeliness, and completeness. The policy should clearly define the processes concerning how to store, archive, back up, and secure the data. In addition, the company should create a set of procedures identifying accessibility levels for employees. Then, the firm should deploy controls and procedures that enforce government regulations and compliance with mandates such as Sarbanes-Oxley.
\item  
\end{itemize}

\note[item]{}
\end{frame}
\begin{frame}
\frametitle{Unrelated Title}

\begin{center}
\includegraphics[width=0.9\textwidth,height=0.9\textheight,keepaspectratio]{/Users/I516998/Library/Application Support/Anki2/User 1/collection.media/paste-e618af9a44a5e71993755825bf2b936c3856d244.jpg}
\end{center}

\begin{itemize}
\item Virtualization is the creation of a virtual (rather than actual) version of computing resources, such as an operating system, a server, a storage device, or network resources. With big data it is now possible to virtualize data so that it can be stored efficiently and cost-effectively. Improvements in network speed and network reliability have removed the physical limitations of being able to manage massive amounts of data at an acceptable pace.
\end{itemize}

\note[item]{}
\end{frame}
\begin{frame}
\frametitle{Unrelated Title}


\begin{itemize}
\item Learning Outcome 8.1
\item Identify the four common characteristics of big data.
\item The four V’s of big data include variety, veracity, voluminous, and
\item velocity. 
\item Variety includes different forms of structured and unstructured data.
\item Veracity includes the uncertainty of data, including biases, noise, and
\item abnormalities.
\item Voluminous is the scale of data. 
\item Velocity is the analysis of
\item streaming data as it travels around the Internet.
\end{itemize}

\note[item]{}
\end{frame}
\begin{frame}
\frametitle{Unrelated Title}


\begin{itemize}
\item Learning Outcome 8.3
\item Explain the importance of data analytics and data visualization.
\item Algorithms are mathematical formulas placed in software that performs an analysis on a data set. Analytics is the science of fact-based decision making. Analytics uses software-based algorithms and statistics to derive meaning from data. Advanced analytics uses data patterns to make forward-looking predictions to explain to the organization where it is headed. Data visualization describes technologies that allow users to see or visualize data to transform information into a business perspective. Data visualization is a powerful way to simplify complex data sets by placing data in a format that is easily grasped and understood far quicker than the raw data alone.
\end{itemize}

\note[item]{}
\end{frame}
\begin{frame}
\frametitle{Unrelated Title}

\begin{center}
\includegraphics[width=0.9\textwidth,height=0.9\textheight,keepaspectratio]{/Users/I516998/Library/Application Support/Anki2/User 1/collection.media/paste-e01f191c323de679104b1f7a73e410ee7e0a6e6b.jpg}
\end{center}

\begin{itemize}
\item Business Understanding: Gain a clear understanding of the business problem that must be solved and how it impacts the company
\item Data Understanding: Analysis of all current data along with identifying any data quality issues
\item Data Preparation: Gather and organize the data in the correct formats and structures for analysis
\item Data Modeling: Apply mathematical techniques to identify trends and patterns in the data
\item Evaluation: Analyze the trends and patterns to assess the potential for solving the business problem
\item Deployment: Deploy the discoveries to the organization for work in everyday business
\end{itemize}

\note[item]{}
\end{frame}
\begin{frame}
\frametitle{Unrelated Title}


\begin{itemize}
\item Common data-mining techniques used to perform advanced analytics.
\item Estimation analysis determines values for an unknown continuous variable behavior or estimated future value. 
\item Affinity grouping analysis reveals the relationship between variables along with the nature and frequency of the relationships. 
\item Cluster analysis is a technique used to divide an information set into mutually exclusive groups such that the members of each group are as close together as possible to one another and the different groups are as far apart as possible.
\item Classification analysis is the process of organizing data into categories or groups for its most effective and efficient use.
\item  
\item Estimation analysis determines values for an unknown continuous variable behavior or estimated future value. Estimation models predict numeric outcomes based on historical data. An estimate is similar to a guess and is one of the least expensive modeling techniques. Many organizations use estimation analysis to determine the overall costs of a project from start to completion or estimates on the profits from introducing a new product line.
\item Affinity grouping analysis reveals the relationship between variables along with the nature and frequency of the relationships. Many people refer to affinity grouping algorithms as association rule generators because they create rules to determine the likelihood of events occurring together at a particular time or following each other in a logical progression. Percentages usually reflect the patterns of these events, for example, “55 percent of the time, events A and B occurred together” or “80 percent of the time that items A and B occurred together, they were followed by item C within three days.”
\item Cluster analysis is a technique used to divide an information set into mutually exclusive groups such that the members of each group are as close together as possible to one another and the different groups are as far apart as possible. A cluster analysis groups similar attributes together to discover segments or clusters, and then examine the attributes and values that define the clusters or segments. A great example of using cluster analysis in business is to create target-marketing strategies based on zip codes.
\item Classification analysis is the process of organizing data into categories or groups for its most effective and efficient use. For example, groups of political affiliation and charity donors. The primary goal of a classification analysis is not to explore data to find interesting segments, but to decide the best way to classify records. It is important to note that classification analysis is similar to cluster analysis because it segments data into distinct segments called classes; however, unlike cluster analysis, a classification analysis requires that all classes are defined before the analysis begins.
\end{itemize}

\note[item]{}
\end{frame}
\begin{frame}
\frametitle{Unrelated Title}


\begin{itemize}
\item Optimization Model: A statistical process that finds the way to make a design, system, or decision as effective as possible, for example, finding the values of controllable variables that determine maximal productivity or minimal waste.
\item Determine which products to produce given a limited amount of ingredientsChoose a combination of projects to maximize overall earningsForecasting Model: Time-series information is time-stamped information collected at a particular frequency. Forecasts are predictions based on time-series information allowing users to manipulate the time series for forecasting activities.
\item Web visits per hourSales per monthCustomer service calls per day  Regression Model: A statistical process for estimating the relationships among variables. Regression models include many techniques for modeling and analyzing several variables when the focus is on the relationship between a dependent variable and one or more independent variables.
\item Predict the winners of a marathon based on gender, height, weight, hours of trainingExplain how the quantity of weekly sales of a popular brand of beer depends on its price at a small chain of supermarkets 
\end{itemize}

\note[item]{}
\end{frame}
\begin{frame}
\frametitle{Unrelated Title}

\begin{center}
\includegraphics[width=0.9\textwidth,height=0.9\textheight,keepaspectratio]{/Users/I516998/Library/Application Support/Anki2/User 1/collection.media/clip_image001-ca1eb9d4fd6bd539d8caf8b9b2c62ef111ffe562.jpg}
\end{center}

\begin{itemize}
\item Wireless Communication Network Categories
\item A personal area network (PAN) provides communication for devices owned by a single user that work over a short distance, such as Bluetooth.
\item A wireless LAN (WLAN) is a local area network that uses radio signals to transmit and receive data over distances of a few hundred feet.
\item A wireless MAN (WMAN) is a metropolitan area network that uses radio signals to transmit and receive data. Worldwide Interoperability for Microwave Access (WiMAX), a communications technology aimed at providing high-speed wireless data over metropolitan area networks.
\item A wireless WAN (WWAN) is a wide area network that uses radio signals to transmit and receive data. WWAN technologies can be divided into two categories: cellular communication systems and satellite communication systems.
\end{itemize}

\note[item]{}
\end{frame}
\begin{frame}
\frametitle{Unrelated Title}

\begin{center}
\includegraphics[width=0.9\textwidth,height=0.9\textheight,keepaspectratio]{/Users/I516998/Library/Application Support/Anki2/User 1/collection.media/clip_image001-a3d8f78443cbad68ebd4844d5199503c74d85000.jpg}
\end{center}

\begin{itemize}
\item Three business applications taking advantage of wireless technologies. RFID: Radio-Frequency IdentificationGPS: Global Positioning SystemGIS: Geographic Information Systems
\end{itemize}

\note[item]{}
\end{frame}
\begin{frame}
\frametitle{Unrelated Title}


\begin{itemize}
\item Learning Outcome 16.1
\item Describe the different wireless network categories. 
\item There are four types of wireless networks—PAN, WLAN, WMAN, and WWAN. A PAN provides communication over a short distance that is intended for use with devices that are owned and operated by a single user. A WLAN is a local area network that uses radio signals to transmit and receive data over distances of a few hundred feet. A WMAN is a metropolitan area network that uses radio signals to transmit and receive data, and a WWAN is a wide area network that uses radio signals to transmit and receive data.
\end{itemize}

\note[item]{}
\end{frame}
\begin{frame}
\frametitle{Unrelated Title}


\begin{itemize}
\item Learning Outcome 16.2
\item Explain the different wireless network business applications. 
\item Mobile and wireless business applications and services are using satellite technologies. These technologies are GPS, GIS, and LBS. GPS is a satellite-based navigation system providing extremely accurate position, time, and speed information. GIS is location information that can be shown on a map. LBSs are applications that use location information to provide a service that both GPS and GIS use.
\end{itemize}

\note[item]{}
\end{frame}
\begin{frame}
\frametitle{Unrelated Title}


\begin{itemize}
\item Mobile business (or mbusiness, mcommerce) is the ability to purchase goods and services through a wireless Internet-enabled device. The emerging technology behind mbusiness is a mobile device equipped with a web-ready micro-browser that can perform the following services: ·         Mobile entertainment—downloads for music, videos, games, voting, ring tones, as well as text-based messaging services. ·         Mobile sales/marketing—advertising, campaigns, discounts, promotions, and coupons. ·         Mobile banking—manage accounts, pay bills, receive alerts, and transfer funds. ·         Mobile ticketing—purchase tickets for entertainment, transportation, and parking including the ability to automatically feed parking meters. ·         Mobile payments—pay for goods and services including in-store purchases, home delivery, vending machines, taxis, gas, and so on.
\end{itemize}

\note[item]{}
\end{frame}
\begin{frame}
\frametitle{Unrelated Title}

\begin{center}
\includegraphics[width=0.9\textwidth,height=0.9\textheight,keepaspectratio]{/Users/I516998/Library/Application Support/Anki2/User 1/collection.media/paste-50d1efbf6a4bf2ab1afccd5d5a08372cc6b83ce9.jpg}
\end{center}

\begin{itemize}
\item Stakeholders drive business strategies, and depending on the stakeholder’s perspective, the business strategy can change. It is not uncommon to find stakeholders’ business strategies have conflicting interests such as investors looking to increase profits by eliminating employee jobs. 
\item A business strategy is a leadership plan that achieves a specific set of goals or objectives such as increasing sales, decreasing costs, entering new markets, or developing new products or services.
\item A stakeholder is a person or group that has an interest or concern in an organization.
\item Common Stakeholder Interests
\end{itemize}

\note[item]{}
\end{frame}
\begin{frame}
\frametitle{Unrelated Title}


\begin{itemize}
\item Learning Outcome 2.1
\item Explain why competitive advantages are temporary along with the four key areas of a SWOT analysis.
\item A SWOT analysis evaluates an organization’s strengths, weaknesses, opportunities, and threats to identify significant influences that work for or against business strategies. Strengths and weaknesses originate inside an organization, or internally. Opportunities and threats originate outside an organization, or externally, and cannot always be anticipated or controlled.
\end{itemize}

\note[item]{}
\end{frame}
\begin{frame}
\frametitle{Unrelated Title}


\begin{itemize}
\item The systems development life cycle (SDLC) is the overall process for developing information systems, from planning and analysis through implementation and maintenance. The SDLC is the foundation for all systems development methods, and hundreds of different activities are associated with each phase. These activities typically include determining budgets, gathering system requirements, and writing detailed user documentation.
\end{itemize}

\note[item]{}
\end{frame}
\begin{frame}
\frametitle{Unrelated Title}


\begin{itemize}
\item Planning is the first and most critical phase of any systems development effort, regardless of whether the effort is to develop a system that allows customers to order products online, determine the best logistical structure for warehouses around the world, or develop a strategic information alliance with another organization.
\end{itemize}

\note[item]{}
\end{frame}
\begin{frame}
\frametitle{Unrelated Title}


\begin{itemize}
\item Four. Each gate consists of executable iterations of the software in development. A project stays in a gate waiting for the stakeholder’s analysis, and then it either moves to the next gate or is cancelled.
\item Gate one: inception. This phase
\item ensures all stakeholders have a shared understanding of the proposed system and
\item what it will do.
\item Gate two: elaboration. This
\item phase expands on the agreed-upon details of the system, including the ability
\item to provide an architecture to support and build it.
\item Gate three: construction. This
\item phase includes building and developing the product.
\item Gate four: transition. Primary
\item questions answered in this phase address ownership of the system and training
\item of key personnel.
\end{itemize}

\note[item]{}
\end{frame}
\begin{frame}
\frametitle{Unrelated Title}

\begin{center}
\includegraphics[width=0.9\textwidth,height=0.9\textheight,keepaspectratio]{/Users/I516998/Library/Application Support/Anki2/User 1/collection.media/paste-3388f1bdaf2a26c247b380b8df8aeb608bc8f9c1.jpg}
\end{center}

\begin{itemize}
\item A PERT (Program Evaluation and Review Technique) chart is a graphical network model that depicts a project’s tasks and the relationships between them. PERT charts define dependency between project tasks before those tasks are scheduled.
\end{itemize}

\note[item]{}
\end{frame}
\begin{frame}
\frametitle{Unrelated Title}

\begin{center}
\includegraphics[width=0.9\textwidth,height=0.9\textheight,keepaspectratio]{/Users/I516998/Library/Application Support/Anki2/User 1/collection.media/paste-f912e94692c634be26fbdb005b52b754f5911893.jpg}
\end{center}

\begin{itemize}
\item A Gantt chart is a simple bar chart that lists project tasks vertically against the project’s time frame, listed horizontally. A Gantt chart works well for representing the project schedule. It also shows actual progress of tasks against the planned duration.
\end{itemize}

\note[item]{}
\end{frame}
\begin{frame}
\frametitle{Unrelated Title}

\begin{center}
\includegraphics[width=0.9\textwidth,height=0.9\textheight,keepaspectratio]{/Users/I516998/Library/Application Support/Anki2/User 1/collection.media/paste-4952867ef01b05359bcfa6750349b38d37a72d9f.jpg}
\end{center}


\note[item]{}
\end{frame}
\begin{frame}
\frametitle{Unrelated Title}


\begin{itemize}
\item Ui/Ni (Einzelhandelsumsatz in Region i / Einzelhandelsrelevante Kaufkraft in Region i)
\end{itemize}

\note[item]{}
\end{frame}
\begin{frame}
\frametitle{Unrelated Title}


\begin{itemize}
\item EH relevante KK pro EW / Einzelhandelsumsatz pro EW aufs ganze Land bezogen
\item Kommt in dem Zähler des Zentralitätskoeffizienten
\end{itemize}

\note[item]{}
\end{frame}
\begin{frame}
\frametitle{Unrelated Title}

\begin{center}
\includegraphics[width=0.9\textwidth,height=0.9\textheight,keepaspectratio]{/Users/I516998/Library/Application Support/Anki2/User 1/collection.media/img3535031075684756134.jpg}
\end{center}


\note[item]{}
\end{frame}
\begin{frame}
\frametitle{Unrelated Title}


\begin{itemize}
\item Besser geeignet sind Werte unter 100%, da Sättigung noch nicht so hoch ist
\end{itemize}

\note[item]{}
\end{frame}
\begin{frame}
\frametitle{Unrelated Title}

\begin{center}
\includegraphics[width=0.9\textwidth,height=0.9\textheight,keepaspectratio]{/Users/I516998/Library/Application Support/Anki2/User 1/collection.media/img2903300120946625274.jpg}
\end{center}


\note[item]{}
\end{frame}
\begin{frame}
\frametitle{Unrelated Title}

\begin{center}
\includegraphics[width=0.9\textwidth,height=0.9\textheight,keepaspectratio]{/Users/I516998/Library/Application Support/Anki2/User 1/collection.media/img1541694571785150617.jpg}
\end{center}


\note[item]{}
\end{frame}
\begin{frame}
\frametitle{Unrelated Title}


\begin{itemize}
\item Nachfrage nach: Konsumgütern, Invest. Gütern, Ausgaben des Staates, Nachfrage des Auslands
\end{itemize}

\note[item]{}
\end{frame}
\begin{frame}
\frametitle{Unrelated Title}


\begin{itemize}
\item Y=C+I+G+Ex
\end{itemize}

\note[item]{}
\end{frame}
\begin{frame}
\frametitle{Unrelated Title}


\begin{itemize}
\item Keynes: Konsum abhängig von laufendem Einkommen (kurzfristig)
\item Absolute Einkommenshypothese
\item Friedmann: Konsum abhängig von zukünftigem Einkommen (langfristig)
\item Permanente Einkommenshypothese
\end{itemize}

\note[item]{}
\end{frame}
\begin{frame}
\frametitle{Unrelated Title}


\begin{itemize}
\item C=Ca+c*Y
\end{itemize}

\note[item]{}
\end{frame}
\begin{frame}
\frametitle{Unrelated Title}


\begin{itemize}
\item S=s*Y-Ca
\end{itemize}

\note[item]{}
\end{frame}
\begin{frame}
\frametitle{Unrelated Title}


\begin{itemize}
\item Klassik: techn. Größe, da von herrschenden Zins abhängig
\item Keynes: psych. Größe, da von Marktzins und erwartetem Risiko (Konjunktur) abhängig
\end{itemize}

\note[item]{}
\end{frame}
\begin{frame}
\frametitle{Unrelated Title}


\begin{itemize}
\item Y= 1/1-c*(Ca+I+Ex+G)
\end{itemize}

\note[item]{}
\end{frame}
\begin{frame}
\frametitle{Unrelated Title}


\begin{itemize}
\item dY=1/1-c*dG(dI,dEx)
\end{itemize}

\note[item]{}
\end{frame}
\begin{frame}
\frametitle{Unrelated Title}


\begin{itemize}
\item dY=1/1-0,7*100€
\item =333,33€
\end{itemize}

\note[item]{}
\end{frame}
\begin{frame}
\frametitle{Unrelated Title}


\begin{itemize}
\item Y=1/1-c*(Ca+I+G+Ex)
\item Y=1/1-0,8*(300+100+200+50)
\item =3250
\end{itemize}

\note[item]{}
\end{frame}
\begin{frame}
\frametitle{Unrelated Title}


\begin{itemize}
\item Veränderung der Steuern
\item "   " der Abschreibungsmöglichkeiten
\end{itemize}

\note[item]{}
\end{frame}
\begin{frame}
\frametitle{Unrelated Title}


\begin{itemize}
\item Veränderung der Staatsausgaben für Güter und Dienstleistungen
\item Veränderung der Transferzahlungen 
\end{itemize}

\note[item]{}
\end{frame}
\begin{frame}
\frametitle{Unrelated Title}


\begin{itemize}
\item Staatsverschuldung steigt
\item Time lags:
\item - Erkennungslag
\item - Handlungslag
\item - Wirkungslag
\end{itemize}

\note[item]{}
\end{frame}
\begin{frame}
\frametitle{Unrelated Title}

\begin{center}
\includegraphics[width=0.9\textwidth,height=0.9\textheight,keepaspectratio]{/Users/I516998/Library/Application Support/Anki2/User 1/collection.media/img3364925077400417586.jpg}
\end{center}


\note[item]{}
\end{frame}
\begin{frame}
\frametitle{Unrelated Title}


\begin{itemize}
\item => Unternehmen investieren selbst dann nicht, wenn Zinsen niedrig sind
\item Gründe:
\item - Vorhandene Produktionskapazität noch nicht ausgelastet
\item - Rendite zwischen Wertpapieranlage und Investition, evtl Wertpapier weniger risikoreich und somit bevorzugt
\end{itemize}

\note[item]{}
\end{frame}
\begin{frame}
\frametitle{Unrelated Title}


\begin{itemize}
\item Marktwirtschaft ist stabil und tendiert zur Vollbeschäftigung 
\item Potenzialorientierte Geldpolitik:
\item Änderung der Geldmenge muss an Wachstumsrate des BIP angepasst sein um Geldwertstabilität und Wirtschaftswachstum zu erreichen. Wenn Geldmengenwachstum höher als BIP Wachstum -> Inflation
\end{itemize}

\note[item]{}
\end{frame}
\begin{frame}
\frametitle{Unrelated Title}

\begin{center}
\includegraphics[width=0.9\textwidth,height=0.9\textheight,keepaspectratio]{/Users/I516998/Library/Application Support/Anki2/User 1/collection.media/img4593919072417901750.jpg}
\end{center}


\note[item]{}
\end{frame}
\begin{frame}
\frametitle{Unrelated Title}

\begin{center}
\includegraphics[width=0.9\textwidth,height=0.9\textheight,keepaspectratio]{/Users/I516998/Library/Application Support/Anki2/User 1/collection.media/img7464202366468965619.jpg}
\end{center}

\begin{itemize}
\item Markt ist nicht gesättigt, Nachfrage ist vorhanden
\item -> Fokus auf Angebotsseite
\end{itemize}

\note[item]{}
\end{frame}
\begin{frame}
\frametitle{Unrelated Title}


\begin{itemize}
\item 1. Say'sches Theorem
\item Jeder Produktion entspricht in gleicher Höhe geschaffenes Einkommen
\item 2. Flexibilität von Preisen
\item Gleichgewicht von Angebot und Nachfrage bei Vollbeschäftigung
\end{itemize}

\note[item]{}
\end{frame}
\begin{frame}
\frametitle{Unrelated Title}


\begin{itemize}
\item Annäherungsfunktion (sinkende Grenzprodukte) an Produktionsmaximum (Grenzprodukt)
\item Wertgrenzprodukt = Wert des Grenzproduktes (Summe)
\end{itemize}

\note[item]{}
\end{frame}
\begin{frame}
\frametitle{Unrelated Title}


\begin{itemize}
\item M*v=P*H
\item M= Geldmenge =Geldnachfrage (Ln)
\item v= Umlaufgeschwindigkeit =1/k(Kassenhaltungsdauer)
\item P= Preisniveau
\item H= Handelsvolumen
\end{itemize}

\note[item]{}
\end{frame}
\begin{frame}
\frametitle{Unrelated Title}


\begin{itemize}
\item dM+dV=dP+dBIP
\item Geldmengenwachstum soll so hoch sein wie BIP Wachstum, sonst Inflation
\end{itemize}

\note[item]{}
\end{frame}
\begin{frame}
\frametitle{Unrelated Title}


\begin{itemize}
\item Laissez-faire Standpunkt
\item Staat hält sich aus Wirtschaft größtenteils raus
\item Soll Land nur gegen Bedrohung von außen schützen
\item Recht und Ordnung wahren
\item Leistungen anbieten, die der private Sektor nicht liefern kann
\end{itemize}

\note[item]{}
\end{frame}
\begin{frame}
\frametitle{Unrelated Title}


\begin{itemize}
\item Ordnungspolitik hat Vorrang
\item Ziel: Preisniveaustabilität
\item Fokus auf Angebotsseite
\item Maßnahmen:
\item Senkung Steuer-, Sozialabgabenbelastung
\item Senkung Staatsquote
\item Abbau Staatsverschuldung
\item Senkung Lohnnebenkostenbelastung
\item Privatisierung
\item Abbau Bürokratie
\end{itemize}

\note[item]{}
\end{frame}
\begin{frame}
\frametitle{Unrelated Title}


\begin{itemize}
\item Gesamte Wirtschaftsleistung Inland pro Periode
\item BIP=Gütermenge*Marktpreise
\item Erfasst Endprodukte
\end{itemize}

\note[item]{}
\end{frame}
\begin{frame}
\frametitle{Unrelated Title}


\begin{itemize}
\item BIP 
\item +Primäreinkommen aus übriger Welt
\item -Primäreinkommen an übrige Welt
\item =BNE
\item BIP: Wert der Produktion, der von inländischen und ausländischen Produktionsfaktoren im Inland hergestellt wurde -> Inlandskonzept
\item BNE: Wert der Produktion der von inländischen Produktionsfaktoren im In- und Ausland hergestellt wurde -> Wohnsitzprinzip
\item In D BNE>BIP
\end{itemize}

\note[item]{}
\end{frame}
\begin{frame}
\frametitle{Unrelated Title}

\begin{center}
\includegraphics[width=0.9\textwidth,height=0.9\textheight,keepaspectratio]{/Users/I516998/Library/Application Support/Anki2/User 1/collection.media/img8350067523351161643.jpg}
\end{center}


\note[item]{}
\end{frame}
\begin{frame}
\frametitle{Unrelated Title}

\begin{center}
\includegraphics[width=0.9\textwidth,height=0.9\textheight,keepaspectratio]{/Users/I516998/Library/Application Support/Anki2/User 1/collection.media/img3243622531907793007.jpg}
\end{center}


\note[item]{}
\end{frame}
\begin{frame}
\frametitle{Unrelated Title}


\begin{itemize}
\item -2020 niedrige Inflationsrate (MwSt. USt. Senkungen)
\item -Nachholeffekte, öffentliche Konjunkturpakete
\item -Anstieg Energiepreise
\item -Co2-Steuer
\item -Lieferengpässe
\end{itemize}

\note[item]{}
\end{frame}
\begin{frame}
\frametitle{Unrelated Title}


\begin{itemize}
\item Höhere Löhne (z.B. von Gewerkschaften gefordert) führen zu höheren Produktionskosten -> Preise steigen, somit such die Inflation -> Gewerkschaften fordern wieder höhere Löhne für die Reallohnsicherung
\item Folge: Internationale Wettbewerbsfähigkeit gerät in Gefahr durch steigende Preise
\end{itemize}

\note[item]{}
\end{frame}
\begin{frame}
\frametitle{Unrelated Title}


\begin{itemize}
\item Kaufkraftverlust
\item Negative Allokations- und Wachstumseffekte
\item Flucht in Sachwerte und ins Ausland
\item Beeinträchtigung der internationalen Wettbewerbsfähigkeit
\item Lohn-Preis-Spirale
\item Umverteilungseffekte
\item -> Transfereinkommen-lag (Hartz IV)
\item -> Rente
\item -> Lohn-lag
\item -> Gläubiger-Schuldner-Hypothese
\item -> Staat als Inflationsgewinner (kalte Progression)
\end{itemize}

\note[item]{}
\end{frame}
\begin{frame}
\frametitle{Unrelated Title}


\begin{itemize}
\item Ärmere, da:
\item - andere Konsummuster -> je reicher die Haushalte, desto höher der Anteil an Waren und Dienstleistungen, deren Preise unterdurchschnittlich gestiegen sind 
\item Z.B. Elektronikartikel, welche sogar günstiger geworden sind
\item Ältere, da:
\item -Preise für Gesundheitswaren und Dienstleistungen stärker gestiegen sind
\end{itemize}

\note[item]{}
\end{frame}
\begin{frame}
\frametitle{Unrelated Title}


\begin{itemize}
\item Staat=Inflationsgewinner, da:
\item Wenn Inflationsrate=Einkommenserhöhung (z.B. 3%), dann steigt auch die Steuerbelastung absolut und prozentual
\item -> bei inflatorischer Wirtschaft sind permanente Steuerreformen "nach unten" erforderlich
\end{itemize}

\note[item]{}
\end{frame}
\begin{frame}
\frametitle{Unrelated Title}


\begin{itemize}
\item Nachfrageinflation: 
\item Kurzfristiger Nachfrageüberhang führt zur Preiserhöhung
\item Geldmengeninflation:
\item Geldmenge steigt stärker als Gütermenge(BIP)
\item Angebotsinflation:
\item Preiserhöhung durch die Anbieter
\end{itemize}

\note[item]{}
\end{frame}
\begin{frame}
\frametitle{Unrelated Title}


\begin{itemize}
\item EZB:
\item -Preisstabilität
\item -Unterstützung allgemeiner Wirtschaftspolitik in der Gemeinschaft
\item -Entpolitisierung der Geldwertstabilität 
\item Priorität: Preisstabilität
\item Institutionell: keine Weisungen von Regierungen
\item Finanziell: verfügen über keinen eigenen Haushalt
\item Fed: 
\item -stabile Preise
\item -Höchstgrad an Beschäftigung 
\item -moderate Langfristzinsen
\item Keine Prioritätensetzung
\end{itemize}

\note[item]{}
\end{frame}
\begin{frame}
\frametitle{Unrelated Title}


\begin{itemize}
\item Geldmenge oder Zinsen
\item 2 Säulen:
\item 1. Geldmengenwachstun
\item 2. Indikatoren bezüglich Aussichten und Risiken für Preisstabilität
\end{itemize}

\note[item]{}
\end{frame}
\begin{frame}
\frametitle{Unrelated Title}

\begin{center}
\includegraphics[width=0.9\textwidth,height=0.9\textheight,keepaspectratio]{/Users/I516998/Library/Application Support/Anki2/User 1/collection.media/img2508146734327055485.jpg}
\end{center}


\note[item]{}
\end{frame}
\begin{frame}
\frametitle{Unrelated Title}


\begin{itemize}
\item Mengentender:
\item Zins ist festgelegt von EZB
\item EZB vergibt Geldmenge anteilsmäßig nach geforderter Geldmenge an Geschäftsbanken (Repartierungsquote) 
\item Folge: Geschäftsbanken fordern immer mehr Geld an, als sie wirklich benötigen, um bessere Quote zu haben
\item Zinstender:
\item Mindestzins von EZB festgelegt
\item Es erfolgt eine Vollverteilung in Reihenfolge der höchsten Zinsangebote der Geschäftsbanken 
\item Ab 2008 Mengentender mit Vollzuweisung:
\item Jeder bekommt das Geld, was er braucht. 
\item Trotzdem: Wertpapiersicherheit
\end{itemize}

\note[item]{}
\end{frame}
\begin{frame}
\frametitle{Unrelated Title}

\begin{center}
\includegraphics[width=0.9\textwidth,height=0.9\textheight,keepaspectratio]{/Users/I516998/Library/Application Support/Anki2/User 1/collection.media/img4622516812879728354.jpg}
\end{center}


\note[item]{}
\end{frame}
\begin{frame}
\frametitle{Unrelated Title}

\begin{center}
\includegraphics[width=0.9\textwidth,height=0.9\textheight,keepaspectratio]{/Users/I516998/Library/Application Support/Anki2/User 1/collection.media/img3698045871804139829.jpg}
\end{center}


\note[item]{}
\end{frame}
\begin{frame}
\frametitle{Unrelated Title}

\begin{center}
\includegraphics[width=0.9\textwidth,height=0.9\textheight,keepaspectratio]{/Users/I516998/Library/Application Support/Anki2/User 1/collection.media/img6404352223392317653.jpg}
\end{center}


\note[item]{}
\end{frame}
\begin{frame}
\frametitle{Unrelated Title}

\begin{center}
\includegraphics[width=0.9\textwidth,height=0.9\textheight,keepaspectratio]{/Users/I516998/Library/Application Support/Anki2/User 1/collection.media/img8417322660249962554.jpg}
\end{center}


\note[item]{}
\end{frame}
\begin{frame}
\frametitle{Unrelated Title}


\begin{itemize}
\item Jede Bank muss Mindestreserven in Höhe von 1% der Kundeneinlagen bei den nationalen Zentralbanken abführen.
\item Diese werden mit dem Refinanzierungssatz(Leitzins) verzinst. 
\end{itemize}

\note[item]{}
\end{frame}
\begin{frame}
\frametitle{Unrelated Title}


\begin{itemize}
\item Investitionsfalle
\item Geschäftsbanken geben Leitzins ungern und zeitverzögert weiter
\item Time lag:
\item Diagnose-lag
\item ->Erkennung der wirtschaftlichen Probleme
\item Handlungs-lag
\item ->bei EZB nur kurz
\item Wirkungs-lag
\item ->Auswirkung auf Handlungen der WI-Subjekte
\item Folge: Maßenahmen könnten prozyklisch wirken
\item Geldpolitik wirkt nur indirekt
\end{itemize}

\note[item]{}
\end{frame}
\begin{frame}
\frametitle{Unrelated Title}


\begin{itemize}
\item Unterschiedliche Geldpolitiken (Südländer locker wegen Arbeitslosigkeit, Nordländer strenger wegen Inflationsgefahr)
\item Staatsverschuldungen
\item Verbor, Kredite an Regierungen zu geben, sonst Verlagerung der Staatsonsolvenz auf EZB
\item EZB muss eine Geldpolitik für unterschiedlich starke Wirtschaften machen
\end{itemize}

\note[item]{}
\end{frame}
\begin{frame}
\frametitle{Unrelated Title}

\begin{center}
\includegraphics[width=0.9\textwidth,height=0.9\textheight,keepaspectratio]{/Users/I516998/Library/Application Support/Anki2/User 1/collection.media/img4180724236643440669.jpg}
\end{center}


\note[item]{}
\end{frame}
\begin{frame}
\frametitle{Unrelated Title}


\begin{itemize}
\item EZB kauft Staatsanleihen auf (durch Geschäftsbanken)
\item PEPP(Pandemic Emergency Purchase Programme)
\item Sollen die wirtschaft ankurbeln
\end{itemize}

\note[item]{}
\end{frame}
\begin{frame}
\frametitle{Unrelated Title}

\begin{center}
\includegraphics[width=0.9\textwidth,height=0.9\textheight,keepaspectratio]{/Users/I516998/Library/Application Support/Anki2/User 1/collection.media/img5435919014300553541.jpg}
\end{center}


\note[item]{}
\end{frame}
\begin{frame}
\frametitle{Unrelated Title}


\begin{itemize}
\item dC/dY
\end{itemize}

\note[item]{}
\end{frame}
\begin{frame}
\frametitle{Unrelated Title}


\begin{itemize}
\item Reduzierung der Geldmenge durch die EZB mittels Erhöhung des Leitzins
\end{itemize}

\note[item]{}
\end{frame}
\begin{frame}
\frametitle{Unrelated Title}


\begin{itemize}
\item Veränderbar
\item Eignung zur Zielerreichung
\item Bedingungen
\item Theoriebezug
\end{itemize}

\note[item]{}
\end{frame}
\begin{frame}
\frametitle{Unrelated Title}


\begin{itemize}
\item Neuorganisation
\item Reorganisation
\item Optimierung
\end{itemize}

\note[item]{}
\end{frame}
\begin{frame}
\frametitle{Unrelated Title}


\begin{itemize}
\item +
\item Entlastungsfunktion (für UN-Leitung)
\item Rationalisierungsfunktion (durch Arbeitsteilung)
\item Kostenminimierung (Ermöglichung Massenproduktion)
\item -
\item Erfolgseinbußen (durch Bürokratie)
\item Motivationseinbußen (durch eingeschränkten Entscheidungsspielraum und mangelnde Identifikation)
\end{itemize}

\note[item]{}
\end{frame}
\begin{frame}
\frametitle{Unrelated Title}


\begin{itemize}
\item Aufbau:
\item Herarchische Ordnung zur Regelung von Rechten und Pflichten
\item Ablauf: 
\item Regelung zur zeitlichen, räumlichen und personellen Festlegeung von Prozessen
\end{itemize}

\note[item]{}
\end{frame}
\begin{frame}
\frametitle{Unrelated Title}


\begin{itemize}
\item Analyse:
\item Aufgaben in Teile zergliedern
\item Synthese: 
\item Aufgabenteile zu neuen Aufgabeneinheiten zusammensetzen
\end{itemize}

\note[item]{}
\end{frame}
\begin{frame}
\frametitle{Unrelated Title}


\begin{itemize}
\item -Einmaligkeit der Bedingungen in ihrer Gesamtheit
\item -Ziel
\item -meist überdurchschnittliches Risiko
\item -Abgrenzung ggü anderen Vorhaben
\item -Projektspezifische Organisation 
\item -Begrenzungen:
\item •zeitlich
\item •finanziell
\item •personell
\end{itemize}

\note[item]{}
\end{frame}
\begin{frame}
\frametitle{Unrelated Title}


\begin{itemize}
\item 1.Ausgangsanalyse
\item 2. Zieldefinition
\item 3. Risiken 
\item 4. Planung
\item 5. Durchführungsphase
\item 6. Soll-/Ist-Vergleich
\end{itemize}

\note[item]{}
\end{frame}
\begin{frame}
\frametitle{Unrelated Title}


\begin{itemize}
\item Organisation
\item Standort
\item Geschäftsprozesse
\item Anwendung
\item Technologie
\item Daten
\end{itemize}

\note[item]{}
\end{frame}
\begin{frame}
\frametitle{Unrelated Title}


\begin{itemize}
\item Quantitative und qualitative Festlegung des Projektinhaltes ( Kosten, Dauer, muss-, kann-Ziele)
\item Immer:
\item -positiv formuliert
\item -sinnvoll und mit Auftraggeber abgestimmt
\item -erreichbar
\item -klar und eindeutig
\item -messbar
\end{itemize}

\note[item]{}
\end{frame}
\begin{frame}
\frametitle{Unrelated Title}


\begin{itemize}
\item Termin - Budget - Leistung
\end{itemize}

\note[item]{}
\end{frame}
\begin{frame}
\frametitle{Unrelated Title}


\begin{itemize}
\item Kaufmännisch
\item Technisch
\item (Termin)
\item Ressourcen
\item Politisch
\item Umwelteinflüsse
\end{itemize}

\note[item]{}
\end{frame}
\begin{frame}
\frametitle{Unrelated Title}


\begin{itemize}
\item Schadenshöhe - Eintrittswahrscheinlichkeit - Risikowert - Maßnahme - Kosten der Maßnahme -RW-Kosten der Maßnahme
\item Für R1, R2....
\end{itemize}

\note[item]{}
\end{frame}
\begin{frame}
\frametitle{Unrelated Title}


\begin{itemize}
\item Folge von Aktivitäten und Meilensteinen ohne Pufferzeit. 
\item Verzögerung führt zur Verschiebung des Endtermins
\end{itemize}

\note[item]{}
\end{frame}
\begin{frame}
\frametitle{Unrelated Title}


\begin{itemize}
\item Verknüpfung von Mitarbeitereinsatzplan und Projektablaufplan
\end{itemize}

\note[item]{}
\end{frame}
\begin{frame}
\frametitle{Unrelated Title}


\begin{itemize}
\item Ab 7-10 Projektmitarbeitern Vollzeit
\item Verantwortlich, dass das Projektziel mit geplanten Aufwänden und Kosten den Termin erreicht
\end{itemize}

\note[item]{}
\end{frame}
\begin{frame}
\frametitle{Unrelated Title}


\begin{itemize}
\item 20-100% ihrer Zeit im Projekt
\item Durchführungsverantwortlicher trägt Verantwortung für Aktivität
\end{itemize}

\note[item]{}
\end{frame}
\begin{frame}
\frametitle{Unrelated Title}


\begin{itemize}
\item Auftraggeber und Auftragnehmer
\item Trifft Entscheidungen, die über dem Kompetenzbereich des Projektleiters liegen
\item Erzhält Projektstatus und Informationen vom Projektleiter (Pl ist selbst nicht Teil des Ausschuss)
\end{itemize}

\note[item]{}
\end{frame}
\begin{frame}
\frametitle{Unrelated Title}

\begin{center}
\includegraphics[width=0.9\textwidth,height=0.9\textheight,keepaspectratio]{/Users/I516998/Library/Application Support/Anki2/User 1/collection.media/img6063616201099554411.jpg}
\end{center}


\note[item]{}
\end{frame}
\begin{frame}
\frametitle{Unrelated Title}


\begin{itemize}
\item +
\item Eindeutige Verantwortung des Projektleiters
\item Kurze Kommunikationswege
\item Geringer Streit um Ressourcen
\item Ausschließlicje Ausrichtung auf Projektziel
\item Schnelle Reaktionsmöglichkeiten
\item Hohe Identifilation der Projekt-MA
\item -
\item Freistellung und Wiedereingliederung
\item Isolierte Projektentwicklung unabhängig vom Unternehmen
\item Abteilungen geben MA ungerne frei, vorallem die guten
\end{itemize}

\note[item]{}
\end{frame}
\begin{frame}
\frametitle{Unrelated Title}

\begin{center}
\includegraphics[width=0.9\textwidth,height=0.9\textheight,keepaspectratio]{/Users/I516998/Library/Application Support/Anki2/User 1/collection.media/img7991818881070564713.jpg}
\end{center}


\note[item]{}
\end{frame}
\begin{frame}
\frametitle{Unrelated Title}

\begin{center}
\includegraphics[width=0.9\textwidth,height=0.9\textheight,keepaspectratio]{/Users/I516998/Library/Application Support/Anki2/User 1/collection.media/img3105253672352086891.jpg}
\end{center}


\note[item]{}
\end{frame}
\begin{frame}
\frametitle{Unrelated Title}


\begin{itemize}
\item Projekt-MA sind Diener zweier Herren -> Verunsicherung
\item Kompetenzabgrenzung zwischen Projekt und Linie kann schwierig sein
\item Linienvorgesetzter kann spontan Ressourcen abziehen
\end{itemize}

\note[item]{}
\end{frame}
\begin{frame}
\frametitle{Unrelated Title}

\begin{center}
\includegraphics[width=0.9\textwidth,height=0.9\textheight,keepaspectratio]{/Users/I516998/Library/Application Support/Anki2/User 1/collection.media/img4827752993490893914.jpg}
\end{center}


\note[item]{}
\end{frame}
\begin{frame}
\frametitle{Unrelated Title}

\begin{center}
\includegraphics[width=0.9\textwidth,height=0.9\textheight,keepaspectratio]{/Users/I516998/Library/Application Support/Anki2/User 1/collection.media/img215392971176289397.jpg}
\end{center}


\note[item]{}
\end{frame}
\begin{frame}
\frametitle{Unrelated Title}

\begin{center}
\includegraphics[width=0.9\textwidth,height=0.9\textheight,keepaspectratio]{/Users/I516998/Library/Application Support/Anki2/User 1/collection.media/img8586311027211236611.jpg}
\includegraphics[width=0.9\textwidth,height=0.9\textheight,keepaspectratio]{/Users/I516998/Library/Application Support/Anki2/User 1/collection.media/img2011067252395769544.jpg}
\end{center}


\note[item]{}
\end{frame}
\begin{frame}
\frametitle{Unrelated Title}


\begin{itemize}
\item Präventivmaßnahmen
\item Korrekturmaßnahmen
\item Eventualmaßnahmen
\item Leistung reduzieren
\item Kapazitäten erhöhen
\item Produktivität steigern
\end{itemize}

\note[item]{}
\end{frame}
\begin{frame}
\frametitle{Unrelated Title}


\begin{itemize}
\item Personalmarketingstrategie
\item Entgeltpolitik
\item Mitbestimmungspolitik
\item Betriebliche Sozialpolitik
\item Personalentwicklungspolitik
\end{itemize}

\note[item]{}
\end{frame}
\begin{frame}
\frametitle{Unrelated Title}


\begin{itemize}
\item Mitwirkung:
\item Informationsrecht ->Planung v Umbauten, Arbeitsplätzen und wirtsch. Angelegenheiten 
\item Vorschlagsrecht ->Personalplanung
\item Anhörungsrecht ->Kündigungen 
\item Beratungsrecht -> Betriebsänderung
\item Mitbestimmung:
\item Widerspruchsrecht (wird vor Gericht geprüft)->Einstellung neuer MA
\item Vetorecht (hält auch bor Gericht stand) ->Änderung der Arbeitsplätze oder Arbeitsumgebung 
\item Initiativrecht->Bildungsmaßnahmen, Kündigung LZKs
\end{itemize}

\note[item]{}
\end{frame}
\begin{frame}
\frametitle{Unrelated Title}


\begin{itemize}
\item Zeitfolge-Analyse:
\item Ausbildungsgang, Arbeitsplatzwechsel, Lücken
\item Positionsanalyse: 
\item beruflicher Auf- und Abstieg, Wechsel von Berufsgebieten
\item Firmen- und Branchen Analyse:
\item Bewerber aufgrund der bisherigen Tätigkeiten für Stellenbesetzung geeignet? 
\item Kontinuitäts-Analyse:
\item Gesamtberufliche Entwicklung
\end{itemize}

\note[item]{}
\end{frame}
\begin{frame}
\frametitle{Unrelated Title}


\begin{itemize}
\item MA-Beurteilung
\item Zielvereinbarung
\item Leistungsmatrix
\item Stellenstruktur
\item Entlohnungssystem
\end{itemize}

\note[item]{}
\end{frame}
\begin{frame}
\frametitle{Unrelated Title}


\begin{itemize}
\item Bildungsarbeit
\item Förderung
\item Organisations-Entwicklung/Arbeitsstrukturierung
\end{itemize}

\note[item]{}
\end{frame}
\begin{frame}
\frametitle{Unrelated Title}


\begin{itemize}
\item Fachkompetenz
\item Methodenkompetenz
\item Soziale Kompetenz
\item Personalkompetenz
\end{itemize}

\note[item]{}
\end{frame}
\begin{frame}
\frametitle{Unrelated Title}


\begin{itemize}
\item 1 Ziele definieren
\item 2 Strategisch planen
\item 3 Zentral steuern 
\item 4 Wissen archivieren
\item 5 Lernsysteme einführen
\end{itemize}

\note[item]{}
\end{frame}
\begin{frame}
\frametitle{Unrelated Title}


\begin{itemize}
\item Ordentlich
\item Außerordentlich
\item Änderungskündigung
\end{itemize}

\note[item]{}
\end{frame}
\begin{frame}
\frametitle{Unrelated Title}


\begin{itemize}
\item Objektorientiert
\item Prozessorientiert
\end{itemize}

\note[item]{}
\end{frame}
\begin{frame}
\frametitle{Unrelated Title}


\begin{itemize}
\item Verrichtungsorientiert
\item Objektorientiert
\end{itemize}

\note[item]{}
\end{frame}
\begin{frame}
\frametitle{Unrelated Title}


\begin{itemize}
\item Validity
\item Velocity
\item Volatility
\item Value
\item Variety
\item Volume
\end{itemize}

\note[item]{}
\end{frame}
\begin{frame}
\frametitle{Unrelated Title}

\begin{center}
\includegraphics[width=0.9\textwidth,height=0.9\textheight,keepaspectratio]{/Users/I516998/Library/Application Support/Anki2/User 1/collection.media/PNKF0NsKQsMFIpqOaRAdAAFBqDme4z1vQjzbVBDzg1dmD_3wXIXd5Y6IqneH1sE-bVAh9B2g7vSO2Ix13Llu7JXqkdZXjYJ9hj3H21a-aEKf9ABUnZI.png}
\end{center}


\note[item]{}
\end{frame}
\begin{frame}
\frametitle{Unrelated Title}


\begin{itemize}
\item Die Zugriffszeiten sind maßgeblich von der Geschwindigkeit des langsameren Speicher abhängig. Die zugriffe auf den schnellen Speicher sind so schnell, dass sie zu vernachlässigen sind. Lediglich das laden eines nicht vorhandenen Datums ist somit Zeitaufwendig!
\end{itemize}

\note[item]{}
\end{frame}
\begin{frame}
\frametitle{Unrelated Title}


\begin{itemize}
\item Das Grundprinzip für Cachining.
\item Betrachtet Zeitliche und Räumliche Lokalität
\end{itemize}

\note[item]{}
\end{frame}
\begin{frame}
\frametitle{Unrelated Title}


\begin{itemize}
\item Daten die oft gemeinsam gelesen werden sollten auch phyisch nah beieinander abgelegt werden um die umpositionierung zu vermeiden.
\end{itemize}

\note[item]{}
\end{frame}
\begin{frame}
\frametitle{Unrelated Title}

\begin{center}
\includegraphics[width=0.9\textwidth,height=0.9\textheight,keepaspectratio]{/Users/I516998/Library/Application Support/Anki2/User 1/collection.media/paste-d89fffb8bee65f2c05ffb247e842d4640f3b9d0c.jpg}
\end{center}


\note[item]{}
\end{frame}
\begin{frame}
\frametitle{Unrelated Title}


\begin{itemize}
\item Daten die potentiell in Zukunft gebraucht werden direkt mitladen.
\end{itemize}

\note[item]{}
\end{frame}
\begin{frame}
\frametitle{Unrelated Title}


\begin{itemize}
\item Sekundärspeicher ohne mechanische Teile
\end{itemize}

\note[item]{}
\end{frame}
\begin{frame}
\frametitle{Unrelated Title}


\begin{itemize}
\item O(1) - Konstanter wahlfreier Zugriff, da keine mechanischen Teile.
\item Funktioniert wie ein Array
\end{itemize}

\note[item]{}
\end{frame}
\begin{frame}
\frametitle{Unrelated Title}


\begin{itemize}
\item Langsames Löschen
\item Begrenze Lebensdauer
\end{itemize}

\note[item]{}
\end{frame}
\begin{frame}
\frametitle{Unrelated Title}

\begin{center}
\includegraphics[width=0.9\textwidth,height=0.9\textheight,keepaspectratio]{/Users/I516998/Library/Application Support/Anki2/User 1/collection.media/paste-33a1301cd559ba6129a323bc9d087ccdc951328a.jpg}
\end{center}

\begin{itemize}
\item Aufteilen der Daten auf mehrere Platten ohne Spiegelung
\end{itemize}

\note[item]{}
\end{frame}
\begin{frame}
\frametitle{Unrelated Title}


\begin{itemize}
\item "Redundant Array of Independent Disks"
\item Speicher verhält sich durch einen vorgeschalteten Controller wie eine Festplatte, egal ob Daten physisch verteilt und/oder gespiegelt sind
\end{itemize}

\note[item]{}
\end{frame}
\begin{frame}
\frametitle{Unrelated Title}

\begin{center}
\includegraphics[width=0.9\textwidth,height=0.9\textheight,keepaspectratio]{/Users/I516998/Library/Application Support/Anki2/User 1/collection.media/paste-a92486830a9d376c35485b0950793e0e13046557.jpg}
\end{center}

\begin{itemize}
\item Spiegelung der Daten
\end{itemize}

\note[item]{}
\end{frame}
\begin{frame}
\frametitle{Unrelated Title}

\begin{center}
\includegraphics[width=0.9\textwidth,height=0.9\textheight,keepaspectratio]{/Users/I516998/Library/Application Support/Anki2/User 1/collection.media/paste-18afbdc6c094b0dd02ae67991fe9f36beaf39dda.jpg}
\end{center}

\begin{itemize}
\item Aufteilung und Spiegelung der Daten
\end{itemize}

\note[item]{}
\end{frame}
\begin{frame}
\frametitle{Unrelated Title}

\begin{center}
\includegraphics[width=0.9\textwidth,height=0.9\textheight,keepaspectratio]{/Users/I516998/Library/Application Support/Anki2/User 1/collection.media/paste-dd66f211084d1b4126834595f2b82e15ffc4f893.jpg}
\end{center}

\begin{itemize}
\item Verteilen der Bits auf mehreren Platten
\item Verwendung von Error Correction Codes (Hamming Code) auf seperaten Platten
\end{itemize}

\note[item]{}
\end{frame}
\begin{frame}
\frametitle{Unrelated Title}

\begin{center}
\includegraphics[width=0.9\textwidth,height=0.9\textheight,keepaspectratio]{/Users/I516998/Library/Application Support/Anki2/User 1/collection.media/300px-RAID_3.svg.png}
\end{center}

\begin{itemize}
\item Verteilung der daten BYTES.
\item Paritätsinformation auf seperaten Platten
\end{itemize}

\note[item]{}
\end{frame}
\begin{frame}
\frametitle{Unrelated Title}

\begin{center}
\includegraphics[width=0.9\textwidth,height=0.9\textheight,keepaspectratio]{/Users/I516998/Library/Application Support/Anki2/User 1/collection.media/300px-RAID_4.svg.png}
\end{center}

\begin{itemize}
\item Verteilung der DatenBLÖCKE.
\item seperate Paritätsplatte.
\end{itemize}

\note[item]{}
\end{frame}
\begin{frame}
\frametitle{Unrelated Title}

\begin{center}
\includegraphics[width=0.9\textwidth,height=0.9\textheight,keepaspectratio]{/Users/I516998/Library/Application Support/Anki2/User 1/collection.media/300px-RAID_5.svg.png}
\end{center}

\begin{itemize}
\item RAID 4 mit verteilung der Paritätsinformation.
\item Keine dedizierten Paritätsplatten mehr.
\end{itemize}

\note[item]{}
\end{frame}
\begin{frame}
\frametitle{Unrelated Title}


\begin{itemize}
\item Zählt beispielsweise alle 1en im Bitstring. Wenn gerade Anzahl dann Paritätsbit = 0.
\item Zeigt an ob ein Fehler passiert ist oder nicht.
\item Zeigt nicht an wo der Fehler passiert ist.
\end{itemize}

\note[item]{}
\end{frame}
\begin{frame}
\frametitle{Unrelated Title}


\begin{itemize}
\item an den zweierpotenzen Paritätslücken im Bitstring schaffen.
\item Dezimalstellen der einser im Bitstring in Binär umrechnen und Positionen XOR.
\item Ergebnis des XOR sind die Paritätsbits die in die Lücken eingetragen werden müssen
\end{itemize}

\note[item]{}
\end{frame}
\begin{frame}
\frametitle{Unrelated Title}


\begin{itemize}
\item Alle dezimalstellen mit einer 1 in binär umrechnen und XOR nehmen.
\item Wenn ergebnis = 0 kein Fehler.
\item Sonst ergibt das Ergebnis die Stelle mit dem Bitflip in binär
\end{itemize}

\note[item]{}
\end{frame}
\begin{frame}
\frametitle{Unrelated Title}


\begin{itemize}
\item Datensätze haben eine eindeutige Addresse aus (Platte#, Spur#, Sektornummer)
\item Die Physischen Blöcke werden auf Seiten verteilt.
\end{itemize}

\note[item]{}
\end{frame}
\begin{frame}
\frametitle{Unrelated Title}


\begin{itemize}
\item Nicht-Spannsatz (Standard)
\item Ein Datensatz/Block der nicht mehr ans Ende einer angefangenen Seite passt muss auf eine neue Seite.
\item Spannsatz
\item Datensätze/Blöcke können sich über mehrere Seiten erstrecken 
\end{itemize}

\note[item]{}
\end{frame}
\begin{frame}
\frametitle{Unrelated Title}


\begin{itemize}
\item Wenn die Blockgröße sehr Groß ist
\end{itemize}

\note[item]{}
\end{frame}
\begin{frame}
\frametitle{Unrelated Title}

\begin{center}
\includegraphics[width=0.9\textwidth,height=0.9\textheight,keepaspectratio]{/Users/I516998/Library/Application Support/Anki2/User 1/collection.media/paste-8e62a943083bd36a866b948eceb5d0e1daf57b92.jpg}
\end{center}


\note[item]{}
\end{frame}
\begin{frame}
\frametitle{Unrelated Title}


\begin{itemize}
\item (Un-)fixierte Position 
\item Feste/Variable Länge
\end{itemize}

\note[item]{}
\end{frame}
\begin{frame}
\frametitle{Unrelated Title}


\begin{itemize}
\item Längeninformation vor jedem Attribut
\item Attributstartpointer (fixer länge) für jedes Attribut zu begin des Tupels
\end{itemize}

\note[item]{}
\end{frame}
\begin{frame}
\frametitle{Unrelated Title}


\begin{itemize}
\item Addressierung eines Tupels (Datensatzes) anhand von Seitennummer und Offset
\end{itemize}

\note[item]{}
\end{frame}
\begin{frame}
\frametitle{Unrelated Title}


\begin{itemize}
\item Run Length Encoding
\item Prefix Encoding
\item Delta Encoding
\item Dictionary Encoding
\item Bitvector Encoding
\end{itemize}

\note[item]{}
\end{frame}
\begin{frame}
\frametitle{Unrelated Title}


\begin{itemize}
\item Statt der Geschlechtsinformationen:
\item wwwmwwwmmmmmmmmmmwww
\item wird gespeichert
\item w3m1w3m8w3
\end{itemize}

\note[item]{}
\end{frame}
\begin{frame}
\frametitle{Unrelated Title}


\begin{itemize}
\item Statt 0001, 0002, 00010, 00053 wird gespeichert:
\item p=000 und p+1, p+2, p+10, p+53
\end{itemize}

\note[item]{}
\end{frame}
\begin{frame}
\frametitle{Unrelated Title}


\begin{itemize}
\item Speichern eines Wertes, der nächste Wert enthält nurnoch die Abweichung von diesem wert. Der dritte wert enthält die abweichung zum 2. 
\end{itemize}

\note[item]{}
\end{frame}
\begin{frame}
\frametitle{Unrelated Title}


\begin{itemize}
\item Für die Werte eines Attributes wird ein Set erstellt. Die Daten werden dann mit den IDs im Set ersetzt. 
\end{itemize}

\note[item]{}
\end{frame}
\begin{frame}
\frametitle{Unrelated Title}

\begin{center}
\includegraphics[width=0.9\textwidth,height=0.9\textheight,keepaspectratio]{/Users/I516998/Library/Application Support/Anki2/User 1/collection.media/paste-0452ef449f40f483e0445fa71a051d10beb921ae.jpg}
\end{center}

\begin{itemize}
\item Jeder Wert eines Attibuts bekommt eine eigene Spalte mit 0/1 jenachdem ob der Wert dieser Zeile vorher dieser Ausprägung ensprach oder nicht
\item  
\end{itemize}

\note[item]{}
\end{frame}
\begin{frame}
\frametitle{Unrelated Title}


\begin{itemize}
\item 512Bytes * 50 = 25K Bytes
\end{itemize}

\note[item]{}
\end{frame}
\begin{frame}
\frametitle{Unrelated Title}


\begin{itemize}
\item 512Bytes * 50 * 2000 = 50.000K Bytes
\end{itemize}

\note[item]{}
\end{frame}
\begin{frame}
\frametitle{Unrelated Title}


\begin{itemize}
\item 50.000KBytes * 5 * 2 = 500 GB
\end{itemize}

\note[item]{}
\end{frame}
\begin{frame}
\frametitle{Unrelated Title}


\begin{itemize}
\item #Spur/Platte = #Zylinder = 2000
\end{itemize}

\note[item]{}
\end{frame}
\begin{frame}
\frametitle{Unrelated Title}


\begin{itemize}
\item 256 nein - ist kleiner als ein Sektor
\item 2048 ja - 4 Sektoren breit
\item 51200 nein - größer als eine gesamte Spur
\end{itemize}

\note[item]{}
\end{frame}
\begin{frame}
\frametitle{Unrelated Title}


\begin{itemize}
\item In jedem Pufferrahmen wird auf der Seite die Seite Nummer überprüft
\item Skaliert sehr schlecht
\end{itemize}

\note[item]{}
\end{frame}
\begin{frame}
\frametitle{Unrelated Title}


\begin{itemize}
\item Suche über Zusatzstruktur: Seitenliste
\item Sortierter Liste mit allen Pufferseiten: Schnell aber hoher Änderungsaufwand
\item Suche in unsortierter Liste - immernoch linear aber nur die verweise und nicht der gesamte Puffer
\end{itemize}

\note[item]{}
\end{frame}
\begin{frame}
\frametitle{Unrelated Title}


\begin{itemize}
\item Lokale Strategien
\item Globale Strategien
\item Seitentypbezogene Strategien
\end{itemize}

\note[item]{}
\end{frame}
\begin{frame}
\frametitle{Unrelated Title}


\begin{itemize}
\item Puffer wird in disjunkte Bereiche für je eine Transaktion eingeteilt
\end{itemize}

\note[item]{}
\end{frame}
\begin{frame}
\frametitle{Unrelated Title}


\begin{itemize}
\item Berücksichtigen das Zugriffsverhalten aller Transaktionen insgesamt. Gemeinsamgenutze Seiten nur einmal im Puffer
\end{itemize}

\note[item]{}
\end{frame}
\begin{frame}
\frametitle{Unrelated Title}


\begin{itemize}
\item Unterschiedliche Behandlung der Seitentypen. Beispielsweise Data-Dictionary und Daten
\end{itemize}

\note[item]{}
\end{frame}
\begin{frame}
\frametitle{Unrelated Title}


\begin{itemize}
\item Demand-Paging (Genau eine Seite geladen)
\item Prefetching (Laden zusätzlicher Seiten die vermutlich gebraucht werden)
\end{itemize}

\note[item]{}
\end{frame}
\begin{frame}
\frametitle{Unrelated Title}


\begin{itemize}
\item Optimal
\item FIFO
\item LRU
\item LFU
\item DGCLOCK
\item Adaptive
\item Random
\end{itemize}

\note[item]{}
\end{frame}
\begin{frame}
\frametitle{Unrelated Title}


\begin{itemize}
\item Kombiniert die Kriterien Alter und Referenzhäufigkeit miteinander. 
\item Das jüngste Verhalten wird durch Gewichtung höher bewertet
\end{itemize}

\note[item]{}
\end{frame}
\begin{frame}
\frametitle{Unrelated Title}


\begin{itemize}
\item Regelmäßige Fullbackups (zb wöchentlich)
\item Incrementelle Backups zwischen Fullbackups (zb Täglich)
\item Wegschreiben aller committed Transaktionen - > Logfile (setz auf IncBackup oder Logfile auf) 
\end{itemize}

\note[item]{}
\end{frame}
\begin{frame}
\frametitle{Unrelated Title}


\begin{itemize}
\item (deutlich) >90% 
\end{itemize}

\note[item]{}
\end{frame}
\begin{frame}
\frametitle{Unrelated Title}


\begin{itemize}
\item DB Admin
\item Network Admin
\item Application Admin
\item Application Dev
\item Security Officer
\item User
\end{itemize}

\note[item]{}
\end{frame}
\begin{frame}
\frametitle{Unrelated Title}


\begin{itemize}
\item Installiert DB-Server und Tools
\item Plant Tablespaces
\item Optimiert DB Strukturen
\item Überwacht Userzugriffe
\item Plant und Exekutiert Backups
\item Archiviert Backups
\end{itemize}

\note[item]{}
\end{frame}
\begin{frame}
\frametitle{Unrelated Title}


\begin{itemize}
\item Enthält:
\item alle Informationen über das komplette Schema (Tabellen, Indizes, Views,
\item User, Proceduren, Trigger, ...)
\item Speicherverwaltung
\item Rechtevergabe
\item Default Werte und Integritätsbesingungen
\item ...
\end{itemize}

\note[item]{}
\end{frame}
\begin{frame}
\frametitle{Unrelated Title}

\begin{center}
\includegraphics[width=0.9\textwidth,height=0.9\textheight,keepaspectratio]{/Users/I516998/Library/Application Support/Anki2/User 1/collection.media/paste-004861274932b8cdd078ddae3f7861b3d41d0c19.jpg}
\end{center}


\note[item]{}
\end{frame}
\begin{frame}
\frametitle{Unrelated Title}

\begin{center}
\includegraphics[width=0.9\textwidth,height=0.9\textheight,keepaspectratio]{/Users/I516998/Library/Application Support/Anki2/User 1/collection.media/paste-11f04e8fa75ede8990528fb360ed224a27a99f80.jpg}
\end{center}


\note[item]{}
\end{frame}
\begin{frame}
\frametitle{Unrelated Title}


\begin{itemize}
\item Daten werden unsortiert gespeichert
\end{itemize}

\note[item]{}
\end{frame}
\begin{frame}
\frametitle{Unrelated Title}


\begin{itemize}
\item Daten werden sortiert gespeichert
\end{itemize}

\note[item]{}
\end{frame}
\begin{frame}
\frametitle{Unrelated Title}


\begin{itemize}
\item Daten werden basierend auf ihrem Hashwert in Buckets gespeichert (gestreut)
\end{itemize}

\note[item]{}
\end{frame}
\begin{frame}
\frametitle{Unrelated Title}


\begin{itemize}
\item ein Composite Index (Index über mehrere Attribute)
\end{itemize}

\note[item]{}
\end{frame}
\begin{frame}
\frametitle{Unrelated Title}


\begin{itemize}
\item Ein Index über die Primärschlüssel einer Relation
\end{itemize}

\note[item]{}
\end{frame}
\begin{frame}
\frametitle{Unrelated Title}


\begin{itemize}
\item Jeder nicht-primäre Zugriffspfad auf eine Relation
\end{itemize}

\note[item]{}
\end{frame}
\begin{frame}
\frametitle{Unrelated Title}


\begin{itemize}
\item Wenn Daten sortiert vorliegen, dann müssen immer nur der erste Wert einer Seite indiziert werden um die Information über alle Werte zu halten
\end{itemize}

\note[item]{}
\end{frame}
\begin{frame}
\frametitle{Unrelated Title}


\begin{itemize}
\item Alle zu indizierenden Werte benötigen einen eigenen Indexeintrag.
\item Immernoch weniger Speicher benötigt als die Originaldatei
\end{itemize}

\note[item]{}
\end{frame}
\begin{frame}
\frametitle{Unrelated Title}


\begin{itemize}
\item Ist in gleicher Form sortiert wie die interne Relation 
\end{itemize}

\note[item]{}
\end{frame}
\begin{frame}
\frametitle{Unrelated Title}


\begin{itemize}
\item Index ist nicht wie die Relation sortiert.
\end{itemize}

\note[item]{}
\end{frame}
\begin{frame}
\frametitle{Unrelated Title}


\begin{itemize}
\item Die Daten und der Index sind gleichartig Sortiert
\end{itemize}

\note[item]{}
\end{frame}
\begin{frame}
\frametitle{Unrelated Title}


\begin{itemize}
\item Der Index ist anders sortiert als die Daten
\end{itemize}

\note[item]{}
\end{frame}
\begin{frame}
\frametitle{Unrelated Title}


\begin{itemize}
\item Insert - einfach anhängen
\item Delete - lookup dann löschbit auf 0 Setzen
\item Lookup - falls kein Index lineare Suche O(n)
\end{itemize}

\note[item]{}
\end{frame}
\begin{frame}
\frametitle{Unrelated Title}


\begin{itemize}
\item Insert - erfordert Lookup und ggf aufteilen der Speicherseiten
\item Delete - Lookup und löschbit auf 0 setzen
\item Lookup - Sortierte Suche, mit index sogar noch schneller
\end{itemize}

\note[item]{}
\end{frame}
\begin{frame}
\frametitle{Unrelated Title}


\begin{itemize}
\item Werte können mehrfach vorkommen und im schlimmsten fall auf jeder Seite mindestens einmal.
\end{itemize}

\note[item]{}
\end{frame}
\begin{frame}
\frametitle{Unrelated Title}


\begin{itemize}
\item Mittels einer Baumstruktur werden Indexdateien Indiziert, bis in den Blättern die echten Datenseiten indiziert werden
\end{itemize}

\note[item]{}
\end{frame}
\begin{frame}
\frametitle{Unrelated Title}


\begin{itemize}
\item Die Hashfunktion sollte eine möglichst gleichmäßige Verteilung der Daten verursachen.
\end{itemize}

\note[item]{}
\end{frame}
\begin{frame}
\frametitle{Unrelated Title}

\begin{center}
\includegraphics[width=0.9\textwidth,height=0.9\textheight,keepaspectratio]{/Users/I516998/Library/Application Support/Anki2/User 1/collection.media/paste-07b3b694139771cf1df77d98aa5a1568f7a30e16.jpg}
\end{center}


\note[item]{}
\end{frame}
\begin{frame}
\frametitle{Unrelated Title}


\begin{itemize}
\item Nein, da die Daten nicht gleichartig Sortiert sind
\end{itemize}

\note[item]{}
\end{frame}
\begin{frame}
\frametitle{Unrelated Title}


\begin{itemize}
\item Ja, die Daten liegen in beiden Fällen gleichartig Sortiert vor.
\item Der dünn besetzte Index ist immer clustered.
\item Aber nicht jeder clustered index ist immer Dünn besetzt
\end{itemize}

\note[item]{}
\end{frame}
\begin{frame}
\frametitle{Unrelated Title}


\begin{itemize}
\item Nein, ein clustered index kann auch dicht besetzt sein
\end{itemize}

\note[item]{}
\end{frame}
\begin{frame}
\frametitle{Unrelated Title}


\begin{itemize}
\item Programmiersprache um prozedurale Strukturen wie Loops etc. mit SQL zu vereinen
\end{itemize}

\note[item]{}
\end{frame}
\begin{frame}
\frametitle{Unrelated Title}


\begin{itemize}
\item Funktion/Methode die auf der Datenbank hinterlegt ist.
\item Extrem schnell.
\item Kann parametrisiert ablaufen.
\end{itemize}

\note[item]{}
\end{frame}
\begin{frame}
\frametitle{Unrelated Title}


\begin{itemize}
\item i.d.R. im Data Dictionary
\end{itemize}

\note[item]{}
\end{frame}
\begin{frame}
\frametitle{Unrelated Title}


\begin{itemize}
\item Embedded SQL steht im Programmcode und wird mittels Precompiler übersetzt.
\item Dynamic SQL greift auf Libraries wie JDBC zurück.
\end{itemize}

\note[item]{}
\end{frame}
\begin{frame}
\frametitle{Unrelated Title}


\begin{itemize}
\item Nein, da die Daten Bytes verteilt sind und nicht die Datenblöcke
\end{itemize}

\note[item]{}
\end{frame}
\begin{frame}
\frametitle{Unrelated Title}


\begin{itemize}
\item 145 = 0b10010001
\item Parity Spacing = 1001_000_1__
\item XOR ones = 12^9^3 = 6
\item 6 = 0b0110 
\item 100100001110
\end{itemize}

\note[item]{}
\end{frame}
\begin{frame}
\frametitle{Unrelated Title}


\begin{itemize}
\item network-attached storage (NAS)
\item Überträgt Files
\end{itemize}

\note[item]{}
\end{frame}
\begin{frame}
\frametitle{Unrelated Title}


\begin{itemize}
\item (Storage Area Network)
\item Beliebige Serverzuordnung (Shared Disk)
\item Im Gegensatz zu DAS und NAS Übertragung von Blöcken möglich (vorteilhaft für DBMS)
\end{itemize}

\note[item]{}
\end{frame}
\begin{frame}
\frametitle{Unrelated Title}


\begin{itemize}
\item Direct Attached Storage
\item Explizit einem Server zugewiesen
\end{itemize}

\note[item]{}
\end{frame}
\begin{frame}
\frametitle{Unrelated Title}


\begin{itemize}
\item Nein
\end{itemize}

\note[item]{}
\end{frame}
\begin{frame}
\frametitle{Unrelated Title}

\begin{center}
\includegraphics[width=0.9\textwidth,height=0.9\textheight,keepaspectratio]{/Users/I516998/Library/Application Support/Anki2/User 1/collection.media/paste-5790b15e9e88a99494d6bfb7e3be7471326ce586.jpg}
\end{center}


\note[item]{}
\end{frame}
\begin{frame}
\frametitle{Unrelated Title}

\begin{center}
\includegraphics[width=0.9\textwidth,height=0.9\textheight,keepaspectratio]{/Users/I516998/Library/Application Support/Anki2/User 1/collection.media/paste-172235b4c4748a2ef13e1dab89114464ed9e3e18.jpg}
\includegraphics[width=0.9\textwidth,height=0.9\textheight,keepaspectratio]{/Users/I516998/Library/Application Support/Anki2/User 1/collection.media/paste-b7e0606e8a6204e4ddce2bde22b87e3713681286.jpg}
\end{center}

\begin{itemize}
\item Dictionary:
\item Data:
\end{itemize}

\note[item]{}
\end{frame}
\begin{frame}
\frametitle{Unrelated Title}


\begin{itemize}
\item 40 + 40 + 60 + 90 + 135 = 265
\item (PCTINCREASE 50 heißt nächster Exten hat 150% der größe des letzten Extends NICHT 50% der gesamten Tabelle o.Ä.)
\end{itemize}

\note[item]{}
\end{frame}
\begin{frame}
\frametitle{Unrelated Title}


\begin{itemize}
\item 4096 da 8^4
\end{itemize}

\note[item]{}
\end{frame}
\begin{frame}
\frametitle{Unrelated Title}


\begin{itemize}
\item lediglich die Ergebnismenge
\end{itemize}

\note[item]{}
\end{frame}
\begin{frame}
\frametitle{Unrelated Title}


\begin{itemize}
\item Einen möglichst effizienten Weg zur errechnung der Ergebnismenge zu finden
\end{itemize}

\note[item]{}
\end{frame}
\begin{frame}
\frametitle{Unrelated Title}


\begin{itemize}
\item Ja
\end{itemize}

\note[item]{}
\end{frame}
\begin{frame}
\frametitle{Unrelated Title}


\begin{itemize}
\item Selektion möglichst früh
\item Operationen zusammenfassen um zwischenergebnisse zu vermeiden
\item Keine redundanten Berechnungen
\item Gleiche Teilausdrücke zusammenfassen: zwischenergebnisse können wiederverwendet werden 
\end{itemize}

\note[item]{}
\end{frame}
\begin{frame}
\frametitle{Unrelated Title}


\begin{itemize}
\item Logische Optimierung
\item Physische Optimierung
\item Kostenbasierte Auswahl
\end{itemize}

\note[item]{}
\end{frame}
\begin{frame}
\frametitle{Unrelated Title}


\begin{itemize}
\item Übersetzung der Anfrage in Relationale Algebra
\item Optimieren dieses Terms (zb frühe selektion)
\end{itemize}

\note[item]{}
\end{frame}
\begin{frame}
\frametitle{Unrelated Title}


\begin{itemize}
\item Ersetzen der Algebraoperationen durch konkrete Algorithmen samt information der Indizes
\end{itemize}

\note[item]{}
\end{frame}
\begin{frame}
\frametitle{Unrelated Title}


\begin{itemize}
\item Auswahl des statistisch besten Plans aus vorangegangen Schichten
\end{itemize}

\note[item]{}
\end{frame}
\begin{frame}
\frametitle{Unrelated Title}


\begin{itemize}
\item Eine DEBUG option für den Entwickler um den Optimizer dazu zu bringen eine Bestimmte Aktion zu bevorzugen oder zu vermeiden
\end{itemize}

\note[item]{}
\end{frame}
\begin{frame}
\frametitle{Unrelated Title}


\begin{itemize}
\item Identifikation ist die Ausweisung einer Identität zb Benutzername
\item Authentisierung ist der Beweis dieser Identität zb Passwort
\item Gemeinsam bilden sie die Authentifizierung
\end{itemize}

\note[item]{}
\end{frame}
\begin{frame}
\frametitle{Unrelated Title}


\begin{itemize}
\item Regeln, die die erlaubten Arten des Zugriffs auf
Sicherheitsobjekte (zb Tabellen oder Tupel) durch Nutzer festlegen
\end{itemize}

\note[item]{}
\end{frame}
\begin{frame}
\frametitle{Unrelated Title}


\begin{itemize}
\item Discretionary Access Controll
\item Geben zu einem Subjekt s die möglichen Zugriffsarten t auf ein Objekt o an und ob s das recht an s' weitergeben darf
\item Quintupel (o, s, t, p, f)
\end{itemize}

\note[item]{}
\end{frame}
\begin{frame}
\frametitle{Unrelated Title}

\begin{center}
\includegraphics[width=0.9\textwidth,height=0.9\textheight,keepaspectratio]{/Users/I516998/Library/Application Support/Anki2/User 1/collection.media/paste-ebe2a3bec58175d00f4ca0251a781ba1eb2a8bd5.jpg}
\end{center}


\note[item]{}
\end{frame}
\begin{frame}
\frametitle{Unrelated Title}


\begin{itemize}
\item Mandatory Access Controll
\item Subjekte und Objekte sind transitiv Klassifiziert
\item Zugriff nur wenn Class(s)>=Class(o)
\end{itemize}

\note[item]{}
\end{frame}
\begin{frame}
\frametitle{Unrelated Title}


\begin{itemize}
\item .... s.Passwort =
\item "‘‘ or ‘x‘ = ‘x‘"
\end{itemize}

\note[item]{}
\end{frame}
\begin{frame}
\frametitle{Unrelated Title}

\begin{center}
\includegraphics[width=0.9\textwidth,height=0.9\textheight,keepaspectratio]{/Users/I516998/Library/Application Support/Anki2/User 1/collection.media/paste-46cc1f4c7c94ee193bff57bfeeb9168673ca173b.jpg}
\end{center}

\begin{itemize}
\item Parametrisierte Prepared Statements
\end{itemize}

\note[item]{}
\end{frame}
\begin{frame}
\frametitle{Unrelated Title}


\begin{itemize}
\item Schließen auf sensible Daten durch Ansammlung und Kombination von nicht sensitiven Daten
\end{itemize}

\note[item]{}
\end{frame}
\begin{frame}
\frametitle{Unrelated Title}


\begin{itemize}
\item einzelne Daten sind nicht sensitiv aber eine große Anzahl von Daten zusammen genommen
schon (Siehe Inferenz-Angriff)
\end{itemize}

\note[item]{}
\end{frame}
\begin{frame}
\frametitle{Unrelated Title}


\begin{itemize}
\item Geschützte Informationen können manchmal auch durch Betrachtung des
Datenwörterbuchs oder von Dateiverzeichnissen erhalten werden.Beispiel: Eine Datei mit dem Namen TomHolland.passwd zeigt, dass TomHolland ein Nutzer ist
\end{itemize}

\note[item]{}
\end{frame}
\begin{frame}
\frametitle{Unrelated Title}


\begin{itemize}
\item Eine Anwendung die sich als gutmütig tarnt, aber etwas anderes macht als erwartet.
\item Beispiel Anmeldung greift Daten ab
\end{itemize}

\note[item]{}
\end{frame}
\begin{frame}
\frametitle{Unrelated Title}


\begin{itemize}
\item Ansonsten extreme verlangsamung durch Wartezeiten.
\item Außerdem CPU resourcen verschwendet wenn nicht voll ausgelastet
\end{itemize}

\note[item]{}
\end{frame}
\begin{frame}
\frametitle{Unrelated Title}


\begin{itemize}
\item Dirty Read
\item Lost Update
\item Phantom Problem/Incorrect Summary
\end{itemize}

\note[item]{}
\end{frame}
\begin{frame}
\frametitle{Unrelated Title}


\begin{itemize}
\item ACID
\end{itemize}

\note[item]{}
\end{frame}
\begin{frame}
\frametitle{Unrelated Title}


\begin{itemize}
\item Atomizität - Eine Transaktion erscheint nach außen wie eine einzelne Operation
\end{itemize}

\note[item]{}
\end{frame}
\begin{frame}
\frametitle{Unrelated Title}


\begin{itemize}
\item Consitency - Die Datenbank ist vor und nach der Transaktion in einem Validen und konsitenten Zustand
\end{itemize}

\note[item]{}
\end{frame}
\begin{frame}
\frametitle{Unrelated Title}


\begin{itemize}
\item Isolation - TAs dürfen sich gegenseitig nicht beeinflussen
\end{itemize}

\note[item]{}
\end{frame}
\begin{frame}
\frametitle{Unrelated Title}


\begin{itemize}
\item Durability - Eine abgeschlossene Transaktion bleibt erhalten
\end{itemize}

\note[item]{}
\end{frame}
\begin{frame}
\frametitle{Unrelated Title}

\begin{center}
\includegraphics[width=0.9\textwidth,height=0.9\textheight,keepaspectratio]{/Users/I516998/Library/Application Support/Anki2/User 1/collection.media/paste-aeabe6dbecff0b37eeb9e4e880b6d3a6a440eab3.jpg}
\end{center}


\note[item]{}
\end{frame}
\begin{frame}
\frametitle{Unrelated Title}


\begin{itemize}
\item Nein Writelocks sind Exklusive Locks und dürfen nur gesetzt werden wenn kein bisheriges Lock vorhanden ist.
\end{itemize}

\note[item]{}
\end{frame}
\begin{frame}
\frametitle{Unrelated Title}


\begin{itemize}
\item Ja, zwei leseoperationen beeinflussen sich nicht
\end{itemize}

\note[item]{}
\end{frame}
\begin{frame}
\frametitle{Unrelated Title}


\begin{itemize}
\item Nein, da die erste Transaktion das Element exklusiv für sich beansprucht hat.
\end{itemize}

\note[item]{}
\end{frame}
\begin{frame}
\frametitle{Unrelated Title}

\begin{center}
\includegraphics[width=0.9\textwidth,height=0.9\textheight,keepaspectratio]{/Users/I516998/Library/Application Support/Anki2/User 1/collection.media/paste-8a14256f6771db38b28cc2d066188808ad8734f2.jpg}
\end{center}


\note[item]{}
\end{frame}
\begin{frame}
\frametitle{Unrelated Title}


\begin{itemize}
\item Eine Transaktion die nach freigegebener Sperre eine Leseoperation macht, obwohl die erste Transaktion ein Rollback durchführt führt zu einem Dirty Read
\end{itemize}

\note[item]{}
\end{frame}
\begin{frame}
\frametitle{Unrelated Title}


\begin{itemize}
\item Alle Sperren werden erst zum ende der Transaktion frei gegeben. Keine Dirty Reads mehr
\end{itemize}

\note[item]{}
\end{frame}
\begin{frame}
\frametitle{Unrelated Title}


\begin{itemize}
\item T1 wartet auf auflösung der Sperre S2 von T2 während
\item T2 wartet auf auflösung der Sperre S1 von T1
\end{itemize}

\note[item]{}
\end{frame}
\begin{frame}
\frametitle{Unrelated Title}


\begin{itemize}
\item Timeoutbasiert
\item Wartegraphen
\item (Dezentrale Deadlock erkennung)
\end{itemize}

\note[item]{}
\end{frame}
\begin{frame}
\frametitle{Unrelated Title}


\begin{itemize}
\item Nach x Zeiteinheiten ohne fortschritt einer Transaktion wird ein Deadlock angenommen.
\item Achtung! Timeoutintervallgröße sehr relevant
\end{itemize}

\note[item]{}
\end{frame}
\begin{frame}
\frametitle{Unrelated Title}


\begin{itemize}
\item Wound-Wait: neuere Transaktion wird zurückgerollt
\item Wait-Die: ältere Transaktion wird zurückgerollt
\end{itemize}

\note[item]{}
\end{frame}
\begin{frame}
\frametitle{Unrelated Title}


\begin{itemize}
\item Knoten als Transaktionen ziehen eine Verbindung zu einer Transaktion auf die sie warten.
\item Zyklen zeigen Deadlocks
\end{itemize}

\note[item]{}
\end{frame}
\begin{frame}
\frametitle{Unrelated Title}


\begin{itemize}
\item Jüngste TA
\item Wenigste Sperren
\item Meiste Sperren
\item Häufigste Deadlockbeteiligung
\item meiste Verklemmungen
\end{itemize}

\note[item]{}
\end{frame}
\begin{frame}
\frametitle{Unrelated Title}


\begin{itemize}
\item T5
\end{itemize}

\note[item]{}
\end{frame}
\begin{frame}
\frametitle{Unrelated Title}


\begin{itemize}
\item T1 (warum nicht T4? T2?)
\end{itemize}

\note[item]{}
\end{frame}
\begin{frame}
\frametitle{Unrelated Title}


\begin{itemize}
\item T4????
\end{itemize}

\note[item]{}
\end{frame}
\begin{frame}
\frametitle{Unrelated Title}


\begin{itemize}
\item T2 (oder T3?)
\end{itemize}

\note[item]{}
\end{frame}
\begin{frame}
\frametitle{Unrelated Title}


\begin{itemize}
\item einzelne Stationen führen Wartegraphen mit jeweils einem External Knoten die dann stellvertretend für alle ressourcen innerhalb der Station stehen
\end{itemize}

\note[item]{}
\end{frame}
\begin{frame}
\frametitle{Unrelated Title}


\begin{itemize}
\item Optimistisch:
\item Transaktion wird immer ausgeführt, erst bei Commit wird entschieden ob sie durchgeht oder
\item zurückgerollt wird
\item Pessimistisch:
\item Erst wird geprüft ob Locks bestehen, nur bei Konfliktfreiheit startet Transaktion
\end{itemize}

\note[item]{}
\end{frame}
\begin{frame}
\frametitle{Unrelated Title}


\begin{itemize}
\item Möglichst kurz laufen lassen
\item Viele lesezugriffe und einzelne Schreibzugriffe möglichst trennen
\item Möglichst nicht ganze Tabellen locken
\end{itemize}

\note[item]{}
\end{frame}
\begin{frame}
\frametitle{Unrelated Title}


\begin{itemize}
\item Verteile Datenbanken Management System
\end{itemize}

\note[item]{}
\end{frame}
\begin{frame}
\frametitle{Unrelated Title}


\begin{itemize}
\item Sammlungen von Datenbanken auf mehreren Rechnern.
\item Rechner sind über ein Kommunikationsnetz miteinander verbunden.
\item Jede Station kann autonom lokale Daten verarbeiten.
\end{itemize}

\note[item]{}
\end{frame}
\begin{frame}
\frametitle{Unrelated Title}


\begin{itemize}
\item Beschreibt die Verteilung der Infromationen in VDB
\item Logisch zusammengehörige Informationsmengen (Relationen) werden in (weitgehend) disjunkte Fragmente
\item (Untereinheiten) zerlegt
\end{itemize}

\note[item]{}
\end{frame}
\begin{frame}
\frametitle{Unrelated Title}


\begin{itemize}
\item Redundante/Replikative Allokation
\item Fragment R1 liegt beispielsweise in S1 und S2
\end{itemize}

\note[item]{}
\end{frame}
\begin{frame}
\frametitle{Unrelated Title}


\begin{itemize}
\item Redundant/Replikativ
\item Fragmente können auf mehreren Stationen liegen
\item Redundanzfrei
\item Jedes Fragment nur auf genau einer Station
\end{itemize}

\note[item]{}
\end{frame}
\begin{frame}
\frametitle{Unrelated Title}


\begin{itemize}
\item Horizontal:
\item Disjunkte Tupelmengen (Selektion)
\item Vertikal:
\item Attribute werden getrennt (Projektion)
\item Kombiniert:
\item Selektion und Projektion
\end{itemize}

\note[item]{}
\end{frame}
\begin{frame}
\frametitle{Unrelated Title}


\begin{itemize}
\item Rekonstruierbarkeit
\item Vollständigkeit
\item Disjunktheit (Bei vertikal ggf. doch nicht disjunkt)
\item !Fragmentierung soll keine Redundanz hervorrufen, lediglich ggf. die Allokation
\end{itemize}

\note[item]{}
\end{frame}
\begin{frame}
\frametitle{Unrelated Title}


\begin{itemize}
\item ggf. große Datenmengen müssen über das Netzwerk laufen
\end{itemize}

\note[item]{}
\end{frame}
\begin{frame}
\frametitle{Unrelated Title}


\begin{itemize}
\item Extract - Extraktion der Daten aus Zielsystem
\item Transport - Syntaktische und Semantische Transformation
\item Load - Übertragung ins Zielsystem
\end{itemize}

\note[item]{}
\end{frame}
\begin{frame}
\frametitle{Unrelated Title}

\begin{center}
\includegraphics[width=0.9\textwidth,height=0.9\textheight,keepaspectratio]{/Users/I516998/Library/Application Support/Anki2/User 1/collection.media/paste-be1e580b04e8eda5bcfd683763086d2a7461393c.jpg}
\end{center}


\note[item]{}
\end{frame}
\begin{frame}
\frametitle{Unrelated Title}


\begin{itemize}
\item Online Transaction Processing
\item Typ der Datenprozessierung optimiert für Transaktionelles und reaktives Arbeiten.
\end{itemize}

\note[item]{}
\end{frame}
\begin{frame}
\frametitle{Unrelated Title}


\begin{itemize}
\item Online analytical processing
\item Art der prozessierung von Daten optimiert für Analysen und Queries
\end{itemize}

\note[item]{}
\end{frame}
\begin{frame}
\frametitle{Unrelated Title}

\begin{center}
\includegraphics[width=0.9\textwidth,height=0.9\textheight,keepaspectratio]{/Users/I516998/Library/Application Support/Anki2/User 1/collection.media/paste-7d47791f66ab605e86d34fa9709ecd5bf1b39f6b.jpg}
\end{center}

\begin{itemize}
\item Daten werden nicht Tupelweise sondern Attributweise abgespeichert. Zusätzlich wird i.d.R. ein Dictionary Encoding angewandt.
\end{itemize}

\note[item]{}
\end{frame}
\begin{frame}
\frametitle{Unrelated Title}


\begin{itemize}
\item Viele Spalten werden gar nicht genutzt.
\item Viele Spalten haben geringe Kardinalität (Geschlecht, Land)
\item Viele Null/Default Werte (ggf noch bessere Kompression)
\end{itemize}

\note[item]{}
\end{frame}
\begin{frame}
\frametitle{Unrelated Title}


\begin{itemize}
\item Blades nutzen eigene Platine mit Mikroprozessor, Arbeisspeicher und
Plattenspeicher (und können auf einen shared Memory zugreifen?)
\end{itemize}

\note[item]{}
\end{frame}
\begin{frame}
\frametitle{Unrelated Title}


\begin{itemize}
\item Partitionierung legt Methode fest nach der die Daten verteilt werden in
\item unabhängige und distinkte Teile.
\item Parallele Ausnutzung mehrerer Core CPUs, die auf ein Shared Memory
\item zugreifen, erhöht Verarbeitungsgeschwindigkeit signifikant.
\end{itemize}

\note[item]{}
\end{frame}
\begin{frame}
\frametitle{Unrelated Title}


\begin{itemize}
\item Aufteilung einer Tabelle nach Attributgruppen (Projektion)
\end{itemize}

\note[item]{}
\end{frame}
\begin{frame}
\frametitle{Unrelated Title}


\begin{itemize}
\item Aufteilung einer Tabelle nach Datensätzen (Selektion)
\item zB. Kontinente, Länder, Geburtstagsbereiche
\end{itemize}

\note[item]{}
\end{frame}
\begin{frame}
\frametitle{Unrelated Title}


\begin{itemize}
\item 1. Alle Seiten aus sek. Speicher, die nicht im Hauptspeicher sind, komprimiert wegschreiben
\item 2. Alle HSP read-only Seiten wegsichern
\item 3. Alle HSP Seiten die committed sind wegsichern
\item 4. Bei Seiten mit offenen Transaktionsänderungen: vorangegangener Status wird mittels ReDo Logfile erstellt und weggesichert
\end{itemize}

\note[item]{}
\end{frame}
\begin{frame}
\frametitle{Unrelated Title}


\begin{itemize}
\item Was ist paasiert?
\item >junge Frau wurde erstochen
\item >wurde von mehreren Nachbarn beobachtet
\item => Verantwortungsdivusion
\item > je mehr Menschen potentiell helfen können desto weniger Menschen helfen
\item => Bystander Effekt
\end{itemize}

\note[item]{}
\end{frame}
\begin{frame}
\frametitle{Unrelated Title}


\begin{itemize}
\item 1. Die Macht der Situation
\item 2. Die Subjektivität der sozialen Situation
\end{itemize}

\note[item]{}
\end{frame}
\begin{frame}
\frametitle{Unrelated Title}


\begin{itemize}
\item 1.Philip Zimbardo
\item 2.
\item >randomisierte Zuweisung der Versuchspersonen als Gefangene oder Wärter
\item >Gefangene wurden nach kurzer Zeit schikaniert
\item >Experiment wurde nach sechs Tagen abgebrochen
\item 3.ethische Sicht (Abbruchkriterien unklar, hoher psych. Sterss)
\item 4.Methodische Unsauberkeit
\end{itemize}

\note[item]{}
\end{frame}
\begin{frame}
\frametitle{Unrelated Title}


\begin{itemize}
\item Fehler werden an der Persönlichkeit festgemacht, die Macht der Situation wird unterschätzt
\end{itemize}

\note[item]{}
\end{frame}
\begin{frame}
\frametitle{Unrelated Title}


\begin{itemize}
\item Menschen reagieren nicht auf objektive Situationen, sondern auf ihre subjektive Konstruktion derselben 
\item > Interpretation des Verhaltes
\item > Beeinflussung des eigenen Verhaltens
\end{itemize}

\note[item]{}
\end{frame}
\begin{frame}
\frametitle{Unrelated Title}


\begin{itemize}
\item Meinen das man Ergebnisse besser hätte Vorhersagen können, als dies wirklich der Fall war.
\end{itemize}

\note[item]{}
\end{frame}
\begin{frame}
\frametitle{Unrelated Title}


\begin{itemize}
\item Abstrakter theoretischer Begriff (Hilfeverhalten, prosoziales Verhalten)
\end{itemize}

\note[item]{}
\end{frame}
\begin{frame}
\frametitle{Unrelated Title}


\begin{itemize}
\item Zusammenhänge zwischen Konstrukten (Theorie der Verantwortungsdiffusion)
\end{itemize}

\note[item]{}
\end{frame}
\begin{frame}
\frametitle{Unrelated Title}


\begin{itemize}
\item >Messbare Repräsentation eines Konstrukts
\item >Abhängige Variable 
\item >Unabhägige Variable
\end{itemize}

\note[item]{}
\end{frame}
\begin{frame}
\frametitle{Unrelated Title}


\begin{itemize}
\item >Beschreibt vermuteten (Kausal)zusammenhang zwischen Variablen
\item >aus der Theorie abgeleitet
\end{itemize}

\note[item]{}
\end{frame}
\begin{frame}
\frametitle{Unrelated Title}


\begin{itemize}
\item Festlegung, wie ein Konstrukt messbar gemacht werden sollen
\end{itemize}

\note[item]{}
\end{frame}
\begin{frame}
\frametitle{Unrelated Title}


\begin{itemize}
\item >Beeinflusst vermutlich abhängige Variable
\item >identische experimentelle Bedingungen für alle Teilnehmenden
\end{itemize}

\note[item]{}
\end{frame}
\begin{frame}
\frametitle{Unrelated Title}


\begin{itemize}
\item >Reaktion, die vermutlich von der unabhängigen Variable beeinflusst
\item > Messung bei allen Teilnehmenden
\end{itemize}

\note[item]{}
\end{frame}
\begin{frame}
\frametitle{Unrelated Title}


\begin{itemize}
\item 1.Interne Validität
\item >Gültigkeit der Schlussfolgerung
\item >Änderung der unabhängigen Variable führt zu Veränderung der abhängigen Variable.
\item (Randomisierung, Kontrollgruppen)
\item 2.Externe Validität
\item >Betrifft die Generalisierbarkeit von Befunden auf andere als die untersuchten Situationen und Populationen
\end{itemize}

\note[item]{}
\end{frame}
\begin{frame}
\frametitle{Unrelated Title}


\begin{itemize}
\item Geheimhaltung von Informationen
\end{itemize}

\note[item]{}
\end{frame}
\begin{frame}
\frametitle{Unrelated Title}


\begin{itemize}
\item Untersuchung von Angriffen gegen kryptografische Methoden
\end{itemize}

\note[item]{}
\end{frame}
\begin{frame}
\frametitle{Unrelated Title}


\begin{itemize}
\item Kryptografie+Kryptoanalyse
\end{itemize}

\note[item]{}
\end{frame}
\begin{frame}
\frametitle{Unrelated Title}


\begin{itemize}
\item Kerckhoff-PrinzipEinfache SchlüsselerzeugungVer/Entschlüsselung für alle gültigen Schlüssel effizient
\end{itemize}

\note[item]{}
\end{frame}
\begin{frame}
\frametitle{Unrelated Title}


\begin{itemize}
\item Ciphertext onlyKnown plaintextChosen plaintextchosen ciphertext
\end{itemize}

\note[item]{}
\end{frame}
\begin{frame}
\frametitle{Unrelated Title}


\begin{itemize}
\item Erpressung ooder Diebstahl des Schlüssels
\end{itemize}

\note[item]{}
\end{frame}
\begin{frame}
\frametitle{Unrelated Title}


\begin{itemize}
\item Jeder mögliche Klartext ist gleich wahrscheinlichGeheimtext liefert nicht genug InformationenFolgerung: #Schlüssel>=#Möglicher Klartexte
\end{itemize}

\note[item]{}
\end{frame}
\begin{frame}
\frametitle{Unrelated Title}


\begin{itemize}
\item Schlüssellänge >= NachrichtenlängeSchlüsselbits wirklich zufälligSchlüssel einmal benutzenProbleme: Schlüsselaustausch und Zufall
\end{itemize}

\note[item]{}
\end{frame}
\begin{frame}
\frametitle{Unrelated Title}


\begin{itemize}
\item Zeichen werden nicht ersetzt, sondern die Reihenfolge geändert
\end{itemize}

\note[item]{}
\end{frame}
\begin{frame}
\frametitle{Unrelated Title}

\begin{center}
\includegraphics[width=0.9\textwidth,height=0.9\textheight,keepaspectratio]{/Users/I516998/Library/Application Support/Anki2/User 1/collection.media/paste-a7198203020f9da4c0c32d75b3723326ec99a86e.jpg}
\end{center}

\begin{itemize}
\item Verwirrung/Confusion: Schwere Ableitbarkeit von Schlüsseln aus Klar/GeheimtextZerstreuung/Diffusion: Schwere Ableitbarkeit von Klartext aus Geheimtext
\end{itemize}

\note[item]{}
\end{frame}
\begin{frame}
\frametitle{Unrelated Title}


\begin{itemize}
\item ExpansionSchlüsseladdition XORS-BoxenPermuation
\end{itemize}

\note[item]{}
\end{frame}
\begin{frame}
\frametitle{Unrelated Title}

\begin{center}
\includegraphics[width=0.9\textwidth,height=0.9\textheight,keepaspectratio]{/Users/I516998/Library/Application Support/Anki2/User 1/collection.media/paste-f906eeb491bcc8e57a64193efb43978f528bf3c1.jpg}
\end{center}


\note[item]{}
\end{frame}
\begin{frame}
\frametitle{Unrelated Title}


\begin{itemize}
\item SubBytes: Nichtlineare Subsitution mittels Multiplikation im endlichen KörperShiftRow: Transposition jeder ZeileMixColumn: Anwenden einer festen Operation auf jede SpalteAddRoundKey: XOR Key
\end{itemize}

\note[item]{}
\end{frame}
\begin{frame}
\frametitle{Unrelated Title}


\begin{itemize}
\item Algebraische Angriffe
\item Seitenkanalangriffe
\end{itemize}

\note[item]{}
\end{frame}
\begin{frame}
\frametitle{Unrelated Title}

\begin{center}
\includegraphics[width=0.9\textwidth,height=0.9\textheight,keepaspectratio]{/Users/I516998/Library/Application Support/Anki2/User 1/collection.media/paste-a79ce444c9c06c725e3435891630f255960a8ddd.jpg}
\end{center}


\note[item]{}
\end{frame}
\begin{frame}
\frametitle{Unrelated Title}

\begin{center}
\includegraphics[width=0.9\textwidth,height=0.9\textheight,keepaspectratio]{/Users/I516998/Library/Application Support/Anki2/User 1/collection.media/paste-48fb9daaeb220ac2408bf5f0d1b123093bcf17d6.jpg}
\end{center}


\note[item]{}
\end{frame}
\begin{frame}
\frametitle{Unrelated Title}


\begin{itemize}
\item Statt mod 𝑛=𝑝∙𝑞 zu rechnen, kann man auch getrennt mod 𝑝 und mod 𝑞 rechnen, und dann das
\item Ergebnis mod 𝑛 „liften“, falls p und q teilerfremd sind.
\end{itemize}

\note[item]{}
\end{frame}
\begin{frame}
\frametitle{Unrelated Title}

\begin{center}
\includegraphics[width=0.9\textwidth,height=0.9\textheight,keepaspectratio]{/Users/I516998/Library/Application Support/Anki2/User 1/collection.media/paste-ba53a9d1237edbc46a08e8e3671075133622489a.jpg}
\end{center}


\note[item]{}
\end{frame}
\begin{frame}
\frametitle{Unrelated Title}

\begin{center}
\includegraphics[width=0.9\textwidth,height=0.9\textheight,keepaspectratio]{/Users/I516998/Library/Application Support/Anki2/User 1/collection.media/paste-ce07972b1233a98a902e0cafc872db67a497a8b3.jpg}
\end{center}


\note[item]{}
\end{frame}
\begin{frame}
\frametitle{Unrelated Title}

\begin{center}
\includegraphics[width=0.9\textwidth,height=0.9\textheight,keepaspectratio]{/Users/I516998/Library/Application Support/Anki2/User 1/collection.media/paste-f452b3974eb9e1428c30bb52631139eb0ad63d87.jpg}
\end{center}


\note[item]{}
\end{frame}
\begin{frame}
\frametitle{Unrelated Title}


\begin{itemize}
\item Length-Extension-Attacks -> SHA3 mit Sponge Konstruktion
\end{itemize}

\note[item]{}
\end{frame}
\begin{frame}
\frametitle{Unrelated Title}

\begin{center}
\includegraphics[width=0.9\textwidth,height=0.9\textheight,keepaspectratio]{/Users/I516998/Library/Application Support/Anki2/User 1/collection.media/paste-35e7ffc86c7aabeb2245271653d916136ccc6fa4.jpg}
\end{center}


\note[item]{}
\end{frame}
\begin{frame}
\frametitle{Unrelated Title}

\begin{center}
\includegraphics[width=0.9\textwidth,height=0.9\textheight,keepaspectratio]{/Users/I516998/Library/Application Support/Anki2/User 1/collection.media/paste-794b77fa8ea9b0535d182a49630e62ea9db6f1f5.jpg}
\end{center}


\note[item]{}
\end{frame}
\begin{frame}
\frametitle{Unrelated Title}


\begin{itemize}
\item Bei H möglichen (gleich wahrscheinlichen) Hashwerten muss man im Mittel ca. 1,18∙Wurzel(𝐻) Werte
\item berechnen, um eine Kollision zu finden.
\end{itemize}

\note[item]{}
\end{frame}
\begin{frame}
\frametitle{Unrelated Title}


\begin{itemize}
\item Damit wir wissen wecher Schlüssel zu welchem Claim gehört. 
\item Durch TTP signierter Hashwert über ein Zertifikat. 
\item Das Zertifikat beweist unsere Identität
\end{itemize}

\note[item]{}
\end{frame}
\begin{frame}
\frametitle{Unrelated Title}


\begin{itemize}
\item Nutzer signieren Schlüssel gegenseitig: Key signing parties
\end{itemize}

\note[item]{}
\end{frame}
\begin{frame}
\frametitle{Unrelated Title}


\begin{itemize}
\item MITM Angriff verhindern
\end{itemize}

\note[item]{}
\end{frame}
\begin{frame}
\frametitle{Unrelated Title}


\begin{itemize}
\item Nutze eine passwortbasierte Schlüsselableitungsfunktion (Key Derivation Function, KDF) KDF), um den Schlüssel vom Passwort abzuleiten (siehe PKCS #5).
\item Nutze abgeleitete Schlüssel, um private/secret keys zu verschlüsseln.
\end{itemize}

\note[item]{}
\end{frame}
\begin{frame}
\frametitle{Unrelated Title}


\begin{itemize}
\item Message Integrität durch HMACMessage Vertraulichkeit durch symetrische VerschlüsselungAuthentizität durch public key Verschlüsselung
\end{itemize}

\note[item]{}
\end{frame}
\begin{frame}
\frametitle{Unrelated Title}

\begin{center}
\includegraphics[width=0.9\textwidth,height=0.9\textheight,keepaspectratio]{/Users/I516998/Library/Application Support/Anki2/User 1/collection.media/paste-d7e73d83d57ba09241c24a69d9b0bf4448bf4992.jpg}
\end{center}


\note[item]{}
\end{frame}
\begin{frame}
\frametitle{Unrelated Title}


\begin{itemize}
\item Verwendete Sitzungsschlüssel sollen nach Beendigung der Sitzung nicht mehr aus dem Langzeitschlüssel rekonstruiert werden können (Aufzeichnungen nicht entschlüsseln)
\end{itemize}

\note[item]{}
\end{frame}
\begin{frame}
\frametitle{Unrelated Title}


\begin{itemize}
\item Nicht vertrauenswürdige CALösung: Certificate pnning: Nicht blind CA vertrauen, weitere Charakteristiken berücksichtigen
\end{itemize}

\note[item]{}
\end{frame}
\begin{frame}
\frametitle{Unrelated Title}


\begin{itemize}
\item Beobachten der Paketgröße
\end{itemize}

\note[item]{}
\end{frame}
\begin{frame}
\frametitle{Unrelated Title}


\begin{itemize}
\item Entfernung unsicherher Verfahren wie SHA-1Unterstützung des AEAD-Modus (Authenticated Encryption with Associated Data) -> kein Padding OracleKein RSA SchlüsselaustauschKompletter Handshake signiertCipher Suites ersetzt durch getrenntes Aushandeln von: Cipher , Key Exchange und Signature Algorithm
\end{itemize}

\note[item]{}
\end{frame}
\begin{frame}
\frametitle{Unrelated Title}


\begin{itemize}
\item ArchitekturImplementierungBetrieb
\end{itemize}

\note[item]{}
\end{frame}
\begin{frame}
\frametitle{Unrelated Title}

\begin{center}
\includegraphics[width=0.9\textwidth,height=0.9\textheight,keepaspectratio]{/Users/I516998/Library/Application Support/Anki2/User 1/collection.media/paste-f2ed7b05fee9eec7c9b74ef51f47f2bfe6880f56.jpg}
\end{center}


\note[item]{}
\end{frame}
\begin{frame}
\frametitle{Unrelated Title}


\begin{itemize}
\item Keine Default PasswörterKeine BeispielnutzerDateien sind schreibgeschütztFehlermeldungen sind allgemein
\end{itemize}

\note[item]{}
\end{frame}
\begin{frame}
\frametitle{Unrelated Title}


\begin{itemize}
\item Prüfe jeden Zugriff auf jedes ObjektNur vertrauenswürdigen Eingaben vertrauenSecure Interface
\end{itemize}

\note[item]{}
\end{frame}
\begin{frame}
\frametitle{Unrelated Title}


\begin{itemize}
\item Teile das System in verschiedene, getrennte Teile aufMinimiere die Privilegien in jedem einzelnen TeilImplementiere keine Alles oder Nichts ModelleBsp: Sandbox-Umgebung
\end{itemize}

\note[item]{}
\end{frame}
\begin{frame}
\frametitle{Unrelated Title}


\begin{itemize}
\item Security Development Lifecycle
\end{itemize}

\note[item]{}
\end{frame}
\begin{frame}
\frametitle{Unrelated Title}


\begin{itemize}
\item Secure by DesignSecure by DefaultSecure in Deployment
\end{itemize}

\note[item]{}
\end{frame}
\begin{frame}
\frametitle{Unrelated Title}

\begin{center}
\includegraphics[width=0.9\textwidth,height=0.9\textheight,keepaspectratio]{/Users/I516998/Library/Application Support/Anki2/User 1/collection.media/paste-fdceb322549ee253abebe12ad6e8c2b67e81e671.jpg}
\end{center}


\note[item]{}
\end{frame}
\begin{frame}
\frametitle{Unrelated Title}


\begin{itemize}
\item Analyse, Bewertung und Entwicklung von Abwehrmaßnahmen der STRIDE Angriffe
\end{itemize}

\note[item]{}
\end{frame}
\begin{frame}
\frametitle{Unrelated Title}

\begin{center}
\includegraphics[width=0.9\textwidth,height=0.9\textheight,keepaspectratio]{/Users/I516998/Library/Application Support/Anki2/User 1/collection.media/paste-60131b18409f206b29242cb5b966d1ce099385ea.jpg}
\end{center}


\note[item]{}
\end{frame}
\begin{frame}
\frametitle{Unrelated Title}

\begin{center}
\includegraphics[width=0.9\textwidth,height=0.9\textheight,keepaspectratio]{/Users/I516998/Library/Application Support/Anki2/User 1/collection.media/paste-0db5c103f82afecca1f2543e688e932fbf428c20.jpg}
\end{center}


\note[item]{}
\end{frame}
\begin{frame}
\frametitle{Unrelated Title}


\begin{itemize}
\item Über seine Möglichkeiten mit dem System zu interagieren
\end{itemize}

\note[item]{}
\end{frame}
\begin{frame}
\frametitle{Unrelated Title}


\begin{itemize}
\item Sitzt am KommunikationswegBeobachtung, Erzeugung, Unterbrechung von NetzwerkverkehrMITM
\end{itemize}

\note[item]{}
\end{frame}
\begin{frame}
\frametitle{Unrelated Title}


\begin{itemize}
\item Reagiert mit einem entfernen System übers NetzwerkKompromitierung des Remote Systems über verfügbare InterfacesZiele: Code Execution, Exfilatrion von Informationen, Denial of Service
\end{itemize}

\note[item]{}
\end{frame}
\begin{frame}
\frametitle{Unrelated Title}


\begin{itemize}
\item Führe beliebigen Code aus (beschränkte Zugriffsrechte)Ziel: zusätzliche RechteBsp: Mehrbenutzersysteme, Schadsoftware
\end{itemize}

\note[item]{}
\end{frame}
\begin{frame}
\frametitle{Unrelated Title}


\begin{itemize}
\item Man in the browser: HTTP Anfragen aus Browser des Nutzers erzeugen (Sessions ausnutzen)XSS Angreifer
\end{itemize}

\note[item]{}
\end{frame}
\begin{frame}
\frametitle{Unrelated Title}


\begin{itemize}
\item 804438080
\end{itemize}

\note[item]{}
\end{frame}
\begin{frame}
\frametitle{Unrelated Title}


\begin{itemize}
\item Hauptanwendung (z.B. Apache)PluginsServletsSkripte
\end{itemize}

\note[item]{}
\end{frame}
\begin{frame}
\frametitle{Unrelated Title}


\begin{itemize}
\item Anfrage oder AntwortzeileHeaderoptionale Nachrichteninhalt
\end{itemize}

\note[item]{}
\end{frame}
\begin{frame}
\frametitle{Unrelated Title}


\begin{itemize}
\item ZustandslosMenschenlesbar da ASCII basiert
\end{itemize}

\note[item]{}
\end{frame}
\begin{frame}
\frametitle{Unrelated Title}


\begin{itemize}
\item Anfragezeile (z.B. GET  /index.html  HTTP/1.1)Liste von HTTP Header (Bsp. Host, Cookie)einer leeren ZeileOptional: Message body
\end{itemize}

\note[item]{}
\end{frame}
\begin{frame}
\frametitle{Unrelated Title}


\begin{itemize}
\item Verb Path ProtocolVerb/Methode ezeichnet die Aktion, die mit der Zielressource ausgeführt werden sollPath gibt die Ressource anProtokoll ist HTTP/1.0 oder 1.1 oder 2.0
\end{itemize}

\note[item]{}
\end{frame}
\begin{frame}
\frametitle{Unrelated Title}


\begin{itemize}
\item Zweck: Ressource aufrufenKeine Seiteneiffekte auf den WebserverKein message-BodyParameter in URL
\end{itemize}

\note[item]{}
\end{frame}
\begin{frame}
\frametitle{Unrelated Title}


\begin{itemize}
\item Finden sich ind Logs und Referer-Headern wieder
\end{itemize}

\note[item]{}
\end{frame}
\begin{frame}
\frametitle{Unrelated Title}


\begin{itemize}
\item Zweck: Daten zum Server sendenDaten im Message BodyZustandsändernde Anfrage
\end{itemize}

\note[item]{}
\end{frame}
\begin{frame}
\frametitle{Unrelated Title}


\begin{itemize}
\item Wie get, nur ohne Response-Body
\end{itemize}

\note[item]{}
\end{frame}
\begin{frame}
\frametitle{Unrelated Title}


\begin{itemize}
\item Management von Ressourcen
\end{itemize}

\note[item]{}
\end{frame}
\begin{frame}
\frametitle{Unrelated Title}


\begin{itemize}
\item GETPOST
\end{itemize}

\note[item]{}
\end{frame}
\begin{frame}
\frametitle{Unrelated Title}


\begin{itemize}
\item Response CodeListe von Response-Headern /z.B. Content-Type)einer LeerzeileResponse-Body
\end{itemize}

\note[item]{}
\end{frame}
\begin{frame}
\frametitle{Unrelated Title}


\begin{itemize}
\item Response Codes können Informationen liefern: 
\item failed requests sind aussagekräftigVersion des WebserversModule
\end{itemize}

\note[item]{}
\end{frame}
\begin{frame}
\frametitle{Unrelated Title}


\begin{itemize}
\item Not Modified -> Browser uses cached version
\end{itemize}

\note[item]{}
\end{frame}
\begin{frame}
\frametitle{Unrelated Title}


\begin{itemize}
\item Bad Request
\end{itemize}

\note[item]{}
\end{frame}
\begin{frame}
\frametitle{Unrelated Title}


\begin{itemize}
\item Unauthorized
\end{itemize}

\note[item]{}
\end{frame}
\begin{frame}
\frametitle{Unrelated Title}


\begin{itemize}
\item Forbidden
\end{itemize}

\note[item]{}
\end{frame}
\begin{frame}
\frametitle{Unrelated Title}


\begin{itemize}
\item Bad Gateway: Fehler beim internen weiterleiten der Nachricht
\end{itemize}

\note[item]{}
\end{frame}
\begin{frame}
\frametitle{Unrelated Title}


\begin{itemize}
\item Name-Wert Paare als Stringsgetrennt durch carriage return CR und line feed LFheader-name: header-value \r \n
\end{itemize}

\note[item]{}
\end{frame}
\begin{frame}
\frametitle{Unrelated Title}


\begin{itemize}
\item Kombination HTTP+SSL (oder TLS)Verschlüsselung schützt vor MITM und Snooping + Schutz der BenutzerauthentifizierungKein Schutz vor Angriffen gegen die Web-Anwendung
\end{itemize}

\note[item]{}
\end{frame}
\begin{frame}
\frametitle{Unrelated Title}


\begin{itemize}
\item HTTP-Verkehr abfangen, untersuchen und weiterleitenZweck: Caching, FilternAlle Anfragen werden an den Proxy geschicktProxy ist verantwortlich für DNS Auflösung
\end{itemize}

\note[item]{}
\end{frame}
\begin{frame}
\frametitle{Unrelated Title}


\begin{itemize}
\item Erzeugen von HTML-Code programmatischCGI (Common Gateway Interface): Nimmt Daten aus dem Request entgegen und beauftragt ein Programm mit HTML Generierung -> HTML in den Standard OutputAlternative zu CGI: Webserver-Modul (Skriptsprache so direkt auf dem Server) oder spezieller Anwendungsserver
\end{itemize}

\note[item]{}
\end{frame}
\begin{frame}
\frametitle{Unrelated Title}


\begin{itemize}
\item Cross-Site Scripting (XSS)Cross-Site Request Forgery (CSRF)
\end{itemize}

\note[item]{}
\end{frame}
\begin{frame}
\frametitle{Unrelated Title}


\begin{itemize}
\item SQL InjectionRemote Code InjectionPath Traversal
\end{itemize}

\note[item]{}
\end{frame}
\begin{frame}
\frametitle{Unrelated Title}


\begin{itemize}
\item SQL InjectionXSSBuffer overflowscookie poisoninghidden field amnipulationremote file inclusion
\end{itemize}

\note[item]{}
\end{frame}
\begin{frame}
\frametitle{Unrelated Title}


\begin{itemize}
\item Z.b. nachfolgende Punkte prüfen, um schädliche Eingaben zu filtern:
\item DatentypZulässige ZeichenmengeMinimale/Maximale Längespezifische Muster
\end{itemize}

\note[item]{}
\end{frame}
\begin{frame}
\frametitle{Unrelated Title}


\begin{itemize}
\item Ziel URLsParameter (URL und Body)HeaderMetadaten übertragener Objekte
\end{itemize}

\note[item]{}
\end{frame}
\begin{frame}
\frametitle{Unrelated Title}

\begin{center}
\includegraphics[width=0.9\textwidth,height=0.9\textheight,keepaspectratio]{/Users/I516998/Library/Application Support/Anki2/User 1/collection.media/paste-de4048c13c84204281e5c94014b596370952b36e.jpg}
\end{center}


\note[item]{}
\end{frame}
\begin{frame}
\frametitle{Unrelated Title}


\begin{itemize}
\item Vollständig unter Kontrolle des AngreifersKann alle Aspekte des Web User-Interfaces lesen und ändern
\end{itemize}

\note[item]{}
\end{frame}
\begin{frame}
\frametitle{Unrelated Title}


\begin{itemize}
\item Angreifer kann solche Validierungen unschädlich machen 
\end{itemize}

\note[item]{}
\end{frame}
\begin{frame}
\frametitle{Unrelated Title}


\begin{itemize}
\item Angreifer kann solche Felder lesen und manipulieren
\end{itemize}

\note[item]{}
\end{frame}
\begin{frame}
\frametitle{Unrelated Title}


\begin{itemize}
\item Wenn die Parameter der Kontrolle des Nutzers unterliegen und nicht sanitized werden, kann der Angreifer eigene befehler an den Interpreter übergeben
\end{itemize}

\note[item]{}
\end{frame}
\begin{frame}
\frametitle{Unrelated Title}


\begin{itemize}
\item Sicherheitsprobleme treten auf, wenn Benutzereingaben in dynamischen Querys verwendet werden
\end{itemize}

\note[item]{}
\end{frame}
\begin{frame}
\frametitle{Unrelated Title}


\begin{itemize}
\item Parameter finden, welchen die Web-Anwendung für eine Datenbankabfrage nutztSchadhafte SQL-Befhle an geeigneter Stelle des Parameters anfügen
\end{itemize}

\note[item]{}
\end{frame}
\begin{frame}
\frametitle{Unrelated Title}

\begin{center}
\includegraphics[width=0.9\textwidth,height=0.9\textheight,keepaspectratio]{/Users/I516998/Library/Application Support/Anki2/User 1/collection.media/paste-4476375e80191dcd2700065014e51d4dde8c80a6.jpg}
\end{center}


\note[item]{}
\end{frame}
\begin{frame}
\frametitle{Unrelated Title}


\begin{itemize}
\item Fehlermeldungen können dem Angreifer helfenBesser: Logfiles für das Debugging
\end{itemize}

\note[item]{}
\end{frame}
\begin{frame}
\frametitle{Unrelated Title}


\begin{itemize}
\item Mittels Subquerys
\end{itemize}

\note[item]{}
\end{frame}
\begin{frame}
\frametitle{Unrelated Title}


\begin{itemize}
\item Hinzufügen boolescher Bedingungen
\end{itemize}

\note[item]{}
\end{frame}
\begin{frame}
\frametitle{Unrelated Title}


\begin{itemize}
\item Durch ORDER BY <number>Fehler bedeutet die Spalte <number> existiert nicht --> Anzahl Spalte = <number>-1
\end{itemize}

\note[item]{}
\end{frame}
\begin{frame}
\frametitle{Unrelated Title}


\begin{itemize}
\item where user_id =='1' or first_name IS NULL;#
\end{itemize}

\note[item]{}
\end{frame}
\begin{frame}
\frametitle{Unrelated Title}


\begin{itemize}
\item where user_id = ‘1 ’ UNION select user,password from mysql.users
\end{itemize}

\note[item]{}
\end{frame}
\begin{frame}
\frametitle{Unrelated Title}


\begin{itemize}
\item Mnache Datenbankfelder sind keine Strings sondern ZahlenStrings können auch mittels chars injiziert werden: insert into users values(666,char(0x63)+char(0x65)...)
\end{itemize}

\note[item]{}
\end{frame}
\begin{frame}
\frametitle{Unrelated Title}


\begin{itemize}
\item Trial and Error Ansatz, da kein Feedback von der ApplikationBsp: select title, description FROM pressReleases where id =5 AND 1=1Bei SQL-Injection Verwundbarkeit wird die Seite zurückgegeben, da 1=1 immer zu true evaluiertWenn die Eingabe validiert werden sollte id=5 AND 1=1 als ein Wert behandelt werden und die Seite sollte nicht zurück gegbeen werden
\end{itemize}

\note[item]{}
\end{frame}
\begin{frame}
\frametitle{Unrelated Title}


\begin{itemize}
\item Erstellte Query ist entweder sehr schnell oder brauch sehr langeAntwortzeit enthält Informationen über den WertSelect active_id FROM mb_active UNION SELECT IF (SUBSTRING(user_password,1,1) = CHAR(52), BENCHMARK (5000000,ENCODE('Slow Down',' by 5 seconds,null) FROM mb_users WHERE user_group=1;
\end{itemize}

\note[item]{}
\end{frame}
\begin{frame}
\frametitle{Unrelated Title}


\begin{itemize}
\item Applikation von SQL trennePersistenz Frameworks nutzenPerpared statements nutzenKeine SQL-Errors oder Fehlermeldungen an Benutzer schicken (nur Code 500)Zugriffsrechte der Anwendung auf DB beschränken
\end{itemize}

\note[item]{}
\end{frame}
\begin{frame}
\frametitle{Unrelated Title}


\begin{itemize}
\item Validiere und escape dynamische Kommanods sorgfältigSystem-User so wenie Privilegien wie möglichBesser. Keine Shell-basierte Ausführung
\end{itemize}

\note[item]{}
\end{frame}
\begin{frame}
\frametitle{Unrelated Title}

\begin{center}
\includegraphics[width=0.9\textwidth,height=0.9\textheight,keepaspectratio]{/Users/I516998/Library/Application Support/Anki2/User 1/collection.media/paste-2d02b36d7b6d0545b71ac490e80714bfb399b32a.jpg}
\end{center}

\begin{itemize}
\item Normalisiseren der Pfade (z.B. ../ entfernen)Whitelists um Zugriff einzuschränken
\end{itemize}

\note[item]{}
\end{frame}
\begin{frame}
\frametitle{Unrelated Title}


\begin{itemize}
\item Baue ich Computer Code zusammen?Falls ja: Welcher Interpreter führt ihn aus?Welche Semantik steckt hinter dem Code?Welcher dynamische Teil kann beeinflusst werden?Wie bereinige ich die Daten?
\end{itemize}

\note[item]{}
\end{frame}
\begin{frame}
\frametitle{Unrelated Title}


\begin{itemize}
\item Durch die Abfolge der HTTP-Request und ResponsesAuf Serverseite ergeben sich die erwarteten HTTP-Requests aus den Aktionen im user-Interface
\end{itemize}

\note[item]{}
\end{frame}
\begin{frame}
\frametitle{Unrelated Title}


\begin{itemize}
\item Angreifer kann HTTP-Requests bis aufs letzte Bit kontrollierenDaher kann nicht auf das durch die UI vorgegebene Schema vertraut werden
\end{itemize}

\note[item]{}
\end{frame}
\begin{frame}
\frametitle{Unrelated Title}


\begin{itemize}
\item Sicherheitseigenschaften werden direkt von HTTP Parametern (oder anderen Teilen) des eingehenden Requests abgeleitetBsp: example.org?isadmin=false --> example.org?isadmin=true
\end{itemize}

\note[item]{}
\end{frame}
\begin{frame}
\frametitle{Unrelated Title}


\begin{itemize}
\item Die Annahme, dass der Benutzer/Angreifer nur auf diejenigen URLs zugreift, die im UI auftauchen.
\end{itemize}

\note[item]{}
\end{frame}
\begin{frame}
\frametitle{Unrelated Title}


\begin{itemize}
\item Verwenden eines beliebigen HTTP-VerbsIm schlechten Fall ist die Default-Authorization ALLOW oder Berechtigungsprüfungen gelten nur für bestimmt HTTP-Verben
\end{itemize}

\note[item]{}
\end{frame}
\begin{frame}
\frametitle{Unrelated Title}


\begin{itemize}
\item Integrität von Workflows darf nicht verletzte werdenschwierig umsetzbar bei mehreren parteien
\end{itemize}

\note[item]{}
\end{frame}
\begin{frame}
\frametitle{Unrelated Title}

\begin{center}
\includegraphics[width=0.9\textwidth,height=0.9\textheight,keepaspectratio]{/Users/I516998/Library/Application Support/Anki2/User 1/collection.media/paste-3df47cfc78cbcd723c7082dbfb5c9e7ee914bbd3.jpg}
\includegraphics[width=0.9\textwidth,height=0.9\textheight,keepaspectratio]{/Users/I516998/Library/Application Support/Anki2/User 1/collection.media/paste-30f75f1f804afada4c8a590a2212bd31f41bec28.jpg}
\includegraphics[width=0.9\textwidth,height=0.9\textheight,keepaspectratio]{/Users/I516998/Library/Application Support/Anki2/User 1/collection.media/paste-ad94820a30a4284ee0f27f2a162199b0829cb7dc.jpg}
\end{center}

\begin{itemize}
\item Schutzmaßnahme:
\item Weiterer Angriff über Fake Shop:
\end{itemize}

\note[item]{}
\end{frame}
\begin{frame}
\frametitle{Unrelated Title}


\begin{itemize}
\item Jeder HTTP-Request ist ein ThreadThreads teilen sich Ressourcen (Filesystem, Datenbank)Problem durch gleichzeitiges eintreffen von Requests (bei z.B. Geld abheben)Problem entsteht, da viele Web-Appllikation-Frameworks keine atomarität bietenLösung: Sorgfältige bedrohungsmodellierung
\end{itemize}

\note[item]{}
\end{frame}
\begin{frame}
\frametitle{Unrelated Title}


\begin{itemize}
\item Dateien sind vom Angreifer kontrolliert (Name und Inhalt)Nach dem Upload ist die Datei auf dem Server und innerhalb der SicherheitsgrenzenDateien müssen nicht das enthalten, was sie vorgeben (z.B. Backdoored Image Files: die XSS-Code enthalten; oder Mediendateien mit JavaScript)
\end{itemize}

\note[item]{}
\end{frame}
\begin{frame}
\frametitle{Unrelated Title}


\begin{itemize}
\item .php oder andere vom Webserver ausgeführte Dateien.htaccess Änderung der Konfiguration des Webservers (z.B. hinzufügen neuere Executables)corssdomain.xml CSRF-Schutz ausebeln.jar Kompromittierung des Rechners oder XSS durch eingeschleuste Applets
\end{itemize}

\note[item]{}
\end{frame}
\begin{frame}
\frametitle{Unrelated Title}


\begin{itemize}
\item Alles prüfen und nur akzeptieren, was erwartet wird (kein Verlass auf Dateiendungen)In DB als BLOB speichern oder selber Dateinamen wählenDateigröße beschränkenBilddateien umwandeln in neues Format
\end{itemize}

\note[item]{}
\end{frame}
\begin{frame}
\frametitle{Unrelated Title}


\begin{itemize}
\item Inline ScriptsExternal ScriptEventhandler
\end{itemize}

\note[item]{}
\end{frame}
\begin{frame}
\frametitle{Unrelated Title}


\begin{itemize}
\item Auf Inhalte darf nur von Objekten zugegriffen werden, die den gleichen Ursprung haben
\end{itemize}

\note[item]{}
\end{frame}
\begin{frame}
\frametitle{Unrelated Title}


\begin{itemize}
\item Protocol+Domain+Port
\end{itemize}

\note[item]{}
\end{frame}
\begin{frame}
\frametitle{Unrelated Title}


\begin{itemize}
\item Auf Dokumentenebene: Allle Elemente die Teil eines Web-Dokuments sind, erheben die Herkunft des sie umschließenden Dokuments
\end{itemize}

\note[item]{}
\end{frame}
\begin{frame}
\frametitle{Unrelated Title}


\begin{itemize}
\item Wert für document.domain Property setzen
\end{itemize}

\note[item]{}
\end{frame}
\begin{frame}
\frametitle{Unrelated Title}


\begin{itemize}
\item XSS ist nichts andere als HTML-Code-Injection
\end{itemize}

\note[item]{}
\end{frame}
\begin{frame}
\frametitle{Unrelated Title}


\begin{itemize}
\item Web-Applikation verarbeitet via Request erhaltene DatenDaten werden auf der Website oft dargestelltIn den Daten enthaltener Code wird von der angegriffenen Web-Applikation ausgeführt
\end{itemize}

\note[item]{}
\end{frame}
\begin{frame}
\frametitle{Unrelated Title}


\begin{itemize}
\item Web-Inhalte ändernprivate Informationen/Passwörter stehlenSessions übernehmenBrowser Übernehmen
\end{itemize}

\note[item]{}
\end{frame}
\begin{frame}
\frametitle{Unrelated Title}

\begin{center}
\includegraphics[width=0.9\textwidth,height=0.9\textheight,keepaspectratio]{/Users/I516998/Library/Application Support/Anki2/User 1/collection.media/paste-af128e8904661216795713ef87a8daf9183633b2.jpg}
\includegraphics[width=0.9\textwidth,height=0.9\textheight,keepaspectratio]{/Users/I516998/Library/Application Support/Anki2/User 1/collection.media/paste-d8238e25af2316fe61da69fc228105c6a27c3e05.jpg}
\end{center}


\note[item]{}
\end{frame}
\begin{frame}
\frametitle{Unrelated Title}


\begin{itemize}
\item URL umschreiben: Id ist in jeder URL enthalten (<a href="some page?SID =g2k42a">...</a>)Form-basierte Session-IDs: SID wird in verstecktem Form-Field gespeichert, entsprechendes Form wird immer mit übertragenCookies: Speichern der SID im Browser und bei Bedarf mitsenden
\end{itemize}

\note[item]{}
\end{frame}
\begin{frame}
\frametitle{Unrelated Title}


\begin{itemize}
\item SID-Leakage durch z.B. Referrer oder Proxylogs
\end{itemize}

\note[item]{}
\end{frame}
\begin{frame}
\frametitle{Unrelated Title}


\begin{itemize}
\item Der Back-Button ("Confirm to re-submit form contents")
\end{itemize}

\note[item]{}
\end{frame}
\begin{frame}
\frametitle{Unrelated Title}


\begin{itemize}
\item Bösartiges JavaScript liest SID aus und teilt SID dem Angreifer mitHTTP-Only Header im Response-Header können dies verhindern
\end{itemize}

\note[item]{}
\end{frame}
\begin{frame}
\frametitle{Unrelated Title}


\begin{itemize}
\item Angriff findet im Browser stattIframe-Inclusion oder XMLHttpRequest-ObjektAngreifer handelt so als sei er das Opfer
\end{itemize}

\note[item]{}
\end{frame}
\begin{frame}
\frametitle{Unrelated Title}


\begin{itemize}
\item Kidnappen des echten Login-Dialogs auf der originalen Seite: Redirect über vergiftete URL (login.php?lang =en<script>...<script>)
\item Ersetzen bzw. ergänzen des ursprünglichen HTML: Imitieren des echten Login-Dialog
\end{itemize}

\note[item]{}
\end{frame}
\begin{frame}
\frametitle{Unrelated Title}


\begin{itemize}
\item Programmbaisert:
\item ReflektiertGespeichertDOM-basiertInfrastruktur-basiert:
\item vom Browser verursachtvom Server verursachtNetzwerk basiert
\end{itemize}

\note[item]{}
\end{frame}
\begin{frame}
\frametitle{Unrelated Title}

\begin{center}
\includegraphics[width=0.9\textwidth,height=0.9\textheight,keepaspectratio]{/Users/I516998/Library/Application Support/Anki2/User 1/collection.media/paste-f1661040943363dca68a186dcc7406fd4374b262.jpg}
\end{center}

\begin{itemize}
\item Eine Webapplikation gibt Nutzereingaben blind/ungeprüft wieder
\end{itemize}

\note[item]{}
\end{frame}
\begin{frame}
\frametitle{Unrelated Title}

\begin{center}
\includegraphics[width=0.9\textwidth,height=0.9\textheight,keepaspectratio]{/Users/I516998/Library/Application Support/Anki2/User 1/collection.media/paste-6a728acd48bda34f9e351bf328b826ceb8f85657.jpg}
\end{center}

\begin{itemize}
\item Web-Applikation speichert Benutzereingaben ohne ausreichende Überprüfung und gibt diese später wieder aus
\end{itemize}

\note[item]{}
\end{frame}
\begin{frame}
\frametitle{Unrelated Title}


\begin{itemize}
\item Web-Applikation enthält clientseitiges JavaScirpt, welches Benutzereingaben unsicher verarbeitet
\item Clientseitige HTML Erzeugung (z.B. .innerHTML oder document.write)Nutzung von Angreifer kontrollierten Werten  (z.B. document.URL)
\end{itemize}

\note[item]{}
\end{frame}
\begin{frame}
\frametitle{Unrelated Title}


\begin{itemize}
\item Der Angreifer muss das JS in die Seite einfügenUND das Opfer muss die vergiftete Seite besuchen
\end{itemize}

\note[item]{}
\end{frame}
\begin{frame}
\frametitle{Unrelated Title}


\begin{itemize}
\item Payload wird dauerhaft Teil der Anwendung -> Angreifer muss nach dem Injizieren nur abwarten
\end{itemize}

\note[item]{}
\end{frame}
\begin{frame}
\frametitle{Unrelated Title}


\begin{itemize}
\item Payload existiert nur in einer einzelnen HTTP-ResponseMethode A: Schicke dem Opfer einen vergifteten LinkMethode B: Verstecke einen IFrame in einer harmlos aussehenden Seite
\end{itemize}

\note[item]{}
\end{frame}
\begin{frame}
\frametitle{Unrelated Title}


\begin{itemize}
\item Der Schadcode läuft im Browser eines betroffenen ClientsAuf dem Server können keine direkten Anzeichen beobachtet werden (wie z.B. beim Angriff auf den Server, welcher Spuren in Logdateien hinterlässt)
\end{itemize}

\note[item]{}
\end{frame}
\begin{frame}
\frametitle{Unrelated Title}


\begin{itemize}
\item Vom Nutzer übergebenen Daten nicht trauenValidieren der Eingaben (erste Verteidigungslinie)Output encoding: Skript-Code vollständig aus allen Website Abschnitten entfernen, die keinen enthalten sollen
\end{itemize}

\note[item]{}
\end{frame}
\begin{frame}
\frametitle{Unrelated Title}


\begin{itemize}
\item Vom Benutzer vorgegebenes HTML darf nicht erlaubt seinNie einzelne < oder > in "user prvided data" zulassenNiemals Fehlermeldungen an den Client durchreichenThirdparty Addons können eine Gefahr seinKeine eigenen Filter verwenden, prüfen ob Framework dieses bietet
\end{itemize}

\note[item]{}
\end{frame}
\begin{frame}
\frametitle{Unrelated Title}


\begin{itemize}
\item Das einzige definsive Hilfsmittel gegen XSS, da das Problem in die Subdomäne eingesperrt wirdEinsatz besonders nützlich gegen Infrastruktur XSS Probleme
\end{itemize}

\note[item]{}
\end{frame}
\begin{frame}
\frametitle{Unrelated Title}


\begin{itemize}
\item Domain relaxation: Sollte nur in eine Richtung möglich sein, damit XSS die gesetzten Grenzen nicht verlassen kannCookies: Cookies können sich auch auf Parent-Domains beziehen --> Session-Fixation: SID der Subdomain wird für einen Login auf der Hauptseite (parent domain) genutzt
\end{itemize}

\note[item]{}
\end{frame}
\begin{frame}
\frametitle{Unrelated Title}


\begin{itemize}
\item Software mit bösartiger Wirkung
\end{itemize}

\note[item]{}
\end{frame}
\begin{frame}
\frametitle{Unrelated Title}


\begin{itemize}
\item Replikation: Malware versucht sich aktiv zu verbreitenPopulationswachstum: Beschreibung der Veränderung der Anzahl der Malware-Instanzen aufgrund von ReplikationenParasitismus: Die Malware benötigt einen Wirt zum existieren
\end{itemize}

\note[item]{}
\end{frame}
\begin{frame}
\frametitle{Unrelated Title}


\begin{itemize}
\item Bestehend aus Nutzlast (böse Aktivität) +Auslöser (boolesche Bedingung)Klein und schwer auffindbarReplikation: NeinPopulationswachstum: KeinesParasitismus: Möglich
\end{itemize}

\note[item]{}
\end{frame}
\begin{frame}
\frametitle{Unrelated Title}


\begin{itemize}
\item Programm, dass eine gewünschte Funktonalität besitztProgramm, dass eine versteckte, nicht gewünschte Funktion besitztReplikation: NeinPopulationswachstum: KeinesParasitismus: Ja
\end{itemize}

\note[item]{}
\end{frame}
\begin{frame}
\frametitle{Unrelated Title}


\begin{itemize}
\item Speziel dafür vorgesehene Malware um Sicherheitsmaßnahmen zu umgehenIn Programm eingefügt oder eigenständige ExistenzReplikation: NeinPopulationswachstum: KeinesParasitismus: Möglich
\end{itemize}

\note[item]{}
\end{frame}
\begin{frame}
\frametitle{Unrelated Title}


\begin{itemize}
\item Verbreitung bei der Ausführung, indem sie sich über ein Netzwerk in Wirtsystemen einnistetEigenständige existenz im WirtssystemReplikation: JaPopulationswachstum: PositivPArasitismus: Nein
\end{itemize}

\note[item]{}
\end{frame}
\begin{frame}
\frametitle{Unrelated Title}


\begin{itemize}
\item Verbreitung während der Ausführung durch selbstständiges Einnisten im WirtInitiale Infektion durch z.B. trojanisches PferdReplikation: JaPopulationswachstum: PositivParasitismus: Ja
\end{itemize}

\note[item]{}
\end{frame}
\begin{frame}
\frametitle{Unrelated Title}


\begin{itemize}
\item Schnelle Replikation, um Ressourcen des Wtsystems zu nutzenReplikation: JaPopulationswachstum: Positivparasitismus: Nein
\end{itemize}

\note[item]{}
\end{frame}
\begin{frame}
\frametitle{Unrelated Title}


\begin{itemize}
\item Versenden vertraulicher Daten vom Wirt an einen verborgenen EmpfängerReplikation: NeinPopulationswachstum: KeinesParasitismus: Nein
\end{itemize}

\note[item]{}
\end{frame}
\begin{frame}
\frametitle{Unrelated Title}


\begin{itemize}
\item Marketing-Spyware zum Anzeigen von Werbung und Erstellen von KonsumerprofilenReplikation: NeinPopulationswachstum: KeinesParasitismus: Nein
\end{itemize}

\note[item]{}
\end{frame}
\begin{frame}
\frametitle{Unrelated Title}


\begin{itemize}
\item Angreifer steht bösartige Funktionalität über das Netzwerk fernsteuerbar zur VerfügungVerbund von Zombies = BotnetzReplikation: NeinPopulationswachstum: NeinParasitismus: Nein
\end{itemize}

\note[item]{}
\end{frame}
\begin{frame}
\frametitle{Unrelated Title}


\begin{itemize}
\item Virus kopiert originalen Bootblock und überschreibt den BootblockKopie des Originals an fester Stellle speichern, da zum Hochfahren benötigtVirus erhält Kontrolle vor Betriebssystem und Anti-Viren-SW
\end{itemize}

\note[item]{}
\end{frame}
\begin{frame}
\frametitle{Unrelated Title}


\begin{itemize}
\item Blockschutz im BIOS
\end{itemize}

\note[item]{}
\end{frame}
\begin{frame}
\frametitle{Unrelated Title}

\begin{center}
\includegraphics[width=0.9\textwidth,height=0.9\textheight,keepaspectratio]{/Users/I516998/Library/Application Support/Anki2/User 1/collection.media/paste-0822d968667a6d18142619cc9372497b8610ea99.jpg}
\end{center}

\begin{itemize}
\item Kopie des originalen Bootblocks wird überschrieben
\end{itemize}

\note[item]{}
\end{frame}
\begin{frame}
\frametitle{Unrelated Title}

\begin{center}
\includegraphics[width=0.9\textwidth,height=0.9\textheight,keepaspectratio]{/Users/I516998/Library/Application Support/Anki2/User 1/collection.media/paste-d228eaf95b182e0099bdbdf85d204be389214477.jpg}
\end{center}

\begin{itemize}
\item Überschreibender Virus meist sehr kruz und ohne Nutzlast
\item Dateianfang überschreiben: Dateigröße unverändert, Wirtsprogramm zerstörtIn der Mitte überschreiben: Dateigröße unverändert, Wirtsprogramm dennoch defektVollständig ersetzen: Leicht entdeckbar aufgrund identischer Dateigrößen
\end{itemize}

\note[item]{}
\end{frame}
\begin{frame}
\frametitle{Unrelated Title}


\begin{itemize}
\item Bei Batch/Shell Dateien parktikabelBei Binärdateien meist zu aufwendig (Relokation)Veränderte Dateigröße
\end{itemize}

\note[item]{}
\end{frame}
\begin{frame}
\frametitle{Unrelated Title}

\begin{center}
\includegraphics[width=0.9\textwidth,height=0.9\textheight,keepaspectratio]{/Users/I516998/Library/Application Support/Anki2/User 1/collection.media/paste-9738808864fdb47af372daf21fe2db70cf470794.jpg}
\end{center}

\begin{itemize}
\item Bei BinärdatenVeränderte Dateigröße
\end{itemize}

\note[item]{}
\end{frame}
\begin{frame}
\frametitle{Unrelated Title}

\begin{center}
\includegraphics[width=0.9\textwidth,height=0.9\textheight,keepaspectratio]{/Users/I516998/Library/Application Support/Anki2/User 1/collection.media/paste-9001d429e116913ec79d799fd1f9aed4aabc23a8.jpg}
\end{center}


\note[item]{}
\end{frame}
\begin{frame}
\frametitle{Unrelated Title}

\begin{center}
\includegraphics[width=0.9\textwidth,height=0.9\textheight,keepaspectratio]{/Users/I516998/Library/Application Support/Anki2/User 1/collection.media/paste-07da002b826142e877c59743ada7d202bd7f000c.jpg}
\end{center}

\begin{itemize}
\item Virus trägt selben Namen wie der ursprüngliche WirtVirus wird vor dem Wirtsprogramm aufgerufen (z.B. Wirt umbenannt, Overlay-Icon, Registry-Maniüulation)Virus ruft Wirtsprogramm auf
\end{itemize}

\note[item]{}
\end{frame}
\begin{frame}
\frametitle{Unrelated Title}

\begin{center}
\includegraphics[width=0.9\textwidth,height=0.9\textheight,keepaspectratio]{/Users/I516998/Library/Application Support/Anki2/User 1/collection.media/paste-69145a49a0d2c99a7aa065414c935079157aacb6.jpg}
\end{center}

\begin{itemize}
\item Virus infiziert Quellcode --> Infektionsstelle nicht offensichtlichWirtsprogramm muss erst kompiliert werden, bevor der Virus wirksam wird --> Harware-Plattform Abhängigkeit
\end{itemize}

\note[item]{}
\end{frame}
\begin{frame}
\frametitle{Unrelated Title}


\begin{itemize}
\item Virus infiziert nicht-ausführbare DokumenteDokumentformat umfasst interpretierbare ProgrammierspracheVirus infiziert Makros, welche von APpliukationen für das Dokumentenformat interpretiert werden
\end{itemize}

\note[item]{}
\end{frame}
\begin{frame}
\frametitle{Unrelated Title}


\begin{itemize}
\item Nutzer: Bemerkt Symptome der InfektionAnti-Viren Programme: Erkennen Eigenschaften des Viruscodes
\end{itemize}

\note[item]{}
\end{frame}
\begin{frame}
\frametitle{Unrelated Title}


\begin{itemize}
\item Statisch: Datemerkmale anpassen wie Größe, Checksumme,...Dynamisch: Unsprüngliche Datei/Bootblockmerkmale zur Laufzeit im Speicher dauerhaft vorhalten und wenn nötig einspielen
\end{itemize}

\note[item]{}
\end{frame}
\begin{frame}
\frametitle{Unrelated Title}


\begin{itemize}
\item Verschlüsselter Virus-BodyStets zufällige Anpassung des neu DekryptorsSelbsterkennung zur Verhinderung von Überinfektionen nötig (muss unabhängig vim Viruscode funktionieren)
\end{itemize}

\note[item]{}
\end{frame}
\begin{frame}
\frametitle{Unrelated Title}


\begin{itemize}
\item Verschlüsselter Teil stets verschiedenDekryptor statisch und daher erkennbarVerschlüsselten Pool statischer Dekryptoren nutzen und bei Weitervrebreitung zufällig auswählen
\end{itemize}

\note[item]{}
\end{frame}
\begin{frame}
\frametitle{Unrelated Title}


\begin{itemize}
\item Äquivalente BefehleÄquivalente BefehlssequenzenBefehlsumordnungUmbenennungDatenumordnung (Umordnen des Speicherorts hat Auswirkungen auf Zeiger in Befehlsoperanden)Spaghettifizierung (Umordnung von Befehlen unter Beibehaltung der Ausführungsreihenfolge)Unnötige BefehleCoder Erzeugung zur LaufzeitNebenläfuigkeite in synchronisierten ThreadsInlining (Einfügen von Subroutinencode)Outlining (Auslagern von Code in Subroutinen)Code-Interpretation (Einfügen von virtuellem Maschinencode)Threaded Outlining (Auslagern von Code in Subroutinenen mit Direktsprüngen)Interleaving (Verschränktes Zusammenführen von Subroutinen)
\end{itemize}

\note[item]{}
\end{frame}
\begin{frame}
\frametitle{Unrelated Title}


\begin{itemize}
\item Erschweren bzw. Verhindern von Reverse EngineeringSuperoptimierung durch Compiler
\end{itemize}

\note[item]{}
\end{frame}
\begin{frame}
\frametitle{Unrelated Title}


\begin{itemize}
\item Identifikation: Durch Erkennung oder seperates VerfahrenDesinfektion: Entfernen des identifizierten Viruses
\end{itemize}

\note[item]{}
\end{frame}
\begin{frame}
\frametitle{Unrelated Title}


\begin{itemize}
\item Durchsuchung von Datenströmen nach bekannten Signaturen (z.B. Strings die Wildcards enthalten)
\end{itemize}

\note[item]{}
\end{frame}
\begin{frame}
\frametitle{Unrelated Title}


\begin{itemize}
\item Auf Anfrage: Nutzer startet SuchvorgangBei Zugriff: Scanner läuft im Hintergrund und durchsucht Daten beim Zugriff
\end{itemize}

\note[item]{}
\end{frame}
\begin{frame}
\frametitle{Unrelated Title}


\begin{itemize}
\item Suche nach virusähnlichem Code, nicht nach spezifischen Byte-MusternDatensammlung um gewichtete Anzeichen zu finden (z.B. Junk Code, Dechiffrier-Schleifen)Komplexe Heuristiken (z.B. Differenz zwischen Einsprung und Dateiende)Vorteil: Erkennung bekannter und unbekannter VirenNachteil: viele fals psoitives und keine Identifikation erkannter Viren --> keine Desinfektion
\end{itemize}

\note[item]{}
\end{frame}
\begin{frame}
\frametitle{Unrelated Title}


\begin{itemize}
\item Unautorisierte Dateiänderungen geben Hinweis auf ein Virus --> Datenbasis auf virusfreiem System zum Vergleich nötig
\end{itemize}

\note[item]{}
\end{frame}
\begin{frame}
\frametitle{Unrelated Title}


\begin{itemize}
\item Periodische PrüfungSelbstprüfung vor ProgrammstartPrüfung vor Ausführung
\end{itemize}

\note[item]{}
\end{frame}
\begin{frame}
\frametitle{Unrelated Title}


\begin{itemize}
\item Vorteil: hohe PerformanceNachteil: aufwendige Wartung, false positives, keine Identifikation und damit keine Desinfektion
\end{itemize}

\note[item]{}
\end{frame}
\begin{frame}
\frametitle{Unrelated Title}


\begin{itemize}
\item Beobachtung des Verhalten ausgeführter Programme im Hinblick auf virusähnliches VerhaltenMonitor kann Aktivitäten blockieren
\end{itemize}

\note[item]{}
\end{frame}
\begin{frame}
\frametitle{Unrelated Title}


\begin{itemize}
\item Spezifizieren unerlaubter VirusaktivitätMonitor abstrahiert von den einzelnen MaschinenbefehlenGrobkörnige Modellierung: Ignorieren von Parametern, Aktivitätsverfolgung in Echtzeit --> falscher Alarm möglichFeinkörnige Modellierung: Modellierung mit Parametern, Vergleich mit späteren Aktivitäten möglichVorteil: Präzise Identifikation des VirusesNachteil: Gerine Erkennungsrate bei unbekannten Viren
\end{itemize}

\note[item]{}
\end{frame}
\begin{frame}
\frametitle{Unrelated Title}


\begin{itemize}
\item Spezifizieren erlaubter Aktivitäten und abweichende Aktivität=Anomalie=VirusVorteil: Erkennung bekannter und unbekannter VirenNachteil: Einstellen des Verfahrens für jedes Programm nötig, keine Identifikaiton erkannter Viren
\end{itemize}

\note[item]{}
\end{frame}
\begin{frame}
\frametitle{Unrelated Title}


\begin{itemize}
\item Virus entschlüsselt sich zur LaufzeitNach Entschlüsselung statisch scannenBeobachtungsdaten zur Laufzeit zur Missbrauchs- oder AnomalieerkennungVorteil: Virus befällt nur emulierte Umgebung, Erkennung bekannter und unbekannter VirenNacteil: hoher Ressourcenaufwand, Virus könnte Emulation erkennen und inaktiv bleiben
\end{itemize}

\note[item]{}
\end{frame}
\begin{frame}
\frametitle{Unrelated Title}


\begin{itemize}
\item CPU EmulationSpeicher EmulationBetriebssystem EmulationHardware Emulation
\end{itemize}

\note[item]{}
\end{frame}
\begin{frame}
\frametitle{Unrelated Title}


\begin{itemize}
\item Funktionsfähigkeit der Anti-Viren-SW unterbindenVirenanalyse erschwerenEntdeckung vermeiden, indem Wissen über die FUnktionsweise der Anti-Viren-SW ausgenutzt wird
\end{itemize}

\note[item]{}
\end{frame}
\begin{frame}
\frametitle{Unrelated Title}


\begin{itemize}
\item Aktives abschalten/Behindern von Anti-Viren Prozessen:
\item Beenden der Prozesse (Liste bekannter Prozessnamen)Dauerhaftes abschalten durch Modifikation im SekundärspeicherLokale IP-Adressübersetzung manipulieren, um Aktualisierungen zu verhindern
\end{itemize}

\note[item]{}
\end{frame}
\begin{frame}
\frametitle{Unrelated Title}


\begin{itemize}
\item Aufruf der ExitProcesss() API-Funktion ersetzen (Virusausführung nach Programmende)Instruktionssequenz ersetzen (Virus führt am Ende ersetzte Codesequenz aus und springt zum Wirtscode zurück)
\end{itemize}

\note[item]{}
\end{frame}
\begin{frame}
\frametitle{Unrelated Title}


\begin{itemize}
\item AussitzenDynamische Heuristiken Vermeiden (Einsprung in Virus erst tief im Wirtscode, Verteilen des Dekryptors, Mehrfaches Entschlüsseln des Viruses)Grenzen testen (Emulator erkennen und ggf. zum Absturz bringen aufgrund limitierter Emulation)
\end{itemize}

\note[item]{}
\end{frame}
\begin{frame}
\frametitle{Unrelated Title}


\begin{itemize}
\item Verzögern des Reverse Engineering, damit Methoden zur Desinfektion möglichst spät entwickelt werden --> weiteres Populationswachstum
\end{itemize}

\note[item]{}
\end{frame}
\begin{frame}
\frametitle{Unrelated Title}


\begin{itemize}
\item Abweichende Laufzeitumgebung erkennen (wie Emulator)Windos: API-Funktion IsDebuggerPresent() afragenBreakpoint-Instruktionen erkennenEinzelschritt-Modus erkennen (CPU löst Interrupt aus--> Ausführungszeit messen oder Interrupt-Adresse prüfen)
\end{itemize}

\note[item]{}
\end{frame}
\begin{frame}
\frametitle{Unrelated Title}


\begin{itemize}
\item Interrupt legt Daten auf den StackIee. Wert auf Stack legen und regelmäßig prüfen
\end{itemize}

\note[item]{}
\end{frame}
\begin{frame}
\frametitle{Unrelated Title}


\begin{itemize}
\item Daten und Code mischen, um präzise Separation nicht zu ermöglichen
\end{itemize}

\note[item]{}
\end{frame}
\begin{frame}
\frametitle{Unrelated Title}


\begin{itemize}
\item Verwundbarkeit ausnutzen (Exploit)Überred / Social Engineering (z.B. Email-Anhang öffnen)Vertrauensbeziehungen ausnutzenDaten im Transit manipulieren/infizieren
\end{itemize}

\note[item]{}
\end{frame}
\begin{frame}
\frametitle{Unrelated Title}


\begin{itemize}
\item Aktuelle UpdatesMaßnahmen gegen ExploitsVerkehrsnormalisierung zur Unterstützung von Netz-IDSUngenutzte Dienste deaktivierenAusgangsverkehr restriktiv Filtern
\end{itemize}

\note[item]{}
\end{frame}
\begin{frame}
\frametitle{Unrelated Title}


\begin{itemize}
\item Honeypots (Würmer einfangen)Aumerksame AdministratorenVerbindungsrate bei Ziel-Auffächerung drosselnIntrusion-Detection
\end{itemize}

\note[item]{}
\end{frame}
\begin{frame}
\frametitle{Unrelated Title}


\begin{itemize}
\item Wieviel Speicher muss reserviert werden?Wie wird mit lokalen Variablen umgegangen?Wohin kehrt die Programmausführung nach der Ausführung des Unterprogramms zurück?
\end{itemize}

\note[item]{}
\end{frame}
\begin{frame}
\frametitle{Unrelated Title}

\begin{center}
\includegraphics[width=0.9\textwidth,height=0.9\textheight,keepaspectratio]{/Users/I516998/Library/Application Support/Anki2/User 1/collection.media/paste-3002e71cce9adcbd236528d0f7fe60b614076958.jpg}
\end{center}


\note[item]{}
\end{frame}
\begin{frame}
\frametitle{Unrelated Title}


\begin{itemize}
\item Erzeuge “ Shellcode ” Binärcode, der die gewünschte Aktion (Angriff) ausführtSpeichere ihn irgendwo im Speicher (hier im Stack)Sorge dafür, dass der IP auf die entsprechende Speicherstelle zeigt.
\end{itemize}

\note[item]{}
\end{frame}
\begin{frame}
\frametitle{Unrelated Title}

\begin{center}
\includegraphics[width=0.9\textwidth,height=0.9\textheight,keepaspectratio]{/Users/I516998/Library/Application Support/Anki2/User 1/collection.media/paste-7c28391003e37833aa1e472d50d044dede015ece.jpg}
\end{center}

\begin{itemize}
\item Errate die Speicheradresse des verwundbaren Puffers auf dem Stack möglichst genau.Fülle den Shellcode auf (Padding)Vor dem Shellcode : So viele NOP Instruktionen wie möglich („NOP slideNach dem Shellcode : Die geschätzte Adresse des Anfangs des ShellcodesDies sollte die ursprüngliche Rücksprungadresse überschreibenDer zusammengesetzte Puffer muss groß genug sein, um die Rücksprungadresse auf dem Stack zu
\item überschreiben
\end{itemize}

\note[item]{}
\end{frame}
\begin{frame}
\frametitle{Unrelated Title}


\begin{itemize}
\item Prävention, Detektion, Reaktion/Wiederherstellung
\end{itemize}

\note[item]{}
\end{frame}
\begin{frame}
\frametitle{Unrelated Title}


\begin{itemize}
\item OffenlegungTäuschungStörungÜbernahme
\end{itemize}

\note[item]{}
\end{frame}
\begin{frame}
\frametitle{Unrelated Title}


\begin{itemize}
\item mögliche Verletzung eines Sicherheitsziels
\end{itemize}

\note[item]{}
\end{frame}
\begin{frame}
\frametitle{Unrelated Title}


\begin{itemize}
\item Schutz gegen vorsätzliche Bedrohungen
\end{itemize}

\note[item]{}
\end{frame}
\begin{frame}
\frametitle{Unrelated Title}


\begin{itemize}
\item Schutz gegen zufällige schadhafte Ereignisse
\end{itemize}

\note[item]{}
\end{frame}
\begin{frame}
\frametitle{Unrelated Title}


\begin{itemize}
\item Spezialfall des Netzwerkangreifers zur Bewertung kryptografischer ProtokolleKann ale Nachrichten abhören und erzeugen, ist nur durch kryptografische Methoden eingeschränkt
\end{itemize}

\note[item]{}
\end{frame}
\begin{frame}
\frametitle{Unrelated Title}


\begin{itemize}
\item Policy/Regel = Aussage was erlaubt istMachanismus = Methode zur Durchsetzung einer Policy
\end{itemize}

\note[item]{}
\end{frame}
\begin{frame}
\frametitle{Unrelated Title}


\begin{itemize}
\item Network Intrusion Detection
\end{itemize}

\note[item]{}
\end{frame}
\begin{frame}
\frametitle{Unrelated Title}


\begin{itemize}
\item Verknüpfung einer Identität mit einem Individuum
\end{itemize}

\note[item]{}
\end{frame}
\begin{frame}
\frametitle{Unrelated Title}


\begin{itemize}
\item Geheimes WissenBesitzIndividuelle Eigenschaften (biometrische Verfahren)
\end{itemize}

\note[item]{}
\end{frame}
\begin{frame}
\frametitle{Unrelated Title}

\begin{center}
\includegraphics[width=0.9\textwidth,height=0.9\textheight,keepaspectratio]{/Users/I516998/Library/Application Support/Anki2/User 1/collection.media/paste-ce4d50d72abb9c6b532bf94e9ae55b5affa60c8b.jpg}
\end{center}


\note[item]{}
\end{frame}
\begin{frame}
\frametitle{Unrelated Title}

\begin{center}
\includegraphics[width=0.9\textwidth,height=0.9\textheight,keepaspectratio]{/Users/I516998/Library/Application Support/Anki2/User 1/collection.media/paste-33e280c3a51356d04f330c9514ff0f9d988b5b12.jpg}
\end{center}


\note[item]{}
\end{frame}
\begin{frame}
\frametitle{Unrelated Title}

\begin{center}
\includegraphics[width=0.9\textwidth,height=0.9\textheight,keepaspectratio]{/Users/I516998/Library/Application Support/Anki2/User 1/collection.media/paste-afc3ae5058b21d44d8e955f04f3227a532fd7d38.jpg}
\end{center}


\note[item]{}
\end{frame}
\begin{frame}
\frametitle{Unrelated Title}

\begin{center}
\includegraphics[width=0.9\textwidth,height=0.9\textheight,keepaspectratio]{/Users/I516998/Library/Application Support/Anki2/User 1/collection.media/paste-9d34063b0a133a9df1e4fb8cc9cb33ff5dfef58b.jpg}
\end{center}


\note[item]{}
\end{frame}
\begin{frame}
\frametitle{Unrelated Title}

\begin{center}
\includegraphics[width=0.9\textwidth,height=0.9\textheight,keepaspectratio]{/Users/I516998/Library/Application Support/Anki2/User 1/collection.media/paste-2b617df9cb5a9185cbef5148f1fc14d6fda860af.jpg}
\end{center}


\note[item]{}
\end{frame}
\begin{frame}
\frametitle{Unrelated Title}

\begin{center}
\includegraphics[width=0.9\textwidth,height=0.9\textheight,keepaspectratio]{/Users/I516998/Library/Application Support/Anki2/User 1/collection.media/paste-dc94c54c26fb96da54cbaf68f851c4607b97c038.jpg}
\end{center}


\note[item]{}
\end{frame}
\begin{frame}
\frametitle{Unrelated Title}

\begin{center}
\includegraphics[width=0.9\textwidth,height=0.9\textheight,keepaspectratio]{/Users/I516998/Library/Application Support/Anki2/User 1/collection.media/paste-06d398edcf85aba7010f12d309317af380a99a11.jpg}
\end{center}


\note[item]{}
\end{frame}
\begin{frame}
\frametitle{Unrelated Title}

\begin{center}
\includegraphics[width=0.9\textwidth,height=0.9\textheight,keepaspectratio]{/Users/I516998/Library/Application Support/Anki2/User 1/collection.media/paste-b8411b2abe7f23769c93795925f0c434d293841e.jpg}
\end{center}


\note[item]{}
\end{frame}
\begin{frame}
\frametitle{Unrelated Title}

\begin{center}
\includegraphics[width=0.9\textwidth,height=0.9\textheight,keepaspectratio]{/Users/I516998/Library/Application Support/Anki2/User 1/collection.media/paste-dc3ebcc6ff70c93bfa72f4bedc0f13ed98b88a41.jpg}
\end{center}


\note[item]{}
\end{frame}
\begin{frame}
\frametitle{Unrelated Title}

\begin{center}
\includegraphics[width=0.9\textwidth,height=0.9\textheight,keepaspectratio]{/Users/I516998/Library/Application Support/Anki2/User 1/collection.media/paste-f3f50c4f727b7e95421dae096828fb5dfb787145.jpg}
\end{center}


\note[item]{}
\end{frame}
\begin{frame}
\frametitle{Unrelated Title}

\begin{center}
\includegraphics[width=0.9\textwidth,height=0.9\textheight,keepaspectratio]{/Users/I516998/Library/Application Support/Anki2/User 1/collection.media/paste-55717073f880abe3f7f04fc3875611b68ec09d80.jpg}
\end{center}


\note[item]{}
\end{frame}
\begin{frame}
\frametitle{Unrelated Title}

\begin{center}
\includegraphics[width=0.9\textwidth,height=0.9\textheight,keepaspectratio]{/Users/I516998/Library/Application Support/Anki2/User 1/collection.media/paste-cf3feb17cc37b3d06d3bbf6dfb5f252e54a61850.jpg}
\end{center}


\note[item]{}
\end{frame}
\begin{frame}
\frametitle{Unrelated Title}

\begin{center}
\includegraphics[width=0.9\textwidth,height=0.9\textheight,keepaspectratio]{/Users/I516998/Library/Application Support/Anki2/User 1/collection.media/paste-ef128ab78df47553f689be762a9a1c33e75842a8.jpg}
\end{center}


\note[item]{}
\end{frame}
\begin{frame}
\frametitle{Unrelated Title}

\begin{center}
\includegraphics[width=0.9\textwidth,height=0.9\textheight,keepaspectratio]{/Users/I516998/Library/Application Support/Anki2/User 1/collection.media/paste-0d50ebfc1f939964122a55e63c047c8323ba0bf0.jpg}
\end{center}


\note[item]{}
\end{frame}
\begin{frame}
\frametitle{Unrelated Title}

\begin{center}
\includegraphics[width=0.9\textwidth,height=0.9\textheight,keepaspectratio]{/Users/I516998/Library/Application Support/Anki2/User 1/collection.media/paste-4abfe751b4ec84c15d5aa8162770bf85d6dbca28.jpg}
\end{center}


\note[item]{}
\end{frame}
\begin{frame}
\frametitle{Unrelated Title}

\begin{center}
\includegraphics[width=0.9\textwidth,height=0.9\textheight,keepaspectratio]{/Users/I516998/Library/Application Support/Anki2/User 1/collection.media/paste-81e6835e8e13310905803ac17c1d04db9ea62ee4.jpg}
\end{center}


\note[item]{}
\end{frame}
\begin{frame}
\frametitle{Unrelated Title}


\begin{itemize}
\item Durch das Network Discovery Protocol (NDP)
\end{itemize}

\note[item]{}
\end{frame}
\begin{frame}
\frametitle{Unrelated Title}

\begin{center}
\includegraphics[width=0.9\textwidth,height=0.9\textheight,keepaspectratio]{/Users/I516998/Library/Application Support/Anki2/User 1/collection.media/paste-c93c1e57d17af79d97abbc9d68d5782210dbc732.jpg}
\end{center}


\note[item]{}
\end{frame}
\begin{frame}
\frametitle{Unrelated Title}

\begin{center}
\includegraphics[width=0.9\textwidth,height=0.9\textheight,keepaspectratio]{/Users/I516998/Library/Application Support/Anki2/User 1/collection.media/paste-e80c4e3f974d36b9fcb7008f9c7283b203d3a432.jpg}
\end{center}


\note[item]{}
\end{frame}
\begin{frame}
\frametitle{Unrelated Title}

\begin{center}
\includegraphics[width=0.9\textwidth,height=0.9\textheight,keepaspectratio]{/Users/I516998/Library/Application Support/Anki2/User 1/collection.media/paste-cf54e5647343c4bf22470fef2e417c99f0b7ead9.jpg}
\end{center}


\note[item]{}
\end{frame}
\begin{frame}
\frametitle{Unrelated Title}

\begin{center}
\includegraphics[width=0.9\textwidth,height=0.9\textheight,keepaspectratio]{/Users/I516998/Library/Application Support/Anki2/User 1/collection.media/paste-28c07239e67df143aaf011f2f5db743bcd412064.jpg}
\end{center}

\begin{itemize}
\item SYN-Pakete zum Server-Port senden-->Warteschlange wird gefülltBei Erschöpfung können keine weiteren Verbindungsanfragen entgegengenommen werdenNach timeout erneut senden-->Keine hohe Datenübertragungsrate notwendig
\end{itemize}

\note[item]{}
\end{frame}
\begin{frame}
\frametitle{Unrelated Title}


\begin{itemize}
\item Findet im Vorfeld eines Angriffes stattPort gilt als offen, wenn eine Anwendung eingehende Anfragen akzeptiertTestpaket an Ziel-Port senden, je nach Antwort ist der Port offen, geschlossen oder gefiltert
\end{itemize}

\note[item]{}
\end{frame}
\begin{frame}
\frametitle{Unrelated Title}


\begin{itemize}
\item TCP-Scan: Ausspähen verfügbarer Dienste (filterbar, in Applikationsschicht/Logfile sichtbar)Half-Open/SYN-Scan: Heimliches Ausspähen verfügbarer Dienste (Filterbar, bleibt auf Transportschicht)FIN-Scan: Heimliches Ausspähen verfügarer Dienste (ggf nicht gefiltert, bleibt auf Transportschicht)
\end{itemize}

\note[item]{}
\end{frame}
\begin{frame}
\frametitle{Unrelated Title}


\begin{itemize}
\item Physikalische AngriffeARP-SpoofingÜbernahme eines ZwischenknotenManipulation von RoutingtabellenDNS-Soofing (durch Veränderung der "hosts" Datei, Unterschieben eines DNS Servers durch manipulierte DHCP-Antworten oder direktes DNS-Spoofing)
\end{itemize}

\note[item]{}
\end{frame}
\begin{frame}
\frametitle{Unrelated Title}


\begin{itemize}
\item Rechtliche Beziehung zwischen AG und AN
\item -> sind aneinander gebunden: Rechte und Pflichten
\item Gegenstand ist der AV: seine Entstehung, Störung und Beendigung
\end{itemize}

\note[item]{}
\end{frame}
\begin{frame}
\frametitle{Unrelated Title}


\begin{itemize}
\item Beziehung arbeitsrechtliche Koalitionen
\item Individuum tritt hinter das Kollektiv zurück
\item Kollektiv regelt die Beziehung durch Abschluss von Tarifverträgen oder Betriebsvereinbarungen
\end{itemize}

\note[item]{}
\end{frame}
\begin{frame}
\frametitle{Unrelated Title}


\begin{itemize}
\item Arbeitsschutzgesetze (MuSchG, BUrG, KSchG etc)
\item Ist dem öffentlichen Recht zugeordnet, da den Arbeitsschutzgesetzen ein formales Gesetzgebungsverfahren vorausgehen muss
\end{itemize}

\note[item]{}
\end{frame}
\begin{frame}
\frametitle{Unrelated Title}

\begin{center}
\includegraphics[width=0.9\textwidth,height=0.9\textheight,keepaspectratio]{/Users/I516998/Library/Application Support/Anki2/User 1/collection.media/img7148503818878948477.jpg}
\end{center}


\note[item]{}
\end{frame}
\begin{frame}
\frametitle{Unrelated Title}


\begin{itemize}
\item Außergesetzliche Anspruchsgrundlage -> Vertrauenshaftung
\item -> klagbarer Anspruch
\item 1. Freiwillige Leistung AG -> bereits durch Tatifvertrag, Gesetz oder Regelungen im AV festgehalten
\item 2. Regelmäßigkeit
\item 3. Kein Freiwilligkeitsvorbehalt
\end{itemize}

\note[item]{}
\end{frame}
\begin{frame}
\frametitle{Unrelated Title}


\begin{itemize}
\item Nach Art. 3 GG -> alle Menschen sind gleich
\item 1. Freiwillige Leistung d. AG -> wenn keine Verpflichtung durch Gesetz, Tarifvertrag oder AV vorliegt
\item 2. Regelhaftigkeit der Leistung -> Transparenzgebot
\item 3. Keine willkürliche Schlechterstellung -> entweder Gruppenbildung (objektiv und nachvollziehbar) oder Besserstellung statt Schlechterstellung
\end{itemize}

\note[item]{}
\end{frame}
\begin{frame}
\frametitle{Unrelated Title}


\begin{itemize}
\item 1. Vertragsfreiheit: Arbeitsbedingungen werden frei im Rahmen des geltenen Rechts verhandelt
\item => Abschlussfreiheit -> jede Partei kann frei entscheiden, ob und mit wem sie ein AV eingeht
\item Inhaltsfreiheit: Regelungen dürfen nicht gegen Gesetze verstoßen
\item Form: formfrei aber Nachweißgesetz -> schriftlicher Nachweis nach einem Tag erforderlich
\item Zeitpunkt der Begründung: nach Einigung der Parteien, unabhängig davon ob Tätigkeit bereits aufgenommen wurde
\item Schadensersatzansprüche bei nicht angetretenem AV möglich
\end{itemize}

\note[item]{}
\end{frame}
\begin{frame}
\frametitle{Unrelated Title}


\begin{itemize}
\item Hauptleistung:
\item AN-> erbringen der vereinbarten Arbeitsleitung
\item AG-> Zahlung der vereinbarten Vergütung
\item Nebenleistungen:
\item AN:
\item Gebot zur Rücksichtnahme
\item Schadensabwendung
\item Mitteilungspflicht
\item Verschwiegenheit
\item Nebentätigleitsanzeige
\item Wettbewerbsverbot
\item Verbot zur Schmiergeldannahme
\item AG:
\item Fürsorgepflicht
\item Urlaubsgewährung
\item Beschäftigungspflicht
\item Ordmungsgemäße Anmeldung und Abführung der Sozialbeiträge und Lohmsteuer
\item Gleichbehandlung
\item Nebenleistungspflicht hat nie Einfluss auf die Hauptleistungspflicht
\item Verletzung von Nebenpflichten führt zu Schadensersatzansprüchen
\item Ohne Arbeit kein Lohn -> Arbeitslestung hat Fixschuldcharakter
\item Kein erbringen der Arbeitsleistung ->$275 I BGB objektive Unmöglichkeit ->Anspruch auf Arbeitsleistung geht unter
\item AG und AN haben keinen Anspruch auf Nachholung
\item Anspruch auf Lohnzahlung geht nach $326 I unter
\end{itemize}

\note[item]{}
\end{frame}
\begin{frame}
\frametitle{Unrelated Title}


\begin{itemize}
\item $3 Entgeltfortzahlung (Krankheit)
\item $ BUrlG (Urlaub)
\item $11 MuSchG 
\item Und weitere
\end{itemize}

\note[item]{}
\end{frame}
\begin{frame}
\frametitle{Unrelated Title}


\begin{itemize}
\item 1. Schuldverhältnis: gültiger AV
\item 2. Pflichtverletzung
\item 3. Verschulden:
\item a) veränderter Haftungsmaßstab im Arbeitsrecht -> keine Haftung bei leichter Fahrlässigkeit
\item Mittlere und grobe Fahrlässigkeit-> anteilige Haftung AN
\item Vorsatz (bewusste und gewollte Schädigung) -> volle Haftung AN
\item b) keine Beweislastumkehr (vgl. $619a BGB)
\item -> AG muss Verschulden des AN positiv beweisen
\item 4. Schaden: jedes unfreiwillige Vermögensopfer, das in Geld messbar ist
\end{itemize}

\note[item]{}
\end{frame}
\begin{frame}
\frametitle{Unrelated Title}


\begin{itemize}
\item $766 BGB immer schriftlich
\item $350HGB bei Erteilung durch Kaufmann entfällt die Schriftpflicht
\end{itemize}

\note[item]{}
\end{frame}
\begin{frame}
\frametitle{Unrelated Title}


\begin{itemize}
\item Ein Gewerbe anmelden -> personenbezogene Dienstleistungen
\end{itemize}

\note[item]{}
\end{frame}
\begin{frame}
\frametitle{Unrelated Title}


\begin{itemize}
\item 2 Stufig
\item 1. Stufe: Gewerbe
\item - äußerlich erkennbar -> viele Geschäftsabschlüsse
\item - Regelmäßigkeit ->auf Dauer ausgelegt
\item - Gewinnerzielungsabsicht
\item - Selbständig $84 I, 2 HGB Arbeitszeit selbstbestimmt
\item - kein freier Beruf -> Ärzte, Steuerberater, Rechtsanwälte etc
\item Liegen alle 5 vor -> Gewerbe
\item 2. Stufe: Kaufmannseigenschaft
\item Ist-Kaufmann nach $1HGB
\item - Kaufmann kraft Betätigung:
\item Umsatz
\item Mitarbeiter
\item Standorte
\item Kredite
\item Bei genug Merkmalen: Kaufmann Kraft Betätigung 
\item Eintragung ins HR wirkt rein deklaratorisch
\item Form Kaufmann nach $6HGB
\item -> Handelsgesellschaften
\item Kann-Kaufmann $2HGB
\item = Kleingewerbetreibender, der sich freiwillig ins HR eintragen lässt
\item Eintragung wirkt konsititutiv
\item Kann Kaufmann nach $5HGB
\item Kauffmannseigenschaft dank bestehender Eintragung im HR
\end{itemize}

\note[item]{}
\end{frame}
\begin{frame}
\frametitle{Unrelated Title}


\begin{itemize}
\item Erteilung:
\item -durch Inhaber eines Handelsgewerbes $48HGB
\item -ausdrücklich, nicht konkludent
\item -persönlich oder gesetzl. Vertreter
\item -gegenüber zukünftigen Prokuristen oder Dritten
\item -Eintragungspflichtig, rein deklaratorisch
\item Umfang:
\item - Alle Geschäfte eines Handelsgewerbes - branchenüblich und branchenunüblich
\item - gerichtliche Handlungen 
\item Beschränkung gem $49 II:
\item - Veräußerung und Belastung v Grundstücken 
\item - Inhabergeschäfte 
\item - Privatgeschäfte d Inhabers
\item - Rechtshandlungen die auf die Freistellung des Betriebs des Handelsgewerbes gerichtet sind
\item - Beschränkung im Innenverhältnis nach $50 I möglich aber Dritten gegenüber immer unwirksam -> sie lösen lediglich einen Schadensersatzanspruch gem $$280 I, 241 II BGB aus
\item Erlöschen der Prokura:
\item - Widerruf $52 I HGB
\item - Beendigung AV
\item - Einstellung d Geschäftsbetriebes 
\item - Tod des Prokuristen
\item - Verlust der Kaufmannseigenschaft
\item BEI TOD DES INHABERS ERLÖSCHT DIE PROKURA NICHT $52 III HGB
\end{itemize}

\note[item]{}
\end{frame}
\begin{frame}
\frametitle{Unrelated Title}


\begin{itemize}
\item $54 HGB
\item Erteilung:
\item - ausdrücklich oder Konkludent
\item - persönlich durch Inhaber oder vertretungsberechtigte Person 
\item - keine Eintragung ins HR
\item - nur natürliche Personen
\item - durch Anscheins- oder Duldungssvollmacht möglich 
\item Arten: 
\item - General-HV
\item - Gattungs - HV
\item - Spezial - HV ( für ein best. Geschäft)
\item Umfang:
\item - beschränkt sich auf die gewöhnlichen Geschäfte -> branchenüblich
\item Gesetzliche Beschränkungen:
\item - keine gerichtlichen Handlungen
\item - keine Darlehensaufnahme
\item - keine Veräußerung oder Belastung von Grundstücken
\item - keine Wechselverbindlichkeiten
\item - interne Beschränkungen sind grundsätzlich möglich
\item BEACHTE: Ein Dritter muss diese Beschränkungen nur dann gegen sich gelten lassen, wenn er sie kannte oder kennen musste ( grob fahrlässige Unkenntnis) $54 III HGB
\item Erlöschen:
\item - $168 BGB Widerruf
\item - Tod des Handlungsbevollmächtigten
\item - Beendigung des AV
\end{itemize}

\note[item]{}
\end{frame}
\begin{frame}
\frametitle{Unrelated Title}


\begin{itemize}
\item $ 56HGB
\item - gesetzliche Anscheinsvollmacht wenn Voraussetzungen d $56 vorliegen
\item Voraussetzungen: 
\item 1. Laden oder offenes Warenlager
\item - Laden= jede, dem Publikum zugängliche, nicht notwendig dauerhafte Verkaufsstätte
\item - offenes Warenlager = Stätte, die zur Lagerung von Waren dient und dem Publikum für Geschäftsabschlüsse zugänglich ist
\item 2. Angestellter:
\item - mit Wissen und Wollen des GF im Laden/Warenlager geschäftlich mit Publikum verkehrend
\item 3. Örtlicher Zusammenhang zwischen Laden und Geschäftsabschluss
\item 4. Umfang:
\item - branchenübliche Geschäfte 
\item - Verkauf
\item - Empfangnahme (Waren Retoure Geld) aber keine Ankäufe!
\item Erlöschen:
\item - Beendigung der Tätigkeit 
\end{itemize}

\note[item]{}
\end{frame}
\begin{frame}
\frametitle{Unrelated Title}


\begin{itemize}
\item Alle Rechtsgeschäfte, die der Kaufmann im Zusammenhang mit seinem Handelsgeschäft vornimmt $343HGB
\end{itemize}

\note[item]{}
\end{frame}
\begin{frame}
\frametitle{Unrelated Title}


\begin{itemize}
\item Handelsgeschäfte entstehen durch Abschluss von Verträgen
\item Handelsbräuche $346HGB z.b. Handelsklauseln wie:
\item Frei Haus
\item Preis freibleibend
\item ab Werk
\item rein Netto
\item Bei internationalen Verträgen gelten standartisierte Incoterms
\item -> kaufmännisches Bestätigungsschreiben
\item - nicht gesetzlich geregelt
\item - gewohnheitsrechtlich anerkannter Handelsbrauch
\item => Schweigen unter Kaufleuten stellt unter folgenden Voraussetzungen eine Vertragsannahme dar:
\item 1. Fortgeschrittene Vertragsverhandlungen ab 3 Verhandlungen
\item 2. Uneinigkeit über einen wesentlichen Vertragsbestandteil
\item 3. Unmittelbar nachfolgendes Bestätigungsschreiben - kann jede Parte senden & Bezeichnung ist unerheblich
\item 4. Kein unverzüglicher Widerspruch -> ohne schuldhaftes Zögern
\item 5. Redlichkeit des Absenders (Vertrauen darauf, dass Schweigen des Empfängers als Annahme gewertet werden kann)
\item (-) bei sich kreuzenden Bestätigungsschreiben
\item (-) wenn bewusst oder unbewusst der bisherige Verhandlungsrahmen verlassen wird
\item RF: liegen alles Voraussetzungen vor, gilt Schweigen als rechtsverbindliche Annahme!
\end{itemize}

\note[item]{}
\end{frame}
\begin{frame}
\frametitle{Unrelated Title}


\begin{itemize}
\item Voraussetzungen:
\item 1. Beidseitiger Handelskauf
\item - Kaufvertrag nach $433
\item - beide Vertragspartner Kaufleute
\item 2. Übergabe der Ware
\item 3. Mangel nach $434BGB 
\item Qualitätsmangel, Falsche Menge, Aliud
\item 4. Mangel muss bei Übergabe bereits vorhanden sein 
\item 5. Keine Arglist
\item Kein vertraglicher Ausschluss $ des $377 HGB
\item UNVERZÜGLICHE Untersuchungspflicht mach Anlieferung der Ware
\item Bei Mangel: unverzügliche Rügepflicht 
\item -> ordnungsgemäß: Gewährleistungsrechte nach $439, 437 BGB
\item -> wird nicht ordnungsgemäß gerügt: Ware gilt als genehmigt
\item BEACHTE: Unterscheidung offene und verdeckte Mängel
\item Bei verdeckten Mängeln gilt $377 III Rügepflicht nach Kenntnis
\end{itemize}

\note[item]{}
\end{frame}
\begin{frame}
\frametitle{Unrelated Title}

\begin{center}
\includegraphics[width=0.9\textwidth,height=0.9\textheight,keepaspectratio]{/Users/I516998/Library/Application Support/Anki2/User 1/collection.media/img7647835416044308515.jpg}
\end{center}


\note[item]{}
\end{frame}
\begin{frame}
\frametitle{Unrelated Title}


\begin{itemize}
\item Provate Personenvereinigung, deren Mitglieder sich rechtsgeschäftlich zur Verfolgung eines gemeinsamen Zwecks zusammengeschlossen haben
\end{itemize}

\note[item]{}
\end{frame}
\begin{frame}
\frametitle{Unrelated Title}


\begin{itemize}
\item Zwang, nur zwischen den gesetzlich vorgesehenen Gesellschaftsarten, deren Organisation und Haltung zu wählen.
\item Soll vor Fantasiegesellschaften schützen-> Transparenz und Rechtssicherheit
\item Lexspecialis
\end{itemize}

\note[item]{}
\end{frame}
\begin{frame}
\frametitle{Unrelated Title}

\begin{center}
\includegraphics[width=0.9\textwidth,height=0.9\textheight,keepaspectratio]{/Users/I516998/Library/Application Support/Anki2/User 1/collection.media/img3238205123631721729.jpg}
\end{center}


\note[item]{}
\end{frame}
\begin{frame}
\frametitle{Unrelated Title}

\begin{center}
\includegraphics[width=0.9\textwidth,height=0.9\textheight,keepaspectratio]{/Users/I516998/Library/Application Support/Anki2/User 1/collection.media/img1603177206773386512.jpg}
\end{center}


\note[item]{}
\end{frame}
\begin{frame}
\frametitle{Unrelated Title}


\begin{itemize}
\item Geschäftsführung - Führung der Geschäfte im Innenverhältnis
\item $709 Führung steht Gesellschaftern gemeinschaftlich zu 
\item Einstimmigkeitsprinzip
\item Vertretung nach außen ggü Dritten -> Einzelvertretung nach $714BGB
\end{itemize}

\note[item]{}
\end{frame}
\begin{frame}
\frametitle{Unrelated Title}


\begin{itemize}
\item Gesamtschuldnerisch, unmittelbar, mit Gesellschafts und Privatvermögen 
\end{itemize}

\note[item]{}
\end{frame}
\begin{frame}
\frametitle{Unrelated Title}

\begin{center}
\includegraphics[width=0.9\textwidth,height=0.9\textheight,keepaspectratio]{/Users/I516998/Library/Application Support/Anki2/User 1/collection.media/img2865082707979991056.jpg}
\end{center}


\note[item]{}
\end{frame}
\begin{frame}
\frametitle{Unrelated Title}

\begin{center}
\includegraphics[width=0.9\textwidth,height=0.9\textheight,keepaspectratio]{/Users/I516998/Library/Application Support/Anki2/User 1/collection.media/img411934721060613731.jpg}
\end{center}


\note[item]{}
\end{frame}
\begin{frame}
\frametitle{Unrelated Title}

\begin{center}
\includegraphics[width=0.9\textwidth,height=0.9\textheight,keepaspectratio]{/Users/I516998/Library/Application Support/Anki2/User 1/collection.media/img8166294333388571460.jpg}
\end{center}


\note[item]{}
\end{frame}
\begin{frame}
\frametitle{Unrelated Title}

\begin{center}
\includegraphics[width=0.9\textwidth,height=0.9\textheight,keepaspectratio]{/Users/I516998/Library/Application Support/Anki2/User 1/collection.media/img7298347679591563130.jpg}
\end{center}


\note[item]{}
\end{frame}
\begin{frame}
\frametitle{Unrelated Title}

\begin{center}
\includegraphics[width=0.9\textwidth,height=0.9\textheight,keepaspectratio]{/Users/I516998/Library/Application Support/Anki2/User 1/collection.media/img7347753548700374353.jpg}
\end{center}


\note[item]{}
\end{frame}
\begin{frame}
\frametitle{Unrelated Title}

\begin{center}
\includegraphics[width=0.9\textwidth,height=0.9\textheight,keepaspectratio]{/Users/I516998/Library/Application Support/Anki2/User 1/collection.media/img5207297591629214564.jpg}
\end{center}


\note[item]{}
\end{frame}
\begin{frame}
\frametitle{Unrelated Title}

\begin{center}
\includegraphics[width=0.9\textwidth,height=0.9\textheight,keepaspectratio]{/Users/I516998/Library/Application Support/Anki2/User 1/collection.media/img8476758750244428131.jpg}
\end{center}


\note[item]{}
\end{frame}
\begin{frame}
\frametitle{Unrelated Title}

\begin{center}
\includegraphics[width=0.9\textwidth,height=0.9\textheight,keepaspectratio]{/Users/I516998/Library/Application Support/Anki2/User 1/collection.media/img5374982749803762437.jpg}
\end{center}


\note[item]{}
\end{frame}
\begin{frame}
\frametitle{Unrelated Title}

\begin{center}
\includegraphics[width=0.9\textwidth,height=0.9\textheight,keepaspectratio]{/Users/I516998/Library/Application Support/Anki2/User 1/collection.media/img6508148348170042369.jpg}
\end{center}


\note[item]{}
\end{frame}
\begin{frame}
\frametitle{Unrelated Title}


\begin{itemize}
\item Aufbau von Wechselbarrieren
\item Einschränkung der "Freiheit"
\item Lock in, Burggraben
\item -> Abos, Betriebssysteme, ERP Systeme
\end{itemize}

\note[item]{}
\end{frame}
\begin{frame}
\frametitle{Unrelated Title}


\begin{itemize}
\item Kunden dauerhaft zufrieden stellen
\item Freiwillige Bindung
\end{itemize}

\note[item]{}
\end{frame}
\begin{frame}
\frametitle{Unrelated Title}


\begin{itemize}
\item Hit&Run:
\item Hohe Marge
\item Einmaliges Geschäft, dann weiterziehen
\item "Ausnutzen" der kurzfristigen Bedürfnisse der Kunden -> Dose Cola bei 40°C
\item Relationship:
\item Image ist wichtig für langfristige Kundenbeziehung/Kundenbindung
\item Horrende Preise nicht möglich wegen Kundenzufriedenheit
\item Auf viele zukünftige Geschäftsabschlüsse ausgerichtet
\end{itemize}

\note[item]{}
\end{frame}
\begin{frame}
\frametitle{Unrelated Title}


\begin{itemize}
\item Entsteht durch den Abgleich der Erwartungen des Kunden mit dem wahrgenommenen IST
\end{itemize}

\note[item]{}
\end{frame}
\begin{frame}
\frametitle{Unrelated Title}

\begin{center}
\includegraphics[width=0.9\textwidth,height=0.9\textheight,keepaspectratio]{/Users/I516998/Library/Application Support/Anki2/User 1/collection.media/img404337859277052286.jpg}
\end{center}


\note[item]{}
\end{frame}
\begin{frame}
\frametitle{Unrelated Title}

\begin{center}
\includegraphics[width=0.9\textwidth,height=0.9\textheight,keepaspectratio]{/Users/I516998/Library/Application Support/Anki2/User 1/collection.media/img1426974243029795533.jpg}
\end{center}


\note[item]{}
\end{frame}
\begin{frame}
\frametitle{Unrelated Title}

\begin{center}
\includegraphics[width=0.9\textwidth,height=0.9\textheight,keepaspectratio]{/Users/I516998/Library/Application Support/Anki2/User 1/collection.media/img1906080333729559248.jpg}
\end{center}


\note[item]{}
\end{frame}
\begin{frame}
\frametitle{Unrelated Title}

\begin{center}
\includegraphics[width=0.9\textwidth,height=0.9\textheight,keepaspectratio]{/Users/I516998/Library/Application Support/Anki2/User 1/collection.media/img3408274160010113958.jpg}
\end{center}


\note[item]{}
\end{frame}
\begin{frame}
\frametitle{Unrelated Title}

\begin{center}
\includegraphics[width=0.9\textwidth,height=0.9\textheight,keepaspectratio]{/Users/I516998/Library/Application Support/Anki2/User 1/collection.media/img52145628627610158.jpg}
\end{center}


\note[item]{}
\end{frame}
\begin{frame}
\frametitle{Unrelated Title}

\begin{center}
\includegraphics[width=0.9\textwidth,height=0.9\textheight,keepaspectratio]{/Users/I516998/Library/Application Support/Anki2/User 1/collection.media/img6381290615392288878.jpg}
\end{center}


\note[item]{}
\end{frame}
\begin{frame}
\frametitle{Unrelated Title}

\begin{center}
\includegraphics[width=0.9\textwidth,height=0.9\textheight,keepaspectratio]{/Users/I516998/Library/Application Support/Anki2/User 1/collection.media/img1751999793452194112.jpg}
\end{center}


\note[item]{}
\end{frame}
\begin{frame}
\frametitle{Unrelated Title}


\begin{itemize}
\item Fehler:
\item Kurzfristperspektive
\item Nicht vorhandene, verstopfte oder gekappte Kommunikationskanäle (Bsp. Kommunikation Betrieb-WG-Vertrieb)
\item Abstraktes bzw. widersprüchliches Unternehmensleitbild
\item Fehlende Orientierung an externen und internen Kunden
\item Mangelnde Motivation bzw. fehlende Anreize (bsp Mitarbeiter bei Kundenfrage) 
\item Mangelnde Fähigkeit bzw. Motivation, aua Unzufriedenheitssignalen konkrete Maßnahmen abzuleiten 
\end{itemize}

\note[item]{}
\end{frame}
\begin{frame}
\frametitle{Unrelated Title}

\begin{center}
\includegraphics[width=0.9\textwidth,height=0.9\textheight,keepaspectratio]{/Users/I516998/Library/Application Support/Anki2/User 1/collection.media/img8611235772270881992.jpg}
\end{center}


\note[item]{}
\end{frame}
\begin{frame}
\frametitle{Unrelated Title}

\begin{center}
\includegraphics[width=0.9\textwidth,height=0.9\textheight,keepaspectratio]{/Users/I516998/Library/Application Support/Anki2/User 1/collection.media/img992529242716381473.jpg}
\end{center}


\note[item]{}
\end{frame}
\begin{frame}
\frametitle{Unrelated Title}

\begin{center}
\includegraphics[width=0.9\textwidth,height=0.9\textheight,keepaspectratio]{/Users/I516998/Library/Application Support/Anki2/User 1/collection.media/img6974748008076676669.jpg}
\end{center}


\note[item]{}
\end{frame}
\begin{frame}
\frametitle{Unrelated Title}

\begin{center}
\includegraphics[width=0.9\textwidth,height=0.9\textheight,keepaspectratio]{/Users/I516998/Library/Application Support/Anki2/User 1/collection.media/img4333930500736648296.jpg}
\end{center}


\note[item]{}
\end{frame}
\begin{frame}
\frametitle{Unrelated Title}

\begin{center}
\includegraphics[width=0.9\textwidth,height=0.9\textheight,keepaspectratio]{/Users/I516998/Library/Application Support/Anki2/User 1/collection.media/img604315609941367486.jpg}
\end{center}


\note[item]{}
\end{frame}
\begin{frame}
\frametitle{Unrelated Title}

\begin{center}
\includegraphics[width=0.9\textwidth,height=0.9\textheight,keepaspectratio]{/Users/I516998/Library/Application Support/Anki2/User 1/collection.media/img7723728124240978484.jpg}
\end{center}


\note[item]{}
\end{frame}
\begin{frame}
\frametitle{Unrelated Title}


\begin{itemize}
\item Externe Referenzpreise:
\item UVP Preise
\item WW Preise
\item Durchgestrichene Preise 
\item Interne Referenzpreise:
\item Preiswissen
\item Latentes Bauchgefühl
\end{itemize}

\note[item]{}
\end{frame}
\begin{frame}
\frametitle{Unrelated Title}


\begin{itemize}
\item E=dx/dp ACHTUNG VORZEICHEN
\item Sacrifice Effekt: negative Elastizität
\item Signaling Effekt: positive Elastizität
\end{itemize}

\note[item]{}
\end{frame}
\begin{frame}
\frametitle{Unrelated Title}

\begin{center}
\includegraphics[width=0.9\textwidth,height=0.9\textheight,keepaspectratio]{/Users/I516998/Library/Application Support/Anki2/User 1/collection.media/img2330234242906708482.jpg}
\end{center}


\note[item]{}
\end{frame}
\begin{frame}
\frametitle{Unrelated Title}

\begin{center}
\includegraphics[width=0.9\textwidth,height=0.9\textheight,keepaspectratio]{/Users/I516998/Library/Application Support/Anki2/User 1/collection.media/img2351264712889164620.jpg}
\end{center}


\note[item]{}
\end{frame}
\begin{frame}
\frametitle{Unrelated Title}

\begin{center}
\includegraphics[width=0.9\textwidth,height=0.9\textheight,keepaspectratio]{/Users/I516998/Library/Application Support/Anki2/User 1/collection.media/img3806626865646412528.jpg}
\includegraphics[width=0.9\textwidth,height=0.9\textheight,keepaspectratio]{/Users/I516998/Library/Application Support/Anki2/User 1/collection.media/img8629459084360296837.jpg}
\end{center}

\begin{itemize}
\item EDLP:
\item + 
\item Weniger Marketingkosten
\item Zuverlässigkeit 
\item Vertrauen
\item -
\item Kein Verkauf durch Rabatte
\item Abwandern der Kunden zu High Lower
\item High Low:
\item +
\item Spielraum in den Preisen
\item Kunden anlocken
\item WKZ (Werbekostenzuschläge)
\item -
\item "Verziehen" die Kunden
\item Erwartungsfunktion
\item Hohe Personal und Werbekosten
\end{itemize}

\note[item]{}
\end{frame}
\begin{frame}
\frametitle{Unrelated Title}

\begin{center}
\includegraphics[width=0.9\textwidth,height=0.9\textheight,keepaspectratio]{/Users/I516998/Library/Application Support/Anki2/User 1/collection.media/img8559147181738842745.jpg}
\end{center}


\note[item]{}
\end{frame}
\begin{frame}
\frametitle{Unrelated Title}


\begin{itemize}
\item Kunde
\item /                  \
\item Kosten.       Wettbewerb
\end{itemize}

\note[item]{}
\end{frame}
\begin{frame}
\frametitle{Unrelated Title}

\begin{center}
\includegraphics[width=0.9\textwidth,height=0.9\textheight,keepaspectratio]{/Users/I516998/Library/Application Support/Anki2/User 1/collection.media/img8043712675439352343.jpg}
\end{center}


\note[item]{}
\end{frame}
\begin{frame}
\frametitle{Unrelated Title}


\begin{itemize}
\item +
\item Einfach, schnell
\item Kosten sind sicher gedeckt
\item -
\item Muss mit Prognosen arbeiten
\item Wettbewerb und Kundenvorstellungen werden außenvor gelassen
\item Verschenkte Potentiale für höhere Marge
\end{itemize}

\note[item]{}
\end{frame}
\begin{frame}
\frametitle{Unrelated Title}

\begin{center}
\includegraphics[width=0.9\textwidth,height=0.9\textheight,keepaspectratio]{/Users/I516998/Library/Application Support/Anki2/User 1/collection.media/img4985526543331598077.jpg}
\end{center}


\note[item]{}
\end{frame}
\begin{frame}
\frametitle{Unrelated Title}


\begin{itemize}
\item eAB= (prozentuale Absatzänderung A) / (prozentuale Preisänderung B)
\end{itemize}

\note[item]{}
\end{frame}
\begin{frame}
\frametitle{Unrelated Title}


\begin{itemize}
\item +
\item Preise übernehmen, um Marge zu verbessern oder "günstig" Image zu sichern
\item -> geht nur bei vergleichbaren Leistungen
\item einfache und schnelle Preisfindung
\item Vermeide Fehltritte
\item -
\item Achtung! Kosten werden außer Acht gelassen
\item riskiere Preiskrieg
\item Kunde wird außenvor gelassen
\end{itemize}

\note[item]{}
\end{frame}
\begin{frame}
\frametitle{Unrelated Title}

\begin{center}
\includegraphics[width=0.9\textwidth,height=0.9\textheight,keepaspectratio]{/Users/I516998/Library/Application Support/Anki2/User 1/collection.media/img8559596549418819920.jpg}
\end{center}


\note[item]{}
\end{frame}
\begin{frame}
\frametitle{Unrelated Title}

\begin{center}
\includegraphics[width=0.9\textwidth,height=0.9\textheight,keepaspectratio]{/Users/I516998/Library/Application Support/Anki2/User 1/collection.media/img8623580756496215278.jpg}
\end{center}


\note[item]{}
\end{frame}
\begin{frame}
\frametitle{Unrelated Title}

\begin{center}
\includegraphics[width=0.9\textwidth,height=0.9\textheight,keepaspectratio]{/Users/I516998/Library/Application Support/Anki2/User 1/collection.media/img1918354605289647638.jpg}
\end{center}


\note[item]{}
\end{frame}
\begin{frame}
\frametitle{Unrelated Title}


\begin{itemize}
\item +
\item Fragen direkt die Entscheider
\item Nicht aufwendig
\item Mindestanzahl an Probanden nötig um sichere Aussagen zu treffen
\item -
\item Nicht für viele Artikel geeignet
\item Hige Fallzahlen -> hohe Kosten
\item Wettbewerb wird nicht mit einbezogen
\end{itemize}

\note[item]{}
\end{frame}
\begin{frame}
\frametitle{Unrelated Title}


\begin{itemize}
\item +
\item Gibt auch die Anzahl der Kunden, die das Produkt kaufen würden
\item Sehr präzise
\item Berechnung Umsatzmaximum
\item Inkludiert Wettbewerb
\item Erkenntnisse über Produktvorlieben der Kunden
\item -
\item Sehr teuer
\item Komplex
\item Aufwendig
\item Nicht für viele Produkte geeignet 
\item Hauptsächlich für hochwertige Elektronikartikel verwendet
\end{itemize}

\note[item]{}
\end{frame}
\begin{frame}
\frametitle{Unrelated Title}

\begin{center}
\includegraphics[width=0.9\textwidth,height=0.9\textheight,keepaspectratio]{/Users/I516998/Library/Application Support/Anki2/User 1/collection.media/img6602071578114699451.jpg}
\end{center}


\note[item]{}
\end{frame}
\begin{frame}
\frametitle{Unrelated Title}

\begin{center}
\includegraphics[width=0.9\textwidth,height=0.9\textheight,keepaspectratio]{/Users/I516998/Library/Application Support/Anki2/User 1/collection.media/img7803165609936347462.jpg}
\end{center}


\note[item]{}
\end{frame}
\begin{frame}
\frametitle{Unrelated Title}

\begin{center}
\includegraphics[width=0.9\textwidth,height=0.9\textheight,keepaspectratio]{/Users/I516998/Library/Application Support/Anki2/User 1/collection.media/img3377758410168689421.jpg}
\end{center}


\note[item]{}
\end{frame}
\begin{frame}
\frametitle{Unrelated Title}


\begin{itemize}
\item Anwendungsbereich KSchG $1,23
\item Sozial gerechtfertigt? -> dringende betr. Erfordernisse die einer Weiterbeschäftigung des AN im Betrieb entgegenstehen. 
\item Liegt nicht an der Person des AN, sondern dass der AG den Arbeitsplatz nicht mehr zur Verfügung stellen kann. 
\item Betriebliche Kündigung ist nur gerechtfertigt, wenn eine unternehmerische Entscheidung vorliegt, die aufgrund inner- oder außerbetrieblicher Gründe seitens des AG getroffen wurde und diese dazu geführt hat, dass der Arbeitsplatz dauerhaft wegfällt. 
\item Unternehmerische Entscheidung wird seitens des AG aufgrund seiner Organisationsmacht im Betrieb getroffen und untersteht seiner unternehmerischen Freiheit
\item -> geschützt durch GG
\item -> Gericht prüft nur ob die unternehmerische Entscheidung vorliegt, nicht ob diese sinnvoll ist
\item Inner- oder außerbetriebliche Gründe:
\item Außerbetrieblich: alles was von außen auf Betrieb einwirkt bsp mangelnde Aufträge, Rohstoffe, Gesetzesänderungen etc
\item Innerbetrieblich: mangelnde Rentabilität, Gewinnrückgang etc. 
\item Beachte: Inner - oder außerbetriebliche Gründe werden vom Gericht geprüft -> AG trägt Darlegungs- und Beweislast
\item Auch dafür, dass die betrieblichen Gründe zum Wegfall des Arbeitsplatzes geführt haben. 
\item Verhältnismäßigkeit: 
\item Ultima Ratio -> darf kein milderes Mittel geben wie z.b. den AN innerhalb des Betriebes weiterzubeschäftigen auf einem Arbeitsplatz, für den er geeignet ist
\item Der Arbeitgeber muss jedoch KEINEN NEUEN Arbeitsplatz schaffen
\item Interessenabwägung: $1 III KSchG 
\item -> AG muss Sozialplan erstellen, der mindestens die Punkte in $1 III KSchG berücksichtigt 
\item -> muss keinen erstellen wenn er eine ganze Abteilung schließt
\item -> Sozialauswahl wird vom Gericht nicht überprüft, wenn der Betriebsrat vorher bereits mitgewirkt hat
\end{itemize}

\note[item]{}
\end{frame}
\begin{frame}
\frametitle{Unrelated Title}


\begin{itemize}
\item A  complete computer installation including hardware, software, users, provedures and data.
\end{itemize}

\note[item]{}
\end{frame}
\begin{frame}
\frametitle{Unrelated Title}


\begin{itemize}
\item An electronic device capable of executing a set of instructions or a 'program'.
\end{itemize}

\note[item]{}
\end{frame}
\begin{frame}
\frametitle{Unrelated Title}


\begin{itemize}
\item Computers that carry out single extremely complex computing tasks such as climate research, cryptanalysis, physical simulations, oil & gas explorations...etc.
\end{itemize}

\note[item]{}
\end{frame}
\begin{frame}
\frametitle{Unrelated Title}


\begin{itemize}
\item FLOPS (Floating Point Operations per Second)
\end{itemize}

\note[item]{}
\end{frame}
\begin{frame}
\frametitle{Unrelated Title}


\begin{itemize}
\item A large Scale Powerful computer with a large storage capacity, a fast Central Processing Unit (CPU).
\end{itemize}

\note[item]{}
\end{frame}
\begin{frame}
\frametitle{Unrelated Title}


\begin{itemize}
\item Central Processing Unit - the brains of the computer where the program instructions are processed or carried out.
\end{itemize}

\note[item]{}
\end{frame}
\begin{frame}
\frametitle{Unrelated Title}


\begin{itemize}
\item The computer's Secondary Storage: hard disk drives and optical drives.
\end{itemize}

\note[item]{}
\end{frame}
\begin{frame}
\frametitle{Unrelated Title}


\begin{itemize}
\item Computer Small Enough to fit on a desk, but also powerful enough for common business tasks and inexpensive enough for home owners.
\end{itemize}

\note[item]{}
\end{frame}
\begin{frame}
\frametitle{Unrelated Title}


\begin{itemize}
\item One that is suitable fopr undertaking a wide range of common computing tasks.
\end{itemize}

\note[item]{}
\end{frame}
\begin{frame}
\frametitle{Unrelated Title}


\begin{itemize}
\item Computer device small enough to hold and operate in one's hand and are easily portable.
\end{itemize}

\note[item]{}
\end{frame}
\begin{frame}
\frametitle{Unrelated Title}


\begin{itemize}
\item A special purpose computer that carries out a specific and dedicated function within a larger electrical or mechanical system, or a combination of both.
\end{itemize}

\note[item]{}
\end{frame}
\begin{frame}
\frametitle{Unrelated Title}


\begin{itemize}
\item Performs a small range of tasks and contains features uniquely designed for use in a particular industry or application.
\end{itemize}

\note[item]{}
\end{frame}
\begin{frame}
\frametitle{Unrelated Title}

\begin{center}
\includegraphics[width=0.9\textwidth,height=0.9\textheight,keepaspectratio]{/Users/I516998/Library/Application Support/Anki2/User 1/collection.media/paste-300049b704d88f2172a8199cf20d4fdcf213f9ba.jpg}
\end{center}


\note[item]{}
\end{frame}
\begin{frame}
\frametitle{Unrelated Title}

\begin{center}
\includegraphics[width=0.9\textwidth,height=0.9\textheight,keepaspectratio]{/Users/I516998/Library/Application Support/Anki2/User 1/collection.media/paste-25f663e10ebec0eb8a9999f33118db17a3df14dd.jpg}
\end{center}


\note[item]{}
\end{frame}
\begin{frame}
\frametitle{Unrelated Title}

\begin{center}
\includegraphics[width=0.9\textwidth,height=0.9\textheight,keepaspectratio]{/Users/I516998/Library/Application Support/Anki2/User 1/collection.media/paste-65708201b7f92164f49dae843be7e32b11b7626a.jpg}
\end{center}


\note[item]{}
\end{frame}
\begin{frame}
\frametitle{Unrelated Title}

\begin{center}
\includegraphics[width=0.9\textwidth,height=0.9\textheight,keepaspectratio]{/Users/I516998/Library/Application Support/Anki2/User 1/collection.media/paste-28d58871657c4802345a94b4a77a66d45e97d1ca.jpg}
\end{center}


\note[item]{}
\end{frame}
\begin{frame}
\frametitle{Unrelated Title}

\begin{center}
\includegraphics[width=0.9\textwidth,height=0.9\textheight,keepaspectratio]{/Users/I516998/Library/Application Support/Anki2/User 1/collection.media/paste-adcb66bad9a26a4198f710b107a751bbc1b3cd7e.jpg}
\end{center}


\note[item]{}
\end{frame}
\begin{frame}
\frametitle{Unrelated Title}

\begin{center}
\includegraphics[width=0.9\textwidth,height=0.9\textheight,keepaspectratio]{/Users/I516998/Library/Application Support/Anki2/User 1/collection.media/paste-773f01d2c38a5c9dde230f9f2de142c507b8a381.jpg}
\end{center}


\note[item]{}
\end{frame}
\begin{frame}
\frametitle{Unrelated Title}

\begin{center}
\includegraphics[width=0.9\textwidth,height=0.9\textheight,keepaspectratio]{/Users/I516998/Library/Application Support/Anki2/User 1/collection.media/paste-8c83c1a8aed81a3fd3e2299011372426b5fd8873.jpg}
\end{center}


\note[item]{}
\end{frame}
\begin{frame}
\frametitle{Unrelated Title}

\begin{center}
\includegraphics[width=0.9\textwidth,height=0.9\textheight,keepaspectratio]{/Users/I516998/Library/Application Support/Anki2/User 1/collection.media/paste-7ae8b9e6147c3d750cb2b78cdb694721fc8c1a18.jpg}
\end{center}


\note[item]{}
\end{frame}
\begin{frame}
\frametitle{Unrelated Title}

\begin{center}
\includegraphics[width=0.9\textwidth,height=0.9\textheight,keepaspectratio]{/Users/I516998/Library/Application Support/Anki2/User 1/collection.media/paste-e79bc7b98152dc056167199ec6e78ca0b258c309.jpg}
\end{center}


\note[item]{}
\end{frame}
\begin{frame}
\frametitle{Unrelated Title}

\begin{center}
\includegraphics[width=0.9\textwidth,height=0.9\textheight,keepaspectratio]{/Users/I516998/Library/Application Support/Anki2/User 1/collection.media/paste-37d0fd72218d4bfae793e25e6fa129d44dbcbfe0.jpg}
\end{center}


\note[item]{}
\end{frame}
\begin{frame}
\frametitle{Unrelated Title}

\begin{center}
\includegraphics[width=0.9\textwidth,height=0.9\textheight,keepaspectratio]{/Users/I516998/Library/Application Support/Anki2/User 1/collection.media/paste-1a8fc4f68b4cb10e65b91e5ce033baf7e42da1d0.jpg}
\end{center}


\note[item]{}
\end{frame}
\begin{frame}
\frametitle{Unrelated Title}

\begin{center}
\includegraphics[width=0.9\textwidth,height=0.9\textheight,keepaspectratio]{/Users/I516998/Library/Application Support/Anki2/User 1/collection.media/paste-c89ec96785fe61a5c1bfb5af3853fb52e059a7a5.jpg}
\end{center}


\note[item]{}
\end{frame}
\begin{frame}
\frametitle{Unrelated Title}

\begin{center}
\includegraphics[width=0.9\textwidth,height=0.9\textheight,keepaspectratio]{/Users/I516998/Library/Application Support/Anki2/User 1/collection.media/paste-7314dfbad262ce3d207f51c1083b4d30e7390c83.jpg}
\end{center}


\note[item]{}
\end{frame}
\begin{frame}
\frametitle{Unrelated Title}

\begin{center}
\includegraphics[width=0.9\textwidth,height=0.9\textheight,keepaspectratio]{/Users/I516998/Library/Application Support/Anki2/User 1/collection.media/paste-94c34524152bb75a2059ef1a4ddb1b7d410f8604.jpg}
\end{center}


\note[item]{}
\end{frame}
\begin{frame}
\frametitle{Unrelated Title}

\begin{center}
\includegraphics[width=0.9\textwidth,height=0.9\textheight,keepaspectratio]{/Users/I516998/Library/Application Support/Anki2/User 1/collection.media/paste-b85eed647bae3507e23910348be19bf39bf901e2.jpg}
\end{center}


\note[item]{}
\end{frame}
\begin{frame}
\frametitle{Unrelated Title}

\begin{center}
\includegraphics[width=0.9\textwidth,height=0.9\textheight,keepaspectratio]{/Users/I516998/Library/Application Support/Anki2/User 1/collection.media/paste-253bed5256ad14e15abc98504f4bd99355fde2ce.jpg}
\end{center}


\note[item]{}
\end{frame}
\begin{frame}
\frametitle{Unrelated Title}

\begin{center}
\includegraphics[width=0.9\textwidth,height=0.9\textheight,keepaspectratio]{/Users/I516998/Library/Application Support/Anki2/User 1/collection.media/paste-e6fdd003281efc7184166e59ccdd93802d84c21d.jpg}
\end{center}


\note[item]{}
\end{frame}
\begin{frame}
\frametitle{Unrelated Title}

\begin{center}
\includegraphics[width=0.9\textwidth,height=0.9\textheight,keepaspectratio]{/Users/I516998/Library/Application Support/Anki2/User 1/collection.media/paste-6792078bf868970119c1ca5d96b97c5f3baff946.jpg}
\end{center}


\note[item]{}
\end{frame}
\begin{frame}
\frametitle{Unrelated Title}

\begin{center}
\includegraphics[width=0.9\textwidth,height=0.9\textheight,keepaspectratio]{/Users/I516998/Library/Application Support/Anki2/User 1/collection.media/paste-9f518ab01a16fd5c8be75ca32fbfd1f13e01aadd.jpg}
\end{center}


\note[item]{}
\end{frame}
\begin{frame}
\frametitle{Unrelated Title}

\begin{center}
\includegraphics[width=0.9\textwidth,height=0.9\textheight,keepaspectratio]{/Users/I516998/Library/Application Support/Anki2/User 1/collection.media/paste-fb8a03e837d5241c1dda7c30dbd6e2cbfdbffd3d.jpg}
\end{center}


\note[item]{}
\end{frame}
\begin{frame}
\frametitle{Unrelated Title}

\begin{center}
\includegraphics[width=0.9\textwidth,height=0.9\textheight,keepaspectratio]{/Users/I516998/Library/Application Support/Anki2/User 1/collection.media/paste-51fc936e937b76ae9462b51aa9a9aa41d0af5091.jpg}
\end{center}


\note[item]{}
\end{frame}
\begin{frame}
\frametitle{Unrelated Title}

\begin{center}
\includegraphics[width=0.9\textwidth,height=0.9\textheight,keepaspectratio]{/Users/I516998/Library/Application Support/Anki2/User 1/collection.media/paste-82c60e0ba364e78161a97e536dd210ecf8f58471.jpg}
\end{center}


\note[item]{}
\end{frame}
\begin{frame}
\frametitle{Unrelated Title}

\begin{center}
\includegraphics[width=0.9\textwidth,height=0.9\textheight,keepaspectratio]{/Users/I516998/Library/Application Support/Anki2/User 1/collection.media/paste-85ada5bdeab538458cd58f9610bba24cd86a5f1f.jpg}
\end{center}


\note[item]{}
\end{frame}
\begin{frame}
\frametitle{Unrelated Title}

\begin{center}
\includegraphics[width=0.9\textwidth,height=0.9\textheight,keepaspectratio]{/Users/I516998/Library/Application Support/Anki2/User 1/collection.media/paste-3a86c55314d0ba02da006ee3de0b88897f238234.jpg}
\end{center}


\note[item]{}
\end{frame}
\begin{frame}
\frametitle{Unrelated Title}


\begin{itemize}
\item Integrated package software contains several relaled prograns in one package.
\end{itemize}

\note[item]{}
\end{frame}
\begin{frame}
\frametitle{Unrelated Title}


\begin{itemize}
\item Special-purpose application is sofiware that is used lo perform narrowly focused tasks.
\end{itemize}

\note[item]{}
\end{frame}
\begin{frame}
\frametitle{Unrelated Title}

\begin{center}
\includegraphics[width=0.9\textwidth,height=0.9\textheight,keepaspectratio]{/Users/I516998/Library/Application Support/Anki2/User 1/collection.media/paste-77a320ea9eb5e9606c50e81349e4324ce2bceaf7.jpg}
\end{center}


\note[item]{}
\end{frame}
\begin{frame}
\frametitle{Unrelated Title}


\begin{itemize}
\item A user interface is the combination of hardware and software that allows users
and computers to communicate with each other effectively.
\end{itemize}

\note[item]{}
\end{frame}
\begin{frame}
\frametitle{Unrelated Title}


\begin{itemize}
\item A hardware interface is those physical hardware components (input and output
devices) that allow the user to manipulate the commputer.
\end{itemize}

\note[item]{}
\end{frame}
\begin{frame}
\frametitle{Unrelated Title}


\begin{itemize}
\item A software interface is the programs that are used to communicate with the
computer via the hardware.
\end{itemize}

\note[item]{}
\end{frame}
\begin{frame}
\frametitle{Unrelated Title}


\begin{itemize}
\item A command line interface is when interaction with a computer is by means ot
individual lines of text.
For example, in the MS-DOS operating system the command "DIR" means 'Display a list of files and subdirectories in a directory'.
\end{itemize}

\note[item]{}
\end{frame}
\begin{frame}
\frametitle{Unrelated Title}


\begin{itemize}
\item A menu-driven interlace is when interaction with a computer is by the user
selecting one option from a list of presented options.
\end{itemize}

\note[item]{}
\end{frame}
\begin{frame}
\frametitle{Unrelated Title}


\begin{itemize}
\item A Graphical user interface (GUI) is when interaction with a computer is by using
a pointing device.
\end{itemize}

\note[item]{}
\end{frame}
\begin{frame}
\frametitle{Unrelated Title}


\begin{itemize}
\item Processing Speed at a basic level is how quickly a microprocessor operates.
\end{itemize}

\note[item]{}
\end{frame}
\begin{frame}
\frametitle{Unrelated Title}

\begin{center}
\includegraphics[width=0.9\textwidth,height=0.9\textheight,keepaspectratio]{/Users/I516998/Library/Application Support/Anki2/User 1/collection.media/paste-436a6df9194ac1216ca462b7ee5d82af4a6d2d36.jpg}
\end{center}


\note[item]{}
\end{frame}
\begin{frame}
\frametitle{Unrelated Title}


\begin{itemize}
\item Data is raw, unprocessed facts. This may be facts about persons, places, things or
events that have been collected through observation or measurement.
\end{itemize}

\note[item]{}
\end{frame}
\begin{frame}
\frametitle{Unrelated Title}


\begin{itemize}
\item fomation is meaningtul knowledge derived from raw data. Data that
has been processed, organised or put into context so that it is meaningful to the
user may be regarded as information.
\end{itemize}

\note[item]{}
\end{frame}
\begin{frame}
\frametitle{Unrelated Title}


\begin{itemize}
\item Interview experts in a field; people related to your research.Books, Newspapers, magazines and journals.Computer databases, company databases.Websites on the internet.
\end{itemize}

\note[item]{}
\end{frame}
\begin{frame}
\frametitle{Unrelated Title}


\begin{itemize}
\item A source document is a docunent that contains data for input into an information procesing system.
\end{itemize}

\note[item]{}
\end{frame}
\begin{frame}
\frametitle{Unrelated Title}


\begin{itemize}
\item A document is printed or written, is usually paper based, and is used to collect,
store ard share data.
\end{itemize}

\note[item]{}
\end{frame}
\begin{frame}
\frametitle{Unrelated Title}


\begin{itemize}
\item Turnaround DocumentHuman-Readable DocumentMachine-Readable Document
\end{itemize}

\note[item]{}
\end{frame}
\begin{frame}
\frametitle{Unrelated Title}


\begin{itemize}
\item This is a document produced by a computer that can be used to record information by a human.  The document with its new information can then be input into the computer system for further processing.OMR and OCR are technologies that are widely applied in this process.
\end{itemize}

\note[item]{}
\end{frame}
\begin{frame}
\frametitle{Unrelated Title}


\begin{itemize}
\item Human readable documents , simply put, can be read by humans.Machine readable documents contain  parts that can only be read by a computer system and not by humans.  E.g. barcodes, QR codes etc.
\end{itemize}

\note[item]{}
\end{frame}
\begin{frame}
\frametitle{Unrelated Title}


\begin{itemize}
\item A machine-readable document is a document that can be read directly an
understood by commputer systems. Examples are documents that include barcodes
and QR (scan) codes. QR codes are like barcodes but are made up of sall black
and white squares instead ol lines.
\end{itemize}

\note[item]{}
\end{frame}
\begin{frame}
\frametitle{Unrelated Title}


\begin{itemize}
\item he degree or extent to which the content of the information can be depended on to be accurate.
\end{itemize}

\note[item]{}
\end{frame}
\begin{frame}
\frametitle{Unrelated Title}


\begin{itemize}
\item authenticity,currency,relevance,lack of bias
\end{itemize}

\note[item]{}
\end{frame}
\begin{frame}
\frametitle{Unrelated Title}


\begin{itemize}
\item Credibility refers to the amount of trust we place in an information source being correct.
\end{itemize}

\note[item]{}
\end{frame}
\begin{frame}
\frametitle{Unrelated Title}


\begin{itemize}
\item This refers to the integrity of information or how correct the information is.  Usually, we try to verify gathered information against other sources to establish that the information is accurate.
\end{itemize}

\note[item]{}
\end{frame}
\begin{frame}
\frametitle{Unrelated Title}


\begin{itemize}
\item This refers to the correctness of discrete items in the information source and how complete the information is for it’s application.
\end{itemize}

\note[item]{}
\end{frame}
\begin{frame}
\frametitle{Unrelated Title}


\begin{itemize}
\item Information should be current, and recent enough to make problem solving and decision making reasonable and reliable.In problem solving and decision making, all information is considered to have a life span for which it is appropriate.
\end{itemize}

\note[item]{}
\end{frame}
\begin{frame}
\frametitle{Unrelated Title}


\begin{itemize}
\item Information gathered should be relevant to the topic being researched and free from extraneous, cosmetic, and irrelevant details.It should be appropriate for the tasks of problem solving and decision making.
\end{itemize}

\note[item]{}
\end{frame}
\begin{frame}
\frametitle{Unrelated Title}


\begin{itemize}
\item Gathered information should be free from bias, i.e the information is objective and free from preconceptions.A reputable institution should support source without bias in the information.
\end{itemize}

\note[item]{}
\end{frame}
\begin{frame}
\frametitle{Unrelated Title}


\begin{itemize}
\item Verification is a process during which data that has already been input or captured is checked to ensure that it matches the data on the source document.
\end{itemize}

\note[item]{}
\end{frame}
\begin{frame}
\frametitle{Unrelated Title}


\begin{itemize}
\item A typographical error is a typing error that affects the text, such as missing or
additional characters. Examples are: Guyyana and Britsh Virgins Islanbs.
\end{itemize}

\note[item]{}
\end{frame}
\begin{frame}
\frametitle{Unrelated Title}


\begin{itemize}
\item A Transposition error is one caused by switching the position of two adjacent
characters in a number or text string. Examples are typing $5,450 instead of $5,540
or Gyuana instead of Guyana.
\end{itemize}

\note[item]{}
\end{frame}
\begin{frame}
\frametitle{Unrelated Title}


\begin{itemize}
\item Validation is a checking process in a program which is aimed finding out if the data is genuine.
\end{itemize}

\note[item]{}
\end{frame}
\begin{frame}
\frametitle{Unrelated Title}

\begin{center}
\includegraphics[width=0.9\textwidth,height=0.9\textheight,keepaspectratio]{/Users/I516998/Library/Application Support/Anki2/User 1/collection.media/paste-8290ee00c843d4194b67b1a5b832020bf437e461.jpg}
\end{center}


\note[item]{}
\end{frame}
\begin{frame}
\frametitle{Unrelated Title}


\begin{itemize}
\item The main diflerence between veritication and validation is that verification checks
the data being input to the system while validation authenticates the data once it is in the system. Verification is carried out by humans: validation is carried out by the computer.
\end{itemize}

\note[item]{}
\end{frame}
\begin{frame}
\frametitle{Unrelated Title}


\begin{itemize}
\item Double EntryProofreading
\end{itemize}

\note[item]{}
\end{frame}
\begin{frame}
\frametitle{Unrelated Title}

\begin{center}
\includegraphics[width=0.9\textwidth,height=0.9\textheight,keepaspectratio]{/Users/I516998/Library/Application Support/Anki2/User 1/collection.media/paste-9e22ccb145d0bfa741284deb2895a7da15b84374.jpg}
\end{center}


\note[item]{}
\end{frame}
\begin{frame}
\frametitle{Unrelated Title}

\begin{center}
\includegraphics[width=0.9\textwidth,height=0.9\textheight,keepaspectratio]{/Users/I516998/Library/Application Support/Anki2/User 1/collection.media/paste-10b592f737c51d3ce5af8bfce145ec1a53cf8558.jpg}
\end{center}


\note[item]{}
\end{frame}
\begin{frame}
\frametitle{Unrelated Title}


\begin{itemize}
\item Range CheckReasonableness CheckData Type CheckConsistency CheckPresence CheckFormat CheckLength Check
\end{itemize}

\note[item]{}
\end{frame}
\begin{frame}
\frametitle{Unrelated Title}


\begin{itemize}
\item A file is a contiainer in a computer system for storing data, information or programs,
Files usually exist permanenetly on a seconlary storage media. Disk drives (HDD
and SSD), USB pen drives and DVDs are all types of secondary storage media.
\end{itemize}

\note[item]{}
\end{frame}
\begin{frame}
\frametitle{Unrelated Title}


\begin{itemize}
\item 1. How data is stored in the file.
2. How files are stored on the storage medla.
\end{itemize}

\note[item]{}
\end{frame}
\begin{frame}
\frametitle{Unrelated Title}


\begin{itemize}
\item SerialSequentialRandom & Direct
\end{itemize}

\note[item]{}
\end{frame}
\begin{frame}
\frametitle{Unrelated Title}


\begin{itemize}
\item Serial access is where data can only be accessed in the order that it was written. This is usually because of the inherent nature of the storage medium for example a magnetic tape.
\end{itemize}

\note[item]{}
\end{frame}
\begin{frame}
\frametitle{Unrelated Title}


\begin{itemize}
\item Sequential access is a form of Serial access, when we organize our data in some order before storing the Data on the storage medium.
\end{itemize}

\note[item]{}
\end{frame}
\begin{frame}
\frametitle{Unrelated Title}


\begin{itemize}
\item Direct access is where the location of the data on the storage medium is known beforehand, thus accessing a logical file on a direct access medium does not require accessing any other data but that specific file.
\end{itemize}

\note[item]{}
\end{frame}
\begin{frame}
\frametitle{Unrelated Title}


\begin{itemize}
\item Backup is the copying of files to a separate removable storage device so that they can be restored to the original location if the original data is ever lost or destroyed.
\end{itemize}

\note[item]{}
\end{frame}
\begin{frame}
\frametitle{Unrelated Title}


\begin{itemize}
\item A copy of all the files on a computer system that can be used to restore the whole computer after a case of all hardware failure or data loss.
\end{itemize}

\note[item]{}
\end{frame}
\begin{frame}
\frametitle{Unrelated Title}


\begin{itemize}
\item An archive is a historical copy of information that is important to an organization. Archives can be used to look back on past decision making, or to gain a better understanding of an organization’s history.
\end{itemize}

\note[item]{}
\end{frame}
\begin{frame}
\frametitle{Unrelated Title}


\begin{itemize}
\item Direct access storage media is secondary storage where each file has a specific location or unique address in the storage, allowing it to be accessed quickly.
\end{itemize}

\note[item]{}
\end{frame}
\begin{frame}
\frametitle{Unrelated Title}


\begin{itemize}
\item A real-time system is a cormputer system where response time is critical. Examples; computer systems controlling sell-driving cars have to be real-time systems.
\end{itemize}

\note[item]{}
\end{frame}
\begin{frame}
\frametitle{Unrelated Title}


\begin{itemize}
\item A temporary area of memory where applications can store items for future use.
\end{itemize}

\note[item]{}
\end{frame}
\begin{frame}
\frametitle{Unrelated Title}


\begin{itemize}
\item Mailings or Mail Merge Wizard > Start Mail Merge > Letters
\end{itemize}

\note[item]{}
\end{frame}
\begin{frame}
\frametitle{Unrelated Title}


\begin{itemize}
\item A mail merge is the term used when a database of recipients and other fields are placed (or merged) onto a template letter to produce a mass mailing.Basically , a template document is created using a list of names.
\end{itemize}

\note[item]{}
\end{frame}
\begin{frame}
\frametitle{Unrelated Title}


\begin{itemize}
\item References > Table of Contents
\end{itemize}

\note[item]{}
\end{frame}
\begin{frame}
\frametitle{Unrelated Title}


\begin{itemize}
\item All headings must be definined as either a Heading 1, 2, 3 etc.
\end{itemize}

\note[item]{}
\end{frame}
\begin{frame}
\frametitle{Unrelated Title}


\begin{itemize}
\item File > Info > Protect Document > Encrypt with Password
\end{itemize}

\note[item]{}
\end{frame}
\begin{frame}
\frametitle{Unrelated Title}


\begin{itemize}
\item Developer Tab
\end{itemize}

\note[item]{}
\end{frame}
\begin{frame}
\frametitle{Unrelated Title}


\begin{itemize}
\item File > Options > Custom Ribbon > Developer
\end{itemize}

\note[item]{}
\end{frame}
\begin{frame}
\frametitle{Unrelated Title}


\begin{itemize}
\item Developer > Controls
\end{itemize}

\note[item]{}
\end{frame}
\begin{frame}
\frametitle{Unrelated Title}


\begin{itemize}
\item WWW: Teilmenge des Internets, ermöglicht Informationsaustausch, Textformatierungen und Grafikübertragungen
\item Teile des WWW: URL, HTML, CSS, HTTP
\item Internet: Verbund an Rechengeräten, das den Teilnehmern ermöglicht, sich auszutauschen und auf ein nahezu grenzenloses Archiv an Daten zuzugreifen
\end{itemize}

\note[item]{}
\end{frame}
\begin{frame}
\frametitle{Unrelated Title}


\begin{itemize}
\item Hypertext Markup Language
\item Informationen sollen nicht linear, sondern durch Querverweise erhältlich sein
\item Fokus auf Inhalt und Struktur
\end{itemize}

\note[item]{}
\end{frame}
\begin{frame}
\frametitle{Unrelated Title}

\begin{center}
\includegraphics[width=0.9\textwidth,height=0.9\textheight,keepaspectratio]{/Users/I516998/Library/Application Support/Anki2/User 1/collection.media/img6992079051075251177.jpg}
\end{center}


\note[item]{}
\end{frame}
\begin{frame}
\frametitle{Unrelated Title}


\begin{itemize}
\item IP Adresse (dynamisch): öffentliche IP Adresse wird vom Internetanbieter an Router vergeben, private IP bleibt verborgen
\item Private IP für Kommunikation innerhalb des internen Netzwerkes
\item V4, V6 
\item MAC Adresse = physische Adresse (statisch) an Netzwerkadapter gekoppelt
\end{itemize}

\note[item]{}
\end{frame}
\begin{frame}
\frametitle{Unrelated Title}


\begin{itemize}
\item Repeater: verstärkt das Wlan Signal der Basis (Router), halbiert aber meistens den Durchsatz, außer er hat 2 Funkmodule für Empfang und Senden
\item Netzwerk wird erst gewechselt bei Verbindungsabbruch oder Manuell
\item Whatsapp anruf
\item Z.t. unterschiedliche Namen der Netzwerke -> ungeeignet für Smart Home
\item Mesh: kommuniziert untereinander und stimmt ab, welcher Node welches Endgerät bedient -> keine schlechten Empfänge oder Verbindungsabbrüche
\end{itemize}

\note[item]{}
\end{frame}
\begin{frame}
\frametitle{Unrelated Title}


\begin{itemize}
\item Ausspähen des Browserverlsufs bei unverschlüsselten Websites
\item Angriffe auf Geräte im Netzwerk über Router
\item Malware im Router
\end{itemize}

\note[item]{}
\end{frame}
\begin{frame}
\frametitle{Unrelated Title}


\begin{itemize}
\item Hypertext transfer Protokol (Secure)
\item Wird genutzt, um Websiten vom Server in Webbrowser zu laden
\item Secure erfolgt über SSL Übertragungsprotokoll (Verschlüsselung)
\item Garantiert die Echtheit eines Servers durch ein Zertifikat
\item Zertifikate werden von Zertifizierungsstellen ausgegeben z.b. Global Sign
\end{itemize}

\note[item]{}
\end{frame}
\begin{frame}
\frametitle{Unrelated Title}


\begin{itemize}
\item Bandbreite
\item Keine Brandgefahr
\item Geringe Signaldämpfung
\item Abhörsicherer
\item Rohstoffsicherheit
\item Unempfindlich ggü Störungen
\end{itemize}

\note[item]{}
\end{frame}
\begin{frame}
\frametitle{Unrelated Title}


\begin{itemize}
\item Zwischen Router und Internet oder Funkmast und Anwender
\item System das Datenverkehr liest, analysiert und ggf blockt
\item Soll unerwünschte Systemzugriffe abwehren
\item Kann Software oder Hardware Komponente sein
\end{itemize}

\note[item]{}
\end{frame}
\begin{frame}
\frametitle{Unrelated Title}


\begin{itemize}
\item Indexierung: Analyse vom Text, Bild und Videodateien auf den Websiten und Speicherung in großer Datenbank (diese wird mit der Nutzung der Google Suche durchsucht)
\item Ranking: Sortierung nach Relevanz bei Suchanfragen ( Schlagwörter, Traffic etc.)
\item Keywords: Eingegebener Begriff in der Suchmaske -> wird durch Suchalgorythmen mit dem jeweiligen Index abgeglichen
\item Entscheident dafür, wie gut eine Website von google gefunden und geranked wird, da die Suche mit den Keywords auf der Website übereinstimmen muss bzw die Website so die Relevanz erhöht
\item Crawling: Crawler besuchen die Internetseiten über die Hyperlinks unf bereits indexierten Seiten. Der ausgelesene Inhalt wird dann ausgewertet und ebenfalls indexiert, wenn nicht vom Urheber verboten
\end{itemize}

\note[item]{}
\end{frame}
\begin{frame}
\frametitle{Unrelated Title}


\begin{itemize}
\item Search Engine Optimization -> Platzierung im Ranking verbessern
\item Wichtige Faktoren:
\item Anzahl Nutzer 
\item HTTPS
\item Optimierung für Mobilgeräte
\item Schnelle Ladezeiten
\item Keine doppelten Inhalte
\item Keywords
\item Nicht kaufbar
\end{itemize}

\note[item]{}
\end{frame}
\begin{frame}
\frametitle{Unrelated Title}


\begin{itemize}
\item Search Engine Advertising
\item Bezahlte Ads in Suchmaschinen
\item Kosten: pay per klick, kosten pro Keyword 
\item Strategien:
\item Dual Visibility: SEO und SEA zusammen optimieren, in dem Strategien aus den Analyseergebnissen des jeweils anderen genutzt werden. Langfristig sollen beide zu einer hohen Platzierung im Ranking kommen
\item Up and Down: SEA Anzeigen werden genutzt, um SEO zu verbessern, allerdings werden die SEA nicht aggressiv gepusht und somit hohe Kosten vermieden
\item Pullback: zuerst aggressives pushen der Anzeigen über SEA, dadurch nachziehen der SEO, dann erfolgt eine starke Reduzierung des Pushens von SEA und somit eine Kostenreduktion
\item SEO bleibt dennoch für eine gewisse Zeit in den Top Anzeigen, während SEA fällt 
\item Generell: sowohl SEO als auch SEA funktionieren mit Keywords
\item Während man aber bei SEA für eine gute Platzierung zahlen kann, geht das bei SEO nicht. Deshalb versucht man häufig, über SEA Erkenntnisse zu sammeln, mit denen man dann auch SEO optimieren kann
\end{itemize}

\note[item]{}
\end{frame}
\begin{frame}
\frametitle{Unrelated Title}


\begin{itemize}
\item Ein Kunde besucht eine Seite und zeigt Interesse an einem Produkt, kauft es aber nicht und verlässt die Seite wieder. 
\item Beim Retargeting bekommt er dann im späteren Verlauf eine Anzeige, die dieses Interesse wieder aufgreift mit dem Ziel, ihn erneut auf die Website zu lotsen und ihm das Produkt zu verkaufen
\end{itemize}

\note[item]{}
\end{frame}
\begin{frame}
\frametitle{Unrelated Title}


\begin{itemize}
\item Textdateien, die beim ersten betreten einer Website angelegt werden
\item Damit:
\item - können Websiten Rechner wiedererkennen und entsprechend Werbung schalten oder auch einen gespeicherten Warenkorb aufrufen
\item - kann nutzerseitige Einstellungen wie Sprache auf Websiten abgespeichert werden
\item - kann die Zeit, welche der Nutzer auf verschiedenen Seiten verbringt, getracked werden
\item Manche Cookies bleiben jahrelang bestehen, andere werden bei verlassen der Website gelöscht
\end{itemize}

\note[item]{}
\end{frame}
\begin{frame}
\frametitle{Unrelated Title}

\begin{center}
\includegraphics[width=0.9\textwidth,height=0.9\textheight,keepaspectratio]{/Users/I516998/Library/Application Support/Anki2/User 1/collection.media/img6871965820759583788.jpg}
\end{center}


\note[item]{}
\end{frame}
\begin{frame}
\frametitle{Unrelated Title}

\begin{center}
\includegraphics[width=0.9\textwidth,height=0.9\textheight,keepaspectratio]{/Users/I516998/Library/Application Support/Anki2/User 1/collection.media/img2071491201227969905.jpg}
\end{center}


\note[item]{}
\end{frame}
\begin{frame}
\frametitle{Unrelated Title}

\begin{center}
\includegraphics[width=0.9\textwidth,height=0.9\textheight,keepaspectratio]{/Users/I516998/Library/Application Support/Anki2/User 1/collection.media/img2725029130207736222.jpg}
\end{center}


\note[item]{}
\end{frame}
\begin{frame}
\frametitle{Unrelated Title}


\begin{itemize}
\item Geräte nicht sperren 
\item USB-Stick
\item Updaterisiko
\item E-Mail
\end{itemize}

\note[item]{}
\end{frame}
\begin{frame}
\frametitle{Unrelated Title}

\begin{center}
\includegraphics[width=0.9\textwidth,height=0.9\textheight,keepaspectratio]{/Users/I516998/Library/Application Support/Anki2/User 1/collection.media/img5278661368488018207.jpg}
\end{center}


\note[item]{}
\end{frame}
\begin{frame}
\frametitle{Unrelated Title}


\begin{itemize}
\item SPF Check (Sender Policy Framework) 
\item Server wird angefragt, um zu erfahren ob der Sender berechtigt ist, unter dem Namen Mails zu senden
\end{itemize}

\note[item]{}
\end{frame}
\begin{frame}
\frametitle{Unrelated Title}

\begin{center}
\includegraphics[width=0.9\textwidth,height=0.9\textheight,keepaspectratio]{/Users/I516998/Library/Application Support/Anki2/User 1/collection.media/img758409713893331183.jpg}
\end{center}


\note[item]{}
\end{frame}
\begin{frame}
\frametitle{Unrelated Title}

\begin{center}
\includegraphics[width=0.9\textwidth,height=0.9\textheight,keepaspectratio]{/Users/I516998/Library/Application Support/Anki2/User 1/collection.media/img4972201952102541415.jpg}
\end{center}


\note[item]{}
\end{frame}
\begin{frame}
\frametitle{Unrelated Title}


\begin{itemize}
\item Datenschutzgrundverordnung 2018
\item Soll Grundrecht und Grundfreiheiten natürlicher Personen schützen, insbesondere personenbezogene Daten
\item 7 Grundsätze:
\item Rechtmäßigkeit, Treu und Glauben, Transparenz
\item Zweckbindung
\item Datenminimierung
\item Richtigkeit
\item Speicherbegrenzung
\item Integrität und Vertraulichkeit
\item Rechenschaftspflicht
\end{itemize}

\note[item]{}
\end{frame}
\begin{frame}
\frametitle{Unrelated Title}


\begin{itemize}
\item Vorteile: 
\item - Höhere Glaubwürdigkeit 
\item - Vermeidung Kontaktabbruch
\item Sonderformen:
\item - Verbal Placement
\item - Negative Placement 
\item - Reverse Product Placement
\end{itemize}

\note[item]{}
\end{frame}
\begin{frame}
\frametitle{Unrelated Title}


\begin{itemize}
\item Finanzielle oder sachliche Unterstützung von Personen(gruppen), Veranstaltungen etc. 
\item Bsp:
\item Sportsponsoring
\item Kunstsponsoring
\item Soziosponsoring
\item Mediensponsoring
\item Vorteile:
\item Image Transfer
\item Motivation von Angestellten 
\item Steigerung Bekanntheitsgrad
\item Neukundengewinnung
\end{itemize}

\note[item]{}
\end{frame}
\begin{frame}
\frametitle{Unrelated Title}


\begin{itemize}
\item Online Anwendungen im WWW zu Kommunikationszwecken
\item Corporate Website
\item Micro Website
\item Werbebanner
\item Social Media
\item Advergames
\item SEA
\end{itemize}

\note[item]{}
\end{frame}
\begin{frame}
\frametitle{Unrelated Title}

\begin{center}
\includegraphics[width=0.9\textwidth,height=0.9\textheight,keepaspectratio]{/Users/I516998/Library/Application Support/Anki2/User 1/collection.media/img4791069336614457193.jpg}
\end{center}


\note[item]{}
\end{frame}
\begin{frame}
\frametitle{Unrelated Title}


\begin{itemize}
\item Unkonventionelle Kommunikationsmaßnahmen sollen große Aufmerksamkeit erzielen ( Verbreitung durch Mundpropaganda)
\item Unternehmen müssen beachten:
\item Wird Maßnahme der Marke zugeordnet?
\item Passt die Strategie zum Markenimage?
\item Spricht die Strategie die Zielgruppe an? 
\item Eignung des Mediums für integrierte Kommunikation?
\item Wie lässt sich Werbewirkung messen?
\end{itemize}

\note[item]{}
\end{frame}
\begin{frame}
\frametitle{Unrelated Title}

\begin{center}
\includegraphics[width=0.9\textwidth,height=0.9\textheight,keepaspectratio]{/Users/I516998/Library/Application Support/Anki2/User 1/collection.media/img5765838193858908159.jpg}
\end{center}


\note[item]{}
\end{frame}
\begin{frame}
\frametitle{Unrelated Title}


\begin{itemize}
\item Zeitlich befristete Maßnahmen mit Aktionscharakter
\item Sollen Absatz fördern
\item Handels-Promotions z.b. WKZ, Displays
\item Hersteller->Handel->Kunde
\item Hersteller Promotions z.b. Produktzugaben, Rückerstattungen
\item Hersteller->Kunde
\item Händler Promotions z.b. Beilagen, Sammelaktionen
\item Handel->Kunde
\end{itemize}

\note[item]{}
\end{frame}
\begin{frame}
\frametitle{Unrelated Title}


\begin{itemize}
\item Von UN initiierte Erlebniswelten.
\item Sollen Einzigartigkeit der Marke kommunizieren -> Kundenbindung
\item Bsp Swarovski - Kristallwelten
\item Erfolgsfaktoren:
\item - Standort und Anbindung, inhaltlicher Bezug des Standortes zu Marke
\item - Einzigartigkeit
\item - Etablierung Wahrzeichen (meist Architektur)
\item - Hauptattraktion 
\item - Interaktivität 
\item - Familienattraktivität
\end{itemize}

\note[item]{}
\end{frame}
\begin{frame}
\frametitle{Unrelated Title}


\begin{itemize}
\item Qualität - Inhalt der Emotion ( Freude, Wut)
\item Intensität - Stärke der Ausprägung des psychischen Zustands
\item Dauer - Dauer der psychischen Zustands
\item Reizgeneriertheit - Reiz, der die Emotionen auslöst ( bsp Angst vor etwas)
\item Physiologische Veränderung - messbare körperliche Veränderung
\item Verhaltensaspekt - Emotionsspezifische Verhaltensweise
\end{itemize}

\note[item]{}
\end{frame}
\begin{frame}
\frametitle{Unrelated Title}


\begin{itemize}
\item Positive
\item Negative
\item Mixed
\end{itemize}

\note[item]{}
\end{frame}
\begin{frame}
\frametitle{Unrelated Title}


\begin{itemize}
\item Auf subjektiver Ebene -> verbale Ratingskalen, Programmanalysator
\item Auf motorischer Ebene -> GfK Emo Scan
\item Auf physiologischer Ebene -> Magnetresonanztomograph
\end{itemize}

\note[item]{}
\end{frame}
\begin{frame}
\frametitle{Unrelated Title}


\begin{itemize}
\item 3/6:
\item Rationalisierungsdruck (Preisdruck v. Kunden)
\item Globalisierung der Märkte
\item Buying Center (mehrere Abteilungen müssen überzeugt werden)
\end{itemize}

\note[item]{}
\end{frame}
\begin{frame}
\frametitle{Unrelated Title}


\begin{itemize}
\item Bonität
\item Großkunden (80:20, ABC)
\item Entwicklungskunden
\item Hoher DB
\item Marktbedeutung
\item Imageführer
\item Know-How Träger
\end{itemize}

\note[item]{}
\end{frame}
\begin{frame}
\frametitle{Unrelated Title}

\begin{center}
\includegraphics[width=0.9\textwidth,height=0.9\textheight,keepaspectratio]{/Users/I516998/Library/Application Support/Anki2/User 1/collection.media/img603450184796109488.jpg}
\end{center}


\note[item]{}
\end{frame}
\begin{frame}
\frametitle{Unrelated Title}

\begin{center}
\includegraphics[width=0.9\textwidth,height=0.9\textheight,keepaspectratio]{/Users/I516998/Library/Application Support/Anki2/User 1/collection.media/img8273403634246863073.jpg}
\end{center}


\note[item]{}
\end{frame}
\begin{frame}
\frametitle{Unrelated Title}

\begin{center}
\includegraphics[width=0.9\textwidth,height=0.9\textheight,keepaspectratio]{/Users/I516998/Library/Application Support/Anki2/User 1/collection.media/img4602150364689668104.jpg}
\end{center}


\note[item]{}
\end{frame}
\begin{frame}
\frametitle{Unrelated Title}

\begin{center}
\includegraphics[width=0.9\textwidth,height=0.9\textheight,keepaspectratio]{/Users/I516998/Library/Application Support/Anki2/User 1/collection.media/img8509426281760944356.jpg}
\end{center}


\note[item]{}
\end{frame}
\begin{frame}
\frametitle{Unrelated Title}

\begin{center}
\includegraphics[width=0.9\textwidth,height=0.9\textheight,keepaspectratio]{/Users/I516998/Library/Application Support/Anki2/User 1/collection.media/img9039724954348666440.jpg}
\end{center}


\note[item]{}
\end{frame}
\begin{frame}
\frametitle{Unrelated Title}

\begin{center}
\includegraphics[width=0.9\textwidth,height=0.9\textheight,keepaspectratio]{/Users/I516998/Library/Application Support/Anki2/User 1/collection.media/img6042410958402642054.jpg}
\end{center}


\note[item]{}
\end{frame}
\begin{frame}
\frametitle{Unrelated Title}


\begin{itemize}
\item Grundsätze der Unternehmensführung unter Beachtung der geltenden Gesetze, Unternehmensricht- und leitlinien sowie Interessen der Stake- und Shareholder
\item 3 Bereiche:
\item - Unternehmensorganisation
\item - Risikomanagement
\item - Compliance Management
\item Zunehmende Bedeutung aufgrund von Globalisierung der Wirtschaft, Liberalisierung der Kapitalmärkte
\item Internes Kontrollsystem nötig
\item GRC-Ansatz Governance, Compliance, Risiko
\end{itemize}

\note[item]{}
\end{frame}
\begin{frame}
\frametitle{Unrelated Title}


\begin{itemize}
\item Verantwortlichkeiten der einzelnen Unternehmenseinheiten exakt festlegen
\item Aufbau und Ablauf (Arbeitsprozesse) 
\item Innerhalb und zwischen Abteilungen
\end{itemize}

\note[item]{}
\end{frame}
\begin{frame}
\frametitle{Unrelated Title}


\begin{itemize}
\item Ökonomische Risiken erkennen, analysieren und bewerten -> Maßnahmen ableiten -> Nachteile vermeiden 
\item VUCA
\end{itemize}

\note[item]{}
\end{frame}
\begin{frame}
\frametitle{Unrelated Title}


\begin{itemize}
\item Rechtliche und ethische Risiken im Blick haben, analysieren und bewerten -> geeignete Maßnahmen ableiten
\item Rechtsrisiken resultieren meistens in ökonomischen Risiken -> Compliance-Management kann ohne Risikomanagement nicht bestehen! 
\item Generell Interdependenzen der 3 Bereiche -> Probleme und Lösungen verstricken sich häufig mit den anderen Bereichen 
\item -> integrierte Betrachtung der 3 Bereiche
\item Vorstand muss für Einhaltung der gesetzlichen Bestimmungen sorgen und Maßnahmen offenlegen. Beschäftigte und Dritte müssen die Möglichkeit haben, Hinweise auf Rechtsverstöße im Unternehmen zu geben
\item Funktionen von Compliance für Unternehmen:
\item 1. Schutzfunktion (Haftung)
\item 2. Image- Reputationsfunktion
\item 3. Beratungs- und Informationsfunktion
\item 4. Qualitätssicherung und Innovation
\item 5. Überwachungsfunktion
\item Wichtige Bereiche im UN für Compliance:
\item Bsp: Steuern, IT, Datenschutz, Mitarbeiter....
\end{itemize}

\note[item]{}
\end{frame}
\begin{frame}
\frametitle{Unrelated Title}


\begin{itemize}
\item Entwicklung unternehmensinterner Richtlinien
\item Schulung Aufsichtsrat
\item Beratung Geschäftsleitung 
\item Berichterstattung über Teilsysteme des GRCs
\end{itemize}

\note[item]{}
\end{frame}
\begin{frame}
\frametitle{Unrelated Title}


\begin{itemize}
\item Bei Instituten im Risikomanagement
\item Wertpapierdienstleistungsunternehmen nicht direkt, nur organisatorische Pflichten
\item Versicherungsunternehmen ausdrücklich mit konkreten Angaben
\item Freiwillig ergibt Compliance schon aus Selbstschutz Sinn! 
\end{itemize}

\note[item]{}
\end{frame}
\begin{frame}
\frametitle{Unrelated Title}


\begin{itemize}
\item Aufstellen von Ethikregeln
\item Definition des Verhaltens bei Behördenbesuchen
\item Ausgestaltung der Beschaffungsrichtlinien
\item Incentive-Richtlinie (Umgang mit Einladungen und Geschenken)
\item Kommunikations und Spendenrichtlinie
\end{itemize}

\note[item]{}
\end{frame}
\begin{frame}
\frametitle{Unrelated Title}


\begin{itemize}
\item Internes Kontrollsystem:
\item Die Gesamtheit aller von der Unternehmensleitung angeordneten
\item - Vorgänge
\item - Methoden
\item - Maßnahmen
\item die dazu dienen, den ordnungsgemäßen und rechtskonformen Ablauf im Betrieb zu sichern
\item Unternehmesorga
\item Risikomanagement
\item Compliance-Management
\item Interne Revision
\end{itemize}

\note[item]{}
\end{frame}
\begin{frame}
\frametitle{Unrelated Title}


\begin{itemize}
\item Kontrolle der Einhaltung von:
\item Zielen
\item Vorgaben 
\item Vorstellungen der Unternehmensführung 
\item Formelle und materielle Ordnungsmäßigkeit
\item Einhaltung von Richtlinien unf Anweisungen
\item Berücksichtigung Wirtschaftlichkeit
\item Implementierung Best Practise
\item -> Mehrwert schaffen und Prozesse verbessern, Unterstützung bei der Zielerreichung, Unabhängig und objektiv
\item -> Systemgeber haften wenn der Prozess nicht passt, da Überwachungsfunktion Aufgabe der UN-Führung
\item Herausforderungen für GF wenn sie die Revision selbst übernehmen müssten: 
\item Konzentration auf die gestaltende Tätigkeit
\item Räumliche und entscheidungsmäßige Dezentralisation
\item Komplexität der Arbeitsabläufe 
\item -> bei kleinen Unternehmen machbar, bei mittleren schwierig -> Umbruch zum Aufbau einer Revision sinnvoll oder Externalisierung um Übersicht und Kontrolle zu behalten
\item Deutsche Unternehmen müssen keine interne Revisionsabteilung haben
\item Kreditinstitute schon
\end{itemize}

\note[item]{}
\end{frame}
\begin{frame}
\frametitle{Unrelated Title}

\begin{center}
\includegraphics[width=0.9\textwidth,height=0.9\textheight,keepaspectratio]{/Users/I516998/Library/Application Support/Anki2/User 1/collection.media/img1878190916869420365.jpg}
\end{center}

\begin{itemize}
\item Mehrjahresplanung -> Risikoorientiert
\item Prüfungsvorbereitung:
\item -> mit Standort Kontakt aufnehmen, Prüfungshinweise sammeln wie Richtlinien, Auswertungen, ZDF
\item ->Prüfungstermin abstimmen
\item -> Lastenheft
\item Prüfungsdurchführung: 
\item -> Kick-Off Meeting, terminliche Abstimmung, Ablaufplan
\item -> Lastenheft Top-Downs abarbeiten - Top Management bis Mitarbeiter
\item -> Soll-Ist Abgleich
\item Prüfungsinterviews
\item Berichterstellung:
\item -> kurze Zusammenfassung zu Beginn mit Ursache - Wirkung Prinzip -> mit Referenzen für Details
\item -> Handlungsempfehlung
\item Follow-UP
\item - nachhalten zur Umsetzung
\item Ca 1-2 Jahre
\end{itemize}

\note[item]{}
\end{frame}
\begin{frame}
\frametitle{Unrelated Title}


\begin{itemize}
\item Definition:
\item Technische unf systematische Regeln des Steuerns und Kontrollen zum Einhalten von Richtlinien und Abwehr von Schäden 
\item Unterscheidung von Kontrollen:
\item Nach Art der Durchführung 
\item Nach dem Zyklus
\item Detektiv oder präventiv
\item Vier Prinzipien des IKS
\item Transparenz
\item 4 Augen Prinzip 
\item Funktionstrennung
\item Mindestinformation
\item Ziele:
\item Funktionsfähigkeit und Wirtschaftlichkeit von Geschäftsprozessen
\item Zuverlässigkeit von betrieblichen Informationen
\item Vermögenssicherung
\item Regeleinhaltung
\end{itemize}

\note[item]{}
\end{frame}
\begin{frame}
\frametitle{Unrelated Title}


\begin{itemize}
\item Dimensionen: 
\item Leistungen für Key Accounts
\item Prozesse&Spielregeln
\item Key Account Manager
\item Definition Spielregeln
\item Steuerung 
\item Werkzeuge
\item Organisation & KA Teams
\item Key Account Identifikation
\end{itemize}

\note[item]{}
\end{frame}
\begin{frame}
\frametitle{Unrelated Title}


\begin{itemize}
\item Komplexere Kunden 
\item Vergleichbarere Produkte
\item Ein- und Verkauf ist Teamsache
\item Pareto bei Wichtigkeit der Unternehmen
\item Steigende Erwartungshaltung der Kunden 
\end{itemize}

\note[item]{}
\end{frame}
\begin{frame}
\frametitle{Unrelated Title}

\begin{center}
\includegraphics[width=0.9\textwidth,height=0.9\textheight,keepaspectratio]{/Users/I516998/Library/Application Support/Anki2/User 1/collection.media/img1058273733205024411.jpg}
\end{center}


\note[item]{}
\end{frame}
\begin{frame}
\frametitle{Unrelated Title}

\begin{center}
\includegraphics[width=0.9\textwidth,height=0.9\textheight,keepaspectratio]{/Users/I516998/Library/Application Support/Anki2/User 1/collection.media/img50794720308936637.jpg}
\end{center}


\note[item]{}
\end{frame}
\begin{frame}
\frametitle{Unrelated Title}


\begin{itemize}
\item Sender - Message - Medium - Receiver
\end{itemize}

\note[item]{}
\end{frame}
\begin{frame}
\frametitle{Unrelated Title}


\begin{itemize}
\item Abschotten
\item Umgehen (aufsichen werbereduziertes Umfeld)
\item Vereinfachen (nur Schlüsselinformationen)
\item Picken (grober Überblick zu Infoangebot und vertiefen was interessant ist)
\end{itemize}

\note[item]{}
\end{frame}
\begin{frame}
\frametitle{Unrelated Title}


\begin{itemize}
\item Emotion
\item Information
\item Emotion und Information
\item Aktualisierung
\end{itemize}

\note[item]{}
\end{frame}
\begin{frame}
\frametitle{Unrelated Title}

\begin{center}
\includegraphics[width=0.9\textwidth,height=0.9\textheight,keepaspectratio]{/Users/I516998/Library/Application Support/Anki2/User 1/collection.media/img9139203124758012556.jpg}
\end{center}


\note[item]{}
\end{frame}
\begin{frame}
\frametitle{Unrelated Title}


\begin{itemize}
\item Systematische Anwendung von verhaltenswissenschaftlichen Gesetzmäßigkeiten zur Beeinflussung von Menschen 
\item Wirksamer Einsatz von Sozialtechniken
\item Kontakt herstellen
\item Verständnis erreichen
\item Emotionen vermitteln
\item Gedächtnis verankern
\end{itemize}

\note[item]{}
\end{frame}
\begin{frame}
\frametitle{Unrelated Title}


\begin{itemize}
\item 1. Frequenztechniken
\item 2. Aktivierungstechniken
\end{itemize}

\note[item]{}
\end{frame}
\begin{frame}
\frametitle{Unrelated Title}

\begin{center}
\includegraphics[width=0.9\textwidth,height=0.9\textheight,keepaspectratio]{/Users/I516998/Library/Application Support/Anki2/User 1/collection.media/img575149739844665286.jpg}
\end{center}


\note[item]{}
\end{frame}
\begin{frame}
\frametitle{Unrelated Title}


\begin{itemize}
\item Beeinflussungsziele:
\item Information
\item emotion
\item Information und Emotion
\item Aktialisierung
\item Aktivierungstechniken:
\item Physisch intensiv
\item Emotional
\item Kognitiv überraschend
\end{itemize}

\note[item]{}
\end{frame}
\begin{frame}
\frametitle{Unrelated Title}


\begin{itemize}
\item Vampir Effekt -> Ablenkung von der eigentlichen Werbebotschaft
\item Bumerang Effekt -> Botschaft entspricht nicht dem Werbeziel
\item Irritation -> Erzeugung eines Gefühls der Störung und Verunsicherung
\end{itemize}

\note[item]{}
\end{frame}
\begin{frame}
\frametitle{Unrelated Title}

\begin{center}
\includegraphics[width=0.9\textwidth,height=0.9\textheight,keepaspectratio]{/Users/I516998/Library/Application Support/Anki2/User 1/collection.media/img6917240902685262185.jpg}
\end{center}


\note[item]{}
\end{frame}
\begin{frame}
\frametitle{Unrelated Title}

\begin{center}
\includegraphics[width=0.9\textwidth,height=0.9\textheight,keepaspectratio]{/Users/I516998/Library/Application Support/Anki2/User 1/collection.media/img6600421730944679943.jpg}
\end{center}


\note[item]{}
\end{frame}
\begin{frame}
\frametitle{Unrelated Title}


\begin{itemize}
\item Direkt -> technischer Nutzen wird gezeigt
\item Kann auch Side by Side Vergleich sein
\item Indirekt -> Assoziation (Produkt+emotionaler Nutzen)oder Analogie(Produkt+technische Eigenschaft) wird hervorgerufen
\end{itemize}

\note[item]{}
\end{frame}
\begin{frame}
\frametitle{Unrelated Title}

\begin{center}
\includegraphics[width=0.9\textwidth,height=0.9\textheight,keepaspectratio]{/Users/I516998/Library/Application Support/Anki2/User 1/collection.media/img8619437816587693926.jpg}
\end{center}


\note[item]{}
\end{frame}
\begin{frame}
\frametitle{Unrelated Title}

\begin{center}
\includegraphics[width=0.9\textwidth,height=0.9\textheight,keepaspectratio]{/Users/I516998/Library/Application Support/Anki2/User 1/collection.media/img2795973069000002824.jpg}
\end{center}


\note[item]{}
\end{frame}
\begin{frame}
\frametitle{Unrelated Title}

\begin{center}
\includegraphics[width=0.9\textwidth,height=0.9\textheight,keepaspectratio]{/Users/I516998/Library/Application Support/Anki2/User 1/collection.media/img5269568806252710749.jpg}
\end{center}


\note[item]{}
\end{frame}
\begin{frame}
\frametitle{Unrelated Title}


\begin{itemize}
\item Cascading Style Sheet
\item Fokussiert sich auf Darstellung auf HTML Seiten
\end{itemize}

\note[item]{}
\end{frame}
\begin{frame}
\frametitle{Unrelated Title}

\begin{center}
\includegraphics[width=0.9\textwidth,height=0.9\textheight,keepaspectratio]{/Users/I516998/Library/Application Support/Anki2/User 1/collection.media/paste-487dca53e63a7fd06f0d57a0e85f4e0f70f94606.jpg}
\end{center}


\note[item]{}
\end{frame}
\begin{frame}
\frametitle{Unrelated Title}

\begin{center}
\includegraphics[width=0.9\textwidth,height=0.9\textheight,keepaspectratio]{/Users/I516998/Library/Application Support/Anki2/User 1/collection.media/paste-18988a7fec30508c949de7b3c9b02d37297e4b52.jpg}
\end{center}


\note[item]{}
\end{frame}
\begin{frame}
\frametitle{Unrelated Title}


\begin{itemize}
\item Web page layout is the positioning and sizing of text and images on the web page.
\end{itemize}

\note[item]{}
\end{frame}
\begin{frame}
\frametitle{Unrelated Title}


\begin{itemize}
\item A website theme is the use of common elements such as page layout. navigation, hyperlinks, colours and pictures on every web page.
\end{itemize}

\note[item]{}
\end{frame}
\begin{frame}
\frametitle{Unrelated Title}


\begin{itemize}
\item The website's content (all the .html files and the associated images)A domain nameA web-hosting service
\end{itemize}

\note[item]{}
\end{frame}
\begin{frame}
\frametitle{Unrelated Title}


\begin{itemize}
\item Hosting is the act or making a website available on the World Wide Web.
\end{itemize}

\note[item]{}
\end{frame}
\begin{frame}
\frametitle{Unrelated Title}


\begin{itemize}
\item Before publishing a website, it is important that you ensure that:
\item • the content is complete. accurate and up to date
\item • the navigation and hyperlinks work correctly
\item • the pages display effectively on a range of devices and platforms.
\end{itemize}

\note[item]{}
\end{frame}
\begin{frame}
\frametitle{Unrelated Title}


\begin{itemize}
\item how the website looks if the viewer is using an older/newer version of your web browserhow the website looks if the viewer is using a different browserhow the website looks when viewed on a tablet or smart mobile phonehow the website looks to the visually impaired or colour-blind.
\end{itemize}

\note[item]{}
\end{frame}
\begin{frame}
\frametitle{Unrelated Title}


\begin{itemize}
\item Website maintenance is an integral part of updating and keeping your site working properly. When properly maintained a website will:
\item • be free of broken hyperlinks
\item • provide current and relevant content to its users
\item • have an information section at the bottom showing the date the page was last updated and the details of the authors or organisation.
\item It is important that you regularly check and update your website. Information can
\item soon become out of date.
\end{itemize}

\note[item]{}
\end{frame}
\begin{frame}
\frametitle{Unrelated Title}


\begin{itemize}
\item A Computer Network is an interconnection of 2 or more computers to enable the sharing of
resources and communication. Resources include files, printers and specialized network services e.g a
web server or a chat server.
\end{itemize}

\note[item]{}
\end{frame}
\begin{frame}
\frametitle{Unrelated Title}


\begin{itemize}
\item A local area network (LAN) is an interconnection of a group of computers within a limited geographical
location (a room or building). E.g A school’s computer lab.
\end{itemize}

\note[item]{}
\end{frame}
\begin{frame}
\frametitle{Unrelated Title}


\begin{itemize}
\item A wide area network (WAN) spans large geographical locations , it can be implemented as many
computers connected over large distances interconnected with other LANs. The Internet is an example
of a WAN.
\end{itemize}

\note[item]{}
\end{frame}
\begin{frame}
\frametitle{Unrelated Title}


\begin{itemize}
\item A metropolitan area network (MAN) is a network that interconnects users with computer resources in a
geographic area or region larger than that covered by even a large local area network (LAN) but smaller
than the area covered by a wide area network (WAN). 
\end{itemize}

\note[item]{}
\end{frame}
\begin{frame}
\frametitle{Unrelated Title}


\begin{itemize}
\item The mechanisms by which computing devices are connected together. They can be wired or wireless.
\end{itemize}

\note[item]{}
\end{frame}
\begin{frame}
\frametitle{Unrelated Title}


\begin{itemize}
\item twisted pair, coaxial, fibre
\end{itemize}

\note[item]{}
\end{frame}
\begin{frame}
\frametitle{Unrelated Title}


\begin{itemize}
\item infrared, microwave, satellite
\end{itemize}

\note[item]{}
\end{frame}
\begin{frame}
\frametitle{Unrelated Title}


\begin{itemize}
\item Conists of 8 cooper wires, twisted into pairs to reduce interference and enclosed in a plastic covering.
\end{itemize}

\note[item]{}
\end{frame}
\begin{frame}
\frametitle{Unrelated Title}


\begin{itemize}
\item Consists of a single copper core surrounded by a metal sheild to educe interference and all enclosed in a plastic covering.
\end{itemize}

\note[item]{}
\end{frame}
\begin{frame}
\frametitle{Unrelated Title}


\begin{itemize}
\item Consists of flexible fibres with a glass core; they carry information using light.
\end{itemize}

\note[item]{}
\end{frame}
\begin{frame}
\frametitle{Unrelated Title}


\begin{itemize}
\item Part of the electromagnetic radiation spectrum at wavelengths that fall between visible ligh and radio waves at frequencies 300GHz up to 430THz.
\item In terms of Information Technology, infrared communication works in the same way as Wi-Fi but at a different frequency. An example of infrared transmission is communication between a desktop and netbook computer where the distance is small and there is a clear line of sight between the two devices. The requirement for a clear line of sight between devices has limited the popularity of infrared communication.
\end{itemize}

\note[item]{}
\end{frame}
\begin{frame}
\frametitle{Unrelated Title}


\begin{itemize}
\item Microwaves are part of the electromagnetic radiation spectrum at frequencie, 300MHz to 300GHz.
\item Like infrared, microwave communication works in the same way as Wi-F. The range of microwave transmission is many kilometres but there has to be a clear line of sight between the two antennae. Microwave communications can be used to connect remote locations to a network providing there is a clear line of sight: for example, a hilltop weather station may be networked to the valley-bottom laboratory using microwaves.
\end{itemize}

\note[item]{}
\end{frame}
\begin{frame}
\frametitle{Unrelated Title}


\begin{itemize}
\item A satellite is a communication device orbiting in space that relays and amplifie
\item radio telecommunication signals to and from remote locations.
\end{itemize}

\note[item]{}
\end{frame}
\begin{frame}
\frametitle{Unrelated Title}

\begin{center}
\includegraphics[width=0.9\textwidth,height=0.9\textheight,keepaspectratio]{/Users/I516998/Library/Application Support/Anki2/User 1/collection.media/paste-7dc32d06a553a564445009d6500ca9e0ec91939c.jpg}
\end{center}


\note[item]{}
\end{frame}
\begin{frame}
\frametitle{Unrelated Title}


\begin{itemize}
\item A switch is an intelligent device that routes incoming data directly to the specific
\item output port that will take the data towards its intended destination. Instead of blindly forwarding all the frames it receives on one port to all the other ports on the device, a
\item switch will create a MAC address source table and then forward the frame to the port with the correct
\item destination MAC address. This significantly reduces the amount of traffic on the network because there
\item is direct communication between the two devices rather than a one-to-all type of communication.
\end{itemize}

\note[item]{}
\end{frame}
\begin{frame}
\frametitle{Unrelated Title}


\begin{itemize}
\item Hubs are considered Layer 1 (Physical) devices whereas switches are put into Layer 2 (Data Link). This is
where hubs and switches differ. The Data Link layer of the OSI model deals with MAC addresses and
switches look at MAC addresses when they process an incoming frame on a port.
\end{itemize}

\note[item]{}
\end{frame}
\begin{frame}
\frametitle{Unrelated Title}


\begin{itemize}
\item A frame is a data type that is used to carry data on all networking devices, it contains source and destination MAC addresses and source and
destination IP addresses inside the frame. 
\end{itemize}

\note[item]{}
\end{frame}
\begin{frame}
\frametitle{Unrelated Title}


\begin{itemize}
\item A router is an intelligcnt network dcvice that connects two networks together. Routers work at Layer 3 (Network) of the OSI model, which deals with IP addresses.
\end{itemize}

\note[item]{}
\end{frame}
\begin{frame}
\frametitle{Unrelated Title}


\begin{itemize}
\item Modem is an abbreviation for modulator-demodulator, a device that converts
\item signals from analogue to digital and vice versa. A modem allows computers to
\item exchange information through telephone lines.
\end{itemize}

\note[item]{}
\end{frame}
\begin{frame}
\frametitle{Unrelated Title}


\begin{itemize}
\item A network interrace card (NIC) is a computer hardware component installed in a
\item device into which a network cable may be plugged.
\end{itemize}

\note[item]{}
\end{frame}
\begin{frame}
\frametitle{Unrelated Title}


\begin{itemize}
\item A network adapter is a converter that adapts/extends a USB port in order for it to
\item function like a NIC or a wireless NIC.
\end{itemize}

\note[item]{}
\end{frame}
\begin{frame}
\frametitle{Unrelated Title}


\begin{itemize}
\item A common carrier is an entity that provides wired and wireless communication services to the general
public for a fee. A common carrier can be contrasted with a contract carrier, also called a private carrier,
which provides services to a limited number of customers.
\end{itemize}

\note[item]{}
\end{frame}
\begin{frame}
\frametitle{Unrelated Title}


\begin{itemize}
\item Bluetooth is a wireless mobile standard that allows for personal area networking. Personal area
networks are meant to allow for a user’s devices to connect over short distances (~10 m) to a mobile
device.
\end{itemize}

\note[item]{}
\end{frame}
\begin{frame}
\frametitle{Unrelated Title}


\begin{itemize}
\item Wi-Fi is technology for radio wireless local area networking of devices based on the IEEE 802.11
standards.
\end{itemize}

\note[item]{}
\end{frame}
\begin{frame}
\frametitle{Unrelated Title}


\begin{itemize}
\item The Internet is a public, global (wide area) network based on the TCP/
\item IP protocol. The TCP/IP protocol assigns every connected computer a unique
\item Internet address, also called an IP address, so that any two connected
\item computers can locate each other on the network and locate data.
\end{itemize}

\note[item]{}
\end{frame}
\begin{frame}
\frametitle{Unrelated Title}


\begin{itemize}
\item An intranet is a private computer network designed to meet the needs of a
\item single organisation or company that is based on Internet (TCP/IP) technology.
\item It is not necessarily open to the external Internet and definitely not open to
\item outside users. It utilises familiar facilities such as web pages and web browsers.
\end{itemize}

\note[item]{}
\end{frame}
\begin{frame}
\frametitle{Unrelated Title}


\begin{itemize}
\item An extranet is an intranet that has been selectively opened to specially
\item selected individuals or organisations (including customers, suppliers, research
\item associates). An online banking application is an example of an extranet.
\end{itemize}

\note[item]{}
\end{frame}
\begin{frame}
\frametitle{Unrelated Title}


\begin{itemize}
\item The WWW is an information system on the Internet which allows documents to be connected to other
documents by hypertext links, enabling the user to search for information by moving from one
document to another
\end{itemize}

\note[item]{}
\end{frame}
\begin{frame}
\frametitle{Unrelated Title}


\begin{itemize}
\item Hypertext Markup Language, a standardized system for tagging text files to achieve font, colour, graphic,
and hyperlink effects on World Wide Web pages.
\end{itemize}

\note[item]{}
\end{frame}
\begin{frame}
\frametitle{Unrelated Title}


\begin{itemize}
\item HTTP means HyperText Transfer Protocol. HTTP is the underlying protocol used by the World Wide Web
and this protocol defines how messages are formatted and transmitted, and what actions Web servers
and browsers should take in response to various commands.
For example, when you enter a URL in your browser, this actually sends an HTTP command to the Web
server directing it to fetch and transmit the requested Web page.
\end{itemize}

\note[item]{}
\end{frame}
\begin{frame}
\frametitle{Unrelated Title}


\begin{itemize}
\item A hyperlink is an icon, information object, underlined or otherwise emphasized word or phrase that
displays another document ( or resource) when clicked with the mouse.
\end{itemize}

\note[item]{}
\end{frame}
\begin{frame}
\frametitle{Unrelated Title}


\begin{itemize}
\item A web server is special-purpose application software that accepts requests web browser
for information, framed according to the Hyper Text Transport Protocol (HTTP) , processes these
requests and sends the requested document.
\end{itemize}

\note[item]{}
\end{frame}
\begin{frame}
\frametitle{Unrelated Title}


\begin{itemize}
\item A web page is a document, written in Hyper Text Markup Language (HTML), that may contain text,
sound, images, video clips, hyperlinks and other components.
\end{itemize}

\note[item]{}
\end{frame}
\begin{frame}
\frametitle{Unrelated Title}


\begin{itemize}
\item A web browser is special-purpose application software that runs on an
\item Internet- connected computer and uses the HTTP to connect with web servers.
\item All web browsers can decode web pages that have been written (marked) with
\item HTML. A web browser is needed for a web page to be requested, downloaded,
\item decoded and displayed on a user's local machine. The most common web
\item browsers are Internet Explorer. Firefox. Chrome, Safari and Opera.
\end{itemize}

\note[item]{}
\end{frame}
\begin{frame}
\frametitle{Unrelated Title}


\begin{itemize}
\item A uniform resource locator (URL) is a string of characters that uniquely
\item identifies an Internet resource's type and location.
\end{itemize}

\note[item]{}
\end{frame}
\begin{frame}
\frametitle{Unrelated Title}


\begin{itemize}
\item File Transfer Protocol (FTP) is a protocol used to transfer files between
\item FTP servers and clients. An FTP site is like a large filing cabinet.
\end{itemize}

\note[item]{}
\end{frame}
\begin{frame}
\frametitle{Unrelated Title}


\begin{itemize}
\item Uploading means data is being sent from your computer to the Internet.
\end{itemize}

\note[item]{}
\end{frame}
\begin{frame}
\frametitle{Unrelated Title}


\begin{itemize}
\item Downloading means your computer is receiving data from the Internet.
\end{itemize}

\note[item]{}
\end{frame}
\begin{frame}
\frametitle{Unrelated Title}


\begin{itemize}
\item Computer misuse is considered to include:
\item Acts which are likely to cause unauthorized modifcation, removal, or copying of the contents of any computer system.Directly or indirectly obtaining computer services without the proper authorization.Accessing programs or data on a computer with the intent to comit a crime.Unauthorized access to a computer system.
\end{itemize}

\note[item]{}
\end{frame}
\begin{frame}
\frametitle{Unrelated Title}


\begin{itemize}
\item Cyberbullying is a form of bullying or harassment using computer-based communications.
\end{itemize}

\note[item]{}
\end{frame}
\begin{frame}
\frametitle{Unrelated Title}


\begin{itemize}
\item The deliberate use of someone else's identity usually to gain a financial advantage.
\end{itemize}

\note[item]{}
\end{frame}
\begin{frame}
\frametitle{Unrelated Title}


\begin{itemize}
\item • false applications Ior loans and credit cards
\item • fraudulent withdrawals from bank accounts
\item • fraudulent use of online accounts
\item • fraudulently obtaining other goods or scrviccs.
\end{itemize}

\note[item]{}
\end{frame}
\begin{frame}
\frametitle{Unrelated Title}


\begin{itemize}
\item Obscene material is material of a sexual nature or material that offends against
society's morality.
\end{itemize}

\note[item]{}
\end{frame}
\begin{frame}
\frametitle{Unrelated Title}


\begin{itemize}
\item Phishing is the attempt to obtain sensitive information such as usernames
passwords and credit card details by sending emails pretending to be from a
legitimate organisation.
\end{itemize}

\note[item]{}
\end{frame}
\begin{frame}
\frametitle{Unrelated Title}

\begin{center}
\includegraphics[width=0.9\textwidth,height=0.9\textheight,keepaspectratio]{/Users/I516998/Library/Application Support/Anki2/User 1/collection.media/paste-b7e6157f77ac2f4c786fcddcb08d78157af7d38d.jpg}
\end{center}


\note[item]{}
\end{frame}
\begin{frame}
\frametitle{Unrelated Title}


\begin{itemize}
\item Privacy is the right of persons to choose freely under what circumstances and to
\item what extcnt they will reveal information about themselves.
\end{itemize}

\note[item]{}
\end{frame}
\begin{frame}
\frametitle{Unrelated Title}


\begin{itemize}
\item A cookie is a piece of data sent from a website and stored on the user's computer
by the user's web browser while the user is browsing.
\end{itemize}

\note[item]{}
\end{frame}
\begin{frame}
\frametitle{Unrelated Title}


\begin{itemize}
\item Copyright is a legal right that gives the creator of an original work exclusive lights
\item over its use and distribution. Copyright is a form of intellectual property applicable
\item to certain forms of creative work such as books, illustrations, maps, poetry and
\item plays. These rights include reproduction. control over derivative works (works
\item based on the original work), distribution and public performance.
\end{itemize}

\note[item]{}
\end{frame}
\begin{frame}
\frametitle{Unrelated Title}


\begin{itemize}
\item Software piracy is the unauthorised reproduction, distribution or use of software products.
\end{itemize}

\note[item]{}
\end{frame}
\begin{frame}
\frametitle{Unrelated Title}


\begin{itemize}
\item Data theft is the unauthorised copying or removal of data from the legitimate
owner's computer system.
\end{itemize}

\note[item]{}
\end{frame}
\begin{frame}
\frametitle{Unrelated Title}


\begin{itemize}
\item The Dark Net is the term given to the parts of the Intermet that are kept hidden
from the general public and cannot be accessed by standard search engines such as
Google and Bing. Suspect activities such as computer hacking and fraud take place
on Dark Net websites.
\end{itemize}

\note[item]{}
\end{frame}
\begin{frame}
\frametitle{Unrelated Title}


\begin{itemize}
\item A denial-of-service attack (DOS attack) is a cyber-attack where thc intent is to
\item prevent a scrvice being delivered by the target system.
\item The attack could be by an individual hacker exploiting a vulnerability in the
\item target system to gain unauthorised access and so crash the system from within.
\item When the attack is directed from the outside in, it may be a distributed denial-of.
\item service attack.
\end{itemize}

\note[item]{}
\end{frame}
\begin{frame}
\frametitle{Unrelated Title}


\begin{itemize}
\item A distributed denial-ot-service attack (DDOS attack) is a cyber-attack during
\item which the target system is flooded with requests that overload the targeted system.
\end{itemize}

\note[item]{}
\end{frame}
\begin{frame}
\frametitle{Unrelated Title}


\begin{itemize}
\item Malware, malicious software, is software designed to disrupt, damage or gain
\item unauthorised acccss to a computer system. Viruses, worms, trojans, ransomware
\item and spyware are all types of malware.
\end{itemize}

\note[item]{}
\end{frame}
\begin{frame}
\frametitle{Unrelated Title}


\begin{itemize}
\item Electronic eavesdropping is the act of electronically intercepting communications
\item without the knowledge or consent of at least one of the participants.
\end{itemize}

\note[item]{}
\end{frame}
\begin{frame}
\frametitle{Unrelated Title}


\begin{itemize}
\item intercepting emails
\item activating a laptop webcam to watch you remotely
\item Internet-enabled home security systcms that can be hacked so aiminals can view your home
\item voice-controlled TVs that can be hacked to listen into your conversations.
\end{itemize}

\note[item]{}
\end{frame}
\begin{frame}
\frametitle{Unrelated Title}


\begin{itemize}
\item Propaganda is communication of information that is of a biased or misleading
\item nature and that is aimed at influencing the recipient. Propaganda can be used
\item by many different organisations, including activist groups, companies, the media
\item and government bodies, for various purposes. The content is usually repeated and
\item dispersed over a wide variety of media.
\end{itemize}

\note[item]{}
\end{frame}
\begin{frame}
\frametitle{Unrelated Title}


\begin{itemize}
\item Any mechanism that reduces the risk of unauthorized access to a computer system's hardware.
\end{itemize}

\note[item]{}
\end{frame}
\begin{frame}
\frametitle{Unrelated Title}


\begin{itemize}
\item Software countermeasures are a combination of specialised system software and
application software used to protect computer systems.
\end{itemize}

\note[item]{}
\end{frame}
\begin{frame}
\frametitle{Unrelated Title}


\begin{itemize}
\item Personal security practices are countermeasures used by individuals to imnplement computer security and cybersecurity.
\end{itemize}

\note[item]{}
\end{frame}
\begin{frame}
\frametitle{Unrelated Title}


\begin{itemize}
\item managing patient files
\item financial accounting of profit and loss statements
\item billing and managing insurance forms
\item managing employee records
\item accessing the Internet and extranets for immediate access to drugs databases and to check disease symptoms.
\end{itemize}

\note[item]{}
\end{frame}
\begin{frame}
\frametitle{Unrelated Title}


\begin{itemize}
\item Consists of the cash register, magnetic stripe reader for card payments, barcode reader, mini touchscreen monitor, mini thermal printer, weighing scales, weight sensors and keyboard.
\item All EPOS terminals in store are connected to the supermarket's mainframe computer system.
\end{itemize}

\note[item]{}
\end{frame}
\begin{frame}
\frametitle{Unrelated Title}


\begin{itemize}
\item A problem is a discrepancy (or difference) between the data we have and the information we require.
\end{itemize}

\note[item]{}
\end{frame}
\begin{frame}
\frametitle{Unrelated Title}


\begin{itemize}
\item A solution is a set ol instructions that, if followed in order, will produce the required inlormation.
\end{itemize}

\note[item]{}
\end{frame}
\begin{frame}
\frametitle{Unrelated Title}


\begin{itemize}
\item Problem-solving is the process of creating a set of instructions that, when
executed, accepts input data and produces meaningful information.
\end{itemize}

\note[item]{}
\end{frame}
\begin{frame}
\frametitle{Unrelated Title}


\begin{itemize}
\item An algorithm is a sequence of instructions which rigorously defines a solution to a
\item problem. Put more simply, an algorithm is a set of instructions, written in everyday
\item English, that solves a problem.
\end{itemize}

\note[item]{}
\end{frame}
\begin{frame}
\frametitle{Unrelated Title}


\begin{itemize}
\item Pseudocode is a language consisting of English-like statements used to define an
\item algorithm. Pseudocode is a formal way of writing an algorithm using Structured
\item English text. numbers and special characters.
\end{itemize}

\note[item]{}
\end{frame}
\begin{frame}
\frametitle{Unrelated Title}


\begin{itemize}
\item A flowchart is a pictorial way of representing an algorithm using a set of Standard
\item symbols (shapes).
\end{itemize}

\note[item]{}
\end{frame}
\begin{frame}
\frametitle{Unrelated Title}


\begin{itemize}
\item A typical problem-solving process involves six steps:
\item Define the problem.Propose and evaluate solutions.Determine the best solution.Develop the algorithm.Represent the algorithm as pseudocode or a flowchart.Test and validate the solution.
\end{itemize}

\note[item]{}
\end{frame}
\begin{frame}
\frametitle{Unrelated Title}


\begin{itemize}
\item The divide-and-conquer approach is to repeatedly split a large problem into a
number of smaller sub-problems until a complete solution is identified.
\end{itemize}

\note[item]{}
\end{frame}
\begin{frame}
\frametitle{Unrelated Title}


\begin{itemize}
\item To decompose is to break down into a combination of simpler components.
\end{itemize}

\note[item]{}
\end{frame}
\begin{frame}
\frametitle{Unrelated Title}


\begin{itemize}
\item A variable is an area of storage whose value can change during processing.
\end{itemize}

\note[item]{}
\end{frame}
\begin{frame}
\frametitle{Unrelated Title}


\begin{itemize}
\item A Constant is an area of storage whose value cannot be changed during processing. 
\end{itemize}

\note[item]{}
\end{frame}
\begin{frame}
\frametitle{Unrelated Title}


\begin{itemize}
\item The name given to variable or a constant by a programmer is called an identifier.
\end{itemize}

\note[item]{}
\end{frame}
\begin{frame}
\frametitle{Unrelated Title}


\begin{itemize}
\item A data type specifies what sort of values a variable or constant can hold.
\end{itemize}

\note[item]{}
\end{frame}
\begin{frame}
\frametitle{Unrelated Title}


\begin{itemize}
\item integer: whole numbers such as 0.5, 10, 1.024 and -50 (negative 50)floating point (real): any number that contains decimal places such as 0.2, 1.5 and —50.4 (negative 50.4)character: units of data including letters. numbers and symbols such as A. a. 9. S and @string: a sequence of characters such as 'Frank', 'Ruth', 'bus' and '925•0000'boolean: a true or false value. Yes/No.
\end{itemize}

\note[item]{}
\end{frame}
\begin{frame}
\frametitle{Unrelated Title}


\begin{itemize}
\item All algorithms share these basic characteristics:
\item They have a limited (finite) number of steps to be performed.They must contain a finite number of instructions. A complex algorithm may contain thousands of instructions but it must be a finite number, it must always finish.The statements used are precise.An algorithm is precise when it is strictly defined.The statements used are unambiguous.An algorithm is unantbiguous when it is not open to more than one interpretation. Algorithms cannot include words or phrases like 'looks bigger, or 'best'. because these can bc interpreted differently by different programmersThey have instructions that pass the flow of control from onc processfaction to
\item another. Instructions in an algorithm must be followed in a sequence, control from onc process to thc next until the algorithm terminates.They eventually terminate.All algorithms must eventually finish.
\end{itemize}

\note[item]{}
\end{frame}
\begin{frame}
\frametitle{Unrelated Title}


\begin{itemize}
\item The most common Boolean. or logical operators, perform AND. NOT and OR
\item actions. They occur:
\item • before another condition in the case of the NOT operator
\item • between two relational conditions as in the case of thc AND OR operators.
\item The use of a logical operator turns a simple relational condition into a compound
\item condition.
\end{itemize}

\note[item]{}
\end{frame}
\begin{frame}
\frametitle{Unrelated Title}


\begin{itemize}
\item Other important symbols used in programming are those that represent
\item mathematical computations such as addition. subtraction. multiplication and
\item division. These are called the arithmetic operators and these perform +,-,*,/, DIV
\item and MOD actions. The addition. subtraction, multiplication and division operations
\item are the same actions as in mathematics. The DIV and MOD operators may be new
\item to you: they are used in integer arithmetic. Integers are whole numbers.
\end{itemize}

\note[item]{}
\end{frame}
\begin{frame}
\frametitle{Unrelated Title}


\begin{itemize}
\item DIV is the integer division operator which discards the fractional part (remainder)
of the result. For example:
5 DIV 2 produces 2 since the remainder, which is 1, is discarded.
\end{itemize}

\note[item]{}
\end{frame}
\begin{frame}
\frametitle{Unrelated Title}


\begin{itemize}
\item MOD is the integer remainder operator which gives the fractional part (remainder)
of the result. For example:
5 MOD 2 produces 1 since this is the remainder.
\end{itemize}

\note[item]{}
\end{frame}
\begin{frame}
\frametitle{Unrelated Title}


\begin{itemize}
\item Machine languagc consists of binary
code that a microprocessor can cxecute
direcity.
\end{itemize}

\note[item]{}
\end{frame}
\begin{frame}
\frametitle{Unrelated Title}


\begin{itemize}
\item Binary code is a sequence of Is and Os.
Here is an Example of a machine code
instruction written in binary code:
0011001000001110 00010010 00000000
\end{itemize}

\note[item]{}
\end{frame}
\begin{frame}
\frametitle{Unrelated Title}


\begin{itemize}
\item Assembly language uses short words
to represent simple instructions that can
easily be translated into machine code.
\end{itemize}

\note[item]{}
\end{frame}
\begin{frame}
\frametitle{Unrelated Title}


\begin{itemize}
\item Low-level language is a machine-dependent language written in machine
\item or assembly code.
\item Low-Level languages are not useful for moden programming.
\end{itemize}

\note[item]{}
\end{frame}
\begin{frame}
\frametitle{Unrelated Title}


\begin{itemize}
\item High-level languages are machine-indepedent languages written in statements
that closely resemble English. High-level languages include Pascal, Python, Visual
Basic and C.
\end{itemize}

\note[item]{}
\end{frame}
\begin{frame}
\frametitle{Unrelated Title}


\begin{itemize}
\item The  reserved words in a prograimming language have specific meaning and use. It shouldn't come as any surprise that some reserved words appear in most programming languages.
\end{itemize}

\note[item]{}
\end{frame}
\begin{frame}
\frametitle{Unrelated Title}


\begin{itemize}
\item IFTHENELSEREPEATUNTILBEGINENDORNOT
\end{itemize}

\note[item]{}
\end{frame}
\begin{frame}
\frametitle{Unrelated Title}


\begin{itemize}
\item Syntax is a set of rules defining the structure of statements in a programming
language.
\end{itemize}

\note[item]{}
\end{frame}
\begin{frame}
\frametitle{Unrelated Title}


\begin{itemize}
\item A translator converts one language into another.
\end{itemize}

\note[item]{}
\end{frame}
\begin{frame}
\frametitle{Unrelated Title}


\begin{itemize}
\item A compiler is a special translator that converts high-level language into machine
code.
\end{itemize}

\note[item]{}
\end{frame}
\begin{frame}
\frametitle{Unrelated Title}


\begin{itemize}
\item An interpreter is a translator that converts and executes high-level language in
one step.
\end{itemize}

\note[item]{}
\end{frame}
\begin{frame}
\frametitle{Unrelated Title}


\begin{itemize}
\item A Syntax error is an error in how the statements in source code are written in a
particular program.
\end{itemize}

\note[item]{}
\end{frame}
\begin{frame}
\frametitle{Unrelated Title}


\begin{itemize}
\item Corrective maintenance is fixing program bugs or flaws in the program.
\end{itemize}

\note[item]{}
\end{frame}
\begin{frame}
\frametitle{Unrelated Title}


\begin{itemize}
\item Adaptive maintenance is implementing new features or changing the way a
particular feature works.
\end{itemize}

\note[item]{}
\end{frame}
\begin{frame}
\frametitle{Unrelated Title}


\begin{itemize}
\item A logic error is a mistake in the design of the program.
\end{itemize}

\note[item]{}
\end{frame}
\begin{frame}
\frametitle{Unrelated Title}


\begin{itemize}
\item A run-time error is a program error that occurs during program execution.
\end{itemize}

\note[item]{}
\end{frame}
\begin{frame}
\frametitle{Unrelated Title}


\begin{itemize}
\item Test data is a set of data values selected to test a computer program. Test data is
\item used for positive testing when the data values centered are what might be expected
\item during normal operation; in this case the anticipated result should be output. Test
\item data is also used for negativc testing; in negative testing the data values entered
\item are unusual, extreme, exceptional or unexpected, and the question is: Does the
\item program still produce the expected output?
\end{itemize}

\note[item]{}
\end{frame}
\begin{frame}
\frametitle{Unrelated Title}


\begin{itemize}
\item Debugging is the process of finding and removing errors from computer hardware
or software.
\end{itemize}

\note[item]{}
\end{frame}
\begin{frame}
\frametitle{Unrelated Title}


\begin{itemize}
\item Fehlende oder mangelhafte Vertriebs- und Kundenstrategie
\item Fehlende oder mangelhafte Verknüpfung der Vertriebsziele mit der Unternehmensmission
\item Unterschiedliche Zielvorgaben für Vertriebsabteilungen
\item Kein professionell organisiertes Vertriebstean (fehlende Vorgaben) 
\item Silo Problematik (jeder denkt nur in seinem Bereich) keine Kundenorientierte Organisationsstruktur
\end{itemize}

\note[item]{}
\end{frame}
\begin{frame}
\frametitle{Unrelated Title}

\begin{center}
\includegraphics[width=0.9\textwidth,height=0.9\textheight,keepaspectratio]{/Users/I516998/Library/Application Support/Anki2/User 1/collection.media/img4778279323356636258.jpg}
\end{center}

\begin{itemize}
\item Weg von Transaktionsorientiert, hin zu Beziehungsorientiert
\end{itemize}

\note[item]{}
\end{frame}
\begin{frame}
\frametitle{Unrelated Title}


\begin{itemize}
\item Fehlende/mangelnde Priorisierung der Kunden
\item Unzureichender Support des Vertriebsinnendienstes
\item Schlechte Zielvorgaben/ Forecast
\end{itemize}

\note[item]{}
\end{frame}
\begin{frame}
\frametitle{Unrelated Title}


\begin{itemize}
\item Einzelkämpfer-Mentalität
\item Konfliktbehaftete Zusammenarbeit im Marketing/Vertrieb/Produktion
\item Defizite in der MA Qualifikation
\item Fehler innder Mitarbeiterauswahl
\end{itemize}

\note[item]{}
\end{frame}
\begin{frame}
\frametitle{Unrelated Title}


\begin{itemize}
\item Best Practises zur Positionierung
\item Transaktionsmanagement zu  Relationship
\item Kundenorientierung
\item Lieferanten sollen zu Partnern werden
\item Für den Vertrieb:
\item Vom reaktiven Verkauf zum aktiven Verkauf (Berater)
\item Produktvertrieb->Lösungsvertrieb
\item Kundenorientierter, nicht produktionsorientierter Vertrieb
\item Für das KAM:
\item Individuelle Strategie für Vertriebskunden
\item Professioneller Methodeneinsatz 
\item Ziele, KPIs und Forecast-Bewertung gemeinsam mit KA-Manager festlegen
\end{itemize}

\note[item]{}
\end{frame}
\begin{frame}
\frametitle{Unrelated Title}


\begin{itemize}
\item Kundenabwanderung vermeiden
\item Ursachen für Kundenabwanderung identifizieren
\item Maßnahmen ableiten
\end{itemize}

\note[item]{}
\end{frame}
\begin{frame}
\frametitle{Unrelated Title}


\begin{itemize}
\item Operatives CRM
\item - Vertriebsautomation
\item - Kampagnenmanagement
\item - Kundenservice
\item Analytisches CRM
\item - data Mining/Warehousing
\item Kollaboratives CRM
\item - face to face
\item - call-center
\item - web/mail 
\item => Channel Management
\end{itemize}

\note[item]{}
\end{frame}
\begin{frame}
\frametitle{Unrelated Title}

\begin{center}
\includegraphics[width=0.9\textwidth,height=0.9\textheight,keepaspectratio]{/Users/I516998/Library/Application Support/Anki2/User 1/collection.media/img5301577611131610972.jpg}
\end{center}


\note[item]{}
\end{frame}
\begin{frame}
\frametitle{Unrelated Title}


\begin{itemize}
\item Fokus auf Konsumenten als Shopper
\item Kooperative Zusammenarbeit
\item Daten und Fakten als sachliche Entscheidungsgrundlage
\item Strukturierter und permanenter Prozess und Erfolgskontrolle
\end{itemize}

\note[item]{}
\end{frame}
\begin{frame}
\frametitle{Unrelated Title}

\begin{center}
\includegraphics[width=0.9\textwidth,height=0.9\textheight,keepaspectratio]{/Users/I516998/Library/Application Support/Anki2/User 1/collection.media/img7860459893370782041.jpg}
\end{center}


\note[item]{}
\end{frame}
\begin{frame}
\frametitle{Unrelated Title}


\begin{itemize}
\item Unterteilung der Warengruppe in Segmente erfolgt nach Gruppen von Produkten, die als ähnlich angesehen werden
\item Bps.
\item Bedarfsorientiert
\item Erlebnisorientiert
\item Zielgruppenorientiert
\end{itemize}

\note[item]{}
\end{frame}
\begin{frame}
\frametitle{Unrelated Title}


\begin{itemize}
\item Profilierungssortiment
\item -> der beste Anbieter sein
\item Pflichtsortiment
\item -> die Verbrauchererwartungen erfüllen 
\item Impuls- und Saisonsortiment
\item -> saisonbedingter Verbrauchernutzen, zusätzlich gekauft 
\item Ergänzungssortiment
\item -> one stop shopping ermöglichen
\end{itemize}

\note[item]{}
\end{frame}
\begin{frame}
\frametitle{Unrelated Title}


\begin{itemize}
\item Checkliste nach unterschiedlichen Bereichen:
\item Konsument
\item Markt
\item Händler 
\item Hersteller
\end{itemize}

\note[item]{}
\end{frame}
\begin{frame}
\frametitle{Unrelated Title}


\begin{itemize}
\item Unterteilt in:
\item Käufer 
\item Markt
\item Finanzen
\item Produktivität
\end{itemize}

\note[item]{}
\end{frame}
\begin{frame}
\frametitle{Unrelated Title}


\begin{itemize}
\item Frequenz bilden
\item Transaktionswert steigern
\item Gewinn erhöhen
\item Marktanteil verteidigen
\item Kundenbegeisterung erzeugen
\item Image verbessern
\end{itemize}

\note[item]{}
\end{frame}
\begin{frame}
\frametitle{Unrelated Title}


\begin{itemize}
\item Strategien werden Taktiken zugeordnet jeweils für:
\item Sortiment
\item Preis
\item Promotion
\item Präsentation
\end{itemize}

\note[item]{}
\end{frame}
\begin{frame}
\frametitle{Unrelated Title}


\begin{itemize}
\item Fünf Schlüsselkomponenten:
\item Commitment
\item Verantwortlichkeiten
\item Orga- und Ressourcenmamagement
\item Kommunikation
\item Controlling
\end{itemize}

\note[item]{}
\end{frame}
\begin{frame}
\frametitle{Unrelated Title}


\begin{itemize}
\item Erfolgskontrolle
\item Greifen die Maßnahmen? 
\item Durchführung soll-ist Abgleich
\end{itemize}

\note[item]{}
\end{frame}
\begin{frame}
\frametitle{Unrelated Title}

\begin{center}
\includegraphics[width=0.9\textwidth,height=0.9\textheight,keepaspectratio]{/Users/I516998/Library/Application Support/Anki2/User 1/collection.media/img6209679356791583661.jpg}
\end{center}


\note[item]{}
\end{frame}
\begin{frame}
\frametitle{Unrelated Title}


\begin{itemize}
\item Die Konkurrenz zieht mit
\item Direkter Verzicht auf Deckungsbeitrag
\item Ungewissheit über Mengeneffekte
\end{itemize}

\note[item]{}
\end{frame}
\begin{frame}
\frametitle{Unrelated Title}


\begin{itemize}
\item Konkurrenz wird ermutigt mitzuziehen
\item Steigerung Deckungsbeitrag pro Stück
\item Bewusster Verzicht auf Menge, bei höherer Profitabilität
\end{itemize}

\note[item]{}
\end{frame}
\begin{frame}
\frametitle{Unrelated Title}

\begin{center}
\includegraphics[width=0.9\textwidth,height=0.9\textheight,keepaspectratio]{/Users/I516998/Library/Application Support/Anki2/User 1/collection.media/img1710589577545477715.jpg}
\end{center}


\note[item]{}
\end{frame}
\begin{frame}
\frametitle{Unrelated Title}

\begin{center}
\includegraphics[width=0.9\textwidth,height=0.9\textheight,keepaspectratio]{/Users/I516998/Library/Application Support/Anki2/User 1/collection.media/img7624100696557564669.jpg}
\end{center}


\note[item]{}
\end{frame}
\begin{frame}
\frametitle{Unrelated Title}

\begin{center}
\includegraphics[width=0.9\textwidth,height=0.9\textheight,keepaspectratio]{/Users/I516998/Library/Application Support/Anki2/User 1/collection.media/img4780470074833556872.jpg}
\end{center}


\note[item]{}
\end{frame}
\begin{frame}
\frametitle{Unrelated Title}

\begin{center}
\includegraphics[width=0.9\textwidth,height=0.9\textheight,keepaspectratio]{/Users/I516998/Library/Application Support/Anki2/User 1/collection.media/img2839204673960453060.jpg}
\end{center}


\note[item]{}
\end{frame}
\begin{frame}
\frametitle{Unrelated Title}

\begin{center}
\includegraphics[width=0.9\textwidth,height=0.9\textheight,keepaspectratio]{/Users/I516998/Library/Application Support/Anki2/User 1/collection.media/img3608973729222922535.jpg}
\end{center}


\note[item]{}
\end{frame}
\begin{frame}
\frametitle{Unrelated Title}

\begin{center}
\includegraphics[width=0.9\textwidth,height=0.9\textheight,keepaspectratio]{/Users/I516998/Library/Application Support/Anki2/User 1/collection.media/img2070427017049046492.jpg}
\end{center}


\note[item]{}
\end{frame}
\begin{frame}
\frametitle{Unrelated Title}

\begin{center}
\includegraphics[width=0.9\textwidth,height=0.9\textheight,keepaspectratio]{/Users/I516998/Library/Application Support/Anki2/User 1/collection.media/img2173926681079083188.jpg}
\end{center}


\note[item]{}
\end{frame}
\begin{frame}
\frametitle{Unrelated Title}

\begin{center}
\includegraphics[width=0.9\textwidth,height=0.9\textheight,keepaspectratio]{/Users/I516998/Library/Application Support/Anki2/User 1/collection.media/img1384748025338037963.jpg}
\end{center}


\note[item]{}
\end{frame}
\begin{frame}
\frametitle{Unrelated Title}

\begin{center}
\includegraphics[width=0.9\textwidth,height=0.9\textheight,keepaspectratio]{/Users/I516998/Library/Application Support/Anki2/User 1/collection.media/img2136724737121933747.jpg}
\end{center}


\note[item]{}
\end{frame}
\begin{frame}
\frametitle{Unrelated Title}

\begin{center}
\includegraphics[width=0.9\textwidth,height=0.9\textheight,keepaspectratio]{/Users/I516998/Library/Application Support/Anki2/User 1/collection.media/img1016667514237396004.jpg}
\end{center}


\note[item]{}
\end{frame}
\begin{frame}
\frametitle{Unrelated Title}


\begin{itemize}
\item 1. Bestimmung der Globalzufriedenheit
\item 2. Bewertung verschiedener Leistungsattribute
\item 3. Klassifikation der Ergebnisse in unzufriedene und zufriedene Kunden
\item 4. Mit Hilfe einer multiplen Regressionsanalyse Ermittlung der Leistungsattribute die Penalty- und die Reward-Faktoren darstellen
\end{itemize}

\note[item]{}
\end{frame}
\begin{frame}
\frametitle{Unrelated Title}

\begin{center}
\includegraphics[width=0.9\textwidth,height=0.9\textheight,keepaspectratio]{/Users/I516998/Library/Application Support/Anki2/User 1/collection.media/img8699595733733703348.jpg}
\end{center}


\note[item]{}
\end{frame}
\begin{frame}
\frametitle{Unrelated Title}


\begin{itemize}
\item Beschwerdemanagement
\item Kundenbeziehungsmanagement
\item Kundenkontaktmanagement
\item Key Account Management
\end{itemize}

\note[item]{}
\end{frame}
\begin{frame}
\frametitle{Unrelated Title}

\begin{center}
\includegraphics[width=0.9\textwidth,height=0.9\textheight,keepaspectratio]{/Users/I516998/Library/Application Support/Anki2/User 1/collection.media/img7856721232436278353.jpg}
\end{center}


\note[item]{}
\end{frame}
\begin{frame}
\frametitle{Unrelated Title}

\begin{center}
\includegraphics[width=0.9\textwidth,height=0.9\textheight,keepaspectratio]{/Users/I516998/Library/Application Support/Anki2/User 1/collection.media/img4457276675425416607.jpg}
\end{center}


\note[item]{}
\end{frame}
\begin{frame}
\frametitle{Unrelated Title}

\begin{center}
\includegraphics[width=0.9\textwidth,height=0.9\textheight,keepaspectratio]{/Users/I516998/Library/Application Support/Anki2/User 1/collection.media/img5003809063171366718.jpg}
\end{center}


\note[item]{}
\end{frame}
\begin{frame}
\frametitle{Unrelated Title}

\begin{center}
\includegraphics[width=0.9\textwidth,height=0.9\textheight,keepaspectratio]{/Users/I516998/Library/Application Support/Anki2/User 1/collection.media/img492513830173706810.jpg}
\end{center}


\note[item]{}
\end{frame}
\begin{frame}
\frametitle{Unrelated Title}


\begin{itemize}
\item Governance-Risikomanagement-Compliancemanagement
\item -> unternehmensweit angelegte Organisationssteuerung
\item -> festgelegter Risiko Appetit zusammen mit rechtlichen und ethischen Faktoren als Handlungsrahmen
\end{itemize}

\note[item]{}
\end{frame}
\begin{frame}
\frametitle{Unrelated Title}


\begin{itemize}
\item Schutz ( Haftung)
\item Image und Reputation
\item Beratung und Information
\item Qualitätssicherung und Innovation
\item Überwachungsfunktion
\end{itemize}

\note[item]{}
\end{frame}
\begin{frame}
\frametitle{Unrelated Title}

\begin{center}
\includegraphics[width=0.9\textwidth,height=0.9\textheight,keepaspectratio]{/Users/I516998/Library/Application Support/Anki2/User 1/collection.media/img3971938724884213812.jpg}
\end{center}


\note[item]{}
\end{frame}
\begin{frame}
\frametitle{Unrelated Title}


\begin{itemize}
\item Porblem (sales of lollipos are going down)Data (all sales data)Informaiton (lollipops bought by people older than 25 Insight (moms believe lollipops=bad teeth))Value (dentists advertise lollipops)
\end{itemize}

\note[item]{}
\end{frame}
\begin{frame}
\frametitle{Unrelated Title}


\begin{itemize}
\item Additional data sources neededData often semi-/unstructured & noisyData alone is silent (processing of the data is necessary)Right technology and organizazion required
\end{itemize}

\note[item]{}
\end{frame}
\begin{frame}
\frametitle{Unrelated Title}


\begin{itemize}
\item PhilosophicalBusinessTechnical
\end{itemize}

\note[item]{}
\end{frame}
\begin{frame}
\frametitle{Unrelated Title}


\begin{itemize}
\item Big data is an approach to problem solving (look around, collect data, answer questions from data)Traditional CS: Understand the problem, build a model/algorthm, answer questions from its implementation)Success stories e.g. machine translation
\end{itemize}

\note[item]{}
\end{frame}
\begin{frame}
\frametitle{Unrelated Title}


\begin{itemize}
\item MonetiazationPrivacy securityLegal and ethical aspects
\end{itemize}

\note[item]{}
\end{frame}
\begin{frame}
\frametitle{Unrelated Title}


\begin{itemize}
\item Traditional systems: decide upfront what data is important und keep data in well designed tablesBid data approach: collect lots of data and use as neededCheapter than deciding what to keepRelevant data can't be controlled anywayOriginal data formatsPay-as-you-go data integration/cleansing/...--> DATA LAKES
\end{itemize}

\note[item]{}
\end{frame}
\begin{frame}
\frametitle{Unrelated Title}


\begin{itemize}
\item Volume - large amount of dataVelocity - new data is arriving fastVariety - many exchange formats, variety of complex tasks to solveIn short: big, fast, diverse
\end{itemize}

\note[item]{}
\end{frame}
\begin{frame}
\frametitle{Unrelated Title}


\begin{itemize}
\item 



Big data is high-volume, high-velocity and high-variety
information assets that demand cost-effective, innovative
forms of information processing for enhanced insight and
decision making. 



\end{itemize}

\note[item]{}
\end{frame}
\begin{frame}
\frametitle{Unrelated Title}


\begin{itemize}
\item 



Big data refers to datasets whose size is beyond
the ability of typical database software tools to
capture, store, manage, and analyze. SubjectiveMoving targettoo big, too fast, too diverse (for available systems)



\end{itemize}

\note[item]{}
\end{frame}
\begin{frame}
\frametitle{Unrelated Title}


\begin{itemize}
\item 




\end{itemize}

\note[item]{}
\end{frame}
\begin{frame}
\frametitle{Unrelated Title}


\begin{itemize}
\item 


Big data is where the data volume, acquisition
velocity, or data representation limits the ability to
perform effective analysis using traditional relational
approaches or requires the use of significant
horizontal scaling for efficient processing.How data is processed mattersThe conclusion big data follows from many machines



\end{itemize}

\note[item]{}
\end{frame}
\begin{frame}
\frametitle{Unrelated Title}


\begin{itemize}
\item Data storage and accessTransactional processingData queryingStreaming and messagingmachine learning
\end{itemize}

\note[item]{}
\end{frame}
\begin{frame}
\frametitle{Unrelated Title}


\begin{itemize}
\item OLTP (Online transactional processing): transactional system for many transactionsOLAP (Online analytical processing): large queriesHTAP (Hybrid transactional and analyticaal processing): haybrid systems for booth
\end{itemize}

\note[item]{}
\end{frame}
\begin{frame}
\frametitle{Unrelated Title}


\begin{itemize}
\item Parallel: Set of nodes (Multiple cores, disk, machine)Machines are physically close and the network is fastHomogenous softwareGoals: High performance, high availability, extensibilityDistributed: Machines can be far away -> slow connection between subsets of machines leads to substantial communication costsDBMS implemented on top of multiple site that hold their own dataGoals: Share distributed data, transparent access, autonomy of sites
\end{itemize}

\note[item]{}
\end{frame}
\begin{frame}
\frametitle{Unrelated Title}


\begin{itemize}
\item Distributed dataData naturally hosted at multiple sitesGeographically distributed dataautonomy of sites (federation)Goal: manage distributed data3V'sProcessing not possible on a single machineGoal: handle big data (performance, scalability, availability)
\end{itemize}

\note[item]{}
\end{frame}
\begin{frame}
\frametitle{Unrelated Title}


\begin{itemize}
\item Performance means doing work fast
\item Throughput, latencyHigh performance often requires parallel processing
\end{itemize}

\note[item]{}
\end{frame}
\begin{frame}
\frametitle{Unrelated Title}


\begin{itemize}
\item Scalability means that a system can be enlarged to handle growing amount of work
\item more data, transactions and queries
\end{itemize}

\note[item]{}
\end{frame}
\begin{frame}
\frametitle{Unrelated Title}


\begin{itemize}
\item Vertical scaling (scale up)More processors, ram and better componentsLess challanging to implementNo expensive inter-machine communicationLower power consumption and cooling costsExpensiveHarware failure can cause bigger outagesVendor lock-inLimited upgradabilityHorizontal scaling (scale out)more serversscales infinitelycheaper overalleasier to upagradeeasier fault tolerancechallanging to implementexpensive inter-machine communicationmore licensing feesmore space, electricity, cooling and network equipment
\end{itemize}

\note[item]{}
\end{frame}
\begin{frame}
\frametitle{Unrelated Title}


\begin{itemize}
\item 


Big Data often includes substantial amounts of
non-relational data 




\end{itemize}

\note[item]{}
\end{frame}
\begin{frame}
\frametitle{Unrelated Title}


\begin{itemize}
\item Scale outto large data setsto large clusters of commodity machinesto a large number of usersSimple to usetwo functions: Map and Reduce
\end{itemize}

\note[item]{}
\end{frame}
\begin{frame}
\frametitle{Unrelated Title}


\begin{itemize}
\item Map: break sentences into (word, frequency) pairsMepReduce runtime automatically groups frequencies by wordReduce: sum up frequencies
\end{itemize}

\note[item]{}
\end{frame}
\begin{frame}
\frametitle{Unrelated Title}

\begin{center}
\includegraphics[width=0.9\textwidth,height=0.9\textheight,keepaspectratio]{/Users/I516998/Library/Application Support/Anki2/User 1/collection.media/paste-51c27fa6c192693946baf47943a0202e5f0a63cd.jpg}
\end{center}


\note[item]{}
\end{frame}
\begin{frame}
\frametitle{Unrelated Title}


\begin{itemize}
\item not only SQL
\end{itemize}

\note[item]{}
\end{frame}
\begin{frame}
\frametitle{Unrelated Title}


\begin{itemize}
\item Key-value storesDocument storesGraph databases
\end{itemize}

\note[item]{}
\end{frame}
\begin{frame}
\frametitle{Unrelated Title}


\begin{itemize}
\item Non-relational data modelsOperation simplicityHigh availability and fault toleranceAutomatic horizontal scalability on commodity hardware
\end{itemize}

\note[item]{}
\end{frame}
\begin{frame}
\frametitle{Unrelated Title}


\begin{itemize}
\item Distributed system wish list
\item Consistency across all nodes (ACI in ACID)Availability: requests receive timely responsesPartition tolerance: system continues to operate correcetly despite message lossCAP principle: can't have all 3 simultaneously --> trade off is needed
\end{itemize}

\note[item]{}
\end{frame}
\begin{frame}
\frametitle{Unrelated Title}


\begin{itemize}
\item RDBMS promote consostency over availability (ACID)NoSQL database promote availability over consitency (BASE - basically available, soft wate, eventually consistent)
\end{itemize}

\note[item]{}
\end{frame}
\begin{frame}
\frametitle{Unrelated Title}

\begin{center}
\includegraphics[width=0.9\textwidth,height=0.9\textheight,keepaspectratio]{/Users/I516998/Library/Application Support/Anki2/User 1/collection.media/Bildschirmfoto 2022-09-15 um 13.01.51.png}
\end{center}


\note[item]{}
\end{frame}
\begin{frame}
\frametitle{Unrelated Title}


\begin{itemize}
\item Acceptance Testing (does the designed fulfil the users need?)Reliability Testing (reflect the frequency of user input, used for reliability estimation)Usability Testing (checks system from a usability POV / GUI)Defect Testing (discover system defects - reveals presence, NOT absence of defects)Compability Testing (ensures that the application functions in different configured systems and environments)
\end{itemize}

\note[item]{}
\end{frame}
\begin{frame}
\frametitle{Unrelated Title}


\begin{itemize}
\item Component (Unit) Testing individual program components (by: component developer)Integration Testing
\end{itemize}

\note[item]{}
\end{frame}
\begin{frame}
\frametitle{Unrelated Title}


\begin{itemize}
\item Testing is the process of executing a program with the intent of finding errors.
\item destructive process, positive test is one that uncovers an errorBasic Idea: stimulate system with some input, and compare actual output with expected
\end{itemize}

\note[item]{}
\end{frame}
\begin{frame}
\frametitle{Unrelated Title}

\begin{center}
\includegraphics[width=0.9\textwidth,height=0.9\textheight,keepaspectratio]{/Users/I516998/Library/Application Support/Anki2/User 1/collection.media/Bildschirmfoto 2022-09-15 um 13.33.51.png}
\end{center}


\note[item]{}
\end{frame}
\begin{frame}
\frametitle{Unrelated Title}

\begin{center}
\includegraphics[width=0.9\textwidth,height=0.9\textheight,keepaspectratio]{/Users/I516998/Library/Application Support/Anki2/User 1/collection.media/Bildschirmfoto 2022-09-15 um 13.35.33.png}
\end{center}


\note[item]{}
\end{frame}
\begin{frame}
\frametitle{Unrelated Title}

\begin{center}
\includegraphics[width=0.9\textwidth,height=0.9\textheight,keepaspectratio]{/Users/I516998/Library/Application Support/Anki2/User 1/collection.media/Bildschirmfoto 2022-09-15 um 13.42.25.png}
\includegraphics[width=0.9\textwidth,height=0.9\textheight,keepaspectratio]{/Users/I516998/Library/Application Support/Anki2/User 1/collection.media/Bildschirmfoto 2022-09-15 um 13.42.38.png}
\end{center}

\begin{itemize}
\item how difficult it is to configure or set a program to a certain state
\item data abstraction reduces controllability and observability 
\end{itemize}

\note[item]{}
\end{frame}
\begin{frame}
\frametitle{Unrelated Title}


\begin{itemize}
\item How easy it is to write effective testsdocumentationlog mechanisms for tracing function calls etc. available?test creation and test management technologies?
\end{itemize}

\note[item]{}
\end{frame}
\begin{frame}
\frametitle{Unrelated Title}


\begin{itemize}
\item used informallydont use itreally, dontfault? error? failure??
\end{itemize}

\note[item]{}
\end{frame}
\begin{frame}
\frametitle{Unrelated Title}


\begin{itemize}
\item Problem: too many input-possibilities to fully test them
\item Coverage criteria: give systematic ways to search the input space -> find fewest inputs that will find the most problems
\item Advantages:minimize the overlap in the testsgive a stopping rulecan be well supported by powerful tools
\end{itemize}

\note[item]{}
\end{frame}
\begin{frame}
\frametitle{Unrelated Title}


\begin{itemize}
\item Testers need to find the best return on investment (effort spent/ faults discovered)
\item Test Requirementsspecific things that must be satisfied or covered during testing
\end{itemize}

\note[item]{}
\end{frame}
\begin{frame}
\frametitle{Unrelated Title}

\begin{center}
\includegraphics[width=0.9\textwidth,height=0.9\textheight,keepaspectratio]{/Users/I516998/Library/Application Support/Anki2/User 1/collection.media/Bildschirmfoto 2022-09-15 um 14.04.10.png}
\end{center}


\note[item]{}
\end{frame}
\begin{frame}
\frametitle{Unrelated Title}

\begin{center}
\includegraphics[width=0.9\textwidth,height=0.9\textheight,keepaspectratio]{/Users/I516998/Library/Application Support/Anki2/User 1/collection.media/paste-0d79bd4beb60044a7e8add17cc38e1f3eaa78b3f.jpg}
\includegraphics[width=0.9\textwidth,height=0.9\textheight,keepaspectratio]{/Users/I516998/Library/Application Support/Anki2/User 1/collection.media/paste-6de6edf74006853ecd8576118efbfeb6700ea8a5.jpg}
\end{center}

\begin{itemize}
\item Speedup: Workload/ DB size is fixed but the number of nodes increaesesScaleup: Workload/ DB size increases by the same factor as the number of nodesElasticity: Overprovisioning means wasting moneyElasticity goal: the capacity inreases/decreses accordingly to the demand (provisioning/deprovisioning)Elasticity key question: How much tim does it take to provision or deprovision?
\end{itemize}

\note[item]{}
\end{frame}
\begin{frame}
\frametitle{Unrelated Title}


\begin{itemize}
\item Bais architectures:
\item Shared memoryShared diskShared nothingHybrid architectures:NUMAClusters
\end{itemize}

\note[item]{}
\end{frame}
\begin{frame}
\frametitle{Unrelated Title}


\begin{itemize}
\item Try to use cpu cache or memory for most of the operations to be as fast as possibleTry to minimize expensive operations
\end{itemize}

\note[item]{}
\end{frame}
\begin{frame}
\frametitle{Unrelated Title}

\begin{center}
\includegraphics[width=0.9\textwidth,height=0.9\textheight,keepaspectratio]{/Users/I516998/Library/Application Support/Anki2/User 1/collection.media/paste-4b7cf5d2bc414d8f3806d2c9376ed02a0ccb66f9.jpg}
\end{center}

\begin{itemize}
\item Multiple processors share both memory & diskSMP (Symmetric multiprocessing)
\end{itemize}

\note[item]{}
\end{frame}
\begin{frame}
\frametitle{Unrelated Title}

\begin{center}
\includegraphics[width=0.9\textwidth,height=0.9\textheight,keepaspectratio]{/Users/I516998/Library/Application Support/Anki2/User 1/collection.media/paste-c38c5a58dd2c2b1e4975fc9b911be2680c96ba5d.jpg}
\end{center}

\begin{itemize}
\item Memory exclusive, shared diskSAN (Storage area network)Good:Lower costHigher extensibilityHigher availabilityLoad balancingBad:Higher complexity (synchronisation, update problems)Potential performance problems
\end{itemize}

\note[item]{}
\end{frame}
\begin{frame}
\frametitle{Unrelated Title}

\begin{center}
\includegraphics[width=0.9\textwidth,height=0.9\textheight,keepaspectratio]{/Users/I516998/Library/Application Support/Anki2/User 1/collection.media/paste-50fcaef8c2e055c958b381c7585becd7d845958a.jpg}
\end{center}

\begin{itemize}
\item Memory and disk exclusiveMPP (Massively parallel processing)Good:Lower costHigh extensibilityHigh availabilityBad: very high complexity!!!
\end{itemize}

\note[item]{}
\end{frame}
\begin{frame}
\frametitle{Unrelated Title}

\begin{center}
\includegraphics[width=0.9\textwidth,height=0.9\textheight,keepaspectratio]{/Users/I516998/Library/Application Support/Anki2/User 1/collection.media/paste-7968431a30504c81b05ef94dc872d3e81814aae1.jpg}
\end{center}

\begin{itemize}
\item Logically shared memory, but varying access costsFor the precessor it looks like a unified memoryEach node can be SMP
\end{itemize}

\note[item]{}
\end{frame}
\begin{frame}
\frametitle{Unrelated Title}


\begin{itemize}
\item Set of interconnected server nodes, shared disk or shared nothingEach node: shared memory
\end{itemize}

\note[item]{}
\end{frame}
\begin{frame}
\frametitle{Unrelated Title}


\begin{itemize}
\item Like shared nothingDisks can be reconfigured to be assigned them to other machines (often in a single rack) 
\end{itemize}

\note[item]{}
\end{frame}
\begin{frame}
\frametitle{Unrelated Title}


\begin{itemize}
\item Federation means that multiple autonomous site participate in a distrbuted DBMS
\end{itemize}

\note[item]{}
\end{frame}
\begin{frame}
\frametitle{Unrelated Title}


\begin{itemize}
\item Parallel DBMS: Disttribute/partition/sort data to make certain DB operations fastDistrbuted DBMS: Data distrbution is given -> find query processing strategy to minimize cost
\end{itemize}

\note[item]{}
\end{frame}
\begin{frame}
\frametitle{Unrelated Title}


\begin{itemize}
\item Need to decide where to put what
\end{itemize}

\note[item]{}
\end{frame}
\begin{frame}
\frametitle{Unrelated Title}


\begin{itemize}
\item Divide the dataset into fragmentsAlso called partitioning or sharding
\end{itemize}

\note[item]{}
\end{frame}
\begin{frame}
\frametitle{Unrelated Title}


\begin{itemize}
\item Decide where to put each fragmentWIth or without replication
\end{itemize}

\note[item]{}
\end{frame}
\begin{frame}
\frametitle{Unrelated Title}


\begin{itemize}
\item Load balancingEntire relation at one site -> hot spotAt multiple site -> increased cost of updates, storageImproved efficiency/scalabilityDivide single query into a set of subqueries on subrelations (intra-query parallelism)Run multiple queries in parallel (inter-query parallelism)
\end{itemize}

\note[item]{}
\end{frame}
\begin{frame}
\frametitle{Unrelated Title}


\begin{itemize}
\item Reduce communication costs by keeping data items together that are accessed together
\end{itemize}

\note[item]{}
\end{frame}
\begin{frame}
\frametitle{Unrelated Title}


\begin{itemize}
\item Horizonzally: set of tuplesVertically: set of attributes
\end{itemize}

\note[item]{}
\end{frame}
\begin{frame}
\frametitle{Unrelated Title}

\begin{center}
\includegraphics[width=0.9\textwidth,height=0.9\textheight,keepaspectratio]{/Users/I516998/Library/Application Support/Anki2/User 1/collection.media/paste-ca1e7f08a8c1e13796d757df44ddbbe72014360d.jpg}
\end{center}

\begin{itemize}
\item Primary means that the fragmentation only depends on local attributes of  a tuple
\end{itemize}

\note[item]{}
\end{frame}
\begin{frame}
\frametitle{Unrelated Title}


\begin{itemize}
\item CompletenessReconstructabilityOften: disjointness (each data item in at most one fragment) --> in exactly one fragment
\end{itemize}

\note[item]{}
\end{frame}
\begin{frame}
\frametitle{Unrelated Title}


\begin{itemize}
\item Via union 𝑅 =∪𝑖 𝑅𝑖
\end{itemize}

\note[item]{}
\end{frame}
\begin{frame}
\frametitle{Unrelated Title}


\begin{itemize}
\item Round robinHash partitioningRange partitioning
\end{itemize}

\note[item]{}
\end{frame}
\begin{frame}
\frametitle{Unrelated Title}

\begin{center}
\includegraphics[width=0.9\textwidth,height=0.9\textheight,keepaspectratio]{/Users/I516998/Library/Application Support/Anki2/User 1/collection.media/paste-cb9c4c9b3c7af2f78f095c560fc7756949ea8779.jpg}
\end{center}

\begin{itemize}
\item Evenly dstributed dataGood for scanning full relationNot good for point or range queriesUniqueness constraints difficult to enforce
\end{itemize}

\note[item]{}
\end{frame}
\begin{frame}
\frametitle{Unrelated Title}

\begin{center}
\includegraphics[width=0.9\textwidth,height=0.9\textheight,keepaspectratio]{/Users/I516998/Library/Application Support/Anki2/User 1/collection.media/paste-2874651dbe4375f2ee46e5e2f30cad8a87ab57be.jpg}
\end{center}

\begin{itemize}
\item Good for point queries and joins on hash keyNot good for range queries and not for point queries not on keyIf hash function is good and key values now skewed -> even distributionUniqueness easy to handle on keyMay need to move tuple during updates
\end{itemize}

\note[item]{}
\end{frame}
\begin{frame}
\frametitle{Unrelated Title}

\begin{center}
\includegraphics[width=0.9\textwidth,height=0.9\textheight,keepaspectratio]{/Users/I516998/Library/Application Support/Anki2/User 1/collection.media/paste-54ffeae359fd252be0e313eef38a9f2200db31a0.jpg}
\end{center}

\begin{itemize}
\item Good for point&range queries on partioning key ANeed to select good partitioning vector, else imbalanceUniqueness easy to handle on keyMay need to move tuple during updatesPartitioning vector: first element including, following element excluding
\end{itemize}

\note[item]{}
\end{frame}
\begin{frame}
\frametitle{Unrelated Title}

\begin{center}
\includegraphics[width=0.9\textwidth,height=0.9\textheight,keepaspectratio]{/Users/I516998/Library/Application Support/Anki2/User 1/collection.media/paste-c6192b1e091f393d3ba32935fb756f0460f02bb7.jpg}
\end{center}


\note[item]{}
\end{frame}
\begin{frame}
\frametitle{Unrelated Title}

\begin{center}
\includegraphics[width=0.9\textwidth,height=0.9\textheight,keepaspectratio]{/Users/I516998/Library/Application Support/Anki2/User 1/collection.media/paste-551180c1698453208d4776f304a45bcd72f988c4.jpg}
\end{center}


\note[item]{}
\end{frame}
\begin{frame}
\frametitle{Unrelated Title}

\begin{center}
\includegraphics[width=0.9\textwidth,height=0.9\textheight,keepaspectratio]{/Users/I516998/Library/Application Support/Anki2/User 1/collection.media/paste-6163a4f5645f54aff79bc4048ad21333cb28d45e.jpg}
\end{center}

\begin{itemize}
\item E ist the owner-fragmentationP is the member-fragmenten, because the fragmentation P is derived from the fragmentation of E
\end{itemize}

\note[item]{}
\end{frame}
\begin{frame}
\frametitle{Unrelated Title}


\begin{itemize}
\item Completeness: Need referential integrity (every value of the member occurs in the owner)Disjointness: Join attribute should be key of ownerCommon special case: R and S have same partioning rule (on a join attribute)Then R and S are co-partioned
\end{itemize}

\note[item]{}
\end{frame}
\begin{frame}
\frametitle{Unrelated Title}

\begin{center}
\includegraphics[width=0.9\textwidth,height=0.9\textheight,keepaspectratio]{/Users/I516998/Library/Application Support/Anki2/User 1/collection.media/paste-ad7943d4a306d1276023b5cb95f8fdacb1b02606.jpg}
\end{center}

\begin{itemize}
\item We need the primary key in each fragment to reconstruct the relation. If the primary key is repeated in all fragment, the reconstruction is possible with a natural join
\end{itemize}

\note[item]{}
\end{frame}
\begin{frame}
\frametitle{Unrelated Title}


\begin{itemize}
\item The attribute affinity matrix describes how commonly two attributes are accessed together?
\end{itemize}

\note[item]{}
\end{frame}
\begin{frame}
\frametitle{Unrelated Title}


\begin{itemize}
\item What kind of queries? (read-only or write)Where do queries originate?What is the storage capacity and cost at the sites?What is the processing power at sites?What is the communication cost?What is the query processing strategy?Do we replicate fragments? 
\end{itemize}

\note[item]{}
\end{frame}
\begin{frame}
\frametitle{Unrelated Title}


\begin{itemize}
\item What is the best placement and best number of copies of each fragment?
\item Minimize query response timeOr maximize throughputOr minimize "some cost"Possible constrints: available storage, bandwith or power
\end{itemize}

\note[item]{}
\end{frame}
\begin{frame}
\frametitle{Unrelated Title}


\begin{itemize}
\item Transparency = separation of high-level semantics from implementation (hide implementation details from user)Multiple transparency layers: fragmentation, allocation and network transparency
\end{itemize}

\note[item]{}
\end{frame}
\begin{frame}
\frametitle{Unrelated Title}


\begin{itemize}
\item Queries are placed against global schemaFragmentation and allocation are hidden
\end{itemize}

\note[item]{}
\end{frame}
\begin{frame}
\frametitle{Unrelated Title}


\begin{itemize}
\item Queries are placed against fragmentation schemaApplications need to know fragmentation and place their queries against fragmentsAllocation still hidden
\end{itemize}

\note[item]{}
\end{frame}
\begin{frame}
\frametitle{Unrelated Title}


\begin{itemize}
\item Queries are placed against allocation schemaApplications need to know fragmentation and place their queries against fragmentsApplications als decide which fragments to use
\end{itemize}

\note[item]{}
\end{frame}
\begin{frame}
\frametitle{Unrelated Title}


\begin{itemize}
\item Ease of useStability (configuration changes)Ability to control query processing by applicationsDifficulty and overhead cost of providing transparency
\end{itemize}

\note[item]{}
\end{frame}
\begin{frame}
\frametitle{Unrelated Title}

\begin{center}
\includegraphics[width=0.9\textwidth,height=0.9\textheight,keepaspectratio]{/Users/I516998/Library/Application Support/Anki2/User 1/collection.media/Bildschirmfoto 2022-09-21 um 11.16.38.png}
\includegraphics[width=0.9\textwidth,height=0.9\textheight,keepaspectratio]{/Users/I516998/Library/Application Support/Anki2/User 1/collection.media/Bildschirmfoto 2022-09-21 um 11.16.45.png}
\end{center}


\note[item]{}
\end{frame}
\begin{frame}
\frametitle{Unrelated Title}

\begin{center}
\includegraphics[width=0.9\textwidth,height=0.9\textheight,keepaspectratio]{/Users/I516998/Library/Application Support/Anki2/User 1/collection.media/Bildschirmfoto 2022-09-21 um 11.22.08.png}
\end{center}

\begin{itemize}
\item Trait: one-directional (Height, Intelligence, Fluency, ...)Process: bi-directional (Interaction)
\end{itemize}

\note[item]{}
\end{frame}
\begin{frame}
\frametitle{Unrelated Title}


\begin{itemize}
\item Assigned: based on occupying a position within an organization (Team leaders, managers, directors)Emergent: individual perceived by others as the most influential member of a group/organization, regardless of title (emerges over time through communication behaviours)verbal involvementbeing informed seeking opinionsbeing firm but not rigid
\end{itemize}

\note[item]{}
\end{frame}
\begin{frame}
\frametitle{Unrelated Title}


\begin{itemize}
\item associated with both: high intelligence, dominance, general self-efficacy, self-monitoring
\end{itemize}

\note[item]{}
\end{frame}
\begin{frame}
\frametitle{Unrelated Title}

\begin{center}
\includegraphics[width=0.9\textwidth,height=0.9\textheight,keepaspectratio]{/Users/I516998/Library/Application Support/Anki2/User 1/collection.media/Bildschirmfoto 2022-09-21 um 11.34.41.png}
\end{center}


\note[item]{}
\end{frame}
\begin{frame}
\frametitle{Unrelated Title}


\begin{itemize}
\item - examples: Hitler, Jim Jones, David Koresh
\end{itemize}

\note[item]{}
\end{frame}
\begin{frame}
\frametitle{Unrelated Title}

\begin{center}
\includegraphics[width=0.9\textwidth,height=0.9\textheight,keepaspectratio]{/Users/I516998/Library/Application Support/Anki2/User 1/collection.media/Bildschirmfoto 2022-09-21 um 11.37.36.png}
\end{center}

\begin{itemize}
\item Leadership = neutral processMachiavelli: no moral values in decision makingleadership is value-neutral, usable for good or badLeadership = moral processleadership includes ethical considerations (morals, values, goals of followers)
\end{itemize}

\note[item]{}
\end{frame}
\begin{frame}
\frametitle{Unrelated Title}


\begin{itemize}
\item Management - Order and ConsistencyPlanning and BudgetingOrganizing and StaffingControlling and Problem SolvingLeadership - Change and MovementEstablish DirectionAligning PeopleMotivating and Inspiring
\end{itemize}

\note[item]{}
\end{frame}
\begin{frame}
\frametitle{Unrelated Title}


\begin{itemize}
\item All terms that are in the collection (the vocabulary)
\item Excep those which are ignored in preprocessing (stopwords)
\end{itemize}

\note[item]{}
\end{frame}
\begin{frame}
\frametitle{Unrelated Title}


\begin{itemize}
\item Grep all documents for "true boolean expressions" and then take out the documents containing the "negated boolean expressions"
\item Slow for large corporaDoes not support other types of information needs (e.g. find documents where a word appears near to another word)
\end{itemize}

\note[item]{}
\end{frame}
\begin{frame}
\frametitle{Unrelated Title}


\begin{itemize}
\item Queries are boolean expressions and the search engine returns all documents from the collection that satisfy the boolean expression
\end{itemize}

\note[item]{}
\end{frame}
\begin{frame}
\frametitle{Unrelated Title}


\begin{itemize}
\item Query representationQuery is given as a propositional logic formula over index termsIndex terms are connected via boolean operatorsQueries can be transformed into disjunctive normal form (DNF)Document representationEach document in the collection is represented as a bag of wordsThe frequency of terms is irrelevant, only wether the term appears in the documentRelevance of document for queryDocument is relevant if it satisfies the propositional logic formula of the query (in DNF this means at least one on the conjunctive components is satisfied)
\end{itemize}

\note[item]{}
\end{frame}
\begin{frame}
\frametitle{Unrelated Title}


\begin{itemize}
\item Boolean retrieval returns matching documents in no particular orderWell-designed search engines need to rank the relevant results
\end{itemize}

\note[item]{}
\end{frame}
\begin{frame}
\frametitle{Unrelated Title}

\begin{center}
\includegraphics[width=0.9\textwidth,height=0.9\textheight,keepaspectratio]{/Users/I516998/Library/Application Support/Anki2/User 1/collection.media/paste-51786d0a6118ad0e770259db2a15c39173d389fc.jpg}
\end{center}

\begin{itemize}
\item The incidence matric can be used to answer boolean queries. It tells which words appear in which documents.
\end{itemize}

\note[item]{}
\end{frame}
\begin{frame}
\frametitle{Unrelated Title}


\begin{itemize}
\item Create 0/1 vector for each index term from matrixTake vectors or inverted vectors for termsPerform bitwise conjunction of these vectors
\end{itemize}

\note[item]{}
\end{frame}
\begin{frame}
\frametitle{Unrelated Title}


\begin{itemize}
\item Good for small collectionsBad for large collections (incidence matrix is very sparse)Possible optimization: Only store positions of 1's, the rest are 0's
\end{itemize}

\note[item]{}
\end{frame}
\begin{frame}
\frametitle{Unrelated Title}


\begin{itemize}
\item Efficient structure for storing term indices and for computationally efficient retrievalLess storage needed, but effiecient algorithms for finding relevant documentsInverted index contains a list of references to documents for all index termsThe list of documents that contain the termin is called posting listDocument are represented with thei identifier numbers
\end{itemize}

\note[item]{}
\end{frame}
\begin{frame}
\frametitle{Unrelated Title}


\begin{itemize}
\item For efficient merging of multiple lists: If posting lists are sorted, the time of the merge is linear in the total number of posting entriesThe merge is performed by simoultaneously walking thorugh the two postings -> O(x+y)If the posting list would not be sorted, the merge complexity would be quadratic
\end{itemize}

\note[item]{}
\end{frame}
\begin{frame}
\frametitle{Unrelated Title}

\begin{center}
\includegraphics[width=0.9\textwidth,height=0.9\textheight,keepaspectratio]{/Users/I516998/Library/Application Support/Anki2/User 1/collection.media/paste-1e2c9c5e8f3c9db444056748a6913ef7ef8d065c.jpg}
\end{center}


\note[item]{}
\end{frame}
\begin{frame}
\frametitle{Unrelated Title}


\begin{itemize}
\item Yes, it does matterGood general heurstic: Start merging from the shortest posting and perform merges in increasing order of posting length
\end{itemize}

\note[item]{}
\end{frame}
\begin{frame}
\frametitle{Unrelated Title}

\begin{center}
\includegraphics[width=0.9\textwidth,height=0.9\textheight,keepaspectratio]{/Users/I516998/Library/Application Support/Anki2/User 1/collection.media/paste-850772565bd6709c0b4847c5add0692fb84de9f1.jpg}
\end{center}


\note[item]{}
\end{frame}
\begin{frame}
\frametitle{Unrelated Title}


\begin{itemize}
\item With read-only indicesAdding skip pointers to the postings speeds up the merge  
\end{itemize}

\note[item]{}
\end{frame}
\begin{frame}
\frametitle{Unrelated Title}


\begin{itemize}
\item The idea is to skip parts of posting lists that lead to empty resultsInsead of going linearly through the list, we are jumping via pointers
\end{itemize}

\note[item]{}
\end{frame}
\begin{frame}
\frametitle{Unrelated Title}

\begin{center}
\includegraphics[width=0.9\textwidth,height=0.9\textheight,keepaspectratio]{/Users/I516998/Library/Application Support/Anki2/User 1/collection.media/paste-29c21013a4bb5b119770c38e893ba13bacd9207b.jpg}
\end{center}


\note[item]{}
\end{frame}
\begin{frame}
\frametitle{Unrelated Title}


\begin{itemize}
\item More skipsShorter skip spansMore likely to skip, but a lot of skip position comparisionsMore data to store (larger index)Fewer skipsLonger skip spansLess likely to skip, but fewer pointer comparisonsLess data to store (smaller index)
\end{itemize}

\note[item]{}
\end{frame}
\begin{frame}
\frametitle{Unrelated Title}


\begin{itemize}
\item For postings of length L use sqr(L) evenly spaced skip pointersEasy to implement of the index is read-onlyMaintenance required when index is updatedThe heuristic ignores the distribution of terms over indexed documents
\end{itemize}

\note[item]{}
\end{frame}
\begin{frame}
\frametitle{Unrelated Title}


\begin{itemize}
\item Some phrases are not meant to be split into termsIt is no longer enough to store only posting lists for indivudal terms
\end{itemize}

\note[item]{}
\end{frame}
\begin{frame}
\frametitle{Unrelated Title}


\begin{itemize}
\item pairs of consecutive terms are indexedEach of the biwords becomes an index termQuery is also transformed into biwords for lookup and merging
\end{itemize}

\note[item]{}
\end{frame}
\begin{frame}
\frametitle{Unrelated Title}


\begin{itemize}
\item The majority of biwords are not real world conceptsNumber of biwords is larger than the number of terms (combinatorial explosion)
\end{itemize}

\note[item]{}
\end{frame}
\begin{frame}
\frametitle{Unrelated Title}


\begin{itemize}
\item Biwords have to satisfy certain part-of-speech patternse.g. all sequences of POS tags of the form NX*N (N=noun, X=preposition)catcher in the rye (NXXN) --> lookup "catcher rye"
\end{itemize}

\note[item]{}
\end{frame}
\begin{frame}
\frametitle{Unrelated Title}


\begin{itemize}
\item Indexing biwords can lead to false positivesLarge index (combinatorial explosion)
\end{itemize}

\note[item]{}
\end{frame}
\begin{frame}
\frametitle{Unrelated Title}

\begin{center}
\includegraphics[width=0.9\textwidth,height=0.9\textheight,keepaspectratio]{/Users/I516998/Library/Application Support/Anki2/User 1/collection.media/paste-144ff68070e2b9852ee4ee2ef61c301ffb119642.jpg}
\end{center}

\begin{itemize}
\item A generall/extended n-word indexFor each document that contains the index term we store positions of all tokens of the term in the document 
\end{itemize}

\note[item]{}
\end{frame}
\begin{frame}
\frametitle{Unrelated Title}


\begin{itemize}
\item Fetach positional posting lists for each of the terms in the phase querymerge the posting list by considering not only documents but als term positions (for amtching documents)
\end{itemize}

\note[item]{}
\end{frame}
\begin{frame}
\frametitle{Unrelated Title}

\begin{center}
\includegraphics[width=0.9\textwidth,height=0.9\textheight,keepaspectratio]{/Users/I516998/Library/Application Support/Anki2/User 1/collection.media/paste-cf806d8cf82e37e1f9a1d37662c4119470093056.jpg}
\end{center}

\begin{itemize}
\item line 14 and 15 can be ignored
\end{itemize}

\note[item]{}
\end{frame}
\begin{frame}
\frametitle{Unrelated Title}


\begin{itemize}
\item Users define how far apart the query terms may be from each otherthe parameter k is the distance and means "with k words from"Positional indices can be leveraged also for proximity queries
\end{itemize}

\note[item]{}
\end{frame}
\begin{frame}
\frametitle{Unrelated Title}


\begin{itemize}
\item For each term and each document in which it appears we store only (optionally) the frequency with which the term appears in the document
\item 1 integer per term-document pair
\end{itemize}

\note[item]{}
\end{frame}
\begin{frame}
\frametitle{Unrelated Title}


\begin{itemize}
\item  There is only a small set of words that appear often in documents and these words are often removed because they are stopwordsIn average words will only appear in 2-4 documents (this approximization holds only for "English-like" languagesPositional index size is 35-50% of the size of the original text
\end{itemize}

\note[item]{}
\end{frame}
\begin{frame}
\frametitle{Unrelated Title}


\begin{itemize}
\item AdvantagesSimplicity and computational efficiencyPopular in early commercial systemsDisadvantagesExpressing information needs as Boolean expressions is unintuitivePure IR model (no ranking, relative importance of terms is ignored)
\end{itemize}

\note[item]{}
\end{frame}
\begin{frame}
\frametitle{Unrelated Title}

\begin{center}
\includegraphics[width=0.9\textwidth,height=0.9\textheight,keepaspectratio]{/Users/I516998/Library/Application Support/Anki2/User 1/collection.media/paste-7b5a9d185a0e5f64aaf14a84dd7d5411bf03e5de.jpg}
\end{center}


\note[item]{}
\end{frame}
\begin{frame}
\frametitle{Unrelated Title}


\begin{itemize}
\item Same as in centralized systems:
\item NormalizationConvert to standard formDetect incorrect queriesSimplificationRemove redundanciesDetect common subexpressionsAlgebraic rewritingPush selections downPush projections down
\end{itemize}

\note[item]{}
\end{frame}
\begin{frame}
\frametitle{Unrelated Title}


\begin{itemize}
\item Obtain suitable algebraic query tree on fragmentsTakes into account information about the fragmentation of the data
\end{itemize}

\note[item]{}
\end{frame}
\begin{frame}
\frametitle{Unrelated Title}


\begin{itemize}
\item The query to reconstruct a global relation from fragments
\end{itemize}

\note[item]{}
\end{frame}
\begin{frame}
\frametitle{Unrelated Title}


\begin{itemize}
\item Replace each relation by its localization program (producing a localized query tree)Apply reduction techniques (push unions up, predicates/projections down)Simplify (eliminate unnecessary operations)
\end{itemize}

\note[item]{}
\end{frame}
\begin{frame}
\frametitle{Unrelated Title}

\begin{center}
\includegraphics[width=0.9\textwidth,height=0.9\textheight,keepaspectratio]{/Users/I516998/Library/Application Support/Anki2/User 1/collection.media/paste-ef2ee0603239062b2ad4b1c6103b9dca02386410.jpg}
\end{center}

\begin{itemize}
\item Selections and projections can be pushed down unions
\end{itemize}

\note[item]{}
\end{frame}
\begin{frame}
\frametitle{Unrelated Title}

\begin{center}
\includegraphics[width=0.9\textwidth,height=0.9\textheight,keepaspectratio]{/Users/I516998/Library/Application Support/Anki2/User 1/collection.media/paste-bd1a587474efd168959b2282641f9e2d7e4f64fb.jpg}
\end{center}

\begin{itemize}
\item Can be used for horizontal fragmentationEliminates empty selections
\end{itemize}

\note[item]{}
\end{frame}
\begin{frame}
\frametitle{Unrelated Title}

\begin{center}
\includegraphics[width=0.9\textwidth,height=0.9\textheight,keepaspectratio]{/Users/I516998/Library/Application Support/Anki2/User 1/collection.media/paste-ca2b74277277c591de6815c10acaa1db4937466c.jpg}
\end{center}

\begin{itemize}
\item A join of unions can be rewritten into a union of joinsResults in more join and union operations (generallay 1 join, n+m unions --> n*m joins, n*m unions)Advantages:operations can be parallelized and dsitributed across multiple sitessometimes some of the joins can be eliminated from the query tree
\end{itemize}

\note[item]{}
\end{frame}
\begin{frame}
\frametitle{Unrelated Title}

\begin{center}
\includegraphics[width=0.9\textwidth,height=0.9\textheight,keepaspectratio]{/Users/I516998/Library/Application Support/Anki2/User 1/collection.media/paste-77f00884be88b3ca1cced919ef725fc0be584704.jpg}
\end{center}

\begin{itemize}
\item For primary horizontal fragmentationEliminates empty joins
\end{itemize}

\note[item]{}
\end{frame}
\begin{frame}
\frametitle{Unrelated Title}

\begin{center}
\includegraphics[width=0.9\textwidth,height=0.9\textheight,keepaspectratio]{/Users/I516998/Library/Application Support/Anki2/User 1/collection.media/paste-2d05b6773fc644c7d10be096d800a1ba04311eb9.jpg}
\end{center}

\begin{itemize}
\item Eliminates fragments with irrelevant columns
\end{itemize}

\note[item]{}
\end{frame}
\begin{frame}
\frametitle{Unrelated Title}


\begin{itemize}
\item Manages connections between components (e.g. clients and query processor)Load balancing
\end{itemize}

\note[item]{}
\end{frame}
\begin{frame}
\frametitle{Unrelated Title}


\begin{itemize}
\item Retrieves and manages client queriesUses metatinformation about data, queries and transactions (from the metadata directory)Query compilation/optimizationMonitor queryTransaction execution
\end{itemize}

\note[item]{}
\end{frame}
\begin{frame}
\frametitle{Unrelated Title}


\begin{itemize}
\item Stores and manages dataLow-level functions (implementation of database operators)Multiple data processor can be involved in a single query
\end{itemize}

\note[item]{}
\end{frame}
\begin{frame}
\frametitle{Unrelated Title}

\begin{center}
\includegraphics[width=0.9\textwidth,height=0.9\textheight,keepaspectratio]{/Users/I516998/Library/Application Support/Anki2/User 1/collection.media/paste-efd5dc5ccafc1e2dc82b0dab9e9c1e632f38cfab.jpg}
\end{center}

\begin{itemize}
\item QEP = Query execution planThis step is performed by a query processor
\end{itemize}

\note[item]{}
\end{frame}
\begin{frame}
\frametitle{Unrelated Title}


\begin{itemize}
\item Search spaceNew operators for communicationNew algorithms for some operations (sort, join,...)New transformation rulesDistributed cost modelMain components CPU, I/O, communicationDistrbuted DB: communication often dominantParallel DB: consider all threeCost model: Reponse time vs total costLoad balancing importantSearch strategyNew optimization rulesOften hybrid strategies (first static based on cost model, later runtime parameter determined at execution time)Load balancing important: many ways to assign and schedule processors
\end{itemize}

\note[item]{}
\end{frame}
\begin{frame}
\frametitle{Unrelated Title}


\begin{itemize}
\item Parallel execution of multiple queriesGoal: increase throughput
\end{itemize}

\note[item]{}
\end{frame}
\begin{frame}
\frametitle{Unrelated Title}


\begin{itemize}
\item Parallel execution of a single queryGoal: decrease response timeTypes: intra-operator, inter-operator
\end{itemize}

\note[item]{}
\end{frame}
\begin{frame}
\frametitle{Unrelated Title}

\begin{center}
\includegraphics[width=0.9\textwidth,height=0.9\textheight,keepaspectratio]{/Users/I516998/Library/Application Support/Anki2/User 1/collection.media/paste-5cbd79b65b07e56718b6b83819e20dcd297bda45.jpg}
\end{center}

\begin{itemize}
\item Operator decomposed into multiple instancesEach running in parallel on a partition of the data
\end{itemize}

\note[item]{}
\end{frame}
\begin{frame}
\frametitle{Unrelated Title}


\begin{itemize}
\item Multiple operators executed in parallelIndependent executionNo dependencies between operators (e.g. two selects)Pipelined execution (dependencies)Producer/consumer (e.g. join and multiple selects)
\end{itemize}

\note[item]{}
\end{frame}
\begin{frame}
\frametitle{Unrelated Title}

\begin{center}
\includegraphics[width=0.9\textwidth,height=0.9\textheight,keepaspectratio]{/Users/I516998/Library/Application Support/Anki2/User 1/collection.media/paste-7b6dfea4954947a357b3c809985022d840087315.jpg}
\end{center}

\begin{itemize}
\item R is range partitioned on sort attribute KEach fragment is sorted individuallyIf necessary, ship results
\end{itemize}

\note[item]{}
\end{frame}
\begin{frame}
\frametitle{Unrelated Title}


\begin{itemize}
\item Shared nothing: No communication needed for parallel computationShared memory: No saved communication costs, but saved compute time
\end{itemize}

\note[item]{}
\end{frame}
\begin{frame}
\frametitle{Unrelated Title}


\begin{itemize}
\item R is fragmented on multiple disks or sites, but not range-partitioned on sort attribute K
\item Range partition on KRun basic sortIf necessary, ship results
\end{itemize}

\note[item]{}
\end{frame}
\begin{frame}
\frametitle{Unrelated Title}

\begin{center}
\includegraphics[width=0.9\textwidth,height=0.9\textheight,keepaspectratio]{/Users/I516998/Library/Application Support/Anki2/User 1/collection.media/paste-80eabf9045a2a9a92a2005884e6d9bbcd671c92c.jpg}
\end{center}

\begin{itemize}
\item Same as parallel range partitioning sort, but sort firstR1 uses merge operation instead of sort operationMerge operation supports pipeliningMerge operation can be slow if the number of source nodes is large (final n-way merge)
\end{itemize}

\note[item]{}
\end{frame}
\begin{frame}
\frametitle{Unrelated Title}


\begin{itemize}
\item Bad partitioning vector leads to partitioning skewIdeally quantilesSort data and use n paritions -> take values after every 1/n-th of the dataProblem: quantile computationSorting is expensiveHeuristics are riskyBetter: compute histogramsUnderlying problem: distributed selection
\end{itemize}

\note[item]{}
\end{frame}
\begin{frame}
\frametitle{Unrelated Title}

\begin{center}
\includegraphics[width=0.9\textwidth,height=0.9\textheight,keepaspectratio]{/Users/I516998/Library/Application Support/Anki2/User 1/collection.media/Bildschirmfoto 2022-09-26 um 14.58.03.png}
\end{center}


\note[item]{}
\end{frame}
\begin{frame}
\frametitle{Unrelated Title}


\begin{itemize}
\item Great Man Theories (Early 1900s) (dominant leaders)Shift towards relationship and social situation (Intelligence, Masculinity)Meta-analysis extension -> Six traitsTraits associated with social intelligence (Cognitive abilities, emotional intelligence) (2004) Zaccaro et al. 
\end{itemize}

\note[item]{}
\end{frame}
\begin{frame}
\frametitle{Unrelated Title}


\begin{itemize}
\item All mentioned traits (e.g. Intelligence, Creative Thinking etc. ) contribute to a certain extend to leader emergence and effectiveness. There is no one golden trait that is 100% responsible and rules for emergence and effectiveness.Most show correlation between 0.2 and 0.3 (Medium)
\end{itemize}

\note[item]{}
\end{frame}
\begin{frame}
\frametitle{Unrelated Title}


\begin{itemize}
\item Neurotiscism (anxious, insecure) Extraversion (sociable, positive energy)Openness (informed, curious)Agreeableness (accepting, trusting)Conscientiousness (organized, controlled)
\end{itemize}

\note[item]{}
\end{frame}
\begin{frame}
\frametitle{Unrelated Title}

\begin{center}
\includegraphics[width=0.9\textwidth,height=0.9\textheight,keepaspectratio]{/Users/I516998/Library/Application Support/Anki2/User 1/collection.media/Bildschirmfoto 2022-09-26 um 15.31.44.png}
\includegraphics[width=0.9\textwidth,height=0.9\textheight,keepaspectratio]{/Users/I516998/Library/Application Support/Anki2/User 1/collection.media/Bildschirmfoto 2022-09-26 um 15.35.02.png}
\end{center}

\begin{itemize}
\item Effectiveness:
\item low positive correlation between 4 of the big 5 and leadership effectivenessException: Neuroticism Highest: Extraversion-> Regression
\end{itemize}

\note[item]{}
\end{frame}
\begin{frame}
\frametitle{Unrelated Title}


\begin{itemize}
\item focuses exclusively on leadercertrain traits are crucial for effective leadershipapplication: use of personality assessments to find designated leadership profilesprovides directionpinpoints strengths and weaknessesassesses current standing in organization
\end{itemize}

\note[item]{}
\end{frame}
\begin{frame}
\frametitle{Unrelated Title}


\begin{itemize}
\item + Intuitively appealing
\item + long research tradition
\item + gives benchmark for leadership selection
\item - no definitive list of traits, invites formulation of new lists
\item - no situation/followers
\item - not useful for training/development
\end{itemize}

\note[item]{}
\end{frame}
\begin{frame}
\frametitle{Unrelated Title}

\begin{center}
\includegraphics[width=0.9\textwidth,height=0.9\textheight,keepaspectratio]{/Users/I516998/Library/Application Support/Anki2/User 1/collection.media/Bildschirmfoto 2022-09-26 um 16.19.47.png}
\end{center}

\begin{itemize}
\item Shift from innate, stable personality characteristics (traits) to learned and developed skillsStill leader-centered
\end{itemize}

\note[item]{}
\end{frame}
\begin{frame}
\frametitle{Unrelated Title}


\begin{itemize}
\item based on own observations of executivessees leadership as a set of developable skills(skills = what leaders can accomplish, traits = who leaders are)Technical Skills Human Skills Conceptual Skills 
\end{itemize}

\note[item]{}
\end{frame}
\begin{frame}
\frametitle{Unrelated Title}

\begin{center}
\includegraphics[width=0.9\textwidth,height=0.9\textheight,keepaspectratio]{/Users/I516998/Library/Application Support/Anki2/User 1/collection.media/Bildschirmfoto 2022-09-26 um 16.24.47.png}
\end{center}


\note[item]{}
\end{frame}
\begin{frame}
\frametitle{Unrelated Title}

\begin{center}
\includegraphics[width=0.9\textwidth,height=0.9\textheight,keepaspectratio]{/Users/I516998/Library/Application Support/Anki2/User 1/collection.media/Bildschirmfoto 2022-09-26 um 16.29.14.png}
\includegraphics[width=0.9\textwidth,height=0.9\textheight,keepaspectratio]{/Users/I516998/Library/Application Support/Anki2/User 1/collection.media/Bildschirmfoto 2022-09-26 um 16.36.59.png}
\end{center}

\begin{itemize}
\item Attributes = TraitsCompetencies = Skills
\end{itemize}

\note[item]{}
\end{frame}
\begin{frame}
\frametitle{Unrelated Title}


\begin{itemize}
\item General cognitive ability (Intelligence)Crystallized cognitive ability (ability learned over time)Motivation (Social Good)Personality (other)
\end{itemize}

\note[item]{}
\end{frame}
\begin{frame}
\frametitle{Unrelated Title}


\begin{itemize}
\item Problem solving SkillsSocial judgement Skills (social perceptiveness)Knowledge
\end{itemize}

\note[item]{}
\end{frame}
\begin{frame}
\frametitle{Unrelated Title}


\begin{itemize}
\item Effective problem solving (originality and quality of solutions: logical, effective, unique)Performance 
\end{itemize}

\note[item]{}
\end{frame}
\begin{frame}
\frametitle{Unrelated Title}


\begin{itemize}
\item Challenging AssignmentsMentoring Appropriate trainingHands-on experience with novelty
\end{itemize}

\note[item]{}
\end{frame}
\begin{frame}
\frametitle{Unrelated Title}


\begin{itemize}
\item Factors that lie outside leaders' control
\item Social Environment (poor social interaction patterns within team)Technical environment (outdated information / information systems)Inequality in opportunitiesPoor Background growing upCovidvery simple stuff ... 
\end{itemize}

\note[item]{}
\end{frame}
\begin{frame}
\frametitle{Unrelated Title}

\begin{center}
\includegraphics[width=0.9\textwidth,height=0.9\textheight,keepaspectratio]{/Users/I516998/Library/Application Support/Anki2/User 1/collection.media/Bildschirmfoto 2022-09-26 um 16.49.46.png}
\end{center}

\begin{itemize}
\item Distal Attributes: general, stable, traitsProximal Attributes: task-related, skillsLeader Processes: behaviour
\end{itemize}

\note[item]{}
\end{frame}
\begin{frame}
\frametitle{Unrelated Title}


\begin{itemize}
\item descriptivestructures the nature of effective leadership2 principle research perspective:Katz 1955 -> importance of skills varie depending on management hierarchyMumford et al. (2000) -> leadership outcomes are direct result of leaders skilled competency and indirect result of their attributesApplication: not really applied, lack of formal training programs, but helps identify strengths/weaknesses of leaders
\end{itemize}

\note[item]{}
\end{frame}
\begin{frame}
\frametitle{Unrelated Title}


\begin{itemize}
\item + shows importance of skill developing
\item + first approach of conceptualizing the process of leadership around skills
\item + expansive view of leadership
\item - adresses more than just leadership -> less precise
\item - weak in predictive value, no explanation of HOW skills lead to performance
\item - Skill Model: trait-driven, based on army data
\end{itemize}

\note[item]{}
\end{frame}
\begin{frame}
\frametitle{Unrelated Title}


\begin{itemize}
\item A theta join is a join where R and S are arbitrarily partitioned and where the join predicate p is arbitrary
\end{itemize}

\note[item]{}
\end{frame}
\begin{frame}
\frametitle{Unrelated Title}


\begin{itemize}
\item The costs
\end{itemize}

\note[item]{}
\end{frame}
\begin{frame}
\frametitle{Unrelated Title}

\begin{center}
\includegraphics[width=0.9\textwidth,height=0.9\textheight,keepaspectratio]{/Users/I516998/Library/Application Support/Anki2/User 1/collection.media/paste-a38b5dfa326b1a0a0da534731af0a52de0038b83.jpg}
\includegraphics[width=0.9\textwidth,height=0.9\textheight,keepaspectratio]{/Users/I516998/Library/Application Support/Anki2/User 1/collection.media/paste-bdca5e0c985f35bd746caa8af0ec2b74061191a8.jpg}
\includegraphics[width=0.9\textwidth,height=0.9\textheight,keepaspectratio]{/Users/I516998/Library/Application Support/Anki2/User 1/collection.media/paste-de5245d2de6b2e6517f145cf42345e1fe792cd2e.jpg}
\end{center}

\begin{itemize}
\item Replicate relation R to each S-node (broadcast)Each S-node computes the local join Join result ditributed over S-nodes
\end{itemize}

\note[item]{}
\end{frame}
\begin{frame}
\frametitle{Unrelated Title}


\begin{itemize}
\item SimpleGood for theta-joinsInefficient for equi-joins
\end{itemize}

\note[item]{}
\end{frame}
\begin{frame}
\frametitle{Unrelated Title}

\begin{center}
\includegraphics[width=0.9\textwidth,height=0.9\textheight,keepaspectratio]{/Users/I516998/Library/Application Support/Anki2/User 1/collection.media/paste-f74ba5057d4abd67f3d2ba8a5d4f7b9c77135eb8.jpg}
\includegraphics[width=0.9\textwidth,height=0.9\textheight,keepaspectratio]{/Users/I516998/Library/Application Support/Anki2/User 1/collection.media/paste-7e52ccc57cbe490a286913c263aa11d7d1c3340c.jpg}
\end{center}

\begin{itemize}
\item (Re)partition R and S on join attribute AUse the same partitioning funtion for R and S
\end{itemize}

\note[item]{}
\end{frame}
\begin{frame}
\frametitle{Unrelated Title}


\begin{itemize}
\item When using has joins locally (for joins on a node), use a different hash function for global relation --> otherweise the hash join would boil down to a nested loop join, because all tuples would end up in the same bucket
\end{itemize}

\note[item]{}
\end{frame}
\begin{frame}
\frametitle{Unrelated Title}

\begin{center}
\includegraphics[width=0.9\textwidth,height=0.9\textheight,keepaspectratio]{/Users/I516998/Library/Application Support/Anki2/User 1/collection.media/paste-6de5a2748c762469a526888903e05f07820b51c6.jpg}
\end{center}

\begin{itemize}
\item S already partitioned on the join attribute APartition R conformingly
\end{itemize}

\note[item]{}
\end{frame}
\begin{frame}
\frametitle{Unrelated Title}


\begin{itemize}
\item Careful co-partitioning and co-locationing (imporant if multiple relations can be co-partiotioned and if we have to choose which ones)
\end{itemize}

\note[item]{}
\end{frame}
\begin{frame}
\frametitle{Unrelated Title}


\begin{itemize}
\item Reduce the communication in distributed equi joinsAvoid transmitting tuples that do not join with other tuplesTwo approaches: Semi join filtering and bloom filters
\end{itemize}

\note[item]{}
\end{frame}
\begin{frame}
\frametitle{Unrelated Title}

\begin{center}
\includegraphics[width=0.9\textwidth,height=0.9\textheight,keepaspectratio]{/Users/I516998/Library/Application Support/Anki2/User 1/collection.media/paste-16d08413bb1c9953b0a48c7f90831d373d48b168.jpg}
\end{center}

\begin{itemize}
\item Send keys from R to S-nodePerform semi join (to get only tuples from S that have a corresponding tuple in R)Transfer semi join result to R-node and perform join on R-node
\end{itemize}

\note[item]{}
\end{frame}
\begin{frame}
\frametitle{Unrelated Title}

\begin{center}
\includegraphics[width=0.9\textwidth,height=0.9\textheight,keepaspectratio]{/Users/I516998/Library/Application Support/Anki2/User 1/collection.media/paste-6210673c5fc2282d6d75bc5338334984b994012c.jpg}
\end{center}

\begin{itemize}
\item Only communication costs are consideredsize of S and R often needs to be esstimatedSimilar techniques for n-way joins, but many variants (exponential)
\end{itemize}

\note[item]{}
\end{frame}
\begin{frame}
\frametitle{Unrelated Title}

\begin{center}
\includegraphics[width=0.9\textwidth,height=0.9\textheight,keepaspectratio]{/Users/I516998/Library/Application Support/Anki2/User 1/collection.media/paste-4c509b01c6cb247b2b5b6f35e394b8c5937c29e0.jpg}
\end{center}

\begin{itemize}
\item A data structure to support set membership queriesGoal: make data structure as small as possible
\end{itemize}

\note[item]{}
\end{frame}
\begin{frame}
\frametitle{Unrelated Title}

\begin{center}
\includegraphics[width=0.9\textwidth,height=0.9\textheight,keepaspectratio]{/Users/I516998/Library/Application Support/Anki2/User 1/collection.media/paste-a16ea99282f879aa903ea82e17310d9f42fb2435.jpg}
\end{center}

\begin{itemize}
\item Send only the bloom filter which consists of m bits
\end{itemize}

\note[item]{}
\end{frame}
\begin{frame}
\frametitle{Unrelated Title}


\begin{itemize}
\item When the key value is large (e.g. 64 bit integer or string)For small keys the bloom filter is not beneficial
\end{itemize}

\note[item]{}
\end{frame}
\begin{frame}
\frametitle{Unrelated Title}

\begin{center}
\includegraphics[width=0.9\textwidth,height=0.9\textheight,keepaspectratio]{/Users/I516998/Library/Application Support/Anki2/User 1/collection.media/paste-89dbc34d3803ee8968d215806aa4d61a381552de.jpg}
\end{center}

\begin{itemize}
\item Repartition all fragments based on the grouping attributeLocally apply grouping operation 
\end{itemize}

\note[item]{}
\end{frame}
\begin{frame}
\frametitle{Unrelated Title}

\begin{center}
\includegraphics[width=0.9\textwidth,height=0.9\textheight,keepaspectratio]{/Users/I516998/Library/Application Support/Anki2/User 1/collection.media/paste-1ad614122e39669b06c8d9de84cd6f83e3f58f8f.jpg}
\end{center}

\begin{itemize}
\item Before repartitioning the data is grouped to obtain a single value for each key
\end{itemize}

\note[item]{}
\end{frame}
\begin{frame}
\frametitle{Unrelated Title}


\begin{itemize}
\item Sort (in parallel), then duplicates can be eliminated easily because they are next to each otherPartition tuple (range or hash), eliminate duplicates locally
\end{itemize}

\note[item]{}
\end{frame}
\begin{frame}
\frametitle{Unrelated Title}


\begin{itemize}
\item Most users find it difficult to wirte boolean queriesMost users cannot go through thousands of results the boolean retrieval enginge returns on large collections
\end{itemize}

\note[item]{}
\end{frame}
\begin{frame}
\frametitle{Unrelated Title}


\begin{itemize}
\item The ranking of the documents is based on the relevanceThe ranking function captures the extent of relevance of the document for the query
\end{itemize}

\note[item]{}
\end{frame}
\begin{frame}
\frametitle{Unrelated Title}

\begin{center}
\includegraphics[width=0.9\textwidth,height=0.9\textheight,keepaspectratio]{/Users/I516998/Library/Application Support/Anki2/User 1/collection.media/paste-2f3a9c10e85749e5239a5a950be55cb6aded5213.jpg}
\end{center}

\begin{itemize}
\item a measure of overlap of two sets
\end{itemize}

\note[item]{}
\end{frame}
\begin{frame}
\frametitle{Unrelated Title}


\begin{itemize}
\item Term frequency in each document is ignoredOverall frequency of the termin in the collection/language is ignored - rare terms are more informativeThere are more sophisticated ways to normalize for the document length
\end{itemize}

\note[item]{}
\end{frame}
\begin{frame}
\frametitle{Unrelated Title}


\begin{itemize}
\item A measure that denotes how frequently the term t appears in the document d
\end{itemize}

\note[item]{}
\end{frame}
\begin{frame}
\frametitle{Unrelated Title}


\begin{itemize}
\item No
\item Relevance does not increase linearly with term frequencyRaw term frequency does not account for document length
\end{itemize}

\note[item]{}
\end{frame}
\begin{frame}
\frametitle{Unrelated Title}


\begin{itemize}
\item rare terms are more informative than frequent termsDocuments frequency can be used (the number of documents in the collection to account for global frequency of terms)
\end{itemize}

\note[item]{}
\end{frame}
\begin{frame}
\frametitle{Unrelated Title}

\begin{center}
\includegraphics[width=0.9\textwidth,height=0.9\textheight,keepaspectratio]{/Users/I516998/Library/Application Support/Anki2/User 1/collection.media/paste-fb15e01bd7319d4b682c5ceb68fef472541116c1.jpg}
\end{center}

\begin{itemize}
\item The informativeness of the term is inversely proportional to the number of documents in the collection in which the term appears (less documents means bigger weight)
\end{itemize}

\note[item]{}
\end{frame}
\begin{frame}
\frametitle{Unrelated Title}


\begin{itemize}
\item We want to rank documents (and not sentences)e.g. collection frequency could be the same for two terms while document frequency has a high difference
\end{itemize}

\note[item]{}
\end{frame}
\begin{frame}
\frametitle{Unrelated Title}


\begin{itemize}
\item There is no other term that could be more relevant --> so only term frequency can define ranking
\end{itemize}

\note[item]{}
\end{frame}
\begin{frame}
\frametitle{Unrelated Title}


\begin{itemize}
\item It is adding an additional preprocessing stepCosine distance between two vectors is quadratically proprotional to the Euclidean distance between unit-normalized version of those vectorsRanking is the same
\end{itemize}

\note[item]{}
\end{frame}
\begin{frame}
\frametitle{Unrelated Title}

\begin{center}
\includegraphics[width=0.9\textwidth,height=0.9\textheight,keepaspectratio]{/Users/I516998/Library/Application Support/Anki2/User 1/collection.media/Bildschirmfoto 2022-09-28 um 21.08.50.png}
\end{center}


\note[item]{}
\end{frame}
\begin{frame}
\frametitle{Unrelated Title}

\begin{center}
\includegraphics[width=0.9\textwidth,height=0.9\textheight,keepaspectratio]{/Users/I516998/Library/Application Support/Anki2/User 1/collection.media/Bildschirmfoto 2022-09-28 um 21.09.47.png}
\end{center}


\note[item]{}
\end{frame}
\begin{frame}
\frametitle{Unrelated Title}

\begin{center}
\includegraphics[width=0.9\textwidth,height=0.9\textheight,keepaspectratio]{/Users/I516998/Library/Application Support/Anki2/User 1/collection.media/Bildschirmfoto 2022-09-28 um 21.10.33.png}
\end{center}


\note[item]{}
\end{frame}
\begin{frame}
\frametitle{Unrelated Title}

\begin{center}
\includegraphics[width=0.9\textwidth,height=0.9\textheight,keepaspectratio]{/Users/I516998/Library/Application Support/Anki2/User 1/collection.media/Bildschirmfoto 2022-09-28 um 21.11.42.png}
\end{center}


\note[item]{}
\end{frame}
\begin{frame}
\frametitle{Unrelated Title}

\begin{center}
\includegraphics[width=0.9\textwidth,height=0.9\textheight,keepaspectratio]{/Users/I516998/Library/Application Support/Anki2/User 1/collection.media/Bildschirmfoto 2022-09-28 um 21.13.24.png}
\end{center}


\note[item]{}
\end{frame}
\begin{frame}
\frametitle{Unrelated Title}

\begin{center}
\includegraphics[width=0.9\textwidth,height=0.9\textheight,keepaspectratio]{/Users/I516998/Library/Application Support/Anki2/User 1/collection.media/Bildschirmfoto 2022-09-28 um 21.14.02.png}
\end{center}


\note[item]{}
\end{frame}
\begin{frame}
\frametitle{Unrelated Title}

\begin{center}
\includegraphics[width=0.9\textwidth,height=0.9\textheight,keepaspectratio]{/Users/I516998/Library/Application Support/Anki2/User 1/collection.media/Bildschirmfoto 2022-09-28 um 21.15.10.png}
\end{center}


\note[item]{}
\end{frame}
\begin{frame}
\frametitle{Unrelated Title}

\begin{center}
\includegraphics[width=0.9\textwidth,height=0.9\textheight,keepaspectratio]{/Users/I516998/Library/Application Support/Anki2/User 1/collection.media/Bildschirmfoto 2022-09-28 um 21.16.37.png}
\end{center}


\note[item]{}
\end{frame}
\begin{frame}
\frametitle{Unrelated Title}

\begin{center}
\includegraphics[width=0.9\textwidth,height=0.9\textheight,keepaspectratio]{/Users/I516998/Library/Application Support/Anki2/User 1/collection.media/Bildschirmfoto 2022-09-28 um 21.18.04.png}
\includegraphics[width=0.9\textwidth,height=0.9\textheight,keepaspectratio]{/Users/I516998/Library/Application Support/Anki2/User 1/collection.media/Bildschirmfoto 2022-09-28 um 21.18.18.png}
\end{center}


\note[item]{}
\end{frame}
\begin{frame}
\frametitle{Unrelated Title}

\begin{center}
\includegraphics[width=0.9\textwidth,height=0.9\textheight,keepaspectratio]{/Users/I516998/Library/Application Support/Anki2/User 1/collection.media/Bildschirmfoto 2022-09-28 um 21.18.42.png}
\end{center}


\note[item]{}
\end{frame}
\begin{frame}
\frametitle{Unrelated Title}

\begin{center}
\includegraphics[width=0.9\textwidth,height=0.9\textheight,keepaspectratio]{/Users/I516998/Library/Application Support/Anki2/User 1/collection.media/Bildschirmfoto 2022-09-28 um 21.24.47.png}
\includegraphics[width=0.9\textwidth,height=0.9\textheight,keepaspectratio]{/Users/I516998/Library/Application Support/Anki2/User 1/collection.media/Bildschirmfoto 2022-09-28 um 21.28.00.png}
\end{center}


\note[item]{}
\end{frame}
\begin{frame}
\frametitle{Unrelated Title}

\begin{center}
\includegraphics[width=0.9\textwidth,height=0.9\textheight,keepaspectratio]{/Users/I516998/Library/Application Support/Anki2/User 1/collection.media/Bildschirmfoto 2022-09-28 um 21.29.04.png}
\end{center}


\note[item]{}
\end{frame}
\begin{frame}
\frametitle{Unrelated Title}

\begin{center}
\includegraphics[width=0.9\textwidth,height=0.9\textheight,keepaspectratio]{/Users/I516998/Library/Application Support/Anki2/User 1/collection.media/Bildschirmfoto 2022-09-28 um 21.31.21.png}
\includegraphics[width=0.9\textwidth,height=0.9\textheight,keepaspectratio]{/Users/I516998/Library/Application Support/Anki2/User 1/collection.media/Bildschirmfoto 2022-09-28 um 21.32.35.png}
\end{center}


\note[item]{}
\end{frame}
\begin{frame}
\frametitle{Unrelated Title}

\begin{center}
\includegraphics[width=0.9\textwidth,height=0.9\textheight,keepaspectratio]{/Users/I516998/Library/Application Support/Anki2/User 1/collection.media/Bildschirmfoto 2022-09-28 um 22.08.23.png}
\end{center}


\note[item]{}
\end{frame}
\begin{frame}
\frametitle{Unrelated Title}

\begin{center}
\includegraphics[width=0.9\textwidth,height=0.9\textheight,keepaspectratio]{/Users/I516998/Library/Application Support/Anki2/User 1/collection.media/Bildschirmfoto 2022-09-28 um 23.15.00.png}
\end{center}


\note[item]{}
\end{frame}
\begin{frame}
\frametitle{Unrelated Title}

\begin{center}
\includegraphics[width=0.9\textwidth,height=0.9\textheight,keepaspectratio]{/Users/I516998/Library/Application Support/Anki2/User 1/collection.media/Bildschirmfoto 2022-09-28 um 23.15.21.png}
\end{center}


\note[item]{}
\end{frame}
\begin{frame}
\frametitle{Unrelated Title}

\begin{center}
\includegraphics[width=0.9\textwidth,height=0.9\textheight,keepaspectratio]{/Users/I516998/Library/Application Support/Anki2/User 1/collection.media/Bildschirmfoto 2022-09-28 um 23.16.37.png}
\end{center}


\note[item]{}
\end{frame}
\begin{frame}
\frametitle{Unrelated Title}

\begin{center}
\includegraphics[width=0.9\textwidth,height=0.9\textheight,keepaspectratio]{/Users/I516998/Library/Application Support/Anki2/User 1/collection.media/Bildschirmfoto 2022-09-28 um 23.17.23.png}
\end{center}


\note[item]{}
\end{frame}
\begin{frame}
\frametitle{Unrelated Title}

\begin{center}
\includegraphics[width=0.9\textwidth,height=0.9\textheight,keepaspectratio]{/Users/I516998/Library/Application Support/Anki2/User 1/collection.media/Bildschirmfoto 2022-09-28 um 23.18.13.png}
\end{center}


\note[item]{}
\end{frame}
\begin{frame}
\frametitle{Unrelated Title}

\begin{center}
\includegraphics[width=0.9\textwidth,height=0.9\textheight,keepaspectratio]{/Users/I516998/Library/Application Support/Anki2/User 1/collection.media/Bildschirmfoto 2022-09-28 um 23.20.04.png}
\end{center}

\begin{itemize}
\item = tour + is def-clear subpath
\end{itemize}

\note[item]{}
\end{frame}
\begin{frame}
\frametitle{Unrelated Title}

\begin{center}
\includegraphics[width=0.9\textwidth,height=0.9\textheight,keepaspectratio]{/Users/I516998/Library/Application Support/Anki2/User 1/collection.media/Bildschirmfoto 2022-09-28 um 23.20.30.png}
\end{center}


\note[item]{}
\end{frame}
\begin{frame}
\frametitle{Unrelated Title}


\begin{itemize}
\item Four categories of customer preference: Exciters, Satisfiers, Dissatisfiers, Indifferent
\end{itemize}

\note[item]{}
\end{frame}
\begin{frame}
\frametitle{Unrelated Title}


\begin{itemize}
\item Must Have, Should Have, Could Have, Won’t Have
\end{itemize}

\note[item]{}
\end{frame}
\begin{frame}
\frametitle{Unrelated Title}


\begin{itemize}
\item compare multiple criteria to rank importance
\end{itemize}

\note[item]{}
\end{frame}
\begin{frame}
\frametitle{Unrelated Title}


\begin{itemize}
\item A prioritization methodkey stakeholders rank a list of requirements by using 100 points
\end{itemize}

\note[item]{}
\end{frame}
\begin{frame}
\frametitle{Unrelated Title}


\begin{itemize}
\item Using the common, ubiquitous knowledge of t-shirts and their sizes, individuals assign values to user stories. Relative estimating.
\end{itemize}

\note[item]{}
\end{frame}
\begin{frame}
\frametitle{Unrelated Title}


\begin{itemize}
\item Use Fibonacci number to size/estimate effort for user stories
\end{itemize}

\note[item]{}
\end{frame}
\begin{frame}
\frametitle{Unrelated Title}


\begin{itemize}
\item AKA Scrum poker
\end{itemize}

\note[item]{}
\end{frame}
\begin{frame}
\frametitle{Unrelated Title}


\begin{itemize}
\item Quickest method to develop estimates, but is the least accurate. Relative estimating.
\end{itemize}

\note[item]{}
\end{frame}
\begin{frame}
\frametitle{Unrelated Title}


\begin{itemize}
\item Not time-consuming, but may be inaccurate based on integrity of historical information.
\end{itemize}

\note[item]{}
\end{frame}
\begin{frame}
\frametitle{Unrelated Title}


\begin{itemize}
\item Takes unknowns into consideration (Pessimistic, Optimistic, Most-Likely) can be time-consuming
\end{itemize}

\note[item]{}
\end{frame}
\begin{frame}
\frametitle{Unrelated Title}


\begin{itemize}
\item Very accurateGives lower-level managers more responsibilityMay be very time-consuming and can be used only after the WBS has been well-defined. (Not Relative Sizing)
\end{itemize}

\note[item]{}
\end{frame}
\begin{frame}
\frametitle{Unrelated Title}


\begin{itemize}
\item Individuals vote by holding up five fingers for total agreement, a fist for total disagreement, or multiple fingers for somewhere in between
\end{itemize}

\note[item]{}
\end{frame}
\begin{frame}
\frametitle{Unrelated Title}


\begin{itemize}
\item Individuals vote with either a thumbs up (agreement) or thumbs down (disagreement).
\end{itemize}

\note[item]{}
\end{frame}
\begin{frame}
\frametitle{Unrelated Title}


\begin{itemize}
\item Team members share their point of view and, if team is unanimous, they move on.If objections raised, the facilitator works to solve the problem.
\end{itemize}

\note[item]{}
\end{frame}
\begin{frame}
\frametitle{Unrelated Title}


\begin{itemize}
\item Individuals use sticky dots to prioritize items in a list
\end{itemize}

\note[item]{}
\end{frame}
\begin{frame}
\frametitle{Unrelated Title}


\begin{itemize}
\item AKA lump sum contractUsed when scope is fully determinedSet fee will be paid for defined work regardless of the costmaximum protection to buyer but requires a lengthy preparation and bid evaluationSuited for projects with a high degree of certainty about their parameters
\end{itemize}

\note[item]{}
\end{frame}
\begin{frame}
\frametitle{Unrelated Title}


\begin{itemize}
\item • payment to seller for actual costs, plus a fee on seller's profit• often includes incentives for performance targets (cost, schedule, etc)• for projects with uncertain parameters
\end{itemize}

\note[item]{}
\end{frame}
\begin{frame}
\frametitle{Unrelated Title}


\begin{itemize}
\item • Example: calling a mechanic for an immediate repair for A/C unit; he doesn't come on-site to estimate or "plan", so by nature would be time & materials• hybrid of both cost-reimbursable and fixed-price contracts• Combines a negotiated hourly rate and full reimbursement for materials• Cost and value limits• Suited for projects when a precise statement of work cannot be quickly prescribed
\end{itemize}

\note[item]{}
\end{frame}
\begin{frame}
\frametitle{Unrelated Title}


\begin{itemize}
\item • like Time and Materials• Buyers don’t pay beyond a certain point• Suppliers benefit in case of early time-frame changes
\end{itemize}

\note[item]{}
\end{frame}
\begin{frame}
\frametitle{Unrelated Title}


\begin{itemize}
\item • Supplier and customer agree on final price• mutual cost savings if contract value runs below budget• These contracts may allow both parties to face additional costs if it exceeds budget
\end{itemize}

\note[item]{}
\end{frame}
\begin{frame}
\frametitle{Unrelated Title}


\begin{itemize}
\item • Customers review contracts during the contract life cycle at pre-negotiated, designated points of the contract lifecycle.• Customers can make required changes, continue or terminate the project at these points
\end{itemize}

\note[item]{}
\end{frame}
\begin{frame}
\frametitle{Unrelated Title}


\begin{itemize}
\item Cycle for Continuous process and quality improvement
\item Theorized quality was a management issue 85% of the time.
\end{itemize}

\note[item]{}
\end{frame}
\begin{frame}
\frametitle{Unrelated Title}


\begin{itemize}
\item FITNESS FOR USE
\end{itemize}

\note[item]{}
\end{frame}
\begin{frame}
\frametitle{Unrelated Title}


\begin{itemize}
\item Four absolutes (CZAR)- Conform to requirements (conformance by design)- Zero defects- Achieve quality by prevention- Real quality is measured by cost of quality
\end{itemize}

\note[item]{}
\end{frame}
\begin{frame}
\frametitle{Unrelated Title}


\begin{itemize}
\item Method emphasizes quality designed into the products to identify and control variation
\end{itemize}

\note[item]{}
\end{frame}
\begin{frame}
\frametitle{Unrelated Title}


\begin{itemize}
\item Six sigma! Systematically remove defects!
\end{itemize}

\note[item]{}
\end{frame}
\begin{frame}
\frametitle{Unrelated Title}


\begin{itemize}
\item Japanese for continuous improvement100 changes with 1% instead of 1 big change with 100%Many small changesideas come from workersindividual ownership of work / improve own performance
\end{itemize}

\note[item]{}
\end{frame}
\begin{frame}
\frametitle{Unrelated Title}


\begin{itemize}
\item Plan, do, check, act
\end{itemize}

\note[item]{}
\end{frame}
\begin{frame}
\frametitle{Unrelated Title}


\begin{itemize}
\item (4M
+ O + P)/6
\end{itemize}

\note[item]{}
\end{frame}
\begin{frame}
\frametitle{Unrelated Title}


\begin{itemize}
\item LF-EF
\end{itemize}

\note[item]{}
\end{frame}
\begin{frame}
\frametitle{Unrelated Title}


\begin{itemize}
\item EV - AC
\end{itemize}

\note[item]{}
\end{frame}
\begin{frame}
\frametitle{Unrelated Title}


\begin{itemize}
\item EV - PV
\end{itemize}

\note[item]{}
\end{frame}
\begin{frame}
\frametitle{Unrelated Title}


\begin{itemize}
\item EV/PV
\end{itemize}

\note[item]{}
\end{frame}
\begin{frame}
\frametitle{Unrelated Title}


\begin{itemize}
\item EV/AC
\end{itemize}

\note[item]{}
\end{frame}
\begin{frame}
\frametitle{Unrelated Title}


\begin{itemize}
\item BAC/CPI
\end{itemize}

\note[item]{}
\end{frame}
\begin{frame}
\frametitle{Unrelated Title}


\begin{itemize}
\item EAC - AC
\end{itemize}

\note[item]{}
\end{frame}
\begin{frame}
\frametitle{Unrelated Title}


\begin{itemize}
\item BAC - EAC
\end{itemize}

\note[item]{}
\end{frame}
\begin{frame}
\frametitle{Unrelated Title}


\begin{itemize}
\item Probability x Impact
\end{itemize}

\note[item]{}
\end{frame}
\begin{frame}
\frametitle{Unrelated Title}


\begin{itemize}
\item (BAC-EV)/(BAC-AC)
\end{itemize}

\note[item]{}
\end{frame}
\begin{frame}
\frametitle{Unrelated Title}


\begin{itemize}
\item Planned Value - the budgeted value of work PLANNED to be done at a certain time
\end{itemize}

\note[item]{}
\end{frame}
\begin{frame}
\frametitle{Unrelated Title}


\begin{itemize}
\item Earned Value - the budgeted value of work completed at a certain time; value based on cost & schedule performance
\end{itemize}

\note[item]{}
\end{frame}
\begin{frame}
\frametitle{Unrelated Title}


\begin{itemize}
\item Actual Cost - the actual cost of the work completed
\end{itemize}

\note[item]{}
\end{frame}
\begin{frame}
\frametitle{Unrelated Title}


\begin{itemize}
\item Budget at Completion - the original cost baseline subtracting all the approved changes in cost
\end{itemize}

\note[item]{}
\end{frame}
\begin{frame}
\frametitle{Unrelated Title}


\begin{itemize}
\item Estimate at Completion - the total project cost that is expected (forecasted based on performance to date)
\end{itemize}

\note[item]{}
\end{frame}
\begin{frame}
\frametitle{Unrelated Title}


\begin{itemize}
\item Estimate to Completion - the remaining cost of the project that's expected (from now til project completion)
\end{itemize}

\note[item]{}
\end{frame}
\begin{frame}
\frametitle{Unrelated Title}


\begin{itemize}
\item Variance at Completion - expected variance over/under the project budget
\end{itemize}

\note[item]{}
\end{frame}
\begin{frame}
\frametitle{Unrelated Title}


\begin{itemize}
\item Schedule Performance Index - Under 1 is behind schedule; Over 1 is ahead of schedule; Exactly 1 is right on schedule
\end{itemize}

\note[item]{}
\end{frame}
\begin{frame}
\frametitle{Unrelated Title}


\begin{itemize}
\item Cost Performance Index - Under 1 is over budget; Over 1 is under budget; Exactly 1 is on budget
\end{itemize}

\note[item]{}
\end{frame}
\begin{frame}
\frametitle{Unrelated Title}


\begin{itemize}
\item Schedule Variance - Negative is bad, Positive is good, Zero is on schedule
\end{itemize}

\note[item]{}
\end{frame}
\begin{frame}
\frametitle{Unrelated Title}


\begin{itemize}
\item Cost Variance - Negative is bad, Positive is good, Zero is on budget
\end{itemize}

\note[item]{}
\end{frame}
\begin{frame}
\frametitle{Unrelated Title}


\begin{itemize}
\item Code of account identifierDescription of workAssumptions/ConstraintsSchedule MilestonesResponsible OrgAssociated Schedule ActivitiesResources requiredCost estimatesQuality RequirementsAcceptance CriteriaTechnical ReferencesAgreement Information
\end{itemize}

\note[item]{}
\end{frame}
\begin{frame}
\frametitle{Unrelated Title}


\begin{itemize}
\item It provides detailed deliverable, activity and schedule information about each PIECE of the WBS.
\end{itemize}

\note[item]{}
\end{frame}
\begin{frame}
\frametitle{Unrelated Title}


\begin{itemize}
\item ANY document related to management of the project. These are living documents that are updated to reflect changes in project requirements and scope.
\end{itemize}

\note[item]{}
\end{frame}
\begin{frame}
\frametitle{Unrelated Title}


\begin{itemize}
\item Assessing external business environment with: Political, Economical, Social, Technical, Legal, Environmental factors
\end{itemize}

\note[item]{}
\end{frame}
\begin{frame}
\frametitle{Unrelated Title}


\begin{itemize}
\item AKA Plan Do Study ActTests possible solutions, assesses the results, and implement the solutions that work
\end{itemize}

\note[item]{}
\end{frame}
\begin{frame}
\frametitle{Unrelated Title}


\begin{itemize}
\item The lower level of needs have to be met firstTop level: Self-actualizationEsteemBelongingSafetyLowest Level: Physiological
\end{itemize}

\note[item]{}
\end{frame}
\begin{frame}
\frametitle{Unrelated Title}


\begin{itemize}
\item Theory X - Managers that believe they need to watch subordinates; who believe employees are naturally lazyTheory Y - Managers who believe that employees are naturally self-motivated
\end{itemize}

\note[item]{}
\end{frame}
\begin{frame}
\frametitle{Unrelated Title}


\begin{itemize}
\item Believes job satisfaction and dissatisfaction are not opposites.
\end{itemize}

\note[item]{}
\end{frame}
\begin{frame}
\frametitle{Unrelated Title}


\begin{itemize}
\item Humans have three types of emotional needs: Achievement, Power, Affiliation.
\end{itemize}

\note[item]{}
\end{frame}
\begin{frame}
\frametitle{Unrelated Title}


\begin{itemize}
\item Present value of all cash outflows minus Present value of all cash inflowsCompares value of dollar today to dollar in future (inflation and discount rate)
\end{itemize}

\note[item]{}
\end{frame}
\begin{frame}
\frametitle{Unrelated Title}


\begin{itemize}
\item Individuals & InteractionsWorking SoftwareCustomer CollaborationResponding to Change
\end{itemize}

\note[item]{}
\end{frame}
\begin{frame}
\frametitle{Unrelated Title}


\begin{itemize}
\item Increased likelihood of changes, which can result in wasted work and rework, which is time-consuming and costly.
\end{itemize}

\note[item]{}
\end{frame}
\begin{frame}
\frametitle{Unrelated Title}


\begin{itemize}
\item Allows feedback for unfinished work to improve & modify the work
\end{itemize}

\note[item]{}
\end{frame}
\begin{frame}
\frametitle{Unrelated Title}


\begin{itemize}
\item Approach that provides finished deliverables that customer can use immediately
\end{itemize}

\note[item]{}
\end{frame}
\begin{frame}
\frametitle{Unrelated Title}


\begin{itemize}
\item Iterative and IncrementalRefines work & delivers frequently
\end{itemize}

\note[item]{}
\end{frame}
\begin{frame}
\frametitle{Unrelated Title}


\begin{itemize}
\item Traditional, bulk of planning upfront, executing in single pass
\end{itemize}

\note[item]{}
\end{frame}
\begin{frame}
\frametitle{Unrelated Title}


\begin{itemize}
\item Success prototypes and proofs of concept
\end{itemize}

\note[item]{}
\end{frame}
\begin{frame}
\frametitle{Unrelated Title}


\begin{itemize}
\item Timeboxes of equal duration
\end{itemize}

\note[item]{}
\end{frame}
\begin{frame}
\frametitle{Unrelated Title}


\begin{itemize}
\item WBS, WBS Dictionary, Project Scope Statement, Planning package
\end{itemize}

\note[item]{}
\end{frame}
\begin{frame}
\frametitle{Unrelated Title}


\begin{itemize}
\item AKA Version ControlFunctional & physical characteristics of system or deliverables
\end{itemize}

\note[item]{}
\end{frame}
\begin{frame}
\frametitle{Unrelated Title}


\begin{itemize}
\item The features and functions of a product or service
\end{itemize}

\note[item]{}
\end{frame}
\begin{frame}
\frametitle{Unrelated Title}


\begin{itemize}
\item Work that must be done to deliver product/service as output of the project
\end{itemize}

\note[item]{}
\end{frame}
\begin{frame}
\frametitle{Unrelated Title}


\begin{itemize}
\item XP practiceallows product code to benefit from attention of all programmersall programmers can add functions, fix bugs, improve designs and refractor
\end{itemize}

\note[item]{}
\end{frame}
\begin{frame}
\frametitle{Unrelated Title}


\begin{itemize}
\item Technical guidelines for design of product & optimization of specific aspect of the design
\end{itemize}

\note[item]{}
\end{frame}
\begin{frame}
\frametitle{Unrelated Title}


\begin{itemize}
\item inspection & testing of deliverables (cost of conformance)
\end{itemize}

\note[item]{}
\end{frame}
\begin{frame}
\frametitle{Unrelated Title}


\begin{itemize}
\item Graph to show process over time against control limits Shows process stability/predictable performance
\end{itemize}

\note[item]{}
\end{frame}
\begin{frame}
\frametitle{Unrelated Title}


\begin{itemize}
\item Plan Quality Management, Manage Quality, Control Quality
\end{itemize}

\note[item]{}
\end{frame}
\begin{frame}
\frametitle{Unrelated Title}


\begin{itemize}
\item tracks work to be completed in backlog; analyzes variances
\end{itemize}

\note[item]{}
\end{frame}
\begin{frame}
\frametitle{Unrelated Title}


\begin{itemize}
\item Control Schedule
\end{itemize}

\note[item]{}
\end{frame}
\begin{frame}
\frametitle{Unrelated Title}


\begin{itemize}
\item Bar chart showing data that's usually over a period of time; displays cost, resource, defect, schedule or risk data
\end{itemize}

\note[item]{}
\end{frame}
\begin{frame}
\frametitle{Unrelated Title}


\begin{itemize}
\item Plan Resource MngtEstimate Activity ResourcesControl QualityQuantitative Risk Analysis
\end{itemize}

\note[item]{}
\end{frame}
\begin{frame}
\frametitle{Unrelated Title}


\begin{itemize}
\item Vertical bar chart showing number of hours per person; used for resource leveling
\end{itemize}

\note[item]{}
\end{frame}
\begin{frame}
\frametitle{Unrelated Title}


\begin{itemize}
\item Plan Resource ManagementDevelop ScheduleControl Schedule
\end{itemize}

\note[item]{}
\end{frame}
\begin{frame}
\frametitle{Unrelated Title}


\begin{itemize}
\item Shows relationship between two variables
\end{itemize}

\note[item]{}
\end{frame}
\begin{frame}
\frametitle{Unrelated Title}


\begin{itemize}
\item Control Quality
\end{itemize}

\note[item]{}
\end{frame}
\begin{frame}
\frametitle{Unrelated Title}


\begin{itemize}
\item DETERMINES PREVALENT ERRORSChart showing Proximity, Detectability, Impact 
\end{itemize}

\note[item]{}
\end{frame}
\begin{frame}
\frametitle{Unrelated Title}


\begin{itemize}
\item Qualitative Risk Analysis
\end{itemize}

\note[item]{}
\end{frame}
\begin{frame}
\frametitle{Unrelated Title}


\begin{itemize}
\item Shows a processIdentifies and anticipates process problems
\end{itemize}

\note[item]{}
\end{frame}
\begin{frame}
\frametitle{Unrelated Title}


\begin{itemize}
\item Plan Quality, Control Quality, Identify Risk
\end{itemize}

\note[item]{}
\end{frame}
\begin{frame}
\frametitle{Unrelated Title}


\begin{itemize}
\item Graph representing a schedule; shows activities, durations, start and end dates (like a Gantt)
\end{itemize}

\note[item]{}
\end{frame}
\begin{frame}
\frametitle{Unrelated Title}


\begin{itemize}
\item Develop Schedule
\end{itemize}

\note[item]{}
\end{frame}
\begin{frame}
\frametitle{Unrelated Title}


\begin{itemize}
\item Graphical representation of significant points, shows schedule at high-level executives
\end{itemize}

\note[item]{}
\end{frame}
\begin{frame}
\frametitle{Unrelated Title}


\begin{itemize}
\item Develop Schedule
\end{itemize}

\note[item]{}
\end{frame}
\begin{frame}
\frametitle{Unrelated Title}


\begin{itemize}
\item Connection between more work packages and team members
\end{itemize}

\note[item]{}
\end{frame}
\begin{frame}
\frametitle{Unrelated Title}


\begin{itemize}
\item Plan Resource Management
\end{itemize}

\note[item]{}
\end{frame}
\begin{frame}
\frametitle{Unrelated Title}


\begin{itemize}
\item Traces an undesirable effect back to root cause; breaks down potential causes
\end{itemize}

\note[item]{}
\end{frame}
\begin{frame}
\frametitle{Unrelated Title}


\begin{itemize}
\item Plan Quality Management
\end{itemize}

\note[item]{}
\end{frame}
\begin{frame}
\frametitle{Unrelated Title}


\begin{itemize}
\item Shows correlation of risks, activities, risk categories; determines which need more attention
\end{itemize}

\note[item]{}
\end{frame}
\begin{frame}
\frametitle{Unrelated Title}


\begin{itemize}
\item Perform Quantitative Risk Analysis
\end{itemize}

\note[item]{}
\end{frame}
\begin{frame}
\frametitle{Unrelated Title}


\begin{itemize}
\item Shows a situation as a set of outcomes and influences and their relationships; used for decisions with uncertain conditions
\end{itemize}

\note[item]{}
\end{frame}
\begin{frame}
\frametitle{Unrelated Title}


\begin{itemize}
\item Perform Quantitative Risk Analysis
\end{itemize}

\note[item]{}
\end{frame}
\begin{frame}
\frametitle{Unrelated Title}


\begin{itemize}
\item 1. Educate why/how to be agile (business value)2. Mentor, encourage, support3. Help with technical PM activities (i.e. quantitative risk analysis)4. Celebrate Team successes
\end{itemize}

\note[item]{}
\end{frame}
\begin{frame}
\frametitle{Unrelated Title}


\begin{itemize}
\item Range from 3-9 members; 7 -/+ 2
\end{itemize}

\note[item]{}
\end{frame}
\begin{frame}
\frametitle{Unrelated Title}


\begin{itemize}
\item Pairing, Swarming, Mobbing
\end{itemize}

\note[item]{}
\end{frame}
\begin{frame}
\frametitle{Unrelated Title}


\begin{itemize}
\item Forming
\end{itemize}

\note[item]{}
\end{frame}
\begin{frame}
\frametitle{Unrelated Title}


\begin{itemize}
\item Team building events
\end{itemize}

\note[item]{}
\end{frame}
\begin{frame}
\frametitle{Unrelated Title}


\begin{itemize}
\item Develop Project CharterIdentify Stakeholders
\end{itemize}

\note[item]{}
\end{frame}
\begin{frame}
\frametitle{Unrelated Title}


\begin{itemize}
\item 4.2 Develop Project Management Plan5.1 Plan Scope Mngt5.2 Collect Requirements5.3 Define Scope5.4 Create WBS6.1 Plan Schedule Mngt6.2 Define Activities6.3 Sequence Activities6.4 Estimate Activity Durations6.5 Develop Schedule7.1 Plan Cost Mngt7.2 Estimate Costs7.3 Determine Budget8.1 Plan Quality Mngt9.1 Plan Resource Management9.2 Estimate Activity Resources10.1 Plan Communications Mngt11.1 Plan Risk Mngt11.2 Identify RIsks11.3 Perform Qualitative Risk Analysis11.4 Perform Quantitative Risk Analysis11.5 Plan Risk Responses12.1 Plan Procurement Mngt13.2 Plan Stakeholder Engagement
\end{itemize}

\note[item]{}
\end{frame}
\begin{frame}
\frametitle{Unrelated Title}


\begin{itemize}
\item 4.3 Direct & Manage Project Work4.4 Manage Project Knowledge8.2 Manage Quality9.3 Acquire Resources9.4 Develop Team9.5 Manage Team10.2 Manage Communications11.6 Implement Risk Responses12.2 Conduct Procurements13.3 Manage Stakeholder Engagement
\end{itemize}

\note[item]{}
\end{frame}
\begin{frame}
\frametitle{Unrelated Title}


\begin{itemize}
\item 4.5 M/C Project Work4.6 Perform Integrated Change Control5.5 Validate Scope5.6 Control Scope6.6 Control Schedule7.4 Control Costs8.3 Control Quality11.7 Monitor Risks12.3 Control Procurements13.4 Monitor Stakeholder Engagement
\end{itemize}

\note[item]{}
\end{frame}
\begin{frame}
\frametitle{Unrelated Title}


\begin{itemize}
\item 4.7 Close Project
\end{itemize}

\note[item]{}
\end{frame}
\begin{frame}
\frametitle{Unrelated Title}


\begin{itemize}
\item Looks at process and finds where waste can be eliminatedidentifies inefficiencies and bottlenecks
\end{itemize}

\note[item]{}
\end{frame}
\begin{frame}
\frametitle{Unrelated Title}


\begin{itemize}
\item Baseline information to show if the project is on track or not.
\end{itemize}

\note[item]{}
\end{frame}
\begin{frame}
\frametitle{Unrelated Title}


\begin{itemize}
\item Project Charter
\end{itemize}

\note[item]{}
\end{frame}
\begin{frame}
\frametitle{Unrelated Title}


\begin{itemize}
\item Make a change request outlining the regulation requirements
\end{itemize}

\note[item]{}
\end{frame}
\begin{frame}
\frametitle{Unrelated Title}


\begin{itemize}
\item monitor seller's performancebuild seller relationships,improve processes/outcomes involving sellers
\end{itemize}

\note[item]{}
\end{frame}
\begin{frame}
\frametitle{Unrelated Title}


\begin{itemize}
\item Prioritization
\end{itemize}

\note[item]{}
\end{frame}
\begin{frame}
\frametitle{Unrelated Title}


\begin{itemize}
\item Combines grid elements into 3-D model, and helps with creating communication strategies
\end{itemize}

\note[item]{}
\end{frame}
\begin{frame}
\frametitle{Unrelated Title}


\begin{itemize}
\item Classifies stakeholders according to their influence on the work (used when relationships are clearly defined)
\end{itemize}

\note[item]{}
\end{frame}
\begin{frame}
\frametitle{Unrelated Title}


\begin{itemize}
\item For smaller projects with simple stakeholder relationships
\end{itemize}

\note[item]{}
\end{frame}
\begin{frame}
\frametitle{Unrelated Title}


\begin{itemize}
\item Web Conferencing, Email (Electronic communications; websites, web publishing, etc.)
\end{itemize}

\note[item]{}
\end{frame}
\begin{frame}
\frametitle{Unrelated Title}


\begin{itemize}
\item Used for virtual teams; link is started at beginning of workday and closed at end of workday; people can spontaneously engage with each other
\end{itemize}

\note[item]{}
\end{frame}
\begin{frame}
\frametitle{Unrelated Title}


\begin{itemize}
\item Completed/accepted user storiesProduct backlog progressComparison of user stories deliveredSprint/Iteration plans
\end{itemize}

\note[item]{}
\end{frame}
\begin{frame}
\frametitle{Unrelated Title}


\begin{itemize}
\item Validate Scope and Control Scope
\end{itemize}

\note[item]{}
\end{frame}
\begin{frame}
\frametitle{Unrelated Title}


\begin{itemize}
\item Servant LeaderFoster CollaborationAlign Stakeholder needsChange emphasis to coaching if neededNOT the center of coordination
\end{itemize}

\note[item]{}
\end{frame}
\begin{frame}
\frametitle{Unrelated Title}


\begin{itemize}
\item When team addresses ALL the requirements and attempts to do ALL design and ALL the building in a given period. Not a good approach.
\end{itemize}

\note[item]{}
\end{frame}
\begin{frame}
\frametitle{Unrelated Title}


\begin{itemize}
\item Cross-functional team membersProduct OwnerTeam Facilitator
\end{itemize}

\note[item]{}
\end{frame}
\begin{frame}
\frametitle{Unrelated Title}


\begin{itemize}
\item Pro: ExpertiseCon: Lacking relationships
\end{itemize}

\note[item]{}
\end{frame}
\begin{frame}
\frametitle{Unrelated Title}


\begin{itemize}
\item Pro: Strong relationshipsCon: Lack of expertise
\end{itemize}

\note[item]{}
\end{frame}
\begin{frame}
\frametitle{Unrelated Title}


\begin{itemize}
\item Tasks are handed off at the end of every day from one site to the next, many time zones away to speed up development
\end{itemize}

\note[item]{}
\end{frame}
\begin{frame}
\frametitle{Unrelated Title}


\begin{itemize}
\item Using virtual conferencing tools to share screens, voice/video links to collaborate
\end{itemize}

\note[item]{}
\end{frame}
\begin{frame}
\frametitle{Unrelated Title}


\begin{itemize}
\item It’s not optimal because the problem is
np-complete and overly simplified, in practice:
·      Multiple fragments, multiple relations
·      Fragment placement/ cost not independent
·      Complex query processing
·      Cost of integrity enforcement
·      Cost of concurrency control
·      Dynamics
\end{itemize}

\note[item]{}
\end{frame}
\begin{frame}
\frametitle{Unrelated Title}


\begin{itemize}
\item It requires a change request to make any changes to it.
\end{itemize}

\note[item]{}
\end{frame}
\begin{frame}
\frametitle{Unrelated Title}


\begin{itemize}
\item Loss of potential gain from other alternatives when another is chosen.Example: Choosing a $2000 speaking engagement over a $5000 one. The OC would be $5000.
\end{itemize}

\note[item]{}
\end{frame}
\begin{frame}
\frametitle{Unrelated Title}


\begin{itemize}
\item A summary of the project charter
\end{itemize}

\note[item]{}
\end{frame}
\begin{frame}
\frametitle{Unrelated Title}


\begin{itemize}
\item Plan Risk Management - RBS is part of this plan. 
\end{itemize}

\note[item]{}
\end{frame}
\begin{frame}
\frametitle{Unrelated Title}


\begin{itemize}
\item Hierarchically layout of project risks by category and subcategory.Shows various areas and causes of potential risks.
\end{itemize}

\note[item]{}
\end{frame}
\begin{frame}
\frametitle{Unrelated Title}


\begin{itemize}
\item Complexity
\end{itemize}

\note[item]{}
\end{frame}
\begin{frame}
\frametitle{Unrelated Title}


\begin{itemize}
\item List the required skills for the project and organize the project team based on those skills.
\end{itemize}

\note[item]{}
\end{frame}
\begin{frame}
\frametitle{Unrelated Title}


\begin{itemize}
\item Develop appropriate Comms StrategyProject communications(this is clearly defined in the PMBOK)
\end{itemize}

\note[item]{}
\end{frame}
\begin{frame}
\frametitle{Unrelated Title}


\begin{itemize}
\item Conduct a risk assessment for each proposed change, and document the complete impact
\end{itemize}

\note[item]{}
\end{frame}
\begin{frame}
\frametitle{Unrelated Title}


\begin{itemize}
\item Ability to be persuasive, articulate; active listening; awareness of various perspectives
\end{itemize}

\note[item]{}
\end{frame}
\begin{frame}
\frametitle{Unrelated Title}


\begin{itemize}
\item A project that makes screws that adhere to particular standards in the industry. Each screw is measured and weighed and has a notation to show the value in relation to the mean you've established.
\end{itemize}

\note[item]{}
\end{frame}
\begin{frame}
\frametitle{Unrelated Title}


\begin{itemize}
\item Tool and Technique of control qualityTracks repetitive activities or results
\end{itemize}

\note[item]{}
\end{frame}
\begin{frame}
\frametitle{Unrelated Title}


\begin{itemize}
\item Gaining credibility, respect, admiration of others
\end{itemize}

\note[item]{}
\end{frame}
\begin{frame}
\frametitle{Unrelated Title}


\begin{itemize}
\item Networking and having connections/alliances
\end{itemize}

\note[item]{}
\end{frame}
\begin{frame}
\frametitle{Unrelated Title}


\begin{itemize}
\item Being personally charming
\end{itemize}

\note[item]{}
\end{frame}
\begin{frame}
\frametitle{Unrelated Title}


\begin{itemize}
\item Applying flattery to win favor or cooperation
\end{itemize}

\note[item]{}
\end{frame}
\begin{frame}
\frametitle{Unrelated Title}


\begin{itemize}
\item All costs incurred over life of product by investment in preventing nonconformance, appraisal, and rework
\end{itemize}

\note[item]{}
\end{frame}
\begin{frame}
\frametitle{Unrelated Title}


\begin{itemize}
\item Prevention costs, appraisal costs, failure (rework) costs
\end{itemize}

\note[item]{}
\end{frame}
\begin{frame}
\frametitle{Unrelated Title}


\begin{itemize}
\item They are cost efficientTotal throughput/$ is more important than peak performancef
\end{itemize}

\note[item]{}
\end{frame}
\begin{frame}
\frametitle{Unrelated Title}


\begin{itemize}
\item At large scales, even most reliable hardware failsSOftware still needs to be fault-tolerant
\end{itemize}

\note[item]{}
\end{frame}
\begin{frame}
\frametitle{Unrelated Title}


\begin{itemize}
\item Reliable computer infrastructure from clusters of unreliable cmmodity machinesReplicate services across many machines to increase request throughput and availabilityFavor price/performance over peak performances
\end{itemize}

\note[item]{}
\end{frame}
\begin{frame}
\frametitle{Unrelated Title}


\begin{itemize}
\item Overheatingindividual machine failuresrouter failuresslow disks...
\end{itemize}

\note[item]{}
\end{frame}
\begin{frame}
\frametitle{Unrelated Title}


\begin{itemize}
\item Read collection of data itemsProcess each data item individually (Map)Shuffle and group dataAggregate each group of data items (Reduce)Write new collection of data items--> Map and Reduce are user-specified functions to express actual tasks
\end{itemize}

\note[item]{}
\end{frame}
\begin{frame}
\frametitle{Unrelated Title}


\begin{itemize}
\item Map function: extract words from a single document (document id, text) -> list(word, freq)Shuffle: Group counts per wordReduce function: Aggregate counts of words (word, list(freq)) -> list(word, total freq)








\end{itemize}

\note[item]{}
\end{frame}
\begin{frame}
\frametitle{Unrelated Title}


\begin{itemize}
\item Apply user-defined Map function to each k/v pairInput: one k/v pairOutput 0, 1, or more new k/v pairs
\end{itemize}

\note[item]{}
\end{frame}
\begin{frame}
\frametitle{Unrelated Title}


\begin{itemize}
\item Group all Map output pairs with the same key together and collect the values
\end{itemize}

\note[item]{}
\end{frame}
\begin{frame}
\frametitle{Unrelated Title}


\begin{itemize}
\item Apply user-defined Reduce function to each groupInput: key and its associated collection of valuesOutput: 0, 1, or more new k/v pairsMay or may not reduce data size
\end{itemize}

\note[item]{}
\end{frame}
\begin{frame}
\frametitle{Unrelated Title}

\begin{center}
\includegraphics[width=0.9\textwidth,height=0.9\textheight,keepaspectratio]{/Users/I516998/Library/Application Support/Anki2/User 1/collection.media/paste-3e7d46a374a0c3ed0a36c68309266ae5a81f8d82.jpg}
\end{center}

\begin{itemize}
\item Types are arbitraryMap and Reduce also typed
\end{itemize}

\note[item]{}
\end{frame}
\begin{frame}
\frametitle{Unrelated Title}

\begin{center}
\includegraphics[width=0.9\textwidth,height=0.9\textheight,keepaspectratio]{/Users/I516998/Library/Application Support/Anki2/User 1/collection.media/paste-08388fed4910f40a29cc3ca9ecefb0c9dea779b0.jpg}
\end{center}

\begin{itemize}
\item Split data (horizontal fragmentation)Map functionShuffle phase: Send groups to same machine (repartitioning)Reduce
\end{itemize}

\note[item]{}
\end{frame}
\begin{frame}
\frametitle{Unrelated Title}


\begin{itemize}
\item Map: Tasks that can be performed independentlyReduce: Tasks that require access to multiple data items
\end{itemize}

\note[item]{}
\end{frame}
\begin{frame}
\frametitle{Unrelated Title}


\begin{itemize}
\item Text processingdocument -> detect sentiments -> document annotated with sentimentsImage processingimage -> optimize image -> processed imageSimple SQL queries (without duplicate elimination)tuple -> if condition satisfied, emit output -> tuple or nothing
\end{itemize}

\note[item]{}
\end{frame}
\begin{frame}
\frametitle{Unrelated Title}


\begin{itemize}
\item Maximum salary of managers by departmentMap: take person record, when manager output (department, salary)Reduce: take (department, list of salaries), output department and maximum salaryGenerally: Compute an aggregateCount words or n-gramsBuild an inverted indexPerform data preprocessingPerform data mining
\end{itemize}

\note[item]{}
\end{frame}
\begin{frame}
\frametitle{Unrelated Title}


\begin{itemize}
\item Select keys and values such
that the right data items end up togetherStart by determining which keys to useComplex tasks may need to be partition into multiple MapReduce jobs


\end{itemize}

\note[item]{}
\end{frame}
\begin{frame}
\frametitle{Unrelated Title}


\begin{itemize}
\item Combiner functions for partial combination within a map task to save network bandwithPartitioning functions for mapping intermediate k/v to reduce workers (e.g. hash functions to controll which groups are processed at which machines)Ordering of intermediate k/v pairs (e.g. output ordered results)Distributed cache for read-only files
\end{itemize}

\note[item]{}
\end{frame}
\begin{frame}
\frametitle{Unrelated Title}


\begin{itemize}
\item Framework for distributed processing of large datasetsHDFS (Hdoop distributed file system)manage storageYARN (Yer another cluster manager)manage compute and memory
\end{itemize}

\note[item]{}
\end{frame}
\begin{frame}
\frametitle{Unrelated Title}


\begin{itemize}
\item Parition data across cluster of machines -> more machines, more throughputShop code to data: Machines that store data also run Map/Reduce functions on their local part to save communication bandwithQuery fault tolerance: Don't run from sratch -> repeat only parts affected by failure
\end{itemize}

\note[item]{}
\end{frame}
\begin{frame}
\frametitle{Unrelated Title}


\begin{itemize}
\item Large files Streaming data accessWrite once, read many timesAppend onlyTime to read more importantCommodity hardwareNot well-suited forLow latencyLots of small filesMultiple writers  
\end{itemize}

\note[item]{}
\end{frame}
\begin{frame}
\frametitle{Unrelated Title}

\begin{center}
\includegraphics[width=0.9\textwidth,height=0.9\textheight,keepaspectratio]{/Users/I516998/Library/Application Support/Anki2/User 1/collection.media/paste-0b5752ea3ddb49e4b11972075ccb33008dbc7f49.jpg}
\end{center}

\begin{itemize}
\item Files divided into fixed-size blocksMultiple datanodes store blocks on local disks (block is replicated to multiple datanodes)At least one namenode which maintains all file system metadata, block locations and monitors datanodesMultiple clients talk to namenodes for metadata, but to datanodes for read/write
\end{itemize}

\note[item]{}
\end{frame}
\begin{frame}
\frametitle{Unrelated Title}


\begin{itemize}
\item Files can be larger thn a single diskEasy allocation/replication for parallel processingMore fine-grained fault toleranceStorage system simplified (just store metadata, no need to understand data)
\end{itemize}

\note[item]{}
\end{frame}
\begin{frame}
\frametitle{Unrelated Title}

\begin{center}
\includegraphics[width=0.9\textwidth,height=0.9\textheight,keepaspectratio]{/Users/I516998/Library/Application Support/Anki2/User 1/collection.media/paste-822a6ffd4df52c89e510342772ed13f5bd7884cb.jpg}
\end{center}

\begin{itemize}
\item HDFS tries to pick blocks that are close in notion to the distanceNotion is simple: numbers are assigned based on rules
\end{itemize}

\note[item]{}
\end{frame}
\begin{frame}
\frametitle{Unrelated Title}

\begin{center}
\includegraphics[width=0.9\textwidth,height=0.9\textheight,keepaspectratio]{/Users/I516998/Library/Application Support/Anki2/User 1/collection.media/paste-595b1d13e7ed7f5e7b7ec4cd5d782c4f0a991eed.jpg}
\end{center}

\begin{itemize}
\item Client sends out data only onceData is replicated from first datanode to other datanodes in a pipelined way
\end{itemize}

\note[item]{}
\end{frame}
\begin{frame}
\frametitle{Unrelated Title}

\begin{center}
\includegraphics[width=0.9\textwidth,height=0.9\textheight,keepaspectratio]{/Users/I516998/Library/Application Support/Anki2/User 1/collection.media/paste-fb225c99b2a265fbbee68fc8e41e8a3e314239e7.jpg}
\end{center}

\begin{itemize}
\item Division of MapReduce job into tasks
\item Input data divided into M splitsEach split processed by a map taskReduce phase partitioned into R reduce tasks
\end{itemize}

\note[item]{}
\end{frame}
\begin{frame}
\frametitle{Unrelated Title}


\begin{itemize}
\item Master assigns each map task to a free workerConsiders data locality to worker when assigning a task (ship code to data)Worker reas task input (oftn from ocal files) and applies Map operation to each k/v pairWorker reduces R local files containing (sorted) intermediate k/v pairsMaster assigns each reduce task to a free worker
\end{itemize}

\note[item]{}
\end{frame}
\begin{frame}
\frametitle{Unrelated Title}


\begin{itemize}
\item after the merge the k/v pairs are still sorted by the keyk/v pairs can be read sequentially because k/v pairs are grouped by the keys
\end{itemize}

\note[item]{}
\end{frame}
\begin{frame}
\frametitle{Unrelated Title}


\begin{itemize}
\item Small code, large dataReduce network bandwith
\end{itemize}

\note[item]{}
\end{frame}
\begin{frame}
\frametitle{Unrelated Title}

\begin{center}
\includegraphics[width=0.9\textwidth,height=0.9\textheight,keepaspectratio]{/Users/I516998/Library/Application Support/Anki2/User 1/collection.media/paste-623b2fc855c1ccd8e2917ace86fd29b7293d1f57.jpg}
\end{center}

\begin{itemize}
\item Local fetchThousand of machines can read at local disk speed, no network limit
\end{itemize}

\note[item]{}
\end{frame}
\begin{frame}
\frametitle{Unrelated Title}

\begin{center}
\includegraphics[width=0.9\textwidth,height=0.9\textheight,keepaspectratio]{/Users/I516998/Library/Application Support/Anki2/User 1/collection.media/paste-f5b7f928cf76cef7d977e91019a2a4b58e45199f.jpg}
\end{center}

\begin{itemize}
\item Shuffle phase moves data and sortsSynchronization barrier between map and reduce phaseGenerally no locality possible for reduce phase
\end{itemize}

\note[item]{}
\end{frame}
\begin{frame}
\frametitle{Unrelated Title}


\begin{itemize}
\item Master detecs failure via periodic hearbeatsBoth completed and in-progress map tasks on that worker should be re-executed (output stored on local disk is lost)Only in-prgress reduce tasks on that worker should be re-executed (output stored in global file system)All reduce workers will be notified about any map re-executions
\end{itemize}

\note[item]{}
\end{frame}
\begin{frame}
\frametitle{Unrelated Title}


\begin{itemize}
\item State is check-pointed to DFSNew master recovers from there and continues
\end{itemize}

\note[item]{}
\end{frame}
\begin{frame}
\frametitle{Unrelated Title}

\begin{center}
\includegraphics[width=0.9\textwidth,height=0.9\textheight,keepaspectratio]{/Users/I516998/Library/Application Support/Anki2/User 1/collection.media/paste-fdc889e9fa0d61909e4e67218b1ef9055b574d4d.jpg}
\end{center}

\begin{itemize}
\item  Resource manager (coordinate allocation of resources across entire cluster)Node managers (1 per worker to monitor containers (application masters))Application masters (e.g. 1 per MapReduce job)
\end{itemize}

\note[item]{}
\end{frame}
\begin{frame}
\frametitle{Unrelated Title}

\begin{center}
\includegraphics[width=0.9\textwidth,height=0.9\textheight,keepaspectratio]{/Users/I516998/Library/Application Support/Anki2/User 1/collection.media/paste-3c5c7584954a87dc6e5a60a04daa2c6dae27e702.jpg}
\end{center}


\note[item]{}
\end{frame}
\begin{frame}
\frametitle{Unrelated Title}


\begin{itemize}
\item Multiple Map functions run in parallelMultiple Reduce functions run in parallel (on a different key)Partitioning of reduce phase important for scalability (balance work across reduce tasks)Assumes no state and no side effects (order of execution doesn't matter, safely rerun)
\end{itemize}

\note[item]{}
\end{frame}
\begin{frame}
\frametitle{Unrelated Title}


\begin{itemize}
\item Stragglers are slow workersSolution: close to completion, spawn backup copies of the remaining in-progress (speculative execution) -> the one finishes first wins
\end{itemize}

\note[item]{}
\end{frame}
\begin{frame}
\frametitle{Unrelated Title}


\begin{itemize}
\item Completing a release/shipping somethingWhen a few weeks have passedWhen they feel stuckMilestonesDoesn't always have to be iterations
\end{itemize}

\note[item]{}
\end{frame}
\begin{frame}
\frametitle{Unrelated Title}


\begin{itemize}
\item To refine enough stories so the team understands what the stories are and how large they are in relation to each other
\end{itemize}

\note[item]{}
\end{frame}
\begin{frame}
\frametitle{Unrelated Title}


\begin{itemize}
\item Timeboxed research or experiments; used to learn more about a story; can help assess risksUseful for estimation, acceptance criteria definition, understanding flow from user standpointHelpful for understanding technical/function elements
\end{itemize}

\note[item]{}
\end{frame}
\begin{frame}
\frametitle{Unrelated Title}


\begin{itemize}
\item What did I complete since last standup?What am I planning to complete between now and the next?What are my impediments?
\end{itemize}

\note[item]{}
\end{frame}
\begin{frame}
\frametitle{Unrelated Title}


\begin{itemize}
\item Focused on team's throughput (what is needed to advance the work)
\end{itemize}

\note[item]{}
\end{frame}
\begin{frame}
\frametitle{Unrelated Title}


\begin{itemize}
\item The product owner reduces the amount of work to match the reduced capacity
\end{itemize}

\note[item]{}
\end{frame}
\begin{frame}
\frametitle{Unrelated Title}


\begin{itemize}
\item Perform frequent incorporation of work into the wholeRetest to determine if entire product still worksTaking code and putting into the code bank
\end{itemize}

\note[item]{}
\end{frame}
\begin{frame}
\frametitle{Unrelated Title}


\begin{itemize}
\item System-level testing for end-to-end information and testing by unit for building blocks
\end{itemize}

\note[item]{}
\end{frame}
\begin{frame}
\frametitle{Unrelated Title}


\begin{itemize}
\item Entire team discusses acceptance criteria for work product. They consider how to test the work and then complete chunks of value.
\end{itemize}

\note[item]{}
\end{frame}
\begin{frame}
\frametitle{Unrelated Title}


\begin{itemize}
\item Writing automated tests BEFORE project work begins.Helps people design and mistake-proof the product.Considering how to "test drive" the design
\end{itemize}

\note[item]{}
\end{frame}
\begin{frame}
\frametitle{Unrelated Title}


\begin{itemize}
\item Creates cadence for delivery and feedbackFirst part of delivery is a demoReceive feedback and retrospect on how to inspect and adapt
\end{itemize}

\note[item]{}
\end{frame}
\begin{frame}
\frametitle{Unrelated Title}


\begin{itemize}
\item Product vision or product roadmap
\end{itemize}

\note[item]{}
\end{frame}
\begin{frame}
\frametitle{Unrelated Title}


\begin{itemize}
\item Splitting stories Use relative estimationConsider agile modeling or spiking
\end{itemize}

\note[item]{}
\end{frame}
\begin{frame}
\frametitle{Unrelated Title}


\begin{itemize}
\item Team defines definition of done, including acceptance criteria/release criteria
\end{itemize}

\note[item]{}
\end{frame}
\begin{frame}
\frametitle{Unrelated Title}


\begin{itemize}
\item Key stakeholders, not all
\end{itemize}

\note[item]{}
\end{frame}
\begin{frame}
\frametitle{Unrelated Title}


\begin{itemize}
\item Capability Maturity Model Integration - a process improvement framework
\end{itemize}

\note[item]{}
\end{frame}
\begin{frame}
\frametitle{Unrelated Title}


\begin{itemize}
\item Quality
\end{itemize}

\note[item]{}
\end{frame}
\begin{frame}
\frametitle{Unrelated Title}


\begin{itemize}
\item Smoothing out resource usage on a project; may have to let schedule slip as a trade off
\end{itemize}

\note[item]{}
\end{frame}
\begin{frame}
\frametitle{Unrelated Title}


\begin{itemize}
\item Schedule compressionCrashing adds resources to shorten schedule (increased risk/cost) DOESN'T NECESSARILY GUARANTEE ON-TIME DELIVERYFast-tracking performs activities in parallel to reduce time (potential rework)
\end{itemize}

\note[item]{}
\end{frame}
\begin{frame}
\frametitle{Unrelated Title}


\begin{itemize}
\item Words, pictures, drawings, maps, tangible materials people can use
\end{itemize}

\note[item]{}
\end{frame}
\begin{frame}
\frametitle{Unrelated Title}


\begin{itemize}
\item Personal experience, difficult to express, often shared in forms of stories
\end{itemize}

\note[item]{}
\end{frame}
\begin{frame}
\frametitle{Unrelated Title}


\begin{itemize}
\item Strives to reduce defects and increase efficiency
\end{itemize}

\note[item]{}
\end{frame}
\begin{frame}
\frametitle{Unrelated Title}


\begin{itemize}
\item Documents how the project/product scope will be defined, validated and controlled
\end{itemize}

\note[item]{}
\end{frame}
\begin{frame}
\frametitle{Unrelated Title}


\begin{itemize}
\item After scope baseline is created and required resources are estimated
\end{itemize}

\note[item]{}
\end{frame}
\begin{frame}
\frametitle{Unrelated Title}


\begin{itemize}
\item labor, material, other direct costs, overhead
\end{itemize}

\note[item]{}
\end{frame}
\begin{frame}
\frametitle{Unrelated Title}


\begin{itemize}
\item Project schedule network diagramsActivity duration estimatesResource calendars
\end{itemize}

\note[item]{}
\end{frame}
\begin{frame}
\frametitle{Unrelated Title}


\begin{itemize}
\item This means a sampling of the lot must have 10 % or fewer defects
\end{itemize}

\note[item]{}
\end{frame}
\begin{frame}
\frametitle{Unrelated Title}


\begin{itemize}
\item Sampling a small number of units that are assumed to represent the batch
\end{itemize}

\note[item]{}
\end{frame}
\begin{frame}
\frametitle{Unrelated Title}


\begin{itemize}
\item Making sure the deliverables meet the quality requirements set by key stakeholders or the customer
\end{itemize}

\note[item]{}
\end{frame}
\begin{frame}
\frametitle{Unrelated Title}


\begin{itemize}
\item Manage Quality - for the quality reports
\end{itemize}

\note[item]{}
\end{frame}
\begin{frame}
\frametitle{Unrelated Title}


\begin{itemize}
\item Handing over deliverables from one phase, and acts as a check point to make sure the next phase should continue
\end{itemize}

\note[item]{}
\end{frame}
\begin{frame}
\frametitle{Unrelated Title}


\begin{itemize}
\item one-on-one VS. groups meeting with groups
\end{itemize}

\note[item]{}
\end{frame}
\begin{frame}
\frametitle{Unrelated Title}


\begin{itemize}
\item Resource management plan
\end{itemize}

\note[item]{}
\end{frame}
\begin{frame}
\frametitle{Unrelated Title}


\begin{itemize}
\item Conversations where people are telling each other they think they're wrong
\end{itemize}

\note[item]{}
\end{frame}
\begin{frame}
\frametitle{Unrelated Title}


\begin{itemize}
\item Working together collaboratively
\end{itemize}

\note[item]{}
\end{frame}
\begin{frame}
\frametitle{Unrelated Title}


\begin{itemize}
\item Goal is to win - overshadows resolution
\end{itemize}

\note[item]{}
\end{frame}
\begin{frame}
\frametitle{Unrelated Title}


\begin{itemize}
\item Not talking to one anotherSomeone MUST lose
\end{itemize}

\note[item]{}
\end{frame}
\begin{frame}
\frametitle{Unrelated Title}


\begin{itemize}
\item Conflict managementInfluencingMotivationNegotiationTeam BuildingCultural Awareness
\end{itemize}

\note[item]{}
\end{frame}
\begin{frame}
\frametitle{Unrelated Title}


\begin{itemize}
\item How individual team members are performing on a project
\end{itemize}

\note[item]{}
\end{frame}
\begin{frame}
\frametitle{Unrelated Title}


\begin{itemize}
\item Assessing the project teams performance
\end{itemize}

\note[item]{}
\end{frame}
\begin{frame}
\frametitle{Unrelated Title}


\begin{itemize}
\item Used for budget constraint; for contracts with fixed budgetSupplier may offer customer the option to vary project scope at specific points in the projectCustomer can adjust features to fit capacity
\end{itemize}

\note[item]{}
\end{frame}
\begin{frame}
\frametitle{Unrelated Title}


\begin{itemize}
\item When agile supplier delivers sufficient value with only half scope completed; customer not bound to paying the rest if they don't need it
\end{itemize}

\note[item]{}
\end{frame}
\begin{frame}
\frametitle{Unrelated Title}


\begin{itemize}
\item Arguably most collaborative contracting approach; embeds supplier services directly into the customer organization
\end{itemize}

\note[item]{}
\end{frame}
\begin{frame}
\frametitle{Unrelated Title}


\begin{itemize}
\item Limits the budgetNew ideas replace original work, as new ideas come along
\end{itemize}

\note[item]{}
\end{frame}
\begin{frame}
\frametitle{Unrelated Title}


\begin{itemize}
\item Shared financial riskSupplier can be rewarded with higher rate when delivering earlierOR their rate reduces if they deliver late
\end{itemize}

\note[item]{}
\end{frame}
\begin{frame}
\frametitle{Unrelated Title}


\begin{itemize}
\item Pull communication method
\end{itemize}

\note[item]{}
\end{frame}
\begin{frame}
\frametitle{Unrelated Title}


\begin{itemize}
\item When you need to distribute info to large audiences, and the info can be large and complex
\end{itemize}

\note[item]{}
\end{frame}
\begin{frame}
\frametitle{Unrelated Title}


\begin{itemize}
\item Determine when everyone is availableIdentifies working days, shifts, start/end of normal business hours, holidays, etc.
\end{itemize}

\note[item]{}
\end{frame}
\begin{frame}
\frametitle{Unrelated Title}


\begin{itemize}
\item Conduct trainings, create SOPs, other creative ways of promoting team learning
\end{itemize}

\note[item]{}
\end{frame}
\begin{frame}
\frametitle{Unrelated Title}


\begin{itemize}
\item Adjust the data points because your sponsor is asking for a completed report; you would speak with team members if the question didn't require you to get the report out soon
\end{itemize}

\note[item]{}
\end{frame}
\begin{frame}
\frametitle{Unrelated Title}


\begin{itemize}
\item Lessons learned
\end{itemize}

\note[item]{}
\end{frame}
\begin{frame}
\frametitle{Unrelated Title}


\begin{itemize}
\item It is used as an input, and updated as an output in many processes throughout the project
\end{itemize}

\note[item]{}
\end{frame}
\begin{frame}
\frametitle{Unrelated Title}


\begin{itemize}
\item Personnel administration policiesResource management planInfo about competency levels and prior experience (assessing who would be appropriate for a project)
\end{itemize}

\note[item]{}
\end{frame}
\begin{frame}
\frametitle{Unrelated Title}


\begin{itemize}
\item Software development method based on frequent cyclesPractices intended to improve results of software projects (organizational, technical, planning and integration)
\end{itemize}

\note[item]{}
\end{frame}
\begin{frame}
\frametitle{Unrelated Title}


\begin{itemize}
\item Test-Driven developmentPlanning gameOn-site customerPair programmingCode refractoring (continuously improve code)
\end{itemize}

\note[item]{}
\end{frame}
\begin{frame}
\frametitle{Unrelated Title}


\begin{itemize}
\item Fully scopes work & deliverables
\item Tracks cost to right categoriesHelps identify riskHelps designate workCreates ease to oversee project
\end{itemize}

\note[item]{}
\end{frame}
\begin{frame}
\frametitle{Unrelated Title}


\begin{itemize}
\item How requirements will be planned, tracked, reportedConfiguration management activitiesRequirements prioritization processMetrics used Traceability structure
\end{itemize}

\note[item]{}
\end{frame}
\begin{frame}
\frametitle{Unrelated Title}


\begin{itemize}
\item Project schedule model developmentRelease/iteration lengthLevel of accuracyUnits of measure (staff hours, days, weeks, etc.)Control thresholdsSchedule Maintenance& More.
\end{itemize}

\note[item]{}
\end{frame}
\begin{frame}
\frametitle{Unrelated Title}


\begin{itemize}
\item How the scope will be defined, developed, monitored, controlled and validated.
\end{itemize}

\note[item]{}
\end{frame}
\begin{frame}
\frametitle{Unrelated Title}


\begin{itemize}
\item How requirements will be:1. analyzed2. documented3.managed
\end{itemize}

\note[item]{}
\end{frame}
\begin{frame}
\frametitle{Unrelated Title}


\begin{itemize}
\item Criteria and activities for developing, monitoring, controlling the schedule.
\end{itemize}

\note[item]{}
\end{frame}
\begin{frame}
\frametitle{Unrelated Title}


\begin{itemize}
\item How costs will be planned, structured, and controlled.
\end{itemize}

\note[item]{}
\end{frame}
\begin{frame}
\frametitle{Unrelated Title}


\begin{itemize}
\item How an organization's quality policies, methodologies and standards will be implemented in the project.
\end{itemize}

\note[item]{}
\end{frame}
\begin{frame}
\frametitle{Unrelated Title}


\begin{itemize}
\item Guidance on how project resources should be categorized, allocated, managed and released.
\end{itemize}

\note[item]{}
\end{frame}
\begin{frame}
\frametitle{Unrelated Title}


\begin{itemize}
\item How, when, by whom, information about the project will be administered and disseminated.
\end{itemize}

\note[item]{}
\end{frame}
\begin{frame}
\frametitle{Unrelated Title}


\begin{itemize}
\item How activities will be structured and performed.
\end{itemize}

\note[item]{}
\end{frame}
\begin{frame}
\frametitle{Unrelated Title}


\begin{itemize}
\item How project team will acquire goods and services from outside performing organization.
\end{itemize}

\note[item]{}
\end{frame}
\begin{frame}
\frametitle{Unrelated Title}


\begin{itemize}
\item How stakeholders will be engaged in project decisions and execution according to their needs, interest and impact.
\end{itemize}

\note[item]{}
\end{frame}
\begin{frame}
\frametitle{Unrelated Title}


\begin{itemize}
\item Scope baselineSchedule baselineCost baseline
\end{itemize}

\note[item]{}
\end{frame}
\begin{frame}
\frametitle{Unrelated Title}


\begin{itemize}
\item Deals with implementing the planned and systemic activities within the quality system to assure the project team and the external stakeholders that the quality standards are being met (hence the quality audits)
\end{itemize}

\note[item]{}
\end{frame}
\begin{frame}
\frametitle{Unrelated Title}


\begin{itemize}
\item Monitoring specific project results to determine compliance to relevant quality standards
\end{itemize}

\note[item]{}
\end{frame}
\begin{frame}
\frametitle{Unrelated Title}


\begin{itemize}
\item Involves the identification of quality standards relevant to the project
\end{itemize}

\note[item]{}
\end{frame}
\begin{frame}
\frametitle{Unrelated Title}


\begin{itemize}
\item Process of monitoring the status of the project to update project budget and manage changes to the cost baseline
\end{itemize}

\note[item]{}
\end{frame}
\begin{frame}
\frametitle{Unrelated Title}


\begin{itemize}
\item Manage Quality
\end{itemize}

\note[item]{}
\end{frame}
\begin{frame}
\frametitle{Unrelated Title}


\begin{itemize}
\item Evaluating team strengths, weakness, team preferences, future employment and plans of team members, building trust, decision-making amongst team
\end{itemize}

\note[item]{}
\end{frame}
\begin{frame}
\frametitle{Unrelated Title}


\begin{itemize}
\item Identify relationships between team members in smaller projectsIdentifies communication protocols in large projectsHelps identify reporting relationships
\end{itemize}

\note[item]{}
\end{frame}
\begin{frame}
\frametitle{Unrelated Title}


\begin{itemize}
\item Does not determine how work is authorized, nor does it help ensure work is done by the org
\end{itemize}

\note[item]{}
\end{frame}
\begin{frame}
\frametitle{Unrelated Title}


\begin{itemize}
\item As a tool and technique of Acquire Resources (conducted as part of interpersonal and team skills)
\end{itemize}

\note[item]{}
\end{frame}
\begin{frame}
\frametitle{Unrelated Title}


\begin{itemize}
\item This describes Develop Project Charter - so agreement, business case, EEFs and OPAs are the inputs
\end{itemize}

\note[item]{}
\end{frame}
\begin{frame}
\frametitle{Unrelated Title}


\begin{itemize}
\item Project Integration Management
\end{itemize}

\note[item]{}
\end{frame}
\begin{frame}
\frametitle{Unrelated Title}


\begin{itemize}
\item Anonymous risk identification method
\end{itemize}

\note[item]{}
\end{frame}
\begin{frame}
\frametitle{Unrelated Title}


\begin{itemize}
\item Calculating EMV of a decision
\end{itemize}

\note[item]{}
\end{frame}
\begin{frame}
\frametitle{Unrelated Title}


\begin{itemize}
\item Quality controlDefect percentages by typeHigh to low ranking order
\end{itemize}

\note[item]{}
\end{frame}
\begin{frame}
\frametitle{Unrelated Title}


\begin{itemize}
\item Describes an organization's willingness to tolerate risk
\end{itemize}

\note[item]{}
\end{frame}
\begin{frame}
\frametitle{Unrelated Title}


\begin{itemize}
\item Throughout the whole project
\end{itemize}

\note[item]{}
\end{frame}
\begin{frame}
\frametitle{Unrelated Title}


\begin{itemize}
\item Qualitative Risk Analysis
\end{itemize}

\note[item]{}
\end{frame}
\begin{frame}
\frametitle{Unrelated Title}


\begin{itemize}
\item Quantitative Risk Analysis
\end{itemize}

\note[item]{}
\end{frame}
\begin{frame}
\frametitle{Unrelated Title}


\begin{itemize}
\item Behaviors of the productExamples: actions, processes, data, interactions the product should execute
\end{itemize}

\note[item]{}
\end{frame}
\begin{frame}
\frametitle{Unrelated Title}


\begin{itemize}
\item Environmental conditions or qualities required for product to be effective.Examples: reliability, security, performance, safety, level of service, supportability, retention
\end{itemize}

\note[item]{}
\end{frame}
\begin{frame}
\frametitle{Unrelated Title}


\begin{itemize}
\item Nonfunctional requirements
\end{itemize}

\note[item]{}
\end{frame}
\begin{frame}
\frametitle{Unrelated Title}


\begin{itemize}
\item Functional requirement
\end{itemize}

\note[item]{}
\end{frame}
\begin{frame}
\frametitle{Unrelated Title}


\begin{itemize}
\item Qualitative Risk Analysis
\end{itemize}

\note[item]{}
\end{frame}
\begin{frame}
\frametitle{Unrelated Title}


\begin{itemize}
\item - It acknowledges the project exists- It's published under the name of an external manager to project team- Gives PM authority to assign org resources
\end{itemize}

\note[item]{}
\end{frame}
\begin{frame}
\frametitle{Unrelated Title}


\begin{itemize}
\item Minimum Business Increment - a small amount of working software that can be attached to existing software 
\end{itemize}

\note[item]{}
\end{frame}
\begin{frame}
\frametitle{Unrelated Title}


\begin{itemize}
\item Roll-wave planning; the conscious choice to focus on the near term
\end{itemize}

\note[item]{}
\end{frame}
\begin{frame}
\frametitle{Unrelated Title}


\begin{itemize}
\item Ongoing refinement of the project management plan; learn by doing
\end{itemize}

\note[item]{}
\end{frame}
\begin{frame}
\frametitle{Unrelated Title}


\begin{itemize}
\item Cost forecast, Estimate at Completion (EAC)
\end{itemize}

\note[item]{}
\end{frame}
\begin{frame}
\frametitle{Unrelated Title}


\begin{itemize}
\item Between the product backlog creation and sprint backlog creation; estimated in sprint planning meeting
\end{itemize}

\note[item]{}
\end{frame}
\begin{frame}
\frametitle{Unrelated Title}


\begin{itemize}
\item Average stories or story points per iteration
\end{itemize}

\note[item]{}
\end{frame}
\begin{frame}
\frametitle{Unrelated Title}


\begin{itemize}
\item Adaptive, Cross-collaborative
\end{itemize}

\note[item]{}
\end{frame}
\begin{frame}
\frametitle{Unrelated Title}


\begin{itemize}
\item Collect requirementsDefine scopeCreate WBSDefine activitiesSequence activities
\end{itemize}

\note[item]{}
\end{frame}
\begin{frame}
\frametitle{Unrelated Title}


\begin{itemize}
\item Clear, Cohesive, Complete, Concise, Concrete
\end{itemize}

\note[item]{}
\end{frame}
\begin{frame}
\frametitle{Unrelated Title}


\begin{itemize}
\item "Crash takes cash" - if the budget doesn't look good, likely you'd want to fast-track
\end{itemize}

\note[item]{}
\end{frame}
\begin{frame}
\frametitle{Unrelated Title}


\begin{itemize}
\item When you have a failed product, you don't have to recover the costThis would be considered a sunk costThis cost should just be ignored; it's not positive, you'll account for the cost in cost of materials
\end{itemize}

\note[item]{}
\end{frame}
\begin{frame}
\frametitle{Unrelated Title}


\begin{itemize}
\item Economic Model
\end{itemize}

\note[item]{}
\end{frame}
\begin{frame}
\frametitle{Unrelated Title}


\begin{itemize}
\item PV is the S-curve, it’s the baseline of what we have PLANNED to do.
\end{itemize}

\note[item]{}
\end{frame}
\begin{frame}
\frametitle{Unrelated Title}


\begin{itemize}
\item Month 5 - intersection of EV and PV
\end{itemize}

\note[item]{}
\end{frame}
\begin{frame}
\frametitle{Unrelated Title}


\begin{itemize}
\item You got behind schedule
\end{itemize}

\note[item]{}
\end{frame}
\begin{frame}
\frametitle{Unrelated Title}


\begin{itemize}
\item You’re operating over budget the whole time
\end{itemize}

\note[item]{}
\end{frame}
\begin{frame}
\frametitle{Unrelated Title}


\begin{itemize}
\item You’ve spent more money than you’ve accomplished work
\end{itemize}

\note[item]{}
\end{frame}
\begin{frame}
\frametitle{Unrelated Title}

\begin{center}
\includegraphics[width=0.9\textwidth,height=0.9\textheight,keepaspectratio]{/Users/I516998/Library/Application Support/Anki2/User 1/collection.media/image-a1b232af992a7f1ac86240b723cf5e3944a7d388.png}
\end{center}

\begin{itemize}
\item Envisioning: vision, product backlog, roadmapTo Product BacklogTo Sprint BacklogTo Sprint (and a Daily Scrum)To Potentially Shippable Product
\end{itemize}

\note[item]{}
\end{frame}
\begin{frame}
\frametitle{Unrelated Title}


\begin{itemize}
\item Envisioning - n/aSprint Planning - up to 8 hoursSprint - 2-4 weeksDaily Scrum - 15 minSprint Review - up to 4 hoursSprint Retro - up to 3 hours
\end{itemize}

\note[item]{}
\end{frame}
\begin{frame}
\frametitle{Unrelated Title}


\begin{itemize}
\item - Ideas from team/stakeholders- Product owner helps define vision/refines into user stories- Product roadmap and backlog constantly refined**Backlog must be groomed before sprint planning
\end{itemize}

\note[item]{}
\end{frame}
\begin{frame}
\frametitle{Unrelated Title}


\begin{itemize}
\item All of them (Initiation, Planning, Execution, Monitor/Control, Close)
\end{itemize}

\note[item]{}
\end{frame}
\begin{frame}
\frametitle{Unrelated Title}


\begin{itemize}
\item Scoping; Building a schedule based on activities
\end{itemize}

\note[item]{}
\end{frame}
\begin{frame}
\frametitle{Unrelated Title}


\begin{itemize}
\item Aspects of user stories (role, goal, benefit structure)Independent, Negotiable, Valuable, Estimated, Small, Testable
\end{itemize}

\note[item]{}
\end{frame}
\begin{frame}
\frametitle{Unrelated Title}

\begin{center}
\includegraphics[width=0.9\textwidth,height=0.9\textheight,keepaspectratio]{/Users/I516998/Library/Application Support/Anki2/User 1/collection.media/image-2c23eb101c8bf2dd798e65311f7407e0faebc1c8.jpg}
\end{center}


\note[item]{}
\end{frame}
\begin{frame}
\frametitle{Unrelated Title}


\begin{itemize}
\item Cycle Time
\end{itemize}

\note[item]{}
\end{frame}
\begin{frame}
\frametitle{Unrelated Title}


\begin{itemize}
\item Time it take to deliver start to finishTime an items waits before work starts
\end{itemize}

\note[item]{}
\end{frame}
\begin{frame}
\frametitle{Unrelated Title}


\begin{itemize}
\item Features complete, remaining and total
\end{itemize}

\note[item]{}
\end{frame}
\begin{frame}
\frametitle{Unrelated Title}


\begin{itemize}
\item Completed work compared to total expected work at iterations
\end{itemize}

\note[item]{}
\end{frame}
\begin{frame}
\frametitle{Unrelated Title}


\begin{itemize}
\item Large number of ideas to be classified into groups for review and analysis(Example: can group causes of defects into groups)
\end{itemize}

\note[item]{}
\end{frame}
\begin{frame}
\frametitle{Unrelated Title}


\begin{itemize}
\item Regulation mandated by law
\end{itemize}

\note[item]{}
\end{frame}
\begin{frame}
\frametitle{Unrelated Title}

\begin{center}
\includegraphics[width=0.9\textwidth,height=0.9\textheight,keepaspectratio]{/Users/I516998/Library/Application Support/Anki2/User 1/collection.media/Bildschirmfoto 2022-10-11 um 12.08.26.png}
\end{center}


\note[item]{}
\end{frame}
\begin{frame}
\frametitle{Unrelated Title}

\begin{center}
\includegraphics[width=0.9\textwidth,height=0.9\textheight,keepaspectratio]{/Users/I516998/Library/Application Support/Anki2/User 1/collection.media/Bildschirmfoto 2022-10-11 um 12.11.11.png}
\end{center}

\begin{itemize}
\item Focus on leaders impact on performance of small groups2 types of leadership behaviors = scale endpoints (later studies show -> wrong)
\end{itemize}

\note[item]{}
\end{frame}
\begin{frame}
\frametitle{Unrelated Title}

\begin{center}
\includegraphics[width=0.9\textwidth,height=0.9\textheight,keepaspectratio]{/Users/I516998/Library/Application Support/Anki2/User 1/collection.media/Bildschirmfoto 2022-10-11 um 12.16.01.png}
\end{center}


\note[item]{}
\end{frame}
\begin{frame}
\frametitle{Unrelated Title}

\begin{center}
\includegraphics[width=0.9\textwidth,height=0.9\textheight,keepaspectratio]{/Users/I516998/Library/Application Support/Anki2/User 1/collection.media/Bildschirmfoto 2022-10-11 um 12.18.27.png}
\end{center}

\begin{itemize}
\item Concern for people - how leader attends to the peopleConcern for production - how leader is concerned with achieving organizational tasks1 - 9
\end{itemize}

\note[item]{}
\end{frame}
\begin{frame}
\frametitle{Unrelated Title}


\begin{itemize}
\item provides framework for assessing leadership in a broad way (most common tools: LBDQ by Stogdill and Leadership Grid by Blake&McCanse)applies to nearly all actions of leaders, enables self-reflection of leaderapplications: basis for many training and development programs
\end{itemize}

\note[item]{}
\end{frame}
\begin{frame}
\frametitle{Unrelated Title}


\begin{itemize}
\item + adds behavior to traits and skills
\item + helps understand leadership process
\item + differentiating between task and relationship makes approach heuristic, allows action assesment
\item - research: no consistent link between task and relationship behavior and outcomes
\item - no universal style identified that is effective in most cases (high/high = best is not always true)
\end{itemize}

\note[item]{}
\end{frame}
\begin{frame}
\frametitle{Unrelated Title}

\begin{center}
\includegraphics[width=0.9\textwidth,height=0.9\textheight,keepaspectratio]{/Users/I516998/Library/Application Support/Anki2/User 1/collection.media/Bildschirmfoto 2022-10-11 um 12.42.48.png}
\end{center}


\note[item]{}
\end{frame}
\begin{frame}
\frametitle{Unrelated Title}


\begin{itemize}
\item Directive behaviors - assist group members in goal achievement via one-way communication (giving directions, defining roles, evaluations+methods)Supportive behaviors - assist group members via two-way communication in feeling comfortable (asking for input, problem solving, listening)
\end{itemize}

\note[item]{}
\end{frame}
\begin{frame}
\frametitle{Unrelated Title}

\begin{center}
\includegraphics[width=0.9\textwidth,height=0.9\textheight,keepaspectratio]{/Users/I516998/Library/Application Support/Anki2/User 1/collection.media/Bildschirmfoto 2022-10-11 um 12.46.36.png}
\end{center}

\begin{itemize}
\item Directing, Coaching, Supporting, Delegating (S1 - S4)Situation: D1 -> D4 (Competence + Commitment)
\end{itemize}

\note[item]{}
\end{frame}
\begin{frame}
\frametitle{Unrelated Title}


\begin{itemize}
\item leader match theory: effective leadership is contingent on matching a leader's style to the right setting (leader has only one style!!) (styles can be task or relationship motivated)Measure: LPC least prefered coworker (high -> relationship, low -> task) -> shows leadership style
\end{itemize}

\note[item]{}
\end{frame}
\begin{frame}
\frametitle{Unrelated Title}


\begin{itemize}
\item Leader-Member relations (group atmoshphere, degree of confidence and loyalty) Task structure (degree to which task requirements are clear)Position power (amount of authority a leader has to reward/punish)-> High = Good/Strong, Low = Bad/poor
\end{itemize}

\note[item]{}
\end{frame}
\begin{frame}
\frametitle{Unrelated Title}

\begin{center}
\includegraphics[width=0.9\textwidth,height=0.9\textheight,keepaspectratio]{/Users/I516998/Library/Application Support/Anki2/User 1/collection.media/Bildschirmfoto 2022-10-11 um 13.01.15.png}
\end{center}

\begin{itemize}
\item Rating the person you'd like to work the least with, positive = high number, negative = low number.. 
\end{itemize}

\note[item]{}
\end{frame}
\begin{frame}
\frametitle{Unrelated Title}

\begin{center}
\includegraphics[width=0.9\textwidth,height=0.9\textheight,keepaspectratio]{/Users/I516998/Library/Application Support/Anki2/User 1/collection.media/Bildschirmfoto 2022-10-11 um 13.02.24.png}
\end{center}


\note[item]{}
\end{frame}
\begin{frame}
\frametitle{Unrelated Title}

\begin{center}
\includegraphics[width=0.9\textwidth,height=0.9\textheight,keepaspectratio]{/Users/I516998/Library/Application Support/Anki2/User 1/collection.media/Bildschirmfoto 2022-10-11 um 13.04.39.png}
\end{center}


\note[item]{}
\end{frame}
\begin{frame}
\frametitle{Unrelated Title}


\begin{itemize}
\item Situation must be taken into consideration and leadership style / leader must be adapted2 Research perspectives:Hersey: Development levels vary, leaders must be flexible in their styleFiedler: A leader is not good in all situations, as the effectiveness is dependant on the context 
\end{itemize}

\note[item]{}
\end{frame}
\begin{frame}
\frametitle{Unrelated Title}


\begin{itemize}
\item + emphasizes leader flexibility and individual employees
\item + high practicability (SLII) and prescriptive value
\item + proved useful in many trainings and development programs
\item - demographic characteristics missing / low theoretical basis
\item - hersey: ambiguous design of development level (SLII)
\item - fiedler: LPC scale does not correlate well with other leadership measures
\end{itemize}

\note[item]{}
\end{frame}
\begin{frame}
\frametitle{Unrelated Title}

\begin{center}
\includegraphics[width=0.9\textwidth,height=0.9\textheight,keepaspectratio]{/Users/I516998/Library/Application Support/Anki2/User 1/collection.media/Bildschirmfoto 2022-10-11 um 15.16.24.png}
\end{center}


\note[item]{}
\end{frame}
\begin{frame}
\frametitle{Unrelated Title}


\begin{itemize}
\item TransportationInventoryMotionWaitingOverproductionOverprocessingDefects
\end{itemize}

\note[item]{}
\end{frame}
\begin{frame}
\frametitle{Unrelated Title}

\begin{center}
\includegraphics[width=0.9\textwidth,height=0.9\textheight,keepaspectratio]{/Users/I516998/Library/Application Support/Anki2/User 1/collection.media/Bildschirmfoto 2022-10-11 um 15.20.14.png}
\end{center}


\note[item]{}
\end{frame}
\begin{frame}
\frametitle{Unrelated Title}

\begin{center}
\includegraphics[width=0.9\textwidth,height=0.9\textheight,keepaspectratio]{/Users/I516998/Library/Application Support/Anki2/User 1/collection.media/Bildschirmfoto 2022-10-11 um 15.23.04.png}
\end{center}


\note[item]{}
\end{frame}
\begin{frame}
\frametitle{Unrelated Title}

\begin{center}
\includegraphics[width=0.9\textwidth,height=0.9\textheight,keepaspectratio]{/Users/I516998/Library/Application Support/Anki2/User 1/collection.media/Bildschirmfoto 2022-10-11 um 15.28.01.png}
\end{center}

\begin{itemize}
\item need-based theories explain why a person must actthey do not explain why specific actions are chosen in specific situations to
\item obtain specific outcomes + do not account for individual differences 
\end{itemize}

\note[item]{}
\end{frame}
\begin{frame}
\frametitle{Unrelated Title}


\begin{itemize}
\item 



Valuessimilar to needs (direct, arouse), but needs = inborn and values = acquired by experiencetrans-situational goals that serve as guiding principles Goalssimilar to values but more specific, mechanism by which a value leads to action
\end{itemize}

\note[item]{}
\end{frame}
\begin{frame}
\frametitle{Unrelated Title}

\begin{center}
\includegraphics[width=0.9\textwidth,height=0.9\textheight,keepaspectratio]{/Users/I516998/Library/Application Support/Anki2/User 1/collection.media/Bildschirmfoto 2022-10-11 um 15.41.04.png}
\end{center}


\note[item]{}
\end{frame}
\begin{frame}
\frametitle{Unrelated Title}

\begin{center}
\includegraphics[width=0.9\textwidth,height=0.9\textheight,keepaspectratio]{/Users/I516998/Library/Application Support/Anki2/User 1/collection.media/Bildschirmfoto 2022-10-11 um 16.40.15.png}
\end{center}

\begin{itemize}
\item Literature was focused on task and person orientation, however mixed findings in researchLeader's goal: enhance employee performance and satisfaction by focusing on employee motivation 
\end{itemize}

\note[item]{}
\end{frame}
\begin{frame}
\frametitle{Unrelated Title}

\begin{center}
\includegraphics[width=0.9\textwidth,height=0.9\textheight,keepaspectratio]{/Users/I516998/Library/Application Support/Anki2/User 1/collection.media/Bildschirmfoto 2022-10-11 um 16.41.51.png}
\end{center}


\note[item]{}
\end{frame}
\begin{frame}
\frametitle{Unrelated Title}


\begin{itemize}
\item behavior that complements or supplements what is missing in work environmentprovide information, ressources or reward to enhance goal attainment➢ It depends on follower and task characteristics if a type of leader behavior is
motivating 
\end{itemize}

\note[item]{}
\end{frame}
\begin{frame}
\frametitle{Unrelated Title}

\begin{center}
\includegraphics[width=0.9\textwidth,height=0.9\textheight,keepaspectratio]{/Users/I516998/Library/Application Support/Anki2/User 1/collection.media/Bildschirmfoto 2022-10-11 um 16.52.51.png}
\end{center}


\note[item]{}
\end{frame}
\begin{frame}
\frametitle{Unrelated Title}


\begin{itemize}
\item Directive Leadership (set clear standards for performance, clarify how task is to be done)Supportive Leadership (be friendly/approachable, respect followers)Participative Leadership (consult with followers, seek and respect their opinion)Achievement-oriented leadership (seek improvements, high performance expectancy)
\end{itemize}

\note[item]{}
\end{frame}
\begin{frame}
\frametitle{Unrelated Title}


\begin{itemize}
\item 



Need for affiliation -> supportive leadership






Preferences for structure -> directive leadership






Locus of control (internal locus -> participative leadership, external locus -> directive leadership)







Self-perceived level of task ability, if low -> directive leadership



\end{itemize}

\note[item]{}
\end{frame}
\begin{frame}
\frametitle{Unrelated Title}


\begin{itemize}
\item Task (unclear/ambiguos -> provide structure, repetitive -> provide support for motivation)Formal authority system (Weak formal authority -> assist by making reqs and rules clear)Primary work group (weak group norms -> build
cohesiveness and role responsibility)
\end{itemize}

\note[item]{}
\end{frame}
\begin{frame}
\frametitle{Unrelated Title}

\begin{center}
\includegraphics[width=0.9\textwidth,height=0.9\textheight,keepaspectratio]{/Users/I516998/Library/Application Support/Anki2/User 1/collection.media/Bildschirmfoto 2022-10-11 um 17.22.39.png}
\end{center}


\note[item]{}
\end{frame}
\begin{frame}
\frametitle{Unrelated Title}


\begin{itemize}
\item 


Interaction facilitationGroup oriented decision processWork-group representation and networking
Value-based leadership behavior





\end{itemize}

\note[item]{}
\end{frame}
\begin{frame}
\frametitle{Unrelated Title}


\begin{itemize}
\item Set of assumptions how different leadership styles interact with follower characteristics and the work situation to affect employee motivationLeaders should choose fitting style for followers needs and their work, assisting them in reaching their goals
\end{itemize}

\note[item]{}
\end{frame}
\begin{frame}
\frametitle{Unrelated Title}


\begin{itemize}
\item + framework to understand follower's satisfaction and performance
\item + integrate motivation into leadership theory
\item + practical recommendations for leaders to ease followers path to their goals
\item - complex set of assumptions 
\item - not well empirically supported, behavior-motivation relationship not clear
\item - leadership as a one-way event, leader -> follower
\item Application: only few trainings, rather general recommendations
\end{itemize}

\note[item]{}
\end{frame}
\begin{frame}
\frametitle{Unrelated Title}


\begin{itemize}
\item Kano modelMoSCoWPaired Comparisons100 Points
\end{itemize}

\note[item]{}
\end{frame}
\begin{frame}
\frametitle{Unrelated Title}


\begin{itemize}
\item Fist of FiveRoman VotingPollingDot Voting
\end{itemize}

\note[item]{}
\end{frame}
\begin{frame}
\frametitle{Unrelated Title}


\begin{itemize}
\item T-shirt sizingStory PointingPlanning Poker
\end{itemize}

\note[item]{}
\end{frame}
\begin{frame}
\frametitle{Unrelated Title}


\begin{itemize}
\item Data + InformationPhysical or electronic reportData reviewed from baseline plus WPI
\end{itemize}

\note[item]{}
\end{frame}
\begin{frame}
\frametitle{Unrelated Title}


\begin{itemize}
\item Iteration 0 - "sprint 0", is sprint planning/start of agile projectIteration N - definition of done
\end{itemize}

\note[item]{}
\end{frame}
\begin{frame}
\frametitle{Unrelated Title}


\begin{itemize}
\item At the beginning of every sprint
\end{itemize}

\note[item]{}
\end{frame}
\begin{frame}
\frametitle{Unrelated Title}


\begin{itemize}
\item The product backlog
\end{itemize}

\note[item]{}
\end{frame}
\begin{frame}
\frametitle{Unrelated Title}


\begin{itemize}
\item Unanimity - everyone agreesMajority - >50% agreePlurality - largest block of people agree (i.e. 40% agree, 30% for one choice, 10% for another)Agile - Fist of five/Roman Voting
\end{itemize}

\note[item]{}
\end{frame}
\begin{frame}
\frametitle{Unrelated Title}


\begin{itemize}
\item Tool to define scope; asking questions about the product, forming answers to describe its use and other relevant aspects of what will be developed
\end{itemize}

\note[item]{}
\end{frame}
\begin{frame}
\frametitle{Unrelated Title}


\begin{itemize}
\item Structured technique to optimize value in project
\end{itemize}

\note[item]{}
\end{frame}
\begin{frame}
\frametitle{Unrelated Title}


\begin{itemize}
\item AC + (BAC - EV)
\end{itemize}

\note[item]{}
\end{frame}
\begin{frame}
\frametitle{Unrelated Title}


\begin{itemize}
\item Adequate time-cost estimates
\end{itemize}

\note[item]{}
\end{frame}
\begin{frame}
\frametitle{Unrelated Title}


\begin{itemize}
\item In between Control Account and Work Package 
\item Known work content without detailed schedule activities
\end{itemize}

\note[item]{}
\end{frame}
\begin{frame}
\frametitle{Unrelated Title}


\begin{itemize}
\item Scope, budget, actual cost and schedule
\end{itemize}

\note[item]{}
\end{frame}
\begin{frame}
\frametitle{Unrelated Title}


\begin{itemize}
\item Lead speeds up - is a negative number or %Lag slows down - is a positive number or %
\end{itemize}

\note[item]{}
\end{frame}
\begin{frame}
\frametitle{Unrelated Title}


\begin{itemize}
\item 95 days
\end{itemize}

\note[item]{}
\end{frame}
\begin{frame}
\frametitle{Unrelated Title}


\begin{itemize}
\item Longest path through projectRepresents shortest possible durationLeast amount of float
\end{itemize}

\note[item]{}
\end{frame}
\begin{frame}
\frametitle{Unrelated Title}


\begin{itemize}
\item Forward - calculate early numbers (EF and ES)Backward - calculates late numbers (LF and LS)
\end{itemize}

\note[item]{}
\end{frame}
\begin{frame}
\frametitle{Unrelated Title}

\begin{center}
\includegraphics[width=0.9\textwidth,height=0.9\textheight,keepaspectratio]{/Users/I516998/Library/Application Support/Anki2/User 1/collection.media/image-77115e8fb1d1b72300870db469f930bce5e94c28.jpg}
\end{center}


\note[item]{}
\end{frame}
\begin{frame}
\frametitle{Unrelated Title}

\begin{center}
\includegraphics[width=0.9\textwidth,height=0.9\textheight,keepaspectratio]{/Users/I516998/Library/Application Support/Anki2/User 1/collection.media/image-8bfdf38ed0e44b3d40d9763cfe6e46136b5b7880.png}
\end{center}

\begin{itemize}
\item ROMDefinitivePhased Estimate
\end{itemize}

\note[item]{}
\end{frame}
\begin{frame}
\frametitle{Unrelated Title}

\begin{center}
\includegraphics[width=0.9\textwidth,height=0.9\textheight,keepaspectratio]{/Users/I516998/Library/Application Support/Anki2/User 1/collection.media/image-921683d562d78f8ce092a632aaed75f303a0dcdd.jpg}
\end{center}


\note[item]{}
\end{frame}
\begin{frame}
\frametitle{Unrelated Title}


\begin{itemize}
\item Initialize :: Initial -> PartialTransform an input value into a mergeable summaryApplied to every input valueMerge :: (Partial, Partial) -> PartialMerge two summariesRepeat until only one summary remainsTerminate :: Partial -> FinalCompute the final aggregate from final summary
\end{itemize}

\note[item]{}
\end{frame}
\begin{frame}
\frametitle{Unrelated Title}

\begin{center}
\includegraphics[width=0.9\textwidth,height=0.9\textheight,keepaspectratio]{/Users/I516998/Library/Application Support/Anki2/User 1/collection.media/paste-2cc14876a5e20435033666ddf7815e8441d10c99.jpg}
\end{center}


\note[item]{}
\end{frame}
\begin{frame}
\frametitle{Unrelated Title}


\begin{itemize}
\item Intialize and merge all data items on each site locallySend the corresponding summaries to a central site and merge themCompute final result at central siteAssumption: merge associative & commutative
\end{itemize}

\note[item]{}
\end{frame}
\begin{frame}
\frametitle{Unrelated Title}


\begin{itemize}
\item Small cost to store and communicate partial resultsTerminate step (which cannot be parallelized) gets small input
\end{itemize}

\note[item]{}
\end{frame}
\begin{frame}
\frametitle{Unrelated Title}


\begin{itemize}
\item D has n elements, each taking O(1)Small = O(log n) -> count and sum are considered small, but union is notif small = O(1) -> minimum and maximum are small, but count and sum are notif small = O(n) -> union counts as small
\end{itemize}

\note[item]{}
\end{frame}
\begin{frame}
\frametitle{Unrelated Title}


\begin{itemize}
\item Summaries are smallTermiante is identity (Terminate(x) = reutrn x)
\end{itemize}

\note[item]{}
\end{frame}
\begin{frame}
\frametitle{Unrelated Title}

\begin{center}
\includegraphics[width=0.9\textwidth,height=0.9\textheight,keepaspectratio]{/Users/I516998/Library/Application Support/Anki2/User 1/collection.media/paste-c752c000e8eaf9d91cc5cf844f1ab8122e34f38f.jpg}
\end{center}

\begin{itemize}
\item If it can be computed based on a fixed number of distributive aggregate functions
\end{itemize}

\note[item]{}
\end{frame}
\begin{frame}
\frametitle{Unrelated Title}


\begin{itemize}
\item All aggregate functions that are not distributive or algebraicOften not parallelizableSomtimes efficient approcimate computation possibleExample: median, most frequent value
\end{itemize}

\note[item]{}
\end{frame}
\begin{frame}
\frametitle{Unrelated Title}


\begin{itemize}
\item Combiner optionally pre-reduce the mapper outputCan save network bandwith+workloadSame input type as Reduce / output type same as MapDoes not replace reduce step (only local data)With combiners MapReduce is closely related to aggregate fucntions
\end{itemize}

\note[item]{}
\end{frame}
\begin{frame}
\frametitle{Unrelated Title}


\begin{itemize}
\item Evaluate an aggregation fucntion per keyAn aggregation function can be parallelized per pey in MapReduce
\end{itemize}

\note[item]{}
\end{frame}
\begin{frame}
\frametitle{Unrelated Title}


\begin{itemize}
\item Some task do not admit a useful combinerDistributive and algebraic aggregates usually lead to useful combinersFor holistic aggregates a combiner can be used but is rarely beneficial
\end{itemize}

\note[item]{}
\end{frame}
\begin{frame}
\frametitle{Unrelated Title}


\begin{itemize}
\item How to partition keys to Reduce tasks?How to sort keys within a partition?How to group keys for Reduce function?
\end{itemize}

\note[item]{}
\end{frame}
\begin{frame}
\frametitle{Unrelated Title}

\begin{center}
\includegraphics[width=0.9\textwidth,height=0.9\textheight,keepaspectratio]{/Users/I516998/Library/Application Support/Anki2/User 1/collection.media/paste-41c0cfb6b8876e9b258efd2be5685f77c82c3441.jpg}
\end{center}

\begin{itemize}
\item Add value to keyProblem: Map otuput not partitioned by original key
\end{itemize}

\note[item]{}
\end{frame}
\begin{frame}
\frametitle{Unrelated Title}

\begin{center}
\includegraphics[width=0.9\textwidth,height=0.9\textheight,keepaspectratio]{/Users/I516998/Library/Application Support/Anki2/User 1/collection.media/paste-f453ff4e3f25fbf2b25c98996e1fd57495d4c2cf.jpg}
\end{center}

\begin{itemize}
\item Custom partitioner is based only on a part of the keyCustom key comparator sorts by all keysProblem: Partition correct but not grouped by only a part of the key
\end{itemize}

\note[item]{}
\end{frame}
\begin{frame}
\frametitle{Unrelated Title}

\begin{center}
\includegraphics[width=0.9\textwidth,height=0.9\textheight,keepaspectratio]{/Users/I516998/Library/Application Support/Anki2/User 1/collection.media/paste-7590b3bfbbebb6e9863bf059f204db8ddaac0a35.jpg}
\end{center}

\begin{itemize}
\item Custom grouping comparator (only a part of the key)In general: grouping comparator should be consistent with key comparator
\end{itemize}

\note[item]{}
\end{frame}
\begin{frame}
\frametitle{Unrelated Title}


\begin{itemize}
\item Map: selection, projection without duplicate eliminationMapReduce framework: groupingReduce: aggregation, having, duplicate elimination
\end{itemize}

\note[item]{}
\end{frame}
\begin{frame}
\frametitle{Unrelated Title}


\begin{itemize}
\item Parallel external merge sort
\item Mapper: output(sort attribute, tuple) pairsCustom partitioner to run range-partitioningCustom key comparator for sort orderReducer: output all tuplesProblem: need good paritioning vector (often determined from sample)
\end{itemize}

\note[item]{}
\end{frame}
\begin{frame}
\frametitle{Unrelated Title}


\begin{itemize}
\item Great when one relation is small (S)Map over tuples of RMap functions knows S entirelyMap joins each input tuple with S and outputs result
\end{itemize}

\note[item]{}
\end{frame}
\begin{frame}
\frametitle{Unrelated Title}


\begin{itemize}
\item Map over tuples from R and from S (use multiple inputs: relation name, tuple)Map extracts join key from each tuple and outputs: (join key, (relation name, tuple))Recdue obtains all tuples from both relations for each join key -> can join locallyBasically that's a parallel hash join
\end{itemize}

\note[item]{}
\end{frame}
\begin{frame}
\frametitle{Unrelated Title}

\begin{center}
\includegraphics[width=0.9\textwidth,height=0.9\textheight,keepaspectratio]{/Users/I516998/Library/Application Support/Anki2/User 1/collection.media/paste-1e93d5635759ecd028479c7e623869705b176d8f.jpg}
\end{center}


\note[item]{}
\end{frame}
\begin{frame}
\frametitle{Unrelated Title}


\begin{itemize}
\item SimpleScalableHigh througputQuery fault toleranceSupports "data first, purpose later" paradigm of Big Data
\end{itemize}

\note[item]{}
\end{frame}
\begin{frame}
\frametitle{Unrelated Title}


\begin{itemize}
\item Low level, restricted programming modeNo query languageNo indices, updates or transactionsSlowReinvents many DBMS technologies
\end{itemize}

\note[item]{}
\end{frame}
\begin{frame}
\frametitle{Unrelated Title}

\begin{center}
\includegraphics[width=0.9\textwidth,height=0.9\textheight,keepaspectratio]{/Users/I516998/Library/Application Support/Anki2/User 1/collection.media/paste-460d4dafaead9e625d035730b1a4d6eff963dea2.jpg}
\end{center}


\note[item]{}
\end{frame}
\begin{frame}
\frametitle{Unrelated Title}

\begin{center}
\includegraphics[width=0.9\textwidth,height=0.9\textheight,keepaspectratio]{/Users/I516998/Library/Application Support/Anki2/User 1/collection.media/paste-48c6588e4d9e2b5faa8756fb3db795a279a4f5cb.jpg}
\end{center}

\begin{itemize}
\item Rank documents accroding to the angle of their vector close with the vector of the queryAngle can be zero while the euclidean distance can be high
\end{itemize}

\note[item]{}
\end{frame}
\begin{frame}
\frametitle{Unrelated Title}

\begin{center}
\includegraphics[width=0.9\textwidth,height=0.9\textheight,keepaspectratio]{/Users/I516998/Library/Application Support/Anki2/User 1/collection.media/paste-47b76398c4a709a4eff5b43583a59fbe80c80f98.jpg}
\end{center}


\note[item]{}
\end{frame}
\begin{frame}
\frametitle{Unrelated Title}

\begin{center}
\includegraphics[width=0.9\textwidth,height=0.9\textheight,keepaspectratio]{/Users/I516998/Library/Application Support/Anki2/User 1/collection.media/paste-ecf92cdeedb4ef3e4f76769b18ec60a35f590a4e.jpg}
\end{center}

\begin{itemize}
\item -1 to 1 (0°-180°)
\end{itemize}

\note[item]{}
\end{frame}
\begin{frame}
\frametitle{Unrelated Title}

\begin{center}
\includegraphics[width=0.9\textwidth,height=0.9\textheight,keepaspectratio]{/Users/I516998/Library/Application Support/Anki2/User 1/collection.media/paste-63f0034c474bf8717b90f8248877ae67009d6e10.jpg}
\end{center}


\note[item]{}
\end{frame}
\begin{frame}
\frametitle{Unrelated Title}


\begin{itemize}
\item Reduce the total number of cosine computationsPerfiltering (e.g. boolean retrieval) or pre-clusteringReduce the set of query terms we consoderSmaller set of candidates and faster computation (shorter vectors)
\end{itemize}

\note[item]{}
\end{frame}
\begin{frame}
\frametitle{Unrelated Title}


\begin{itemize}
\item Fetch only documents that contain at least one/ N query term(s) (otherwise cosine similarity is 0 anyway)Use inverted index and boolean query (t1 or t2 or t3)
\end{itemize}

\note[item]{}
\end{frame}
\begin{frame}
\frametitle{Unrelated Title}


\begin{itemize}
\item Terms with low IDF scores appear in many documents, thus matching such terms between query and couments does not affect the ranking muchPosting lists of terms with low IDF are long (many cosine computations)
\end{itemize}

\note[item]{}
\end{frame}
\begin{frame}
\frametitle{Unrelated Title}


\begin{itemize}
\item Randomly select sqrt(N) documents, the leadersFor every document in the collectionCompute the similarities with all leadersAdd the document the cluster of the most somilar leaderRandom sampling is desirableFastest strategyLeaders refelct data distribution (Dense regions will have more leaders than sparse regions)Fast retrievalMeasure similarity with cluster leadersSelect leader documentCompute similarity with all documents in the selected leader's cluster2*sqrt(N) computations -> O(sqrt(N))Pre-clustering may lead to lower recall (Some relevant documents are not found)
\end{itemize}

\note[item]{}
\end{frame}
\begin{frame}
\frametitle{Unrelated Title}


\begin{itemize}
\item Reduce length of the vectorsOnly makes sense for queries with many termsRepresent query and document with a significantly shorter vector M and M <<VKey question: Select the vector in such a way that relations between the original cosine similarities are preserved Locality sensitive hashing -> Random projections
\end{itemize}

\note[item]{}
\end{frame}
\begin{frame}
\frametitle{Unrelated Title}


\begin{itemize}
\item A family of dimensionality reducion techniques that map the original vector space into a lower-dimensional spaceMaximize the extent to which the new vector space retains the topology of the original one
\end{itemize}

\note[item]{}
\end{frame}
\begin{frame}
\frametitle{Unrelated Title}


\begin{itemize}
\item A vector space of lower dimensions that retains the distances from the original space
\end{itemize}

\note[item]{}
\end{frame}
\begin{frame}
\frametitle{Unrelated Title}


\begin{itemize}
\item Choose a set of M random vectors in the original high-dimensional vector spaceFor each document tf-idf vector doCompute inner dot product of d and each random vectorHash inner product (e.g. treshold -> 1 or 0)Compute a new vectors of length M that contains all hash values
\end{itemize}

\note[item]{}
\end{frame}
\begin{frame}
\frametitle{Unrelated Title}


\begin{itemize}
\item For each term store only the documents with highest scores w (in these documents the term nis relatively informative)Basically we rank documents according the the TF values sinc idf is the same for allSuch a list is called champion list or fancy listTwo possible options:Take the top N documents with highest scoreTake all docuemnts for which the TF value is above some treshold
\end{itemize}

\note[item]{}
\end{frame}
\begin{frame}
\frametitle{Unrelated Title}


\begin{itemize}
\item Create champion list and regular full posting listAnswer queries from champion listIf the number of hits using the champion list is smaller than the number of results the user is looking for, return hits using the full posting list
\end{itemize}

\note[item]{}
\end{frame}
\begin{frame}
\frametitle{Unrelated Title}

\begin{center}
\includegraphics[width=0.9\textwidth,height=0.9\textheight,keepaspectratio]{/Users/I516998/Library/Application Support/Anki2/User 1/collection.media/paste-7c2333f4ce1af06cccb6f9f4b12733982c8a2698.jpg}
\end{center}

\begin{itemize}
\item General version of the two layer indexPosting list is broken down hierarchically into several lists according to the TF scoreIn each tier the documents are sorted according to the docId, not tfLook-up: Merge postings of the first tierMerge lists until the result contains the requested number of hitsFirst merge and sort the different tiers for one term
\end{itemize}

\note[item]{}
\end{frame}
\begin{frame}
\frametitle{Unrelated Title}


\begin{itemize}
\item Leadership: process that is centered on the interactions between a leader and followersLMX challenges the assumption that leaders treat followers in a collective wayFocus on possible differences between each follower and leader (Dyadic Relationship)
\end{itemize}

\note[item]{}
\end{frame}
\begin{frame}
\frametitle{Unrelated Title}

\begin{center}
\includegraphics[width=0.9\textwidth,height=0.9\textheight,keepaspectratio]{/Users/I516998/Library/Application Support/Anki2/User 1/collection.media/Bildschirmfoto 2022-10-16 um 10.39.31.png}
\end{center}


\note[item]{}
\end{frame}
\begin{frame}
\frametitle{Unrelated Title}

\begin{center}
\includegraphics[width=0.9\textwidth,height=0.9\textheight,keepaspectratio]{/Users/I516998/Library/Application Support/Anki2/User 1/collection.media/Bildschirmfoto 2022-10-16 um 10.41.36.png}
\end{center}


\note[item]{}
\end{frame}
\begin{frame}
\frametitle{Unrelated Title}


\begin{itemize}
\item  Early studies: focus on differences between in-group / out-groupthen: extension towards relationship: LMX - organizational effectivenessempirical results on high-quality LMX (less employee turnover, more positive evaluation)meta-analytic results (pos relationship: LMX - task performance / citizenship performance, neg: LMX - counterproductive performance)
\end{itemize}

\note[item]{}
\end{frame}
\begin{frame}
\frametitle{Unrelated Title}

\begin{center}
\includegraphics[width=0.9\textwidth,height=0.9\textheight,keepaspectratio]{/Users/I516998/Library/Application Support/Anki2/User 1/collection.media/Bildschirmfoto 2022-10-16 um 11.09.01.png}
\end{center}


\note[item]{}
\end{frame}
\begin{frame}
\frametitle{Unrelated Title}


\begin{itemize}
\item central concept: dyadic relationshipLMX describes leadership: (in/outgroups, differences, goal achievement)LMX prescribes leadership: (LMX development, special relationships, new roles/opportunities, goal: whole work group = in-group)
\end{itemize}

\note[item]{}
\end{frame}
\begin{frame}
\frametitle{Unrelated Title}


\begin{itemize}
\item + only approach to consider the dyadic relationship
\item + emphasizes importance of communication
\item + solid research on positive effects of LMX practice to organizational outcomes
\item - may support development of privileged groups
\item - basic theoretical ideas not fully developed
\item - LMX measurement is being questioned
\item application: leaders can use insights to improve their leadership, regarding relationships & networks
\end{itemize}

\note[item]{}
\end{frame}
\begin{frame}
\frametitle{Unrelated Title}


\begin{itemize}
\item Goal difficulty: difficult goals -> higher performancedo-your-best goals -> no external referent, allow wide range of acceptable performance levelsgoal specifity -> reduces performance variation
\end{itemize}

\note[item]{}
\end{frame}
\begin{frame}
\frametitle{Unrelated Title}


\begin{itemize}
\item Goals give directionGoals have energizing functionGoals affect persistenceGoals affect action
\end{itemize}

\note[item]{}
\end{frame}
\begin{frame}
\frametitle{Unrelated Title}


\begin{itemize}
\item Goal Core, Moderators, Mechanisms, Performance, Satisfaction with Performance&Reward, Willingness to Commit to New Challenges
\end{itemize}

\note[item]{}
\end{frame}
\begin{frame}
\frametitle{Unrelated Title}


\begin{itemize}
\item Goal commitment: high -> performance high, facilitating goal commitment: goal importanceself-efficacy, belief in attainmentFeedback (feedback to reveal progress concerning goals is required)Task complexity (higher complexity -> smaller effect of Goal Core = lower performance)
\end{itemize}

\note[item]{}
\end{frame}
\begin{frame}
\frametitle{Unrelated Title}


\begin{itemize}
\item What: Target benefits, Strategic alignmentWhen: Timeframe for realizing benefitsWho: Benefits OwnerWhy: Metrics, Assumptions, RisksHow: Metrics, Assumptions
\end{itemize}

\note[item]{}
\end{frame}
\begin{frame}
\frametitle{Unrelated Title}


\begin{itemize}
\item Business Case Document
\end{itemize}

\note[item]{}
\end{frame}
\begin{frame}
\frametitle{Unrelated Title}


\begin{itemize}
\item CPI - it's a measure of cost efficiency, and is the ratio of earned value to actual costs
\end{itemize}

\note[item]{}
\end{frame}
\begin{frame}
\frametitle{Unrelated Title}


\begin{itemize}
\item Resource Breakdown Structure
\end{itemize}

\note[item]{}
\end{frame}
\begin{frame}
\frametitle{Unrelated Title}


\begin{itemize}
\item - Create environment for participation- Help team create viable solutions- Collaboration/knowledge-sharing
\end{itemize}

\note[item]{}
\end{frame}
\begin{frame}
\frametitle{Unrelated Title}


\begin{itemize}
\item Communication Technology - conversations, written documents, meetings, databases, social media, websites
\end{itemize}

\note[item]{}
\end{frame}
\begin{frame}
\frametitle{Unrelated Title}


\begin{itemize}
\item InteractivePushPull
\end{itemize}

\note[item]{}
\end{frame}
\begin{frame}
\frametitle{Unrelated Title}


\begin{itemize}
\item Plan quality management
\end{itemize}

\note[item]{}
\end{frame}
\begin{frame}
\frametitle{Unrelated Title}


\begin{itemize}
\item Represent communication in most basic form
\end{itemize}

\note[item]{}
\end{frame}
\begin{frame}
\frametitle{Unrelated Title}

\begin{center}
\includegraphics[width=0.9\textwidth,height=0.9\textheight,keepaspectratio]{/Users/I516998/Library/Application Support/Anki2/User 1/collection.media/image-ab47fad8127ce2b447ec1690dd70ba3588254979.jpg}
\end{center}


\note[item]{}
\end{frame}
\begin{frame}
\frametitle{Unrelated Title}


\begin{itemize}
\item Prevention of poor quality in products, deliverables or services
\end{itemize}

\note[item]{}
\end{frame}
\begin{frame}
\frametitle{Unrelated Title}


\begin{itemize}
\item Evaluating, measuring, auditing, testing deliverables
\end{itemize}

\note[item]{}
\end{frame}
\begin{frame}
\frametitle{Unrelated Title}


\begin{itemize}
\item Costs related to nonconformance of products
\end{itemize}

\note[item]{}
\end{frame}
\begin{frame}
\frametitle{Unrelated Title}

\begin{center}
\includegraphics[width=0.9\textwidth,height=0.9\textheight,keepaspectratio]{/Users/I516998/Library/Application Support/Anki2/User 1/collection.media/image-b4f74c0cfb03d3a25f1576991b6cda4c2032aa2c.jpg}
\end{center}


\note[item]{}
\end{frame}
\begin{frame}
\frametitle{Unrelated Title}


\begin{itemize}
\item Availability of features/capabilitiesOrg tolerance for changesTime cadence for subsequent releases
\end{itemize}

\note[item]{}
\end{frame}
\begin{frame}
\frametitle{Unrelated Title}


\begin{itemize}
\item Default should be to resolve problems in favor of the customer whenever possible.
\end{itemize}

\note[item]{}
\end{frame}
\begin{frame}
\frametitle{Unrelated Title}


\begin{itemize}
\item Divisions of a project where extra control is needed to effectively manage the completion of a major deliverable.
\end{itemize}

\note[item]{}
\end{frame}
\begin{frame}
\frametitle{Unrelated Title}


\begin{itemize}
\item Lowest level of WBS for which cost and duration are estimated and managed.
\end{itemize}

\note[item]{}
\end{frame}
\begin{frame}
\frametitle{Unrelated Title}


\begin{itemize}
\item Identify stakeholders - anytime there is significant change, you ought to identify new stakeholders and see if any have been added to or left the project.
\end{itemize}

\note[item]{}
\end{frame}
\begin{frame}
\frametitle{Unrelated Title}


\begin{itemize}
\item Information on specific change requests
\end{itemize}

\note[item]{}
\end{frame}
\begin{frame}
\frametitle{Unrelated Title}


\begin{itemize}
\item All approved/rejected changes to scope
\end{itemize}

\note[item]{}
\end{frame}
\begin{frame}
\frametitle{Unrelated Title}


\begin{itemize}
\item Monitor Risks; Audits
\end{itemize}

\note[item]{}
\end{frame}
\begin{frame}
\frametitle{Unrelated Title}


\begin{itemize}
\item An output
\end{itemize}

\note[item]{}
\end{frame}
\begin{frame}
\frametitle{Unrelated Title}


\begin{itemize}
\item Identifying stakeholders
\end{itemize}

\note[item]{}
\end{frame}
\begin{frame}
\frametitle{Unrelated Title}


\begin{itemize}
\item In the planning process
\end{itemize}

\note[item]{}
\end{frame}
\begin{frame}
\frametitle{Unrelated Title}


\begin{itemize}
\item Enhances brainstorming with a voting process used to rank most useful ideas
\end{itemize}

\note[item]{}
\end{frame}
\begin{frame}
\frametitle{Unrelated Title}


\begin{itemize}
\item A change management methodologyHelps manage resistance to change
\end{itemize}

\note[item]{}
\end{frame}
\begin{frame}
\frametitle{Unrelated Title}


\begin{itemize}
\item Change Management methodology
\end{itemize}

\note[item]{}
\end{frame}
\begin{frame}
\frametitle{Unrelated Title}


\begin{itemize}
\item Change Management ModelAlignment of the 7 S'sStructureStrategySystemsSkillsStyleStaffShared Values
\end{itemize}

\note[item]{}
\end{frame}
\begin{frame}
\frametitle{Unrelated Title}


\begin{itemize}
\item Tracking stakeholder's attitudes currently and where the team desire them to be at.
\end{itemize}

\note[item]{}
\end{frame}
\begin{frame}
\frametitle{Unrelated Title}


\begin{itemize}
\item Quantitative Risk Analysis
\end{itemize}

\note[item]{}
\end{frame}
\begin{frame}
\frametitle{Unrelated Title}


\begin{itemize}
\item Qualitative Risk Analysis
\end{itemize}

\note[item]{}
\end{frame}
\begin{frame}
\frametitle{Unrelated Title}


\begin{itemize}
\item Probability & Impact matrixRisk urgency assessmentExpert JudgmentRisk Data quality assessmentand more.
\end{itemize}

\note[item]{}
\end{frame}
\begin{frame}
\frametitle{Unrelated Title}


\begin{itemize}
\item EEFs
\end{itemize}

\note[item]{}
\end{frame}
\begin{frame}
\frametitle{Unrelated Title}


\begin{itemize}
\item No - the benefits cost management plan is not a document referenced in the PMBOK Guide. 
\end{itemize}

\note[item]{}
\end{frame}
\begin{frame}
\frametitle{Unrelated Title}


\begin{itemize}
\item Plan timeboxed meetingsImprove visibility of goals and activity statusProject visibility among team
\end{itemize}

\note[item]{}
\end{frame}
\begin{frame}
\frametitle{Unrelated Title}


\begin{itemize}
\item 1) Prevent root cause2) Identify/Document change3) Evaluate impact4) Issue change request5) Perform Integrated Change Control (get CCB Approval)6) Update change log/project documents7) Communicate change/get buy-in8) Direct & manage project according to new plan
\end{itemize}

\note[item]{}
\end{frame}
\begin{frame}
\frametitle{Unrelated Title}


\begin{itemize}
\item Refer to the perform integrated change control process to examine the issue
\end{itemize}

\note[item]{}
\end{frame}
\begin{frame}
\frametitle{Unrelated Title}


\begin{itemize}
\item Embraces much changeOnly works on projectsPM has highest level of authorityCon: may have no work at project end
\end{itemize}

\note[item]{}
\end{frame}
\begin{frame}
\frametitle{Unrelated Title}


\begin{itemize}
\item Little/None resources or authorityLess about change, fewer projectsMore about day-to-day operationsFunctional manager controls budget
\end{itemize}

\note[item]{}
\end{frame}
\begin{frame}
\frametitle{Unrelated Title}


\begin{itemize}
\item Weak and Balanced have low resources, low PM authority, low-to-mixed budget control, part-time PM role availability
\end{itemize}

\note[item]{}
\end{frame}
\begin{frame}
\frametitle{Unrelated Title}


\begin{itemize}
\item This means the team is co-located..doesn't refer to a type of organization
\end{itemize}

\note[item]{}
\end{frame}
\begin{frame}
\frametitle{Unrelated Title}


\begin{itemize}
\item Team Charter
\end{itemize}

\note[item]{}
\end{frame}
\begin{frame}
\frametitle{Unrelated Title}


\begin{itemize}
\item Validation
\end{itemize}

\note[item]{}
\end{frame}
\begin{frame}
\frametitle{Unrelated Title}


\begin{itemize}
\item Allows participants time to consider questions before the group session is held
\end{itemize}

\note[item]{}
\end{frame}
\begin{frame}
\frametitle{Unrelated Title}


\begin{itemize}
\item Consolidating ideas into a single map to reflect commonality or difference
\end{itemize}

\note[item]{}
\end{frame}
\begin{frame}
\frametitle{Unrelated Title}


\begin{itemize}
\item Recognition is usually more personal and unexpected.Reward is usually expected for achieving agreed result
\end{itemize}

\note[item]{}
\end{frame}
\begin{frame}
\frametitle{Unrelated Title}


\begin{itemize}
\item WBS and scope statement
\end{itemize}

\note[item]{}
\end{frame}
\begin{frame}
\frametitle{Unrelated Title}


\begin{itemize}
\item This is Develop Project Charter
\end{itemize}

\note[item]{}
\end{frame}
\begin{frame}
\frametitle{Unrelated Title}


\begin{itemize}
\item Team members pull work from the queue
\end{itemize}

\note[item]{}
\end{frame}
\begin{frame}
\frametitle{Unrelated Title}


\begin{itemize}
\item A minimum marketable feature that would be acceptable by customer (although a team may deliver a functional product, it may not meet MMF)
\end{itemize}

\note[item]{}
\end{frame}
\begin{frame}
\frametitle{Unrelated Title}


\begin{itemize}
\item Tying together all elements of the project plan to ensure their implementation (creation of plan, execution, document of changes, etc.)
\end{itemize}

\note[item]{}
\end{frame}
\begin{frame}
\frametitle{Unrelated Title}


\begin{itemize}
\item Monitoring cost performance to detect cost variances from budget and understand what's causing them. This is an ongoing process.
\end{itemize}

\note[item]{}
\end{frame}
\begin{frame}
\frametitle{Unrelated Title}


\begin{itemize}
\item Unknown unknowns; or unidentified risks.
\end{itemize}

\note[item]{}
\end{frame}
\begin{frame}
\frametitle{Unrelated Title}


\begin{itemize}
\item AKA Walking the board?
\end{itemize}

\note[item]{}
\end{frame}
\begin{frame}
\frametitle{Unrelated Title}


\begin{itemize}
\item deliverables, phases, subproject, or some combo.
\end{itemize}

\note[item]{}
\end{frame}
\begin{frame}
\frametitle{Unrelated Title}


\begin{itemize}
\item Understanding of the queryWhether a document satisfies the query or not
\end{itemize}

\note[item]{}
\end{frame}
\begin{frame}
\frametitle{Unrelated Title}


\begin{itemize}
\item Understanding the user's information need is uncertain (good query and query representation?)Estimating document relevanceUncertainty from selection of document representationUncertainty from matching query and documents
\end{itemize}

\note[item]{}
\end{frame}
\begin{frame}
\frametitle{Unrelated Title}


\begin{itemize}
\item The information retrieval system will reach best obtainable efficiency if the documents are ranked decreasingly according to their probability of relevance
\item Probabilistic retrieval models aim to answer the following question: What is the probability that the user will judge this document as relevant for this query? (Compute best estimate from available data)
\end{itemize}

\note[item]{}
\end{frame}
\begin{frame}
\frametitle{Unrelated Title}

\begin{center}
\includegraphics[width=0.9\textwidth,height=0.9\textheight,keepaspectratio]{/Users/I516998/Library/Application Support/Anki2/User 1/collection.media/paste-7c42014b81aed8a47cc5b52bde7eb2e65e99a474.jpg}
\end{center}


\note[item]{}
\end{frame}
\begin{frame}
\frametitle{Unrelated Title}

\begin{center}
\includegraphics[width=0.9\textwidth,height=0.9\textheight,keepaspectratio]{/Users/I516998/Library/Application Support/Anki2/User 1/collection.media/paste-5e767d7b7e0cb78bd60d965a77841cd8eb0b1cb8.jpg}
\end{center}


\note[item]{}
\end{frame}
\begin{frame}
\frametitle{Unrelated Title}

\begin{center}
\includegraphics[width=0.9\textwidth,height=0.9\textheight,keepaspectratio]{/Users/I516998/Library/Application Support/Anki2/User 1/collection.media/paste-c1cc4ca7bd182bc432c7cdca6d864dc209350496.jpg}
\includegraphics[width=0.9\textwidth,height=0.9\textheight,keepaspectratio]{/Users/I516998/Library/Application Support/Anki2/User 1/collection.media/paste-e5204fd2034a361b7bb3c772e6499b657e33e10d.jpg}
\includegraphics[width=0.9\textwidth,height=0.9\textheight,keepaspectratio]{/Users/I516998/Library/Application Support/Anki2/User 1/collection.media/paste-d9545b3bfab72a690ef2c4c88e3cc098a9ae2b45.jpg}
\end{center}


\note[item]{}
\end{frame}
\begin{frame}
\frametitle{Unrelated Title}


\begin{itemize}
\item Making small changes and getting feedback from customer
\end{itemize}

\note[item]{}
\end{frame}
\begin{frame}
\frametitle{Unrelated Title}


\begin{itemize}
\item Alternative dispute resolution - provision for termination of a contract
\end{itemize}

\note[item]{}
\end{frame}
\begin{frame}
\frametitle{Unrelated Title}


\begin{itemize}
\item Governance gateStage gateTollgatePhase EndKill PointQuality Gate
\end{itemize}

\note[item]{}
\end{frame}
\begin{frame}
\frametitle{Unrelated Title}


\begin{itemize}
\item A technique that reviews ALL risks. It assesses the risk exposure events to overall project objectives and determines the confidence levels of achieving those objectives. 
\end{itemize}

\note[item]{}
\end{frame}
\begin{frame}
\frametitle{Unrelated Title}


\begin{itemize}
\item probability x impact
\end{itemize}

\note[item]{}
\end{frame}
\begin{frame}
\frametitle{Unrelated Title}


\begin{itemize}
\item classifies/groups stakeholders on basis of authority, immediate needs, and how appropriate their involvement is
\end{itemize}

\note[item]{}
\end{frame}
\begin{frame}
\frametitle{Unrelated Title}


\begin{itemize}
\item Single Point of Contact - needs to be established for working with external stakeholdersSingle Point of Failure - when you only have one person to do the job (different from SPOC)
\end{itemize}

\note[item]{}
\end{frame}
\begin{frame}
\frametitle{Unrelated Title}


\begin{itemize}
\item Taking sides, resolution is not enough
\end{itemize}

\note[item]{}
\end{frame}
\begin{frame}
\frametitle{Unrelated Title}


\begin{itemize}
\item Every day passing conflict, easy to resolve
\end{itemize}

\note[item]{}
\end{frame}
\begin{frame}
\frametitle{Unrelated Title}


\begin{itemize}
\item 1 Problem to Solve2 Disagreement3 Contest4 Crusade5 World War
\end{itemize}

\note[item]{}
\end{frame}
\begin{frame}
\frametitle{Unrelated Title}


\begin{itemize}
\item A change management model
\end{itemize}

\note[item]{}
\end{frame}
\begin{frame}
\frametitle{Unrelated Title}


\begin{itemize}
\item In the communication plan and during the planning stage
\end{itemize}

\note[item]{}
\end{frame}
\begin{frame}
\frametitle{Unrelated Title}


\begin{itemize}
\item Team Charter (different from escalation)
\end{itemize}

\note[item]{}
\end{frame}
\begin{frame}
\frametitle{Unrelated Title}


\begin{itemize}
\item An EEF & it can have an effect on resource availability, how projects are selected, approved, conducted, etc., as well as PM authority and budget controls
\end{itemize}

\note[item]{}
\end{frame}
\begin{frame}
\frametitle{Unrelated Title}


\begin{itemize}
\item Let them choose the tools they want to use - buy-in from the team
\end{itemize}

\note[item]{}
\end{frame}
\begin{frame}
\frametitle{Unrelated Title}


\begin{itemize}
\item Claims administration technique
\end{itemize}

\note[item]{}
\end{frame}
\begin{frame}
\frametitle{Unrelated Title}


\begin{itemize}
\item Add the regulatory requirement to your requirements documentation; no need for a change request in the planning stage
\end{itemize}

\note[item]{}
\end{frame}
\begin{frame}
\frametitle{Unrelated Title}


\begin{itemize}
\item Only for use during Execution, when all of your baselines are complete, and you need to request a change for one of them
\end{itemize}

\note[item]{}
\end{frame}
\begin{frame}
\frametitle{Unrelated Title}


\begin{itemize}
\item B. Add up raw materials and labor costs for each activity that must be done.
\end{itemize}

\note[item]{}
\end{frame}
\begin{frame}
\frametitle{Unrelated Title}


\begin{itemize}
\item A kick off meeting should be held with stakeholders, project management team members, and other stakeholders to gain commitment from the team for the project, alert them of the phase gate and explain the roles and responsibilities of each stakeholder.
\end{itemize}

\note[item]{}
\end{frame}
\begin{frame}
\frametitle{Unrelated Title}


\begin{itemize}
\item Planning
\end{itemize}

\note[item]{}
\end{frame}
\begin{frame}
\frametitle{Unrelated Title}


\begin{itemize}
\item TrainingDocument processesEquipmentTime to do it right
\end{itemize}

\note[item]{}
\end{frame}
\begin{frame}
\frametitle{Unrelated Title}


\begin{itemize}
\item TestingDestructive Testing LossInspections
\end{itemize}

\note[item]{}
\end{frame}
\begin{frame}
\frametitle{Unrelated Title}


\begin{itemize}
\item Internal - rework and scrapExternal - liabilities, warranty, lost business
\end{itemize}

\note[item]{}
\end{frame}
\begin{frame}
\frametitle{Unrelated Title}


\begin{itemize}
\item Analyze the current information you have prior to making a decision - you don't want to proactively head down the wrong path without doing the proper analysis first.
\end{itemize}

\note[item]{}
\end{frame}
\begin{frame}
\frametitle{Unrelated Title}


\begin{itemize}
\item Ideally go through the change control process
\end{itemize}

\note[item]{}
\end{frame}
\begin{frame}
\frametitle{Unrelated Title}


\begin{itemize}
\item Monitor risks
\end{itemize}

\note[item]{}
\end{frame}
\begin{frame}
\frametitle{Unrelated Title}


\begin{itemize}
\item Go through the change control process!
\end{itemize}

\note[item]{}
\end{frame}
\begin{frame}
\frametitle{Unrelated Title}


\begin{itemize}
\item 11.2 Identify risks - document them! Reviewing historical data/regulatory framework11.3 Perform Qualitative Risk Analysis - categorize/prioritize11.4 Perform Quantitative Risk Analysis - calculate/quantify11.5 Plan Risk Responses (Avoid, mitigate, etc.)
\end{itemize}

\note[item]{}
\end{frame}
\begin{frame}
\frametitle{Unrelated Title}


\begin{itemize}
\item Interpersonal skills
\end{itemize}

\note[item]{}
\end{frame}
\begin{frame}
\frametitle{Unrelated Title}


\begin{itemize}
\item Risks will always be
tracked on the Risk Register, NOT the Issue Log.  Risks will continue to be tracked on the Risk Register.  The Issue Log is for items that had NOT been
previously identified and documented in the Risk Register.  
\end{itemize}

\note[item]{}
\end{frame}
\begin{frame}
\frametitle{Unrelated Title}


\begin{itemize}
\item Risk register, risk report, lessons learned
\end{itemize}

\note[item]{}
\end{frame}
\begin{frame}
\frametitle{Unrelated Title}


\begin{itemize}
\item net present value - should choose the highest value, regardless of timeframe
\end{itemize}

\note[item]{}
\end{frame}
\begin{frame}
\frametitle{Unrelated Title}


\begin{itemize}
\item In the flow of processes, you would next "Identify Stakeholders"
\end{itemize}

\note[item]{}
\end{frame}
\begin{frame}
\frametitle{Unrelated Title}


\begin{itemize}
\item Throughout the project
\end{itemize}

\note[item]{}
\end{frame}
\begin{frame}
\frametitle{Unrelated Title}


\begin{itemize}
\item Control - checking to see if the product is correctManage - did you follow the right process
\end{itemize}

\note[item]{}
\end{frame}
\begin{frame}
\frametitle{Unrelated Title}


\begin{itemize}
\item Conduct a risk assessment for each proposed change
\end{itemize}

\note[item]{}
\end{frame}
\begin{frame}
\frametitle{Unrelated Title}


\begin{itemize}
\item 


- Assigning a challenging goal raises self-efficacy -> expression of confidence by leaderAdditionally: - Goals and self-efficacy mediate the effect of visionary leadership on employee performance and the effect of feedback- Self-set goals + self-efficacy mediate the effect on monetary incentive



\end{itemize}

\note[item]{}
\end{frame}
\begin{frame}
\frametitle{Unrelated Title}

\begin{center}
\includegraphics[width=0.9\textwidth,height=0.9\textheight,keepaspectratio]{/Users/I516998/Library/Application Support/Anki2/User 1/collection.media/Bildschirmfoto 2022-10-20 um 14.03.36.png}
\end{center}

\begin{itemize}
\item Goal Difficulty: Hard -> better performance (correlation coefficient r)
\end{itemize}

\note[item]{}
\end{frame}
\begin{frame}
\frametitle{Unrelated Title}


\begin{itemize}
\item 



Setting high and specific goals can lead to increases in employee productivity


SMART goals (specific, measurable, attainable, relevant, time-bound) Helpful in interviews for selection process (assess an applicant‘s goals or intentions)Self-regulation -> Higher self-efficacy -> setting higher goals -> increases
performance 
\end{itemize}

\note[item]{}
\end{frame}
\begin{frame}
\frametitle{Unrelated Title}


\begin{itemize}
\item 



Learning goals vs. performance goals (in complex tasks: learning goals better)


Risks: high performance goals may result in riskier strategiesConflicts: assigned goals may conflict with personal goalsToo specific/challenging goals -> lose sight of other important aspects :(
\end{itemize}

\note[item]{}
\end{frame}
\begin{frame}
\frametitle{Unrelated Title}

\begin{center}
\includegraphics[width=0.9\textwidth,height=0.9\textheight,keepaspectratio]{/Users/I516998/Library/Application Support/Anki2/User 1/collection.media/Bildschirmfoto 2022-10-20 um 14.15.39.png}
\end{center}


\note[item]{}
\end{frame}
\begin{frame}
\frametitle{Unrelated Title}


\begin{itemize}
\item 



Behaviors differ in the degree to which they are autonomous versus controlled
Autonomous and controlled motivations differ in regulatory processes and their accompanying experiencesIntrinsic motivation = prototypically autonomous



Extrinsic motivation varies -> autonomous vs. controlled 





\end{itemize}

\note[item]{}
\end{frame}
\begin{frame}
\frametitle{Unrelated Title}


\begin{itemize}
\item Initiated and maintained by contingencies external to the personEnergized into action only when the action is instrumental Prototype of controlled/extrinsic motivation
\end{itemize}

\note[item]{}
\end{frame}
\begin{frame}
\frametitle{Unrelated Title}


\begin{itemize}
\item 




Externally and intrinsically regulated behavior = endpoints on a continuum (Degree of internalization can vary)Internalization: people taking in values, attitudes, or regulatory structures: external regulation -> internal regulation & external contingency no longer needed




\end{itemize}

\note[item]{}
\end{frame}
\begin{frame}
\frametitle{Unrelated Title}


\begin{itemize}
\item 


Introjection (Regulation not accepted as own but ego involvement, still directed by rewards/punishments)
Identification (Behavior is aligned with their personal goals and identities, importance)Integration (Behavior is an integral part of who the person is) -> close to intrinsic motivation / difficult to grasp



\end{itemize}

\note[item]{}
\end{frame}
\begin{frame}
\frametitle{Unrelated Title}

\begin{center}
\includegraphics[width=0.9\textwidth,height=0.9\textheight,keepaspectratio]{/Users/I516998/Library/Application Support/Anki2/User 1/collection.media/Bildschirmfoto 2022-10-20 um 15.01.52.png}
\end{center}


\note[item]{}
\end{frame}
\begin{frame}
\frametitle{Unrelated Title}


\begin{itemize}
\item Product requirements; Requirements Documentation
\end{itemize}

\note[item]{}
\end{frame}
\begin{frame}
\frametitle{Unrelated Title}


\begin{itemize}
\item No, it has agile release planning, product roadmap, product vision
\end{itemize}

\note[item]{}
\end{frame}
\begin{frame}
\frametitle{Unrelated Title}

\begin{center}
\includegraphics[width=0.9\textwidth,height=0.9\textheight,keepaspectratio]{/Users/I516998/Library/Application Support/Anki2/User 1/collection.media/paste-39a7ea82366babe9b05fe254745e2c3edb01cc33.jpg}
\end{center}


\note[item]{}
\end{frame}
\begin{frame}
\frametitle{Unrelated Title}

\begin{center}
\includegraphics[width=0.9\textwidth,height=0.9\textheight,keepaspectratio]{/Users/I516998/Library/Application Support/Anki2/User 1/collection.media/paste-ad130535a2096d1de5e1cd90d8de6d29b3f64e08.jpg}
\includegraphics[width=0.9\textwidth,height=0.9\textheight,keepaspectratio]{/Users/I516998/Library/Application Support/Anki2/User 1/collection.media/paste-3e4fff3f461bb4701dacb2d7d6c26d0a3a913c75.jpg}
\end{center}

\begin{itemize}
\item Terms in the documents and query are independentAllows to represent the document probability as product of term probabilities
\end{itemize}

\note[item]{}
\end{frame}
\begin{frame}
\frametitle{Unrelated Title}

\begin{center}
\includegraphics[width=0.9\textwidth,height=0.9\textheight,keepaspectratio]{/Users/I516998/Library/Application Support/Anki2/User 1/collection.media/paste-4fd6d903f8ba25cd8e5b42a0d9720fe167a27f43.jpg}
\includegraphics[width=0.9\textwidth,height=0.9\textheight,keepaspectratio]{/Users/I516998/Library/Application Support/Anki2/User 1/collection.media/paste-b2864a40709c67130061830c621461cdf87afe86.jpg}
\end{center}

\begin{itemize}
\item We must go for two reasonable assuptions:
\item Putting this assumptions into the model we can compute the relevance score for a document without any relevance judgements:
\end{itemize}

\note[item]{}
\end{frame}
\begin{frame}
\frametitle{Unrelated Title}


\begin{itemize}
\item When the expense/hassle outweighs the effects it would cause if it occurred
\end{itemize}

\note[item]{}
\end{frame}
\begin{frame}
\frametitle{Unrelated Title}


\begin{itemize}
\item Risk tolerance and thresholdsProject's current risk exposureMethod of risk management
\end{itemize}

\note[item]{}
\end{frame}
\begin{frame}
\frametitle{Unrelated Title}


\begin{itemize}
\item A comprehensive document that lists the overall project risk exposure
\end{itemize}

\note[item]{}
\end{frame}
\begin{frame}
\frametitle{Unrelated Title}


\begin{itemize}
\item The Project Charter - first documents overall project riskRisk Report - documents overall project risk
\end{itemize}

\note[item]{}
\end{frame}
\begin{frame}
\frametitle{Unrelated Title}


\begin{itemize}
\item Contained in the Risk Management PlanDescribes risk categories and is a way to identify risksLowest level of RBS can be used as a checklist
\end{itemize}

\note[item]{}
\end{frame}
\begin{frame}
\frametitle{Unrelated Title}


\begin{itemize}
\item Specifically for risk identification
\end{itemize}

\note[item]{}
\end{frame}
\begin{frame}
\frametitle{Unrelated Title}


\begin{itemize}
\item Contingency risks are quantitative risk analysis; you would need to do qualitative risk analysis because you need to prioritize the risks first
\end{itemize}

\note[item]{}
\end{frame}
\begin{frame}
\frametitle{Unrelated Title}


\begin{itemize}
\item Increases: cost of changeDecreases: risk and stakeholder influence
\end{itemize}

\note[item]{}
\end{frame}
\begin{frame}
\frametitle{Unrelated Title}


\begin{itemize}
\item PIPPPIM (plan, identify, perform, perform, implement, monitor)
\end{itemize}

\note[item]{}
\end{frame}
\begin{frame}
\frametitle{Unrelated Title}


\begin{itemize}
\item Might involve submitting a change request that modifies type or number of resources, or a procurement activity
\end{itemize}

\note[item]{}
\end{frame}
\begin{frame}
\frametitle{Unrelated Title}


\begin{itemize}
\item Add work packages to the WBS and adjust your schedule
\end{itemize}

\note[item]{}
\end{frame}
\begin{frame}
\frametitle{Unrelated Title}


\begin{itemize}
\item At the beginning because uncertainty is highest - the more you know about a project, the better chance you have at completing it successfully
\end{itemize}

\note[item]{}
\end{frame}
\begin{frame}
\frametitle{Unrelated Title}


\begin{itemize}
\item No, quantitative is not always necessary, but qualitative is
\end{itemize}

\note[item]{}
\end{frame}
\begin{frame}
\frametitle{Unrelated Title}


\begin{itemize}
\item describes the prioritization of user stories/features based on value compared to risk
\end{itemize}

\note[item]{}
\end{frame}
\begin{frame}
\frametitle{Unrelated Title}


\begin{itemize}
\item Variability - uncertainty about some key characteristics (unseasonable weather)Ambiguity - uncertainty about future items (regulations and policy changes)
\end{itemize}

\note[item]{}
\end{frame}
\begin{frame}
\frametitle{Unrelated Title}


\begin{itemize}
\item A computerized analysis that simulates combined effects of individual project risks
\end{itemize}

\note[item]{}
\end{frame}
\begin{frame}
\frametitle{Unrelated Title}


\begin{itemize}
\item diagramming technique for evaluating implications amongst multiple options
\end{itemize}

\note[item]{}
\end{frame}
\begin{frame}
\frametitle{Unrelated Title}


\begin{itemize}
\item Analytical technique where input variables are examined in relation to output to create a statistical relationship
\end{itemize}

\note[item]{}
\end{frame}
\begin{frame}
\frametitle{Unrelated Title}


\begin{itemize}
\item IndividualOverall
\end{itemize}

\note[item]{}
\end{frame}
\begin{frame}
\frametitle{Unrelated Title}


\begin{itemize}
\item An anonymous risk identification technique
\end{itemize}

\note[item]{}
\end{frame}
\begin{frame}
\frametitle{Unrelated Title}


\begin{itemize}
\item Project charter creation
\end{itemize}

\note[item]{}
\end{frame}
\begin{frame}
\frametitle{Unrelated Title}


\begin{itemize}
\item Project charter
\end{itemize}

\note[item]{}
\end{frame}
\begin{frame}
\frametitle{Unrelated Title}


\begin{itemize}
\item Discern project goals and the business case - this is creating the project charter and the goals/business case will be needed.
\end{itemize}

\note[item]{}
\end{frame}
\begin{frame}
\frametitle{Unrelated Title}


\begin{itemize}
\item Monitor/Control; Execution
\end{itemize}

\note[item]{}
\end{frame}
\begin{frame}
\frametitle{Unrelated Title}


\begin{itemize}
\item Application of knowledge, skills, tools and techniques to project activities to meet project requirements
\end{itemize}

\note[item]{}
\end{frame}
\begin{frame}
\frametitle{Unrelated Title}


\begin{itemize}
\item Executing
\end{itemize}

\note[item]{}
\end{frame}
\begin{frame}
\frametitle{Unrelated Title}


\begin{itemize}
\item At the beginning of the project (listing out requirements and whatnot)
\end{itemize}

\note[item]{}
\end{frame}
\begin{frame}
\frametitle{Unrelated Title}


\begin{itemize}
\item Work performance data
\end{itemize}

\note[item]{}
\end{frame}
\begin{frame}
\frametitle{Unrelated Title}


\begin{itemize}
\item  Variance Analysis – when controlling scope of the
project, the PM should be concerned with both the product and project scope.
Variance analysis is a technique used to determine the cause and the degree of
the variance relative to the baseline.
\end{itemize}

\note[item]{}
\end{frame}
\begin{frame}
\frametitle{Unrelated Title}


\begin{itemize}
\item They should choose the ES = LS in all their activities. This means there will be no float or slack as all tasks need to begin ASAP
\end{itemize}

\note[item]{}
\end{frame}
\begin{frame}
\frametitle{Unrelated Title}


\begin{itemize}
\item Sequence activities
\item Estimate activity resourcesEstimate durationsDevelop a schedule!
\end{itemize}

\note[item]{}
\end{frame}
\begin{frame}
\frametitle{Unrelated Title}


\begin{itemize}
\item Using components from previous project's schedule network diagramIdentical portions could help complete it
\end{itemize}

\note[item]{}
\end{frame}
\begin{frame}
\frametitle{Unrelated Title}


\begin{itemize}
\item Modeling techniques - like what-if scenario analysis to examine possible adverse situations
\end{itemize}

\note[item]{}
\end{frame}
\begin{frame}
\frametitle{Unrelated Title}


\begin{itemize}
\item The project will be delayed
\end{itemize}

\note[item]{}
\end{frame}
\begin{frame}
\frametitle{Unrelated Title}


\begin{itemize}
\item comparing actual practices to those of comparable practices to identify best practices and helps measure performance
\end{itemize}

\note[item]{}
\end{frame}
\begin{frame}
\frametitle{Unrelated Title}


\begin{itemize}
\item Smoothing helps prevent people from being overworked (lengthens schedule)
\end{itemize}

\note[item]{}
\end{frame}
\begin{frame}
\frametitle{Unrelated Title}


\begin{itemize}
\item Agile Release Planning - how many sprints, features per sprint, and approximate target date
\end{itemize}

\note[item]{}
\end{frame}
\begin{frame}
\frametitle{Unrelated Title}


\begin{itemize}
\item Schedule baseline is a separate document, it needs a change request to be changed
\end{itemize}

\note[item]{}
\end{frame}
\begin{frame}
\frametitle{Unrelated Title}


\begin{itemize}
\item Interdependencies between various activities
\end{itemize}

\note[item]{}
\end{frame}
\begin{frame}
\frametitle{Unrelated Title}


\begin{itemize}
\item You could seek additional funding to crash critical path tasks to shorten project duration
\end{itemize}

\note[item]{}
\end{frame}
\begin{frame}
\frametitle{Unrelated Title}


\begin{itemize}
\item You may have less experienced resources being used - could account for cost overruns and schedule delays
\end{itemize}

\note[item]{}
\end{frame}
\begin{frame}
\frametitle{Unrelated Title}


\begin{itemize}
\item Schedule
Variance – EV-PV
\end{itemize}

\note[item]{}
\end{frame}
\begin{frame}
\frametitle{Unrelated Title}


\begin{itemize}
\item Control Quality - certifying them as complete and correctly completed
\end{itemize}

\note[item]{}
\end{frame}
\begin{frame}
\frametitle{Unrelated Title}


\begin{itemize}
\item Measurements tightly clustered around mean in a random pattern; within upper/lower control limits
\end{itemize}

\note[item]{}
\end{frame}
\begin{frame}
\frametitle{Unrelated Title}


\begin{itemize}
\item An appraisal COQ
\end{itemize}

\note[item]{}
\end{frame}
\begin{frame}
\frametitle{Unrelated Title}


\begin{itemize}
\item Variable - degree of conformityStatistical - sample from a lot to inspectAttribute - two results, either conform or not
\end{itemize}

\note[item]{}
\end{frame}
\begin{frame}
\frametitle{Unrelated Title}


\begin{itemize}
\item Data Analysis
\end{itemize}

\note[item]{}
\end{frame}
\begin{frame}
\frametitle{Unrelated Title}


\begin{itemize}
\item Audit
\end{itemize}

\note[item]{}
\end{frame}
\begin{frame}
\frametitle{Unrelated Title}


\begin{itemize}
\item Audit - because you're in the control quality process (where you would statistically sample, trend analysis and inspect)
\end{itemize}

\note[item]{}
\end{frame}
\begin{frame}
\frametitle{Unrelated Title}


\begin{itemize}
\item Overall project performance to ensure satisfaction to quality standards; monitoring specific project results to determine compliance to quality standards
\end{itemize}

\note[item]{}
\end{frame}
\begin{frame}
\frametitle{Unrelated Title}


\begin{itemize}
\item Contingency reserves
\end{itemize}

\note[item]{}
\end{frame}
\begin{frame}
\frametitle{Unrelated Title}


\begin{itemize}
\item  Real-world applications often need multiple MR jobs, but MR does not compose well. There are two major liitations:
\item Programmability (Many MR steps lead to boilerplate code)Performance (MR is heavy on disk, no optimization across multiple MR jobs)
\end{itemize}

\note[item]{}
\end{frame}
\begin{frame}
\frametitle{Unrelated Title}

\begin{center}
\includegraphics[width=0.9\textwidth,height=0.9\textheight,keepaspectratio]{/Users/I516998/Library/Application Support/Anki2/User 1/collection.media/paste-7cdb89daebd121cb2768515f1cf860411f7630e2.jpg}
\end{center}

\begin{itemize}
\item Each job recomputes the cluster centersResults/New cluster centers are wirtten to diskCheck if cluster centers should be recomputed (another job)
\end{itemize}

\note[item]{}
\end{frame}
\begin{frame}
\frametitle{Unrelated Title}


\begin{itemize}
\item Higher level languagesPrograms are written in a suitable higher-level language and are compiled into MapReduce jobsImprove programmabilityUse dataflow enginesInstead of compiling to MapReduce, a more suitable execution environment is usedImprove performance
\end{itemize}

\note[item]{}
\end{frame}
\begin{frame}
\frametitle{Unrelated Title}

\begin{center}
\includegraphics[width=0.9\textwidth,height=0.9\textheight,keepaspectratio]{/Users/I516998/Library/Application Support/Anki2/User 1/collection.media/paste-1683773955d771a66c2ba71e38c26eef6c4b65e9.jpg}
\end{center}

\begin{itemize}
\item Imperative programFocus on control flow, data at restDataflow programFocus on data flow, program at restModelled as a directed acyclic graph
\end{itemize}

\note[item]{}
\end{frame}
\begin{frame}
\frametitle{Unrelated Title}


\begin{itemize}
\item MapReduce:
\item Two operators, fixed logical dataflowOperators are parameterized by user defined functionsAutomatic parallelization at runtimeDataflow engines improvements:More operators (filtering, join ...)Improved implementationLogical/physical optimization
\end{itemize}

\note[item]{}
\end{frame}
\begin{frame}
\frametitle{Unrelated Title}


\begin{itemize}
\item Programmers provide logical dataflow (operators, function paratmeters, connections)Dataflow engine create phyiscal dataflow (logical and physical optimization)Execution an a cluster (automatic parallelization and failure handling)
\end{itemize}

\note[item]{}
\end{frame}
\begin{frame}
\frametitle{Unrelated Title}

\begin{center}
\includegraphics[width=0.9\textwidth,height=0.9\textheight,keepaspectratio]{/Users/I516998/Library/Application Support/Anki2/User 1/collection.media/paste-6c088e6040ba7898aa11c8c21d402283617da4c9.jpg}
\end{center}

\begin{itemize}
\item Resilient distributed dataset:
\item Conceptually: Collection of objectsImmutable, partitioned, fault-tolerantCan be operated on in parallelSpread across a cluster (stored in RAM, on disk or on both)Built through parallel transformationsAutomatically rebuilt on failureOperationsTransformations (map, filter, ...)Actions (count, collect, ...)
\end{itemize}

\note[item]{}
\end{frame}
\begin{frame}
\frametitle{Unrelated Title}


\begin{itemize}
\item Batch processing, interactive, real-time processingNative integration with Java/scala/PythonHigh-level abstractionsLibraries on top
\end{itemize}

\note[item]{}
\end{frame}
\begin{frame}
\frametitle{Unrelated Title}

\begin{center}
\includegraphics[width=0.9\textwidth,height=0.9\textheight,keepaspectratio]{/Users/I516998/Library/Application Support/Anki2/User 1/collection.media/paste-efa0018c8d3016aae001bd903ba322c0013afbbe.jpg}
\end{center}


\note[item]{}
\end{frame}
\begin{frame}
\frametitle{Unrelated Title}

\begin{center}
\includegraphics[width=0.9\textwidth,height=0.9\textheight,keepaspectratio]{/Users/I516998/Library/Application Support/Anki2/User 1/collection.media/paste-66fde2dc7f9cfab78afb12755c057dedb4938b08.jpg}
\end{center}

\begin{itemize}
\item Transofrmation takes an RDD and produces an RDDComputation is lazily
\end{itemize}

\note[item]{}
\end{frame}
\begin{frame}
\frametitle{Unrelated Title}

\begin{center}
\includegraphics[width=0.9\textwidth,height=0.9\textheight,keepaspectratio]{/Users/I516998/Library/Application Support/Anki2/User 1/collection.media/paste-5b36f7c50da8f7dd0493b8dd2768e8470573d5c5.jpg}
\end{center}

\begin{itemize}
\item Action takes an RDD and returns a value (reduce(), collect())
\end{itemize}

\note[item]{}
\end{frame}
\begin{frame}
\frametitle{Unrelated Title}

\begin{center}
\includegraphics[width=0.9\textwidth,height=0.9\textheight,keepaspectratio]{/Users/I516998/Library/Application Support/Anki2/User 1/collection.media/paste-17b18d82761977563eba4fa8ca2352d50f4ebc7a.jpg}
\end{center}

\begin{itemize}
\item Simulate Map: use flatMap or map transofrmationSumlate Shuffle:Use groupByKey transformationDefined on RDDs of k/v pairsCollects all values for each key and produces RDD of (key, collection of values)Simulate Reduce: use flatMap or map transformation
\end{itemize}

\note[item]{}
\end{frame}
\begin{frame}
\frametitle{Unrelated Title}

\begin{center}
\includegraphics[width=0.9\textwidth,height=0.9\textheight,keepaspectratio]{/Users/I516998/Library/Application Support/Anki2/User 1/collection.media/paste-84f578d414f155a1fc32e71b71112b6820e0d8bb.jpg}
\end{center}

\begin{itemize}
\item RDDs correspond to logical dataflowsan RDD is a dataflow that produces a collection of values when executedtransformations are computed lazilyTransformations combine/add operatorsActions compile and execute a dataflow
\end{itemize}

\note[item]{}
\end{frame}
\begin{frame}
\frametitle{Unrelated Title}

\begin{center}
\includegraphics[width=0.9\textwidth,height=0.9\textheight,keepaspectratio]{/Users/I516998/Library/Application Support/Anki2/User 1/collection.media/paste-19913b9ab8f387ba9890e523de343c478319bbee.jpg}
\end{center}

\begin{itemize}
\item Driver submits jobCluster master manages cluster workers and launches executorsCluster workers are machines that do the workExecutors are containers that run tasksA task can run in parallel on multiple executors (one per partition)
\end{itemize}

\note[item]{}
\end{frame}
\begin{frame}
\frametitle{Unrelated Title}


\begin{itemize}
\item When using an action, Sparks translates the logical representation into a physical execution planMultiple operations can be merged into tasksTasks are grouped into stages (Inside a stage no data movement is needed)Communication between tasks: pipeliningCommunication between stages: network (Spark generally uses memory)
\end{itemize}

\note[item]{}
\end{frame}
\begin{frame}
\frametitle{Unrelated Title}

\begin{center}
\includegraphics[width=0.9\textwidth,height=0.9\textheight,keepaspectratio]{/Users/I516998/Library/Application Support/Anki2/User 1/collection.media/paste-584c2da44c747ee492645b896926bc191686c3a7.jpg}
\includegraphics[width=0.9\textwidth,height=0.9\textheight,keepaspectratio]{/Users/I516998/Library/Application Support/Anki2/User 1/collection.media/paste-f106e5bd69f680b4d1b0f1d75fbbe879fb63a57e.jpg}
\end{center}

\begin{itemize}
\item List of partitionsList of dependencies on parent RDDsFunction to compute a partition from its parentsOptional: partitionerOptional: preferred locations for each partitionExample: FIlteredRDD
\end{itemize}

\note[item]{}
\end{frame}
\begin{frame}
\frametitle{Unrelated Title}

\begin{center}
\includegraphics[width=0.9\textwidth,height=0.9\textheight,keepaspectratio]{/Users/I516998/Library/Application Support/Anki2/User 1/collection.media/paste-5018c8072901e6ca35fa29778c9dff31cff768f2.jpg}
\end{center}

\begin{itemize}
\item Sets a flag to cache the intermediate resultsSpeed up computations that reuse the result
\end{itemize}

\note[item]{}
\end{frame}
\begin{frame}
\frametitle{Unrelated Title}

\begin{center}
\includegraphics[width=0.9\textwidth,height=0.9\textheight,keepaspectratio]{/Users/I516998/Library/Application Support/Anki2/User 1/collection.media/paste-9bc5fd2a89fd2ee856fbc6b87de3243367d1cc96.jpg}
\includegraphics[width=0.9\textwidth,height=0.9\textheight,keepaspectratio]{/Users/I516998/Library/Application Support/Anki2/User 1/collection.media/paste-9efa81abdfaacb8704d9a4b883cd64763ca666f6.jpg}
\end{center}


\note[item]{}
\end{frame}
\begin{frame}
\frametitle{Unrelated Title}


\begin{itemize}
\item Improved efficiencyImproved usabilitySpark unifies various big data analytics tasksFault tolerance and parallel processing retained
\end{itemize}

\note[item]{}
\end{frame}
\begin{frame}
\frametitle{Unrelated Title}

\begin{center}
\includegraphics[width=0.9\textwidth,height=0.9\textheight,keepaspectratio]{/Users/I516998/Library/Application Support/Anki2/User 1/collection.media/paste-02830b27ed2761d12efa5f95a05bfc8985b398be.jpg}
\end{center}

\begin{itemize}
\item Only the document term frequency is consideredDocument length, overall term frequency or document frequency are not considered
\end{itemize}

\note[item]{}
\end{frame}
\begin{frame}
\frametitle{Unrelated Title}

\begin{center}
\includegraphics[width=0.9\textwidth,height=0.9\textheight,keepaspectratio]{/Users/I516998/Library/Application Support/Anki2/User 1/collection.media/paste-9eeb3f4b676409b4086658c84d6400e69201f8a0.jpg}
\includegraphics[width=0.9\textwidth,height=0.9\textheight,keepaspectratio]{/Users/I516998/Library/Application Support/Anki2/User 1/collection.media/paste-8059a89fb7653d5440c6924b7cca103d63592f8e.jpg}
\end{center}


\note[item]{}
\end{frame}
\begin{frame}
\frametitle{Unrelated Title}


\begin{itemize}
\item Binary boolean representation of documents, queries and relevanceTerms are considered to be mutually independentOut-of-query terms do not affect retrievalDocument relevance scores are independent
\end{itemize}

\note[item]{}
\end{frame}
\begin{frame}
\frametitle{Unrelated Title}

\begin{center}
\includegraphics[width=0.9\textwidth,height=0.9\textheight,keepaspectratio]{/Users/I516998/Library/Application Support/Anki2/User 1/collection.media/paste-b2bd2abc228749300a62e09c645823b6cfb4f33f.jpg}
\end{center}

\begin{itemize}
\item Term frequency is consideredAssumption: all documents are equally longConstant k is used to normalize the scaling
\end{itemize}

\note[item]{}
\end{frame}
\begin{frame}
\frametitle{Unrelated Title}

\begin{center}
\includegraphics[width=0.9\textwidth,height=0.9\textheight,keepaspectratio]{/Users/I516998/Library/Application Support/Anki2/User 1/collection.media/paste-e8b0c7de116e5f9e981dd957ac3aee60c2372837.jpg}
\end{center}

\begin{itemize}
\item Model corrects the Two Poisson Model weight scaling to account for different document lengthsEffect of the model: If the document is long the scaling factor will be lower
\end{itemize}

\note[item]{}
\end{frame}
\begin{frame}
\frametitle{Unrelated Title}

\begin{center}
\includegraphics[width=0.9\textwidth,height=0.9\textheight,keepaspectratio]{/Users/I516998/Library/Application Support/Anki2/User 1/collection.media/paste-abb5c2cabb0693e54ed55771b4cb9a9b91f1c2b2.jpg}
\end{center}

\begin{itemize}
\item Goal: controll the amout of correction for the document length using an additional parameterNo real explanation why it is better, it's only proven empirically
\end{itemize}

\note[item]{}
\end{frame}
\begin{frame}
\frametitle{Unrelated Title}

\begin{center}
\includegraphics[width=0.9\textwidth,height=0.9\textheight,keepaspectratio]{/Users/I516998/Library/Application Support/Anki2/User 1/collection.media/paste-5acb016c74fbc00cf897e7be5e243a3e13a28505.jpg}
\end{center}


\note[item]{}
\end{frame}
\begin{frame}
\frametitle{Unrelated Title}

\begin{center}
\includegraphics[width=0.9\textwidth,height=0.9\textheight,keepaspectratio]{/Users/I516998/Library/Application Support/Anki2/User 1/collection.media/paste-9e88e62d35217150b436f2a6196aa7348a7ace43.jpg}
\end{center}

\begin{itemize}
\item Language models for IR are probalistic models and model the query generation processGoal: estimate the probability of a query q being sampled from the language model of the document d
\end{itemize}

\note[item]{}
\end{frame}
\begin{frame}
\frametitle{Unrelated Title}


\begin{itemize}
\item Stakeholder analysis, meetings, expert judgment
\end{itemize}

\note[item]{}
\end{frame}
\begin{frame}
\frametitle{Unrelated Title}


\begin{itemize}
\item Where we identify assumptions and constraints?
\end{itemize}

\note[item]{}
\end{frame}
\begin{frame}
\frametitle{Unrelated Title}


\begin{itemize}
\item Relevant:Org quality policyRequirements docsStakeholder register
\end{itemize}

\note[item]{}
\end{frame}
\begin{frame}
\frametitle{Unrelated Title}


\begin{itemize}
\item Project Management PlanProject Documents (lessons learned, project team assignments, resource breakdown structure, etc.)DeliverablesEEFsOPAs
\end{itemize}

\note[item]{}
\end{frame}
\begin{frame}
\frametitle{Unrelated Title}


\begin{itemize}
\item Ensure that this policy is included in Team Charter; i.e. could be considered "Ground Rules"
\end{itemize}

\note[item]{}
\end{frame}
\begin{frame}
\frametitle{Unrelated Title}


\begin{itemize}
\item C. Inspection
\end{itemize}

\note[item]{}
\end{frame}
\begin{frame}
\frametitle{Unrelated Title}


\begin{itemize}
\item EV/BAC
\end{itemize}

\note[item]{}
\end{frame}
\begin{frame}
\frametitle{Unrelated Title}


\begin{itemize}
\item Code of accounts
\end{itemize}

\note[item]{}
\end{frame}
\begin{frame}
\frametitle{Unrelated Title}


\begin{itemize}
\item Contract terms and conditions
\end{itemize}

\note[item]{}
\end{frame}
\begin{frame}
\frametitle{Unrelated Title}


\begin{itemize}
\item Project team and project manager
\end{itemize}

\note[item]{}
\end{frame}
\begin{frame}
\frametitle{Unrelated Title}


\begin{itemize}
\item Monitoring/evaluating process variations
\end{itemize}

\note[item]{}
\end{frame}
\begin{frame}
\frametitle{Unrelated Title}


\begin{itemize}
\item To negotiate a deal that both parties are comfortable with
\end{itemize}

\note[item]{}
\end{frame}
\begin{frame}
\frametitle{Unrelated Title}


\begin{itemize}
\item CPIF - Cost Plus Incentive Fee
\end{itemize}

\note[item]{}
\end{frame}
\begin{frame}
\frametitle{Unrelated Title}


\begin{itemize}
\item Summary of validation info for productSummary level description of the projectIncludes information on schedule and cost performance
\end{itemize}

\note[item]{}
\end{frame}
\begin{frame}
\frametitle{Unrelated Title}


\begin{itemize}
\item Review your Comms Management Plan to make sure an appropriate approach was defined based on stakeholder needs.
\end{itemize}

\note[item]{}
\end{frame}
\begin{frame}
\frametitle{Unrelated Title}


\begin{itemize}
\item Advertising
\end{itemize}

\note[item]{}
\end{frame}
\begin{frame}
\frametitle{Unrelated Title}


\begin{itemize}
\item Looking at the effectiveness of the risk management process
\end{itemize}

\note[item]{}
\end{frame}
\begin{frame}
\frametitle{Unrelated Title}


\begin{itemize}
\item Effectiveness of responses to overall project risk and identified individual project risks
\end{itemize}

\note[item]{}
\end{frame}
\begin{frame}
\frametitle{Unrelated Title}


\begin{itemize}
\item Fixed Cost
\end{itemize}

\note[item]{}
\end{frame}
\begin{frame}
\frametitle{Unrelated Title}


\begin{itemize}
\item Project Charter
\end{itemize}

\note[item]{}
\end{frame}
\begin{frame}
\frametitle{Unrelated Title}


\begin{itemize}
\item Prioritize the change yourself without your project team.
\end{itemize}

\note[item]{}
\end{frame}
\begin{frame}
\frametitle{Unrelated Title}


\begin{itemize}
\item 300
\end{itemize}

\note[item]{}
\end{frame}
\begin{frame}
\frametitle{Unrelated Title}


\begin{itemize}
\item Recognition and Rewards
\end{itemize}

\note[item]{}
\end{frame}
\begin{frame}
\frametitle{Unrelated Title}


\begin{itemize}
\item Does the WBS capture all of the sub-items required to be done for each level-one entry?
\end{itemize}

\note[item]{}
\end{frame}
\begin{frame}
\frametitle{Unrelated Title}


\begin{itemize}
\item Discuss the solutions and choose the best option
\end{itemize}

\note[item]{}
\end{frame}
\begin{frame}
\frametitle{Unrelated Title}


\begin{itemize}
\item Qualified sellers list
\end{itemize}

\note[item]{}
\end{frame}
\begin{frame}
\frametitle{Unrelated Title}


\begin{itemize}
\item -/+1 = 68%-/+2 = 95%-/+3 = 99.7%-/+6 = 99.9%
\end{itemize}

\note[item]{}
\end{frame}
\begin{frame}
\frametitle{Unrelated Title}


\begin{itemize}
\item + better behavioral effectiveness
\item + better subjective well-being
\item + better assimilation within social group
\item -> integrated regulation still NOT intrinsic motivation (not activity itself!!)
\item -> SDT - internalization model NOT a stage theory, where people must move through
\end{itemize}

\note[item]{}
\end{frame}
\begin{frame}
\frametitle{Unrelated Title}

\begin{center}
\includegraphics[width=0.9\textwidth,height=0.9\textheight,keepaspectratio]{/Users/I516998/Library/Application Support/Anki2/User 1/collection.media/Bildschirmfoto 2022-10-23 um 10.27.25.png}
\end{center}

\begin{itemize}
\item The higher the internalization degree, the higher the degree of desirable outcomes (performance, OCB) and lower the degree of undesirable outcomes (burnout, absenteeism).
\end{itemize}

\note[item]{}
\end{frame}
\begin{frame}
\frametitle{Unrelated Title}


\begin{itemize}
\item In SDT: something is a need if its satisfaction promotes psychological health and vice versa3 basic needs: Competency, Automony, RelatednessConditions supporting needs foster: high quality forms of motivation & performance/creativity
\end{itemize}

\note[item]{}
\end{frame}
\begin{frame}
\frametitle{Unrelated Title}


\begin{itemize}
\item Integration degree depends on: prior experiences & situational factors:Competence (higher competence -> enhance intrinsic motivation)Autonomy (experienced self-determination -> enhance intrinsic motivation)Relatedness (sense of security/relatedness -> enhance intrinsic motivation)-> Workenvironment should promote satisfactions of the three basic psychological needs
\end{itemize}

\note[item]{}
\end{frame}
\begin{frame}
\frametitle{Unrelated Title}


\begin{itemize}
\item - not clear what causes intrinsic motivation
\item - too much attention to individuals' needs can frustrate and dissatisfy
\item - not possible to respond to all followers' needs
\end{itemize}

\note[item]{}
\end{frame}
\begin{frame}
\frametitle{Unrelated Title}

\begin{center}
\includegraphics[width=0.9\textwidth,height=0.9\textheight,keepaspectratio]{/Users/I516998/Library/Application Support/Anki2/User 1/collection.media/paste-2bc2995ea13014dcce4358048c2c8306328065af.jpg}
\end{center}

\begin{itemize}
\item We assume word independenceWe estimate P(term) for every term in the text
\end{itemize}

\note[item]{}
\end{frame}
\begin{frame}
\frametitle{Unrelated Title}

\begin{center}
\includegraphics[width=0.9\textwidth,height=0.9\textheight,keepaspectratio]{/Users/I516998/Library/Application Support/Anki2/User 1/collection.media/paste-15b3d02fd6af808862d887dae40b719bff56e731.jpg}
\end{center}

\begin{itemize}
\item The probability of a word depends on the immediately preceding wordWe estimate P(term | previous term) for every pair of terms that appear on after another
\end{itemize}

\note[item]{}
\end{frame}
\begin{frame}
\frametitle{Unrelated Title}

\begin{center}
\includegraphics[width=0.9\textwidth,height=0.9\textheight,keepaspectratio]{/Users/I516998/Library/Application Support/Anki2/User 1/collection.media/paste-f9f14b736b06eed6ff21d5fae91c3447eba4fca3.jpg}
\end{center}


\note[item]{}
\end{frame}
\begin{frame}
\frametitle{Unrelated Title}

\begin{center}
\includegraphics[width=0.9\textwidth,height=0.9\textheight,keepaspectratio]{/Users/I516998/Library/Application Support/Anki2/User 1/collection.media/paste-a26681cd1804e166932aaa97d496f486615e5e66.jpg}
\end{center}

\begin{itemize}
\item Zero frequency problem:
\item If a term from the query does not occur in the document the probability for the term tis null and therefore the document is considered as irrelevant
\end{itemize}

\note[item]{}
\end{frame}
\begin{frame}
\frametitle{Unrelated Title}

\begin{center}
\includegraphics[width=0.9\textwidth,height=0.9\textheight,keepaspectratio]{/Users/I516998/Library/Application Support/Anki2/User 1/collection.media/paste-46a88e33f6ee922ea7e1ee64ffd96e2fff65ac73.jpg}
\end{center}

\begin{itemize}
\item |V| is the number of distinc terms in the documentShortcoming: all unseen words are equally likely
\end{itemize}

\note[item]{}
\end{frame}
\begin{frame}
\frametitle{Unrelated Title}

\begin{center}
\includegraphics[width=0.9\textwidth,height=0.9\textheight,keepaspectratio]{/Users/I516998/Library/Application Support/Anki2/User 1/collection.media/paste-3124edbf87c324de781625aff786c5ff46ed518c.jpg}
\end{center}

\begin{itemize}
\item Unseen words get some probability depending on its frequency in the whole collectionIf the term does not occur in the whole colection it can be removed from the query because it can't be used for any ranking
\end{itemize}

\note[item]{}
\end{frame}
\begin{frame}
\frametitle{Unrelated Title}

\begin{center}
\includegraphics[width=0.9\textwidth,height=0.9\textheight,keepaspectratio]{/Users/I516998/Library/Application Support/Anki2/User 1/collection.media/paste-6fbe7a96a4f9021f7ae4fed9b5c4974edd351e1e.jpg}
\end{center}

\begin{itemize}
\item Less frequency words in the document get more probability from the global component
\end{itemize}

\note[item]{}
\end{frame}
\begin{frame}
\frametitle{Unrelated Title}


\begin{itemize}
\item Plan Risk Responses
\end{itemize}

\note[item]{}
\end{frame}
\begin{frame}
\frametitle{Unrelated Title}


\begin{itemize}
\item Project Charter, Project Mngt Plan, Project Documents, EEFs, OPAs
\end{itemize}

\note[item]{}
\end{frame}
\begin{frame}
\frametitle{Unrelated Title}


\begin{itemize}
\item Update WBS Dictionary bc the planning package is converted to work packages
\end{itemize}

\note[item]{}
\end{frame}
\begin{frame}
\frametitle{Unrelated Title}


\begin{itemize}
\item It's not on the critical path; Critical path has ZERO FLOAT; For this question, it has a float of 10 days
\end{itemize}

\note[item]{}
\end{frame}
\begin{frame}
\frametitle{Unrelated Title}


\begin{itemize}
\item Tells us how efficient we need to be with cost.
\end{itemize}

\note[item]{}
\end{frame}
\begin{frame}
\frametitle{Unrelated Title}


\begin{itemize}
\item TCPI
\end{itemize}

\note[item]{}
\end{frame}
\begin{frame}
\frametitle{Unrelated Title}


\begin{itemize}
\item Procurement
\end{itemize}

\note[item]{}
\end{frame}
\begin{frame}
\frametitle{Unrelated Title}


\begin{itemize}
\item At the beginning of every phase of a multi-phase project
\end{itemize}

\note[item]{}
\end{frame}
\begin{frame}
\frametitle{Unrelated Title}


\begin{itemize}
\item Cost plus is good for evolving work
\end{itemize}

\note[item]{}
\end{frame}
\begin{frame}
\frametitle{Unrelated Title}


\begin{itemize}
\item Frequently, through built-in steps throughout the project. Recurring retrospectives regularly check in on the effectiveness of quality processes and suggest plans for improvement
\end{itemize}

\note[item]{}
\end{frame}
\begin{frame}
\frametitle{Unrelated Title}


\begin{itemize}
\item Inform them that this is known as a planning packageA planning package is a work breakdown structure component below the control account with known work content, but without detailed schedule activities. A work package is the work defined at the lowest level of the WBS for which cost and duration are estimated and managed.
\end{itemize}

\note[item]{}
\end{frame}
\begin{frame}
\frametitle{Unrelated Title}


\begin{itemize}
\item Product Owner - represents customers, accountable for the solutionTeam - creates the solutionFacilitator - Ensures tools and resources are availableSponsor - provides funding
\end{itemize}

\note[item]{}
\end{frame}
\begin{frame}
\frametitle{Unrelated Title}


\begin{itemize}
\item Variances atypical - this means use the EAC formula that is atypical
\end{itemize}

\note[item]{}
\end{frame}
\begin{frame}
\frametitle{Unrelated Title}


\begin{itemize}
\item Top-ranked results should be relevant
\item Plenty of relevant documents, but users only look at top-ranked results
\end{itemize}

\note[item]{}
\end{frame}
\begin{frame}
\frametitle{Unrelated Title}


\begin{itemize}
\item Adding new termsterms that are sementically related to the terms of the queryMaking the query vector more similar to vectors of relevant documentsindication of relevance needed
\end{itemize}

\note[item]{}
\end{frame}
\begin{frame}
\frametitle{Unrelated Title}


\begin{itemize}
\item Global method: query expansionAdding new terms that are related to original termsLocal MethodsRelevance feedback (input from user)Pseudo relevance feedback (automated assumption of relevance)Relevance model (automated assumption of relevance)
\end{itemize}

\note[item]{}
\end{frame}
\begin{frame}
\frametitle{Unrelated Title}


\begin{itemize}
\item User issues a initial querySearch engine ranks the resultsUser explicitly marks some of the results as relevant/ non-relevantThe search engines computes a new query based on the feedbackMultiple iterations possibleAssumption: it's difficult to formulate a good query when you don't know the collection well
\end{itemize}

\note[item]{}
\end{frame}
\begin{frame}
\frametitle{Unrelated Title}

\begin{center}
\includegraphics[width=0.9\textwidth,height=0.9\textheight,keepaspectratio]{/Users/I516998/Library/Application Support/Anki2/User 1/collection.media/paste-86a758c76ab137ed607a0eecb63d22454871394c.jpg}
\end{center}


\note[item]{}
\end{frame}
\begin{frame}
\frametitle{Unrelated Title}

\begin{center}
\includegraphics[width=0.9\textwidth,height=0.9\textheight,keepaspectratio]{/Users/I516998/Library/Application Support/Anki2/User 1/collection.media/paste-1161af8aa23125325143424b4b60268cf098d4d6.jpg}
\end{center}


\note[item]{}
\end{frame}
\begin{frame}
\frametitle{Unrelated Title}


\begin{itemize}
\item Used to incroporate relevance ffedback into the vector space modelIdea: rewrite the query so that it's vector isAs similar as possible to the vectors of relevant documentsAs dissimilar as possible to thevectors of non-relevant documents
\end{itemize}

\note[item]{}
\end{frame}
\begin{frame}
\frametitle{Unrelated Title}

\begin{center}
\includegraphics[width=0.9\textwidth,height=0.9\textheight,keepaspectratio]{/Users/I516998/Library/Application Support/Anki2/User 1/collection.media/paste-01072e2087553a15b00c8ee2d7247737ea100d12.jpg}
\includegraphics[width=0.9\textwidth,height=0.9\textheight,keepaspectratio]{/Users/I516998/Library/Application Support/Anki2/User 1/collection.media/paste-09aabf9511b666190ea2749cbd1cd5e81672ca72.jpg}
\end{center}

\begin{itemize}
\item Compute the centroid of a set of documents to aggregate relevant/ non-relevant documents
\end{itemize}

\note[item]{}
\end{frame}
\begin{frame}
\frametitle{Unrelated Title}

\begin{center}
\includegraphics[width=0.9\textwidth,height=0.9\textheight,keepaspectratio]{/Users/I516998/Library/Application Support/Anki2/User 1/collection.media/paste-4c6f60eb7da67fcd55c21c554cfaf3d87cba850d.jpg}
\end{center}

\begin{itemize}
\item Problem: Users provide relevance judgements only for a small number of documents --> leads to a wrong centroidRe-estimate the query by combining
\end{itemize}

\note[item]{}
\end{frame}
\begin{frame}
\frametitle{Unrelated Title}


\begin{itemize}
\item Non-relevant documents can be mutually differentIt's not likely that non-relevant documents form a clusterTherefore many systems allow oly positive feedback
\end{itemize}

\note[item]{}
\end{frame}
\begin{frame}
\frametitle{Unrelated Title}


\begin{itemize}
\item User has sufficient knowledge for the inital queryRelevance for the inital query behaves nicelyRelevant documents are tightly clustered around a single relevance vetorOr there are several clusters of relevant documents, but they have a significant overlapSimilarties between relevant and non-relevant documents are small
\end{itemize}

\note[item]{}
\end{frame}
\begin{frame}
\frametitle{Unrelated Title}


\begin{itemize}
\item User has insufficient knowledgeMultiple relevance vector clusters with little overlap
\end{itemize}

\note[item]{}
\end{frame}
\begin{frame}
\frametitle{Unrelated Title}


\begin{itemize}
\item EfficiencyExtending the query may lead to long queries that are inefficient processUser issuesUsers often reluctant to provide explicit feedbackHard for users to understand why a particular document was retrieved after applying relevance feedback
\end{itemize}

\note[item]{}
\end{frame}
\begin{frame}
\frametitle{Unrelated Title}

\begin{center}
\includegraphics[width=0.9\textwidth,height=0.9\textheight,keepaspectratio]{/Users/I516998/Library/Application Support/Anki2/User 1/collection.media/paste-58b2fad16984cbb655555b7bdb67ab7b681d8b98.jpg}
\end{center}

\begin{itemize}
\item Automated relevance feedbackAlgorithm:
\end{itemize}

\note[item]{}
\end{frame}
\begin{frame}
\frametitle{Unrelated Title}

\begin{center}
\includegraphics[width=0.9\textwidth,height=0.9\textheight,keepaspectratio]{/Users/I516998/Library/Application Support/Anki2/User 1/collection.media/paste-0a3e00fbffae5d69c943134e1b88a932bd57d346.jpg}
\end{center}


\note[item]{}
\end{frame}
\begin{frame}
\frametitle{Unrelated Title}


\begin{itemize}
\item Performance crucially depends on how good the initally top-ranked K document are (on average better th query expansion)Pitfall: query drift --> rewriting the query multiple times may lead to another focus of the query (no good default number for the number of iterations)
\end{itemize}

\note[item]{}
\end{frame}
\begin{frame}
\frametitle{Unrelated Title}


\begin{itemize}
\item Eplicit query expansionInitial query is executedExpanded versions are proposed to the userImplicit query expansionOriginal query is implicitly expanded and results are obtained by execution the expanded query
\end{itemize}

\note[item]{}
\end{frame}
\begin{frame}
\frametitle{Unrelated Title}


\begin{itemize}
\item Controlled vocabularyThere is a canonical term for each concept, all other terms for the same concept are replaced by the canonical termManual thesaurusSet of synonymous terms build by human annotatorsAutomatically generated thesaurusCo-occurence statistics on a large external corpus are used to automatically induce a large thesaurusQuery log miningQuery reformulations done by users are logged and used for reformulations
\end{itemize}

\note[item]{}
\end{frame}
\begin{frame}
\frametitle{Unrelated Title}


\begin{itemize}
\item Simple algorithm: For each term t from the initial query q_0 look up synonyms and related terms in the thesaurusOptionally assign new terms lower weightsGenerally increases recall, but may significantly decrease precisionManually producing a thasaurus is time-consuming and expensive, constant updates neededAutomated thesaurus generation:Detecting relatedness of terms in a alrge corporaDistributional hypothesis: words are similar if they occur in similar contextsRelated words: words that often co-appear are semantically related
\end{itemize}

\note[item]{}
\end{frame}
\begin{frame}
\frametitle{Unrelated Title}

\begin{center}
\includegraphics[width=0.9\textwidth,height=0.9\textheight,keepaspectratio]{/Users/I516998/Library/Application Support/Anki2/User 1/collection.media/paste-055ff4890cccbbfdc8b7ed41ba6101d54373d1c7.jpg}
\end{center}

\begin{itemize}
\item Simple way to compute a thesaurus base on term-term similarities
\end{itemize}

\note[item]{}
\end{frame}
\begin{frame}
\frametitle{Unrelated Title}


\begin{itemize}
\item Distributional vectors cannot distinguish similarity from relatednessDistributional vectors encode only similarity, not relatednessQuality of produced term associations often not good enough for IR
\end{itemize}

\note[item]{}
\end{frame}
\begin{frame}
\frametitle{Unrelated Title}


\begin{itemize}
\item User interaction is loggedAssumption: ranked documents clicked by th user are relevant for the queryClickstream mining: Global metric of document relevance on the web
\end{itemize}

\note[item]{}
\end{frame}
\begin{frame}
\frametitle{Unrelated Title}


\begin{itemize}
\item Higher positions receive more attention and more clicksTrue, even when the or is reversedUser behaviour is an intriguing source of relevance --> Cliks are informative but biasedOther problems: User behavior can be noisy, spam
\end{itemize}

\note[item]{}
\end{frame}
\begin{frame}
\frametitle{Unrelated Title}


\begin{itemize}
\item Partitioning: How to synchronize partitions?For scalability, throughput, latencyReplication: How to synchronize replicas?For robustness, throughputCaching: What happens to cache when underlying data changes?For throughput, latency
\end{itemize}

\note[item]{}
\end{frame}
\begin{frame}
\frametitle{Unrelated Title}


\begin{itemize}
\item Atomicity: all or nothing in terms of txn actionsConsistency: DB is consistent after every txnIsolation: concurrent execution of txns leads to same result as serial executionDurability: results of comitted txns are permanent
\end{itemize}

\note[item]{}
\end{frame}
\begin{frame}
\frametitle{Unrelated Title}


\begin{itemize}
\item Txn's allowed to touch only one siteSimple and efficient when possibleHandle txns that span multiple sitesChallenging, often unavoidable
\end{itemize}

\note[item]{}
\end{frame}
\begin{frame}
\frametitle{Unrelated Title}


\begin{itemize}
\item Transaction failuresleads to ABORTResaons: initiated by application, concurrency control, deadlock, constraint violationSite failuresMemory lost but secondary storage keptMedia failuresSecondary storage lost but handled locally (backups)Communication failuresErroneous messagesIncorrect message orderDuplicate messagesLost messagesNetwork partitions
\end{itemize}

\note[item]{}
\end{frame}
\begin{frame}
\frametitle{Unrelated Title}


\begin{itemize}
\item Dealing with failures
\end{itemize}

\note[item]{}
\end{frame}
\begin{frame}
\frametitle{Unrelated Title}


\begin{itemize}
\item There ist no protocol that uses a finite number of messages and solves the two-generals problem.
\item Uncertainty can be reduced sending multiple messagese.g. if a messenger gets through with probability p, how many do you need to send to get 99% confidence?
\end{itemize}

\note[item]{}
\end{frame}
\begin{frame}
\frametitle{Unrelated Title}


\begin{itemize}
\item Sites must agree but no simulatenous action
\end{itemize}

\note[item]{}
\end{frame}
\begin{frame}
\frametitle{Unrelated Title}

\begin{center}
\includegraphics[width=0.9\textwidth,height=0.9\textheight,keepaspectratio]{/Users/I516998/Library/Application Support/Anki2/User 1/collection.media/paste-55e26e8d6ccd69fc1ba8f4dbc2cc350cc3e090e9.jpg}
\end{center}

\begin{itemize}
\item Txns state their begin (BOT) and end (EOT) with COMMIT or ABORTAtomicity need to be ensuredCOMMIT -> all changes at all local site need to be comittedABORT -> changes at all local sites aborted
\end{itemize}

\note[item]{}
\end{frame}
\begin{frame}
\frametitle{Unrelated Title}

\begin{center}
\includegraphics[width=0.9\textwidth,height=0.9\textheight,keepaspectratio]{/Users/I516998/Library/Application Support/Anki2/User 1/collection.media/paste-c84f9075d7a88f655eec571bc1c0b2b00cafc105.jpg}
\end{center}

\begin{itemize}
\item One site acts as coordinatorOther site act as agentsPhase 1: each agent votes whether it agress to a commitPhase 2: coordinator makes global commit/abort decision based on votes and tells agents
\end{itemize}

\note[item]{}
\end{frame}
\begin{frame}
\frametitle{Unrelated Title}

\begin{center}
\includegraphics[width=0.9\textwidth,height=0.9\textheight,keepaspectratio]{/Users/I516998/Library/Application Support/Anki2/User 1/collection.media/paste-66a9f0fa174dbc179197b18643020b863f9bbc75.jpg}
\end{center}


\note[item]{}
\end{frame}
\begin{frame}
\frametitle{Unrelated Title}

\begin{center}
\includegraphics[width=0.9\textwidth,height=0.9\textheight,keepaspectratio]{/Users/I516998/Library/Application Support/Anki2/User 1/collection.media/paste-a22f59e9bd2b137d3b4d33d86e17ba1895d0cc77.jpg}
\end{center}


\note[item]{}
\end{frame}
\begin{frame}
\frametitle{Unrelated Title}


\begin{itemize}
\item Guarantees that it is able to COMMIT (red-logs written)Gurantess that is is able to ABORT (holds resources, undo-logs)can COMMIT or ABORT only if instructed by coordinator
\end{itemize}

\note[item]{}
\end{frame}
\begin{frame}
\frametitle{Unrelated Title}


\begin{itemize}
\item Crash of coordinatorCrash of siteLoss of messageLogs (WALs) are used to recover local and/or global decison
\end{itemize}

\note[item]{}
\end{frame}
\begin{frame}
\frametitle{Unrelated Title}


\begin{itemize}
\item Before replying to PREPARE: coordinator decides ABORT after timeoutAfter replying to PREPARE: coordinator may have reached global decision -> recoverRecover after restart:No ready found -> abort and tell coordinatorRead found but no commit/abort -> ask coordinator for global decision, then redo/undi txncommit/abort found -> redo/undo txn
\end{itemize}

\note[item]{}
\end{frame}
\begin{frame}
\frametitle{Unrelated Title}


\begin{itemize}
\item Before it logs commit/abort: send abort to all agents after restartAfter it logged commit/abort: send global decision again2PC is blocking: agents in ready state need to wait for coordinatorImprovement: agents use state of surviving agentswhen one in initial state: decide abort, log abort, tell otherswhen one in abort state: tell others: txn abortetwhen one in commit state: tell others: txn comittedwhen all in ready state: still need to wait
\end{itemize}

\note[item]{}
\end{frame}
\begin{frame}
\frametitle{Unrelated Title}


\begin{itemize}
\item PREPARE / VOTE message lost: coordinator votes abort after timeoutABORT / COMMIT message lost:agents in ready state keep waitingagents can periodically ask coordinator for decision
\end{itemize}

\note[item]{}
\end{frame}
\begin{frame}
\frametitle{Unrelated Title}


\begin{itemize}
\item Pessimistic concurrency control: two-phase lockingOptimistic concurrency control: work on a copy, validate at end
\end{itemize}

\note[item]{}
\end{frame}
\begin{frame}
\frametitle{Unrelated Title}

\begin{center}
\includegraphics[width=0.9\textwidth,height=0.9\textheight,keepaspectratio]{/Users/I516998/Library/Application Support/Anki2/User 1/collection.media/paste-f146fc7931b5a175d8cb625d31316de555db283a.jpg}
\end{center}

\begin{itemize}
\item Local serialization of a transaction at each site is not enough
\end{itemize}

\note[item]{}
\end{frame}
\begin{frame}
\frametitle{Unrelated Title}

\begin{center}
\includegraphics[width=0.9\textwidth,height=0.9\textheight,keepaspectratio]{/Users/I516998/Library/Application Support/Anki2/User 1/collection.media/paste-2286d7cd090575bd73a1b6683a6725935ff1abea.jpg}
\end{center}

\begin{itemize}
\item Central site runs global lock managerEach site runs a transaction managerTM aquires/releases locks from central lock manager
\end{itemize}

\note[item]{}
\end{frame}
\begin{frame}
\frametitle{Unrelated Title}


\begin{itemize}
\item SimpleBottleneck at central citeReduce reliabilityReduced autonomy of sites
\end{itemize}

\note[item]{}
\end{frame}
\begin{frame}
\frametitle{Unrelated Title}

\begin{center}
\includegraphics[width=0.9\textwidth,height=0.9\textheight,keepaspectratio]{/Users/I516998/Library/Application Support/Anki2/User 1/collection.media/paste-8fef99b8f101a0a5b778a64c5c88c6f65b520dac.jpg}
\end{center}

\begin{itemize}
\item Each site runs a lock manager which is responsible for local data itemsAdvantageHigher performance, reliability, autonomyLower communication costDisadvantageDeadlock handling more complex
\end{itemize}

\note[item]{}
\end{frame}
\begin{frame}
\frametitle{Unrelated Title}


\begin{itemize}
\item PreventionPreclaim all resourcesOr order resources and insist on lock request to be consistent with this orderDifficult since data access usually not know at start of txnAvoidanceUse timestamps to prioritze tr´xnsDetection and resolutionDetect and resolve deadlocks by selection of victim txns to abort
\end{itemize}

\note[item]{}
\end{frame}
\begin{frame}
\frametitle{Unrelated Title}


\begin{itemize}
\item Build local wait-for graphsCreate global wait-for graph out of local wait-for graphs
\end{itemize}

\note[item]{}
\end{frame}
\begin{frame}
\frametitle{Unrelated Title}


\begin{itemize}
\item One site runs a designated deadlock detectorEach site periodically sends its LWFGDeadlock dectors create GWFGGood: simple and natural for centralized 2PLBad: bad relibaility and communication overhead
\end{itemize}

\note[item]{}
\end{frame}
\begin{frame}
\frametitle{Unrelated Title}


\begin{itemize}
\item Each txn has globally unique timestampWait / Die (non-preemptive)Requesting older transactions wait for younger oneRequesting young transactions abort themselvesWound / Wait (preemtive)Requesting old transactions abort younger onesRequsting young transactions wait for older ones
\end{itemize}

\note[item]{}
\end{frame}
\begin{frame}
\frametitle{Unrelated Title}


\begin{itemize}
\item TL = part of "New Leadership" paradigm, more attention to charismatic and affective elementsTL = process that changes/transforms individuals & is concerned with emotions, values and long-term goalsTL = very popular concept in contemporary leadership research
\end{itemize}

\note[item]{}
\end{frame}
\begin{frame}
\frametitle{Unrelated Title}


\begin{itemize}
\item Burns (1978): transformational and transactional leadership 
\item –  Transactional: focus on exchange between leader and followers 
\item –  Transformational: connection between leader and followers that raises the level
\item of motivation and morality in both parties
\item • Example: Gandhi
\item – Pseudotransformational: leadership that is self-consumed,
\item exploitive, and power oriented (e.g., Hitler, Saddam Hussein) 
\end{itemize}

\note[item]{}
\end{frame}
\begin{frame}
\frametitle{Unrelated Title}

\begin{center}
\includegraphics[width=0.9\textwidth,height=0.9\textheight,keepaspectratio]{/Users/I516998/Library/Application Support/Anki2/User 1/collection.media/Bildschirmfoto 2022-11-01 um 15.49.11.png}
\includegraphics[width=0.9\textwidth,height=0.9\textheight,keepaspectratio]{/Users/I516998/Library/Application Support/Anki2/User 1/collection.media/Bildschirmfoto 2022-11-01 um 15.50.23.png}
\end{center}


\note[item]{}
\end{frame}
\begin{frame}
\frametitle{Unrelated Title}


\begin{itemize}
\item 


➢ Expanded and refined version of work done by Burns and House:



More attention to followers’ rather than leaders’ needsSuggested TL could apply to outcomes that were not positive

transactional and transformational leadership as a continuum (wrong assumption)➢ Extended House’s work by:



More attention to emotional elements & origins of charisma



Suggested charisma is a necessary but not sufficient condition for TL
Continuum: TL, Transactional, Laissez-faire (wrong assumption)




\end{itemize}

\note[item]{}
\end{frame}
\begin{frame}
\frametitle{Unrelated Title}

\begin{center}
\includegraphics[width=0.9\textwidth,height=0.9\textheight,keepaspectratio]{/Users/I516998/Library/Application Support/Anki2/User 1/collection.media/Bildschirmfoto 2022-11-01 um 15.55.44.png}
\end{center}


\note[item]{}
\end{frame}
\begin{frame}
\frametitle{Unrelated Title}


\begin{itemize}
\item Idealized Influence - vision and a sense of missionInspirational Motivation - symbols and emotional appeals 
\item feedback Laissez-Faire - hands-off, let-things-ride approach
\end{itemize}

\note[item]{}
\end{frame}
\begin{frame}
\frametitle{Unrelated Title}


\begin{itemize}
\item 


Conceptualization seriously flawedmissing definition independent of effectsmissing theory about dimensionsmissing configurational model for explanationCausal model underdeveloped - idiosyncratic micro-theoriesRatings of charismatic–transformational leadership highly reflective of
liking for the leader 


\end{itemize}

\note[item]{}
\end{frame}
\begin{frame}
\frametitle{Unrelated Title}


\begin{itemize}
\item Describes how leaders can initiate & carry out changes in organizationsTransformational Leaders empower followers & are effective at working with peopleTL does not provide a clear set of assumptionsUsing MLQ to detect strength and weaknesses
\end{itemize}

\note[item]{}
\end{frame}
\begin{frame}
\frametitle{Unrelated Title}


\begin{itemize}
\item 


Broadly researchedIntuitive appealProcess-focusedExpansive leadership viewEmphasizes followerEffectiveness

\end{itemize}

\note[item]{}
\end{frame}
\begin{frame}
\frametitle{Unrelated Title}


\begin{itemize}
\item 


- Lacks conceptual clarity, Dimensions are not clearly delimited, overlaps with other theories- Measurement questioned (MLQ)- Leadership more as a personality trait or predisposition than a behavior that
can be taught- Potential to be abused Application: provides general way of thinking about leadership that emphasizes ideals, inspiration,
innovations & visions


\end{itemize}

\note[item]{}
\end{frame}
\begin{frame}
\frametitle{Unrelated Title}


\begin{itemize}
\item Robustness (availability)Goal is robustness against failuresIf one site fails, this should not affect transactionsPerformanceThe site used can be choosedIncreases throughputDecreases latency
\end{itemize}

\note[item]{}
\end{frame}
\begin{frame}
\frametitle{Unrelated Title}


\begin{itemize}
\item Synchronization costs:
\item Multiple replica may actually decrease availabilityMultiple replica may actually increase latency
\end{itemize}

\note[item]{}
\end{frame}
\begin{frame}
\frametitle{Unrelated Title}


\begin{itemize}
\item All replicas of each item have identical values
\end{itemize}

\note[item]{}
\end{frame}
\begin{frame}
\frametitle{Unrelated Title}


\begin{itemize}
\item System behaves as if there were only one copyOne-copy equivalence
\end{itemize}

\note[item]{}
\end{frame}
\begin{frame}
\frametitle{Unrelated Title}


\begin{itemize}
\item Read(x)Obtain shared lock on each copyRead any copyWrite(x)Obtain exclusive lock on each copyWrite all copiesAt the end of txn, release all locks
\end{itemize}

\note[item]{}
\end{frame}
\begin{frame}
\frametitle{Unrelated Title}


\begin{itemize}
\item Not robust: x cannot be accessed when one copy downHigh latency: update speed restricted by slowest machine
\end{itemize}

\note[item]{}
\end{frame}
\begin{frame}
\frametitle{Unrelated Title}


\begin{itemize}
\item Similar to RAWA but:
\item Readers lock and read only one copyAdvantage: Favors reads (robustness, throughput)Disadvantage: High latency like for RAWA
\end{itemize}

\note[item]{}
\end{frame}
\begin{frame}
\frametitle{Unrelated Title}


\begin{itemize}
\item RAWA: Read-lock all, write allROWA: Read-one, write allRPWP (deferred): Read-primary, wirte-primaryRPWP (synchronous): Read-primary, wirte-primary
\end{itemize}

\note[item]{}
\end{frame}
\begin{frame}
\frametitle{Unrelated Title}


\begin{itemize}
\item Each data items has a primary site Readers lock and read primary copyWriters lock and write primary copyPrimary copy forwards updates to replicasEither synchronous (distributed commit)Or deffered (local commit and then send refresh txns)
\end{itemize}

\note[item]{}
\end{frame}
\begin{frame}
\frametitle{Unrelated Title}


\begin{itemize}
\item With synchronous updatesMaintains up-to-date backupImportant: distributed commit at end of each txnWith deffered updatesMaintains slightly outdated backupImportant: automatic updates need to be applied according to globally consistent order of txnsCan allow reads on other replicasSerializability is given but may read stale data
\end{itemize}

\note[item]{}
\end{frame}
\begin{frame}
\frametitle{Unrelated Title}

\begin{center}
\includegraphics[width=0.9\textwidth,height=0.9\textheight,keepaspectratio]{/Users/I516998/Library/Application Support/Anki2/User 1/collection.media/paste-ebdda5801120697913e9b9439f825b3dc1deaff8.jpg}
\end{center}


\note[item]{}
\end{frame}
\begin{frame}
\frametitle{Unrelated Title}

\begin{center}
\includegraphics[width=0.9\textwidth,height=0.9\textheight,keepaspectratio]{/Users/I516998/Library/Application Support/Anki2/User 1/collection.media/paste-6ddbfee74454ff41f58a57f88b9973fa05501ece.jpg}
\includegraphics[width=0.9\textwidth,height=0.9\textheight,keepaspectratio]{/Users/I516998/Library/Application Support/Anki2/User 1/collection.media/paste-6f263c49e8f42bf60a774e8a6c9a049479616bee.jpg}
\end{center}


\note[item]{}
\end{frame}
\begin{frame}
\frametitle{Unrelated Title}


\begin{itemize}
\item Advantages:
\item Flexibility in choice of weightsTolerate site failuresSimple recovery after site failureDisadvantages:Readers need to access multiple copiesMultiple copies necessary to tolerate site failuresVersion trackingCopies must be known in advancw
\end{itemize}

\note[item]{}
\end{frame}
\begin{frame}
\frametitle{Unrelated Title}


\begin{itemize}
\item 


Intrapersonal definition





AL is based on the leaders‘ self-concepts and how these
self-concepts are related to their actions.Developmental definitionAL is something that can be nurtured in a leader rather than a fixed trait, can be triggered by life event 



Interpersonal definitionAL is relational, created by leaders and followers together






\end{itemize}

\note[item]{}
\end{frame}
\begin{frame}
\frametitle{Unrelated Title}

\begin{center}
\includegraphics[width=0.9\textwidth,height=0.9\textheight,keepaspectratio]{/Users/I516998/Library/Application Support/Anki2/User 1/collection.media/Bildschirmfoto 2022-11-11 um 16.13.25.png}
\end{center}


\note[item]{}
\end{frame}
\begin{frame}
\frametitle{Unrelated Title}


\begin{itemize}
\item Self-awareness (personal insights of the leader) Internalized moral perspective (internal moral standards rather than outside pressure) Balanced processing (ability to analyze information objectively and explore other people‘s opinion) Relational transparency (share core feeling and motives, be honest) 
\end{itemize}

\note[item]{}
\end{frame}
\begin{frame}
\frametitle{Unrelated Title}


\begin{itemize}
\item 


Positive psychological capacities (Confidence, Resilience)Moral reasoning (Capacity to make ethical decisions)Critical life events (By understanding own life events, leaders become more authentic)



\end{itemize}

\note[item]{}
\end{frame}
\begin{frame}
\frametitle{Unrelated Title}


\begin{itemize}
\item + explicit moral dimension
\item + emphasizes need for trustworthy leadership
\item + provides guidelines to help become authentic leader
\item - More empirical evidence necessary
\item - Role of positive psychological capacities needs further clarification
\item - Underlying processes between moral and AL not explained
\end{itemize}

\note[item]{}
\end{frame}
\begin{frame}
\frametitle{Unrelated Title}


\begin{itemize}
\item Servant leadership as a paradox: How can a person be leader and servant?Servant leaders:Put followers first, focus on their developmentDemonstrate moral behaviorServe the greater good, have social responsibility
\end{itemize}

\note[item]{}
\end{frame}
\begin{frame}
\frametitle{Unrelated Title}


\begin{itemize}
\item 

AccountabilityCourage
EmpowermentForgivenessStanding back



\end{itemize}

\note[item]{}
\end{frame}
\begin{frame}
\frametitle{Unrelated Title}

\begin{center}
\includegraphics[width=0.9\textwidth,height=0.9\textheight,keepaspectratio]{/Users/I516998/Library/Application Support/Anki2/User 1/collection.media/Bildschirmfoto 2022-11-11 um 16.39.15.png}
\end{center}


\note[item]{}
\end{frame}
\begin{frame}
\frametitle{Unrelated Title}


\begin{itemize}
\item 


+ Adds altruism to the leadership process+  Counterintuitive and thought-provoking+ Servant Leadership Questionaire (SLQ) validated measure

−  Paradox that leaders should follow is not fully explained −  No common definition or framework−  Often prescriptive overtone




\end{itemize}

\note[item]{}
\end{frame}
\begin{frame}
\frametitle{Unrelated Title}


\begin{itemize}
\item Level 1: Preconventional morality - based on avoiding
punishment, and rewards Level 2: Conventional morality - based on society’s views and
expectationsLevel 3: Postconventional morality - based on coscience and creating a just society 
\end{itemize}

\note[item]{}
\end{frame}
\begin{frame}
\frametitle{Unrelated Title}


\begin{itemize}
\item (2 per Level)
\item Obedience and PunishmentIndividualism and ExchangeInterpersonal Accord and ComformityMaintaining the Social OrderSocial Contract and Individual RightsUniversal Principles
\end{itemize}

\note[item]{}
\end{frame}
\begin{frame}
\frametitle{Unrelated Title}

\begin{center}
\includegraphics[width=0.9\textwidth,height=0.9\textheight,keepaspectratio]{/Users/I516998/Library/Application Support/Anki2/User 1/collection.media/Bildschirmfoto 2022-11-11 um 16.53.42.png}
\end{center}

\begin{itemize}
\item Ethical Egoism: create the greatest good for
\item own person
\item Utilitarianism: create the greatest good for the
\item greatest number
\item purpose is to create good for others
\end{itemize}

\note[item]{}
\end{frame}
\begin{frame}
\frametitle{Unrelated Title}

\begin{center}
\includegraphics[width=0.9\textwidth,height=0.9\textheight,keepaspectratio]{/Users/I516998/Library/Application Support/Anki2/User 1/collection.media/Bildschirmfoto 2022-11-11 um 16.59.34.png}
\end{center}


\note[item]{}
\end{frame}
\begin{frame}
\frametitle{Unrelated Title}


\begin{itemize}
\item Destructive Leaders: Personalized Power, Narcissism, Ideology of hateSusceptible Followers: Conformers (Low maturity) & Colluders (Bad values)Conducive Environments: Instability, Perceived Threat
\end{itemize}

\note[item]{}
\end{frame}
\begin{frame}
\frametitle{Unrelated Title}


\begin{itemize}
\item Results not surprisingDestructive -> neg. correlation with outcomes (job satisfaction, commitment, well being)Constructive (different concepts) -> pos correlation
\end{itemize}

\note[item]{}
\end{frame}
\begin{frame}
\frametitle{Unrelated Title}


\begin{itemize}
\item Baseline: what Transformational Leadership alone explains of outcome variables (R^2 from Regression Analysis, with multiple predictors)Adding AL, SL and EL always explains more of the outcome variables, higher R^2 ValuesLittle improvements, but not game changer
\end{itemize}

\note[item]{}
\end{frame}
\begin{frame}
\frametitle{Unrelated Title}


\begin{itemize}
\item Gain scailability,Availabilityand flexibility
\end{itemize}

\note[item]{}
\end{frame}
\begin{frame}
\frametitle{Unrelated Title}


\begin{itemize}
\item Do not support concurrent transactionsSupport only certain kinds of transactionsUse a weaker notion of consistency (relax ACID)Give up the relational model for something more flexibleDo not support joinsEach option may be ok for some applications but not for others
\end{itemize}

\note[item]{}
\end{frame}
\begin{frame}
\frametitle{Unrelated Title}


\begin{itemize}
\item Key-value-storesWide-column storesDocument storesGraph databases
\end{itemize}

\note[item]{}
\end{frame}
\begin{frame}
\frametitle{Unrelated Title}


\begin{itemize}
\item Not only SQL
\end{itemize}

\note[item]{}
\end{frame}
\begin{frame}
\frametitle{Unrelated Title}


\begin{itemize}
\item Non-relational data modelDesigned for scale-out architectures (one logical DB, many machines)No full SQL supportNo full ACID supportOften open-source
\end{itemize}

\note[item]{}
\end{frame}
\begin{frame}
\frametitle{Unrelated Title}


\begin{itemize}
\item Arbitrary values or dasic data typesSimple CRUD operationsAdditional operations possible, e.g. query a range of keysExample use casesStoring user profiles/preferencesAs a cacheParameter servers for training large machine learning models
\end{itemize}

\note[item]{}
\end{frame}
\begin{frame}
\frametitle{Unrelated Title}


\begin{itemize}
\item Store JSON or XML --> flexible data modelSometimes used to aggregate relevant data about a single objectCRUDExample use cases:Event loggingConent management systems (blogs)Web analyticsProduct data management (inventory)
\end{itemize}

\note[item]{}
\end{frame}
\begin{frame}
\frametitle{Unrelated Title}


\begin{itemize}
\item Logically like relational databasesOrganized by columnsDynamic schema --> many sparse columnsNo full SQL and transaction supportHBase: focus on features that can scale; no joins, complex queries, triggers or viewsExample use cases:Same as KV or document storesIndexesCounters and aggregatesExpiring data
\end{itemize}

\note[item]{}
\end{frame}
\begin{frame}
\frametitle{Unrelated Title}

\begin{center}
\includegraphics[width=0.9\textwidth,height=0.9\textheight,keepaspectratio]{/Users/I516998/Library/Application Support/Anki2/User 1/collection.media/paste-3d89a214e31c99ea995bb6b45ad087bf16793685.jpg}
\end{center}

\begin{itemize}
\item Data is stored in tablesRows are identified by a keyRequires a good key designRows have columnsCan be added dynamicallyCell = value of a column for a rowColumns are grouped into column familiesAsoociated with a table and specified upfront
\end{itemize}

\note[item]{}
\end{frame}
\begin{frame}
\frametitle{Unrelated Title}


\begin{itemize}
\item Property graph data modelNamed vertices and named edgesEach vertex and edge can have propertiesRDF data modelSubject-predicate-object propertiesExample use cases:Knowledge representation (question answering)Social data
\end{itemize}

\note[item]{}
\end{frame}
\begin{frame}
\frametitle{Unrelated Title}


\begin{itemize}
\item Limited query capabilitiesLimited consistency and transaction supportNeed to think about how to use system best
\end{itemize}

\note[item]{}
\end{frame}
\begin{frame}
\frametitle{Unrelated Title}


\begin{itemize}
\item - Intercorrelations of Charisma, Individual consideration, Intellectual Stimulation, Contingent Reward, and Management by Exception
\item - high intercorrelations between 3 I's (0.68 - 0.85) -> not independant
\item - also high between I's and Contingent Rewards -> also go in line with 4 I's
\item - low correlations between MBE and others
\end{itemize}

\note[item]{}
\end{frame}
\begin{frame}
\frametitle{Unrelated Title}


\begin{itemize}
\item TL: highest mean correlation with 0.38Transactual: Contingent Reward: second with 0.32then MBE - activenegative for MBE passive and laissez-faire
\end{itemize}

\note[item]{}
\end{frame}
\begin{frame}
\frametitle{Unrelated Title}


\begin{itemize}
\item There is a lexical gap between query and relevant documents if there is no term overlap
\end{itemize}

\note[item]{}
\end{frame}
\begin{frame}
\frametitle{Unrelated Title}


\begin{itemize}
\item They present documents and queries semantically so that semantically similar words and phrases have similar representationsThey produce numeric vectors to present the meaning of portions of text
\end{itemize}

\note[item]{}
\end{frame}
\begin{frame}
\frametitle{Unrelated Title}


\begin{itemize}
\item Latent semantic analysis (LSA)Docomposiotion of word-document co-occurence matrixProbabilistic topic modellingGenerative model assuming that documents and words are probabilistic distributions over a set of latent toipicsText embeddingsDense vector representation that is learned  
\end{itemize}

\note[item]{}
\end{frame}
\begin{frame}
\frametitle{Unrelated Title}


\begin{itemize}
\item Query terms do not need to be exactly matchedRecall is as important as precisionThere are many relevant documents with lexical gap to the query
\end{itemize}

\note[item]{}
\end{frame}
\begin{frame}
\frametitle{Unrelated Title}

\begin{center}
\includegraphics[width=0.9\textwidth,height=0.9\textheight,keepaspectratio]{/Users/I516998/Library/Application Support/Anki2/User 1/collection.media/paste-ea7ab003863e1d604178456b769bfa51451b255e.jpg}
\end{center}

\begin{itemize}
\item Word-document occurence matrix A of dimension M x NM = #wordsN = #documentsRows correspond to words, Columns correspond to documentsElements can be raw frequencies or tf-idfRows of A are distributional vectors of wordsColumns of A are distributional vectors of documentsBoth distributions are sparse!
\end{itemize}

\note[item]{}
\end{frame}
\begin{frame}
\frametitle{Unrelated Title}

\begin{center}
\includegraphics[width=0.9\textwidth,height=0.9\textheight,keepaspectratio]{/Users/I516998/Library/Application Support/Anki2/User 1/collection.media/paste-08d9db500c7aa713758612b246e55a59ea6f8d53.jpg}
\includegraphics[width=0.9\textwidth,height=0.9\textheight,keepaspectratio]{/Users/I516998/Library/Application Support/Anki2/User 1/collection.media/paste-a2f77c7d6cd2170ea682d1eb1b7f532bedce0963.jpg}
\includegraphics[width=0.9\textwidth,height=0.9\textheight,keepaspectratio]{/Users/I516998/Library/Application Support/Anki2/User 1/collection.media/paste-2e6f41ef708b5b0d7f714f86e8cfd402173018bb.jpg}
\includegraphics[width=0.9\textwidth,height=0.9\textheight,keepaspectratio]{/Users/I516998/Library/Application Support/Anki2/User 1/collection.media/paste-e5e766fd72a82058435694d24f7ef4669ec556d8.jpg}
\end{center}

\begin{itemize}
\item SVD = matrix factorizationDecomposition into factor matrices which we use to obtain dense vector representations of words and documentsComparing dense vectors better captures the meaning of words and documents
\end{itemize}

\note[item]{}
\end{frame}
\begin{frame}
\frametitle{Unrelated Title}

\begin{center}
\includegraphics[width=0.9\textwidth,height=0.9\textheight,keepaspectratio]{/Users/I516998/Library/Application Support/Anki2/User 1/collection.media/paste-e2f40208f6374808149330f2123fc8812e619658.jpg}
\end{center}

\begin{itemize}
\item By reducing the rank of the matrix with singular values only the K most prominent topics are leftOther topics are assumed to be noise
\end{itemize}

\note[item]{}
\end{frame}
\begin{frame}
\frametitle{Unrelated Title}

\begin{center}
\includegraphics[width=0.9\textwidth,height=0.9\textheight,keepaspectratio]{/Users/I516998/Library/Application Support/Anki2/User 1/collection.media/paste-7c7a07a6bd8679964ec90101abe23e4eb58aa21b.jpg}
\includegraphics[width=0.9\textwidth,height=0.9\textheight,keepaspectratio]{/Users/I516998/Library/Application Support/Anki2/User 1/collection.media/paste-c731362eff113f4527f7e74faa124cff9d5653a8.jpg}
\end{center}


\note[item]{}
\end{frame}
\begin{frame}
\frametitle{Unrelated Title}


\begin{itemize}
\item Latent topics are numerically justified - SVD ensures the best lower-dimensional approximationLSI latent topics are often not interpretable by humans
\end{itemize}

\note[item]{}
\end{frame}
\begin{frame}
\frametitle{Unrelated Title}

\begin{center}
\includegraphics[width=0.9\textwidth,height=0.9\textheight,keepaspectratio]{/Users/I516998/Library/Application Support/Anki2/User 1/collection.media/paste-2b554f73704adab4e640ef7cb9b6238f79fa3777.jpg}
\end{center}


\note[item]{}
\end{frame}
\begin{frame}
\frametitle{Unrelated Title}

\begin{center}
\includegraphics[width=0.9\textwidth,height=0.9\textheight,keepaspectratio]{/Users/I516998/Library/Application Support/Anki2/User 1/collection.media/paste-2b81154039a0fb91032c8c9296bb93aadb971582.jpg}
\end{center}


\note[item]{}
\end{frame}
\begin{frame}
\frametitle{Unrelated Title}

\begin{center}
\includegraphics[width=0.9\textwidth,height=0.9\textheight,keepaspectratio]{/Users/I516998/Library/Application Support/Anki2/User 1/collection.media/paste-59196fa0f8fd51fc4f848f601260ed2f0dc631c2.jpg}
\includegraphics[width=0.9\textwidth,height=0.9\textheight,keepaspectratio]{/Users/I516998/Library/Application Support/Anki2/User 1/collection.media/paste-51463b14d4426e06e1b785cfdc5603ad3ad38219.jpg}
\includegraphics[width=0.9\textwidth,height=0.9\textheight,keepaspectratio]{/Users/I516998/Library/Application Support/Anki2/User 1/collection.media/paste-87e2b125b78042a6d9a385cd471b52e9eded4dbb.jpg}
\end{center}

\begin{itemize}
\item Latent model that assumes that the collection of documents was generated by a particular dirichlet distributionCollection of M documents, voacabulary of N terms, K latent topicsEach K topic is a concrete multinomial distribution over N terms (N-1 parameters)For each position in each of the M documents we obtain the observed word by:Randomly selecting one of the topics (from dirichlet distribution)Randomly select the term from the multinomial distribution of the topic that was randomly selectedLDA generative View:
\end{itemize}

\note[item]{}
\end{frame}
\begin{frame}
\frametitle{Unrelated Title}


\begin{itemize}
\item Dense semantic vector representations of wordsGrounded on predicting representation vectors of words based on the context (surrounding words)Sparse representation: one-hot encodingDense representation: real-valued vector of dimension orders of magnitude lower than the size of vocabularyCapture meaning of words
\end{itemize}

\note[item]{}
\end{frame}
\begin{frame}
\frametitle{Unrelated Title}

\begin{center}
\includegraphics[width=0.9\textwidth,height=0.9\textheight,keepaspectratio]{/Users/I516998/Library/Application Support/Anki2/User 1/collection.media/paste-94668ae17daa403d24c2c206994db5cb450877f3.jpg}
\includegraphics[width=0.9\textwidth,height=0.9\textheight,keepaspectratio]{/Users/I516998/Library/Application Support/Anki2/User 1/collection.media/paste-88de09976c66632d28744a4907feffbbf4f630e3.jpg}
\includegraphics[width=0.9\textwidth,height=0.9\textheight,keepaspectratio]{/Users/I516998/Library/Application Support/Anki2/User 1/collection.media/paste-0975df2af8bc0bde449a51ec58c093e620264da8.jpg}
\includegraphics[width=0.9\textwidth,height=0.9\textheight,keepaspectratio]{/Users/I516998/Library/Application Support/Anki2/User 1/collection.media/paste-4ecc482feb24e21c38987628664e35786283a47e.jpg}
\includegraphics[width=0.9\textwidth,height=0.9\textheight,keepaspectratio]{/Users/I516998/Library/Application Support/Anki2/User 1/collection.media/paste-8d954567473d27f324801a21526134154a275392.jpg}
\end{center}


\note[item]{}
\end{frame}
\begin{frame}
\frametitle{Unrelated Title}

\begin{center}
\includegraphics[width=0.9\textwidth,height=0.9\textheight,keepaspectratio]{/Users/I516998/Library/Application Support/Anki2/User 1/collection.media/paste-3715b9be46160f2ec3a1f4a4de0f770947e88f2f.jpg}
\end{center}


\note[item]{}
\end{frame}
\begin{frame}
\frametitle{Unrelated Title}


\begin{itemize}
\item Gender & leadership ignored until 1970sPress reported differences: women inferior to men, lacked necessary skills/traits (1970s)Current questions: differences regarding style/effectiveness? Why underrepresented?
\end{itemize}

\note[item]{}
\end{frame}
\begin{frame}
\frametitle{Unrelated Title}


\begin{itemize}
\item Woman concentrated in lower-level and lower-authority leadership vs. menAlso: ethnic and racial minoritiesMotives for removing barriers: fulfill equal opportunity, find most talented regardless of gender
\end{itemize}

\note[item]{}
\end{frame}
\begin{frame}
\frametitle{Unrelated Title}


\begin{itemize}
\item Women NOT found to lead in more interpersonally / less task-oriented manner (considering effectsize d, pos: stereotypical)Only gender-difference: women more participative/democratic, d=0.34Additional meta-analysis until 2000: similar results
\end{itemize}

\note[item]{}
\end{frame}
\begin{frame}
\frametitle{Unrelated Title}


\begin{itemize}
\item Women tend to show more TL + Transactional (contingent) -> linked with leadership effectivenessbut all small effect sizesMen: more MBE + LFComparison may be inaccurate due to smaller women selection in high positions
\end{itemize}

\note[item]{}
\end{frame}
\begin{frame}
\frametitle{Unrelated Title}


\begin{itemize}
\item Women were devalued when they worked in male-dominated environmentWomen were devalued when they use directive/autocratic styleWomen/Men were evaluated favorably when they use democratic style
\end{itemize}

\note[item]{}
\end{frame}
\begin{frame}
\frametitle{Unrelated Title}

\begin{center}
\includegraphics[width=0.9\textwidth,height=0.9\textheight,keepaspectratio]{/Users/I516998/Library/Application Support/Anki2/User 1/collection.media/Bildschirmfoto 2022-11-23 um 13.02.00.png}
\end{center}


\note[item]{}
\end{frame}
\begin{frame}
\frametitle{Unrelated Title}


\begin{itemize}
\item Pipeline Theory, not long enough in managerial positionsSelf-selection by choosing mommy track (empirically not supported)
\end{itemize}

\note[item]{}
\end{frame}
\begin{frame}
\frametitle{Unrelated Title}


\begin{itemize}
\item Same level of identification / commitment to paid employment rolesWomen less likely to emerge as group leaders, less likely to negotiate and ask for what they wantMen more assertiveness, less integrityResearch: androgynous mixture of traits
\end{itemize}

\note[item]{}
\end{frame}
\begin{frame}
\frametitle{Unrelated Title}


\begin{itemize}
\item Men: confidence, independence, rationality, decisivenessWomen: concern, sensitivity, warmth, helpful
\end{itemize}

\note[item]{}
\end{frame}
\begin{frame}
\frametitle{Unrelated Title}


\begin{itemize}
\item Pressure of tokenism (not just because of gender)Assimilation to stereotype (less likely to desire)Counter the stereotype (more desire to assume leadership position)
\end{itemize}

\note[item]{}
\end{frame}
\begin{frame}
\frametitle{Unrelated Title}


\begin{itemize}
\item Number of women in leadership is increasing, but still smaller Gender Gap not only at work but in society in generalmore androgynous conception of leadership helps choose fitting leadership stylesresearch took place in western cultures, differences
\end{itemize}

\note[item]{}
\end{frame}
\begin{frame}
\frametitle{Unrelated Title}


\begin{itemize}
\item Globalization -> multinational organizations, manage culturally diverse employeesCulture = Software of the mind (learned beliefs, values, rules, traditions), make groups unique
\end{itemize}

\note[item]{}
\end{frame}
\begin{frame}
\frametitle{Unrelated Title}


\begin{itemize}
\item Extends Hofstede's work on cultural dimensionsGoal: develop theory to describe, understand and predict cultural impact on leadership/organizational processesDeveloped a classification of 9 cultural dimensions
\end{itemize}

\note[item]{}
\end{frame}
\begin{frame}
\frametitle{Unrelated Title}


\begin{itemize}
\item Uncertainty avoidance (relies on established social norms)Power Distance (power should be shared unequally)Institutional Collectivism (encourage insitutional collective action)In-Group Collectivism (express pride, loyalty and cohesiveness)Gender Egalitarianism (promote gender equality)Assertiveness (determined, confrontational, aggressive)Future Orientation (planning, investing in the future)Performance Orientation (reward excellence)Humane Orientation (reward altruistic)
\end{itemize}

\note[item]{}
\end{frame}
\begin{frame}
\frametitle{Unrelated Title}


\begin{itemize}
\item Charismatic/value-based L - inspire/motivate/expect high performanceTeam-oriented L - team-building/common purposeParticipative L - involve others in decisionsHumane-oriented L - supportive, considerateAutonomous L - individualisticSelf-protective L - ensure safety/security
\end{itemize}

\note[item]{}
\end{frame}
\begin{frame}
\frametitle{Unrelated Title}

\begin{center}
\includegraphics[width=0.9\textwidth,height=0.9\textheight,keepaspectratio]{/Users/I516998/Library/Application Support/Anki2/User 1/collection.media/Bildschirmfoto 2022-11-23 um 19.17.48.png}
\includegraphics[width=0.9\textwidth,height=0.9\textheight,keepaspectratio]{/Users/I516998/Library/Application Support/Anki2/User 1/collection.media/Bildschirmfoto 2022-11-23 um 19.19.26.png}
\end{center}


\note[item]{}
\end{frame}
\begin{frame}
\frametitle{Unrelated Title}


\begin{itemize}
\item process of executing a software
system to determine whether,it matches its specification andexecutes in its intended environment

\end{itemize}

\note[item]{}
\end{frame}
\begin{frame}
\frametitle{Unrelated Title}


\begin{itemize}
\item acceptance testingreliability testingusability testingdefect testingcompatibility testingperformance testingstress testing

\end{itemize}

\note[item]{}
\end{frame}
\begin{frame}
\frametitle{Unrelated Title}


\begin{itemize}
\item component (unit) testingtesting
of individual program componentsresponsibility
of the component developmentintegration testingtesting
of groups of components integrated to create a (sub)systemresponsibility
of an independent testing teamusually
includes regression testing: re-execute testssystem testingtesting
the entire software in its expected environment
\end{itemize}

\note[item]{}
\end{frame}
\begin{frame}
\frametitle{Unrelated Title}


\begin{itemize}
\item reachability: location of fault must
be reached during executinginfections: a software error must
occurpropagation: error must propagate to
cause a failure (i.e. false output)

\end{itemize}

\note[item]{}
\end{frame}
\begin{frame}
\frametitle{Unrelated Title}


\begin{itemize}
\item Software Controllabilityhow
easy it is to provide a program with needed inputseasy:
keyboard inputsharder:
hardware sensors or distributed software inputsSoftware Observabilityhow
easy it is to observe the behavior of a program (outputs, environment,…)hard:
hardware devices, databases or remote files
= > data abstraction reduces controllability and observabilitysoftware understandability:
documentationsoftware traceability: logssoftware test support capability:
common test management (i.e. Junit)
\end{itemize}

\note[item]{}
\end{frame}
\begin{frame}
\frametitle{Unrelated Title}


\begin{itemize}
\item used informallysometimes it means fault, sometimes failure, sometimes error
-> should be avoided
\end{itemize}

\note[item]{}
\end{frame}
\begin{frame}
\frametitle{Unrelated Title}


\begin{itemize}
\item even small programs have too many
inputs to test alltry to find the fewest inputs that
will find the most problemscoverage criteria give systematic ways to
search the input spaceadvantages:thorough
search of input spaceminimize
overlap in testsmaximize
“bang for the buck”tells
testers when testing is finishedcan
be supported by toolsin practice, testers need to find
the best “ROI” in terms of effort spent and faults discovered

\end{itemize}

\note[item]{}
\end{frame}
\begin{frame}
\frametitle{Unrelated Title}


\begin{itemize}
\item GraphsLogical ExpressionsInput Domain CharacterizationSyntactic Structures

\end{itemize}

\note[item]{}
\end{frame}
\begin{frame}
\frametitle{Unrelated Title}


\begin{itemize}
\item most commonly used structure for testingGraphs can come from many sourcesinteraction diagramcontrol flow graphsdesign structurestate diagramsetc.tests usually intend to “cover” the graph in some way 
\end{itemize}

\note[item]{}
\end{frame}
\begin{frame}
\frametitle{Unrelated Title}


\begin{itemize}
\item 
number all the statements of a programnumbered statements represent nodes of the control flow
graph



an edge from one node to another node exists



if execution of statement of first node can result in transfer of control to the other node



\end{itemize}

\note[item]{}
\end{frame}
\begin{frame}
\frametitle{Unrelated Title}


\begin{itemize}
\item 




a non-empty set of nodes N



a set N0 of initial nodes (non-empty subset of N)



a set Nf of final nodes (non-empty subset of N)



a set E of edges (ni, nj) with I predecessor and j successor



path: sequence of nodes



length: number of edges (single node is a path of length 0)



subpath: subsequence of nodes in a path 




\end{itemize}

\note[item]{}
\end{frame}
\begin{frame}
\frametitle{Unrelated Title}


\begin{itemize}
\item 

path that starts at an initial node and ends at a final node
represents execution of test casessome test paths can be executed by many testssome test paths cannot be executed by any test



\end{itemize}

\note[item]{}
\end{frame}
\begin{frame}
\frametitle{Unrelated Title}


\begin{itemize}
\item 










A test path p visits node n if n is in p 
A test path p visits edge e if e is in p 






\end{itemize}

\note[item]{}
\end{frame}
\begin{frame}
\frametitle{Unrelated Title}


\begin{itemize}
\item 

path(t): test path executed by test tpath(T): set of test paths executed by set of tests Teach test executes one and only one test path (deterministic software)non-deterministic software: a test can execute different test paths



\end{itemize}

\note[item]{}
\end{frame}
\begin{frame}
\frametitle{Unrelated Title}


\begin{itemize}
\item 

properties of test paths



\end{itemize}

\note[item]{}
\end{frame}
\begin{frame}
\frametitle{Unrelated Title}


\begin{itemize}
\item 




for every syntactically reachable node n in N there is some path p in path(T) such that
p visits n



or TR contains each reachable node in G 




\end{itemize}

\note[item]{}
\end{frame}
\begin{frame}
\frametitle{Unrelated Title}


\begin{itemize}
\item 



TR contains each reachable path of length up to 1 in G


“length up to 1” allows for graphs with one node and no edge
edge coverage subsumes node coverage

NC and EC are only different when there is an edge and another subpath between a pair of nodes (i.e. if-else statement) 



\end{itemize}

\note[item]{}
\end{frame}
\begin{frame}
\frametitle{Unrelated Title}


\begin{itemize}
\item 




requires pairs of edges, or subpaths of length 2

formal: TR contains each reachable path of

length up to 2 in G





\end{itemize}

\note[item]{}
\end{frame}
\begin{frame}
\frametitle{Unrelated Title}


\begin{itemize}
\item 

TR contains all paths in G-> impossible when the graph has a loop 


\end{itemize}

\note[item]{}
\end{frame}
\begin{frame}
\frametitle{Unrelated Title}


\begin{itemize}
\item 
no node appears twice, except the first and last nodes are the same

-> no internal loops-> a loop is a simple path
-> any path can be composed of simple
paths


\end{itemize}

\note[item]{}
\end{frame}
\begin{frame}
\frametitle{Unrelated Title}


\begin{itemize}
\item TR contains each prime path in Grequires loops to be executed as well as skippedwill tour all paths of length 0 and 1subsumes NC and ECdoesn’t subsume EPC (if a node has an edge to itself)
\end{itemize}

\note[item]{}
\end{frame}
\begin{frame}
\frametitle{Unrelated Title}


\begin{itemize}
\item 

prime paths do not have internal loops – test paths mighttest path p can tours subpath q withsidetrips if every edge in q is also in p in the
same orderdetours if every node in q is also in p in the same
ordertest path that tours a path without sidetrip or detour
tours the path directly (excursion-free) 


\end{itemize}

\note[item]{}
\end{frame}
\begin{frame}
\frametitle{Unrelated Title}


\begin{itemize}
\item 



cannot be satisfied (i.e. unreachable statement)



most test criteria have some infeasible test requirements



it is usually undecidable whether all test requirements are feasible



without excursions, many structural criteria have more infeasible requirements



always allowing excursions weakens test criteria (detours are weaker than sidetrips)


Practical recommendation – Best Effort Touring
satisfy as many test requirements as possible without excursionsallow excursions to satisfy test requirements that would otherwise be unsatisfiable 
\end{itemize}

\note[item]{}
\end{frame}
\begin{frame}
\frametitle{Unrelated Title}


\begin{itemize}
\item 
try to ensure that values are computed and used correctlyDefinition: assignment of value to a variable at a nodedef(n): set of variables defined at node nUse: a use of a value in a variable at a nodeuse(n): set of variables used at node n 



\end{itemize}

\note[item]{}
\end{frame}
\begin{frame}
\frametitle{Unrelated Title}


\begin{itemize}
\item 




excursions can be used, just as with previous
testing



but test paths have to du-tour subpaths 




a test path p du-tours subpath d w.r.t. v if p tours d and
the subpath taken is def-clear w.r.t. v 






\end{itemize}

\note[item]{}
\end{frame}
\begin{frame}
\frametitle{Unrelated Title}


\begin{itemize}
\item predicates
can containBoolean variablesnon-boolean variables related by
>,<,==,!=Boolean function callsinternal
structure is created by logical operators: negation, and, or, implication, XOR,
equivalenceclause:
predicate with no logical operatorspredicates
can occur in decisions in programs, requirements,. ..tests
are intended to choose some subset of the total number of truth assignments

\end{itemize}

\note[item]{}
\end{frame}
\begin{frame}
\frametitle{Unrelated Title}


\begin{itemize}
\item develop model of the software as one
or more predicatesrequire tests to satisfy some
combination of clausesabbreviations:P
is the set of predicatesp
is a single predicate in PC
is the set of clauses in PCp
is the set of clauses in predicate pc
is a single clause in C
\end{itemize}

\note[item]{}
\end{frame}
\begin{frame}
\frametitle{Unrelated Title}


\begin{itemize}
\item for each p in P, TR contains two
requirements: p true and p falsewhen the predicates come from
conditions on edges, this is equivalent to edge coverage

\end{itemize}

\note[item]{}
\end{frame}
\begin{frame}
\frametitle{Unrelated Title}


\begin{itemize}
\item for each c in C, TR contains two
requirements: c evaluates to true and to falseCC does not always ensure PC => bad
\end{itemize}

\note[item]{}
\end{frame}
\begin{frame}
\frametitle{Unrelated Title}


\begin{itemize}
\item for each p in P, TR contains clauses
in Cp to evaluate each possible combinations of truth valuessimple, but expensive (2n
tests for n clauses)
\end{itemize}

\note[item]{}
\end{frame}
\begin{frame}
\frametitle{Unrelated Title}


\begin{itemize}
\item A clause ci in predicate
p, called the major clause, determines p if and only if the values of the
remaining minor clauses cj are such that changing ci
changes the value of p=> This is considered to make the
clause activeexample: p = a v b: if b=false, a
determines p
\end{itemize}

\note[item]{}
\end{frame}
\begin{frame}
\frametitle{Unrelated Title}


\begin{itemize}
\item for each p in P and each major
clause ci in Cp , choose minor clauses cj so
that ci determines p. TR has two requirements for each ci:
true and false
\end{itemize}

\note[item]{}
\end{frame}
\begin{frame}
\frametitle{Unrelated Title}


\begin{itemize}
\item focus on test requirements that do
not affect the predicatesFor each p in P and each major
clause ci in Cp, choose minor clauses cj, so
that ci does not determine p. TR has four requirements for each ci:
(1) ci evaluates to true with p true, (2) ci evaluates to
false with p true, (3) ci evaluates to true with p false, and (4) ci
evaluates to false with p falsepredicate coverage is always guaranteedany row in truth table is either
active or inactive for a certain variable

\end{itemize}

\note[item]{}
\end{frame}
\begin{frame}
\frametitle{Unrelated Title}


\begin{itemize}
\item easy for simple predicates, but what
about more complicated onespc=true is predicate p
with every occurrence of c replaced by truepc=false is predicate p
with every occurrence of c replaced by falsepc = pc=true
XOR pc=false describes values needed for c to determine pcan also be used to determine if a predicate is
good (if pc = true/false, other variables are irrelevant)
\end{itemize}

\note[item]{}
\end{frame}
\begin{frame}
\frametitle{Unrelated Title}

\begin{center}
\includegraphics[width=0.9\textwidth,height=0.9\textheight,keepaspectratio]{/Users/I516998/Library/Application Support/Anki2/User 1/collection.media/Bildschirm­foto 2022-11-27 um 13.35.19.png}
\end{center}


\note[item]{}
\end{frame}
\begin{frame}
\frametitle{Unrelated Title}


\begin{itemize}
\item also possible with logic coverageinfeasible test requirements have to
be recognized and ignoredproblem is hard
\end{itemize}

\note[item]{}
\end{frame}
\begin{frame}
\frametitle{Unrelated Title}

\begin{center}
\includegraphics[width=0.9\textwidth,height=0.9\textheight,keepaspectratio]{/Users/I516998/Library/Application Support/Anki2/User 1/collection.media/paste-fec7cbb0fcf81e40a27b2342ec5566bd1204fd23.jpg}
\end{center}


\note[item]{}
\end{frame}
\begin{frame}
\frametitle{Unrelated Title}

\begin{center}
\includegraphics[width=0.9\textwidth,height=0.9\textheight,keepaspectratio]{/Users/I516998/Library/Application Support/Anki2/User 1/collection.media/paste-e6d0fd2e0e76006fb464572edc018fe0b8210e8c.jpg}
\end{center}


\note[item]{}
\end{frame}
\begin{frame}
\frametitle{Unrelated Title}


\begin{itemize}
\item input domain to a program contains
all possible inputsvery large / infinite even for small
programstesting is fundamentally about
choosing finite sets of values from input domaininput parameters define scope of
input domain (method parameters, data reads, user inputs, …)domain for each input parameter is
partitioned into regions => choose at least one value from each region
\end{itemize}

\note[item]{}
\end{frame}
\begin{frame}
\frametitle{Unrelated Title}


\begin{itemize}
\item Domain D (a set of all combinations
of all possible inputs)Partition scheme q of DPartition q defines a set of blocks
Bq = b1, b2, …, bQpartition must be pairwise disjoint
and must cover the domain D
\end{itemize}

\note[item]{}
\end{frame}
\begin{frame}
\frametitle{Unrelated Title}


\begin{itemize}
\item value from each partition -> each value assumed to be equally
usefulfind characteristics in inputs,
partition each characteristic, choose tests by combining values from
characteristicsexample characteristics: input is
null, order of an input array, …
\end{itemize}

\note[item]{}
\end{frame}
\begin{frame}
\frametitle{Unrelated Title}


\begin{itemize}
\item easy to get wrong, example:order
of an array, blocks are sorted ascending, descending or no orderproblem:
if array only has one element, it will be in all blockssolution: Each characteristic should address
just one property => blocks with true/false for characteristic
\end{itemize}

\note[item]{}
\end{frame}
\begin{frame}
\frametitle{Unrelated Title}


\begin{itemize}
\item Interface-based approach (no
knowledge about program needed)develops
characteristics directly from individual input parameterssimplest
to applycan
be partially automatedFunctionality-based approachdevelops
characteristics from a behavioral view of the program under testharder
to developmay
result in better/fewer tests
\end{itemize}

\note[item]{}
\end{frame}
\begin{frame}
\frametitle{Unrelated Title}


\begin{itemize}
\item a creative engineering stepcandidates:parameter
types (interface-based)preconditions
and postconditionsrelationships
among variablesspecial
values (zero, null, blank)should not use source code but the
input domainbetter to have more characteristics
with few blocks (fewer mistakes, fewer tests)
\end{itemize}

\note[item]{}
\end{frame}
\begin{frame}
\frametitle{Unrelated Title}


\begin{itemize}
\item creative engineering stepoften follows directly from
characteristics definition (often done together)strategies for identifying blocks:include
valid, invalid and special valuessub-partition
some blocksexplore
boundaries of domainsinclude
values that represent “normal use”check for completeness and
disjointness
\end{itemize}

\note[item]{}
\end{frame}
\begin{frame}
\frametitle{Unrelated Title}


\begin{itemize}
\item all combinations of blocks from all characteristics
must be usednumber of tests = product of number
of blocks in each characteristicrarely feasible
\end{itemize}

\note[item]{}
\end{frame}
\begin{frame}
\frametitle{Unrelated Title}


\begin{itemize}
\item one value from each block for each
characteristic must be used at least oncenumber of tests = number of blocks
in largest characteristiccheap, but yields few tests
\end{itemize}

\note[item]{}
\end{frame}
\begin{frame}
\frametitle{Unrelated Title}


\begin{itemize}
\item a value from each block for each
characteristic must be combined with a value from every block for each other
characteristicnumber of tests = at least product
of two largest characteristics
\end{itemize}

\note[item]{}
\end{frame}
\begin{frame}
\frametitle{Unrelated Title}


\begin{itemize}
\item Uses domain knowledge of the programa base choice block is chosen for each
characteristic and a base test is formed by using the
base choice for each characteristicsubsequent tests are chosen by
holding all but one base choice constant and using each non-base choice in each
other characteristicnumber of tests = one base test +
one test for each other block in each partitionOnly one solution
\end{itemize}

\note[item]{}
\end{frame}
\begin{frame}
\frametitle{Unrelated Title}


\begin{itemize}
\item one or more base choice blocks are
chosen for each characteristic and base tests are formed by using each
base choice for each characteristic



Subsequent tests are chosen
by holding all but one base choice constant for each base test and
using each non-base choices in each other characteristic 



\end{itemize}

\note[item]{}
\end{frame}
\begin{frame}
\frametitle{Unrelated Title}


\begin{itemize}
\item some combinations of blocks are
infeasible => represented as constraintstwo types of constraintsa
block from one characteristic cannot be combined with a specific block from
anothera
block from one characteristic can only be combined with a specific block from
anotherhandling constraints depends on the
criterion usedACoC
and PW: drop infeasible pairsBC,
MBC: change a value to another non-base choice
\end{itemize}

\note[item]{}
\end{frame}
\begin{frame}
\frametitle{Unrelated Title}


\begin{itemize}
\item lots of software artifacts follow
strict syntax rulessyntax is often expressed as a
grammarartifacts governed by grammars can
come from many sourcessource
codeintegration
elementsdesign
documents input
descriptionstests are created with two goals:
cover syntax & violate syntax

\end{itemize}

\note[item]{}
\end{frame}
\begin{frame}
\frametitle{Unrelated Title}


\begin{itemize}
\item software engineering makes
widespread use of automata theoryprogramming
languages defined in Backus-Naur-Form (BNF)program
behavior described as finite state machinesallowable inputs defined by grammars
such as regular expressionsa test case is a string or sequence
of strings that is in the grammar
\end{itemize}

\note[item]{}
\end{frame}
\begin{frame}
\frametitle{Unrelated Title}


\begin{itemize}
\item BNF / contexfree grammar
\end{itemize}

\note[item]{}
\end{frame}
\begin{frame}
\frametitle{Unrelated Title}


\begin{itemize}
\item recognizer: determine when strings
are in the grammar (parsing)generator: derive strings in the
grammar
\end{itemize}

\note[item]{}
\end{frame}
\begin{frame}
\frametitle{Unrelated Title}


\begin{itemize}
\item TR contains each terminal symbol in
the grammareach terminal symbol must appear in
at least one of the strings in the set of tests
\end{itemize}

\note[item]{}
\end{frame}
\begin{frame}
\frametitle{Unrelated Title}


\begin{itemize}
\item TR contains each production in the
grammareach production rule choice must be
used in the derivation of at least one of the strings in the set of testsPDC subsumes TSC
\end{itemize}

\note[item]{}
\end{frame}
\begin{frame}
\frametitle{Unrelated Title}


\begin{itemize}
\item TR contains every possible string
that can be derived from the grammarsubsumes PDC and TSCpossibly infinite number of tests
\end{itemize}

\note[item]{}
\end{frame}
\begin{frame}
\frametitle{Unrelated Title}


\begin{itemize}
\item mutants can be used to cover strings
that violate the grammarmutant = variation of a valid stringmutants may be valid or invalid
stringsmutants are created with “mutation
operators”
\end{itemize}

\note[item]{}
\end{frame}
\begin{frame}
\frametitle{Unrelated Title}


\begin{itemize}
\item ground string: a string in the
grammarmutation operator: a rule that
specifies syntactic variations of stringsmutant = result of one application
of a mutation operator => either in or very close to the grammarwell-designed mutation operators
lead to effective testsmost programming languages have mutation operators defined for them
\end{itemize}

\note[item]{}
\end{frame}
\begin{frame}
\frametitle{Unrelated Title}


\begin{itemize}
\item for each mutation operator, TR
contains exactly one derived string
\end{itemize}

\note[item]{}
\end{frame}
\begin{frame}
\frametitle{Unrelated Title}


\begin{itemize}
\item for each mutation operator and each
production that the operator can be applied to, TR contains a mutated string
from that production
\end{itemize}

\note[item]{}
\end{frame}
\begin{frame}
\frametitle{Unrelated Title}


\begin{itemize}
\item a program is essentially a string
and can thus be mutatedmutated strings that are valid can
be executedprogram-based mutation testing aims
to mutate programs to see if a test will detect a difference in behavior to the
originalgrammar
= programming language syntaxground
strings = original programmutated
string = alternative program (must be in the grammar)Killing Mutants: given mutant m of
derivation D and test t, t kills m if and only if the output of t on D is
different from the output of t on m

\end{itemize}

\note[item]{}
\end{frame}
\begin{frame}
\frametitle{Unrelated Title}


\begin{itemize}
\item for each mutation m of a derivation
D which is not behaviorally equivalent to D, TR contains exactly one requirement
to kill mif mutation coverage is not
possible, the proportion of non-equivalent mutants killed is called the
mutation scorenumber of tests depends on the
syntax of the grammar and the mutation operatorsvery difficult to apply by handvery effective (considered “gold
standard” of testing)
\end{itemize}

\note[item]{}
\end{frame}
\begin{frame}
\frametitle{Unrelated Title}


\begin{itemize}
\item usually
not, multiple mutations can interfereless
of an issue for input-oriented mutation testingshould
be avoided for program-based mutation testing

\end{itemize}

\note[item]{}
\end{frame}
\begin{frame}
\frametitle{Unrelated Title}


\begin{itemize}
\item aims to unite component-based development, model-driven architecture and product-line engineering
\item has nested component hierarchy
\end{itemize}

\note[item]{}
\end{frame}
\begin{frame}
\frametitle{Unrelated Title}


\begin{itemize}
\item describe externally visible
properties of a systemdefines the system’s supplied
services (interface)imported
services (used components)basis for contract between system
and its clientsrequirements the realizations of the
system have to fulfill
\end{itemize}

\note[item]{}
\end{frame}
\begin{frame}
\frametitle{Unrelated Title}


\begin{itemize}
\item most methods take a relaxed view
about models (developer can decide)disadvantages: no concept of
completeness, weak concept of correctnessKobrA: content and form of a view
(model) should be described very precisely => strong concept of completeness and
correctnessdiagrams can be checked against each
other
\end{itemize}

\note[item]{}
\end{frame}
\begin{frame}
\frametitle{Unrelated Title}


\begin{itemize}
\item Structural Viewdata manipulated by the component, its
environment and any externally visible structureone
or more class diagramszero
or more object diagramsFunctional Viewcomputations
performed by the componentone
operation specification for each operationBehavioral Viewstates
exhibited by the component and the events that change themzero
or more statechart diagrams
\end{itemize}

\note[item]{}
\end{frame}
\begin{frame}
\frametitle{Unrelated Title}


\begin{itemize}
\item non-functional requirements
specification (quality characteristics)quality measuresdictionary: tables of model entities
and their roletest cases...
\end{itemize}

\note[item]{}
\end{frame}
\begin{frame}
\frametitle{Unrelated Title}


\begin{itemize}
\item specification of the effects/behavior of
system operations in terms of state changes and output eventsstate of system is represented by
the set of objects and their relationshipsa system operation maycreate/delete class instanceschange
attribute of existing objectadd
or delete objects from relationshipssend
event to a component
\end{itemize}

\note[item]{}
\end{frame}
\begin{frame}
\frametitle{Unrelated Title}


\begin{itemize}
\item used to define operationswritten as descriptive logical
predicates (true or false)operations are treated as black
boxes (no intermediate steps)
\end{itemize}

\note[item]{}
\end{frame}
\begin{frame}
\frametitle{Unrelated Title}


\begin{itemize}
\item potential subclauses of result
clause must be purely declarativemeaning of result clause must not
depend on order of subclausessystem operations must respect
invariantsa specification is satisfiable if
for all initial values satisfying the assumes clause, there exist final values
satisfying the result clause
\end{itemize}

\note[item]{}
\end{frame}
\begin{frame}
\frametitle{Unrelated Title}


\begin{itemize}
\item describes how system behaves in
response to external stimulithree important conecpts:events:
indivisible occurrences that have a location in space and timeoperations:
encapsulated unit of functionalitystates:
a period of time in between events that determines how the system will respond
to a subsequent eventin KobrA: events are invocations of
system’s operationsset of transitions from a state
define the operations that are possible in that stateexternal
transitions move the system into a new stateinternal
transitions leave the system in the same state
\end{itemize}

\note[item]{}
\end{frame}
\begin{frame}
\frametitle{Unrelated Title}


\begin{itemize}
\item only
components can have an operation compartmentif
the subject component has an operation compartment, it must list all the
operations of the systemif
an acquired component has an operation compartment, it should only list the
operations invoked by subject
\end{itemize}

\note[item]{}
\end{frame}
\begin{frame}
\frametitle{Unrelated Title}


\begin{itemize}
\item must
agree on the nature of the entitiesclasses,
attributes, relationships in operation specification must be in class diagramevery
class / attribute / operation in class diagram should appear in at least one
operation specification
\end{itemize}

\note[item]{}
\end{frame}
\begin{frame}
\frametitle{Unrelated Title}


\begin{itemize}
\item many failures of software systems
come down to specification problemsspecification
itself can be wrongimplementation
may not match the specifications (e.g. testing mistakes)software specifications need to bereadable,
precise and easy-to-understandaccompanied
by comprehensive test definitionswriting tests at specification often
reveals weaknesses in the specification

\end{itemize}

\note[item]{}
\end{frame}
\begin{frame}
\frametitle{Unrelated Title}


\begin{itemize}
\item KobrA specifications provide an
ideal basis for black-box testsbehavioral model => graph-based criteriafunctional model => logic-based criteriafunctional model and structural
model => input space partitioning
\end{itemize}

\note[item]{}
\end{frame}
\begin{frame}
\frametitle{Unrelated Title}


\begin{itemize}
\item apply graph-based criteriaapply logic-based criteriaapply data partitioning criteriaconsolidate all requirements and
define test sets to fulfill themdefine concrete tests so satisfy
requirements in a minimal way
\end{itemize}

\note[item]{}
\end{frame}
\begin{frame}
\frametitle{Unrelated Title}


\begin{itemize}
\item KobrA state: values of externally
visible attributes + externally visible abstract statesChanger (setter) operations: may
change the state, reflected in postconditionsInspector (getter) operations: never
change state
\end{itemize}

\note[item]{}
\end{frame}
\begin{frame}
\frametitle{Unrelated Title}


\begin{itemize}
\item each state is mapped to a state node
Sieach external transition is mapped
to a transition node Ti (because formal graphs don’t have multiple
edges between nodes)
each internal transition caused by a changer
operation is mapped to a node plus two edges
add defs and uses based on operation
specification for each attribute
\end{itemize}

\note[item]{}
\end{frame}
\begin{frame}
\frametitle{Unrelated Title}


\begin{itemize}
\item combine all structural and data-flow
based test requirementsremove duplicates and subpaths
\end{itemize}

\note[item]{}
\end{frame}
\begin{frame}
\frametitle{Unrelated Title}


\begin{itemize}
\item separate changer and inspector
operationsin general we can ignore inspector
operationswill take strict assumption of
preconditionslogic coverage only relevant for
operations with predicates
\end{itemize}

\note[item]{}
\end{frame}
\begin{frame}
\frametitle{Unrelated Title}


\begin{itemize}
\item also strict assumption for
preconditionsapply input space partitioning to
operations
\end{itemize}

\note[item]{}
\end{frame}
\begin{frame}
\frametitle{Unrelated Title}


\begin{itemize}
\item 




A predicate is an expression that evaluates to a boolean valuecan contain: boolean values, non boolean values related with <, >, != etc., boolean function callsInternal structure with: logical operators (and, or, negation etc.)Clauses = predicate with no logical operators




\end{itemize}

\note[item]{}
\end{frame}
\begin{frame}
\frametitle{Unrelated Title}


\begin{itemize}
\item 





P is the set of predicates



p is a single predicate in P



C is the set of clauses in P



Cp is the set of clauses in predicate p



c is a single clause in C 





\end{itemize}

\note[item]{}
\end{frame}
\begin{frame}
\frametitle{Unrelated Title}


\begin{itemize}
\item 



Predicate Coverage (PC) : For each p in P, TR contains two
requirements: p evaluates to true, and p evaluates to false. 



■ When the predicates come from
conditions on edges, this is equivalent
to edge coverage 






\end{itemize}

\note[item]{}
\end{frame}
\begin{frame}
\frametitle{Unrelated Title}


\begin{itemize}
\item 


Clause Coverage (CC) : For each c in C, TR contains two
requirements: c evaluates to true, and c evaluates to false.PC does not evaluate all the clauses, so clause coverage is
defined to ensure the predicate is executed with each clause set
to true and to false. 



\end{itemize}

\note[item]{}
\end{frame}
\begin{frame}
\frametitle{Unrelated Title}


\begin{itemize}
\item 



PC does not fully exercise all the clauses



CC does not always ensure PC



That is, we can satisfy CC without causing the predicate
to be both true and false 





\end{itemize}

\note[item]{}
\end{frame}
\begin{frame}
\frametitle{Unrelated Title}


\begin{itemize}
\item 



Combinatorial Coverage (CoC) : For each p in P, TR has test
requirements for the clauses in Cp to evaluate to each possible
combination of truth values. -> expensive





2^n tests, where n is the number of clauses



impractical for predicates with more than 3 or 4 clauses 








\end{itemize}

\note[item]{}
\end{frame}
\begin{frame}
\frametitle{Unrelated Title}


\begin{itemize}
\item 


A clause ci in predicate p, called the major clause,
determines p if and only if the values of the remaining minor clauses cj are such that changing ci changes the
value of p.



\end{itemize}

\note[item]{}
\end{frame}
\begin{frame}
\frametitle{Unrelated Title}


\begin{itemize}
\item 



Active Clause Coverage (ACC) : For each p in P and each major clause
ci in Cp, choose minor clauses cj, j != i, so that ci determines p. TR has
two requirements for each ci : ci evaluates to true and ci evaluates to
false. 



\end{itemize}

\note[item]{}
\end{frame}
\begin{frame}
\frametitle{Unrelated Title}


\begin{itemize}
\item 



General Active Clause Coverage : For each p in P and each major
clause ci in Cp, choose minor clauses cj, j != i, so that ci determines p.
TR has two requirements for each ci : ci evaluates to true and ci
evaluates to false. The values chosen for the minor clauses cj do not
need to be the same when ci is true as when ci is false -> most relaxed, possible to satisfy GACC without PC :( 



\end{itemize}

\note[item]{}
\end{frame}
\begin{frame}
\frametitle{Unrelated Title}


\begin{itemize}
\item 


Restricted Active Clause Coverage RACC: For each p in P and each major
clause ci in Cp, choose minor clauses cj, j != i, so that ci determines p.
TR has two requirements for each ci: ci evaluates to true and ci
evaluates to false. The values chosen for the minor clauses cj must
be the same when ci is true as when ci is false.-> often leads to infeasible test requirements 



\end{itemize}

\note[item]{}
\end{frame}
\begin{frame}
\frametitle{Unrelated Title}


\begin{itemize}
\item 


Correlated Active Clause Coverage CACC: For each p in P and each major
clause ci in Cp, choose minor clauses cj, j != i, so that ci determines p.
TR has two requirements for each ci: ci evaluates to true and ci
evaluates to false. The values chosen for the minor clauses cj must
cause p to be true for one value of the major clause ci and false for
the other 





Implicitly allows minor clauses to have different values



Explicitly satisfies (subsumes) predicate coverage 








\end{itemize}

\note[item]{}
\end{frame}
\begin{frame}
\frametitle{Unrelated Title}


\begin{itemize}
\item 



Inactive Clause Coverage (ICC) : For each p in P and each major
clause ci in Cp, choose minor clauses cj, j != i, so that ci does not
determine p. TR has four requirements for each ci: (1) ci evaluates to
true with p true, (2) ci evaluates to false with p true, (3) ci evaluates to
true with p false, and (4) ci evaluates to false with p false. 



General Inactive Clause Coverage (GICC) : The values chosen for the minor
clauses cj do not need to be the same when ci is true as when ci is false 



Restricted Inactive Clause Coverage (RICC) : The values chosen for the minor
clauses cj must be the same when ci is true as when ci is false 









\end{itemize}

\note[item]{}
\end{frame}
\begin{frame}
\frametitle{Unrelated Title}


\begin{itemize}
\item 




Let:



pc=true is predicate p with every occurrence of c replaced by true



pc=false is predicate p with every occurrence of c replaced by false





To find values for the minor clauses, connect pc=true and pc=false

with exclusive OR
pc = pc=true XOR pc=false



After solving, pc describes exactly the values needed for c to
determine p 

Example:










\end{itemize}

\note[item]{}
\end{frame}
\begin{frame}
\frametitle{Unrelated Title}


\begin{itemize}
\item XOR: only true if a/b different



a ∧ ¬b: only true if a true and b false


a XOR a∧b: same



¬a ∧ b: only true if  a false and b true






a XOR a∨b : same






De Morgan's laws ¬(a ∨ b) = ¬ a ∧ ¬ b ¬(a ∧ b) = ¬ a ∨ ¬ b 



\end{itemize}

\note[item]{}
\end{frame}
\begin{frame}
\frametitle{Unrelated Title}

\begin{center}
\includegraphics[width=0.9\textwidth,height=0.9\textheight,keepaspectratio]{/Users/I516998/Library/Application Support/Anki2/User 1/collection.media/Bildschirmfoto 2023-01-03 um 15.06.18.png}
\end{center}


\note[item]{}
\end{frame}
\begin{frame}
\frametitle{Unrelated Title}

\begin{center}
\includegraphics[width=0.9\textwidth,height=0.9\textheight,keepaspectratio]{/Users/I516998/Library/Application Support/Anki2/User 1/collection.media/Bildschirmfoto 2023-01-06 um 12.49.48.png}
\end{center}


\note[item]{}
\end{frame}
\begin{frame}
\frametitle{Unrelated Title}

\begin{center}
\includegraphics[width=0.9\textwidth,height=0.9\textheight,keepaspectratio]{/Users/I516998/Library/Application Support/Anki2/User 1/collection.media/Bildschirmfoto 2023-01-06 um 12.50.32.png}
\end{center}


\note[item]{}
\end{frame}
\begin{frame}
\frametitle{Unrelated Title}


\begin{itemize}
\item 1. find characteristics in input (parameters, semantic, description) (order, input)
\item 2. partition each characteristic (ascending, descending, null ...)
\item 3. choose tests by combining values from characteristics
\item -> each characteristic should address just one property!! File F sorted ascending
\item -> otherwise not disjoint or complete
\end{itemize}

\note[item]{}
\end{frame}
\begin{frame}
\frametitle{Unrelated Title}

\begin{center}
\includegraphics[width=0.9\textwidth,height=0.9\textheight,keepaspectratio]{/Users/I516998/Library/Application Support/Anki2/User 1/collection.media/Bildschirmfoto 2023-01-06 um 12.59.53.png}
\end{center}

\begin{itemize}
\item 1. Interface-based approach (from individual input parameters)
\item 2. Functionality-based approach (from behavioral view) - harder - better/more effective
\end{itemize}

\note[item]{}
\end{frame}
\begin{frame}
\frametitle{Unrelated Title}

\begin{center}
\includegraphics[width=0.9\textwidth,height=0.9\textheight,keepaspectratio]{/Users/I516998/Library/Application Support/Anki2/User 1/collection.media/Bildschirmfoto 2023-01-06 um 13.02.17.png}
\includegraphics[width=0.9\textwidth,height=0.9\textheight,keepaspectratio]{/Users/I516998/Library/Application Support/Anki2/User 1/collection.media/Bildschirmfoto 2023-01-06 um 13.02.29.png}
\end{center}


\note[item]{}
\end{frame}
\begin{frame}
\frametitle{Unrelated Title}


\begin{itemize}
\item invalid, valid, special values?boundaries of domains?values that represent Normal Use?subpartition some blocks?-> check completeness and disjointness
\end{itemize}

\note[item]{}
\end{frame}
\begin{frame}
\frametitle{Unrelated Title}

\begin{center}
\includegraphics[width=0.9\textwidth,height=0.9\textheight,keepaspectratio]{/Users/I516998/Library/Application Support/Anki2/User 1/collection.media/Bildschirmfoto 2023-01-06 um 13.08.05.png}
\includegraphics[width=0.9\textwidth,height=0.9\textheight,keepaspectratio]{/Users/I516998/Library/Application Support/Anki2/User 1/collection.media/Bildschirmfoto 2023-01-06 um 13.08.36.png}
\end{center}


\note[item]{}
\end{frame}
\begin{frame}
\frametitle{Unrelated Title}

\begin{center}
\includegraphics[width=0.9\textwidth,height=0.9\textheight,keepaspectratio]{/Users/I516998/Library/Application Support/Anki2/User 1/collection.media/Bildschirmfoto 2023-01-06 um 13.09.57.png}
\includegraphics[width=0.9\textwidth,height=0.9\textheight,keepaspectratio]{/Users/I516998/Library/Application Support/Anki2/User 1/collection.media/Bildschirmfoto 2023-01-06 um 13.10.07.png}
\end{center}

\begin{itemize}
\item - ACoC: num tests = product of num blocks
\item - EC: num tests = highest num blocks
\end{itemize}

\note[item]{}
\end{frame}
\begin{frame}
\frametitle{Unrelated Title}

\begin{center}
\includegraphics[width=0.9\textwidth,height=0.9\textheight,keepaspectratio]{/Users/I516998/Library/Application Support/Anki2/User 1/collection.media/Bildschirmfoto 2023-01-06 um 13.12.07.png}
\end{center}

\begin{itemize}
\item - num tests = min. product of 2 largest characteristics
\end{itemize}

\note[item]{}
\end{frame}
\begin{frame}
\frametitle{Unrelated Title}

\begin{center}
\includegraphics[width=0.9\textwidth,height=0.9\textheight,keepaspectratio]{/Users/I516998/Library/Application Support/Anki2/User 1/collection.media/Bildschirmfoto 2023-01-06 um 13.17.46.png}
\includegraphics[width=0.9\textwidth,height=0.9\textheight,keepaspectratio]{/Users/I516998/Library/Application Support/Anki2/User 1/collection.media/Bildschirmfoto 2023-01-06 um 13.20.36.png}
\end{center}

\begin{itemize}
\item - num tests: 1 + sum of all other blocks per partition
\end{itemize}

\note[item]{}
\end{frame}
\begin{frame}
\frametitle{Unrelated Title}


\begin{itemize}
\item Some combinations = infeasible -> represented as constraints (either NOT combined with or ONLY combined with)ACoC + PW -> drop infeasible pairsBC + MBC -> change a value to another non-base choice to find feasible combination
\end{itemize}

\note[item]{}
\end{frame}
\begin{frame}
\frametitle{Unrelated Title}

\begin{center}
\includegraphics[width=0.9\textwidth,height=0.9\textheight,keepaspectratio]{/Users/I516998/Library/Application Support/Anki2/User 1/collection.media/Bildschirmfoto 2023-01-06 um 13.23.23.png}
\end{center}


\note[item]{}
\end{frame}
\begin{frame}
\frametitle{Unrelated Title}


\begin{itemize}
\item Software often follows syntax rules, expressed as grammar in a languageTest goals: cover syntax in some way or violate the syntax
\end{itemize}

\note[item]{}
\end{frame}
\begin{frame}
\frametitle{Unrelated Title}


\begin{itemize}
\item Allowable inputs defined by grammars such as regular expressionstest case: string / sequence of strings that is "in the grammar" = satisfies the derivation rules
\end{itemize}

\note[item]{}
\end{frame}
\begin{frame}
\frametitle{Unrelated Title}

\begin{center}
\includegraphics[width=0.9\textwidth,height=0.9\textheight,keepaspectratio]{/Users/I516998/Library/Application Support/Anki2/User 1/collection.media/Bildschirmfoto 2023-01-06 um 15.32.11.png}
\end{center}


\note[item]{}
\end{frame}
\begin{frame}
\frametitle{Unrelated Title}


\begin{itemize}
\item Recognizer: determines when strings are in the grammar (parsing)Generator: derive strings in the grammar
\end{itemize}

\note[item]{}
\end{frame}
\begin{frame}
\frametitle{Unrelated Title}

\begin{center}
\includegraphics[width=0.9\textwidth,height=0.9\textheight,keepaspectratio]{/Users/I516998/Library/Application Support/Anki2/User 1/collection.media/Bildschirmfoto 2023-01-06 um 15.35.07.png}
\includegraphics[width=0.9\textwidth,height=0.9\textheight,keepaspectratio]{/Users/I516998/Library/Application Support/Anki2/User 1/collection.media/Bildschirmfoto 2023-01-06 um 15.35.22.png}
\end{center}

\begin{itemize}
\item max number of tests: number of termination symbols
\item number of tests: number of productions
\end{itemize}

\note[item]{}
\end{frame}
\begin{frame}
\frametitle{Unrelated Title}

\begin{center}
\includegraphics[width=0.9\textwidth,height=0.9\textheight,keepaspectratio]{/Users/I516998/Library/Application Support/Anki2/User 1/collection.media/Bildschirmfoto 2023-01-06 um 15.36.32.png}
\end{center}

\begin{itemize}
\item number of DC tests in stream grammar: infinityyyy
\end{itemize}

\note[item]{}
\end{frame}
\begin{frame}
\frametitle{Unrelated Title}


\begin{itemize}
\item mutant = variation of a valid string, can be valid or invalidcreated by mutation operators
\end{itemize}

\note[item]{}
\end{frame}
\begin{frame}
\frametitle{Unrelated Title}


\begin{itemize}
\item GS: string in the grammarMO: rule that specifies syntactic variations of strings generated from a grammar, KEY TO EFFECTIVE TESTINGdefined for many languages: programming, specification, modeling, input grammarsM: result of one application of a mutation operator
\end{itemize}

\note[item]{}
\end{frame}
\begin{frame}
\frametitle{Unrelated Title}

\begin{center}
\includegraphics[width=0.9\textwidth,height=0.9\textheight,keepaspectratio]{/Users/I516998/Library/Application Support/Anki2/User 1/collection.media/Bildschirmfoto 2023-01-06 um 15.48.20.png}
\end{center}


\note[item]{}
\end{frame}
\begin{frame}
\frametitle{Unrelated Title}

\begin{center}
\includegraphics[width=0.9\textwidth,height=0.9\textheight,keepaspectratio]{/Users/I516998/Library/Application Support/Anki2/User 1/collection.media/Bildschirmfoto 2023-01-06 um 15.52.14.png}
\end{center}

\begin{itemize}
\item mutate programs to see if a test will detect difference in behavior to originalavoid using multiple mutation operators at the same time!! can interfere
\end{itemize}

\note[item]{}
\end{frame}
\begin{frame}
\frametitle{Unrelated Title}

\begin{center}
\includegraphics[width=0.9\textwidth,height=0.9\textheight,keepaspectratio]{/Users/I516998/Library/Application Support/Anki2/User 1/collection.media/Bildschirmfoto 2023-01-06 um 15.53.12.png}
\end{center}

\begin{itemize}
\item number of tests: depends on syntax of artefact and on mutation operators
\end{itemize}

\note[item]{}
\end{frame}
\begin{frame}
\frametitle{Unrelated Title}

\begin{center}
\includegraphics[width=0.9\textwidth,height=0.9\textheight,keepaspectratio]{/Users/I516998/Library/Application Support/Anki2/User 1/collection.media/Bildschirmfoto 2023-01-06 um 15.55.51.png}
\end{center}


\note[item]{}
\end{frame}
\begin{frame}
\frametitle{Unrelated Title}

\begin{center}
\includegraphics[width=0.9\textwidth,height=0.9\textheight,keepaspectratio]{/Users/I516998/Library/Application Support/Anki2/User 1/collection.media/Bildschirmfoto 2023-01-06 um 15.59.23.png}
\end{center}


\note[item]{}
\end{frame}
\begin{frame}
\frametitle{Unrelated Title}

\begin{center}
\includegraphics[width=0.9\textwidth,height=0.9\textheight,keepaspectratio]{/Users/I516998/Library/Application Support/Anki2/User 1/collection.media/Bildschirmfoto 2023-01-07 um 16.50.34.png}
\end{center}


\note[item]{}
\end{frame}
\begin{frame}
\frametitle{Unrelated Title}

\begin{center}
\includegraphics[width=0.9\textwidth,height=0.9\textheight,keepaspectratio]{/Users/I516998/Library/Application Support/Anki2/User 1/collection.media/Bildschirmfoto 2023-01-07 um 16.52.10.png}
\end{center}


\note[item]{}
\end{frame}
\begin{frame}
\frametitle{Unrelated Title}

\begin{center}
\includegraphics[width=0.9\textwidth,height=0.9\textheight,keepaspectratio]{/Users/I516998/Library/Application Support/Anki2/User 1/collection.media/Bildschirmfoto 2023-01-07 um 16.52.58.png}
\end{center}


\note[item]{}
\end{frame}
\begin{frame}
\frametitle{Unrelated Title}

\begin{center}
\includegraphics[width=0.9\textwidth,height=0.9\textheight,keepaspectratio]{/Users/I516998/Library/Application Support/Anki2/User 1/collection.media/Bildschirmfoto 2023-01-07 um 16.53.57.png}
\end{center}


\note[item]{}
\end{frame}
\begin{frame}
\frametitle{Unrelated Title}


\begin{itemize}
\item OpSpec = form based specification of the effects of system ops (state changes + output events)OpSpec: specify the behavior of system ops declarativelystate of system: represented by set of objects + their relationshipsA system op may: create/delete instance of class, change attribute value, add/delete objects from relationships....
\end{itemize}

\note[item]{}
\end{frame}
\begin{frame}
\frametitle{Unrelated Title}

\begin{center}
\includegraphics[width=0.9\textwidth,height=0.9\textheight,keepaspectratio]{/Users/I516998/Library/Application Support/Anki2/User 1/collection.media/Bildschirmfoto 2023-01-07 um 16.57.46.png}
\end{center}

\begin{itemize}
\item specified in the functional model by means of pre and post conditionswritten as descriptive clauses / logical predicatestreated as black boxes for specification
\end{itemize}

\note[item]{}
\end{frame}
\begin{frame}
\frametitle{Unrelated Title}


\begin{itemize}
\item NameDescription - purpose + normal/exceptional effectsRules - constraints for realizationReceives - info inputReturns - info returnedSends - signals to imported systemsReads - visible info accessedChanges - visible info changedAssumes - precondition on visible system state + inputs to garantuee post conditionResult - strongest postcondition on visible system state + returns after execution with true assumes 
\end{itemize}

\note[item]{}
\end{frame}
\begin{frame}
\frametitle{Unrelated Title}


\begin{itemize}
\item Result clause + subclauses -> purely declarative + no flow of controlSystem ops -> respect invariantsSpecification = satisfiable, if for all initial values satisfying the assumes clause, there exists final values satisfying the result clauseOften better: conditional expression for results in normal vs. exceptional
\end{itemize}

\note[item]{}
\end{frame}
\begin{frame}
\frametitle{Unrelated Title}


\begin{itemize}
\item Behavioural Model: describes how a system reacts to external stimuli
\end{itemize}

\note[item]{}
\end{frame}
\begin{frame}
\frametitle{Unrelated Title}

\begin{center}
\includegraphics[width=0.9\textwidth,height=0.9\textheight,keepaspectratio]{/Users/I516998/Library/Application Support/Anki2/User 1/collection.media/Bildschirmfoto 2023-01-07 um 17.15.14.png}
\end{center}


\note[item]{}
\end{frame}
\begin{frame}
\frametitle{Unrelated Title}


\begin{itemize}
\item only components can have operation compartmentif subject component has operation compartment -> must list all system opsif acquired component has operation compartment -> only list ops that subject component invokes (sends clause)
\end{itemize}

\note[item]{}
\end{frame}
\begin{frame}
\frametitle{Unrelated Title}


\begin{itemize}
\item outgoing transitions of each state must be disjoint (behavioral model must be deterministic)
\end{itemize}

\note[item]{}
\end{frame}
\begin{frame}
\frametitle{Unrelated Title}


\begin{itemize}
\item every operation from component under specification must have Op Specassumes + result clause -> must be boolean expressionssends, receives, returns, changes (terms) -> must refer to items mentioned in assumes and result clause
\end{itemize}

\note[item]{}
\end{frame}
\begin{frame}
\frametitle{Unrelated Title}


\begin{itemize}
\item the two models must agree on the nature of the entities that exist in order for operations to do their work
\end{itemize}

\note[item]{}
\end{frame}
\begin{frame}
\frametitle{Unrelated Title}


\begin{itemize}
\item OD can only define instances of entities in CD of same system specification
\end{itemize}

\note[item]{}
\end{frame}
\begin{frame}
\frametitle{Unrelated Title}


\begin{itemize}
\item atts in statechart diagram must must match those in respective classconditions in statechart diagram only defined by entities from CD
\end{itemize}

\note[item]{}
\end{frame}
\begin{frame}
\frametitle{Unrelated Title}


\begin{itemize}
\item instance names from OpSpec match those from OD
\end{itemize}

\note[item]{}
\end{frame}
\begin{frame}
\frametitle{Unrelated Title}


\begin{itemize}
\item must agree on events/operations/statesOpSpec only mention states/events from SDSD only mention ops that have OpSpec
\end{itemize}

\note[item]{}
\end{frame}
\begin{frame}
\frametitle{Unrelated Title}


\begin{itemize}
\item wrong specification (wrong/missing reqs)implementation not matching specification
\end{itemize}

\note[item]{}
\end{frame}
\begin{frame}
\frametitle{Unrelated Title}


\begin{itemize}
\item Structural model and Functional model --> input space partitioningFunctional model --> logic based criteriaBehavioural model --> graph-based criteria
\end{itemize}

\note[item]{}
\end{frame}
\begin{frame}
\frametitle{Unrelated Title}


\begin{itemize}
\item apply graph based criteria ( identify + define reqs du-pairs)apply logic based criteria (use sysOpSpec for predicate coverage reqs)apply data partitioning criteria (create partition model from sysOps)consolidate all reqs and define test setsdefine concrete tests to realize in minimal way
\end{itemize}

\note[item]{}
\end{frame}
\begin{frame}
\frametitle{Unrelated Title}


\begin{itemize}
\item state of system captured by: external atts (struct. + behav. views) external visible abstract states (behav. view)Changer Ops: sometimes change state after execution, represented in postcondition/change of opSpecconstructor/setter opsInspector Ops:never change system state, no changes clausegetter ops
\end{itemize}

\note[item]{}
\end{frame}
\begin{frame}
\frametitle{Unrelated Title}


\begin{itemize}
\item State -> mapped to state node Siexternal Transition -> mapped to transition node Ti + 2 edges (from initial state + to new state)internal Transition -> same as external but back to same FINAL state
\end{itemize}

\note[item]{}
\end{frame}
\begin{frame}
\frametitle{Unrelated Title}

\begin{center}
\includegraphics[width=0.9\textwidth,height=0.9\textheight,keepaspectratio]{/Users/I516998/Library/Application Support/Anki2/User 1/collection.media/Bildschirmfoto 2023-01-09 um 13.14.49.png}
\end{center}


\note[item]{}
\end{frame}
\begin{frame}
\frametitle{Unrelated Title}

\begin{center}
\includegraphics[width=0.9\textwidth,height=0.9\textheight,keepaspectratio]{/Users/I516998/Library/Application Support/Anki2/User 1/collection.media/Bildschirmfoto 2023-01-09 um 13.20.04.png}
\includegraphics[width=0.9\textwidth,height=0.9\textheight,keepaspectratio]{/Users/I516998/Library/Application Support/Anki2/User 1/collection.media/Bildschirmfoto 2023-01-09 um 13.20.42.png}
\end{center}


\note[item]{}
\end{frame}
\begin{frame}
\frametitle{Unrelated Title}


\begin{itemize}
\item Combine structural (e.g. Edge Coverage) and data-flow based (e.g. DU Path Coverage) reqsremove any duplicates or subpath
\end{itemize}

\note[item]{}
\end{frame}
\begin{frame}
\frametitle{Unrelated Title}


\begin{itemize}
\item only for changer ops with predicatesformulate Predicate Coverage Test Requirements (e.g. push(): elems=max-1-> True/False)
\end{itemize}

\note[item]{}
\end{frame}
\begin{frame}
\frametitle{Unrelated Title}


\begin{itemize}
\item only for changer opscharacteristics for each changer op (e.g. priorState of Stack, Object)formulate Base Choices
\end{itemize}

\note[item]{}
\end{frame}
\begin{frame}
\frametitle{Unrelated Title}


\begin{itemize}
\item establish whether a system fulfils users and customers’ expectationsperformed on (partly) finished system
\end{itemize}

\note[item]{}
\end{frame}
\begin{frame}
\frametitle{Unrelated Title}


\begin{itemize}
\item tests designed to reflect the frequency of user inputsused for reliability estimation
\end{itemize}

\note[item]{}
\end{frame}
\begin{frame}
\frametitle{Unrelated Title}


\begin{itemize}
\item checks the system from the perspective of the GUIis all functionality efficiently accessible?
\end{itemize}

\note[item]{}
\end{frame}
\begin{frame}
\frametitle{Unrelated Title}


\begin{itemize}
\item designed to discover defects in the systemcan’t reveal the absence of errorsonly exhaustive testing (all combinations) can show a program is free from defects -> impossiblegood testing is about making the system fail => destructive process positive tests should be seen as uncovering an errorbasic idea: stimulate program with input and compare output with expected
\end{itemize}

\note[item]{}
\end{frame}
\begin{frame}
\frametitle{Unrelated Title}


\begin{itemize}
\item ensures that the application works with differently configured systems
\end{itemize}

\note[item]{}
\end{frame}
\begin{frame}
\frametitle{Unrelated Title}


\begin{itemize}
\item evaluate scalability for more users / higher data volumeimportant to identify bottlenecks
\end{itemize}

\note[item]{}
\end{frame}
\begin{frame}
\frametitle{Unrelated Title}


\begin{itemize}
\item exercises the system beyond its maximum designed loadchecks for unacceptable loss of service or data
\end{itemize}

\note[item]{}
\end{frame}
\begin{frame}
\frametitle{Unrelated Title}


\begin{itemize}
\item static defect in the software
\end{itemize}

\note[item]{}
\end{frame}
\begin{frame}
\frametitle{Unrelated Title}


\begin{itemize}
\item incorrect internal state as manifestation of fault
\end{itemize}

\note[item]{}
\end{frame}
\begin{frame}
\frametitle{Unrelated Title}


\begin{itemize}
\item externally visible incorrect behavior of the system
\end{itemize}

\note[item]{}
\end{frame}
\begin{frame}
\frametitle{Unrelated Title}


\begin{itemize}
\item specific things that must be covered during testing
\end{itemize}

\note[item]{}
\end{frame}
\begin{frame}
\frametitle{Unrelated Title}


\begin{itemize}
\item collection of rules that define test requirements
\end{itemize}

\note[item]{}
\end{frame}
\begin{frame}
\frametitle{Unrelated Title}


\begin{itemize}
\item useful for visualizing the structure of source code
\end{itemize}

\note[item]{}
\end{frame}
\begin{frame}
\frametitle{Unrelated Title}


\begin{itemize}
\item Single-Entry-Single-Exit (SESE) graphs
\item all test paths start at same node and end at same node 
\end{itemize}

\note[item]{}
\end{frame}
\begin{frame}
\frametitle{Unrelated Title}


\begin{itemize}
\item 





A test path p tours subpath q if q is a subpath of p 





\end{itemize}

\note[item]{}
\end{frame}
\begin{frame}
\frametitle{Unrelated Title}


\begin{itemize}
\item develop a model of the software as a graphrequire tests to visit or tour specific nodes, edges or subpaths
\end{itemize}

\note[item]{}
\end{frame}
\begin{frame}
\frametitle{Unrelated Title}


\begin{itemize}
\item Rules that define test requirements
\end{itemize}

\note[item]{}
\end{frame}
\begin{frame}
\frametitle{Unrelated Title}


\begin{itemize}
\item 


Given a set of test requirements TR for a criterion C, a set
of tests T satisfies C on a graph if and only if for every test requirement in
TR, there is a test path in path(T) that meets the test requirement tr 


\end{itemize}

\note[item]{}
\end{frame}
\begin{frame}
\frametitle{Unrelated Title}


\begin{itemize}
\item location in graph can be reached from another location if there is a pathsyntactic reach: a subpath exists in the graphsemantic reach: a test exists that can execute that subpath 
\end{itemize}

\note[item]{}
\end{frame}
\begin{frame}
\frametitle{Unrelated Title}


\begin{itemize}
\item A simple path that is not a proper subpath of another simple path 
\end{itemize}

\note[item]{}
\end{frame}
\begin{frame}
\frametitle{Unrelated Title}


\begin{itemize}
\item For each set of du-paths S = du(n, v), TR contains at least one path d in Severy dev reaches a use
\end{itemize}

\note[item]{}
\end{frame}
\begin{frame}
\frametitle{Unrelated Title}


\begin{itemize}
\item For each set of du-paths to uses S = du(n, m, v), TR contains at least one path d in S
\end{itemize}

\note[item]{}
\end{frame}
\begin{frame}
\frametitle{Unrelated Title}


\begin{itemize}
\item For each set S = du(n, m, v), TR contains every path d in Severy path between defs and uses is covered 
\end{itemize}

\note[item]{}
\end{frame}
\begin{frame}
\frametitle{Unrelated Title}


\begin{itemize}
\item pair of nodes (n, m) such that variable v is defined at n and used at m
\end{itemize}

\note[item]{}
\end{frame}
\begin{frame}
\frametitle{Unrelated Title}


\begin{itemize}
\item path is def-clear from n to m w.r.t. variable v if v is not given another value anywhere in the path
\end{itemize}

\note[item]{}
\end{frame}
\begin{frame}
\frametitle{Unrelated Title}


\begin{itemize}
\item simple subpath that is def-clear w.r.t v from a def of v to a use of v
\item du(n, m, v): set of du-paths from n to m w.r.t vdu(n, v): set of du-paths that start from n w.r.t v 
\end{itemize}

\note[item]{}
\end{frame}
\begin{frame}
\frametitle{Unrelated Title}


\begin{itemize}
\item Boolean variablesnon-boolean variables related by >,<,==,!=Boolean function calls
\end{itemize}

\note[item]{}
\end{frame}
\begin{frame}
\frametitle{Unrelated Title}


\begin{itemize}
\item Ambiguity: do the minor clauses have to have the same values when the major clause is true and false?
\item three separate criteria:
\end{itemize}

\note[item]{}
\end{frame}
\begin{frame}
\frametitle{Unrelated Title}


\begin{itemize}
\item 



For each p in P and each major
clause ci in Cp, choose minor clauses cj, j != i, so that ci determines p.
TR has two requirements for each ci : ci evaluates to true and ci
evaluates to false. The values chosen for the minor clauses cj do not
need to be the same when ci is true as when ci is false


minor clauses do not need to be the samedoes not imply predicate coverage => bad
\end{itemize}

\note[item]{}
\end{frame}
\begin{frame}
\frametitle{Unrelated Title}


\begin{itemize}
\item 



For each p in P and each major
clause ci in Cp, choose minor clauses cj, j != i, so that ci determines p.
TR has two requirements for each ci: ci evaluates to true and ci
evaluates to false. The values chosen for the minor clauses cj must
be the same when ci is true as when ci is false


minor clauses do need to be the sameoften leads to infeasible test requirementsthere is no logical reason for such a restriction
\end{itemize}

\note[item]{}
\end{frame}
\begin{frame}
\frametitle{Unrelated Title}


\begin{itemize}
\item 



For each p in P and each major
clause ci in Cp, choose minor clauses cj, j != i, so that ci determines p.
TR has two requirements for each ci: ci evaluates to true and ci
evaluates to false. The values chosen for the minor clauses cj must
cause p to be true for one value of the major clause ci and false for
the other


the values chosen for the minor clauses must cause p to be true for one value of the major clause and false for the otherallows minor clauses to have different valuessatisfies predicate coverage
\end{itemize}

\note[item]{}
\end{frame}
\begin{frame}
\frametitle{Unrelated Title}


\begin{itemize}
\item weakness of clause coverage: values do not always make difference
\end{itemize}

\note[item]{}
\end{frame}
\begin{frame}
\frametitle{Unrelated Title}


\begin{itemize}
\item 



The values chosen for the minor
clauses cj do not need to be the same when ci is true as when ci is false 



\end{itemize}

\note[item]{}
\end{frame}
\begin{frame}
\frametitle{Unrelated Title}


\begin{itemize}
\item 



The values chosen for the minor
clauses cj must be the same when ci is true as when ci is false 



\end{itemize}

\note[item]{}
\end{frame}
\begin{frame}
\frametitle{Unrelated Title}


\begin{itemize}
\item object diagrams can only define instances of entities defined in class diagram
\end{itemize}

\note[item]{}
\end{frame}
\begin{frame}
\frametitle{Unrelated Title}


\begin{itemize}
\item attributes in statechart diagram must appear in the class representing the subject
\end{itemize}

\note[item]{}
\end{frame}
\begin{frame}
\frametitle{Unrelated Title}


\begin{itemize}
\item instances names used in operation specifications must match those in the object diagram
\end{itemize}

\note[item]{}
\end{frame}
\begin{frame}
\frametitle{Unrelated Title}


\begin{itemize}
\item must agree on states, events and operations
\end{itemize}

\note[item]{}
\end{frame}
\begin{frame}
\frametitle{Unrelated Title}


\begin{itemize}
\item must be deterministic
\end{itemize}

\note[item]{}
\end{frame}
\begin{frame}
\frametitle{Unrelated Title}


\begin{itemize}
\item there must be one for each operationassumes and result clauses must be Boolean expressionsterms in sends, receives, returns and changes clauses should only refer to items mentioned in assumes and result clause
\end{itemize}

\note[item]{}
\end{frame}
\begin{frame}
\frametitle{Unrelated Title}


\begin{itemize}
\item map behavior model to directed graphdefine structural coverage requirements (e.g. PPC)identify DU pairs based on operation specificationsdefine DU-based coverage requirementsmerge into unified graph coverage requirements
\end{itemize}

\note[item]{}
\end{frame}
\begin{frame}
\frametitle{Unrelated Title}


\begin{itemize}
\item clarify assumptions for preconditionsuse system operation specifications to define predicate coverage requirements
\end{itemize}

\note[item]{}
\end{frame}
\begin{frame}
\frametitle{Unrelated Title}


\begin{itemize}
\item use system operations and structural model to create partition modelif BCC is used, pick base choicesdefine input partitioning coverage requirements
\end{itemize}

\note[item]{}
\end{frame}
\begin{frame}
\frametitle{Unrelated Title}

\begin{center}
\includegraphics[width=0.9\textwidth,height=0.9\textheight,keepaspectratio]{/Users/I516998/Library/Application Support/Anki2/User 1/collection.media/Bildschirmfoto 2023-01-29 um 16.33.31.png}
\end{center}


\note[item]{}
\end{frame}
\begin{frame}
\frametitle{Unrelated Title}

\begin{center}
\includegraphics[width=0.9\textwidth,height=0.9\textheight,keepaspectratio]{/Users/I516998/Library/Application Support/Anki2/User 1/collection.media/Bildschirmfoto 2023-01-29 um 16.34.15.png}
\end{center}


\note[item]{}
\end{frame}
\begin{frame}
\frametitle{Unrelated Title}

\begin{center}
\includegraphics[width=0.9\textwidth,height=0.9\textheight,keepaspectratio]{/Users/I516998/Library/Application Support/Anki2/User 1/collection.media/Bildschirmfoto 2023-01-29 um 16.34.45.png}
\end{center}


\note[item]{}
\end{frame}
\begin{frame}
\frametitle{Unrelated Title}

\begin{center}
\includegraphics[width=0.9\textwidth,height=0.9\textheight,keepaspectratio]{/Users/I516998/Library/Application Support/Anki2/User 1/collection.media/Bildschirmfoto 2023-01-29 um 16.35.23.png}
\end{center}


\note[item]{}
\end{frame}
\begin{frame}
\frametitle{Unrelated Title}


\begin{itemize}
\item Frequentist Interpretation (probs are frequencies of events)Subjective Interpretation (probs are subjective degrees of belief)attributing degree possible via betting games
\end{itemize}

\note[item]{}
\end{frame}
\begin{frame}
\frametitle{Unrelated Title}


\begin{itemize}
\item contains empty and trivial (all events)closed under union, complementation, intersection, difference
\end{itemize}

\note[item]{}
\end{frame}
\begin{frame}
\frametitle{Unrelated Title}


\begin{itemize}
\item n! / (k! * (n-k)!) -> nCr auf taschenrechner
\end{itemize}

\note[item]{}
\end{frame}
\begin{frame}
\frametitle{Unrelated Title}

\begin{center}
\includegraphics[width=0.9\textwidth,height=0.9\textheight,keepaspectratio]{/Users/I516998/Library/Application Support/Anki2/User 1/collection.media/Bildschirmfoto 2023-01-29 um 17.04.22.png}
\end{center}


\note[item]{}
\end{frame}
\begin{frame}
\frametitle{Unrelated Title}

\begin{center}
\includegraphics[width=0.9\textwidth,height=0.9\textheight,keepaspectratio]{/Users/I516998/Library/Application Support/Anki2/User 1/collection.media/Bildschirmfoto 2023-01-29 um 17.18.42.png}
\end{center}


\note[item]{}
\end{frame}
\begin{frame}
\frametitle{Unrelated Title}

\begin{center}
\includegraphics[width=0.9\textwidth,height=0.9\textheight,keepaspectratio]{/Users/I516998/Library/Application Support/Anki2/User 1/collection.media/Bildschirmfoto 2023-01-29 um 17.53.36.png}
\end{center}


\note[item]{}
\end{frame}
\begin{frame}
\frametitle{Unrelated Title}

\begin{center}
\includegraphics[width=0.9\textwidth,height=0.9\textheight,keepaspectratio]{/Users/I516998/Library/Application Support/Anki2/User 1/collection.media/Bildschirmfoto 2023-01-29 um 17.57.32.png}
\end{center}

\begin{itemize}
\item 1. alpha * sum over all rest * conditional probs
\item 2. Sort based on sums
\item 3. calculate for both b and NOT b
\item 4. normalize
\end{itemize}

\note[item]{}
\end{frame}
\begin{frame}
\frametitle{Unrelated Title}

\begin{center}
\includegraphics[width=0.9\textwidth,height=0.9\textheight,keepaspectratio]{/Users/I516998/Library/Application Support/Anki2/User 1/collection.media/Bildschirmfoto 2023-01-29 um 18.05.09.png}
\includegraphics[width=0.9\textwidth,height=0.9\textheight,keepaspectratio]{/Users/I516998/Library/Application Support/Anki2/User 1/collection.media/Bildschirmfoto 2023-01-29 um 18.04.36.png}
\end{center}

\begin{itemize}
\item Sum over all that are over the target var!! then all related Ps
\end{itemize}

\note[item]{}
\end{frame}
\begin{frame}
\frametitle{Unrelated Title}


\begin{itemize}
\item start with full joint distribution with sums !! want to eliminate what we dont needassign terms with dependent variables -> f1(A,B,C)1. use evidence = assign values -> f1(A,C) with B set to true, drop P(B)
\item 2. sum over C -> f1(A) with C trues und C false together (just ignore C)
\item 3. multiply (and sum?) -> f1(A) * f2(A,C) on A
\item 4. when only one final left -> normalize!!! 
\end{itemize}

\note[item]{}
\end{frame}
\begin{frame}
\frametitle{Unrelated Title}

\begin{center}
\includegraphics[width=0.9\textwidth,height=0.9\textheight,keepaspectratio]{/Users/I516998/Library/Application Support/Anki2/User 1/collection.media/Bildschirmfoto 2023-01-30 um 10.11.26.png}
\end{center}

\begin{itemize}
\item 1. prediction (prior * transitions) -> sum all possible ways = prob1
\item 2. correction (for each possible outcome: prob1*observationProb OF OBSERVATION / that fOR ALL (normalization)
\end{itemize}

\note[item]{}
\end{frame}
\begin{frame}
\frametitle{Unrelated Title}

\begin{center}
\includegraphics[width=0.9\textwidth,height=0.9\textheight,keepaspectratio]{/Users/I516998/Library/Application Support/Anki2/User 1/collection.media/Bildschirmfoto 2023-01-30 um 10.12.51.png}
\end{center}


\note[item]{}
\end{frame}
\begin{frame}
\frametitle{Unrelated Title}


\begin{itemize}
\item 1. for each possible outcome: find potential state sequences
\item 2. calc sequences with observationProbs and sum them for each outcome
\item 3. finally: normalize all outcomes, choose highest
\end{itemize}

\note[item]{}
\end{frame}
\begin{frame}
\frametitle{Unrelated Title}

\begin{center}
\includegraphics[width=0.9\textwidth,height=0.9\textheight,keepaspectratio]{/Users/I516998/Library/Application Support/Anki2/User 1/collection.media/Bildschirmfoto 2023-01-30 um 10.18.56.png}
\end{center}

\begin{itemize}
\item proportion of samples that are aligned with targetdoes NOT work with evidence
\end{itemize}

\note[item]{}
\end{frame}
\begin{frame}
\frametitle{Unrelated Title}

\begin{center}
\includegraphics[width=0.9\textwidth,height=0.9\textheight,keepaspectratio]{/Users/I516998/Library/Application Support/Anki2/User 1/collection.media/Bildschirmfoto 2023-01-30 um 10.19.57.png}
\end{center}

\begin{itemize}
\item Reject all samples NOT consistent with evidence, then count proportionP(a|n,c)- - requires MANY samples (especially if low prob)
\end{itemize}

\note[item]{}
\end{frame}
\begin{frame}
\frametitle{Unrelated Title}

\begin{center}
\includegraphics[width=0.9\textwidth,height=0.9\textheight,keepaspectratio]{/Users/I516998/Library/Application Support/Anki2/User 1/collection.media/Bildschirmfoto 2023-01-30 um 16.22.52.png}
\end{center}

\begin{itemize}
\item avoids inefficiancy of rejection samplinggenerate only samples consistent with evidenceUSE CONDITIONAL PROBS (table?) OF EVIDENCE ONLY to calc WEIGHT of each sample.. -> want P(a|c,x), generate c,x samples, weights = condProb c* condProb xresult = weight of relevants (a) / all weightse.g. weight = 0.9 * 0.8
\end{itemize}

\note[item]{}
\end{frame}
\begin{frame}
\frametitle{Unrelated Title}


\begin{itemize}
\item traverses a set of states, FIX EVIDENCEgenerate next state by sampling value of ONE var given its markov Blanketreqs -> evidence must be fixed, only one change per iteration
\end{itemize}

\note[item]{}
\end{frame}
\begin{frame}
\frametitle{Unrelated Title}


\begin{itemize}
\item Parents, Children, Childrens Parents
\end{itemize}

\note[item]{}
\end{frame}
\begin{frame}
\frametitle{Unrelated Title}


\begin{itemize}
\item Used if exact inference not feasible (large networks)approximation algorithmdirect sampling (prior, rejection, likelihood)MCMC sampling (gibbs)prior: no evidencerejection: evidence, infeasible if evidence growslikelihood + gibbs insensitive to topology
\end{itemize}

\note[item]{}
\end{frame}
\begin{frame}
\frametitle{Unrelated Title}

\begin{center}
\includegraphics[width=0.9\textwidth,height=0.9\textheight,keepaspectratio]{/Users/I516998/Library/Application Support/Anki2/User 1/collection.media/Bildschirmfoto 2023-01-30 um 10.55.17.png}
\includegraphics[width=0.9\textwidth,height=0.9\textheight,keepaspectratio]{/Users/I516998/Library/Application Support/Anki2/User 1/collection.media/Bildschirmfoto 2023-01-30 um 10.57.03.png}
\end{center}

\begin{itemize}
\item VARIANT 1 (marginalizing out for each CHOICE and UTILITY-DETERMINER, then expected utility)VARIANT 2 (direct just all paths + expected utility
\end{itemize}

\note[item]{}
\end{frame}
\begin{frame}
\frametitle{Unrelated Title}


\begin{itemize}
\item 1. P(pass) = sum over qs of P(pass|q)*P(q) = 0.9* p(q) + 0.2 * P(not q)
\item 2. fail same same
\item 3. then use bayes for all 4 combinations of q | pass / fail
\item no normalization
\end{itemize}

\note[item]{}
\end{frame}
\begin{frame}
\frametitle{Unrelated Title}

\begin{center}
\includegraphics[width=0.9\textwidth,height=0.9\textheight,keepaspectratio]{/Users/I516998/Library/Application Support/Anki2/User 1/collection.media/Bildschirmfoto 2023-01-30 um 11.12.48.png}
\end{center}

\begin{itemize}
\item assumption: test does not cost anything! not considered in utility
\item ALSO CONSIDER TRIVIAL Options (not buy but passed -> 0)
\end{itemize}

\note[item]{}
\end{frame}
\begin{frame}
\frametitle{Unrelated Title}


\begin{itemize}
\item Value of information =
\item EU with information (p(Info=pass) * EU with pass + p(Info=fail) * EU with fail) - EU without info
\item -> if value lower than cost, dont buy!! 
\item -> may have new EU after Info! consider!
\end{itemize}

\note[item]{}
\end{frame}
\begin{frame}
\frametitle{Unrelated Title}

\begin{center}
\includegraphics[width=0.9\textwidth,height=0.9\textheight,keepaspectratio]{/Users/I516998/Library/Application Support/Anki2/User 1/collection.media/Bildschirmfoto 2023-01-30 um 11.47.53.png}
\end{center}

\begin{itemize}
\item -> All possible states! leave 0 ggf
\end{itemize}

\note[item]{}
\end{frame}
\begin{frame}
\frametitle{Unrelated Title}


\begin{itemize}
\item n = number of states that require action
\item a = number of different actions
\item a^n
\end{itemize}

\note[item]{}
\end{frame}
\begin{frame}
\frametitle{Unrelated Title}


\begin{itemize}
\item v(s) = value of state s, initially 0!!
\item r(s) = direct reward of state s
\item a = pi(s) given policy / action
\item ONE SWEEP  = 1 full iteration of all states, start with s=0
\item v(0) = reward(0) + discount * (SUM transitionProbs * their utility-value(initially 0 / REUSE calculated)
\end{itemize}

\note[item]{}
\end{frame}
\begin{frame}
\frametitle{Unrelated Title}


\begin{itemize}
\item Compare current expected utility with all other possible, choose highest
\item Action1 = transitionProb1 * value1 + transitionProb2 * value2 etc . . . . .. .
\item Action2 = transitionProb1 * value4 + transitionProb2 * value7 etc... 
\item etc. . .. . 
\end{itemize}

\note[item]{}
\end{frame}
\begin{frame}
\frametitle{Unrelated Title}


\begin{itemize}
\item FORMULA WAY reward + discount * MAXAction (trans * v)again, set all v(s)= 0v(0) = reward + discount * maxAction (SUM all transitionProbs * v of that state)REUSE within iterationwould do multiple iterations until only marginal changesOTHER FORMAT - no smart MAX, first everything and then choose max result
\end{itemize}

\note[item]{}
\end{frame}
\begin{frame}
\frametitle{Unrelated Title}


\begin{itemize}
\item smaller gamma -> smaller utility values + diffs -> more focus on direct rewards
\item larger gamma -> larger """ -> more focus on future states -> REQ MORE ITERATIONS 
\end{itemize}

\note[item]{}
\end{frame}
\begin{frame}
\frametitle{Unrelated Title}

\begin{center}
\includegraphics[width=0.9\textwidth,height=0.9\textheight,keepaspectratio]{/Users/I516998/Library/Application Support/Anki2/User 1/collection.media/Bildschirmfoto 2023-01-30 um 12.42.16.png}
\end{center}

\begin{itemize}
\item new Q value = old value + alpha (step size) * (target = FUNCTION(reward + discount * MaxPrevQvalue=max discounted reward when in s' - old value))
\end{itemize}

\note[item]{}
\end{frame}
\begin{frame}
\frametitle{Unrelated Title}

\begin{center}
\includegraphics[width=0.9\textwidth,height=0.9\textheight,keepaspectratio]{/Users/I516998/Library/Application Support/Anki2/User 1/collection.media/Bildschirmfoto 2023-01-30 um 12.46.33.png}
\end{center}

\begin{itemize}
\item smaller alpha values -> more trust in old estimations
\item larger alpha values -> more exploration + more trust in new estimations 
\end{itemize}

\note[item]{}
\end{frame}
\begin{frame}
\frametitle{Unrelated Title}


\begin{itemize}
\item used when transition probs/rewards are not known (vs. value/policy iteration / MDP)exploring different combinations vs. exploiting newly gained knowledge two parameters, e = probability to explore (start with 1) and EEE = factor of exploring in next game/round ... e*EEE = new e, 1-e probability to exploit
\end{itemize}

\note[item]{}
\end{frame}
\begin{frame}
\frametitle{Unrelated Title}


\begin{itemize}
\item ___|___P2________ P2 
\item P1 | (P1, P2). . . . . (P1, P2)
\item P1 | etc. 
\end{itemize}

\note[item]{}
\end{frame}
\begin{frame}
\frametitle{Unrelated Title}


\begin{itemize}
\item Dominant Strategy - best no matter what the other player chooses (-> One LINE/COLUMN better in every case?)
\item Nash Equilibria - steady state, no player would change action (can be more)
\end{itemize}

\note[item]{}
\end{frame}
\begin{frame}
\frametitle{Unrelated Title}

\begin{center}
\includegraphics[width=0.9\textwidth,height=0.9\textheight,keepaspectratio]{/Users/I516998/Library/Application Support/Anki2/User 1/collection.media/Bildschirmfoto 2023-02-01 um 15.03.28.png}
\includegraphics[width=0.9\textwidth,height=0.9\textheight,keepaspectratio]{/Users/I516998/Library/Application Support/Anki2/User 1/collection.media/Bildschirmfoto 2023-02-01 um 15.03.37.png}
\end{center}


\note[item]{}
\end{frame}
\begin{frame}
\frametitle{Unrelated Title}


\begin{itemize}
\item if P(X|Y) = P(X) or P(X,Y) = P(X) * P(Y)
\item if P(X|Y,Z) = P(X|Z)
\end{itemize}

\note[item]{}
\end{frame}
\begin{frame}
\frametitle{Unrelated Title}


\begin{itemize}
\item data structure repesenting any full joint prob distributionno directed cyclescauses to symptoms (causal)node Xi has conditional prob distr P(Xi | Parents(Xi))
\end{itemize}

\note[item]{}
\end{frame}
\begin{frame}
\frametitle{Unrelated Title}


\begin{itemize}
\item Inference by Enumeration (SUMMING OUT hidden vars with cond probs, alpha!!) O(n*2^n)Inference by Var Elimination (reuse Factors) O(n) if single connected, otherwise O(2^n)
\end{itemize}

\note[item]{}
\end{frame}
\begin{frame}
\frametitle{Unrelated Title}


\begin{itemize}
\item transition model P(Xt|Xt-1)
\item observation model P(Yt|Xt)
\item prior distribution P(Xt=1)
\end{itemize}

\note[item]{}
\end{frame}
\begin{frame}
\frametitle{Unrelated Title}


\begin{itemize}
\item probablistic graphical models are limited (compact distribitions only, infeasible inference -> exponential treewidth)with SPNs -> inference guaranteed to be polynomial complexity
\end{itemize}

\note[item]{}
\end{frame}
\begin{frame}
\frametitle{Unrelated Title}

\begin{center}
\includegraphics[width=0.9\textwidth,height=0.9\textheight,keepaspectratio]{/Users/I516998/Library/Application Support/Anki2/User 1/collection.media/Bildschirmfoto 2023-02-02 um 16.12.12.png}
\end{center}


\note[item]{}
\end{frame}
\begin{frame}
\frametitle{Unrelated Title}


\begin{itemize}
\item Complete: for every Sum-Node -> children have the same scopeDecomposable: for every Product-Node -> unidentical children have disjoint scopesvalid: complete and decomposable
\end{itemize}

\note[item]{}
\end{frame}
\begin{frame}
\frametitle{Unrelated Title}


\begin{itemize}
\item Complete Evidence(Probabilities for vars)Product for product nodesWEIGHTED sum for sum nodesMarginal InferenceSame but set other Vars to 1
\end{itemize}

\note[item]{}
\end{frame}
\begin{frame}
\frametitle{Unrelated Title}


\begin{itemize}
\item start with data matrix (column -> random var, row = samples)split columns and create PRODUCT node if independence foundsplit rows and create SUM node with weights of cluster sizes with cluster algorithm if < n samples -> product node with all remaining vars as children
\end{itemize}

\note[item]{}
\end{frame}
\begin{frame}
\frametitle{Unrelated Title}

\begin{center}
\includegraphics[width=0.9\textwidth,height=0.9\textheight,keepaspectratio]{/Users/I516998/Library/Application Support/Anki2/User 1/collection.media/Bildschirmfoto 2023-02-02 um 16.27.18.png}
\end{center}

\begin{itemize}
\item Depends on agents utility function U(s)Choose action with max EU
\end{itemize}

\note[item]{}
\end{frame}
\begin{frame}
\frametitle{Unrelated Title}


\begin{itemize}
\item Chance Nodes (oval) = standard random vars ( bayesian network)
\item Decision Nodes (rectangles) = choices
\item Utility (diamonds) = utility function
\end{itemize}

\note[item]{}
\end{frame}
\begin{frame}
\frametitle{Unrelated Title}


\begin{itemize}
\item Orderability (1 of 3 holds)Transitivity (wenn 2x kleiner, auch 1 und 3 kleiner)Continuity (p zw A und C sodass gleich B)Substitutability (wenn a&b gleich viel wert, austauschbar in lottery)MonotonicityDecomposability
\end{itemize}

\note[item]{}
\end{frame}
\begin{frame}
\frametitle{Unrelated Title}


\begin{itemize}
\item L = [p, A; 1-p, B]
\end{itemize}

\note[item]{}
\end{frame}
\begin{frame}
\frametitle{Unrelated Title}


\begin{itemize}
\item design standard lottory with [p, great outcome; 1-p, worst outcome]start with p = 0, shift towards 1as soon as people are indifferent between S and lottory -> p is utility
\end{itemize}

\note[item]{}
\end{frame}
\begin{frame}
\frametitle{Unrelated Title}


\begin{itemize}
\item Allais -> people prefer certain gaines (100% chance)Ellsbergs -> ambiguity aversion and choose lotteries with kn own chances
\end{itemize}

\note[item]{}
\end{frame}
\begin{frame}
\frametitle{Unrelated Title}


\begin{itemize}
\item Vpi(s) is expected discounted reward of following policy pi in state s
\item Qpi(s, a) is total expected discounted reward value of doing a in state s then following policy pi
\end{itemize}

\note[item]{}
\end{frame}
\begin{frame}
\frametitle{Unrelated Title}


\begin{itemize}
\item 



Identification of the purpose of the operation, followed by an
informal description of the normal and exceptional effects. 



\end{itemize}

\note[item]{}
\end{frame}
\begin{frame}
\frametitle{Unrelated Title}


\begin{itemize}
\item 



Properties that constrain the realization and implementation of the
system. 



\end{itemize}

\note[item]{}
\end{frame}
\begin{frame}
\frametitle{Unrelated Title}


\begin{itemize}
\item 



Information input to the operation by the invoker. 



\end{itemize}

\note[item]{}
\end{frame}
\begin{frame}
\frametitle{Unrelated Title}


\begin{itemize}
\item Information returned to the invoker by the operation.
\end{itemize}

\note[item]{}
\end{frame}
\begin{frame}
\frametitle{Unrelated Title}


\begin{itemize}
\item 



Signals which the operation sends to imported systems. These can
be events or operation invocations. 



\end{itemize}

\note[item]{}
\end{frame}
\begin{frame}
\frametitle{Unrelated Title}


\begin{itemize}
\item 



Externally visible information accessed by the operation. 



\end{itemize}

\note[item]{}
\end{frame}
\begin{frame}
\frametitle{Unrelated Title}


\begin{itemize}
\item 



Externally visible information changed by the operation. 



\end{itemize}

\note[item]{}
\end{frame}
\begin{frame}
\frametitle{Unrelated Title}


\begin{itemize}
\item 



Precondition on the externally visible state of the system and on
the inputs (in receives clause) that must be true for the system to
guarantee the post condition (result clause). 



\end{itemize}

\note[item]{}
\end{frame}
\begin{frame}
\frametitle{Unrelated Title}


\begin{itemize}
\item 



Strongest post condition on the externally visible properties of the
system and the returned entities (returns clause) that become true
after execution of the operation with true assumes clause. 



\end{itemize}

\note[item]{}
\end{frame}
\begin{frame}
\frametitle{Unrelated Title}


\begin{itemize}
\item Learn model from observation samples (unknown world)sequence of transition samples (s, a, r, s')collect new state and actionP ss' a (proportion of samples where we end up in s' from s with a / all from s with a)R ss' a (average reward with s,a,s')O(N^2 * A)
\end{itemize}

\note[item]{}
\end{frame}
\begin{frame}
\frametitle{Unrelated Title}

\begin{center}
\includegraphics[width=0.9\textwidth,height=0.9\textheight,keepaspectratio]{/Users/I516998/Library/Application Support/Anki2/User 1/collection.media/Bildschirmfoto 2023-02-03 um 09.55.45.png}
\end{center}


\note[item]{}
\end{frame}
\begin{frame}
\frametitle{Unrelated Title}


\begin{itemize}
\item Pure: always same choice, deterministicMixed: probability distr over choices, player assigns a prob to each choice
\end{itemize}

\note[item]{}
\end{frame}
\begin{frame}
\frametitle{Unrelated Title}


\begin{itemize}
\item MaxMin -> Fix P2 for every P2 choice, set up linear function with P1 payouts and P1 probsgleichsetzen -> p, einsetzen = expected Utility (0?)kurve UNTEN, höhepunkt = intersectionMinMax -> Fix P1for every P1 choice, set up linear function with P1 payouts AND P2 PROBSgleichsetzen -> q, einsetzen = expected Utility (0?)kurve OBEN, tiefpunkt = intersection
\end{itemize}

\note[item]{}
\end{frame}
\begin{frame}
\frametitle{Unrelated Title}


\begin{itemize}
\item both refuse until other testifies once, then only testifystart with refuse and then always mimick other player
\end{itemize}

\note[item]{}
\end{frame}
\begin{frame}
\frametitle{Unrelated Title}


\begin{itemize}
\item true language understandingintegration of prior and learned knowledgelarge time scale planningconceptual breakthroughs
\end{itemize}

\note[item]{}
\end{frame}
\begin{frame}
\frametitle{Unrelated Title}


\begin{itemize}
\item + + civilization/weath = result of intelligence -> more intelligence, more civilization/wealth
\item - - difficult to understand (human vs. gorillas)
\end{itemize}

\note[item]{}
\end{frame}
\begin{frame}
\frametitle{Unrelated Title}


\begin{itemize}
\item Wrong Goal implementedpositive incentive to prevent being switched off
\end{itemize}

\note[item]{}
\end{frame}
\begin{frame}
\frametitle{Unrelated Title}


\begin{itemize}
\item AI: intelligent, to the extent that their actions lead to achievement of their goalsbetter: beneficial, to the extent that their actions lead to achievement of OUR goalsapproach: human behavior CAN influence machine behavior, used for inference of human goal (unknown)
\end{itemize}

\note[item]{}
\end{frame}
\begin{frame}
\frametitle{Unrelated Title}


\begin{itemize}
\item Robot has choice: ACTION + reward (+9999 or -9999 reward) or shut down (0 reward) -> SMART to waitHuman as choice: nothing or shut downRobot now knows, if i am still on, reward is ATLEAST 0 :) inferes this by human behavior
\end{itemize}

\note[item]{}
\end{frame}
\begin{frame}
\frametitle{Unrelated Title}


\begin{itemize}
\item limited computational poweremotional behavioruncertainty over and variability of preferences
\end{itemize}

\note[item]{}
\end{frame}
\begin{frame}
\frametitle{Unrelated Title}

\begin{center}
\includegraphics[width=0.9\textwidth,height=0.9\textheight,keepaspectratio]{/Users/I516998/Library/Application Support/Anki2/User 1/collection.media/Bildschirmfoto 2023-02-04 um 12.36.16.png}
\includegraphics[width=0.9\textwidth,height=0.9\textheight,keepaspectratio]{/Users/I516998/Library/Application Support/Anki2/User 1/collection.media/Bildschirmfoto 2023-02-04 um 12.37.11.png}
\end{center}

\begin{itemize}
\item pride and envy are mathematically identical to sadism, ignoring them might not be sensible (fundamental)
\end{itemize}

\note[item]{}
\end{frame}
\begin{frame}
\frametitle{Unrelated Title}


\begin{itemize}
\item Eavesdropping dataData is forwarded between several servers/hopsManipulating dataImpersonation
\end{itemize}

\note[item]{}
\end{frame}
\begin{frame}
\frametitle{Unrelated Title}


\begin{itemize}
\item Establish communicationCheck identityAgree on common secret keyEncrpyted communication
\end{itemize}

\note[item]{}
\end{frame}
\begin{frame}
\frametitle{Unrelated Title}


\begin{itemize}
\item TLs runs on top of http to provide a secure connection
\end{itemize}

\note[item]{}
\end{frame}
\begin{frame}
\frametitle{Unrelated Title}

\begin{center}
\includegraphics[width=0.9\textwidth,height=0.9\textheight,keepaspectratio]{/Users/I516998/Library/Application Support/Anki2/User 1/collection.media/paste-ecab948a61c383dcc55ed1a44d6d7fed5d2a5b95.jpg}
\end{center}

\begin{itemize}
\item Generate a secret keyUse the key to encrpyt the messageTransfer only the ciphertext
\end{itemize}

\note[item]{}
\end{frame}
\begin{frame}
\frametitle{Unrelated Title}


\begin{itemize}
\item Three sets:
\item Key space KMessage space MCiphertext space CThree algorithms:Key generationEncryption algorithm Enc: K x M -> CDecryption algorithm Dec: K x C -> MThe scheme has to fulfill the correctness requirement: For each message m in M and each k in K, it holds that Dec(k, Enc(k,m)) = m
\end{itemize}

\note[item]{}
\end{frame}
\begin{frame}
\frametitle{Unrelated Title}


\begin{itemize}
\item Public Key Encrpytion, e.g. RSAKey agreement protocol, e.g. Diffie-Hellman-protocol
\end{itemize}

\note[item]{}
\end{frame}
\begin{frame}
\frametitle{Unrelated Title}

\begin{center}
\includegraphics[width=0.9\textwidth,height=0.9\textheight,keepaspectratio]{/Users/I516998/Library/Application Support/Anki2/User 1/collection.media/paste-4fdfea194f5d6151baf41dea6307b41b4f01ce8d.jpg}
\end{center}

\begin{itemize}
\item Public key is used for encryptionPrivate key is used for decryptionFormal definition is then the same as for symmetric encryption
\end{itemize}

\note[item]{}
\end{frame}
\begin{frame}
\frametitle{Unrelated Title}


\begin{itemize}
\item Symmetric key encryption is usually more efficient
\end{itemize}

\note[item]{}
\end{frame}
\begin{frame}
\frametitle{Unrelated Title}

\begin{center}
\includegraphics[width=0.9\textwidth,height=0.9\textheight,keepaspectratio]{/Users/I516998/Library/Application Support/Anki2/User 1/collection.media/paste-78dd0539efe3cc97130ed2c36c914ec75e5d454d.jpg}
\includegraphics[width=0.9\textwidth,height=0.9\textheight,keepaspectratio]{/Users/I516998/Library/Application Support/Anki2/User 1/collection.media/paste-cd7fd5b57d3cca73cbd3eed6d1298de0d98e3bdd.jpg}
\end{center}

\begin{itemize}
\item At the beginning no common secretCommunicate over a public channel and aim for common secret keyKey k should remain secret even in presence of an eavesdropperBoth parties chose a local secret to communicate secret k
\end{itemize}

\note[item]{}
\end{frame}
\begin{frame}
\frametitle{Unrelated Title}

\begin{center}
\includegraphics[width=0.9\textwidth,height=0.9\textheight,keepaspectratio]{/Users/I516998/Library/Application Support/Anki2/User 1/collection.media/paste-a943b316c7b9abc6162363ae3f71e58cdf6ba372.jpg}
\end{center}

\begin{itemize}
\item Signature algorithm: SK x m -> SVerification algorithm: PK x M x S -> True/FalseCorrectness is ensured if Verify(pk,m,Sign(sk,m)) = True
\end{itemize}

\note[item]{}
\end{frame}
\begin{frame}
\frametitle{Unrelated Title}


\begin{itemize}
\item Authenticity: Ensure the source of dataNon-repudiation: The sender of a message cannot deny the creation of the messageIntegrity: Messages have not been modified in transit
\end{itemize}

\note[item]{}
\end{frame}
\begin{frame}
\frametitle{Unrelated Title}


\begin{itemize}
\item A server authenticates by providing a certificateCertificate contains information about a public keyThe certificate is signed by a trusted third partyThird parties are called Certificate Authorities (CA)
\end{itemize}

\note[item]{}
\end{frame}
\begin{frame}
\frametitle{Unrelated Title}


\begin{itemize}
\item German consitution (Artikel 10 Grundgesetz)Fundamental right to secrecy of communicationOnly applies to information in transitLawful interception (Telekommunikationsüberwachung)Regulates interception of communicationRequires warrant and probable causeCounters encrpyted communication (e.g. through federal trojan)
\end{itemize}

\note[item]{}
\end{frame}
\begin{frame}
\frametitle{Unrelated Title}


\begin{itemize}
\item Law that regulates electronic surveillanceFISA court:Without court order it's only allowed foreign intelligence information
\end{itemize}

\note[item]{}
\end{frame}
\begin{frame}
\frametitle{Unrelated Title}


\begin{itemize}
\item Secret court rders allow NSA to collect phone recordsNSA spies on foreign countries and word leadersNSA tried to crack encryption protcolsPRISM project: collect data with help of companies
\end{itemize}

\note[item]{}
\end{frame}
\begin{frame}
\frametitle{Unrelated Title}


\begin{itemize}
\item Beziehung von 2 Organisationsmitgliedern ist asymmetrisch strukturiert, d.h. der Vorgesetzte kann auf Machtpotentiale zugreifen, die dem Mitarbeiter nicht zur Verfügung stehen
\end{itemize}

\note[item]{}
\end{frame}
\begin{frame}
\frametitle{Unrelated Title}


\begin{itemize}
\item Grundidee: Führungserfolg abhängig von bestimmten Eigenschaften - angeboren oder erworben
\item Früher vor allem physische Eigenschaften
\item Heute vor allem psychische Führungseigenschaften:
\item -Emphatisch, Stressresistent, Konsequent, Intelligenz
\end{itemize}

\note[item]{}
\end{frame}
\begin{frame}
\frametitle{Unrelated Title}


\begin{itemize}
\item Individuelle Eigenschaften sind wichtig, aber auch Gruppenverhalten und situative Bedingungen relevant
\item Die meisten MA sind gleichzeitig Vorgesetzte und Untergebene
\item -> Theory betrachtet Führungsperson aber nicht Führungssituation
\end{itemize}

\note[item]{}
\end{frame}
\begin{frame}
\frametitle{Unrelated Title}

\begin{center}
\includegraphics[width=0.9\textwidth,height=0.9\textheight,keepaspectratio]{/Users/I516998/Library/Application Support/Anki2/User 1/collection.media/img4079933088116596153.jpg}
\end{center}


\note[item]{}
\end{frame}
\begin{frame}
\frametitle{Unrelated Title}


\begin{itemize}
\item Nicht mehr Eigenschaften der Führungskraft entscheidend, sondern ihr persönlicher Führungsstil
\item Fokus auf Beziehung zwischen Führenden und Geführten
\item Mitarbeiter in seinem Umfeld -> Kollegen, Vorgesetzte, Untergebene
\item Besondere Aufmerksamkeit: Verhalten von Gruppenmitgliedern bei unterschiedlichen Führungsstilen
\item Ziel: Ermittlung Führungsstil mit besten Resultaten beim Mitarbeiter
\end{itemize}

\note[item]{}
\end{frame}
\begin{frame}
\frametitle{Unrelated Title}


\begin{itemize}
\item Konkrete Situationen werden nicht berücksichtigt
\end{itemize}

\note[item]{}
\end{frame}
\begin{frame}
\frametitle{Unrelated Title}

\begin{center}
\includegraphics[width=0.9\textwidth,height=0.9\textheight,keepaspectratio]{/Users/I516998/Library/Application Support/Anki2/User 1/collection.media/img8420023671091990483.jpg}
\end{center}


\note[item]{}
\end{frame}
\begin{frame}
\frametitle{Unrelated Title}

\begin{center}
\includegraphics[width=0.9\textwidth,height=0.9\textheight,keepaspectratio]{/Users/I516998/Library/Application Support/Anki2/User 1/collection.media/img807959781689941209.jpg}
\includegraphics[width=0.9\textwidth,height=0.9\textheight,keepaspectratio]{/Users/I516998/Library/Application Support/Anki2/User 1/collection.media/img7960803321639598579.jpg}
\end{center}


\note[item]{}
\end{frame}
\begin{frame}
\frametitle{Unrelated Title}

\begin{center}
\includegraphics[width=0.9\textwidth,height=0.9\textheight,keepaspectratio]{/Users/I516998/Library/Application Support/Anki2/User 1/collection.media/img3307622951431745582.jpg}
\end{center}


\note[item]{}
\end{frame}
\begin{frame}
\frametitle{Unrelated Title}


\begin{itemize}
\item Charakteristika des Vorgesetzten:
\item - Wertesystem
\item - Führungsqualitäten
\item - Vertrauen in die Mitarbeiter
\item Charakteristika des Mitarbeiter:
\item - Fachliche Kompetenz und Erfahrung
\item -Engagement für das Problem
\item - Ansprüche bzgl. beruflicher/ persönlicher Entwicklung
\item Charakteristika der Situation:
\item - Art der Organisation
\item - Eigenschaften der Gruppe
\end{itemize}

\note[item]{}
\end{frame}
\begin{frame}
\frametitle{Unrelated Title}

\begin{center}
\includegraphics[width=0.9\textwidth,height=0.9\textheight,keepaspectratio]{/Users/I516998/Library/Application Support/Anki2/User 1/collection.media/img3563727479963715995.jpg}
\end{center}

\begin{itemize}
\item Kriterium: Partizipation in Entscheidungssituationen
\item Von autoritär bis demokratisch
\item + Real beibachtbares Führungsverhalten
\item - zu eindimensional, da nur das Verhaltensmerkmal Partizipation berücksichtigt wird
\end{itemize}

\note[item]{}
\end{frame}
\begin{frame}
\frametitle{Unrelated Title}

\begin{center}
\includegraphics[width=0.9\textwidth,height=0.9\textheight,keepaspectratio]{/Users/I516998/Library/Application Support/Anki2/User 1/collection.media/img8523554459944908845.jpg}
\includegraphics[width=0.9\textwidth,height=0.9\textheight,keepaspectratio]{/Users/I516998/Library/Application Support/Anki2/User 1/collection.media/img7387305191732861610.jpg}
\end{center}


\note[item]{}
\end{frame}
\begin{frame}
\frametitle{Unrelated Title}

\begin{center}
\includegraphics[width=0.9\textwidth,height=0.9\textheight,keepaspectratio]{/Users/I516998/Library/Application Support/Anki2/User 1/collection.media/img7424717825182870842.jpg}
\end{center}


\note[item]{}
\end{frame}
\begin{frame}
\frametitle{Unrelated Title}

\begin{center}
\includegraphics[width=0.9\textwidth,height=0.9\textheight,keepaspectratio]{/Users/I516998/Library/Application Support/Anki2/User 1/collection.media/img4812587603710560804.jpg}
\end{center}


\note[item]{}
\end{frame}
\begin{frame}
\frametitle{Unrelated Title}

\begin{center}
\includegraphics[width=0.9\textwidth,height=0.9\textheight,keepaspectratio]{/Users/I516998/Library/Application Support/Anki2/User 1/collection.media/img4366320952469223561.jpg}
\end{center}


\note[item]{}
\end{frame}
\begin{frame}
\frametitle{Unrelated Title}

\begin{center}
\includegraphics[width=0.9\textwidth,height=0.9\textheight,keepaspectratio]{/Users/I516998/Library/Application Support/Anki2/User 1/collection.media/img394746988988963405.jpg}
\end{center}


\note[item]{}
\end{frame}
\begin{frame}
\frametitle{Unrelated Title}

\begin{center}
\includegraphics[width=0.9\textwidth,height=0.9\textheight,keepaspectratio]{/Users/I516998/Library/Application Support/Anki2/User 1/collection.media/img295896826956620220.jpg}
\end{center}


\note[item]{}
\end{frame}
\begin{frame}
\frametitle{Unrelated Title}

\begin{center}
\includegraphics[width=0.9\textwidth,height=0.9\textheight,keepaspectratio]{/Users/I516998/Library/Application Support/Anki2/User 1/collection.media/img8572884421289393189.jpg}
\includegraphics[width=0.9\textwidth,height=0.9\textheight,keepaspectratio]{/Users/I516998/Library/Application Support/Anki2/User 1/collection.media/img2265042845515365257.jpg}
\includegraphics[width=0.9\textwidth,height=0.9\textheight,keepaspectratio]{/Users/I516998/Library/Application Support/Anki2/User 1/collection.media/img6342221131236279820.jpg}
\end{center}


\note[item]{}
\end{frame}
\begin{frame}
\frametitle{Unrelated Title}

\begin{center}
\includegraphics[width=0.9\textwidth,height=0.9\textheight,keepaspectratio]{/Users/I516998/Library/Application Support/Anki2/User 1/collection.media/img5240834646208121505.jpg}
\end{center}


\note[item]{}
\end{frame}
\begin{frame}
\frametitle{Unrelated Title}

\begin{center}
\includegraphics[width=0.9\textwidth,height=0.9\textheight,keepaspectratio]{/Users/I516998/Library/Application Support/Anki2/User 1/collection.media/img1166407564181507573.jpg}
\includegraphics[width=0.9\textwidth,height=0.9\textheight,keepaspectratio]{/Users/I516998/Library/Application Support/Anki2/User 1/collection.media/img2225391585954926665.jpg}
\end{center}


\note[item]{}
\end{frame}
\begin{frame}
\frametitle{Unrelated Title}


\begin{itemize}
\item Ursachen:
\item - Verhaltensweisen
\item - Machtkämpfe
\item - Antipathie
\item Wirkungen: ...
\end{itemize}

\note[item]{}
\end{frame}
\begin{frame}
\frametitle{Unrelated Title}

\begin{center}
\includegraphics[width=0.9\textwidth,height=0.9\textheight,keepaspectratio]{/Users/I516998/Library/Application Support/Anki2/User 1/collection.media/img5024543253194916568.jpg}
\end{center}


\note[item]{}
\end{frame}
\begin{frame}
\frametitle{Unrelated Title}


\begin{itemize}
\item Übereinstimmung von Unternehmensinteresse und Mitarbeiterinteresse
\end{itemize}

\note[item]{}
\end{frame}
\begin{frame}
\frametitle{Unrelated Title}


\begin{itemize}
\item Kennzeichen für generellen Mangelzustand -> Handlungsbereitschaft des Individuums
\item Teilweise genetisch festgelegt
\item Teilweise erlernt
\item Unterliegen kulturellem Einfluss
\end{itemize}

\note[item]{}
\end{frame}
\begin{frame}
\frametitle{Unrelated Title}


\begin{itemize}
\item In der gleichen Situation verhalten sich Menschen unterschiedlich 
\item In verschiedenen Situationen verhält sich eine Person nicht unterschiedlich, sondern gleich
\item Ein bestimmtes Verhalten wird auch gegen erhebliche Widerstände durchgesetzt
\end{itemize}

\note[item]{}
\end{frame}
\begin{frame}
\frametitle{Unrelated Title}

\begin{center}
\includegraphics[width=0.9\textwidth,height=0.9\textheight,keepaspectratio]{/Users/I516998/Library/Application Support/Anki2/User 1/collection.media/img7563718182746965637.jpg}
\end{center}


\note[item]{}
\end{frame}
\begin{frame}
\frametitle{Unrelated Title}

\begin{center}
\includegraphics[width=0.9\textwidth,height=0.9\textheight,keepaspectratio]{/Users/I516998/Library/Application Support/Anki2/User 1/collection.media/img1970006901952693998.jpg}
\end{center}


\note[item]{}
\end{frame}
\begin{frame}
\frametitle{Unrelated Title}

\begin{center}
\includegraphics[width=0.9\textwidth,height=0.9\textheight,keepaspectratio]{/Users/I516998/Library/Application Support/Anki2/User 1/collection.media/img2666972517455349087.jpg}
\end{center}


\note[item]{}
\end{frame}
\begin{frame}
\frametitle{Unrelated Title}

\begin{center}
\includegraphics[width=0.9\textwidth,height=0.9\textheight,keepaspectratio]{/Users/I516998/Library/Application Support/Anki2/User 1/collection.media/img1844434170634361061.jpg}
\includegraphics[width=0.9\textwidth,height=0.9\textheight,keepaspectratio]{/Users/I516998/Library/Application Support/Anki2/User 1/collection.media/img6180989447104983762.jpg}
\includegraphics[width=0.9\textwidth,height=0.9\textheight,keepaspectratio]{/Users/I516998/Library/Application Support/Anki2/User 1/collection.media/img5089068217936499399.jpg}
\end{center}


\note[item]{}
\end{frame}
\begin{frame}
\frametitle{Unrelated Title}

\begin{center}
\includegraphics[width=0.9\textwidth,height=0.9\textheight,keepaspectratio]{/Users/I516998/Library/Application Support/Anki2/User 1/collection.media/img8168960627413803901.jpg}
\end{center}


\note[item]{}
\end{frame}
\begin{frame}
\frametitle{Unrelated Title}

\begin{center}
\includegraphics[width=0.9\textwidth,height=0.9\textheight,keepaspectratio]{/Users/I516998/Library/Application Support/Anki2/User 1/collection.media/img632669422270913131.jpg}
\end{center}


\note[item]{}
\end{frame}
\begin{frame}
\frametitle{Unrelated Title}

\begin{center}
\includegraphics[width=0.9\textwidth,height=0.9\textheight,keepaspectratio]{/Users/I516998/Library/Application Support/Anki2/User 1/collection.media/img4033745349226268570.jpg}
\end{center}


\note[item]{}
\end{frame}
\begin{frame}
\frametitle{Unrelated Title}

\begin{center}
\includegraphics[width=0.9\textwidth,height=0.9\textheight,keepaspectratio]{/Users/I516998/Library/Application Support/Anki2/User 1/collection.media/img2205843441576543087.jpg}
\end{center}


\note[item]{}
\end{frame}
\begin{frame}
\frametitle{Unrelated Title}

\begin{center}
\includegraphics[width=0.9\textwidth,height=0.9\textheight,keepaspectratio]{/Users/I516998/Library/Application Support/Anki2/User 1/collection.media/img4955501983881624075.jpg}
\end{center}


\note[item]{}
\end{frame}
\begin{frame}
\frametitle{Unrelated Title}

\begin{center}
\includegraphics[width=0.9\textwidth,height=0.9\textheight,keepaspectratio]{/Users/I516998/Library/Application Support/Anki2/User 1/collection.media/img5087973709685647224.jpg}
\end{center}


\note[item]{}
\end{frame}
\begin{frame}
\frametitle{Unrelated Title}

\begin{center}
\includegraphics[width=0.9\textwidth,height=0.9\textheight,keepaspectratio]{/Users/I516998/Library/Application Support/Anki2/User 1/collection.media/img572932084152350471.jpg}
\end{center}


\note[item]{}
\end{frame}
\begin{frame}
\frametitle{Unrelated Title}


\begin{itemize}
\item - Kommunizieren von Unternehmenszielen
\item - Frühzeitige Intervention bei Konflikten
\item - "objektiver" als einseitige Beurteilung
\item - Potenzialeinschätzung
\item - Höhere Mitarbeiterzufriedenheit
\end{itemize}

\note[item]{}
\end{frame}
\begin{frame}
\frametitle{Unrelated Title}

\begin{center}
\includegraphics[width=0.9\textwidth,height=0.9\textheight,keepaspectratio]{/Users/I516998/Library/Application Support/Anki2/User 1/collection.media/img5119553164663321126.jpg}
\end{center}


\note[item]{}
\end{frame}
\begin{frame}
\frametitle{Unrelated Title}

\begin{center}
\includegraphics[width=0.9\textwidth,height=0.9\textheight,keepaspectratio]{/Users/I516998/Library/Application Support/Anki2/User 1/collection.media/img6553902016469141756.jpg}
\end{center}


\note[item]{}
\end{frame}
\begin{frame}
\frametitle{Unrelated Title}

\begin{center}
\includegraphics[width=0.9\textwidth,height=0.9\textheight,keepaspectratio]{/Users/I516998/Library/Application Support/Anki2/User 1/collection.media/img1300489607124405770.jpg}
\end{center}


\note[item]{}
\end{frame}
\begin{frame}
\frametitle{Unrelated Title}

\begin{center}
\includegraphics[width=0.9\textwidth,height=0.9\textheight,keepaspectratio]{/Users/I516998/Library/Application Support/Anki2/User 1/collection.media/img1913854068513466310.jpg}
\end{center}


\note[item]{}
\end{frame}
\begin{frame}
\frametitle{Unrelated Title}

\begin{center}
\includegraphics[width=0.9\textwidth,height=0.9\textheight,keepaspectratio]{/Users/I516998/Library/Application Support/Anki2/User 1/collection.media/img3488154690458807905.jpg}
\end{center}


\note[item]{}
\end{frame}
\begin{frame}
\frametitle{Unrelated Title}

\begin{center}
\includegraphics[width=0.9\textwidth,height=0.9\textheight,keepaspectratio]{/Users/I516998/Library/Application Support/Anki2/User 1/collection.media/img6213268564838646614.jpg}
\end{center}


\note[item]{}
\end{frame}
\begin{frame}
\frametitle{Unrelated Title}

\begin{center}
\includegraphics[width=0.9\textwidth,height=0.9\textheight,keepaspectratio]{/Users/I516998/Library/Application Support/Anki2/User 1/collection.media/img1157787941094978035.jpg}
\end{center}


\note[item]{}
\end{frame}
\begin{frame}
\frametitle{Unrelated Title}

\begin{center}
\includegraphics[width=0.9\textwidth,height=0.9\textheight,keepaspectratio]{/Users/I516998/Library/Application Support/Anki2/User 1/collection.media/img8222217253768808903.jpg}
\end{center}


\note[item]{}
\end{frame}
\begin{frame}
\frametitle{Unrelated Title}

\begin{center}
\includegraphics[width=0.9\textwidth,height=0.9\textheight,keepaspectratio]{/Users/I516998/Library/Application Support/Anki2/User 1/collection.media/img4390904293988560039.jpg}
\end{center}


\note[item]{}
\end{frame}
\begin{frame}
\frametitle{Unrelated Title}

\begin{center}
\includegraphics[width=0.9\textwidth,height=0.9\textheight,keepaspectratio]{/Users/I516998/Library/Application Support/Anki2/User 1/collection.media/img7330068746680850088.jpg}
\end{center}


\note[item]{}
\end{frame}
\begin{frame}
\frametitle{Unrelated Title}

\begin{center}
\includegraphics[width=0.9\textwidth,height=0.9\textheight,keepaspectratio]{/Users/I516998/Library/Application Support/Anki2/User 1/collection.media/img2072914604036040733.jpg}
\end{center}


\note[item]{}
\end{frame}
\begin{frame}
\frametitle{Unrelated Title}

\begin{center}
\includegraphics[width=0.9\textwidth,height=0.9\textheight,keepaspectratio]{/Users/I516998/Library/Application Support/Anki2/User 1/collection.media/img975019157587045245.jpg}
\end{center}


\note[item]{}
\end{frame}
\begin{frame}
\frametitle{Unrelated Title}

\begin{center}
\includegraphics[width=0.9\textwidth,height=0.9\textheight,keepaspectratio]{/Users/I516998/Library/Application Support/Anki2/User 1/collection.media/img2092111447105193981.jpg}
\end{center}


\note[item]{}
\end{frame}
\begin{frame}
\frametitle{Unrelated Title}

\begin{center}
\includegraphics[width=0.9\textwidth,height=0.9\textheight,keepaspectratio]{/Users/I516998/Library/Application Support/Anki2/User 1/collection.media/img1586682114946342816.jpg}
\end{center}


\note[item]{}
\end{frame}
\begin{frame}
\frametitle{Unrelated Title}

\begin{center}
\includegraphics[width=0.9\textwidth,height=0.9\textheight,keepaspectratio]{/Users/I516998/Library/Application Support/Anki2/User 1/collection.media/img2345162774797570480.jpg}
\end{center}


\note[item]{}
\end{frame}
\begin{frame}
\frametitle{Unrelated Title}

\begin{center}
\includegraphics[width=0.9\textwidth,height=0.9\textheight,keepaspectratio]{/Users/I516998/Library/Application Support/Anki2/User 1/collection.media/img1444150476482618047.jpg}
\end{center}


\note[item]{}
\end{frame}
\begin{frame}
\frametitle{Unrelated Title}

\begin{center}
\includegraphics[width=0.9\textwidth,height=0.9\textheight,keepaspectratio]{/Users/I516998/Library/Application Support/Anki2/User 1/collection.media/img342716503036666088.jpg}
\end{center}


\note[item]{}
\end{frame}
\begin{frame}
\frametitle{Unrelated Title}

\begin{center}
\includegraphics[width=0.9\textwidth,height=0.9\textheight,keepaspectratio]{/Users/I516998/Library/Application Support/Anki2/User 1/collection.media/img4647291255965927514.jpg}
\end{center}


\note[item]{}
\end{frame}
\begin{frame}
\frametitle{Unrelated Title}

\begin{center}
\includegraphics[width=0.9\textwidth,height=0.9\textheight,keepaspectratio]{/Users/I516998/Library/Application Support/Anki2/User 1/collection.media/img6730972955762682902.jpg}
\end{center}


\note[item]{}
\end{frame}
\begin{frame}
\frametitle{Unrelated Title}

\begin{center}
\includegraphics[width=0.9\textwidth,height=0.9\textheight,keepaspectratio]{/Users/I516998/Library/Application Support/Anki2/User 1/collection.media/img8436070651118151415.jpg}
\end{center}


\note[item]{}
\end{frame}
\begin{frame}
\frametitle{Unrelated Title}

\begin{center}
\includegraphics[width=0.9\textwidth,height=0.9\textheight,keepaspectratio]{/Users/I516998/Library/Application Support/Anki2/User 1/collection.media/img7258401884471535373.jpg}
\end{center}


\note[item]{}
\end{frame}
\begin{frame}
\frametitle{Unrelated Title}


\begin{itemize}
\item Belastung: Neutral -> objektive sachbezogene Anforderung an den Menschen, die sich auf alle gleich auswirken
\item Ergeben sich aus der Durchführung der Arbeitsaufgabe, Einflüssen der Arbeitsumgebung und Arbeitsorganisation
\item Beanspruchung: individuell unterschiedlich
\item Durch unterschiedliche individuelle Eigenschaften des Menschen geprägte Reaktion auf einwirkende Belastungen. 
\item Körperliche Eigenschaften sind zum Beispiel Leistungsfähigkeit oder kognitive Leistungsfähigkeit
\end{itemize}

\note[item]{}
\end{frame}
\begin{frame}
\frametitle{Unrelated Title}


\begin{itemize}
\item 1. Ideenprozess
\item 2. Produktentwicklung
\item 3. Lieferantensourcing & Ausschreibung
\item 4. Verhandlung
\item 5. Designentwicklung & Martkeinführung
\end{itemize}

\note[item]{}
\end{frame}
\begin{frame}
\frametitle{Unrelated Title}


\begin{itemize}
\item Interne Zahlen
\item Marktforschung
\item A-Marke
\item Lieferanten
\item Wettbewerb
\end{itemize}

\note[item]{}
\end{frame}
\begin{frame}
\frametitle{Unrelated Title}


\begin{itemize}
\item Eigene Forschung vs Dienstleister
\item Welchen Institut?
\item Trendreport
\item Messen
\item Erfahrungen von Lieferanten nutzen
\item - relevanter Markt?
\item - erfolgreiche Produkte
\item - Marktführer
\item - wachsende Segmente
\item - A Marke
\end{itemize}

\note[item]{}
\end{frame}
\begin{frame}
\frametitle{Unrelated Title}


\begin{itemize}
\item Produktsicherheit
\item Verpackung
\item Nachhaltigkeit (CSR im Einkauf)
\item Nachhaltigkeit Rohstoffe
\item Nachhaltigkeit Verpackung
\item Rezepturentwicklung
\end{itemize}

\note[item]{}
\end{frame}
\begin{frame}
\frametitle{Unrelated Title}


\begin{itemize}
\item Ist wichtig, damit man auch das gewünschte Produkt geliefert bekommt
\item Besteht u.a. aus:
\item - Rezepturvorgaben (sensorisch, visuell, analytisch)
\item - Orientierung an A-Marke und Wettbewerb
\item - Wertgebende Bestandteile
\item - Vorgaben Nachhaltigkeit der Rohstoffe
\item Ableitungen wie Ernährungsparameter sind notwendig, damit eine gute Bewertung bei Verbrauchermagazinen erreicht werden kann
\end{itemize}

\note[item]{}
\end{frame}
\begin{frame}
\frametitle{Unrelated Title}

\begin{center}
\includegraphics[width=0.9\textwidth,height=0.9\textheight,keepaspectratio]{/Users/I516998/Library/Application Support/Anki2/User 1/collection.media/img3155029952092373378.jpg}
\includegraphics[width=0.9\textwidth,height=0.9\textheight,keepaspectratio]{/Users/I516998/Library/Application Support/Anki2/User 1/collection.media/img433160940437691139.jpg}
\end{center}


\note[item]{}
\end{frame}
\begin{frame}
\frametitle{Unrelated Title}


\begin{itemize}
\item Einzelverpackung (EVP) 
\item - Glas vs PET
\item - Anzahl Druckfarben
\item - Qualität der Bestandteile (Einweg, Mehrweg, Nachhaltig, hochwertiges Design...)
\item - Tray Sichtbarkeit, Stabilität und Entnehmbarkeit
\end{itemize}

\note[item]{}
\end{frame}
\begin{frame}
\frametitle{Unrelated Title}


\begin{itemize}
\item - Messen
\item - Wettbewerb ( oft auf Verpackung aufgedruckt)
\item - Dienstleister z.b. Wabel
\item - Landesgesellschaften des Händlers
\item - Internetrecherche
\item - Handelskammern
\item - Empfehlungen
\item - Botschaften
\end{itemize}

\note[item]{}
\end{frame}
\begin{frame}
\frametitle{Unrelated Title}


\begin{itemize}
\item Global Sourcing
\item + Weltweite Auswahl
\item + Risikoverteilung
\item + Ausnutzen von Wechselkursschwankungen
\item - Versorgungsrisiko (Lieferkette)
\item -Währungsrisiko
\item - Qualitätsrisiko
\item - lange Vorlaufzeiten
\item - Probleme müssen in der Distanz geregelt werden
\end{itemize}

\note[item]{}
\end{frame}
\begin{frame}
\frametitle{Unrelated Title}


\begin{itemize}
\item Local Sourcing
\item + geringe Transportkosten
\item + gleiche Sprache 
\item + teilweise einfach notwendig aufgrund der Besonderheiten eines Produktes
\item + Kundenbindung durch "lokal"
\item - geringe Marktübersicht
\item - Preisintransparenz aufgrund mangelnde Vergleichsmöglichkeiten
\end{itemize}

\note[item]{}
\end{frame}
\begin{frame}
\frametitle{Unrelated Title}


\begin{itemize}
\item Insourcing
\item -> Integration von Leistungsumfängen, die zunächst ausgelagert waren
\item + Versorgungssicherheit
\item + weniger Abhängigkeit 
\item + Outsourcing nicht wirtschaftlich
\item - hohe Investition
\item - benötigt Spezialisten
\end{itemize}

\note[item]{}
\end{frame}
\begin{frame}
\frametitle{Unrelated Title}


\begin{itemize}
\item Outsourcing
\item -> auslagern von Teilprozessen
\item + Kostensenkung
\item + Know How des Dienstleisters
\item + Komzentration auf Kernkompetenz
\item - Qualitätseinbußen
\item - Abhängigkeit 
\item - Kontrollverlust
\end{itemize}

\note[item]{}
\end{frame}
\begin{frame}
\frametitle{Unrelated Title}


\begin{itemize}
\item -> mehrere Lieferanten
\item + Versorgungssicherheit
\item + Förderung Wettbewerb
\item + Vermeidung von Abhängigkeiten 
\item - Höhere operative Prozesskosten
\item - Höherer Abstimmungsaufwand
\item - Macht-Einflussverlust auf Seite des Abnehmers
\end{itemize}

\note[item]{}
\end{frame}
\begin{frame}
\frametitle{Unrelated Title}


\begin{itemize}
\item Anzahl der Beschaffer
\item Rechtlich selbständige Unternehmen stimmen die Beschaffung von Gütern und Dienstleistungen ab
\item Es findet eine Zusammenarbeit der Einkaufsabteilungen des verschiedenen Unternehmen statt
\item + Höhere Verhandlungsmacht
\item + Zugang zu neuen Märkten
\item - Höherer Koordinationsaufwand -> lange Verhandlugnszeit
\item - Know How Abfluss (Gefahr bei Auflösung der Kooperation)
\item - Verschärfung der Rechtlage führen zu Auflösung
\end{itemize}

\note[item]{}
\end{frame}
\begin{frame}
\frametitle{Unrelated Title}


\begin{itemize}
\item Werksbesichtigung
\item Systemaudit
\item Zertifizierungen
\item Gespräche 
\item Bei bestehendem Lieferant: KPI
\item Für Finanzlage:
\item Jahresbericht bei AGs
\item Bundesanzeiger
\item Interviews
\end{itemize}

\note[item]{}
\end{frame}
\begin{frame}
\frametitle{Unrelated Title}


\begin{itemize}
\item Klassische Ausschreibung
\item - per Mail erhält Lieferant alle notwendigen Infos zu Spezifikation, Menge, Liefervorgaben
\item -> schickt Angebot mit ausgefüllten Unterlagen zurück
\item Auktion:
\item - Lieferant wird über Internetplattform kontaktiert, erhält im Vorfeld alle wichtigen Infos zu Spezifikation, Menge, Liefervorgaben
\item - Einkauf gibt Rahmen der Auktion vor
\item - Lieferanten unterbieten gegenseitig den Startpreis
\end{itemize}

\note[item]{}
\end{frame}
\begin{frame}
\frametitle{Unrelated Title}


\begin{itemize}
\item Jahresgespräche für Preise und andere Konditionen
\item Grundsätzlich wichtig, eine Strategie zu haben, die Marktpreise und die Lieferanten zu kennen und weiter Kostensteigerungen oder Minderungen im Blick zu haben 
\item Im Food richten sich die Zeiten nach den Ernten
\item Incoterms -> Gefahrenübergang beachten
\end{itemize}

\note[item]{}
\end{frame}
\begin{frame}
\frametitle{Unrelated Title}

\begin{center}
\includegraphics[width=0.9\textwidth,height=0.9\textheight,keepaspectratio]{/Users/I516998/Library/Application Support/Anki2/User 1/collection.media/img527686641624662746.jpg}
\end{center}


\note[item]{}
\end{frame}
\begin{frame}
\frametitle{Unrelated Title}


\begin{itemize}
\item Markenrecht:
\item Formen der Markenanmeldung
\item Close to brand
\item Verpackungsaufbau
\item - Barcode
\item - lebensmittelrechtliche Vorgaben
\item - Design der A Marke
\item - Sortenfarbe
\item - Abbildungen
\item Druckumsetzung:
\item Design vs Proof vs Druckmuster
\item Dual Quality
\end{itemize}

\note[item]{}
\end{frame}
\begin{frame}
\frametitle{Unrelated Title}


\begin{itemize}
\item Analytik und Sensorik
\item Reklamationsmanagement & Risikomanagement
\item Platzierung
\item Erfolgskennzahlen
\end{itemize}

\note[item]{}
\end{frame}
\begin{frame}
\frametitle{Unrelated Title}

\begin{center}
\includegraphics[width=0.9\textwidth,height=0.9\textheight,keepaspectratio]{/Users/I516998/Library/Application Support/Anki2/User 1/collection.media/img2678022455124712416.jpg}
\end{center}


\note[item]{}
\end{frame}
\begin{frame}
\frametitle{Unrelated Title}

\begin{center}
\includegraphics[width=0.9\textwidth,height=0.9\textheight,keepaspectratio]{/Users/I516998/Library/Application Support/Anki2/User 1/collection.media/img3203732803504961182.jpg}
\end{center}


\note[item]{}
\end{frame}
\begin{frame}
\frametitle{Unrelated Title}

\begin{center}
\includegraphics[width=0.9\textwidth,height=0.9\textheight,keepaspectratio]{/Users/I516998/Library/Application Support/Anki2/User 1/collection.media/img674224451194535186.jpg}
\includegraphics[width=0.9\textwidth,height=0.9\textheight,keepaspectratio]{/Users/I516998/Library/Application Support/Anki2/User 1/collection.media/img8544966834789239070.jpg}
\end{center}


\note[item]{}
\end{frame}
\begin{frame}
\frametitle{Unrelated Title}


\begin{itemize}
\item Eyetracking
\item Stapelhöhe
\item Auspackware vs Trayplatzierung
\end{itemize}

\note[item]{}
\end{frame}
\begin{frame}
\frametitle{Unrelated Title}


\begin{itemize}
\item Umsatz, Absatz, Rohertrag, Spanne
\item Aktionsfrei vs Aktion
\item Käuferpotential
\item Unsatzpotential
\item Käuferpotentialausschöpfung
\item Umsatzpotentialausschöpfung
\item Loyalität = Bedarfsdeckung
\end{itemize}

\note[item]{}
\end{frame}
\begin{frame}
\frametitle{Unrelated Title}


\begin{itemize}
\item 1. Shopperorientierung
\item 2. Kooperative Einstellung
\item 3. Basis: Daten und Fakten
\item 4. Strukturierter und permanenter Prozess
\end{itemize}

\note[item]{}
\end{frame}
\begin{frame}
\frametitle{Unrelated Title}


\begin{itemize}
\item . 
\end{itemize}

\note[item]{}
\end{frame}
\begin{frame}
\frametitle{Unrelated Title}

\begin{center}
\includegraphics[width=0.9\textwidth,height=0.9\textheight,keepaspectratio]{/Users/I516998/Library/Application Support/Anki2/User 1/collection.media/img2079538057667589205.jpg}
\end{center}


\note[item]{}
\end{frame}
\begin{frame}
\frametitle{Unrelated Title}

\begin{center}
\includegraphics[width=0.9\textwidth,height=0.9\textheight,keepaspectratio]{/Users/I516998/Library/Application Support/Anki2/User 1/collection.media/img7963052542352816794.jpg}
\end{center}


\note[item]{}
\end{frame}
\begin{frame}
\frametitle{Unrelated Title}


\begin{itemize}
\item Aufgaben und Tätigkeiten eines Einkäufers bzw. der Einkaufsorganisation können entweder operativer oder strategischer Natur sein
\item Während der operative Einkauf administrative und ausführende Einkaufstätigkeiten wahrnimmt, trägt der strategische Einkauf die Verantwortung für langfristige Aufgaben. 
\item Aufgaben operativer Einkauf:
\item - Bestellabwicklung
\item - Reklamationsbearbeitung
\item - Terminkoordination
\item - Rechnungsprüfung
\item - Bearbeitung von Auftragsbestätigungen
\item Aufgaben des strategischen Einkaufs:
\item - Beschaffungsmarltforschung
\item - Marktanalysen (Rohstoffe, Finanzmarkt)
\item - Lieferantenmanagement (Recherche, Auswahl, Bewertung)
\item - Definition der Spezifikation von Beschaffungsartikeln
\item - Verhandlung von Rahmenverträgen
\item - Umsetzung der Einkaufsstrategie
\item - Qualitätssicherung 
\end{itemize}

\note[item]{}
\end{frame}
\begin{frame}
\frametitle{Unrelated Title}


\begin{itemize}
\item Sie ist die Strategie, in der zunächst allgemeine Leitlinien festgelegt werden
\item Sie wird aus der Unternehmensstrategie abgeleitet und beinhaltet u.a. die folgenden Punkte:
\item - Einbezug des Einkaufs
\item - Abgrenzung der Tätigkeitsfelder
\item - Markt-Strategien
\item - Sourcing-Strategien
\item - Aufstellung des Einkaufs im Geschäftsmodell
\item Wird u.a. aus Vertriebszielen und aus Anforderungen der internen Kunden abgeleitet
\item Wird außerdem mit Geschäftsführung und Produktbereich abgesprochen
\item Flexibilität ist wichtig, um sich an interne und externe Veränderungen schnell anzupassen
\item Im Einkaufshandbuch wird die Einkaufsstrategie festgehalten, um die Inhalte an die Mitarbeiter zu kommunizieren
\end{itemize}

\note[item]{}
\end{frame}
\begin{frame}
\frametitle{Unrelated Title}


\begin{itemize}
\item Im Warengruppenmanagement oder auch Materialgruppenmamagement werden Artikel des Beschaffungsportfolios eines Unternehmens anhand gemeinsamer Merkmale in Gruppen zusammengefasst. 
\item Ziel im Warengruppenmanagement ist es, die eigene Position zu erkennen und daraus Maßnahmen abzuleiten. Zudem dient sie der Identifikation von Optimierungspotentialen
\end{itemize}

\note[item]{}
\end{frame}
\begin{frame}
\frametitle{Unrelated Title}


\begin{itemize}
\item - Bedeutung der Warengruppe für das Unternehmen
\item - Marktstruktur und Marktstrategie (lokal/global)
\item - Lieferantenstrategie
\item - Preisstrategie
\item - Risikobewertung
\item - interne Kundenanforderungen 
\end{itemize}

\note[item]{}
\end{frame}
\begin{frame}
\frametitle{Unrelated Title}


\begin{itemize}
\item Direkt: gekaufte Produkte gehen direkt ins Endprodukt ein
\item Indirekt: gekaufte Produkte sind zur Nutzung im Unternehmen bspw. IT, Dienstleistungen
\end{itemize}

\note[item]{}
\end{frame}
\begin{frame}
\frametitle{Unrelated Title}


\begin{itemize}
\item - Hebelmaterialien
\item - Strategische Materialien
\item - Standardmaterialien
\item - Engpassmaterialien
\item Hebelmaterialien:
\item Sind in der Regel einfach zu beschaffen, d.h. das Beschaffungsrisiko ist gering. 
\item Sie haben allerdings einen hohen Einfluss auf das Betriebsergebnis.
\item Ihr wertmäßiger Anteil am Beschaffungsvolumen ist hoch. 
\item Bei Hebelmaterialien handelt es sich meist um relativ einfache Einzelteile, die möglichst standardisiert sein sollten, um eine bestmögliche Mengenbündelung zu gewährleisten. 
\item Geeignete Vorgehensweisen für diese Warengruppen sind z.B.:
\item - Ausnutzung des Marktpotentials zur Preisreduzierung
\item - Wettbewerb fördern (Lieferantenauswahl, Produktauswahl)
\item - global Sourcing
\item - gezielte Preis- und Verhandlungsstrategien
\item - gute Marktdaten, kurz- bis mittelfristige Bedarfsplanung
\item - Prognose Preisentwicklung und Frachtraten
\end{itemize}

\note[item]{}
\end{frame}
\begin{frame}
\frametitle{Unrelated Title}

\begin{center}
\includegraphics[width=0.9\textwidth,height=0.9\textheight,keepaspectratio]{/Users/I516998/Library/Application Support/Anki2/User 1/collection.media/img9000719233675480630.jpg}
\includegraphics[width=0.9\textwidth,height=0.9\textheight,keepaspectratio]{/Users/I516998/Library/Application Support/Anki2/User 1/collection.media/img4464656561506579547.jpg}
\end{center}


\note[item]{}
\end{frame}
\begin{frame}
\frametitle{Unrelated Title}

\begin{center}
\includegraphics[width=0.9\textwidth,height=0.9\textheight,keepaspectratio]{/Users/I516998/Library/Application Support/Anki2/User 1/collection.media/img8380354398245441706.jpg}
\end{center}

\begin{itemize}
\item Ist nach $4 ArbSchG Allgemeine Grundsätze bestimmt
\end{itemize}

\note[item]{}
\end{frame}
\begin{frame}
\frametitle{Unrelated Title}


\begin{itemize}
\item Nach $5ArbSchG
\item Ermittlung, Bewertung und Beurteilung von Ursachen und Bedingungen, die zu Unfällen bei der Arbeit und abrietsbedingten Gesundheitsgefahren führen können
\item - regelmäßige Überprüfung der Gefährdungsbeurteilung
\item - Dokumentation der Gefährdungsbeurteilung
\end{itemize}

\note[item]{}
\end{frame}
\begin{frame}
\frametitle{Unrelated Title}

\begin{center}
\includegraphics[width=0.9\textwidth,height=0.9\textheight,keepaspectratio]{/Users/I516998/Library/Application Support/Anki2/User 1/collection.media/img13095018683887477.jpg}
\end{center}


\note[item]{}
\end{frame}
\begin{frame}
\frametitle{Unrelated Title}

\begin{center}
\includegraphics[width=0.9\textwidth,height=0.9\textheight,keepaspectratio]{/Users/I516998/Library/Application Support/Anki2/User 1/collection.media/img1168662391210337834.jpg}
\end{center}


\note[item]{}
\end{frame}
\begin{frame}
\frametitle{Unrelated Title}

\begin{center}
\includegraphics[width=0.9\textwidth,height=0.9\textheight,keepaspectratio]{/Users/I516998/Library/Application Support/Anki2/User 1/collection.media/img3140933321436732550.jpg}
\end{center}

\begin{itemize}
\item Bei Ordnungswidrigkeiten wird schon der Verstoß sanktioniert
\item Im Strafrecht muss es zu einem Schaden gekommen sein
\end{itemize}

\note[item]{}
\end{frame}
\begin{frame}
\frametitle{Unrelated Title}

\begin{center}
\includegraphics[width=0.9\textwidth,height=0.9\textheight,keepaspectratio]{/Users/I516998/Library/Application Support/Anki2/User 1/collection.media/img4017105029270189568.jpg}
\includegraphics[width=0.9\textwidth,height=0.9\textheight,keepaspectratio]{/Users/I516998/Library/Application Support/Anki2/User 1/collection.media/img3360053424931545684.jpg}
\includegraphics[width=0.9\textwidth,height=0.9\textheight,keepaspectratio]{/Users/I516998/Library/Application Support/Anki2/User 1/collection.media/img826948142288479274.jpg}
\includegraphics[width=0.9\textwidth,height=0.9\textheight,keepaspectratio]{/Users/I516998/Library/Application Support/Anki2/User 1/collection.media/img3479737243347082685.jpg}
\end{center}


\note[item]{}
\end{frame}
\begin{frame}
\frametitle{Unrelated Title}


\begin{itemize}
\item Präventiv
\item Korrektiv
\end{itemize}

\note[item]{}
\end{frame}
\begin{frame}
\frametitle{Unrelated Title}


\begin{itemize}
\item Liegt vor, wenn eine versicherte Person wührend ihrer Tätigkeit einer schädigenden Einwirkung ausgesetzt ist und eine daraus resultierende Erkrankung in der Berfuslrankheiten-Verordnung aufgeführt ist. 
\item Bsp Bandscheibenvorfall
\end{itemize}

\note[item]{}
\end{frame}
\begin{frame}
\frametitle{Unrelated Title}


\begin{itemize}
\item Prävention
\item Rehabilitation
\item Entschädigung 
\end{itemize}

\note[item]{}
\end{frame}
\begin{frame}
\frametitle{Unrelated Title}


\begin{itemize}
\item Wenn eine versicherte Person sich bei einer versicherten Tätigkeit verletzt und diese Tätigkeit zum Unfall geführt hat.
\item -> Berufsgenosschenschaft übernimmt Rehabilitation
\end{itemize}

\note[item]{}
\end{frame}
\begin{frame}
\frametitle{Unrelated Title}

\begin{center}
\includegraphics[width=0.9\textwidth,height=0.9\textheight,keepaspectratio]{/Users/I516998/Library/Application Support/Anki2/User 1/collection.media/img5018538578624065929.jpg}
\end{center}

\begin{itemize}
\item Fachkräfte für Arbeitssicherheit und Sicherheitsbeauftragte wirken in Stabfunktionen mit und üben eine beratende Tätigkeit aus -> tragen keine Verantwortung, (ausschließliche die) Fachkräfte für Arbeitssicherheit sind aber für Richtigkeit der Beratung verantwortlich
\item Vorteile: überschaubare, klare Funktionsteilung mit entsprechender Verantwortlichkeit bleibt erhalten
\item Mangel an spezieller sicherheitstechnischer Fachkunde wird durch die Fachkräfte des Stabes ausgeglichen
\end{itemize}

\note[item]{}
\end{frame}













 \end{document}



