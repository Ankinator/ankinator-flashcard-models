\documentclass[
    11pt,
    aspectratio=169,
]{beamer}



\graphicspath{{images/}{./}}
\usepackage{booktabs}
\usepackage{calligra}

\usepackage{fnpct}
\usepackage{listings}





\usepackage[style=numeric, sorting=none, backend=biber]{biblatex}
\addbibresource{bib.bib}
\setbeamerfont{footnote}{size=\tiny}

\usepackage{xpatch}
\xapptobibmacro{cite}{\setunit{\nametitledelim} \printfield{title} \setunit{\nametitledelim} \printfield{year}}











\DeclareCiteCommand{\firstcite}
{\usebibmacro{prenote}}
{
    \footnotemark[\thefield{labelnumber}]
    \footnotetext[\thefield{labelnumber}]{
        \printnames{labelname}
        \setunit{\printdelim{nametitledelim}}

        \printfield[citetitle]{labeltitle}
        \setunit{\printdelim{nametitledelim}}
        \printfield{year}
        \newunit{\adddot}
    }
}
{\multicitedelim}
{\usebibmacro{postnote}}

\DeclareCiteCommand{\secondcite}
{\usebibmacro{prenote}}
{
    \footnotemark[\thefield{labelnumber}]
}
{\multicitedelim}
{\usebibmacro{postnote}}

\usetheme{Madrid}





\definecolor{smuBlue}{RGB}{21, 28, 85}

\definecolor{scisGold}{RGB}{198, 146, 0}

\setbeamercolor*{structure}{bg=smuBlue, fg=smuBlue}

\setbeamercolor*{palette primary}{use=structure, fg=white,bg=smuBlue}
\setbeamercolor*{palette secondary}{use=structure, fg=smuBlue, bg=white}

\setbeamercolor*{palette tertiary}{use=structure, fg=white, bg=smuBlue} 


\setbeamercolor{frametitle}{bg=smuBlue, fg=white}
\setbeamercolor*{titlelike}{parent=palette primary}







    \setbeamercolor{section in head/foot}{fg=white, bg=scisGold}
    \setbeamercolor{headline}{fg=white, bg=scisGold}






\setbeamercolor{item projected}{bg=scisGold}
\setbeamertemplate{enumerate items}{bg=scisGold}

\setbeamercolor{itemize item}{fg=scisGold}
\setbeamercolor{itemize subitem}{fg=scisGold}

\setbeamercolor{button}{bg=scisGold}


\setbeamercolor{section in toc}{fg=black}
\setbeamercolor{subsection in toc}{fg=black}



    \setbeamercolor{block title}{bg=scisGold, fg=white}
    \setbeamercolor{block body}{bg=scisGold!20}

    \setbeamercolor{block title alerted}{bg=orange, fg=white}
    \setbeamercolor{block body alerted}{bg=orange!10}

    \setbeamercolor{block title example}{bg=smuBlue, fg=white}
    \setbeamercolor{block body example}{bg=smuBlue!10}

\usefonttheme{default}
\usepackage{palatino}
\usepackage[default]{opensans}
\useinnertheme{circles}




\useoutertheme{miniframes}








\title[SMU Beamer Template]{SMU Presentation Template}

\subtitle{Subtitle}
\author[For Internal Use]{Author: Felix Nie}

\institute[]{School of Computing and Information Systems \\ \smallskip \textit{email@smu.edu.sg}}
\date[Spring 2023]




\logo{\includegraphics[width=2.5cm]{logo.eps}}

\lstset{
    basicstyle=\ttfamily\small,
    keywordstyle=\bfseries\color{blue},
    emphstyle=\ttfamily\color{red},   
    stringstyle=\color{green},
    numbers=left,
    numberstyle=\small\color{gray},
    rulesepcolor=\color{red!20!green!20!blue!20},
    frame=shadowbox,
    xleftmargin=1cm,
    xrightmargin=1cm,
}




\pdfstringdefDisableCommands{
    \def\translate#1{#1}
}


\setbeamertemplate{headline}{
    \nointerlineskip
    \begin{beamercolorbox}[wd=\paperwidth,ht=7.0ex]{headline}
        \insertnavigation{\paperwidth}\vspace*{2.0ex}
    \end{beamercolorbox}
}


\setbeamertemplate{navigation symbols}{}

\begin{document}

\begin{frame}
\frametitle{Unrelated Title}


\begin{itemize}
\item Eine Zahlung an Käufer und Verkäufer mit dem Ziel,die Einkommen zu erhöhen oder die Produktionskosten zu senkenund dadurch dem Empfänger der Subvention einen Vorteil zuverschaffen.
\end{itemize}

\note[item]{}
\end{frame}
\begin{frame}
\frametitle{Unrelated Title}


\begin{itemize}
\item 1. Vollständigkeit und Rangordnungsfähigkeit   •Konsumenten können Güterbündel vergleichen und sie aufreihen2. Für die meisten Güter gilt: "Mehr" ist besser als "weniger"   •Nichtsättigung und "kostenlose Entsorgung3.Transitivität   •sorgt für logisch konsistente Präferenzen und Rankings4.Je mehr ein Konsument von einem bestimmten Gut hat, desto weniger ist er bereit, ein anderes Gut aufzugeben, um noch mehr von diesem Gut zu erhalten.    •Wird auch als "abnehmender Grenznutzen" bezeichnet
\end{itemize}

\note[item]{}
\end{frame}
\begin{frame}
\frametitle{Unrelated Title}


\begin{itemize}
\item ist der zusätzliche Nutzen, den ein Konsument durch eine zusätzliche Einheit eines materiellen Gutes oder einer Dienstleistung erhält.
\end{itemize}

\note[item]{}
\end{frame}
\begin{frame}
\frametitle{Unrelated Title}

\begin{center}
\includegraphics[width=0.9\textwidth]{/Users/I516998/Library/Application Support/Anki2/User 1/collection.media/image-45db37487146d9d5f7917a9754f9da632f978ca7.png}
\end{center}


\note[item]{}
\end{frame}
\begin{frame}
\frametitle{Unrelated Title}

\begin{center}
\includegraphics[width=0.9\textwidth]{/Users/I516998/Library/Application Support/Anki2/User 1/collection.media/image-0a9614154db41c2ebd6b3d30c33a1fb425c96c48.png}
\end{center}


\note[item]{}
\end{frame}
\begin{frame}
\frametitle{Unrelated Title}

\begin{center}
\includegraphics[width=0.9\textwidth]{/Users/I516998/Library/Application Support/Anki2/User 1/collection.media/image-ef994671bb0c8ac6883de922a7ea246535a2d2e4.png}
\includegraphics[width=0.9\textwidth]{/Users/I516998/Library/Application Support/Anki2/User 1/collection.media/image-60ca01bdccb293a49e7e6deaf15572d3eaa80085.png}
\includegraphics[width=0.9\textwidth]{/Users/I516998/Library/Application Support/Anki2/User 1/collection.media/image-2f8649070fea63721f938a7f7a63b027f3f5c9c9.png}
\end{center}


\note[item]{}
\end{frame}
\begin{frame}
\frametitle{Unrelated Title}

\begin{center}
\includegraphics[width=0.9\textwidth]{/Users/I516998/Library/Application Support/Anki2/User 1/collection.media/image-433b56172a96c65f963fc6a6f9e15ecefca3c96b.png}
\end{center}


\note[item]{}
\end{frame}
\begin{frame}
\frametitle{Unrelated Title}


\begin{itemize}
\item (Realismus, Konstruktivismus,.. )-> weder zu beweisen noch zu widerlegen 
\end{itemize}

\note[item]{}
\end{frame}
\begin{frame}
\frametitle{Unrelated Title}


\begin{itemize}
\item 1. Definiendum ( Begriff, dessen Bedeutung festgelegt werden soll) 2. Definiens (Begriffe, die den Inhalt des Definiendums festlegen und begrenzen) -> weder wahr noch falsch-> Konventionen, die nichts über die Wirklichkeit aussagen -> keine Aussage über das "Wesen" von Tatbeständen 
\end{itemize}

\note[item]{}
\end{frame}
\begin{frame}
\frametitle{Unrelated Title}


\begin{itemize}
\item Begriff, dessen Bedeutung festgelegt werden soll
\end{itemize}

\note[item]{}
\end{frame}
\begin{frame}
\frametitle{Unrelated Title}


\begin{itemize}
\item Begriffe, die den Inhalt des Definiendums festlegen und begrenzen
\end{itemize}

\note[item]{}
\end{frame}
\begin{frame}
\frametitle{Unrelated Title}


\begin{itemize}
\item -> Im Definiens Begriff enthalten, deren Bedeutung nicht festgelegt ist => Folgedefinitionen nötigFolge -> unendliches definieren Abbruchkriterium -> Bedeutung der Begriffe kann als bekannt vorausgesetzt werden
\end{itemize}

\note[item]{}
\end{frame}
\begin{frame}
\frametitle{Unrelated Title}


\begin{itemize}
\item -> Definiendum ist Teil des Definiens 
\end{itemize}

\note[item]{}
\end{frame}
\begin{frame}
\frametitle{Unrelated Title}


\begin{itemize}
\item Variablen sind zusammenfassende Klassen von Prädikatoren  -> Beispiel: Geschlecht, Einkommen, Stellung im Beruf 
\end{itemize}

\note[item]{}
\end{frame}
\begin{frame}
\frametitle{Unrelated Title}


\begin{itemize}
\item -> Sonderform von Variablen-> Dispositionen sind situationsübergreifende Reaktionstendenzen Beispiel: Einstellungen, Handlungsbereitschaft, Fähigkeit 
\end{itemize}

\note[item]{}
\end{frame}
\begin{frame}
\frametitle{Unrelated Title}


\begin{itemize}
\item -> Kausalanalyse (z.B. familiäres Umfeld und Bildungserfolg-> Dimensionsanalyse (z.B Messung einer Dispositionsvariable) -> Konstruktion von Typologien ( z.B Schichten, Milieus) 
\end{itemize}

\note[item]{}
\end{frame}
\begin{frame}
\frametitle{Unrelated Title}


\begin{itemize}
\item -> Nichtzirkularität-> Präzision-> Konsistenz -> Relevanz:            -> empirisch            -> theoretisch            -> Praktisch  
\end{itemize}

\note[item]{}
\end{frame}
\begin{frame}
\frametitle{Unrelated Title}


\begin{itemize}
\item Wahrheitskriterium: Übereinstimmung mit den Tatsachen (Wirklichkeit), Korrespondenztheorie der Wirklichkeit 
\end{itemize}

\note[item]{}
\end{frame}
\begin{frame}
\frametitle{Unrelated Title}


\begin{itemize}
\item Wahrheitskriterium: Übereinstimmung mit übergeordneten Werten (Normen), wissenschaftlich keine Letztbegründung 
\end{itemize}

\note[item]{}
\end{frame}
\begin{frame}
\frametitle{Unrelated Title}


\begin{itemize}
\item >Singuläre Aussage -> raumzeitlich fixierte Aussage über Individuen und Ereignisse; auch sozialwissenschaftliche Aussagen 
\end{itemize}

\note[item]{}
\end{frame}

\end{document}