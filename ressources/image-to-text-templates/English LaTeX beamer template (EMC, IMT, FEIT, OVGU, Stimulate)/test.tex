



\documentclass[aspectratio=3218]{beamer}


% input encoding
\usepackage[utf8]{inputenc}

% output font encoding
\usepackage[T1]{fontenc}

% package for the English language
\usepackage[english]{babel}

% default graphics path
\graphicspath{{logos/}{figures/}}

% package to create diagrams and plot with TikZ
\usepackage{pgfplots}
% always use the newest version
\pgfplotsset{compat=newest}
% define the style of the connection of two lines
\pgfplotsset{every axis/.append style={line join=bevel}}
% generate a new style template
\mode<beamer>{
	\pgfplotsset{
		beamer/.style={
			width=0.8\textwidth,
			height=0.45\textwidth,
			legend style={font=\scriptsize},
			tick label style={font=\footnotesize},
			label style={font=\small},
			max space between ticks=28,
		}
	}
}
\mode<handout>{
	\pgfplotsset{
		beamer/.style={
			width=0.8\textwidth,
			height=0.45\textwidth,
			legend style={font=\scriptsize},
			tick label style={font=\footnotesize},
			label style={font=\small},
			max space between ticks=25,
		}
	}
}
\mode<article>{
	\pgfplotsset{
		beamer/.style={
			width=0.8\textwidth,
			height=0.45\textwidth,
			max space between ticks=35,
		}
	}
}
% format and size template for two plots side-by-side
\pgfplotsset{
	scriptsize/.style={
		width=0.34\textwidth,
		height=0.1768\textwidth,
		legend style={font=\scriptsize},
		tick label style={font=\scriptsize},
		label style={font=\footnotesize},
		title style={font=\footnotesize},
		every axis title shift=0pt,
		max space between ticks=25,
		every mark/.append style={mark size=7},
		major tick length=0.1cm,
		minor tick length=0.066cm,
	}
}
\pgfplotsset{
	small/.style={
		width=6.5cm,
		height=,
		tick label style={font=\footnotesize},
		label style={font=\small},
		legend style={font=\footnotesize},
		max space between ticks=30,
	}
}
% align legends entries to the left
\pgfplotsset{legend cell align=left}
% draw a main grid
\pgfplotsset{xmajorgrids}
\pgfplotsset{ymajorgrids}
% number of minor ticks between two major tickmarks
%\pgfplotsset{minor x tick num={3}}
%\pgfplotsset{minor y tick num={3}}
% draw a minor grid
%\pgfplotsset{xminorgrids}
%\pgfplotsset{yminorgrids}
% scale only the axis using a certain width and height
\pgfplotsset{scale only axis}
% define colors like in MATLAB
\definecolor{matlab1}{rgb}{0,0,1}
\definecolor{matlab2}{rgb}{0,0.5,0}
\definecolor{matlab3}{rgb}{1,0,0}
\definecolor{matlab4}{rgb}{0,0.75,0.75}
\definecolor{matlab5}{rgb}{0.75,0,0.75}
\definecolor{matlab6}{rgb}{0.75,0.75,0}
\definecolor{matlab7}{rgb}{0.25,0.25,0.25}
% define a color cycle list like in MATLAB
\pgfplotscreateplotcyclelist{matlab}{
	{matlab1,solid},
	{matlab2,dashed},
	{matlab3,dashdotted},
	{matlab4,dotted},
	{matlab5,densely dashed},
	{matlab6,densely dashdotted},
	{matlab7,densely dotted}%this prevents an error
}
% use the earlier defined color list
\pgfplotsset{cycle list name=matlab}
% use the standard color list of pgfplots
%\pgfplotsset{cycle list name=color list}
% use only grayscale lines
%\pgfplotsset{cycle list name=linestyles}
% use a line width of 1pt
\pgfplotsset{every axis plot/.append style={line width=1pt}}
% use a comma as the dot separator for german documents
\addto\extrasngerman{\pgfplotsset{/pgf/number format/.cd,set decimal separator={{{,}}}}}
% use a half space as the 1000 separator
\pgfplotsset{/pgf/number format/.cd,1000 sep={\,}}
% define legend positions
\pgfplotsset{/pgfplots/legend pos/north/.style={/pgfplots/legend style={at={(0.50,0.97)},anchor=north}}}
\pgfplotsset{/pgfplots/legend pos/south/.style={/pgfplots/legend style={at={(0.50,0.03)},anchor=south}}}
\pgfplotsset{/pgfplots/legend pos/east/.style={/pgfplots/legend style={at={(0.97,0.50)},anchor=east}}}
\pgfplotsset{/pgfplots/legend pos/west/.style={/pgfplots/legend style={at={(0.03,0.50)},anchor=west}}}
\pgfplotsset{/pgfplots/legend pos/outer north/.style={/pgfplots/legend style={at={(0.50,1.03)},anchor=south}}}

% package for SI units
\usepackage{siunitx}
% real fractions
\sisetup{per-mode=fraction,mode=math}
% decimal marker in dependence from the language
\addto\extrasngerman{\sisetup{output-decimal-marker={,}}}
\addto\extrasenglish{\sisetup{output-decimal-marker={.}}}
% range phrase in dependence from the language
\addto\extrasngerman{\sisetup{range-phrase={ bis~}}} 
\addto\extrasenglish{\sisetup{range-phrase={ to~}}}

% A stand-alone package that implements several commands for including external animation and sound
% files in a pdf document. The package can be used together with both dvips plus ps2pdf and pdflatex,
% though the special sound support is available only in pdflatex.
\usepackage{multimedia}

% package to dissable floating in the article mode
\usepackage{float}

% package to change the line spacing
\usepackage{setspace}

% package for nice fractions in the text mode
\usepackage{xfrac}
% default with with a slash as in the usual math mode
\UseCollection{xfrac}{plainmath}

% nicer tables
\usepackage{booktabs}

% nicer quotation marks
\usepackage{csquotes}

% define labeling environment
\makeatletter
\newenvironment{labeling}[2][]{%
  \def\sc@septext{#1}%
  \list{}{\settowidth{\labelwidth}{{%
        #2%
          \sc@septext%
      }}%
    \leftmargin\labelwidth \advance\leftmargin by \labelsep
    \let\makelabel\labelinglabel
  }%
}{%
  \endlist
}
\newcommand\labelinglabel[1]{%
  #1\hfil
    \sc@septext%
}
\makeatother

% BibLaTeX package for direct citations
\usepackage[%
	style=ieee,%				format as used at the IEEE
	backend=biber,%			suppress warning
]{biblatex}

% database of references
\addbibresource{literature.bib}

% How to remove some pages from navigation bullets in beamer?
% see: https://tex.stackexchange.com/questions/37127/how-to-remove-some-pages-from-the-navigation-bullets-in-beamer
\makeatletter
\let\beamer@writeslidentry@miniframeson=\beamer@writeslidentry
\def\beamer@writeslidentry@miniframesoff{%
	\expandafter\beamer@ifempty\expandafter{\beamer@framestartpage}{}% does not happen normally
	{%else
		% removed \addtocontents commands
		\clearpage\beamer@notesactions%
	}
}
\newcommand*{\miniframeson}{\let\beamer@writeslidentry=\beamer@writeslidentry@miniframeson}
\newcommand*{\miniframesoff}{\let\beamer@writeslidentry=\beamer@writeslidentry@miniframesoff}
\makeatother

% Distance between points of subsections in miniframe navigation
% siehe: http://tex.stackexchange.com/questions/288548/beamer-miniframes-horizontal-space-between-dots-when-using-the-compress-option
\makeatletter
\def\slideentry#1#2#3#4#5#6{%
  %section number, subsection number, slide number, first/last frame, page number, part number
  \ifnum#6=\c@part\ifnum#2>0\ifnum#3>0%
    \ifbeamer@compress%
      \advance\beamer@xpos by1\relax%
      \ifnum1=#3                        %%% NOTICE
				\ifnum1=#2\else
					\advance\beamer@xpos by1\relax% %%% THESE
				\fi
      \fi                               %%% LINE
    \else%
      \beamer@xpos=#3\relax%
      \beamer@ypos=#2\relax%
    \fi%
  \hbox to 0pt{%
    \beamer@tempdim=-\beamer@vboxoffset%
    \advance\beamer@tempdim by-\beamer@boxsize%
    \multiply\beamer@tempdim by\beamer@ypos%
    \advance\beamer@tempdim by -.05cm%
    \raise\beamer@tempdim\hbox{%
      \beamer@tempdim=\beamer@boxsize%
      \multiply\beamer@tempdim by\beamer@xpos%
      \advance\beamer@tempdim by -\beamer@boxsize%
      \advance\beamer@tempdim by 1pt%
      \kern\beamer@tempdim
      \global\beamer@section@min@dim\beamer@tempdim
      \hbox{\beamer@link(#4){%
          \usebeamerfont{mini frame}%
          \ifnum\c@section=#1%
            \ifnum\c@subsection=#2%
              \usebeamercolor[fg]{mini frame}%
              \ifnum\c@subsectionslide=#3%
                \usebeamertemplate{mini frame}%\beamer@minislidehilight%
              \else%
                \usebeamertemplate{mini frame in current subsection}%\beamer@minisliderowhilight%
              \fi%
            \else%
              \usebeamercolor{mini frame}%
              %\color{fg!50!bg}%
              \usebeamertemplate{mini frame in other subsection}%\beamer@minislide%
            \fi%
          \else%
            \usebeamercolor{mini frame}%
            %\color{fg!50!bg}%
            \usebeamertemplate{mini frame in other subsection}%\beamer@minislide%
          \fi%
        }}}\hskip-10cm plus 1fil%
  }\fi\fi%
  \else%
  \fakeslideentry{#1}{#2}{#3}{#4}{#5}{#6}%
  \fi\ignorespaces
}










\usepackage{beamer_ovgu_169}



\title[Short Title]{Title of the Presentation}
\author{John Doe}
\institute[Chair for EMC]{
	Chair for Electromagnetic Compatibility \\
	Institute for Medical Engineering \\
	Otto-von-Guericke-University Magdeburg, Germany
}
\mode<presentation>{\keywords{Put some keywords here, separated by commas, e.g. Non-Technical Project, Academic Writing}}
\date[01/01/2023]{Date of the Presentation, e.g. January 1th 2023}

\begin{document}

\begin{frame}
\frametitle<presentation>{Unrelated Title}


\begin{itemize}
\item Eine Zahlung an Käufer und Verkäufer mit dem Ziel,die Einkommen zu erhöhen oder die Produktionskosten zu senkenund dadurch dem Empfänger der Subvention einen Vorteil zuverschaffen.
\end{itemize}

\note[item]{}
\end{frame}
\begin{frame}
\frametitle<presentation>{Unrelated Title}


\begin{itemize}
\item 1. Vollständigkeit und Rangordnungsfähigkeit   •Konsumenten können Güterbündel vergleichen und sie aufreihen2. Für die meisten Güter gilt: "Mehr" ist besser als "weniger"   •Nichtsättigung und "kostenlose Entsorgung3.Transitivität   •sorgt für logisch konsistente Präferenzen und Rankings4.Je mehr ein Konsument von einem bestimmten Gut hat, desto weniger ist er bereit, ein anderes Gut aufzugeben, um noch mehr von diesem Gut zu erhalten.    •Wird auch als "abnehmender Grenznutzen" bezeichnet
\end{itemize}

\note[item]{}
\end{frame}
\begin{frame}
\frametitle<presentation>{Unrelated Title}


\begin{itemize}
\item ist der zusätzliche Nutzen, den ein Konsument durch eine zusätzliche Einheit eines materiellen Gutes oder einer Dienstleistung erhält.
\end{itemize}

\note[item]{}
\end{frame}
\begin{frame}
\frametitle<presentation>{Unrelated Title}

\begin{center}
\includegraphics[width=0.9\textwidth]{/Users/I516998/Library/Application Support/Anki2/User 1/collection.media/image-45db37487146d9d5f7917a9754f9da632f978ca7.png}
\end{center}


\note[item]{}
\end{frame}
\begin{frame}
\frametitle<presentation>{Unrelated Title}

\begin{center}
\includegraphics[width=0.9\textwidth]{/Users/I516998/Library/Application Support/Anki2/User 1/collection.media/image-0a9614154db41c2ebd6b3d30c33a1fb425c96c48.png}
\end{center}


\note[item]{}
\end{frame}
\begin{frame}
\frametitle<presentation>{Unrelated Title}

\begin{center}
\includegraphics[width=0.9\textwidth]{/Users/I516998/Library/Application Support/Anki2/User 1/collection.media/image-ef994671bb0c8ac6883de922a7ea246535a2d2e4.png}
\includegraphics[width=0.9\textwidth]{/Users/I516998/Library/Application Support/Anki2/User 1/collection.media/image-60ca01bdccb293a49e7e6deaf15572d3eaa80085.png}
\includegraphics[width=0.9\textwidth]{/Users/I516998/Library/Application Support/Anki2/User 1/collection.media/image-2f8649070fea63721f938a7f7a63b027f3f5c9c9.png}
\end{center}


\note[item]{}
\end{frame}
\begin{frame}
\frametitle<presentation>{Unrelated Title}

\begin{center}
\includegraphics[width=0.9\textwidth]{/Users/I516998/Library/Application Support/Anki2/User 1/collection.media/image-433b56172a96c65f963fc6a6f9e15ecefca3c96b.png}
\end{center}


\note[item]{}
\end{frame}
\begin{frame}
\frametitle<presentation>{Unrelated Title}


\begin{itemize}
\item (Realismus, Konstruktivismus,.. )-> weder zu beweisen noch zu widerlegen 
\end{itemize}

\note[item]{}
\end{frame}
\begin{frame}
\frametitle<presentation>{Unrelated Title}


\begin{itemize}
\item 1. Definiendum ( Begriff, dessen Bedeutung festgelegt werden soll) 2. Definiens (Begriffe, die den Inhalt des Definiendums festlegen und begrenzen) -> weder wahr noch falsch-> Konventionen, die nichts über die Wirklichkeit aussagen -> keine Aussage über das "Wesen" von Tatbeständen 
\end{itemize}

\note[item]{}
\end{frame}
\begin{frame}
\frametitle<presentation>{Unrelated Title}


\begin{itemize}
\item Begriff, dessen Bedeutung festgelegt werden soll
\end{itemize}

\note[item]{}
\end{frame}
\begin{frame}
\frametitle<presentation>{Unrelated Title}


\begin{itemize}
\item Begriffe, die den Inhalt des Definiendums festlegen und begrenzen
\end{itemize}

\note[item]{}
\end{frame}
\begin{frame}
\frametitle<presentation>{Unrelated Title}


\begin{itemize}
\item -> Im Definiens Begriff enthalten, deren Bedeutung nicht festgelegt ist => Folgedefinitionen nötigFolge -> unendliches definieren Abbruchkriterium -> Bedeutung der Begriffe kann als bekannt vorausgesetzt werden
\end{itemize}

\note[item]{}
\end{frame}
\begin{frame}
\frametitle<presentation>{Unrelated Title}


\begin{itemize}
\item -> Definiendum ist Teil des Definiens 
\end{itemize}

\note[item]{}
\end{frame}
\begin{frame}
\frametitle<presentation>{Unrelated Title}


\begin{itemize}
\item Variablen sind zusammenfassende Klassen von Prädikatoren  -> Beispiel: Geschlecht, Einkommen, Stellung im Beruf 
\end{itemize}

\note[item]{}
\end{frame}
\begin{frame}
\frametitle<presentation>{Unrelated Title}


\begin{itemize}
\item -> Sonderform von Variablen-> Dispositionen sind situationsübergreifende Reaktionstendenzen Beispiel: Einstellungen, Handlungsbereitschaft, Fähigkeit 
\end{itemize}

\note[item]{}
\end{frame}
\begin{frame}
\frametitle<presentation>{Unrelated Title}


\begin{itemize}
\item -> Kausalanalyse (z.B. familiäres Umfeld und Bildungserfolg-> Dimensionsanalyse (z.B Messung einer Dispositionsvariable) -> Konstruktion von Typologien ( z.B Schichten, Milieus) 
\end{itemize}

\note[item]{}
\end{frame}
\begin{frame}
\frametitle<presentation>{Unrelated Title}


\begin{itemize}
\item -> Nichtzirkularität-> Präzision-> Konsistenz -> Relevanz:            -> empirisch            -> theoretisch            -> Praktisch  
\end{itemize}

\note[item]{}
\end{frame}
\begin{frame}
\frametitle<presentation>{Unrelated Title}


\begin{itemize}
\item Wahrheitskriterium: Übereinstimmung mit den Tatsachen (Wirklichkeit), Korrespondenztheorie der Wirklichkeit 
\end{itemize}

\note[item]{}
\end{frame}
\begin{frame}
\frametitle<presentation>{Unrelated Title}


\begin{itemize}
\item Wahrheitskriterium: Übereinstimmung mit übergeordneten Werten (Normen), wissenschaftlich keine Letztbegründung 
\end{itemize}

\note[item]{}
\end{frame}
\begin{frame}
\frametitle<presentation>{Unrelated Title}


\begin{itemize}
\item >Singuläre Aussage -> raumzeitlich fixierte Aussage über Individuen und Ereignisse; auch sozialwissenschaftliche Aussagen 
\end{itemize}

\note[item]{}
\end{frame}

\end{document}